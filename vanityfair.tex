% This document is a LaTeX-ification of a document in the Public Domain.
% LaTeX-ification by David W. Hogg (NYU).

% WARNING: DO NOT USE THIS FILE:
% YOUR EYES WILL BURN.

% to-do list:
% -----------
% - fix all occurences of all-caps.
% - check and deal with all-caps words and phrases
% - make a blockquote environment consistent with Bringhurst.
% - find quoted correspondence and notes and set them properly with margins.
% - almost all italics from my edition have been eradicated here... fix these?
% - almost all foreign punctuation from my edition has been eradicated here... fix these!
% - develop a policy about what foreign words get \foreign{} -- eg, place names?
% - decide how to typeset money amounts -- there are inconsistencies now.
% - find all HOGG in text and fix.

\documentclass[10pt, a5paper]{book}
\usepackage{fancyhdr, textgreek}
\usepackage[a5paper]{geometry}
\usepackage{tocloft}

% fonts
\renewcommand{\bfseries}{\relax} % remove all bold forever
\renewcommand{\sffamily}{\relax} % remove all sans-serif forever

% spacings
\geometry{hmargin={2cm, 2cm}}
\geometry{vmargin={2.5cm, 2.5cm}}
\newlength{\hoggskip}
\setlength{\hoggskip}{1.0833\baselineskip}
\linespread{1.0833} % 10/13 line spacing
\setlength{\parindent}{\hoggskip} % make the indents like square blocks

\makeatletter
\renewenvironment{quotation} % idea from http://tex.stackexchange.com/questions/29640/how-to-modify-spacing-around-quotation-environment
               {\list{}{\listparindent=\hoggskip
                        \itemindent=\listparindent
                        \leftmargin=1.5\hoggskip
                        \rightmargin=0.5\hoggskip
                        \topsep=0pt
                        \parsep        \z@ \@plus\p@}%
                \item\relax}
               {\endlist}
\makeatother
\newcommand{\longindent}{\noindent\rule{0.4\textwidth}{0pt}}

% header hacking
\renewcommand{\MakeUppercase}[1]{{#1}}
\pagestyle{fancy}
\renewcommand{\thechapter}{\Roman{chapter}.}
\renewcommand{\chaptermark}[1]{\markboth{~}{\thechapter~#1}}
\fancyhead[LE]{{\footnotesize{\textit{Vanity Fair} by William Makepeace Thackeray}}}
\fancyhead[RE]{}
\fancyhead[LO]{}
\fancyhead[RO]{{\footnotesize\rightmark}}

% toc hacking
\newlength{\nnn}
\settowidth{\nnn}{~999}
\cftsetpnumwidth{\nnn}
\newlength{\XXXVIII}
\settowidth{\XXXVIII}{XXXVIII.~}
\renewcommand{\cftchapnumwidth}{\XXXVIII}
\setlength{\cftbeforepartskip}{0ex}
\setlength{\cftbeforechapskip}{0ex}

% word typesetting
\newcommand{\foreign}[1]{\textsl{#1}}

\begin{document}%
\sloppy\sloppypar\raggedbottom%
\markboth{William Makepeace Thackeray \ \textit{Vanity Fair}}{}

\begin{centering}\thispagestyle{empty}
~
\vfill

Vanity Fair

\vfill
~

\vfill
~
\end{centering}

\clearpage
~\thispagestyle{empty}

\cleardoublepage
% HOGG: Fix these hacks!
\chapter*{Vanity Fair}\thispagestyle{empty}
\section*{A Novel Without a Hero}
~
\subsection*{by William Makepeace Thackeray}
~

\clearpage
\begin{centering}\thispagestyle{plain}
~
\vfill

All content is in the Public Domain.

\vfill

This book has no ISBN. \\
It is a one-off original \emph{made by hand} by David W. Hogg. \\
See the end matter starting on page~\pageref{about} for more information.

\vfill
~
\end{centering}

\cleardoublepage
~\thispagestyle{plain}

%% HOGG: If you want a table of contents, you are going to have to work for it
\cleardoublepage
\tableofcontents

\chapter*{Before the Curtain}
\markboth{~}{Before the Curtain}
\addcontentsline{toc}{chapter}{\numberline{~}Before the Curtain}

As the manager of the Performance sits before the curtain on the boards
and looks into the Fair, a feeling of profound melancholy comes over
him in his survey of the bustling place. There is a great quantity of
eating and drinking, making love and jilting, laughing and the
contrary, smoking, cheating, fighting, dancing and fiddling; there are
bullies pushing about, bucks ogling the women, knaves picking pockets,
policemen on the look-out, quacks (\emph{other} quacks, plague take them!)
bawling in front of their booths, and yokels looking up at the
tinselled dancers and poor old rouged tumblers, while the
light-fingered folk are operating upon their pockets behind. Yes, this
is \textsc{Vanity Fair}; not a moral place certainly; nor a merry one, though
very noisy.  Look at the faces of the actors and buffoons when they
come off from their business; and Tom Fool washing the paint off his
cheeks before he sits down to dinner with his wife and the little Jack
Puddings behind the canvas.   The curtain will be up presently, and he
will be turning over head and heels, and crying, ``How are you?''

A man with a reflective turn of mind, walking through an exhibition of
this sort, will not be oppressed, I take it, by his own or other
people's hilarity.   An episode of humour or kindness touches and
amuses him here and there---a pretty child looking at a gingerbread
stall; a pretty girl blushing whilst her lover talks to her and chooses
her fairing; poor Tom Fool, yonder behind the waggon, mumbling his bone
with the honest family which lives by his tumbling; but the general
impression is one more melancholy than mirthful.  When you come home
you sit down in a sober, contemplative, not uncharitable frame of mind,
and apply yourself to your books or your business.

I have no other moral than this to tag to the present story of ``Vanity
Fair.'' Some people consider Fairs immoral altogether, and eschew such,
with their servants and families: very likely they are right.  But
persons who think otherwise, and are of a lazy, or a benevolent, or a
sarcastic mood, may perhaps like to step in for half an hour, and look
at the performances. There are scenes of all sorts; some dreadful
combats, some grand and lofty horse-riding, some scenes of high life,
and some of very middling indeed; some love-making for the sentimental,
and some light comic business; the whole accompanied by appropriate
scenery and brilliantly illuminated with the Author's own candles.

What more has the Manager of the Performance to say?---To acknowledge
the kindness with which it has been received in all the principal towns
of England through which the Show has passed, and where it has been
most favourably noticed by the respected conductors of the public
Press, and by the Nobility and Gentry.  He is proud to think that his
Puppets have given satisfaction to the very best company in this
empire.  The famous little Becky Puppet has been pronounced to be
uncommonly flexible in the joints, and lively on the wire; the Amelia
Doll, though it has had a smaller circle of admirers, has yet been
carved and dressed with the greatest care by the artist; the Dobbin
Figure, though apparently clumsy, yet dances in a very amusing and
natural manner; the Little Boys' Dance has been liked by some; and
please to remark the richly dressed figure of the Wicked Nobleman, on
which no expense has been spared, and which Old Nick will fetch away at
the end of this singular performance.

And with this, and a profound bow to his patrons, the Manager retires,
and the curtain rises.

\smallskip
\noindent
\textsc{London} \hfill \textsl{June 28, 1848}

%% HOGG: Use the following to check that the Chapters are complete and
%% numbered correctly in the LaTeX output.

%% CONTENTS

%%        I  Chiswick Mall
%%       II  In Which Miss Sharp and Miss Sedley Prepare to Open the
%%           Campaign
%%      III  Rebecca Is in Presence of the Enemy
%%       IV  The Green Silk Purse
%%        V  Dobbin of Ours
%%       VI  Vauxhall
%%      VII  Crawley of Queen's Crawley
%%     VIII  Private and Confidential
%%       IX  Family Portraits
%%        X  Miss Sharp Begins to Make Friends
%%       XI  Arcadian Simplicity
%%      XII  Quite a Sentimental Chapter
%%     XIII  Sentimental and Otherwise
%%      XIV  Miss Crawley at Home
%%       XV  In Which Rebecca's Husband Appears for a Short Time
%%      XVI  The Letter on the Pincushion
%%     XVII  How Captain Dobbin Bought a Piano
%%    XVIII  Who Played on the Piano Captain Dobbin Bought
%%      XIX  Miss Crawley at Nurse
%%       XX  In Which Captain Dobbin Acts as the Messenger of Hymen
%%      XXI  A Quarrel About an Heiress
%%     XXII  A Marriage and Part of a Honeymoon
%%    XXIII  Captain Dobbin Proceeds on His Canvass
%%     XXIV  In Which Mr.\ Osborne Takes Down the Family Bible
%%      XXV  In Which All the Principal Personages Think Fit to Leave
%%           Brighton
%%     XXVI  Between London and Chatham
%%    XXVII  In Which Amelia Joins Her Regiment
%%   XXVIII  In Which Amelia Invades the Low Countries
%%     XXIX  Brussels
%%      XXX  ``The Girl I Left Behind Me''
%%     XXXI  In Which Jos Sedley Takes Care of His Sister
%%    XXXII  In Which Jos Takes Flight, and the War Is Brought to a Close
%%   XXXIII  In Which Miss Crawley's Relations Are Very Anxious About Her
%%    XXXIV  James Crawley's Pipe Is Put Out
%%     XXXV  Widow and Mother
%%    XXXVI  How to Live Well on Nothing a Year
%%   XXXVII  The Subject Continued
%%  XXXVIII  A Family in a Very Small Way
%%    XXXIX  A Cynical Chapter
%%       XL  In Which Becky Is Recognized by the Family
%%      XLI  In Which Becky Revisits the Halls of Her Ancestors
%%     XLII  Which Treats of the Osborne Family
%%    XLIII  In Which the Reader Has to Double the Cape
%%     XLIV  A Round-about Chapter between London and Hampshire
%%      XLV  Between Hampshire and London
%%     XLVI  Struggles and Trials
%%    XLVII  Gaunt House
%%   XLVIII  In Which the Reader Is Introduced to the Very Best of Company
%%     XLIX  In Which We Enjoy Three Courses and a Dessert
%%        L  Contains a Vulgar Incident
%%       LI  In Which a Charade Is Acted Which May or May Not Puzzle the
%%           Reader
%%      LII  In Which Lord Steyne Shows Himself in a Most Amiable Light
%%     LIII  A Rescue and a Catastrophe
%%      LIV  Sunday After the Battle
%%       LV  In Which the Same Subject is Pursued
%%      LVI  Georgy is Made a Gentleman
%%     LVII  Eothen
%%    LVIII  Our Friend the Major
%%      LIX  The Old Piano
%%       LX  Returns to the Genteel World
%%      LXI  In Which Two Lights are Put Out
%%     LXII  Am Rhein
%%    LXIII  In Which We Meet an Old Acquaintance
%%     LXIV  A Vagabond Chapter
%%      LXV  Full of Business and Pleasure
%%     LXVI  Amantium Irae
%%    LXVII  Which Contains Births, Marriages, and Deaths

\chapter{Chiswick Mall}

While the present century was in its teens, and on one sunshiny morning
in June, there drove up to the great iron gate of Miss Pinkerton's
academy for young ladies, on Chiswick Mall, a large family coach, with
two fat horses in blazing harness, driven by a fat coachman in a
three-cornered hat and wig, at the rate of four miles an hour.  A black
servant, who reposed on the box beside the fat coachman, uncurled his
bandy legs as soon as the equipage drew up opposite Miss Pinkerton's
shining brass plate, and as he pulled the bell at least a score of
young heads were seen peering out of the narrow windows of the stately
old brick house.  Nay, the acute observer might have recognized the
little red nose of good-natured Miss Jemima Pinkerton herself, rising
over some geranium pots in the window of that lady's own drawing-room.

``It is Mrs.\ Sedley's coach, sister,'' said Miss Jemima. ``Sambo, the
black servant, has just rung the bell; and the coachman has a new red
waistcoat.''

``Have you completed all the necessary preparations incident to Miss
Sedley's departure, Miss Jemima?'' asked Miss Pinkerton herself, that
majestic lady; the Semiramis of Hammersmith, the friend of Doctor
Johnson, the correspondent of Mrs.\ Chapone herself.

``The girls were up at four this morning, packing her trunks, sister,''
replied Miss Jemima; ``we have made her a bow-pot.''

``Say a bouquet, sister Jemima, 'tis more genteel.''

``Well, a booky as big almost as a haystack; I have put up two bottles
of the gillyflower water for Mrs.\ Sedley, and the receipt for making
it, in Amelia's box.''

``And I trust, Miss Jemima, you have made a copy of Miss Sedley's
account.  This is it, is it? Very good---ninety-three pounds, four
shillings.  Be kind enough to address it to John Sedley, Esquire, and
to seal this billet which I have written to his lady.''

In Miss Jemima's eyes an autograph letter of her sister, Miss
Pinkerton, was an object of as deep veneration as would have been a
letter from a sovereign.  Only when her pupils quitted the
establishment, or when they were about to be married, and once, when
poor Miss Birch died of the scarlet fever, was Miss Pinkerton known to
write personally to the parents of her pupils; and it was Jemima's
opinion that if anything \emph{could} console Mrs.\ Birch for her daughter's
loss, it would be that pious and eloquent composition in which Miss
Pinkerton announced the event.

In the present instance Miss Pinkerton's ``billet'' was to the following
effect:---
\begin{quotation}
\longindent The Mall, Chiswick, June 15, 18---

\noindent Madam,

After her six years' residence at the Mall, I have the honour
and happiness of presenting Miss Amelia Sedley to her parents, as a
young lady not unworthy to occupy a fitting position in their polished
and refined circle.  Those virtues which characterize the young English
gentlewoman, those accomplishments which become her birth and station,
will not be found wanting in the amiable Miss Sedley, whose \emph{industry}
and \emph{obedience} have endeared her to her instructors, and whose
delightful sweetness of temper has charmed her \emph{aged} and her \emph{youthful}
companions.

In music, in dancing, in orthography, in every variety of embroidery
and needlework, she will be found to have realized her friends' fondest
wishes.  In geography there is still much to be desired; and a careful
and undeviating use of the backboard, for four hours daily during the
next three years, is recommended as necessary to the acquirement of
that dignified \emph{deportment and carriage}, so requisite for every young
lady of \emph{fashion}.

In the principles of religion and morality, Miss Sedley will be found
worthy of an establishment which has been honoured by the presence of
\emph{The Great Lexicographer}, and the patronage of the admirable Mrs.\ %
Chapone.  In leaving the Mall, Miss Amelia carries with her the hearts
of her companions, and the affectionate regards of her mistress, who
has the honour to subscribe herself,

\longindent Madam,

\longindent Your most obliged humble servant,

\longindent Barbara Pinkerton

P. S.---Miss Sharp accompanies Miss Sedley.  It is particularly requested
that Miss Sharp's stay in Russell Square may not exceed ten days.  The
family of distinction with whom she is engaged, desire to avail
themselves of her services as soon as possible.
\end{quotation}

This letter completed, Miss Pinkerton proceeded to write her own name,
and Miss Sedley's, in the fly-leaf of a Johnson's Dictionary---the
interesting work which she invariably presented to her scholars, on
their departure from the Mall.  On the cover was inserted a copy of
``Lines addressed to a young lady on quitting Miss Pinkerton's school,
at the Mall; by the late revered Doctor Samuel Johnson.'' In fact, the
Lexicographer's name was always on the lips of this majestic woman, and
a visit he had paid to her was the cause of her reputation and her
fortune.

Being commanded by her elder sister to get ``the Dictionary'' from the
cupboard, Miss Jemima had extracted two copies of the book from the
receptacle in question.  When Miss Pinkerton had finished the
inscription in the first, Jemima, with rather a dubious and timid air,
handed her the second.

``For whom is this, Miss Jemima?'' said Miss Pinkerton, with awful
coldness.

``For Becky Sharp,'' answered Jemima, trembling very much, and blushing
over her withered face and neck, as she turned her back on her sister.
``For Becky Sharp: she's going too.''

``MISS JEMIMA!'' exclaimed Miss Pinkerton, in the largest capitals. ``Are
you in your senses? Replace the Dixonary in the closet, and never
venture to take such a liberty in future.''

``Well, sister, it's only two-and-ninepence, and poor Becky will be
miserable if she don't get one.''

``Send Miss Sedley instantly to me,'' said Miss Pinkerton. And so
venturing not to say another word, poor Jemima trotted off, exceedingly
flurried and nervous.

Miss Sedley's papa was a merchant in London, and a man of some wealth;
whereas Miss Sharp was an articled pupil, for whom Miss Pinkerton had
done, as she thought, quite enough, without conferring upon her at
parting the high honour of the Dixonary.

Although schoolmistresses' letters are to be trusted no more nor less
than churchyard epitaphs; yet, as it sometimes happens that a person
departs this life who is really deserving of all the praises the stone
cutter carves over his bones; who \emph{is} a good Christian, a good parent,
child, wife, or husband; who actually \emph{does} leave a disconsolate family
to mourn his loss; so in academies of the male and female sex it occurs
every now and then that the pupil is fully worthy of the praises
bestowed by the disinterested instructor. Now, Miss Amelia Sedley was a
young lady of this singular species; and deserved not only all that
Miss Pinkerton said in her praise, but had many charming qualities
which that pompous old Minerva of a woman could not see, from the
differences of rank and age between her pupil and herself.

For she could not only sing like a lark, or a Mrs.\ Billington, and
dance like Hillisberg or Parisot; and embroider beautifully; and spell
as well as a Dixonary itself; but she had such a kindly, smiling,
tender, gentle, generous heart of her own, as won the love of everybody
who came near her, from Minerva herself down to the poor girl in the
scullery, and the one-eyed tart-woman's daughter, who was permitted to
vend her wares once a week to the young ladies in the Mall.  She had
twelve intimate and bosom friends out of the twenty-four young ladies.
Even envious Miss Briggs never spoke ill of her; high and mighty Miss
Saltire (Lord Dexter's granddaughter) allowed that her figure was
genteel; and as for Miss Swartz, the rich woolly-haired mulatto from
St.\ Kitt's, on the day Amelia went away, she was in such a passion of
tears that they were obliged to send for Dr. Floss, and half tipsify
her with salvolatile.  Miss Pinkerton's attachment was, as may be
supposed from the high position and eminent virtues of that lady, calm
and dignified; but Miss Jemima had already whimpered several times at
the idea of Amelia's departure; and, but for fear of her sister, would
have gone off in downright hysterics, like the heiress (who paid
double) of St.\ Kitt's.  Such luxury of grief, however, is only allowed
to parlour-boarders. Honest Jemima had all the bills, and the washing,
and the mending, and the puddings, and the plate and crockery, and the
servants to superintend.  But why speak about her?  It is probable that
we shall not hear of her again from this moment to the end of time, and
that when the great filigree iron gates are once closed on her, she and
her awful sister will never issue therefrom into this little world of
history.

But as we are to see a great deal of Amelia, there is no harm in
saying, at the outset of our acquaintance, that she was a dear little
creature; and a great mercy it is, both in life and in novels, which
(and the latter especially) abound in villains of the most sombre sort,
that we are to have for a constant companion so guileless and
good-natured a person.  As she is not a heroine, there is no need to
describe her person; indeed I am afraid that her nose was rather short
than otherwise, and her cheeks a great deal too round and red for a
heroine; but her face blushed with rosy health, and her lips with the
freshest of smiles, and she had a pair of eyes which sparkled with the
brightest and honestest good-humour, except indeed when they filled
with tears, and that was a great deal too often; for the silly thing
would cry over a dead canary-bird; or over a mouse, that the cat haply
had seized upon; or over the end of a novel, were it ever so stupid;
and as for saying an unkind word to her, were any persons hard-hearted
enough to do so---why, so much the worse for them.  Even Miss Pinkerton,
that austere and godlike woman, ceased scolding her after the first
time, and though she no more comprehended sensibility than she did
Algebra, gave all masters and teachers particular orders to treat Miss
Sedley with the utmost gentleness, as harsh treatment was injurious to
her.

So that when the day of departure came, between her two customs of
laughing and crying, Miss Sedley was greatly puzzled how to act. She
was glad to go home, and yet most woefully sad at leaving school.  For
three days before, little Laura Martin, the orphan, followed her about
like a little dog.  She had to make and receive at least fourteen
presents---to make fourteen solemn promises of writing every week:
``Send my letters under cover to my grandpapa, the Earl of Dexter,'' said
Miss Saltire (who, by the way, was rather shabby).  ``Never mind the
postage, but write every day, you dear darling,'' said the impetuous and
woolly-headed, but generous and affectionate Miss Swartz; and the
orphan little Laura Martin (who was just in round-hand), took her
friend's hand and said, looking up in her face wistfully, ``Amelia, when
I write to you I shall call you Mamma.'' All which details, I have no
doubt, \textsc{Jones}, who reads this book at his Club, will pronounce to be
excessively foolish, trivial, twaddling, and ultra-sentimental.  Yes; I
can see Jones at this minute (rather flushed with his joint of mutton
and half pint of wine), taking out his pencil and scoring under the
words ``foolish, twaddling,'' \&c., and adding to them his own remark of
``\emph{quite true}.'' Well, he is a lofty man of genius, and admires the great
and heroic in life and novels; and so had better take warning and go
elsewhere.

Well, then.  The flowers, and the presents, and the trunks, and
bonnet-boxes of Miss Sedley having been arranged by Mr.\ Sambo in the
carriage, together with a very small and weather-beaten old cow's-skin
trunk with Miss Sharp's card neatly nailed upon it, which was delivered
by Sambo with a grin, and packed by the coachman with a corresponding
sneer---the hour for parting came; and the grief of that moment was
considerably lessened by the admirable discourse which Miss Pinkerton
addressed to her pupil.  Not that the parting speech caused Amelia to
philosophise, or that it armed her in any way with a calmness, the
result of argument; but it was intolerably dull, pompous, and tedious;
and having the fear of her schoolmistress greatly before her eyes, Miss
Sedley did not venture, in her presence, to give way to any ebullitions
of private grief.  A seed-cake and a bottle of wine were produced in
the drawing-room, as on the solemn occasions of the visits of parents,
and these refreshments being partaken of, Miss Sedley was at liberty to
depart.

``You'll go in and say good-by to Miss Pinkerton, Becky!'' said Miss
Jemima to a young lady of whom nobody took any notice, and who was
coming downstairs with her own bandbox.

``I suppose I must,'' said Miss Sharp calmly, and much to the wonder of
Miss Jemima; and the latter having knocked at the door, and receiving
permission to come in, Miss Sharp advanced in a very unconcerned
manner, and said in French, and with a perfect accent, ``\foreign{Mademoiselle,
je viens vous faire mes adieux.}'' % HOGG: period inside or outside the \foreign??

Miss Pinkerton did not understand French; she only directed those who
did: but biting her lips and throwing up her venerable and Roman-nosed
head (on the top of which figured a large and solemn turban), she said,
``Miss Sharp, I wish you a good morning.'' As the Hammersmith Semiramis
spoke, she waved one hand, both by way of adieu, and to give Miss Sharp
an opportunity of shaking one of the fingers of the hand which was left
out for that purpose.

Miss Sharp only folded her own hands with a very frigid smile and bow,
and quite declined to accept the proffered honour; on which Semiramis
tossed up her turban more indignantly than ever.  In fact, it was a
little battle between the young lady and the old one, and the latter
was worsted.  ``Heaven bless you, my child,'' said she, embracing Amelia,
and scowling the while over the girl's shoulder at Miss Sharp.  ``Come
away, Becky,'' said Miss Jemima, pulling the young woman away in great
alarm, and the drawing-room door closed upon them for ever.

Then came the struggle and parting below.  Words refuse to tell it. All
the servants were there in the hall---all the dear friends---all the young
ladies---the dancing-master who had just arrived; and there was such a
scuffling, and hugging, and kissing, and crying, with the hysterical
\emph{yoops} of Miss Swartz, the parlour-boarder, from her room, as no pen can % HOGG ``yoops''?
depict, and as the tender heart would fain pass over. The embracing was
over; they parted---that is, Miss Sedley parted from her friends.  Miss
Sharp had demurely entered the carriage some minutes before.  Nobody
cried for leaving \emph{her}.

Sambo of the bandy legs slammed the carriage door on his young weeping
mistress.  He sprang up behind the carriage.  ``Stop!'' cried Miss
Jemima, rushing to the gate with a parcel.

``It's some sandwiches, my dear,'' said she to Amelia. ``You may be
hungry, you know; and Becky, Becky Sharp, here's a book for you that my
sister---that is, I---Johnson's Dixonary, you know; you mustn't leave us
without that.  Good-by.  Drive on, coachman.  God bless you!''

And the kind creature retreated into the garden, overcome with emotion.

But, lo!\ and just as the coach drove off, Miss Sharp put her pale face % HOGG: note spacing after lo!
out of the window and actually flung the book back into the garden.

This almost caused Jemima to faint with terror.  ``Well, I never''---said
she---``what an audacious''---Emotion prevented her from completing either
sentence.  The carriage rolled away; the great gates were closed; the
bell rang for the dancing lesson.  The world is before the two young
ladies; and so, farewell to Chiswick Mall.



\chapter{In Which Miss Sharp and Miss Sedley Prepare to Open the Campaign}

When Miss Sharp had performed the heroical act mentioned in the last
chapter, and had seen the Dixonary, flying over the pavement of the
little garden, fall at length at the feet of the astonished Miss
Jemima, the young lady's countenance, which had before worn an almost
livid look of hatred, assumed a smile that perhaps was scarcely more
agreeable, and she sank back in the carriage in an easy frame of mind,
saying---``So much for the Dixonary; and, thank God, I'm out of Chiswick.''

Miss Sedley was almost as flurried at the act of defiance as Miss
Jemima had been; for, consider, it was but one minute that she had left
school, and the impressions of six years are not got over in that space
of time.  Nay, with some persons those awes and terrors of youth last
for ever and ever.  I know, for instance, an old gentleman of
sixty-eight, who said to me one morning at breakfast, with a very
agitated countenance, ``I dreamed last night that I was flogged by Dr.
Raine.'' Fancy had carried him back five-and-fifty years in the course
of that evening.  Dr. Raine and his rod were just as awful to him in
his heart, then, at sixty-eight, as they had been at thirteen.  If the
Doctor, with a large birch, had appeared bodily to him, even at the age
of threescore and eight, and had said in awful voice, ``Boy, take down
your pant---''? Well, well, Miss Sedley was exceedingly alarmed at this
act of insubordination.

``How could you do so, Rebecca?'' at last she said, after a pause.

``Why, do you think Miss Pinkerton will come out and order me back to
the black-hole?'' said Rebecca, laughing.

``No: but---''

``I hate the whole house,'' continued Miss Sharp in a fury.  ``I hope I
may never set eyes on it again.  I wish it were in the bottom of the
Thames, I do; and if Miss Pinkerton were there, I wouldn't pick her
out, that I wouldn't.  O how I should like to see her floating in the
water yonder, turban and all, with her train streaming after her, and
her nose like the beak of a wherry.''

``Hush!'' cried Miss Sedley.

``Why, will the black footman tell tales?'' cried Miss Rebecca, laughing.
``He may go back and tell Miss Pinkerton that I hate her with all my
soul; and I wish he would; and I wish I had a means of proving it, too.
For two years I have only had insults and outrage from her. I have been
treated worse than any servant in the kitchen. I have never had a
friend or a kind word, except from you.  I have been made to tend the
little girls in the lower schoolroom, and to talk French to the Misses,
until I grew sick of my mother tongue. But that talking French to Miss
Pinkerton was capital fun, wasn't it? She doesn't know a word of
French, and was too proud to confess it.  I believe it was that which
made her part with me; and so thank Heaven for French.  Vive la France!
Vive l'Empereur! Vive Bonaparte!''

``O Rebecca, Rebecca, for shame!'' cried Miss Sedley; for this was the
greatest blasphemy Rebecca had as yet uttered; and in those days, in
England, to say, ``Long live Bonaparte!'' was as much as to say, ``Long
live Lucifer!'' ``How can you---how dare you have such wicked, revengeful
thoughts?''

``Revenge may be wicked, but it's natural,'' answered Miss Rebecca. ``I'm
no angel.'' And, to say the truth, she certainly was not.

For it may be remarked in the course of this little conversation (which
took place as the coach rolled along lazily by the river side) that
though Miss Rebecca Sharp has twice had occasion to thank Heaven, it
has been, in the first place, for ridding her of some person whom she
hated, and secondly, for enabling her to bring her enemies to some sort
of perplexity or confusion; neither of which are very amiable motives
for religious gratitude, or such as would be put forward by persons of
a kind and placable disposition.  Miss Rebecca was not, then, in the
least kind or placable.  All the world used her ill, said this young
misanthropist, and we may be pretty certain that persons whom all the
world treats ill, deserve entirely the treatment they get.  The world
is a looking-glass, and gives back to every man the reflection of his
own face.  Frown at it, and it will in turn look sourly upon you; laugh
at it and with it, and it is a jolly kind companion; and so let all
young persons take their choice. This is certain, that if the world
neglected Miss Sharp, she never was known to have done a good action in
behalf of anybody; nor can it be expected that twenty-four young ladies
should all be as amiable as the heroine of this work, Miss Sedley (whom
we have selected for the very reason that she was the best-natured of
all, otherwise what on earth was to have prevented us from putting up
Miss Swartz, or Miss Crump, or Miss Hopkins, as heroine in her place!)
it could not be expected that every one should be of the humble and
gentle temper of Miss Amelia Sedley; should take every opportunity to
vanquish Rebecca's hard-heartedness and ill-humour; and, by a thousand
kind words and offices, overcome, for once at least, her hostility to
her kind.

Miss Sharp's father was an artist, and in that quality had given
lessons of drawing at Miss Pinkerton's school. He was a clever man; a
pleasant companion; a careless student; with a great propensity for
running into debt, and a partiality for the tavern.  When he was drunk,
he used to beat his wife and daughter; and the next morning, with a
headache, he would rail at the world for its neglect of his genius, and
abuse, with a good deal of cleverness, and sometimes with perfect
reason, the fools, his brother painters.  As it was with the utmost
difficulty that he could keep himself, and as he owed money for a mile
round Soho, where he lived, he thought to better his circumstances by
marrying a young woman of the French nation, who was by profession an
opera-girl.  The humble calling of her female parent Miss Sharp never
alluded to, but used to state subsequently that the Entrechats were a
noble family of Gascony, and took great pride in her descent from them.
And curious it is that as she advanced in life this young lady's
ancestors increased in rank and splendour.

Rebecca's mother had had some education somewhere, and her daughter
spoke French with purity and a Parisian accent.  It was in those days
rather a rare accomplishment, and led to her engagement with the
orthodox Miss Pinkerton.  For her mother being dead, her father,
finding himself not likely to recover, after his third attack of
delirium tremens, wrote a manly and pathetic letter to Miss Pinkerton,
recommending the orphan child to her protection, and so descended to
the grave, after two bailiffs had quarrelled over his corpse.  Rebecca
was seventeen when she came to Chiswick, and was bound over as an
articled pupil; her duties being to talk French, as we have seen; and
her privileges to live cost free, and, with a few guineas a year, to
gather scraps of knowledge from the professors who attended the school.

She was small and slight in person; pale, sandy-haired, and with eyes
habitually cast down: when they looked up they were very large, odd,
and attractive; so attractive that the Reverend Mr.\ Crisp, fresh from
Oxford, and curate to the Vicar of Chiswick, the Reverend Mr.\ %
Flowerdew, fell in love with Miss Sharp; being shot dead by a glance of
her eyes which was fired all the way across Chiswick Church from the
school-pew to the reading-desk.  This infatuated young man used
sometimes to take tea with Miss Pinkerton, to whom he had been
presented by his mamma, and actually proposed something like marriage
in an intercepted note, which the one-eyed apple-woman was charged to
deliver.  Mrs.\ Crisp was summoned from Buxton, and abruptly carried off
her darling boy; but the idea, even, of such an eagle in the Chiswick
dovecot caused a great flutter in the breast of Miss Pinkerton, who
would have sent away Miss Sharp but that she was bound to her under a
forfeit, and who never could thoroughly believe the young lady's
protestations that she had never exchanged a single word with Mr.\ %
Crisp, except under her own eyes on the two occasions when she had met
him at tea.

By the side of many tall and bouncing young ladies in the
establishment, Rebecca Sharp looked like a child.  But she had the
dismal precocity of poverty.  Many a dun had she talked to, and turned
away from her father's door; many a tradesman had she coaxed and
wheedled into good-humour, and into the granting of one meal more. She
sate commonly with her father, who was very proud of her wit, and heard
the talk of many of his wild companions---often but ill-suited for a
girl to hear.  But she never had been a girl, she said; she had been a
woman since she was eight years old.  Oh, why did Miss Pinkerton let
such a dangerous bird into her cage?

The fact is, the old lady believed Rebecca to be the meekest creature
in the world, so admirably, on the occasions when her father brought
her to Chiswick, used Rebecca to perform the part of the ingenue; and
only a year before the arrangement by which Rebecca had been admitted
into her house, and when Rebecca was sixteen years old, Miss Pinkerton
majestically, and with a little speech, made her a present of a
doll---which was, by the way, the confiscated property of Miss Swindle,
discovered surreptitiously nursing it in school-hours. How the father
and daughter laughed as they trudged home together after the evening
party (it was on the occasion of the speeches, when all the professors
were invited) and how Miss Pinkerton would have raged had she seen the
caricature of herself which the little mimic, Rebecca, managed to make
out of her doll. Becky used to go through dialogues with it; it formed
the delight of Newman Street, Gerrard Street, and the Artists' quarter:
and the young painters, when they came to take their gin-and-water with
their lazy, dissolute, clever, jovial senior, used regularly to ask
Rebecca if Miss Pinkerton was at home: she was as well known to them,
poor soul! as Mr.\ Lawrence or President West.  Once Rebecca had the
honour to pass a few days at Chiswick; after which she brought back
Jemima, and erected another doll as Miss Jemmy: for though that honest
creature had made and given her jelly and cake enough for three
children, and a seven-shilling piece at parting, the girl's sense of
ridicule was far stronger than her gratitude, and she sacrificed Miss
Jemmy quite as pitilessly as her sister.

The catastrophe came, and she was brought to the Mall as to her home.
The rigid formality of the place suffocated her: the prayers and the
meals, the lessons and the walks, which were arranged with a conventual
regularity, oppressed her almost beyond endurance; and she looked back
to the freedom and the beggary of the old studio in Soho with so much
regret, that everybody, herself included, fancied she was consumed with
grief for her father.  She had a little room in the garret, where the
maids heard her walking and sobbing at night; but it was with rage, and
not with grief.  She had not been much of a dissembler, until now her
loneliness taught her to feign. She had never mingled in the society of
women: her father, reprobate as he was, was a man of talent; his
conversation was a thousand times more agreeable to her than the talk
of such of her own sex as she now encountered. The pompous vanity of
the old schoolmistress, the foolish good-humour of her sister, the
silly chat and scandal of the elder girls, and the frigid correctness
of the governesses equally annoyed her; and she had no soft maternal
heart, this unlucky girl, otherwise the prattle and talk of the younger
children, with whose care she was chiefly intrusted, might have soothed
and interested her; but she lived among them two years, and not one was
sorry that she went away.  The gentle tender-hearted Amelia Sedley was
the only person to whom she could attach herself in the least; and who
could help attaching herself to Amelia?

The happiness---the superior advantages of the young women round about
her, gave Rebecca inexpressible pangs of envy.  ``What airs that girl
gives herself, because she is an Earl's grand-daughter,'' she said of
one.  ``How they cringe and bow to that Creole, because of her hundred
thousand pounds!  I am a thousand times cleverer and more charming than
that creature, for all her wealth. I am as well bred as the Earl's
grand-daughter, for all her fine pedigree; and yet every one passes me
by here.  And yet, when I was at my father's, did not the men give up
their gayest balls and parties in order to pass the evening with me?''
She determined at any rate to get free from the prison in which she
found herself, and now began to act for herself, and for the first time
to make connected plans for the future.

She took advantage, therefore, of the means of study the place offered
her; and as she was already a musician and a good linguist, she
speedily went through the little course of study which was considered
necessary for ladies in those days.  Her music she practised
incessantly, and one day, when the girls were out, and she had remained
at home, she was overheard to play a piece so well that Minerva
thought, wisely, she could spare herself the expense of a master for
the juniors, and intimated to Miss Sharp that she was to instruct them
in music for the future.

The girl refused; and for the first time, and to the astonishment of
the majestic mistress of the school.  ``I am here to speak French with
the children,'' Rebecca said abruptly, ``not to teach them music, and
save money for you.  Give me money, and I will teach them.''

Minerva was obliged to yield, and, of course, disliked her from that
day.  ``For five-and-thirty years,'' she said, and with great justice, ``I
never have seen the individual who has dared in my own house to
question my authority.  I have nourished a viper in my bosom.''

``A viper---a fiddlestick,'' said Miss Sharp to the old lady, almost
fainting with astonishment.  ``You took me because I was useful. There
is no question of gratitude between us.  I hate this place, and want to
leave it.  I will do nothing here but what I am obliged to do.''

It was in vain that the old lady asked her if she was aware she was
speaking to Miss Pinkerton?  Rebecca laughed in her face, with a horrid
sarcastic demoniacal laughter, that almost sent the schoolmistress into
fits. ``Give me a sum of money,'' said the girl, ``and get rid of me---or,
if you like better, get me a good place as governess in a nobleman's
family---you can do so if you please.''  And in their further disputes
she always returned to this point, ``Get me a situation---we hate each
other, and I am ready to go.''

Worthy Miss Pinkerton, although she had a Roman nose and a turban, and
was as tall as a grenadier, and had been up to this time an
irresistible princess, had no will or strength like that of her little
apprentice, and in vain did battle against her, and tried to overawe
her. Attempting once to scold her in public, Rebecca hit upon the
before-mentioned plan of answering her in French, which quite routed
the old woman.  In order to maintain authority in her school, it became
necessary to remove this rebel, this monster, this serpent, this
firebrand; and hearing about this time that Sir Pitt Crawley's family
was in want of a governess, she actually recommended Miss Sharp for the
situation, firebrand and serpent as she was.  ``I cannot, certainly,''
she said, ``find fault with Miss Sharp's conduct, except to myself; and
must allow that her talents and accomplishments are of a high order. As
far as the head goes, at least, she does credit to the educational
system pursued at my establishment.''

And so the schoolmistress reconciled the recommendation to her
conscience, and the indentures were cancelled, and the apprentice was
free.  The battle here described in a few lines, of course, lasted for
some months.  And as Miss Sedley, being now in her seventeenth year,
was about to leave school, and had a friendship for Miss Sharp (``'tis
the only point in Amelia's behaviour,'' said Minerva, ``which has not
been satisfactory to her mistress''), Miss Sharp was invited by her
friend to pass a week with her at home, before she entered upon her
duties as governess in a private family.

Thus the world began for these two young ladies.  For Amelia it was
quite a new, fresh, brilliant world, with all the bloom upon it.  It
was not quite a new one for Rebecca---(indeed, if the truth must be told
with respect to the Crisp affair, the tart-woman hinted to somebody,
who took an affidavit of the fact to somebody else, that there was a
great deal more than was made public regarding Mr.\ Crisp and Miss
Sharp, and that his letter was in answer to another letter).  But who
can tell you the real truth of the matter? At all events, if Rebecca
was not beginning the world, she was beginning it over again.

By the time the young ladies reached Kensington turnpike, Amelia had
not forgotten her companions, but had dried her tears, and had blushed
very much and been delighted at a young officer of the Life Guards, who
spied her as he was riding by, and said, ``A dem fine gal, egad!'' and
before the carriage arrived in Russell Square, a great deal of
conversation had taken place about the Drawing-room, and whether or not
young ladies wore powder as well as hoops when presented, and whether
she was to have that honour: to the Lord Mayor's ball she knew she was
to go.  And when at length home was reached, Miss Amelia Sedley skipped
out on Sambo's arm, as happy and as handsome a girl as any in the whole
big city of London.  Both he and coachman agreed on this point, and so
did her father and mother, and so did every one of the servants in the
house, as they stood bobbing, and curtseying, and smiling, in the hall
to welcome their young mistress.

You may be sure that she showed Rebecca over every room of the house,
and everything in every one of her drawers; and her books, and her
piano, and her dresses, and all her necklaces, brooches, laces, and
gimcracks. She insisted upon Rebecca accepting the white cornelian and
the turquoise rings, and a sweet sprigged muslin, which was too small
for her now, though it would fit her friend to a nicety; and she
determined in her heart to ask her mother's permission to present her
white Cashmere shawl to her friend.  Could she not spare it? and had
not her brother Joseph just brought her two from India?

When Rebecca saw the two magnificent Cashmere shawls which Joseph
Sedley had brought home to his sister, she said, with perfect truth,
``that it must be delightful to have a brother,'' and easily got the pity
of the tender-hearted Amelia for being alone in the world, an orphan
without friends or kindred.

``Not alone,'' said Amelia; ``you know, Rebecca, I shall always be your
friend, and love you as a sister---indeed I will.''

``Ah, but to have parents, as you have---kind, rich, affectionate
parents, who give you everything you ask for; and their love, which is
more precious than all! My poor papa could give me nothing, and I had
but two frocks in all the world! And then, to have a brother, a dear
brother! Oh, how you must love him!''

Amelia laughed.

``What! don't you love him? you, who say you love everybody?''

``Yes, of course, I do---only---''

``Only what?''

``Only Joseph doesn't seem to care much whether I love him or not. He
gave me two fingers to shake when he arrived after ten years' absence!
He is very kind and good, but he scarcely ever speaks to me; I think he
loves his pipe a great deal better than his''---but here Amelia checked
herself, for why should she speak ill of her brother? ``He was very kind
to me as a child,'' she added; ``I was but five years old when he went
away.''

``Isn't he very rich?'' said Rebecca.  ``They say all Indian nabobs are
enormously rich.''

``I believe he has a very large income.''

``And is your sister-in-law a nice pretty woman?''

``La! Joseph is not married,'' said Amelia, laughing again.

Perhaps she had mentioned the fact already to Rebecca, but that young
lady did not appear to have remembered it; indeed, vowed and protested
that she expected to see a number of Amelia's nephews and nieces.  She
was quite disappointed that Mr.\ Sedley was not married; she was sure
Amelia had said he was, and she doted so on little children.

``I think you must have had enough of them at Chiswick,'' said Amelia,
rather wondering at the sudden tenderness on her friend's part; and
indeed in later days Miss Sharp would never have committed herself so
far as to advance opinions, the untruth of which would have been so
easily detected.  But we must remember that she is but nineteen as yet,
unused to the art of deceiving, poor innocent creature! and making her
own experience in her own person.  The meaning of the above series of
queries, as translated in the heart of this ingenious young woman, was
simply this: ``If Mr.\ Joseph Sedley is rich and unmarried, why should I
not marry him? I have only a fortnight, to be sure, but there is no
harm in trying.'' And she determined within herself to make this
laudable attempt.  She redoubled her caresses to Amelia; she kissed the
white cornelian necklace as she put it on; and vowed she would never,
never part with it.  When the dinner-bell rang she went downstairs with
her arm round her friend's waist, as is the habit of young ladies. She
was so agitated at the drawing-room door, that she could hardly find
courage to enter.  ``Feel my heart, how it beats, dear!'' said she to her
friend.

``No, it doesn't,'' said Amelia.  ``Come in, don't be frightened.  Papa
won't do you any harm.''



\chapter{Rebecca Is in Presence of the Enemy}

A VERY stout, puffy man, in buckskins and Hessian boots, with several
immense neckcloths that rose almost to his nose, with a red striped
waistcoat and an apple green coat with steel buttons almost as large as
crown pieces (it was the morning costume of a dandy or blood of those
days) was reading the paper by the fire when the two girls entered, and
bounced off his arm-chair, and blushed excessively, and hid his entire
face almost in his neckcloths at this apparition.

``It's only your sister, Joseph,'' said Amelia, laughing and shaking the
two fingers which he held out.  ``I've come home FOR GOOD, you know; and
this is my friend, Miss Sharp, whom you have heard me mention.''

``No, never, upon my word,'' said the head under the neckcloth, shaking
very much---``that is, yes---what abominably cold weather, Miss''---and
herewith he fell to poking the fire with all his might, although it was
in the middle of June.

``He's very handsome,'' whispered Rebecca to Amelia, rather loud.

``Do you think so?'' said the latter.  ``I'll tell him.''

``Darling! not for worlds,'' said Miss Sharp, starting back as timid as a
fawn.  She had previously made a respectful virgin-like curtsey to the
gentleman, and her modest eyes gazed so perseveringly on the carpet
that it was a wonder how she should have found an opportunity to see
him.

``Thank you for the beautiful shawls, brother,'' said Amelia to the fire
poker.  ``Are they not beautiful, Rebecca?''

``O heavenly!'' said Miss Sharp, and her eyes went from the carpet
straight to the chandelier.

Joseph still continued a huge clattering at the poker and tongs,
puffing and blowing the while, and turning as red as his yellow face
would allow him.  ``I can't make you such handsome presents, Joseph,''
continued his sister, ``but while I was at school, I have embroidered
for you a very beautiful pair of braces.''

``Good Gad! Amelia,'' cried the brother, in serious alarm, ``what do you
mean?'' and plunging with all his might at the bell-rope, that article
of furniture came away in his hand, and increased the honest fellow's
confusion.  ``For heaven's sake see if my buggy's at the door.  I CAN'T
wait.  I must go.  D--------- that groom of mine. I must go.''

At this minute the father of the family walked in, rattling his seals
like a true British merchant.  ``What's the matter, Emmy?'' says he.

``Joseph wants me to see if his---his buggy is at the door.  What is a
buggy, Papa?''

``It is a one-horse palanquin,'' said the old gentleman, who was a wag in
his way.

Joseph at this burst out into a wild fit of laughter; in which,
encountering the eye of Miss Sharp, he stopped all of a sudden, as if
he had been shot.

``This young lady is your friend? Miss Sharp, I am very happy to see
you.  Have you and Emmy been quarrelling already with Joseph, that he
wants to be off?''

``I promised Bonamy of our service, sir,'' said Joseph, ``to dine with
him.''

``O fie! didn't you tell your mother you would dine here?''

``But in this dress it's impossible.''

``Look at him, isn't he handsome enough to dine anywhere, Miss Sharp?''

On which, of course, Miss Sharp looked at her friend, and they both set
off in a fit of laughter, highly agreeable to the old gentleman.

``Did you ever see a pair of buckskins like those at Miss Pinkerton's?''
continued he, following up his advantage.

``Gracious heavens! Father,'' cried Joseph.

``There now, I have hurt his feelings.  Mrs.\ Sedley, my dear, I have
hurt your son's feelings.  I have alluded to his buckskins.  Ask Miss
Sharp if I haven't? Come, Joseph, be friends with Miss Sharp, and let
us all go to dinner.''

``There's a pillau, Joseph, just as you like it, and Papa has brought
home the best turbot in Billingsgate.''

``Come, come, sir, walk downstairs with Miss Sharp, and I will follow
with these two young women,'' said the father, and he took an arm of
wife and daughter and walked merrily off.

If Miss Rebecca Sharp had determined in her heart upon making the
conquest of this big beau, I don't think, ladies, we have any right to
blame her; for though the task of husband-hunting is generally, and
with becoming modesty, entrusted by young persons to their mammas,
recollect that Miss Sharp had no kind parent to arrange these delicate
matters for her, and that if she did not get a husband for herself,
there was no one else in the wide world who would take the trouble off
her hands.  What causes young people to ``come out,'' but the noble
ambition of matrimony? What sends them trooping to watering-places?
What keeps them dancing till five o'clock in the morning through a
whole mortal season? What causes them to labour at pianoforte sonatas,
and to learn four songs from a fashionable master at a guinea a lesson,
and to play the harp if they have handsome arms and neat elbows, and to
wear Lincoln Green toxophilite hats and feathers, but that they may
bring down some ``desirable'' young man with those killing bows and
arrows of theirs? What causes respectable parents to take up their
carpets, set their houses topsy-turvy, and spend a fifth of their
year's income in ball suppers and iced champagne? Is it sheer love of
their species, and an unadulterated wish to see young people happy and
dancing? Psha! they want to marry their daughters; and, as honest Mrs.\ %
Sedley has, in the depths of her kind heart, already arranged a score
of little schemes for the settlement of her Amelia, so also had our
beloved but unprotected Rebecca determined to do her very best to
secure the husband, who was even more necessary for her than for her
friend. She had a vivid imagination; she had, besides, read the Arabian
Nights and Guthrie's Geography; and it is a fact that while she was
dressing for dinner, and after she had asked Amelia whether her brother
was very rich, she had built for herself a most magnificent castle in
the air, of which she was mistress, with a husband somewhere in the
background (she had not seen him as yet, and his figure would not
therefore be very distinct); she had arrayed herself in an infinity of
shawls, turbans, and diamond necklaces, and had mounted upon an
elephant to the sound of the march in Bluebeard, in order to pay a
visit of ceremony to the Grand Mogul. Charming Alnaschar visions! it is
the happy privilege of youth to construct you, and many a fanciful
young creature besides Rebecca Sharp has indulged in these delightful
day-dreams ere now!

Joseph Sedley was twelve years older than his sister Amelia.  He was in
the East India Company's Civil Service, and his name appeared, at the
period of which we write, in the Bengal division of the East India
Register, as collector of Boggley Wollah, an honourable and lucrative
post, as everybody knows: in order to know to what higher posts Joseph
rose in the service, the reader is referred to the same periodical.

Boggley Wollah is situated in a fine, lonely, marshy, jungly district,
famous for snipe-shooting, and where not unfrequently you may flush a
tiger.  Ramgunge, where there is a magistrate, is only forty miles off,
and there is a cavalry station about thirty miles farther; so Joseph
wrote home to his parents, when he took possession of his
collectorship.  He had lived for about eight years of his life, quite
alone, at this charming place, scarcely seeing a Christian face except
twice a year, when the detachment arrived to carry off the revenues
which he had collected, to Calcutta.

Luckily, at this time he caught a liver complaint, for the cure of
which he returned to Europe, and which was the source of great comfort
and amusement to him in his native country.  He did not live with his
family while in London, but had lodgings of his own, like a gay young
bachelor.  Before he went to India he was too young to partake of the
delightful pleasures of a man about town, and plunged into them on his
return with considerable assiduity.  He drove his horses in the Park;
he dined at the fashionable taverns (for the Oriental Club was not as
yet invented); he frequented the theatres, as the mode was in those
days, or made his appearance at the opera, laboriously attired in
tights and a cocked hat.

On returning to India, and ever after, he used to talk of the pleasure
of this period of his existence with great enthusiasm, and give you to
understand that he and Brummel were the leading bucks of the day.  But
he was as lonely here as in his jungle at Boggley Wollah.  He scarcely
knew a single soul in the metropolis: and were it not for his doctor,
and the society of his blue-pill, and his liver complaint, he must have
died of loneliness. He was lazy, peevish, and a bon-vivant; the
appearance of a lady frightened him beyond measure; hence it was but
seldom that he joined the paternal circle in Russell Square, where
there was plenty of gaiety, and where the jokes of his good-natured old
father frightened his amour-propre.  His bulk caused Joseph much
anxious thought and alarm; now and then he would make a desperate
attempt to get rid of his superabundant fat; but his indolence and love
of good living speedily got the better of these endeavours at reform,
and he found himself again at his three meals a day.  He never was well
dressed; but he took the hugest pains to adorn his big person, and
passed many hours daily in that occupation. His valet made a fortune
out of his wardrobe: his toilet-table was covered with as many pomatums
and essences as ever were employed by an old beauty: he had tried, in
order to give himself a waist, every girth, stay, and waistband then
invented.  Like most fat men, he would have his clothes made too tight,
and took care they should be of the most brilliant colours and youthful
cut.  When dressed at length, in the afternoon, he would issue forth to
take a drive with nobody in the Park; and then would come back in order
to dress again and go and dine with nobody at the Piazza Coffee-House.
He was as vain as a girl; and perhaps his extreme shyness was one of
the results of his extreme vanity. If Miss Rebecca can get the better
of him, and at her first entrance into life, she is a young person of
no ordinary cleverness.

The first move showed considerable skill.  When she called Sedley a
very handsome man, she knew that Amelia would tell her mother, who
would probably tell Joseph, or who, at any rate, would be pleased by
the compliment paid to her son.  All mothers are.  If you had told
Sycorax that her son Caliban was as handsome as Apollo, she would have
been pleased, witch as she was.  Perhaps, too, Joseph Sedley would
overhear the compliment---Rebecca spoke loud enough---and he did hear,
and (thinking in his heart that he was a very fine man) the praise
thrilled through every fibre of his big body, and made it tingle with
pleasure.  Then, however, came a recoil.  ``Is the girl making fun of
me?'' he thought, and straightway he bounced towards the bell, and was
for retreating, as we have seen, when his father's jokes and his
mother's entreaties caused him to pause and stay where he was.  He
conducted the young lady down to dinner in a dubious and agitated frame
of mind. ``Does she really think I am handsome?'' thought he, ``or is she
only making game of me?'' We have talked of Joseph Sedley being as vain
as a girl.  Heaven help us! the girls have only to turn the tables, and
say of one of their own sex, ``She is as vain as a man,'' and they will
have perfect reason.  The bearded creatures are quite as eager for
praise, quite as finikin over their toilettes, quite as proud of their
personal advantages, quite as conscious of their powers of fascination,
as any coquette in the world.

Downstairs, then, they went, Joseph very red and blushing, Rebecca very
modest, and holding her green eyes downwards.  She was dressed in
white, with bare shoulders as white as snow---the picture of youth,
unprotected innocence, and humble virgin simplicity. ``I must be very
quiet,'' thought Rebecca, ``and very much interested about India.''

Now we have heard how Mrs.\ Sedley had prepared a fine curry for her
son, just as he liked it, and in the course of dinner a portion of this
dish was offered to Rebecca.  ``What is it?'' said she, turning an
appealing look to Mr.\ Joseph.

``Capital,'' said he.  His mouth was full of it: his face quite red with
the delightful exercise of gobbling. ``Mother, it's as good as my own
curries in India.''

``Oh, I must try some, if it is an Indian dish,'' said Miss Rebecca. ``I
am sure everything must be good that comes from there.''

``Give Miss Sharp some curry, my dear,'' said Mr.\ Sedley, laughing.

Rebecca had never tasted the dish before.

``Do you find it as good as everything else from India?'' said Mr.\ Sedley.

``Oh, excellent!'' said Rebecca, who was suffering tortures with the
cayenne pepper.

``Try a chili with it, Miss Sharp,'' said Joseph, really interested.

``A chili,'' said Rebecca, gasping.  ``Oh yes!''  She thought a chili was
something cool, as its name imported, and was served with some. ``How
fresh and green they look,'' she said, and put one into her mouth.  It
was hotter than the curry; flesh and blood could bear it no longer.
She laid down her fork.  ``Water, for Heaven's sake, water!'' she cried.
Mr.\ Sedley burst out laughing (he was a coarse man, from the Stock
Exchange, where they love all sorts of practical jokes).  ``They are
real Indian, I assure you,'' said he.  ``Sambo, give Miss Sharp some
water.''

The paternal laugh was echoed by Joseph, who thought the joke capital.
The ladies only smiled a little.  They thought poor Rebecca suffered
too much.  She would have liked to choke old Sedley, but she swallowed
her mortification as well as she had the abominable curry before it,
and as soon as she could speak, said, with a comical, good-humoured
air, ``I ought to have remembered the pepper which the Princess of
Persia puts in the cream-tarts in the Arabian Nights.  Do you put
cayenne into your cream-tarts in India, sir?''

Old Sedley began to laugh, and thought Rebecca was a good-humoured
girl.  Joseph simply said, ``Cream-tarts, Miss? Our cream is very bad in
Bengal.  We generally use goats' milk; and, 'gad, do you know, I've got
to prefer it!''

``You won't like EVERYTHING from India now, Miss Sharp,'' said the old
gentleman; but when the ladies had retired after dinner, the wily old
fellow said to his son, ``Have a care, Joe; that girl is setting her cap
at you.''

``Pooh! nonsense!'' said Joe, highly flattered.  ``I recollect, sir, there
was a girl at Dumdum, a daughter of Cutler of the Artillery, and
afterwards married to Lance, the surgeon, who made a dead set at me in
the year '4---at me and Mulligatawney, whom I mentioned to you before
dinner---a devilish good fellow Mulligatawney---he's a magistrate at
Budgebudge, and sure to be in council in five years. Well, sir, the
Artillery gave a ball, and Quintin, of the King's 14th, said to me,
'Sedley,' said he, 'I bet you thirteen to ten that Sophy Cutler hooks
either you or Mulligatawney before the rains.' 'Done,' says I; and
egad, sir---this claret's very good.  Adamson's or Carbonell's?''

A slight snore was the only reply: the honest stockbroker was asleep,
and so the rest of Joseph's story was lost for that day. But he was
always exceedingly communicative in a man's party, and has told this
delightful tale many scores of times to his apothecary, Dr. Gollop,
when he came to inquire about the liver and the blue-pill.

Being an invalid, Joseph Sedley contented himself with a bottle of
claret besides his Madeira at dinner, and he managed a couple of plates
full of strawberries and cream, and twenty-four little rout cakes that
were lying neglected in a plate near him, and certainly (for novelists
have the privilege of knowing everything) he thought a great deal about
the girl upstairs.  ``A nice, gay, merry young creature,'' thought he to
himself.  ``How she looked at me when I picked up her handkerchief at
dinner!  She dropped it twice.  Who's that singing in the drawing-room?
'Gad! shall I go up and see?''

But his modesty came rushing upon him with uncontrollable force. His
father was asleep: his hat was in the hall: there was a hackney-coach
standing hard by in Southampton Row.  ``I'll go and see the Forty
Thieves,'' said he, ``and Miss Decamp's dance''; and he slipped away
gently on the pointed toes of his boots, and disappeared, without
waking his worthy parent.

``There goes Joseph,'' said Amelia, who was looking from the open windows
of the drawing-room, while Rebecca was singing at the piano.

``Miss Sharp has frightened him away,'' said Mrs.\ Sedley.  ``Poor Joe, why
WILL he be so shy?''



\chapter{The Green Silk Purse}

Poor Joe's panic lasted for two or three days; during which he did not
visit the house, nor during that period did Miss Rebecca ever mention
his name.  She was all respectful gratitude to Mrs.\ Sedley; delighted
beyond measure at the Bazaars; and in a whirl of wonder at the theatre,
whither the good-natured lady took her.  One day, Amelia had a
headache, and could not go upon some party of pleasure to which the two
young people were invited: nothing could induce her friend to go
without her. ``What! you who have shown the poor orphan what happiness
and love are for the first time in her life---quit YOU?  Never!''  and
the green eyes looked up to Heaven and filled with tears; and Mrs.\ %
Sedley could not but own that her daughter's friend had a charming kind
heart of her own.

As for Mr.\ Sedley's jokes, Rebecca laughed at them with a cordiality
and perseverance which not a little pleased and softened that
good-natured gentleman.  Nor was it with the chiefs of the family alone
that Miss Sharp found favour.  She interested Mrs.\ Blenkinsop by
evincing the deepest sympathy in the raspberry-jam preserving, which
operation was then going on in the Housekeeper's room; she persisted in
calling Sambo ``Sir,'' and ``Mr.\ Sambo,'' to the delight of that attendant;
and she apologised to the lady's maid for giving her trouble in
venturing to ring the bell, with such sweetness and humility, that the
Servants' Hall was almost as charmed with her as the Drawing Room.

Once, in looking over some drawings which Amelia had sent from school,
Rebecca suddenly came upon one which caused her to burst into tears and
leave the room. It was on the day when Joe Sedley made his second
appearance.

Amelia hastened after her friend to know the cause of this display of
feeling, and the good-natured girl came back without her companion,
rather affected too.  ``You know, her father was our drawing-master,
Mamma, at Chiswick, and used to do all the best parts of our drawings.''

``My love! I'm sure I always heard Miss Pinkerton say that he did not
touch them---he only mounted them.'' ``It was called mounting, Mamma.
Rebecca remembers the drawing, and her father working at it, and the
thought of it came upon her rather suddenly---and so, you know, she---''

``The poor child is all heart,'' said Mrs.\ Sedley.

``I wish she could stay with us another week,'' said Amelia.

``She's devilish like Miss Cutler that I used to meet at Dumdum, only
fairer.  She's married now to Lance, the Artillery Surgeon.  Do you
know, Ma'am, that once Quintin, of the 14th, bet me---''

``O Joseph, we know that story,'' said Amelia, laughing. ``Never mind about
telling that; but persuade Mamma to write to Sir Something Crawley for
leave of absence for poor dear Rebecca: here she comes, her eyes red
with weeping.''

``I'm better, now,'' said the girl, with the sweetest smile possible,
taking good-natured Mrs.\ Sedley's extended hand and kissing it
respectfully.  ``How kind you all are to me! All,'' she added, with a
laugh, ``except you, Mr.\ Joseph.''

``Me!'' said Joseph, meditating an instant departure. ``Gracious Heavens!
Good Gad! Miss Sharp!'

``Yes; how could you be so cruel as to make me eat that horrid
pepper-dish at dinner, the first day I ever saw you? You are not so
good to me as dear Amelia.''

``He doesn't know you so well,'' cried Amelia.

``I defy anybody not to be good to you, my dear,'' said her mother.

``The curry was capital; indeed it was,'' said Joe, quite gravely.
``Perhaps there was NOT enough citron juice in it---no, there was NOT.''

``And the chilis?''

``By Jove, how they made you cry out!'' said Joe, caught by the ridicule
of the circumstance, and exploding in a fit of laughter which ended
quite suddenly, as usual.

``I shall take care how I let YOU choose for me another time,'' said
Rebecca, as they went down again to dinner.  ``I didn't think men were
fond of putting poor harmless girls to pain.''

``By Gad, Miss Rebecca, I wouldn't hurt you for the world.''

``No,'' said she, ``I KNOW you wouldn't''; and then she gave him ever so
gentle a pressure with her little hand, and drew it back quite
frightened, and looked first for one instant in his face, and then down
at the carpet-rods; and I am not prepared to say that Joe's heart did
not thump at this little involuntary, timid, gentle motion of regard on
the part of the simple girl.

It was an advance, and as such, perhaps, some ladies of indisputable
correctness and gentility will condemn the action as immodest; but, you
see, poor dear Rebecca had all this work to do for herself.  If a
person is too poor to keep a servant, though ever so elegant, he must
sweep his own rooms: if a dear girl has no dear Mamma to settle matters
with the young man, she must do it for herself.  And oh, what a mercy
it is that these women do not exercise their powers oftener! We can't
resist them, if they do.  Let them show ever so little inclination, and
men go down on their knees at once: old or ugly, it is all the same.
And this I set down as a positive truth. A woman with fair
opportunities, and without an absolute hump, may marry WHOM SHE LIKES.
Only let us be thankful that the darlings are like the beasts of the
field, and don't know their own power.  They would overcome us entirely
if they did.

``Egad!'' thought Joseph, entering the dining-room, ``I exactly begin to
feel as I did at Dumdum with Miss Cutler.'' Many sweet little appeals,
half tender, half jocular, did Miss Sharp make to him about the dishes
at dinner; for by this time she was on a footing of considerable
familiarity with the family, and as for the girls, they loved each
other like sisters.  Young unmarried girls always do, if they are in a
house together for ten days.

As if bent upon advancing Rebecca's plans in every way---what must
Amelia do, but remind her brother of a promise made last Easter
holidays---``When I was a girl at school,'' said she, laughing---a promise
that he, Joseph, would take her to Vauxhall.  ``Now,'' she said, ``that
Rebecca is with us, will be the very time.''

``O, delightful!'' said Rebecca, going to clap her hands; but she
recollected herself, and paused, like a modest creature, as she was.

``To-night is not the night,'' said Joe.

``Well, to-morrow.''

``To-morrow your Papa and I dine out,'' said Mrs.\ Sedley.

``You don't suppose that I'm going, Mrs.\ Sed?'' said her husband, ``and
that a woman of your years and size is to catch cold, in such an
abominable damp place?''

``The children must have someone with them,'' cried Mrs.\ Sedley.

``Let Joe go,'' said-his father, laughing.  ``He's big enough.'' At which
speech even Mr.\ Sambo at the sideboard burst out laughing, and poor fat
Joe felt inclined to become a parricide almost.

``Undo his stays!'' continued the pitiless old gentleman. ``Fling some
water in his face, Miss Sharp, or carry him upstairs: the dear
creature's fainting.  Poor victim! carry him up; he's as light as a
feather!''

``If I stand this, sir, I'm d---------!'' roared Joseph.

``Order Mr.\ Jos's elephant, Sambo!'' cried the father. ``Send to Exeter
'Change, Sambo''; but seeing Jos ready almost to cry with vexation, the
old joker stopped his laughter, and said, holding out his hand to his
son, ``It's all fair on the Stock Exchange, Jos---and, Sambo, never mind
the elephant, but give me and Mr.\ Jos a glass of Champagne.  Boney
himself hasn't got such in his cellar, my boy!''

A goblet of Champagne restored Joseph's equanimity, and before the
bottle was emptied, of which as an invalid he took two-thirds, he had
agreed to take the young ladies to Vauxhall.

``The girls must have a gentleman apiece,'' said the old gentleman. ``Jos
will be sure to leave Emmy in the crowd, he will be so taken up with
Miss Sharp here.  Send to 96, and ask George Osborne if he'll come.''

At this, I don't know in the least for what reason, Mrs.\ Sedley looked
at her husband and laughed.  Mr.\ Sedley's eyes twinkled in a manner
indescribably roguish, and he looked at Amelia; and Amelia, hanging
down her head, blushed as only young ladies of seventeen know how to
blush, and as Miss Rebecca Sharp never blushed in her life---at least
not since she was eight years old, and when she was caught stealing jam
out of a cupboard by her godmother.  ``Amelia had better write a note,''
said her father; ``and let George Osborne see what a beautiful
handwriting we have brought back from Miss Pinkerton's.  Do you
remember when you wrote to him to come on Twelfth-night, Emmy, and
spelt twelfth without the f?''

``That was years ago,'' said Amelia.

``It seems like yesterday, don't it, John?'' said Mrs.\ Sedley to her
husband; and that night in a conversation which took place in a front
room in the second floor, in a sort of tent, hung round with chintz of
a rich and fantastic India pattern, and double with calico of a tender
rose-colour; in the interior of which species of marquee was a
featherbed, on which were two pillows, on which were two round red
faces, one in a laced nightcap, and one in a simple cotton one, ending
in a tassel---in a CURTAIN LECTURE, I say, Mrs.\ Sedley took her husband
to task for his cruel conduct to poor Joe.

``It was quite wicked of you, Mr.\ Sedley,'' said she, ``to torment the
poor boy so.''

``My dear,'' said the cotton-tassel in defence of his conduct, ``Jos is a
great deal vainer than you ever were in your life, and that's saying a
good deal.  Though, some thirty years ago, in the year seventeen
hundred and eighty---what was it?---perhaps you had a right to be vain---I
don't say no.  But I've no patience with Jos and his dandified modesty.
It is out-Josephing Joseph, my dear, and all the while the boy is only
thinking of himself, and what a fine fellow he is.  I doubt, Ma'am, we
shall have some trouble with him yet.  Here is Emmy's little friend
making love to him as hard as she can; that's quite clear; and if she
does not catch him some other will. That man is destined to be a prey
to woman, as I am to go on 'Change every day.  It's a mercy he did not
bring us over a black daughter-in-law, my dear.  But, mark my words,
the first woman who fishes for him, hooks him.''

``She shall go off to-morrow, the little artful creature,'' said Mrs.\ %
Sedley, with great energy.

``Why not she as well as another, Mrs.\ Sedley? The girl's a white face
at any rate.  I don't care who marries him.  Let Joe please himself.''

And presently the voices of the two speakers were hushed, or were
replaced by the gentle but unromantic music of the nose; and save when
the church bells tolled the hour and the watchman called it, all was
silent at the house of John Sedley, Esquire, of Russell Square, and the
Stock Exchange.

When morning came, the good-natured Mrs.\ Sedley no longer thought of
executing her threats with regard to Miss Sharp; for though nothing is
more keen, nor more common, nor more justifiable, than maternal
jealousy, yet she could not bring herself to suppose that the little,
humble, grateful, gentle governess would dare to look up to such a
magnificent personage as the Collector of Boggley Wollah. The petition,
too, for an extension of the young lady's leave of absence had already
been despatched, and it would be difficult to find a pretext for
abruptly dismissing her.

And as if all things conspired in favour of the gentle Rebecca, the
very elements (although she was not inclined at first to acknowledge
their action in her behalf) interposed to aid her.  For on the evening
appointed for the Vauxhall party, George Osborne having come to dinner,
and the elders of the house having departed, according to invitation,
to dine with Alderman Balls at Highbury Barn, there came on such a
thunder-storm as only happens on Vauxhall nights, and as obliged the
young people, perforce, to remain at home.  Mr.\ Osborne did not seem in
the least disappointed at this occurrence. He and Joseph Sedley drank a
fitting quantity of port-wine, tete-a-tete, in the dining-room, during
the drinking of which Sedley told a number of his best Indian stories;
for he was extremely talkative in man's society; and afterwards Miss
Amelia Sedley did the honours of the drawing-room; and these four young
persons passed such a comfortable evening together, that they declared
they were rather glad of the thunder-storm than otherwise, which had
caused them to put off their visit to Vauxhall.

Osborne was Sedley's godson, and had been one of the family any time
these three-and-twenty years.  At six weeks old, he had received from
John Sedley a present of a silver cup; at six months old, a coral with
gold whistle and bells; from his youth upwards he was ``tipped''
regularly by the old gentleman at Christmas: and on going back to
school, he remembered perfectly well being thrashed by Joseph Sedley,
when the latter was a big, swaggering hobbadyhoy, and George an
impudent urchin of ten years old.  In a word, George was as familiar
with the family as such daily acts of kindness and intercourse could
make him.

``Do you remember, Sedley, what a fury you were in, when I cut off the
tassels of your Hessian boots, and how Miss---hem!---how Amelia rescued
me from a beating, by falling down on her knees and crying out to her
brother Jos, not to beat little George?''

Jos remembered this remarkable circumstance perfectly well, but vowed
that he had totally forgotten it.

``Well, do you remember coming down in a gig to Dr. Swishtail's to see
me, before you went to India, and giving me half a guinea and a pat on
the head? I always had an idea that you were at least seven feet high,
and was quite astonished at your return from India to find you no
taller than myself.''

``How good of Mr.\ Sedley to go to your school and give you the money!''
exclaimed Rebecca, in accents of extreme delight.

``Yes, and after I had cut the tassels of his boots too. Boys never
forget those tips at school, nor the givers.''

``I delight in Hessian boots,'' said Rebecca.  Jos Sedley, who admired
his own legs prodigiously, and always wore this ornamental chaussure,
was extremely pleased at this remark, though he drew his legs under his
chair as it was made.

``Miss Sharp!'' said George Osborne, ``you who are so clever an artist,
you must make a grand historical picture of the scene of the boots.
Sedley shall be represented in buckskins, and holding one of the
injured boots in one hand; by the other he shall have hold of my
shirt-frill.  Amelia shall be kneeling near him, with her little hands
up; and the picture shall have a grand allegorical title, as the
frontispieces have in the Medulla and the spelling-book.''

``I shan't have time to do it here,'' said Rebecca.  ``I'll do it
when---when I'm gone.'' And she dropped her voice, and looked so sad and
piteous, that everybody felt how cruel her lot was, and how sorry they
would be to part with her.

``O that you could stay longer, dear Rebecca,'' said Amelia.

``Why?'' answered the other, still more sadly.  ``That I may be only the
more unhap---unwilling to lose you?'' And she turned away her head.
Amelia began to give way to that natural infirmity of tears which, we
have said, was one of the defects of this silly little thing.  George
Osborne looked at the two young women with a touched curiosity; and
Joseph Sedley heaved something very like a sigh out of his big chest,
as he cast his eyes down towards his favourite Hessian boots.

``Let us have some music, Miss Sedley---Amelia,'' said George, who felt at
that moment an extraordinary, almost irresistible impulse to seize the
above-mentioned young woman in his arms, and to kiss her in the face of
the company; and she looked at him for a moment, and if I should say
that they fell in love with each other at that single instant of time,
I should perhaps be telling an untruth, for the fact is that these two
young people had been bred up by their parents for this very purpose,
and their banns had, as it were, been read in their respective families
any time these ten years.  They went off to the piano, which was
situated, as pianos usually are, in the back drawing-room; and as it
was rather dark, Miss Amelia, in the most unaffected way in the world,
put her hand into Mr.\ Osborne's, who, of course, could see the way
among the chairs and ottomans a great deal better than she could.  But
this arrangement left Mr.\ Joseph Sedley tete-a-tete with Rebecca, at
the drawing-room table, where the latter was occupied in knitting a
green silk purse.

``There is no need to ask family secrets,'' said Miss Sharp.  ``Those two
have told theirs.''

``As soon as he gets his company,'' said Joseph, ``I believe the affair is
settled.  George Osborne is a capital fellow.''

``And your sister the dearest creature in the world,'' said Rebecca.
``Happy the man who wins her!'' With this, Miss Sharp gave a great sigh.

When two unmarried persons get together, and talk upon such delicate
subjects as the present, a great deal of confidence and intimacy is
presently established between them.  There is no need of giving a
special report of the conversation which now took place between Mr.\ %
Sedley and the young lady; for the conversation, as may be judged from
the foregoing specimen, was not especially witty or eloquent; it seldom
is in private societies, or anywhere except in very high-flown and
ingenious novels. As there was music in the next room, the talk was
carried on, of course, in a low and becoming tone, though, for the
matter of that, the couple in the next apartment would not have been
disturbed had the talking been ever so loud, so occupied were they with
their own pursuits.

Almost for the first time in his life, Mr.\ Sedley found himself
talking, without the least timidity or hesitation, to a person of the
other sex.  Miss Rebecca asked him a great number of questions about
India, which gave him an opportunity of narrating many interesting
anecdotes about that country and himself.  He described the balls at
Government House, and the manner in which they kept themselves cool in
the hot weather, with punkahs, tatties, and other contrivances; and he
was very witty regarding the number of Scotchmen whom Lord Minto, the
Governor-General, patronised; and then he described a tiger-hunt; and
the manner in which the mahout of his elephant had been pulled off his
seat by one of the infuriated animals.  How delighted Miss Rebecca was
at the Government balls, and how she laughed at the stories of the
Scotch aides-de-camp, and called Mr.\ Sedley a sad wicked satirical
creature; and how frightened she was at the story of the elephant! ``For
your mother's sake, dear Mr.\ Sedley,'' she said, ``for the sake of all
your friends, promise NEVER to go on one of those horrid expeditions.''

``Pooh, pooh, Miss Sharp,'' said he, pulling up his shirt-collars; ``the
danger makes the sport only the pleasanter.'' He had never been but once
at a tiger-hunt, when the accident in question occurred, and when he
was half killed---not by the tiger, but by the fright. And as he talked
on, he grew quite bold, and actually had the audacity to ask Miss
Rebecca for whom she was knitting the green silk purse? He was quite
surprised and delighted at his own graceful familiar manner.

``For any one who wants a purse,'' replied Miss Rebecca, looking at him
in the most gentle winning way. Sedley was going to make one of the
most eloquent speeches possible, and had begun---``O Miss Sharp, how---''
when some song which was performed in the other room came to an end,
and caused him to hear his own voice so distinctly that he stopped,
blushed, and blew his nose in great agitation.

``Did you ever hear anything like your brother's eloquence?'' whispered
Mr.\ Osborne to Amelia.  ``Why, your friend has worked miracles.''

``The more the better,'' said Miss Amelia; who, like almost all women who
are worth a pin, was a match-maker in her heart, and would have been
delighted that Joseph should carry back a wife to India.  She had, too,
in the course of this few days' constant intercourse, warmed into a
most tender friendship for Rebecca, and discovered a million of virtues
and amiable qualities in her which she had not perceived when they were
at Chiswick together.  For the affection of young ladies is of as rapid
growth as Jack's bean-stalk, and reaches up to the sky in a night.  It
is no blame to them that after marriage this Sehnsucht nach der Liebe
subsides.  It is what sentimentalists, who deal in very big words, call
a yearning after the Ideal, and simply means that women are commonly
not satisfied until they have husbands and children on whom they may
centre affections, which are spent elsewhere, as it were, in small
change.

Having expended her little store of songs, or having stayed long enough
in the back drawing-room, it now appeared proper to Miss Amelia to ask
her friend to sing.  ``You would not have listened to me,'' she said to
Mr.\ Osborne (though she knew she was telling a fib), ``had you heard
Rebecca first.''

``I give Miss Sharp warning, though,'' said Osborne, ``that, right or
wrong, I consider Miss Amelia Sedley the first singer in the world.''

``You shall hear,'' said Amelia; and Joseph Sedley was actually polite
enough to carry the candles to the piano. Osborne hinted that he should
like quite as well to sit in the dark; but Miss Sedley, laughing,
declined to bear him company any farther, and the two accordingly
followed Mr.\ Joseph.  Rebecca sang far better than her friend (though
of course Osborne was free to keep his opinion), and exerted herself to
the utmost, and, indeed, to the wonder of Amelia, who had never known
her perform so well.  She sang a French song, which Joseph did not
understand in the least, and which George confessed he did not
understand, and then a number of those simple ballads which were the
fashion forty years ago, and in which British tars, our King, poor
Susan, blue-eyed Mary, and the like, were the principal themes. They
are not, it is said, very brilliant, in a musical point of view, but
contain numberless good-natured, simple appeals to the affections,
which people understood better than the milk-and-water lagrime,
sospiri, and felicita of the eternal Donizettian music with which we
are favoured now-a-days.

Conversation of a sentimental sort, befitting the subject, was carried
on between the songs, to which Sambo, after he had brought the tea, the
delighted cook, and even Mrs.\ Blenkinsop, the housekeeper, condescended
to listen on the landing-place.

Among these ditties was one, the last of the concert, and to the
following effect:

Ah! bleak and barren was the moor, Ah! loud and piercing was the storm,
The cottage roof was shelter'd sure, The cottage hearth was bright and
warm---An orphan boy the lattice pass'd, And, as he mark'd its cheerful
glow, Felt doubly keen the midnight blast, And doubly cold the fallen
snow.

They mark'd him as he onward prest, With fainting heart and weary limb;
Kind voices bade him turn and rest, And gentle faces welcomed him. The
dawn is up---the guest is gone, The cottage hearth is blazing still;
Heaven pity all poor wanderers lone! Hark to the wind upon the hill!

It was the sentiment of the before-mentioned words, ``When I'm gone,''
over again.  As she came to the last words, Miss Sharp's ``deep-toned
voice faltered.'' Everybody felt the allusion to her departure, and to
her hapless orphan state.  Joseph Sedley, who was fond of music, and
soft-hearted, was in a state of ravishment during the performance of
the song, and profoundly touched at its conclusion. If he had had the
courage; if George and Miss Sedley had remained, according to the
former's proposal, in the farther room, Joseph Sedley's bachelorhood
would have been at an end, and this work would never have been written.
But at the close of the ditty, Rebecca quitted the piano, and giving
her hand to Amelia, walked away into the front drawing-room twilight;
and, at this moment, Mr.\ Sambo made his appearance with a tray,
containing sandwiches, jellies, and some glittering glasses and
decanters, on which Joseph Sedley's attention was immediately fixed.
When the parents of the house of Sedley returned from their
dinner-party, they found the young people so busy in talking, that they
had not heard the arrival of the carriage, and Mr.\ Joseph was in the
act of saying, ``My dear Miss Sharp, one little teaspoonful of jelly to
recruit you after your immense---your---your delightful exertions.''

``Bravo, Jos!'' said Mr.\ Sedley; on hearing the bantering of which
well-known voice, Jos instantly relapsed into an alarmed silence, and
quickly took his departure. He did not lie awake all night thinking
whether or not he was in love with Miss Sharp; the passion of love
never interfered with the appetite or the slumber of Mr.\ Joseph Sedley;
but he thought to himself how delightful it would be to hear such songs
as those after Cutcherry---what a distinguee girl she was---how she could
speak French better than the Governor-General's lady herself---and what
a sensation she would make at the Calcutta balls. ``It's evident the
poor devil's in love with me,'' thought he.  ``She is just as rich as
most of the girls who come out to India.  I might go farther, and fare
worse, egad!'' And in these meditations he fell asleep.

How Miss Sharp lay awake, thinking, will he come or not to-morrow? need
not be told here.  To-morrow came, and, as sure as fate, Mr.\ Joseph
Sedley made his appearance before luncheon.  He had never been known
before to confer such an honour on Russell Square. George Osborne was
somehow there already (sadly ``putting out'' Amelia, who was writing to
her twelve dearest friends at Chiswick Mall), and Rebecca was employed
upon her yesterday's work.  As Joe's buggy drove up, and while, after
his usual thundering knock and pompous bustle at the door, the
ex-Collector of Boggley Wollah laboured up stairs to the drawing-room,
knowing glances were telegraphed between Osborne and Miss Sedley, and
the pair, smiling archly, looked at Rebecca, who actually blushed as
she bent her fair ringlets over her knitting.  How her heart beat as
Joseph appeared---Joseph, puffing from the staircase in shining creaking
boots---Joseph, in a new waistcoat, red with heat and nervousness, and
blushing behind his wadded neckcloth.  It was a nervous moment for all;
and as for Amelia, I think she was more frightened than even the people
most concerned.

Sambo, who flung open the door and announced Mr.\ Joseph, followed
grinning, in the Collector's rear, and bearing two handsome nosegays of
flowers, which the monster had actually had the gallantry to purchase
in Covent Garden Market that morning---they were not as big as the
haystacks which ladies carry about with them now-a-days, in cones of
filigree paper; but the young women were delighted with the gift, as
Joseph presented one to each, with an exceedingly solemn bow.

``Bravo, Jos!'' cried Osborne.

``Thank you, dear Joseph,'' said Amelia, quite ready to kiss her brother,
if he were so minded.  (And I think for a kiss from such a dear
creature as Amelia, I would purchase all Mr.\ Lee's conservatories out
of hand.)

``O heavenly, heavenly flowers!'' exclaimed Miss Sharp, and smelt them
delicately, and held them to her bosom, and cast up her eyes to the
ceiling, in an ecstasy of admiration.  Perhaps she just looked first
into the bouquet, to see whether there was a billet-doux hidden among
the flowers; but there was no letter.

``Do they talk the language of flowers at Boggley Wollah, Sedley?'' asked
Osborne, laughing.

``Pooh, nonsense!'' replied the sentimental youth. ``Bought 'em at
Nathan's; very glad you like 'em; and eh, Amelia, my dear, I bought a
pine-apple at the same time, which I gave to Sambo.  Let's have it for
tiffin; very cool and nice this hot weather.'' Rebecca said she had
never tasted a pine, and longed beyond everything to taste one.

So the conversation went on.  I don't know on what pretext Osborne left
the room, or why, presently, Amelia went away, perhaps to superintend
the slicing of the pine-apple; but Jos was left alone with Rebecca, who
had resumed her work, and the green silk and the shining needles were
quivering rapidly under her white slender fingers.

``What a beautiful, BYOO-OOTIFUL song that was you sang last night, dear
Miss Sharp,'' said the Collector.  ``It made me cry almost; 'pon my
honour it did.''

``Because you have a kind heart, Mr.\ Joseph; all the Sedleys have, I
think.''

``It kept me awake last night, and I was trying to hum it this morning,
in bed; I was, upon my honour.  Gollop, my doctor, came in at eleven
(for I'm a sad invalid, you know, and see Gollop every day), and, 'gad!
there I was, singing away like---a robin.''

``O you droll creature! Do let me hear you sing it.''

``Me? No, you, Miss Sharp; my dear Miss Sharp, do sing it.''  ``Not now,
Mr.\ Sedley,'' said Rebecca, with a sigh.  ``My spirits are not equal to
it; besides, I must finish the purse.  Will you help me, Mr.\ Sedley?''
And before he had time to ask how, Mr.\ Joseph Sedley, of the East India
Company's service, was actually seated tete-a-tete with a young lady,
looking at her with a most killing expression; his arms stretched out
before her in an imploring attitude, and his hands bound in a web of
green silk, which she was unwinding.

In this romantic position Osborne and Amelia found the interesting
pair, when they entered to announce that tiffin was ready.  The skein
of silk was just wound round the card; but Mr.\ Jos had never spoken.

``I am sure he will to-night, dear,'' Amelia said, as she pressed
Rebecca's hand; and Sedley, too, had communed with his soul, and said
to himself, ``'Gad, I'll pop the question at Vauxhall.''



\chapter{Dobbin of Ours}

Cuff's fight with Dobbin, and the unexpected issue of that contest,
will long be remembered by every man who was educated at Dr.
Swishtail's famous school.  The latter Youth (who used to be called
Heigh-ho Dobbin, Gee-ho Dobbin, and by many other names indicative of
puerile contempt) was the quietest, the clumsiest, and, as it seemed,
the dullest of all Dr. Swishtail's young gentlemen. His parent was a
grocer in the city: and it was bruited abroad that he was admitted into
Dr. Swishtail's academy upon what are called ``mutual principles''---that
is to say, the expenses of his board and schooling were defrayed by his
father in goods, not money; and he stood there---most at the bottom of
the school---in his scraggy corduroys and jacket, through the seams of
which his great big bones were bursting---as the representative of so
many pounds of tea, candles, sugar, mottled-soap, plums (of which a
very mild proportion was supplied for the puddings of the
establishment), and other commodities.  A dreadful day it was for young
Dobbin when one of the youngsters of the school, having run into the
town upon a poaching excursion for hardbake and polonies, espied the
cart of Dobbin \& Rudge, Grocers and Oilmen, Thames Street, London, at
the Doctor's door, discharging a cargo of the wares in which the firm
dealt.

Young Dobbin had no peace after that.  The jokes were frightful, and
merciless against him.  ``Hullo, Dobbin,'' one wag would say, ``here's
good news in the paper.  Sugars is ris', my boy.'' Another would set a
sum---``If a pound of mutton-candles cost sevenpence-halfpenny, how much
must Dobbin cost?'' and a roar would follow from all the circle of young
knaves, usher and all, who rightly considered that the selling of goods
by retail is a shameful and infamous practice, meriting the contempt
and scorn of all real gentlemen.

``Your father's only a merchant, Osborne,'' Dobbin said in private to the
little boy who had brought down the storm upon him.  At which the
latter replied haughtily, ``My father's a gentleman, and keeps his
carriage''; and Mr.\ William Dobbin retreated to a remote outhouse in the
playground, where he passed a half-holiday in the bitterest sadness and
woe.  Who amongst us is there that does not recollect similar hours of
bitter, bitter childish grief? Who feels injustice; who shrinks before
a slight; who has a sense of wrong so acute, and so glowing a gratitude
for kindness, as a generous boy? and how many of those gentle souls do
you degrade, estrange, torture, for the sake of a little loose
arithmetic, and miserable dog-latin?

Now, William Dobbin, from an incapacity to acquire the rudiments of the
above language, as they are propounded in that wonderful book the Eton
Latin Grammar, was compelled to remain among the very last of Doctor
Swishtail's scholars, and was ``taken down'' continually by little
fellows with pink faces and pinafores when he marched up with the lower
form, a giant amongst them, with his downcast, stupefied look, his
dog's-eared primer, and his tight corduroys.  High and low, all made
fun of him.  They sewed up those corduroys, tight as they were. They
cut his bed-strings.  They upset buckets and benches, so that he might
break his shins over them, which he never failed to do.  They sent him
parcels, which, when opened, were found to contain the paternal soap
and candles.  There was no little fellow but had his jeer and joke at
Dobbin; and he bore everything quite patiently, and was entirely dumb
and miserable.

Cuff, on the contrary, was the great chief and dandy of the Swishtail
Seminary.  He smuggled wine in.  He fought the town-boys. Ponies used
to come for him to ride home on Saturdays.  He had his top-boots in his
room, in which he used to hunt in the holidays.  He had a gold
repeater: and took snuff like the Doctor.  He had been to the Opera,
and knew the merits of the principal actors, preferring Mr.\ Kean to Mr.\ %
Kemble.  He could knock you off forty Latin verses in an hour.  He
could make French poetry. What else didn't he know, or couldn't he do?
They said even the Doctor himself was afraid of him.

Cuff, the unquestioned king of the school, ruled over his subjects, and
bullied them, with splendid superiority. This one blacked his shoes:
that toasted his bread, others would fag out, and give him balls at
cricket during whole summer afternoons.  ``Figs'' was the fellow whom he
despised most, and with whom, though always abusing him, and sneering
at him, he scarcely ever condescended to hold personal communication.

One day in private, the two young gentlemen had had a difference. Figs,
alone in the schoolroom, was blundering over a home letter; when Cuff,
entering, bade him go upon some message, of which tarts were probably
the subject.

``I can't,'' says Dobbin; ``I want to finish my letter.''

``You CAN'T?'' says Mr.\ Cuff, laying hold of that document (in which many
words were scratched out, many were mis-spelt, on which had been spent
I don't know how much thought, and labour, and tears; for the poor
fellow was writing to his mother, who was fond of him, although she was
a grocer's wife, and lived in a back parlour in Thames Street).  ``You
CAN'T?'' says Mr.\ Cuff: ``I should like to know why, pray? Can't you
write to old Mother Figs to-morrow?''

``Don't call names,'' Dobbin said, getting off the bench very nervous.

``Well, sir, will you go?'' crowed the cock of the school.

``Put down the letter,'' Dobbin replied; ``no gentleman readth letterth.''

``Well, NOW will you go?'' says the other.

``No, I won't.  Don't strike, or I'll THMASH you,'' roars out Dobbin,
springing to a leaden inkstand, and looking so wicked, that Mr.\ Cuff
paused, turned down his coat sleeves again, put his hands into his
pockets, and walked away with a sneer.  But he never meddled personally
with the grocer's boy after that; though we must do him the justice to
say he always spoke of Mr.\ Dobbin with contempt behind his back.

Some time after this interview, it happened that Mr.\ Cuff, on a
sunshiny afternoon, was in the neighbourhood of poor William Dobbin,
who was lying under a tree in the playground, spelling over a favourite
copy of the Arabian Nights which he had apart from the rest of the
school, who were pursuing their various sports---quite lonely, and
almost happy.  If people would but leave children to themselves; if
teachers would cease to bully them; if parents would not insist upon
directing their thoughts, and dominating their feelings---those feelings
and thoughts which are a mystery to all (for how much do you and I know
of each other, of our children, of our fathers, of our neighbour, and
how far more beautiful and sacred are the thoughts of the poor lad or
girl whom you govern likely to be, than those of the dull and
world-corrupted person who rules him?)---if, I say, parents and masters
would leave their children alone a little more, small harm would
accrue, although a less quantity of as in praesenti might be acquired.

Well, William Dobbin had for once forgotten the world, and was away
with Sindbad the Sailor in the Valley of Diamonds, or with Prince Ahmed
and the Fairy Peribanou in that delightful cavern where the Prince
found her, and whither we should all like to make a tour; when shrill
cries, as of a little fellow weeping, woke up his pleasant reverie; and
looking up, he saw Cuff before him, belabouring a little boy.

It was the lad who had peached upon him about the grocer's cart; but he
bore little malice, not at least towards the young and small. ``How dare
you, sir, break the bottle?'' says Cuff to the little urchin, swinging a
yellow cricket-stump over him.

The boy had been instructed to get over the playground wall (at a
selected spot where the broken glass had been removed from the top, and
niches made convenient in the brick); to run a quarter of a mile; to
purchase a pint of rum-shrub on credit; to brave all the Doctor's
outlying spies, and to clamber back into the playground again; during
the performance of which feat, his foot had slipt, and the bottle was
broken, and the shrub had been spilt, and his pantaloons had been
damaged, and he appeared before his employer a perfectly guilty and
trembling, though harmless, wretch.

``How dare you, sir, break it?'' says Cuff; ``you blundering little thief.
You drank the shrub, and now you pretend to have broken the bottle.
Hold out your hand, sir.''

Down came the stump with a great heavy thump on the child's hand.  A
moan followed.  Dobbin looked up. The Fairy Peribanou had fled into the
inmost cavern with Prince Ahmed: the Roc had whisked away Sindbad the
Sailor out of the Valley of Diamonds out of sight, far into the clouds:
and there was everyday life before honest William; and a big boy
beating a little one without cause.

``Hold out your other hand, sir,'' roars Cuff to his little schoolfellow,
whose face was distorted with pain. Dobbin quivered, and gathered
himself up in his narrow old clothes.

``Take that, you little devil!'' cried Mr.\ Cuff, and down came the wicket
again on the child's hand.---Don't be horrified, ladies, every boy at a
public school has done it. Your children will so do and be done by, in
all probability.  Down came the wicket again; and Dobbin started up.

I can't tell what his motive was.  Torture in a public school is as
much licensed as the knout in Russia.  It would be ungentlemanlike (in
a manner) to resist it. Perhaps Dobbin's foolish soul revolted against
that exercise of tyranny; or perhaps he had a hankering feeling of
revenge in his mind, and longed to measure himself against that
splendid bully and tyrant, who had all the glory, pride, pomp,
circumstance, banners flying, drums beating, guards saluting, in the
place.  Whatever may have been his incentive, however, up he sprang,
and screamed out, ``Hold off, Cuff; don't bully that child any more; or
I'll---''

``Or you'll what?'' Cuff asked in amazement at this interruption. ``Hold
out your hand, you little beast.''

``I'll give you the worst thrashing you ever had in your life,'' Dobbin
said, in reply to the first part of Cuff's sentence; and little
Osborne, gasping and in tears, looked up with wonder and incredulity at
seeing this amazing champion put up suddenly to defend him: while
Cuff's astonishment was scarcely less.  Fancy our late monarch George
III when he heard of the revolt of the North American colonies: fancy
brazen Goliath when little David stepped forward and claimed a meeting;
and you have the feelings of Mr.\ Reginald Cuff when this rencontre was
proposed to him.

``After school,'' says he, of course; after a pause and a look, as much
as to say, ``Make your will, and communicate your last wishes to your
friends between this time and that.''

``As you please,'' Dobbin said.  ``You must be my bottle holder, Osborne.''

``Well, if you like,'' little Osborne replied; for you see his papa kept
a carriage, and he was rather ashamed of his champion.

Yes, when the hour of battle came, he was almost ashamed to say, ``Go
it, Figs''; and not a single other boy in the place uttered that cry for
the first two or three rounds of this famous combat; at the
commencement of which the scientific Cuff, with a contemptuous smile on
his face, and as light and as gay as if he was at a ball, planted his
blows upon his adversary, and floored that unlucky champion three times
running.  At each fall there was a cheer; and everybody was anxious to
have the honour of offering the conqueror a knee.

``What a licking I shall get when it's over,'' young Osborne thought,
picking up his man.  ``You'd best give in,'' he said to Dobbin; ``it's
only a thrashing, Figs, and you know I'm used to it.'' But Figs, all
whose limbs were in a quiver, and whose nostrils were breathing rage,
put his little bottle-holder aside, and went in for a fourth time.

As he did not in the least know how to parry the blows that were aimed
at himself, and Cuff had begun the attack on the three preceding
occasions, without ever allowing his enemy to strike, Figs now
determined that he would commence the engagement by a charge on his own
part; and accordingly, being a left-handed man, brought that arm into
action, and hit out a couple of times with all his might---once at Mr.\ %
Cuff's left eye, and once on his beautiful Roman nose.

Cuff went down this time, to the astonishment of the assembly. ``Well
hit, by Jove,'' says little Osborne, with the air of a connoisseur,
clapping his man on the back. ``Give it him with the left, Figs my boy.''

Figs's left made terrific play during all the rest of the combat. Cuff
went down every time.  At the sixth round, there were almost as many
fellows shouting out, ``Go it, Figs,'' as there were youths exclaiming,
``Go it, Cuff.'' At the twelfth round the latter champion was all abroad,
as the saying is, and had lost all presence of mind and power of attack
or defence.  Figs, on the contrary, was as calm as a quaker.  His face
being quite pale, his eyes shining open, and a great cut on his
underlip bleeding profusely, gave this young fellow a fierce and
ghastly air, which perhaps struck terror into many spectators.
Nevertheless, his intrepid adversary prepared to close for the
thirteenth time.

If I had the pen of a Napier, or a Bell's Life, I should like to
describe this combat properly.  It was the last charge of the
Guard---(that is, it would have been, only Waterloo had not yet taken
place)---it was Ney's column breasting the hill of La Haye Sainte,
bristling with ten thousand bayonets, and crowned with twenty
eagles---it was the shout of the beef-eating British, as leaping down
the hill they rushed to hug the enemy in the savage arms of battle---in
other words, Cuff coming up full of pluck, but quite reeling and
groggy, the Fig-merchant put in his left as usual on his adversary's
nose, and sent him down for the last time.

``I think that will do for him,'' Figs said, as his opponent dropped as
neatly on the green as I have seen Jack Spot's ball plump into the
pocket at billiards; and the fact is, when time was called, Mr.\ %
Reginald Cuff was not able, or did not choose, to stand up again.

And now all the boys set up such a shout for Figs as would have made
you think he had been their darling champion through the whole battle;
and as absolutely brought Dr. Swishtail out of his study, curious to
know the cause of the uproar.  He threatened to flog Figs violently, of
course; but Cuff, who had come to himself by this time, and was washing
his wounds, stood up and said, ``It's my fault, sir---not Figs'---not
Dobbin's.  I was bullying a little boy; and he served me right.'' By
which magnanimous speech he not only saved his conqueror a whipping,
but got back all his ascendancy over the boys which his defeat had
nearly cost him.

Young Osborne wrote home to his parents an account of the transaction.


Sugarcane House, Richmond, March, 18---

DEAR MAMA,---I hope you are quite well.  I should be much obliged to you
to send me a cake and five shillings. There has been a fight here
between Cuff \& Dobbin. Cuff, you know, was the Cock of the School.
They fought thirteen rounds, and Dobbin Licked.  So Cuff is now Only
Second Cock.  The fight was about me.  Cuff was licking me for breaking
a bottle of milk, and Figs wouldn't stand it.  We call him Figs because
his father is a Grocer---Figs \& Rudge, Thames St., City---I think as he
fought for me you ought to buy your Tea \& Sugar at his father's.  Cuff
goes home every Saturday, but can't this, because he has 2 Black Eyes.
He has a white Pony to come and fetch him, and a groom in livery on a
bay mare.  I wish my Papa would let me have a Pony, and I am

Your dutiful Son, GEORGE SEDLEY OSBORNE

P.S.---Give my love to little Emmy.  I am cutting her out a Coach in
cardboard.  Please not a seed-cake, but a plum-cake.


In consequence of Dobbin's victory, his character rose prodigiously in
the estimation of all his schoolfellows, and the name of Figs, which
had been a byword of reproach, became as respectable and popular a
nickname as any other in use in the school.  ``After all, it's not his
fault that his father's a grocer,'' George Osborne said, who, though a
little chap, had a very high popularity among the Swishtail youth; and
his opinion was received with great applause. It was voted low to sneer
at Dobbin about this accident of birth. ``Old Figs'' grew to be a name of
kindness and endearment; and the sneak of an usher jeered at him no
longer.

And Dobbin's spirit rose with his altered circumstances. He made
wonderful advances in scholastic learning.  The superb Cuff himself, at
whose condescension Dobbin could only blush and wonder, helped him on
with his Latin verses; ``coached'' him in play-hours: carried him
triumphantly out of the little-boy class into the middle-sized form;
and even there got a fair place for him.  It was discovered, that
although dull at classical learning, at mathematics he was uncommonly
quick.  To the contentment of all he passed third in algebra, and got a
French prize-book at the public Midsummer examination. You should have
seen his mother's face when Telemaque (that delicious romance) was
presented to him by the Doctor in the face of the whole school and the
parents and company, with an inscription to Gulielmo Dobbin.  All the
boys clapped hands in token of applause and sympathy.  His blushes, his
stumbles, his awkwardness, and the number of feet which he crushed as
he went back to his place, who shall describe or calculate? Old Dobbin,
his father, who now respected him for the first time, gave him two
guineas publicly; most of which he spent in a general tuck-out for the
school: and he came back in a tail-coat after the holidays.

Dobbin was much too modest a young fellow to suppose that this happy
change in all his circumstances arose from his own generous and manly
disposition: he chose, from some perverseness, to attribute his good
fortune to the sole agency and benevolence of little George Osborne, to
whom henceforth he vowed such a love and affection as is only felt by
children---such an affection, as we read in the charming fairy-book,
uncouth Orson had for splendid young Valentine his conqueror.  He flung
himself down at little Osborne's feet, and loved him. Even before they
were acquainted, he had admired Osborne in secret.  Now he was his
valet, his dog, his man Friday.  He believed Osborne to be the
possessor of every perfection, to be the handsomest, the bravest, the
most active, the cleverest, the most generous of created boys.  He
shared his money with him: bought him uncountable presents of knives,
pencil-cases, gold seals, toffee, Little Warblers, and romantic books,
with large coloured pictures of knights and robbers, in many of which
latter you might read inscriptions to George Sedley Osborne, Esquire,
from his attached friend William Dobbin---the which tokens of homage
George received very graciously, as became his superior merit.

So that Lieutenant Osborne, when coming to Russell Square on the day of
the Vauxhall party, said to the ladies, ``Mrs.\ Sedley, Ma'am, I hope you
have room; I've asked Dobbin of ours to come and dine here, and go with
us to Vauxhall.  He's almost as modest as Jos.''

``Modesty! pooh,'' said the stout gentleman, casting a vainqueur look at
Miss Sharp.

``He is---but you are incomparably more graceful, Sedley,'' Osborne added,
laughing.  ``I met him at the Bedford, when I went to look for you; and
I told him that Miss Amelia was come home, and that we were all bent on
going out for a night's pleasuring; and that Mrs.\ Sedley had forgiven
his breaking the punch-bowl at the child's party. Don't you remember
the catastrophe, Ma'am, seven years ago?''

``Over Mrs.\ Flamingo's crimson silk gown,'' said good-natured Mrs.\ %
Sedley.  ``What a gawky it was! And his sisters are not much more
graceful.  Lady Dobbin was at Highbury last night with three of them.
Such figures! my dears.''

``The Alderman's very rich, isn't he?'' Osborne said archly.  ``Don't you
think one of the daughters would be a good spec for me, Ma'am?''

``You foolish creature! Who would take you, I should like to know, with
your yellow face?''

``Mine a yellow face? Stop till you see Dobbin.  Why, he had the yellow
fever three times; twice at Nassau, and once at St.\ Kitts.''

``Well, well; yours is quite yellow enough for us.  Isn't it, Emmy?''
Mrs.\ Sedley said: at which speech Miss Amelia only made a smile and a
blush; and looking at Mr.\ George Osborne's pale interesting
countenance, and those beautiful black, curling, shining whiskers,
which the young gentleman himself regarded with no ordinary
complacency, she thought in her little heart that in His Majesty's
army, or in the wide world, there never was such a face or such a hero.
``I don't care about Captain Dobbin's complexion,'' she said, ``or about
his awkwardness. I shall always like him, I know,'' her little reason
being, that he was the friend and champion of George.

``There's not a finer fellow in the service,'' Osborne said, ``nor a
better officer, though he is not an Adonis, certainly.'' And he looked
towards the glass himself with much naivete; and in so doing, caught
Miss Sharp's eye fixed keenly upon him, at which he blushed a little,
and Rebecca thought in her heart, ``Ah, mon beau Monsieur! I think I
have YOUR gauge''---the little artful minx!

That evening, when Amelia came tripping into the drawing-room in a
white muslin frock, prepared for conquest at Vauxhall, singing like a
lark, and as fresh as a rose---a very tall ungainly gentleman, with
large hands and feet, and large ears, set off by a closely cropped head
of black hair, and in the hideous military frogged coat and cocked hat
of those times, advanced to meet her, and made her one of the clumsiest
bows that was ever performed by a mortal.

This was no other than Captain William Dobbin, of His Majesty's
Regiment of Foot, returned from yellow fever, in the West Indies, to
which the fortune of the service had ordered his regiment, whilst so
many of his gallant comrades were reaping glory in the Peninsula.

He had arrived with a knock so very timid and quiet that it was
inaudible to the ladies upstairs: otherwise, you may be sure Miss
Amelia would never have been so bold as to come singing into the room.
As it was, the sweet fresh little voice went right into the Captain's
heart, and nestled there.  When she held out her hand for him to shake,
before he enveloped it in his own, he paused, and thought---``Well, is it
possible---are you the little maid I remember in the pink frock, such a
short time ago---the night I upset the punch-bowl, just after I was
gazetted? Are you the little girl that George Osborne said should marry
him?  What a blooming young creature you seem, and what a prize the
rogue has got!'' All this he thought, before he took Amelia's hand into
his own, and as he let his cocked hat fall.

His history since he left school, until the very moment when we have
the pleasure of meeting him again, although not fully narrated, has
yet, I think, been indicated sufficiently for an ingenious reader by
the conversation in the last page.  Dobbin, the despised grocer, was
Alderman Dobbin---Alderman Dobbin was Colonel of the City Light Horse,
then burning with military ardour to resist the French Invasion.
Colonel Dobbin's corps, in which old Mr.\ Osborne himself was but an
indifferent corporal, had been reviewed by the Sovereign and the Duke
of York; and the colonel and alderman had been knighted.  His son had
entered the army: and young Osborne followed presently in the same
regiment.  They had served in the West Indies and in Canada.  Their
regiment had just come home, and the attachment of Dobbin to George
Osborne was as warm and generous now as it had been when the two were
schoolboys.

So these worthy people sat down to dinner presently. They talked about
war and glory, and Boney and Lord Wellington, and the last Gazette.  In
those famous days every gazette had a victory in it, and the two
gallant young men longed to see their own names in the glorious list,
and cursed their unlucky fate to belong to a regiment which had been
away from the chances of honour.  Miss Sharp kindled with this exciting
talk, but Miss Sedley trembled and grew quite faint as she heard it.
Mr.\ Jos told several of his tiger-hunting stories, finished the one
about Miss Cutler and Lance the surgeon; helped Rebecca to everything
on the table, and himself gobbled and drank a great deal.

He sprang to open the door for the ladies, when they retired, with the
most killing grace---and coming back to the table, filled himself bumper
after bumper of claret, which he swallowed with nervous rapidity.

``He's priming himself,'' Osborne whispered to Dobbin, and at length the
hour and the carriage arrived for Vauxhall.



\chapter{Vauxhall}

I know that the tune I am piping is a very mild one (although there are
some terrific chapters coming presently), and must beg the good-natured
reader to remember that we are only discoursing at present about a
stockbroker's family in Russell Square, who are taking walks, or
luncheon, or dinner, or talking and making love as people do in common
life, and without a single passionate and wonderful incident to mark
the progress of their loves.  The argument stands thus---Osborne, in
love with Amelia, has asked an old friend to dinner and to
Vauxhall---Jos Sedley is in love with Rebecca.  Will he marry her? That
is the great subject now in hand.

We might have treated this subject in the genteel, or in the romantic,
or in the facetious manner.  Suppose we had laid the scene in Grosvenor
Square, with the very same adventures---would not some people have
listened? Suppose we had shown how Lord Joseph Sedley fell in love, and
the Marquis of Osborne became attached to Lady Amelia, with the full
consent of the Duke, her noble father: or instead of the supremely
genteel, suppose we had resorted to the entirely low, and described
what was going on in Mr.\ Sedley's kitchen---how black Sambo was in love
with the cook (as indeed he was), and how he fought a battle with the
coachman in her behalf; how the knife-boy was caught stealing a cold
shoulder of mutton, and Miss Sedley's new femme de chambre refused to
go to bed without a wax candle; such incidents might be made to provoke
much delightful laughter, and be supposed to represent scenes of
``life.'' Or if, on the contrary, we had taken a fancy for the terrible,
and made the lover of the new femme de chambre a professional burglar,
who bursts into the house with his band, slaughters black Sambo at the
feet of his master, and carries off Amelia in her night-dress, not to
be let loose again till the third volume, we should easily have
constructed a tale of thrilling interest, through the fiery chapters of
which the reader should hurry, panting.  But my readers must hope for
no such romance, only a homely story, and must be content with a
chapter about Vauxhall, which is so short that it scarce deserves to be
called a chapter at all.  And yet it is a chapter, and a very important
one too.  Are not there little chapters in everybody's life, that seem
to be nothing, and yet affect all the rest of the history?

Let us then step into the coach with the Russell Square party, and be
off to the Gardens.  There is barely room between Jos and Miss Sharp,
who are on the front seat.  Mr.\ Osborne sitting bodkin opposite,
between Captain Dobbin and Amelia.

Every soul in the coach agreed that on that night Jos would propose to
make Rebecca Sharp Mrs.\ Sedley.  The parents at home had acquiesced in
the arrangement, though, between ourselves, old Mr.\ Sedley had a
feeling very much akin to contempt for his son.  He said he was vain,
selfish, lazy, and effeminate.  He could not endure his airs as a man
of fashion, and laughed heartily at his pompous braggadocio stories.
``I shall leave the fellow half my property,'' he said; ``and he will
have, besides, plenty of his own; but as I am perfectly sure that if
you, and I, and his sister were to die to-morrow, he would say 'Good
Gad!' and eat his dinner just as well as usual, I am not going to make
myself anxious about him. Let him marry whom he likes.  It's no affair
of mine.''

Amelia, on the other hand, as became a young woman of her prudence and
temperament, was quite enthusiastic for the match.  Once or twice Jos
had been on the point of saying something very important to her, to
which she was most willing to lend an ear, but the fat fellow could not
be brought to unbosom himself of his great secret, and very much to his
sister's disappointment he only rid himself of a large sigh and turned
away.

This mystery served to keep Amelia's gentle bosom in a perpetual
flutter of excitement.  If she did not speak with Rebecca on the tender
subject, she compensated herself with long and intimate conversations
with Mrs.\ Blenkinsop, the housekeeper, who dropped some hints to the
lady's-maid, who may have cursorily mentioned the matter to the cook,
who carried the news, I have no doubt, to all the tradesmen, so that
Mr.\ Jos's marriage was now talked of by a very considerable number of
persons in the Russell Square world.

It was, of course, Mrs.\ Sedley's opinion that her son would demean
himself by a marriage with an artist's daughter.  ``But, lor', Ma'am,''
ejaculated Mrs.\ Blenkinsop, ``we was only grocers when we married Mr.\ %
S., who was a stock-broker's clerk, and we hadn't five hundred pounds
among us, and we're rich enough now.'' And Amelia was entirely of this
opinion, to which, gradually, the good-natured Mrs.\ Sedley was brought.

Mr.\ Sedley was neutral.  ``Let Jos marry whom he likes,'' he said; ``it's
no affair of mine.  This girl has no fortune; no more had Mrs.\ Sedley.
She seems good-humoured and clever, and will keep him in order,
perhaps.  Better she, my dear, than a black Mrs.\ Sedley, and a dozen of
mahogany grandchildren.''

So that everything seemed to smile upon Rebecca's fortunes.  She took
Jos's arm, as a matter of course, on going to dinner; she had sate by
him on the box of his open carriage (a most tremendous ``buck'' he was,
as he sat there, serene, in state, driving his greys), and though
nobody said a word on the subject of the marriage, everybody seemed to
understand it.  All she wanted was the proposal, and ah! how Rebecca
now felt the want of a mother!---a dear, tender mother, who would have
managed the business in ten minutes, and, in the course of a little
delicate confidential conversation, would have extracted the
interesting avowal from the bashful lips of the young man!

Such was the state of affairs as the carriage crossed Westminster
bridge.

The party was landed at the Royal Gardens in due time. As the majestic
Jos stepped out of the creaking vehicle the crowd gave a cheer for the
fat gentleman, who blushed and looked very big and mighty, as he walked
away with Rebecca under his arm.  George, of course, took charge of
Amelia.  She looked as happy as a rose-tree in sunshine.

``I say, Dobbin,'' says George, ``just look to the shawls and things,
there's a good fellow.'' And so while he paired off with Miss Sedley,
and Jos squeezed through the gate into the gardens with Rebecca at his
side, honest Dobbin contented himself by giving an arm to the shawls,
and by paying at the door for the whole party.

He walked very modestly behind them.  He was not willing to spoil
sport.  About Rebecca and Jos he did not care a fig.  But he thought
Amelia worthy even of the brilliant George Osborne, and as he saw that
good-looking couple threading the walks to the girl's delight and
wonder, he watched her artless happiness with a sort of fatherly
pleasure.  Perhaps he felt that he would have liked to have something
on his own arm besides a shawl (the people laughed at seeing the gawky
young officer carrying this female burthen); but William Dobbin was
very little addicted to selfish calculation at all; and so long as his
friend was enjoying himself, how should he be discontented? And the
truth is, that of all the delights of the Gardens; of the hundred
thousand extra lamps, which were always lighted; the fiddlers in cocked
hats, who played ravishing melodies under the gilded cockle-shell in
the midst of the gardens; the singers, both of comic and sentimental
ballads, who charmed the ears there; the country dances, formed by
bouncing cockneys and cockneyesses, and executed amidst jumping,
thumping and laughter; the signal which announced that Madame Saqui was
about to mount skyward on a slack-rope ascending to the stars; the
hermit that always sat in the illuminated hermitage; the dark walks, so
favourable to the interviews of young lovers; the pots of stout handed
about by the people in the shabby old liveries; and the twinkling
boxes, in which the happy feasters made-believe to eat slices of almost
invisible ham---of all these things, and of the gentle Simpson, that
kind smiling idiot, who, I daresay, presided even then over the
place---Captain William Dobbin did not take the slightest notice.

He carried about Amelia's white cashmere shawl, and having attended
under the gilt cockle-shell, while Mrs.\ Salmon performed the Battle of
Borodino (a savage cantata against the Corsican upstart, who had lately
met with his Russian reverses)---Mr.\ Dobbin tried to hum it as he walked
away, and found he was humming---the tune which Amelia Sedley sang on
the stairs, as she came down to dinner.

He burst out laughing at himself; for the truth is, he could sing no
better than an owl.

It is to be understood, as a matter of course, that our young people,
being in parties of two and two, made the most solemn promises to keep
together during the evening, and separated in ten minutes afterwards.
Parties at Vauxhall always did separate, but 'twas only to meet again
at supper-time, when they could talk of their mutual adventures in the
interval.

What were the adventures of Mr.\ Osborne and Miss Amelia? That is a
secret.  But be sure of this---they were perfectly happy, and correct in
their behaviour; and as they had been in the habit of being together
any time these fifteen years, their tete-a-tete offered no particular
novelty.

But when Miss Rebecca Sharp and her stout companion lost themselves in
a solitary walk, in which there were not above five score more of
couples similarly straying, they both felt that the situation was
extremely tender and critical, and now or never was the moment Miss
Sharp thought, to provoke that declaration which was trembling on the
timid lips of Mr.\ Sedley.  They had previously been to the panorama of
Moscow, where a rude fellow, treading on Miss Sharp's foot, caused her
to fall back with a little shriek into the arms of Mr.\ Sedley, and this
little incident increased the tenderness and confidence of that
gentleman to such a degree, that he told her several of his favourite
Indian stories over again for, at least, the sixth time.

``How I should like to see India!'' said Rebecca.

``SHOULD you?'' said Joseph, with a most killing tenderness; and was no
doubt about to follow up this artful interrogatory by a question still
more tender (for he puffed and panted a great deal, and Rebecca's hand,
which was placed near his heart, could count the feverish pulsations of
that organ), when, oh, provoking! the bell rang for the fireworks, and,
a great scuffling and running taking place, these interesting lovers
were obliged to follow in the stream of people.

Captain Dobbin had some thoughts of joining the party at supper: as, in
truth, he found the Vauxhall amusements not particularly lively---but he
paraded twice before the box where the now united couples were met, and
nobody took any notice of him.  Covers were laid for four.  The mated
pairs were prattling away quite happily, and Dobbin knew he was as
clean forgotten as if he had never existed in this world.

``I should only be de trop,'' said the Captain, looking at them rather
wistfully.  ``I'd best go and talk to the hermit,''---and so he strolled
off out of the hum of men, and noise, and clatter of the banquet, into
the dark walk, at the end of which lived that well-known pasteboard
Solitary.  It wasn't very good fun for Dobbin---and, indeed, to be alone
at Vauxhall, I have found, from my own experience, to be one of the
most dismal sports ever entered into by a bachelor.

The two couples were perfectly happy then in their box: where the most
delightful and intimate conversation took place.  Jos was in his glory,
ordering about the waiters with great majesty.  He made the salad; and
uncorked the Champagne; and carved the chickens; and ate and drank the
greater part of the refreshments on the tables. Finally, he insisted
upon having a bowl of rack punch; everybody had rack punch at Vauxhall.
``Waiter, rack punch.''

That bowl of rack punch was the cause of all this history.  And why not
a bowl of rack punch as well as any other cause? Was not a bowl of
prussic acid the cause of Fair Rosamond's retiring from the world? Was
not a bowl of wine the cause of the demise of Alexander the Great, or,
at least, does not Dr. Lempriere say so?---so did this bowl of rack
punch influence the fates of all the principal characters in this
``Novel without a Hero,'' which we are now relating.  It influenced their
life, although most of them did not taste a drop of it.

The young ladies did not drink it; Osborne did not like it; and the
consequence was that Jos, that fat gourmand, drank up the whole
contents of the bowl; and the consequence of his drinking up the whole
contents of the bowl was a liveliness which at first was astonishing,
and then became almost painful; for he talked and laughed so loud as to
bring scores of listeners round the box, much to the confusion of the
innocent party within it; and, volunteering to sing a song (which he
did in that maudlin high key peculiar to gentlemen in an inebriated
state), he almost drew away the audience who were gathered round the
musicians in the gilt scollop-shell, and received from his hearers a
great deal of applause.

``Brayvo, Fat un!'' said one; ``Angcore, Daniel Lambert!'' said another;
``What a figure for the tight-rope!'' exclaimed another wag, to the
inexpressible alarm of the ladies, and the great anger of Mr.\ Osborne.

``For Heaven's sake, Jos, let us get up and go,'' cried that gentleman,
and the young women rose.

``Stop, my dearest diddle-diddle-darling,'' shouted Jos, now as bold as a
lion, and clasping Miss Rebecca round the waist.  Rebecca started, but
she could not get away her hand.  The laughter outside redoubled.  Jos
continued to drink, to make love, and to sing; and, winking and waving
his glass gracefully to his audience, challenged all or any to come in
and take a share of his punch.

Mr.\ Osborne was just on the point of knocking down a gentleman in
top-boots, who proposed to take advantage of this invitation, and a
commotion seemed to be inevitable, when by the greatest good luck a
gentleman of the name of Dobbin, who had been walking about the
gardens, stepped up to the box.  ``Be off, you fools!'' said this
gentleman---shouldering off a great number of the crowd, who vanished
presently before his cocked hat and fierce appearance---and he entered
the box in a most agitated state.

``Good Heavens! Dobbin, where have you been?'' Osborne said, seizing the
white cashmere shawl from his friend's arm, and huddling up Amelia in
it.---``Make yourself useful, and take charge of Jos here, whilst I take
the ladies to the carriage.''

Jos was for rising to interfere---but a single push from Osborne's
finger sent him puffing back into his seat again, and the lieutenant
was enabled to remove the ladies in safety.  Jos kissed his hand to
them as they retreated, and hiccupped out ``Bless you! Bless you!'' Then,
seizing Captain Dobbin's hand, and weeping in the most pitiful way, he
confided to that gentleman the secret of his loves.  He adored that
girl who had just gone out; he had broken her heart, he knew he had, by
his conduct; he would marry her next morning at St.\ George's, Hanover
Square; he'd knock up the Archbishop of Canterbury at Lambeth: he
would, by Jove! and have him in readiness; and, acting on this hint,
Captain Dobbin shrewdly induced him to leave the gardens and hasten to
Lambeth Palace, and, when once out of the gates, easily conveyed Mr.\ %
Jos Sedley into a hackney-coach, which deposited him safely at his
lodgings.

George Osborne conducted the girls home in safety: and when the door
was closed upon them, and as he walked across Russell Square, laughed
so as to astonish the watchman.  Amelia looked very ruefully at her
friend, as they went up stairs, and kissed her, and went to bed without
any more talking.

``He must propose to-morrow,'' thought Rebecca.  ``He called me his soul's
darling, four times; he squeezed my hand in Amelia's presence.  He must
propose to-morrow.'' And so thought Amelia, too. And I dare say she
thought of the dress she was to wear as bridesmaid, and of the presents
which she should make to her nice little sister-in-law, and of a
subsequent ceremony in which she herself might play a principal part,
\&c., and \&c., and \&c., and \&c.

Oh, ignorant young creatures! How little do you know the effect of rack
punch! What is the rack in the punch, at night, to the rack in the head
of a morning? To this truth I can vouch as a man; there is no headache
in the world like that caused by Vauxhall punch. Through the lapse of
twenty years, I can remember the consequence of two glasses! two
wine-glasses! but two, upon the honour of a gentleman; and Joseph
Sedley, who had a liver complaint, had swallowed at least a quart of
the abominable mixture.

That next morning, which Rebecca thought was to dawn upon her fortune,
found Sedley groaning in agonies which the pen refuses to describe.
Soda-water was not invented yet.  Small beer---will it be believed!---was
the only drink with which unhappy gentlemen soothed the fever of their
previous night's potation.  With this mild beverage before him, George
Osborne found the ex-Collector of Boggley Wollah groaning on the sofa
at his lodgings.  Dobbin was already in the room, good-naturedly
tending his patient of the night before.  The two officers, looking at
the prostrate Bacchanalian, and askance at each other, exchanged the
most frightful sympathetic grins.  Even Sedley's valet, the most solemn
and correct of gentlemen, with the muteness and gravity of an
undertaker, could hardly keep his countenance in order, as he looked at
his unfortunate master.

``Mr.\ Sedley was uncommon wild last night, sir,'' he whispered in
confidence to Osborne, as the latter mounted the stair.  ``He wanted to
fight the 'ackney-coachman, sir. The Capting was obliged to bring him
upstairs in his harms like a babby.'' A momentary smile flickered over
Mr.\ Brush's features as he spoke; instantly, however, they relapsed
into their usual unfathomable calm, as he flung open the drawing-room
door, and announced ``Mr.\ Hosbin.''

``How are you, Sedley?'' that young wag began, after surveying his
victim.  ``No bones broke? There's a hackney-coachman downstairs with a
black eye, and a tied-up head, vowing he'll have the law of you.''

``What do you mean---law?'' Sedley faintly asked.

``For thrashing him last night---didn't he, Dobbin? You hit out, sir,
like Molyneux.  The watchman says he never saw a fellow go down so
straight.  Ask Dobbin.''

``You DID have a round with the coachman,'' Captain Dobbin said, ``and
showed plenty of fight too.''

``And that fellow with the white coat at Vauxhall! How Jos drove at him!
How the women screamed! By Jove, sir, it did my heart good to see you.
I thought you civilians had no pluck; but I'll never get in your way
when you are in your cups, Jos.''

``I believe I'm very terrible, when I'm roused,'' ejaculated Jos from the
sofa, and made a grimace so dreary and ludicrous, that the Captain's
politeness could restrain him no longer, and he and Osborne fired off a
ringing volley of laughter.

Osborne pursued his advantage pitilessly.  He thought Jos a milksop. He
had been revolving in his mind the marriage question pending between
Jos and Rebecca, and was not over well pleased that a member of a
family into which he, George Osborne, of the ---th, was going to marry,
should make a mesalliance with a little nobody---a little upstart
governess.  ``You hit, you poor old fellow!'' said Osborne. ``You
terrible! Why, man, you couldn't stand---you made everybody laugh in the
Gardens, though you were crying yourself.  You were maudlin, Jos.
Don't you remember singing a song?''

``A what?'' Jos asked.

``A sentimental song, and calling Rosa, Rebecca, what's her name,
Amelia's little friend---your dearest diddle-diddle-darling?'' And this
ruthless young fellow, seizing hold of Dobbin's hand, acted over the
scene, to the horror of the original performer, and in spite of
Dobbin's good-natured entreaties to him to have mercy.

``Why should I spare him?'' Osborne said to his friend's remonstrances,
when they quitted the invalid, leaving him under the hands of Doctor
Gollop.  ``What the deuce right has he to give himself his patronizing
airs, and make fools of us at Vauxhall? Who's this little schoolgirl
that is ogling and making love to him? Hang it, the family's low enough
already, without HER.  A governess is all very well, but I'd rather
have a lady for my sister-in-law. I'm a liberal man; but I've proper
pride, and know my own station: let her know hers.  And I'll take down
that great hectoring Nabob, and prevent him from being made a greater
fool than he is.  That's why I told him to look out, lest she brought
an action against him.''

``I suppose you know best,'' Dobbin said, though rather dubiously. ``You
always were a Tory, and your family's one of the oldest in England.
But---''

``Come and see the girls, and make love to Miss Sharp yourself,'' the
lieutenant here interrupted his friend; but Captain Dobbin declined to
join Osborne in his daily visit to the young ladies in Russell Square.

As George walked down Southampton Row, from Holborn, he laughed as he
saw, at the Sedley Mansion, in two different stories two heads on the
look-out.

The fact is, Miss Amelia, in the drawing-room balcony, was looking very
eagerly towards the opposite side of the Square, where Mr.\ Osborne
dwelt, on the watch for the lieutenant himself; and Miss Sharp, from
her little bed-room on the second floor, was in observation until Mr.\ %
Joseph's great form should heave in sight.

``Sister Anne is on the watch-tower,'' said he to Amelia, ``but there's
nobody coming''; and laughing and enjoying the joke hugely, he described
in the most ludicrous terms to Miss Sedley, the dismal condition of her
brother.

``I think it's very cruel of you to laugh, George,'' she said, looking
particularly unhappy; but George only laughed the more at her piteous
and discomfited mien, persisted in thinking the joke a most diverting
one, and when Miss Sharp came downstairs, bantered her with a great
deal of liveliness upon the effect of her charms on the fat civilian.

``O Miss Sharp! if you could but see him this morning,'' he
said---``moaning in his flowered dressing-gown---writhing on his sofa; if
you could but have seen him lolling out his tongue to Gollop the
apothecary.''

``See whom?'' said Miss Sharp.

``Whom? O whom?  Captain Dobbin, of course, to whom we were all so
attentive, by the way, last night.''

``We were very unkind to him,'' Emmy said, blushing very much.  ``I---I
quite forgot him.''

``Of course you did,'' cried Osborne, still on the laugh.

``One can't be ALWAYS thinking about Dobbin, you know, Amelia.  Can one,
Miss Sharp?''

``Except when he overset the glass of wine at dinner,'' Miss Sharp said,
with a haughty air and a toss of the head, ``I never gave the existence
of Captain Dobbin one single moment's consideration.''

``Very good, Miss Sharp, I'll tell him,'' Osborne said; and as he spoke
Miss Sharp began to have a feeling of distrust and hatred towards this
young officer, which he was quite unconscious of having inspired.  ``He
is to make fun of me, is he?'' thought Rebecca.  ``Has he been laughing
about me to Joseph?  Has he frightened him? Perhaps he won't come.''---A
film passed over her eyes, and her heart beat quite quick.

``You're always joking,'' said she, smiling as innocently as she could.
``Joke away, Mr.\ George; there's nobody to defend ME.'' And George
Osborne, as she walked away---and Amelia looked reprovingly at him---felt
some little manly compunction for having inflicted any unnecessary
unkindness upon this helpless creature.  ``My dearest Amelia,'' said he,
``you are too good---too kind.  You don't know the world.  I do.  And
your little friend Miss Sharp must learn her station.''

``Don't you think Jos will---''

``Upon my word, my dear, I don't know.  He may, or may not.  I'm not his
master.  I only know he is a very foolish vain fellow, and put my dear
little girl into a very painful and awkward position last night.  My
dearest diddle-diddle-darling!'' He was off laughing again, and he did
it so drolly that Emmy laughed too.

All that day Jos never came.  But Amelia had no fear about this; for
the little schemer had actually sent away the page, Mr.\ Sambo's
aide-de-camp, to Mr.\ Joseph's lodgings, to ask for some book he had
promised, and how he was; and the reply through Jos's man, Mr.\ Brush,
was, that his master was ill in bed, and had just had the doctor with
him.  He must come to-morrow, she thought, but she never had the
courage to speak a word on the subject to Rebecca; nor did that young
woman herself allude to it in any way during the whole evening after
the night at Vauxhall.

The next day, however, as the two young ladies sate on the sofa,
pretending to work, or to write letters, or to read novels, Sambo came
into the room with his usual engaging grin, with a packet under his
arm, and a note on a tray.  ``Note from Mr.\ Jos, Miss,'' says Sambo.

How Amelia trembled as she opened it!

So it ran:

Dear Amelia,---I send you the ``Orphan of the Forest.'' I was too ill to
come yesterday.  I leave town to-day for Cheltenham.  Pray excuse me,
if you can, to the amiable Miss Sharp, for my conduct at Vauxhall, and
entreat her to pardon and forget every word I may have uttered when
excited by that fatal supper.  As soon as I have recovered, for my
health is very much shaken, I shall go to Scotland for some months, and
am

Truly yours, Jos Sedley

It was the death-warrant.  All was over.  Amelia did not dare to look
at Rebecca's pale face and burning eyes, but she dropt the letter into
her friend's lap; and got up, and went upstairs to her room, and cried
her little heart out.

Blenkinsop, the housekeeper, there sought her presently with
consolation, on whose shoulder Amelia wept confidentially, and relieved
herself a good deal.  ``Don't take on, Miss.  I didn't like to tell you.
But none of us in the house have liked her except at fust.  I sor her
with my own eyes reading your Ma's letters.  Pinner says she's always
about your trinket-box and drawers, and everybody's drawers, and she's
sure she's put your white ribbing into her box.''

``I gave it her, I gave it her,'' Amelia said.

But this did not alter Mrs.\ Blenkinsop's opinion of Miss Sharp.  ``I
don't trust them governesses, Pinner,'' she remarked to the maid. ``They
give themselves the hairs and hupstarts of ladies, and their wages is
no better than you nor me.''

It now became clear to every soul in the house, except poor Amelia,
that Rebecca should take her departure, and high and low (always with
the one exception) agreed that that event should take place as speedily
as possible. Our good child ransacked all her drawers, cupboards,
reticules, and gimcrack boxes---passed in review all her gowns, fichus,
tags, bobbins, laces, silk stockings, and fallals---selecting this thing
and that and the other, to make a little heap for Rebecca.  And going
to her Papa, that generous British merchant, who had promised to give
her as many guineas as she was years old---she begged the old gentleman
to give the money to dear Rebecca, who must want it, while she lacked
for nothing.

She even made George Osborne contribute, and nothing loth (for he was
as free-handed a young fellow as any in the army), he went to Bond
Street, and bought the best hat and spenser that money could buy.

``That's George's present to you, Rebecca, dear,'' said Amelia, quite
proud of the bandbox conveying these gifts.  ``What a taste he has!
There's nobody like him.''

``Nobody,'' Rebecca answered.  ``How thankful I am to him!'' She was
thinking in her heart, ``It was George Osborne who prevented my
marriage.''---And she loved George Osborne accordingly.

She made her preparations for departure with great equanimity; and
accepted all the kind little Amelia's presents, after just the proper
degree of hesitation and reluctance.  She vowed eternal gratitude to
Mrs.\ Sedley, of course; but did not intrude herself upon that good lady
too much, who was embarrassed, and evidently wishing to avoid her.  She
kissed Mr.\ Sedley's hand, when he presented her with the purse; and
asked permission to consider him for the future as her kind, kind
friend and protector.  Her behaviour was so affecting that he was going
to write her a cheque for twenty pounds more; but he restrained his
feelings: the carriage was in waiting to take him to dinner, so he
tripped away with a ``God bless you, my dear, always come here when you
come to town, you know.---Drive to the Mansion House, James.''

Finally came the parting with Miss Amelia, over which picture I intend
to throw a veil.  But after a scene in which one person was in earnest
and the other a perfect performer---after the tenderest caresses, the
most pathetic tears, the smelling-bottle, and some of the very best
feelings of the heart, had been called into requisition---Rebecca and
Amelia parted, the former vowing to love her friend for ever and ever
and ever.



\chapter{Crawley of Queen's Crawley}

Among the most respected of the names beginning in C which the
Court-Guide contained, in the year 18---, was that of Crawley, Sir Pitt,
Baronet, Great Gaunt Street, and Queen's Crawley, Hants.  This
honourable name had figured constantly also in the Parliamentary list
for many years, in conjunction with that of a number of other worthy
gentlemen who sat in turns for the borough.

It is related, with regard to the borough of Queen's Crawley, that
Queen Elizabeth in one of her progresses, stopping at Crawley to
breakfast, was so delighted with some remarkably fine Hampshire beer
which was then presented to her by the Crawley of the day (a handsome
gentleman with a trim beard and a good leg), that she forthwith erected
Crawley into a borough to send two members to Parliament; and the
place, from the day of that illustrious visit, took the name of Queen's
Crawley, which it holds up to the present moment.  And though, by the
lapse of time, and those mutations which age produces in empires,
cities, and boroughs, Queen's Crawley was no longer so populous a place
as it had been in Queen Bess's time---nay, was come down to that
condition of borough which used to be denominated rotten---yet, as Sir
Pitt Crawley would say with perfect justice in his elegant way,
``Rotten! be hanged---it produces me a good fifteen hundred a year.''

Sir Pitt Crawley (named after the great Commoner) was the son of
Walpole Crawley, first Baronet, of the Tape and Sealing-Wax Office in
the reign of George II., when he was impeached for peculation, as were
a great number of other honest gentlemen of those days; and Walpole
Crawley was, as need scarcely be said, son of John Churchill Crawley,
named after the celebrated military commander of the reign of Queen
Anne.  The family tree (which hangs up at Queen's Crawley) furthermore
mentions Charles Stuart, afterwards called Barebones Crawley, son of
the Crawley of James the First's time; and finally, Queen Elizabeth's
Crawley, who is represented as the foreground of the picture in his
forked beard and armour.  Out of his waistcoat, as usual, grows a tree,
on the main branches of which the above illustrious names are
inscribed.  Close by the name of Sir Pitt Crawley, Baronet (the subject
of the present memoir), are written that of his brother, the Reverend
Bute Crawley (the great Commoner was in disgrace when the reverend
gentleman was born), rector of Crawley-cum-Snailby, and of various
other male and female members of the Crawley family.

Sir Pitt was first married to Grizzel, sixth daughter of Mungo Binkie,
Lord Binkie, and cousin, in consequence, of Mr.\ Dundas.  She brought
him two sons: Pitt, named not so much after his father as after the
heaven-born minister; and Rawdon Crawley, from the Prince of Wales's
friend, whom his Majesty George IV forgot so completely. Many years
after her ladyship's demise, Sir Pitt led to the altar Rosa, daughter
of Mr.\ G. Dawson, of Mudbury, by whom he had two daughters, for whose
benefit Miss Rebecca Sharp was now engaged as governess.  It will be
seen that the young lady was come into a family of very genteel
connexions, and was about to move in a much more distinguished circle
than that humble one which she had just quitted in Russell Square.

She had received her orders to join her pupils, in a note which was
written upon an old envelope, and which contained the following words:

Sir Pitt Crawley begs Miss Sharp and baggidge may be hear on Tuesday,
as I leaf for Queen's Crawley to-morrow morning ERLY.

Great Gaunt Street.

Rebecca had never seen a Baronet, as far as she knew, and as soon as
she had taken leave of Amelia, and counted the guineas which
good-natured Mr.\ Sedley had put into a purse for her, and as soon as
she had done wiping her eyes with her handkerchief (which operation she
concluded the very moment the carriage had turned the corner of the
street), she began to depict in her own mind what a Baronet must be. ``I
wonder, does he wear a star?'' thought she, ``or is it only lords that
wear stars? But he will be very handsomely dressed in a court suit,
with ruffles, and his hair a little powdered, like Mr.\ Wroughton at
Covent Garden.  I suppose he will be awfully proud, and that I shall be
treated most contemptuously.  Still I must bear my hard lot as well as
I can---at least, I shall be amongst GENTLEFOLKS, and not with vulgar
city people'': and she fell to thinking of her Russell Square friends
with that very same philosophical bitterness with which, in a certain
apologue, the fox is represented as speaking of the grapes.

Having passed through Gaunt Square into Great Gaunt Street, the
carriage at length stopped at a tall gloomy house between two other
tall gloomy houses, each with a hatchment over the middle drawing-room
window; as is the custom of houses in Great Gaunt Street, in which
gloomy locality death seems to reign perpetual.  The shutters of the
first-floor windows of Sir Pitt's mansion were closed---those of the
dining-room were partially open, and the blinds neatly covered up in
old newspapers.

John, the groom, who had driven the carriage alone, did not care to
descend to ring the bell; and so prayed a passing milk-boy to perform
that office for him.  When the bell was rung, a head appeared between
the interstices of the dining-room shutters, and the door was opened by
a man in drab breeches and gaiters, with a dirty old coat, a foul old
neckcloth lashed round his bristly neck, a shining bald head, a leering
red face, a pair of twinkling grey eyes, and a mouth perpetually on the
grin.

``This Sir Pitt Crawley's?'' says John, from the box.

``Ees,'' says the man at the door, with a nod.

``Hand down these 'ere trunks then,'' said John.

``Hand 'n down yourself,'' said the porter.

``Don't you see I can't leave my hosses? Come, bear a hand, my fine
feller, and Miss will give you some beer,'' said John, with a
horse-laugh, for he was no longer respectful to Miss Sharp, as her
connexion with the family was broken off, and as she had given nothing
to the servants on coming away.

The bald-headed man, taking his hands out of his breeches pockets,
advanced on this summons, and throwing Miss Sharp's trunk over his
shoulder, carried it into the house.

``Take this basket and shawl, if you please, and open the door,'' said
Miss Sharp, and descended from the carriage in much indignation.  ``I
shall write to Mr.\ Sedley and inform him of your conduct,'' said she to
the groom.

``Don't,'' replied that functionary.  ``I hope you've forgot nothink? Miss
'Melia's gownds---have you got them---as the lady's maid was to have 'ad?
I hope they'll fit you. Shut the door, Jim, you'll get no good out of
'ER,'' continued John, pointing with his thumb towards Miss Sharp: ``a
bad lot, I tell you, a bad lot,'' and so saying, Mr.\ Sedley's groom
drove away.  The truth is, he was attached to the lady's maid in
question, and indignant that she should have been robbed of her
perquisites.

On entering the dining-room, by the orders of the individual in
gaiters, Rebecca found that apartment not more cheerful than such rooms
usually are, when genteel families are out of town.  The faithful
chambers seem, as it were, to mourn the absence of their masters.  The
turkey carpet has rolled itself up, and retired sulkily under the
sideboard: the pictures have hidden their faces behind old sheets of
brown paper: the ceiling lamp is muffled up in a dismal sack of brown
holland: the window-curtains have disappeared under all sorts of shabby
envelopes: the marble bust of Sir Walpole Crawley is looking from its
black corner at the bare boards and the oiled fire-irons, and the empty
card-racks over the mantelpiece: the cellaret has lurked away behind
the carpet: the chairs are turned up heads and tails along the walls:
and in the dark corner opposite the statue, is an old-fashioned crabbed
knife-box, locked and sitting on a dumb waiter.

Two kitchen chairs, and a round table, and an attenuated old poker and
tongs were, however, gathered round the fire-place, as was a saucepan
over a feeble sputtering fire.  There was a bit of cheese and bread,
and a tin candlestick on the table, and a little black porter in a
pint-pot.

``Had your dinner, I suppose? It is not too warm for you? Like a drop of
beer?''

``Where is Sir Pitt Crawley?'' said Miss Sharp majestically.

``He, he! I'm Sir Pitt Crawley.  Reklect you owe me a pint for bringing
down your luggage.  He, he! Ask Tinker if I aynt.  Mrs.\ Tinker, Miss
Sharp; Miss Governess, Mrs.\ Charwoman.  Ho, ho!''

The lady addressed as Mrs.\ Tinker at this moment made her appearance
with a pipe and a paper of tobacco, for which she had been despatched a
minute before Miss Sharp's arrival; and she handed the articles over to
Sir Pitt, who had taken his seat by the fire.

``Where's the farden?'' said he.  ``I gave you three halfpence. Where's
the change, old Tinker?''

``There!'' replied Mrs.\ Tinker, flinging down the coin; ``it's only
baronets as cares about farthings.''

``A farthing a day is seven shillings a year,'' answered the M.P.; ``seven
shillings a year is the interest of seven guineas.  Take care of your
farthings, old Tinker, and your guineas will come quite nat'ral.''

``You may be sure it's Sir Pitt Crawley, young woman,'' said Mrs.\ Tinker,
surlily; ``because he looks to his farthings.  You'll know him better
afore long.''

``And like me none the worse, Miss Sharp,'' said the old gentleman, with
an air almost of politeness.  ``I must be just before I'm generous.''

``He never gave away a farthing in his life,'' growled Tinker.

``Never, and never will: it's against my principle.  Go and get another
chair from the kitchen, Tinker, if you want to sit down; and then we'll
have a bit of supper.''

Presently the baronet plunged a fork into the saucepan on the fire, and
withdrew from the pot a piece of tripe and an onion, which he divided
into pretty equal portions, and of which he partook with Mrs.\ Tinker.
``You see, Miss Sharp, when I'm not here Tinker's on board wages: when
I'm in town she dines with the family. Haw! haw! I'm glad Miss Sharp's
not hungry, ain't you, Tink?'' And they fell to upon their frugal supper.

After supper Sir Pitt Crawley began to smoke his pipe; and when it
became quite dark, he lighted the rushlight in the tin candlestick, and
producing from an interminable pocket a huge mass of papers, began
reading them, and putting them in order.

``I'm here on law business, my dear, and that's how it happens that I
shall have the pleasure of such a pretty travelling companion
to-morrow.''

``He's always at law business,'' said Mrs.\ Tinker, taking up the pot of
porter.

``Drink and drink about,'' said the Baronet.  ``Yes; my dear, Tinker is
quite right: I've lost and won more lawsuits than any man in England.
Look here at Crawley, Bart. v. Snaffle.  I'll throw him over, or my
name's not Pitt Crawley.  Podder and another versus Crawley, Bart.
Overseers of Snaily parish against Crawley, Bart. They can't prove it's
common: I'll defy 'em; the land's mine. It no more belongs to the
parish than it does to you or Tinker here.  I'll beat 'em, if it cost
me a thousand guineas. Look over the papers; you may if you like, my
dear. Do you write a good hand? I'll make you useful when we're at
Queen's Crawley, depend on it, Miss Sharp. Now the dowager's dead I
want some one.''

``She was as bad as he,'' said Tinker.  ``She took the law of every one of
her tradesmen; and turned away forty-eight footmen in four year.''

``She was close---very close,'' said the Baronet, simply; ``but she was a
valyble woman to me, and saved me a steward.''---And in this confidential
strain, and much to the amusement of the new-comer, the conversation
continued for a considerable time.  Whatever Sir Pitt Crawley's
qualities might be, good or bad, he did not make the least disguise of
them.  He talked of himself incessantly, sometimes in the coarsest and
vulgarest Hampshire accent; sometimes adopting the tone of a man of the
world.  And so, with injunctions to Miss Sharp to be ready at five in
the morning, he bade her good night. ``You'll sleep with Tinker
to-night,'' he said; ``it's a big bed, and there's room for two. Lady
Crawley died in it.  Good night.''

Sir Pitt went off after this benediction, and the solemn Tinker,
rushlight in hand, led the way up the great bleak stone stairs, past
the great dreary drawing-room doors, with the handles muffled up in
paper, into the great front bedroom, where Lady Crawley had slept her
last.  The bed and chamber were so funereal and gloomy, you might have
fancied, not only that Lady Crawley died in the room, but that her
ghost inhabited it.  Rebecca sprang about the apartment, however, with
the greatest liveliness, and had peeped into the huge wardrobes, and
the closets, and the cupboards, and tried the drawers which were
locked, and examined the dreary pictures and toilette appointments,
while the old charwoman was saying her prayers.  ``I shouldn't like to
sleep in this yeer bed without a good conscience, Miss,'' said the old
woman.  ``There's room for us and a half-dozen of ghosts in it,'' says
Rebecca.  ``Tell me all about Lady Crawley and Sir Pitt Crawley, and
everybody, my DEAR Mrs.\ Tinker.''

But old Tinker was not to be pumped by this little cross-questioner;
and signifying to her that bed was a place for sleeping, not
conversation, set up in her corner of the bed such a snore as only the
nose of innocence can produce.  Rebecca lay awake for a long, long
time, thinking of the morrow, and of the new world into which she was
going, and of her chances of success there.  The rushlight flickered in
the basin.  The mantelpiece cast up a great black shadow, over half of
a mouldy old sampler, which her defunct ladyship had worked, no doubt,
and over two little family pictures of young lads, one in a college
gown, and the other in a red jacket like a soldier. When she went to
sleep, Rebecca chose that one to dream about.

At four o'clock, on such a roseate summer's morning as even made Great
Gaunt Street look cheerful, the faithful Tinker, having wakened her
bedfellow, and bid her prepare for departure, unbarred and unbolted the
great hall door (the clanging and clapping whereof startled the
sleeping echoes in the street), and taking her way into Oxford Street,
summoned a coach from a stand there.  It is needless to particularize
the number of the vehicle, or to state that the driver was stationed
thus early in the neighbourhood of Swallow Street, in hopes that some
young buck, reeling homeward from the tavern, might need the aid of his
vehicle, and pay him with the generosity of intoxication.

It is likewise needless to say that the driver, if he had any such
hopes as those above stated, was grossly disappointed; and that the
worthy Baronet whom he drove to the City did not give him one single
penny more than his fare.  It was in vain that Jehu appealed and
stormed; that he flung down Miss Sharp's bandboxes in the gutter at the
'Necks, and swore he would take the law of his fare.

``You'd better not,'' said one of the ostlers; ``it's Sir Pitt Crawley.''

``So it is, Joe,'' cried the Baronet, approvingly; ``and I'd like to see
the man can do me.''

``So should oi,'' said Joe, grinning sulkily, and mounting the Baronet's
baggage on the roof of the coach.

``Keep the box for me, Leader,'' exclaims the Member of Parliament to the
coachman; who replied, ``Yes, Sir Pitt,'' with a touch of his hat, and
rage in his soul (for he had promised the box to a young gentleman from
Cambridge, who would have given a crown to a certainty), and Miss Sharp
was accommodated with a back seat inside the carriage, which might be
said to be carrying her into the wide world.

How the young man from Cambridge sulkily put his five great-coats in
front; but was reconciled when little Miss Sharp was made to quit the
carriage, and mount up beside him---when he covered her up in one of his
Benjamins, and became perfectly good-humoured---how the asthmatic
gentleman, the prim lady, who declared upon her sacred honour she had
never travelled in a public carriage before (there is always such a
lady in a coach---Alas! was; for the coaches, where are they?), and the
fat widow with the brandy-bottle, took their places inside---how the
porter asked them all for money, and got sixpence from the gentleman
and five greasy halfpence from the fat widow---and how the carriage at
length drove away---now threading the dark lanes of Aldersgate, anon
clattering by the Blue Cupola of St.\ Paul's, jingling rapidly by the
strangers' entry of Fleet-Market, which, with Exeter 'Change, has now
departed to the world of shadows---how they passed the White Bear in
Piccadilly, and saw the dew rising up from the market-gardens of
Knightsbridge---how Turnhamgreen, Brentwood, Bagshot, were passed---need
not be told here. But the writer of these pages, who has pursued in
former days, and in the same bright weather, the same remarkable
journey, cannot but think of it with a sweet and tender regret.  Where
is the road now, and its merry incidents of life? Is there no Chelsea
or Greenwich for the old honest pimple-nosed coachmen?  I wonder where
are they, those good fellows? Is old Weller alive or dead? and the
waiters, yea, and the inns at which they waited, and the cold rounds of
beef inside, and the stunted ostler, with his blue nose and clinking
pail, where is he, and where is his generation?  To those great
geniuses now in petticoats, who shall write novels for the beloved
reader's children, these men and things will be as much legend and
history as Nineveh, or Coeur de Lion, or Jack Sheppard.  For them
stage-coaches will have become romances---a team of four bays as
fabulous as Bucephalus or Black Bess.  Ah, how their coats shone, as
the stable-men pulled their clothes off, and away they went---ah, how
their tails shook, as with smoking sides at the stage's end they
demurely walked away into the inn-yard.  Alas!  we shall never hear the
horn sing at midnight, or see the pike-gates fly open any more.
Whither, however, is the light four-inside Trafalgar coach carrying us?
Let us be set down at Queen's Crawley without further divagation, and
see how Miss Rebecca Sharp speeds there.



\chapter{Private and Confidential}

Miss Rebecca Sharp to Miss Amelia Sedley, Russell Square, London.
(Free.---Pitt Crawley.)

MY DEAREST, SWEETEST AMELIA,

With what mingled joy and sorrow do I take up the pen to write to my
dearest friend!  Oh, what a change between to-day and yesterday! Now I
am friendless and alone; yesterday I was at home, in the sweet company
of a sister, whom I shall ever, ever cherish!

I will not tell you in what tears and sadness I passed the fatal night
in which I separated from you.  YOU went on Tuesday to joy and
happiness, with your mother and YOUR DEVOTED YOUNG SOLDIER by your
side; and I thought of you all night, dancing at the Perkins's, the
prettiest, I am sure, of all the young ladies at the Ball.  I was
brought by the groom in the old carriage to Sir Pitt Crawley's town
house, where, after John the groom had behaved most rudely and
insolently to me (alas! 'twas safe to insult poverty and misfortune!),
I was given over to Sir P.'s care, and made to pass the night in an old
gloomy bed, and by the side of a horrid gloomy old charwoman, who keeps
the house.  I did not sleep one single wink the whole night.

Sir Pitt is not what we silly girls, when we used to read Cecilia at
Chiswick, imagined a baronet must have been.  Anything, indeed, less
like Lord Orville cannot be imagined.  Fancy an old, stumpy, short,
vulgar, and very dirty man, in old clothes and shabby old gaiters, who
smokes a horrid pipe, and cooks his own horrid supper in a saucepan.
He speaks with a country accent, and swore a great deal at the old
charwoman, at the hackney coachman who drove us to the inn where the
coach went from, and on which I made the journey OUTSIDE FOR THE
GREATER PART OF THE WAY.

I was awakened at daybreak by the charwoman, and having arrived at the
inn, was at first placed inside the coach.  But, when we got to a place
called Leakington, where the rain began to fall very heavily---will you
believe it?---I was forced to come outside; for Sir Pitt is a proprietor
of the coach, and as a passenger came at Mudbury, who wanted an inside
place, I was obliged to go outside in the rain, where, however, a young
gentleman from Cambridge College sheltered me very kindly in one of his
several great coats.

This gentleman and the guard seemed to know Sir Pitt very well, and
laughed at him a great deal.  They both agreed in calling him an old
screw; which means a very stingy, avaricious person.  He never gives
any money to anybody, they said (and this meanness I hate); and the
young gentleman made me remark that we drove very slow for the last two
stages on the road, because Sir Pitt was on the box, and because he is
proprietor of the horses for this part of the journey.  ``But won't I
flog 'em on to Squashmore, when I take the ribbons?'' said the young
Cantab.  ``And sarve 'em right, Master Jack,'' said the guard.  When I
comprehended the meaning of this phrase, and that Master Jack intended
to drive the rest of the way, and revenge himself on Sir Pitt's horses,
of course I laughed too.

A carriage and four splendid horses, covered with armorial bearings,
however, awaited us at Mudbury, four miles from Queen's Crawley, and we
made our entrance to the baronet's park in state.  There is a fine
avenue of a mile long leading to the house, and the woman at the
lodge-gate (over the pillars of which are a serpent and a dove, the
supporters of the Crawley arms), made us a number of curtsies as she
flung open the old iron carved doors, which are something like those at
odious Chiswick.

``There's an avenue,'' said Sir Pitt, ``a mile long. There's six thousand
pound of timber in them there trees.  Do you call that nothing?'' He
pronounced avenue---EVENUE, and nothing---NOTHINK, so droll; and he had a
Mr.\ Hodson, his hind from Mudbury, into the carriage with him, and they
talked about distraining, and selling up, and draining and subsoiling,
and a great deal about tenants and farming---much more than I could
understand.  Sam Miles had been caught poaching, and Peter Bailey had
gone to the workhouse at last. ``Serve him right,'' said Sir Pitt; ``him
and his family has been cheating me on that farm these hundred and
fifty years.'' Some old tenant, I suppose, who could not pay his rent.
Sir Pitt might have said ``he and his family,'' to be sure; but rich
baronets do not need to be careful about grammar, as poor governesses
must be.

As we passed, I remarked a beautiful church-spire rising above some old
elms in the park; and before them, in the midst of a lawn, and some
outhouses, an old red house with tall chimneys covered with ivy, and
the windows shining in the sun.  ``Is that your church, sir?'' I said.

``Yes, hang it,'' (said Sir Pitt, only he used, dear, A MUCH WICKEDER
WORD); ``how's Buty, Hodson? Buty's my brother Bute, my dear---my brother
the parson.  Buty and the Beast I call him, ha, ha!''

Hodson laughed too, and then looking more grave and nodding his head,
said, ``I'm afraid he's better, Sir Pitt.  He was out on his pony
yesterday, looking at our corn.''

``Looking after his tithes, hang'un (only he used the same wicked word).
Will brandy and water never kill him? He's as tough as old
whatdyecallum---old Methusalem.''

Mr.\ Hodson laughed again.  ``The young men is home from college. They've
whopped John Scroggins till he's well nigh dead.''

``Whop my second keeper!'' roared out Sir Pitt.

``He was on the parson's ground, sir,'' replied Mr.\ Hodson; and Sir Pitt
in a fury swore that if he ever caught 'em poaching on his ground, he'd
transport 'em, by the lord he would.  However, he said, ``I've sold the
presentation of the living, Hodson; none of that breed shall get it, I
war'nt''; and Mr.\ Hodson said he was quite right: and I have no doubt
from this that the two brothers are at variance---as brothers often are,
and sisters too.  Don't you remember the two Miss Scratchleys at
Chiswick, how they used always to fight and quarrel---and Mary Box, how
she was always thumping Louisa?

Presently, seeing two little boys gathering sticks in the wood, Mr.\ %
Hodson jumped out of the carriage, at Sir Pitt's order, and rushed upon
them with his whip.  ``Pitch into 'em, Hodson,'' roared the baronet;
``flog their little souls out, and bring 'em up to the house, the
vagabonds; I'll commit 'em as sure as my name's Pitt.'' And presently we
heard Mr.\ Hodson's whip cracking on the shoulders of the poor little
blubbering wretches, and Sir Pitt, seeing that the malefactors were in
custody, drove on to the hall.

All the servants were ready to meet us, and . . .

Here, my dear, I was interrupted last night by a dreadful thumping at
my door: and who do you think it was? Sir Pitt Crawley in his night-cap
and dressing-gown, such a figure! As I shrank away from such a visitor,
he came forward and seized my candle.  ``No candles after eleven
o'clock, Miss Becky,'' said he.  ``Go to bed in the dark, you pretty
little hussy'' (that is what he called me), ``and unless you wish me to
come for the candle every night, mind and be in bed at eleven.'' And
with this, he and Mr.\ Horrocks the butler went off laughing.  You may
be sure I shall not encourage any more of their visits.  They let loose
two immense bloodhounds at night, which all last night were yelling and
howling at the moon.  ``I call the dog Gorer,'' said Sir Pitt; ``he's
killed a man that dog has, and is master of a bull, and the mother I
used to call Flora; but now I calls her Aroarer, for she's too old to
bite.  Haw, haw!''

Before the house of Queen's Crawley, which is an odious old-fashioned
red brick mansion, with tall chimneys and gables of the style of Queen
Bess, there is a terrace flanked by the family dove and serpent, and on
which the great hall-door opens.  And oh, my dear, the great hall I am
sure is as big and as glum as the great hall in the dear castle of
Udolpho.  It has a large fireplace, in which we might put half Miss
Pinkerton's school, and the grate is big enough to roast an ox at the
very least.  Round the room hang I don't know how many generations of
Crawleys, some with beards and ruffs, some with huge wigs and toes
turned out, some dressed in long straight stays and gowns that look as
stiff as towers, and some with long ringlets, and oh, my dear! scarcely
any stays at all.  At one end of the hall is the great staircase all in
black oak, as dismal as may be, and on either side are tall doors with
stags' heads over them, leading to the billiard-room and the library,
and the great yellow saloon and the morning-rooms.  I think there are
at least twenty bedrooms on the first floor; one of them has the bed in
which Queen Elizabeth slept; and I have been taken by my new pupils
through all these fine apartments this morning.  They are not rendered
less gloomy, I promise you, by having the shutters always shut; and
there is scarce one of the apartments, but when the light was let into
it, I expected to see a ghost in the room.  We have a schoolroom on the
second floor, with my bedroom leading into it on one side, and that of
the young ladies on the other.  Then there are Mr.\ Pitt's
apartments---Mr.\ Crawley, he is called---the eldest son, and Mr.\ Rawdon
Crawley's rooms---he is an officer like SOMEBODY, and away with his
regiment.  There is no want of room I assure you.  You might lodge all
the people in Russell Square in the house, I think, and have space to
spare.

Half an hour after our arrival, the great dinner-bell was rung, and I
came down with my two pupils (they are very thin insignificant little
chits of ten and eight years old).  I came down in your dear muslin
gown (about which that odious Mrs.\ Pinner was so rude, because you gave
it me); for I am to be treated as one of the family, except on company
days, when the young ladies and I are to dine upstairs.

Well, the great dinner-bell rang, and we all assembled in the little
drawing-room where my Lady Crawley sits.  She is the second Lady
Crawley, and mother of the young ladies.  She was an ironmonger's
daughter, and her marriage was thought a great match.  She looks as if
she had been handsome once, and her eyes are always weeping for the
loss of her beauty.  She is pale and meagre and high-shouldered, and
has not a word to say for herself, evidently.  Her stepson Mr.\ Crawley,
was likewise in the room.  He was in full dress, as pompous as an
undertaker.  He is pale, thin, ugly, silent; he has thin legs, no
chest, hay-coloured whiskers, and straw-coloured hair.  He is the very
picture of his sainted mother over the mantelpiece---Griselda of the
noble house of Binkie.

``This is the new governess, Mr.\ Crawley,'' said Lady Crawley, coming
forward and taking my hand.  ``Miss Sharp.''

``O!'' said Mr.\ Crawley, and pushed his head once forward and began again
to read a great pamphlet with which he was busy.

``I hope you will be kind to my girls,'' said Lady Crawley, with her pink
eyes always full of tears.

``Law, Ma, of course she will,'' said the eldest: and I saw at a glance
that I need not be afraid of THAT woman. ``My lady is served,'' says the
butler in black, in an immense white shirt-frill, that looked as if it
had been one of the Queen Elizabeth's ruffs depicted in the hall; and
so, taking Mr.\ Crawley's arm, she led the way to the dining-room,
whither I followed with my little pupils in each hand.

Sir Pitt was already in the room with a silver jug.  He had just been
to the cellar, and was in full dress too; that is, he had taken his
gaiters off, and showed his little dumpy legs in black worsted
stockings.  The sideboard was covered with glistening old plate---old
cups, both gold and silver; old salvers and cruet-stands, like Rundell
and Bridge's shop.  Everything on the table was in silver too, and two
footmen, with red hair and canary-coloured liveries, stood on either
side of the sideboard.

Mr.\ Crawley said a long grace, and Sir Pitt said amen, and the great
silver dish-covers were removed.

``What have we for dinner, Betsy?'' said the Baronet.

``Mutton broth, I believe, Sir Pitt,'' answered Lady Crawley.

``Mouton aux navets,'' added the butler gravely (pronounce, if you
please, moutongonavvy); ``and the soup is potage de mouton a
l'Ecossaise.  The side-dishes contain pommes de terre au naturel, and
choufleur a l'eau.''

``Mutton's mutton,'' said the Baronet, ``and a devilish good thing. What
SHIP was it, Horrocks, and when did you kill?''  ``One of the black-faced
Scotch, Sir Pitt: we killed on Thursday.''

``Who took any?''

``Steel, of Mudbury, took the saddle and two legs, Sir Pitt; but he says
the last was too young and confounded woolly, Sir Pitt.''

``Will you take some potage, Miss ah---Miss Blunt? said Mr.\ Crawley.

``Capital Scotch broth, my dear,'' said Sir Pitt, ``though they call it by
a French name.''

``I believe it is the custom, sir, in decent society,'' said Mr.\ Crawley,
haughtily, ``to call the dish as I have called it''; and it was served to
us on silver soup plates by the footmen in the canary coats, with the
mouton aux navets.  Then ``ale and water'' were brought, and served to us
young ladies in wine-glasses.  I am not a judge of ale, but I can say
with a clear conscience I prefer water.

While we were enjoying our repast, Sir Pitt took occasion to ask what
had become of the shoulders of the mutton.

``I believe they were eaten in the servants' hall,'' said my lady, humbly.

``They was, my lady,'' said Horrocks, ``and precious little else we get
there neither.''

Sir Pitt burst into a horse-laugh, and continued his conversation with
Mr.\ Horrocks.  ``That there little black pig of the Kent sow's breed
must be uncommon fat now.''

``It's not quite busting, Sir Pitt,'' said the butler with the gravest
air, at which Sir Pitt, and with him the young ladies, this time, began
to laugh violently.

``Miss Crawley, Miss Rose Crawley,'' said Mr.\ Crawley, ``your laughter
strikes me as being exceedingly out of place.''

``Never mind, my lord,'' said the Baronet, ``we'll try the porker on
Saturday.  Kill un on Saturday morning, John Horrocks.  Miss Sharp
adores pork, don't you, Miss Sharp?''

And I think this is all the conversation that I remember at dinner.
When the repast was concluded a jug of hot water was placed before Sir
Pitt, with a case-bottle containing, I believe, rum.  Mr.\ Horrocks
served myself and my pupils with three little glasses of wine, and a
bumper was poured out for my lady.  When we retired, she took from her
work-drawer an enormous interminable piece of knitting; the young
ladies began to play at cribbage with a dirty pack of cards.  We had
but one candle lighted, but it was in a magnificent old silver
candlestick, and after a very few questions from my lady, I had my
choice of amusement between a volume of sermons, and a pamphlet on the
corn-laws, which Mr.\ Crawley had been reading before dinner.

So we sat for an hour until steps were heard.

``Put away the cards, girls,'' cried my lady, in a great tremor; ``put
down Mr.\ Crawley's books, Miss Sharp''; and these orders had been
scarcely obeyed, when Mr.\ Crawley entered the room.

``We will resume yesterday's discourse, young ladies,'' said he, ``and you
shall each read a page by turns; so that Miss a---Miss Short may have an
opportunity of hearing you''; and the poor girls began to spell a long
dismal sermon delivered at Bethesda Chapel, Liverpool, on behalf of the
mission for the Chickasaw Indians. Was it not a charming evening?

At ten the servants were told to call Sir Pitt and the household to
prayers.  Sir Pitt came in first, very much flushed, and rather
unsteady in his gait; and after him the butler, the canaries, Mr.\ %
Crawley's man, three other men, smelling very much of the stable, and
four women, one of whom, I remarked, was very much overdressed, and who
flung me a look of great scorn as she plumped down on her knees.

After Mr.\ Crawley had done haranguing and expounding, we received our
candles, and then we went to bed; and then I was disturbed in my
writing, as I have described to my dearest sweetest Amelia.

Good night.  A thousand, thousand, thousand kisses!

Saturday.---This morning, at five, I heard the shrieking of the little
black pig.  Rose and Violet introduced me to it yesterday; and to the
stables, and to the kennel, and to the gardener, who was picking fruit
to send to market, and from whom they begged hard a bunch of hot-house
grapes; but he said that Sir Pitt had numbered every ``Man Jack'' of
them, and it would be as much as his place was worth to give any away.
The darling girls caught a colt in a paddock, and asked me if I would
ride, and began to ride themselves, when the groom, coming with horrid
oaths, drove them away.

Lady Crawley is always knitting the worsted.  Sir Pitt is always tipsy,
every night; and, I believe, sits with Horrocks, the butler. Mr.\ %
Crawley always reads sermons in the evening, and in the morning is
locked up in his study, or else rides to Mudbury, on county business,
or to Squashmore, where he preaches, on Wednesdays and Fridays, to the
tenants there.

A hundred thousand grateful loves to your dear papa and mamma.  Is your
poor brother recovered of his rack-punch? Oh, dear! Oh, dear! How men
should beware of wicked punch!

Ever and ever thine own REBECCA

Everything considered, I think it is quite as well for our dear Amelia
Sedley, in Russell Square, that Miss Sharp and she are parted.  Rebecca
is a droll funny creature, to be sure; and those descriptions of the
poor lady weeping for the loss of her beauty, and the gentleman ``with
hay-coloured whiskers and straw-coloured hair,'' are very smart,
doubtless, and show a great knowledge of the world.  That she might,
when on her knees, have been thinking of something better than Miss
Horrocks's ribbons, has possibly struck both of us.  But my kind reader
will please to remember that this history has ``Vanity Fair'' for a
title, and that Vanity Fair is a very vain, wicked, foolish place, full
of all sorts of humbugs and falsenesses and pretensions.  And while the
moralist, who is holding forth on the cover ( an accurate portrait of
your humble servant), professes to wear neither gown nor bands, but
only the very same long-eared livery in which his congregation is
arrayed: yet, look you, one is bound to speak the truth as far as one
knows it, whether one mounts a cap and bells or a shovel hat; and a
deal of disagreeable matter must come out in the course of such an
undertaking.

I have heard a brother of the story-telling trade, at Naples, preaching
to a pack of good-for-nothing honest lazy fellows by the sea-shore,
work himself up into such a rage and passion with some of the villains
whose wicked deeds he was describing and inventing, that the audience
could not resist it; and they and the poet together would burst out
into a roar of oaths and execrations against the fictitious monster of
the tale, so that the hat went round, and the bajocchi tumbled into it,
in the midst of a perfect storm of sympathy.

At the little Paris theatres, on the other hand, you will not only hear
the people yelling out ``Ah gredin! Ah monstre:'' and cursing the tyrant
of the play from the boxes; but the actors themselves positively refuse
to play the wicked parts, such as those of infames Anglais, brutal
Cossacks, and what not, and prefer to appear at a smaller salary, in
their real characters as loyal Frenchmen.  I set the two stories one
against the other, so that you may see that it is not from mere
mercenary motives that the present performer is desirous to show up and
trounce his villains; but because he has a sincere hatred of them,
which he cannot keep down, and which must find a vent in suitable abuse
and bad language.

I warn my ``kyind friends,'' then, that I am going to tell a story of
harrowing villainy and complicated---but, as I trust, intensely
interesting---crime.  My rascals are no milk-and-water rascals, I
promise you.  When we come to the proper places we won't spare fine
language---No, no! But when we are going over the quiet country we must
perforce be calm.  A tempest in a slop-basin is absurd.  We will
reserve that sort of thing for the mighty ocean and the lonely
midnight.  The present Chapter is very mild.  Others---But we will not
anticipate THOSE.

And, as we bring our characters forward, I will ask leave, as a man and
a brother, not only to introduce them, but occasionally to step down
from the platform, and talk about them: if they are good and kindly, to
love them and shake them by the hand: if they are silly, to laugh at
them confidentially in the reader's sleeve: if they are wicked and
heartless, to abuse them in the strongest terms which politeness admits
of.

Otherwise you might fancy it was I who was sneering at the practice of
devotion, which Miss Sharp finds so ridiculous; that it was I who
laughed good-humouredly at the reeling old Silenus of a
baronet---whereas the laughter comes from one who has no reverence
except for prosperity, and no eye for anything beyond success. Such
people there are living and flourishing in the world---Faithless,
Hopeless, Charityless: let us have at them, dear friends, with might
and main. Some there are, and very successful too, mere quacks and
fools: and it was to combat and expose such as those, no doubt, that
Laughter was made.



\chapter{Family Portraits}

Sir Pitt Crawley was a philosopher with a taste for what is called low
life.  His first marriage with the daughter of the noble Binkie had
been made under the auspices of his parents; and as he often told Lady
Crawley in her lifetime she was such a confounded quarrelsome high-bred
jade that when she died he was hanged if he would ever take another of
her sort, at her ladyship's demise he kept his promise, and selected
for a second wife Miss Rose Dawson, daughter of Mr.\ John Thomas Dawson,
ironmonger, of Mudbury. What a happy woman was Rose to be my Lady
Crawley!

Let us set down the items of her happiness.  In the first place, she
gave up Peter Butt, a young man who kept company with her, and in
consequence of his disappointment in love, took to smuggling, poaching,
and a thousand other bad courses.  Then she quarrelled, as in duty
bound, with all the friends and intimates of her youth, who, of course,
could not be received by my Lady at Queen's Crawley---nor did she find
in her new rank and abode any persons who were willing to welcome her.
Who ever did? Sir Huddleston Fuddleston had three daughters who all
hoped to be Lady Crawley.  Sir Giles Wapshot's family were insulted
that one of the Wapshot girls had not the preference in the marriage,
and the remaining baronets of the county were indignant at their
comrade's misalliance.  Never mind the commoners, whom we will leave to
grumble anonymously.

Sir Pitt did not care, as he said, a brass farden for any one of them.
He had his pretty Rose, and what more need a man require than to please
himself? So he used to get drunk every night: to beat his pretty Rose
sometimes: to leave her in Hampshire when he went to London for the
parliamentary session, without a single friend in the wide world.  Even
Mrs.\ Bute Crawley, the Rector's wife, refused to visit her, as she said
she would never give the pas to a tradesman's daughter.

As the only endowments with which Nature had gifted Lady Crawley were
those of pink cheeks and a white skin, and as she had no sort of
character, nor talents, nor opinions, nor occupations, nor amusements,
nor that vigour of soul and ferocity of temper which often falls to the
lot of entirely foolish women, her hold upon Sir Pitt's affections was
not very great.  Her roses faded out of her cheeks, and the pretty
freshness left her figure after the birth of a couple of children, and
she became a mere machine in her husband's house of no more use than
the late Lady Crawley's grand piano. Being a light-complexioned woman,
she wore light clothes, as most blondes will, and appeared, in
preference, in draggled sea-green, or slatternly sky-blue.  She worked
that worsted day and night, or other pieces like it.  She had
counterpanes in the course of a few years to all the beds in Crawley.
She had a small flower-garden, for which she had rather an affection;
but beyond this no other like or disliking.  When her husband was rude
to her she was apathetic: whenever he struck her she cried.  She had
not character enough to take to drinking, and moaned about, slipshod
and in curl-papers all day.  O Vanity Fair---Vanity Fair! This might
have been, but for you, a cheery lass---Peter Butt and Rose a happy man
and wife, in a snug farm, with a hearty family; and an honest portion
of pleasures, cares, hopes and struggles---but a title and a coach and
four are toys more precious than happiness in Vanity Fair: and if Harry
the Eighth or Bluebeard were alive now, and wanted a tenth wife, do you
suppose he could not get the prettiest girl that shall be presented
this season?

The languid dulness of their mamma did not, as it may be supposed,
awaken much affection in her little daughters, but they were very happy
in the servants' hall and in the stables; and the Scotch gardener
having luckily a good wife and some good children, they got a little
wholesome society and instruction in his lodge, which was the only
education bestowed upon them until Miss Sharp came.

Her engagement was owing to the remonstrances of Mr.\ Pitt Crawley, the
only friend or protector Lady Crawley ever had, and the only person,
besides her children, for whom she entertained a little feeble
attachment.  Mr.\ Pitt took after the noble Binkies, from whom he was
descended, and was a very polite and proper gentleman.  When he grew to
man's estate, and came back from Christchurch, he began to reform the
slackened discipline of the hall, in spite of his father, who stood in
awe of him.  He was a man of such rigid refinement, that he would have
starved rather than have dined without a white neckcloth.  Once, when
just from college, and when Horrocks the butler brought him a letter
without placing it previously on a tray, he gave that domestic a look,
and administered to him a speech so cutting, that Horrocks ever after
trembled before him; the whole household bowed to him: Lady Crawley's
curl-papers came off earlier when he was at home: Sir Pitt's muddy
gaiters disappeared; and if that incorrigible old man still adhered to
other old habits, he never fuddled himself with rum-and-water in his
son's presence, and only talked to his servants in a very reserved and
polite manner; and those persons remarked that Sir Pitt never swore at
Lady Crawley while his son was in the room.

It was he who taught the butler to say, ``My lady is served,'' and who
insisted on handing her ladyship in to dinner.  He seldom spoke to her,
but when he did it was with the most powerful respect; and he never let
her quit the apartment without rising in the most stately manner to
open the door, and making an elegant bow at her egress.

At Eton he was called Miss Crawley; and there, I am sorry to say, his
younger brother Rawdon used to lick him violently.  But though his
parts were not brilliant, he made up for his lack of talent by
meritorious industry, and was never known, during eight years at
school, to be subject to that punishment which it is generally thought
none but a cherub can escape.

At college his career was of course highly creditable. And here he
prepared himself for public life, into which he was to be introduced by
the patronage of his grandfather, Lord Binkie, by studying the ancient
and modern orators with great assiduity, and by speaking unceasingly at
the debating societies.  But though he had a fine flux of words, and
delivered his little voice with great pomposity and pleasure to
himself, and never advanced any sentiment or opinion which was not
perfectly trite and stale, and supported by a Latin quotation; yet he
failed somehow, in spite of a mediocrity which ought to have insured
any man a success.  He did not even get the prize poem, which all his
friends said he was sure of.

After leaving college he became Private Secretary to Lord Binkie, and
was then appointed Attache to the Legation at Pumpernickel, which post
he filled with perfect honour, and brought home despatches, consisting
of Strasburg pie, to the Foreign Minister of the day.  After remaining
ten years Attache (several years after the lamented Lord Binkie's
demise), and finding the advancement slow, he at length gave up the
diplomatic service in some disgust, and began to turn country gentleman.

He wrote a pamphlet on Malt on returning to England (for he was an
ambitious man, and always liked to be before the public), and took a
strong part in the Negro Emancipation question.  Then he became a
friend of Mr.\ Wilberforce's, whose politics he admired, and had that
famous correspondence with the Reverend Silas Hornblower, on the
Ashantee Mission.  He was in London, if not for the Parliament session,
at least in May, for the religious meetings.  In the country he was a
magistrate, and an active visitor and speaker among those destitute of
religious instruction.  He was said to be paying his addresses to Lady
Jane Sheepshanks, Lord Southdown's third daughter, and whose sister,
Lady Emily, wrote those sweet tracts, ``The Sailor's True Binnacle,'' and
``The Applewoman of Finchley Common.''

Miss Sharp's accounts of his employment at Queen's Crawley were not
caricatures.  He subjected the servants there to the devotional
exercises before mentioned, in which (and so much the better) he
brought his father to join.  He patronised an Independent meeting-house
in Crawley parish, much to the indignation of his uncle the Rector, and
to the consequent delight of Sir Pitt, who was induced to go himself
once or twice, which occasioned some violent sermons at Crawley parish
church, directed point-blank at the Baronet's old Gothic pew there.
Honest Sir Pitt, however, did not feel the force of these discourses,
as he always took his nap during sermon-time.

Mr.\ Crawley was very earnest, for the good of the nation and of the
Christian world, that the old gentleman should yield him up his place
in Parliament; but this the elder constantly refused to do. Both were
of course too prudent to give up the fifteen hundred a year which was
brought in by the second seat (at this period filled by Mr.\ Quadroon,
with carte blanche on the Slave question); indeed the family estate was
much embarrassed, and the income drawn from the borough was of great
use to the house of Queen's Crawley.

It had never recovered the heavy fine imposed upon Walpole Crawley,
first baronet, for peculation in the Tape and Sealing Wax Office. Sir
Walpole was a jolly fellow, eager to seize and to spend money (alieni
appetens, sui profusus, as Mr.\ Crawley would remark with a sigh), and
in his day beloved by all the county for the constant drunkenness and
hospitality which was maintained at Queen's Crawley. The cellars were
filled with burgundy then, the kennels with hounds, and the stables
with gallant hunters; now, such horses as Queen's Crawley possessed
went to plough, or ran in the Trafalgar Coach; and it was with a team
of these very horses, on an off-day, that Miss Sharp was brought to the
Hall; for boor as he was, Sir Pitt was a stickler for his dignity while
at home, and seldom drove out but with four horses, and though he dined
off boiled mutton, had always three footmen to serve it.

If mere parsimony could have made a man rich, Sir Pitt Crawley might
have become very wealthy---if he had been an attorney in a country town,
with no capital but his brains, it is very possible that he would have
turned them to good account, and might have achieved for himself a very
considerable influence and competency. But he was unluckily endowed
with a good name and a large though encumbered estate, both of which
went rather to injure than to advance him.  He had a taste for law,
which cost him many thousands yearly; and being a great deal too clever
to be robbed, as he said, by any single agent, allowed his affairs to
be mismanaged by a dozen, whom he all equally mistrusted. He was such a
sharp landlord, that he could hardly find any but bankrupt tenants; and
such a close farmer, as to grudge almost the seed to the ground,
whereupon revengeful Nature grudged him the crops which she granted to
more liberal husbandmen. He speculated in every possible way; he worked
mines; bought canal-shares; horsed coaches; took government contracts,
and was the busiest man and magistrate of his county.  As he would not
pay honest agents at his granite quarry, he had the satisfaction of
finding that four overseers ran away, and took fortunes with them to
America.  For want of proper precautions, his coal-mines filled with
water: the government flung his contract of damaged beef upon his
hands: and for his coach-horses, every mail proprietor in the kingdom
knew that he lost more horses than any man in the country, from
underfeeding and buying cheap. In disposition he was sociable, and far
from being proud; nay, he rather preferred the society of a farmer or a
horse-dealer to that of a gentleman, like my lord, his son: he was fond
of drink, of swearing, of joking with the farmers' daughters: he was
never known to give away a shilling or to do a good action, but was of
a pleasant, sly, laughing mood, and would cut his joke and drink his
glass with a tenant and sell him up the next day; or have his laugh
with the poacher he was transporting with equal good humour.  His
politeness for the fair sex has already been hinted at by Miss Rebecca
Sharp---in a word, the whole baronetage, peerage, commonage of England,
did not contain a more cunning, mean, selfish, foolish, disreputable
old man.  That blood-red hand of Sir Pitt Crawley's would be in
anybody's pocket except his own; and it is with grief and pain, that,
as admirers of the British aristocracy, we find ourselves obliged to
admit the existence of so many ill qualities in a person whose name is
in Debrett.

One great cause why Mr.\ Crawley had such a hold over the affections of
his father, resulted from money arrangements.  The Baronet owed his son
a sum of money out of the jointure of his mother, which he did not find
it convenient to pay; indeed he had an almost invincible repugnance to
paying anybody, and could only be brought by force to discharge his
debts.  Miss Sharp calculated (for she became, as we shall hear
speedily, inducted into most of the secrets of the family) that the
mere payment of his creditors cost the honourable Baronet several
hundreds yearly; but this was a delight he could not forego; he had a
savage pleasure in making the poor wretches wait, and in shifting from
court to court and from term to term the period of satisfaction.
What's the good of being in Parliament, he said, if you must pay your
debts? Hence, indeed, his position as a senator was not a little useful
to him.

Vanity Fair---Vanity Fair!  Here was a man, who could not spell, and did
not care to read---who had the habits and the cunning of a boor: whose
aim in life was pettifogging: who never had a taste, or emotion, or
enjoyment, but what was sordid and foul; and yet he had rank, and
honours, and power, somehow: and was a dignitary of the land, and a
pillar of the state.  He was high sheriff, and rode in a golden coach.
Great ministers and statesmen courted him; and in Vanity Fair he had a
higher place than the most brilliant genius or spotless virtue.

Sir Pitt had an unmarried half-sister who inherited her mother's large
fortune, and though the Baronet proposed to borrow this money of her on
mortgage, Miss Crawley declined the offer, and preferred the security
of the funds. She had signified, however, her intention of leaving her
inheritance between Sir Pitt's second son and the family at the
Rectory, and had once or twice paid the debts of Rawdon Crawley in his
career at college and in the army. Miss Crawley was, in consequence, an
object of great respect when she came to Queen's Crawley, for she had a
balance at her banker's which would have made her beloved anywhere.

What a dignity it gives an old lady, that balance at the banker's! How
tenderly we look at her faults if she is a relative (and may every
reader have a score of such), what a kind good-natured old creature we
find her!  How the junior partner of Hobbs and Dobbs leads her smiling
to the carriage with the lozenge upon it, and the fat wheezy coachman!
How, when she comes to pay us a visit, we generally find an opportunity
to let our friends know her station in the world!  We say (and with
perfect truth) I wish I had Miss MacWhirter's signature to a cheque for
five thousand pounds.  She wouldn't miss it, says your wife.  She is my
aunt, say you, in an easy careless way, when your friend asks if Miss
MacWhirter is any relative.  Your wife is perpetually sending her
little testimonies of affection, your little girls work endless worsted
baskets, cushions, and footstools for her.  What a good fire there is
in her room when she comes to pay you a visit, although your wife laces
her stays without one!  The house during her stay assumes a festive,
neat, warm, jovial, snug appearance not visible at other seasons. You
yourself, dear sir, forget to go to sleep after dinner, and find
yourself all of a sudden (though you invariably lose) very fond of a
rubber.  What good dinners you have---game every day, Malmsey-Madeira,
and no end of fish from London.  Even the servants in the kitchen share
in the general prosperity; and, somehow, during the stay of Miss
MacWhirter's fat coachman, the beer is grown much stronger, and the
consumption of tea and sugar in the nursery (where her maid takes her
meals) is not regarded in the least.  Is it so, or is it not so?  I
appeal to the middle classes.  Ah, gracious powers! I wish you would
send me an old aunt---a maiden aunt---an aunt with a lozenge on her
carriage, and a front of light coffee-coloured hair---how my children
should work workbags for her, and my Julia and I would make her
comfortable! Sweet---sweet vision! Foolish---foolish dream!



\chapter{Miss Sharp Begins to Make Friends}

And now, being received as a member of the amiable family whose
portraits we have sketched in the foregoing pages, it became naturally
Rebecca's duty to make herself, as she said, agreeable to her
benefactors, and to gain their confidence to the utmost of her power.
Who can but admire this quality of gratitude in an unprotected orphan;
and, if there entered some degree of selfishness into her calculations,
who can say but that her prudence was perfectly justifiable?  ``I am
alone in the world,'' said the friendless girl.  ``I have nothing to look
for but what my own labour can bring me; and while that little
pink-faced chit Amelia, with not half my sense, has ten thousand pounds
and an establishment secure, poor Rebecca (and my figure is far better
than hers) has only herself and her own wits to trust to.  Well, let us
see if my wits cannot provide me with an honourable maintenance, and if
some day or the other I cannot show Miss Amelia my real superiority
over her. Not that I dislike poor Amelia: who can dislike such a
harmless, good-natured creature?---only it will be a fine day when I can
take my place above her in the world, as why, indeed, should I not?''
Thus it was that our little romantic friend formed visions of the
future for herself---nor must we be scandalised that, in all her castles
in the air, a husband was the principal inhabitant.  Of what else have
young ladies to think, but husbands? Of what else do their dear mammas
think?  ``I must be my own mamma,'' said Rebecca; not without a tingling
consciousness of defeat, as she thought over her little misadventure
with Jos Sedley.

So she wisely determined to render her position with the Queen's
Crawley family comfortable and secure, and to this end resolved to make
friends of every one around her who could at all interfere with her
comfort.

As my Lady Crawley was not one of these personages, and a woman,
moreover, so indolent and void of character as not to be of the least
consequence in her own house, Rebecca soon found that it was not at all
necessary to cultivate her good will---indeed, impossible to gain it.
She used to talk to her pupils about their ``poor mamma''; and, though
she treated that lady with every demonstration of cool respect, it was
to the rest of the family that she wisely directed the chief part of
her attentions.

With the young people, whose applause she thoroughly gained, her method
was pretty simple.  She did not pester their young brains with too much
learning, but, on the contrary, let them have their own way in regard
to educating themselves; for what instruction is more effectual than
self-instruction? The eldest was rather fond of books, and as there was
in the old library at Queen's Crawley a considerable provision of works
of light literature of the last century, both in the French and English
languages (they had been purchased by the Secretary of the Tape and
Sealing Wax Office at the period of his disgrace), and as nobody ever
troubled the bookshelves but herself, Rebecca was enabled agreeably,
and, as it were, in playing, to impart a great deal of instruction to
Miss Rose Crawley.

She and Miss Rose thus read together many delightful French and English
works, among which may be mentioned those of the learned Dr. Smollett,
of the ingenious Mr.\ Henry Fielding, of the graceful and fantastic
Monsieur Crebillon the younger, whom our immortal poet Gray so much
admired, and of the universal Monsieur de Voltaire. Once, when Mr.\ %
Crawley asked what the young people were reading, the governess replied
``Smollett.'' ``Oh, Smollett,'' said Mr.\ Crawley, quite satisfied.  ``His
history is more dull, but by no means so dangerous as that of Mr.\ Hume.
It is history you are reading?'' ``Yes,'' said Miss Rose; without,
however, adding that it was the history of Mr.\ Humphrey Clinker.  On
another occasion he was rather scandalised at finding his sister with a
book of French plays; but as the governess remarked that it was for the
purpose of acquiring the French idiom in conversation, he was fain to
be content.  Mr.\ Crawley, as a diplomatist, was exceedingly proud of
his own skill in speaking the French language (for he was of the world
still), and not a little pleased with the compliments which the
governess continually paid him upon his proficiency.

Miss Violet's tastes were, on the contrary, more rude and boisterous
than those of her sister.  She knew the sequestered spots where the
hens laid their eggs.  She could climb a tree to rob the nests of the
feathered songsters of their speckled spoils.  And her pleasure was to
ride the young colts, and to scour the plains like Camilla. She was the
favourite of her father and of the stablemen. She was the darling, and
withal the terror of the cook; for she discovered the haunts of the
jam-pots, and would attack them when they were within her reach. She
and her sister were engaged in constant battles.  Any of which
peccadilloes, if Miss Sharp discovered, she did not tell them to Lady
Crawley; who would have told them to the father, or worse, to Mr.\ %
Crawley; but promised not to tell if Miss Violet would be a good girl
and love her governess.

With Mr.\ Crawley Miss Sharp was respectful and obedient.  She used to
consult him on passages of French which she could not understand,
though her mother was a Frenchwoman, and which he would construe to her
satisfaction: and, besides giving her his aid in profane literature, he
was kind enough to select for her books of a more serious tendency, and
address to her much of his conversation.  She admired, beyond measure,
his speech at the Quashimaboo-Aid Society; took an interest in his
pamphlet on malt: was often affected, even to tears, by his discourses
of an evening, and would say---``Oh, thank you, sir,'' with a sigh, and a
look up to heaven, that made him occasionally condescend to shake hands
with her.  ``Blood is everything, after all,'' would that aristocratic
religionist say. ``How Miss Sharp is awakened by my words, when not one
of the people here is touched.  I am too fine for them---too delicate. I
must familiarise my style---but she understands it.  Her mother was a
Montmorency.''

Indeed it was from this famous family, as it appears, that Miss Sharp,
by the mother's side, was descended. Of course she did not say that her
mother had been on the stage; it would have shocked Mr.\ Crawley's
religious scruples.  How many noble emigres had this horrid revolution
plunged in poverty!  She had several stories about her ancestors ere
she had been many months in the house; some of which Mr.\ Crawley
happened to find in D'Hozier's dictionary, which was in the library,
and which strengthened his belief in their truth, and in the
high-breeding of Rebecca.  Are we to suppose from this curiosity and
prying into dictionaries, could our heroine suppose that Mr.\ Crawley
was interested in her?---no, only in a friendly way.  Have we not stated
that he was attached to Lady Jane Sheepshanks?

He took Rebecca to task once or twice about the propriety of playing at
backgammon with Sir Pitt, saying that it was a godless amusement, and
that she would be much better engaged in reading ``Thrump's Legacy,'' or
``The Blind Washerwoman of Moorfields,'' or any work of a more serious
nature; but Miss Sharp said her dear mother used often to play the same
game with the old Count de Trictrac and the venerable Abbe du Cornet,
and so found an excuse for this and other worldly amusements.

But it was not only by playing at backgammon with the Baronet, that the
little governess rendered herself agreeable to her employer. She found
many different ways of being useful to him.  She read over, with
indefatigable patience, all those law papers, with which, before she
came to Queen's Crawley, he had promised to entertain her.  She
volunteered to copy many of his letters, and adroitly altered the
spelling of them so as to suit the usages of the present day.  She
became interested in everything appertaining to the estate, to the
farm, the park, the garden, and the stables; and so delightful a
companion was she, that the Baronet would seldom take his
after-breakfast walk without her (and the children of course), when she
would give her advice as to the trees which were to be lopped in the
shrubberies, the garden-beds to be dug, the crops which were to be cut,
the horses which were to go to cart or plough. Before she had been a
year at Queen's Crawley she had quite won the Baronet's confidence; and
the conversation at the dinner-table, which before used to be held
between him and Mr.\ Horrocks the butler, was now almost exclusively
between Sir Pitt and Miss Sharp. She was almost mistress of the house
when Mr.\ Crawley was absent, but conducted herself in her new and
exalted situation with such circumspection and modesty as not to offend
the authorities of the kitchen and stable, among whom her behaviour was
always exceedingly modest and affable.  She was quite a different
person from the haughty, shy, dissatisfied little girl whom we have
known previously, and this change of temper proved great prudence, a
sincere desire of amendment, or at any rate great moral courage on her
part.  Whether it was the heart which dictated this new system of
complaisance and humility adopted by our Rebecca, is to be proved by
her after-history.  A system of hypocrisy, which lasts through whole
years, is one seldom satisfactorily practised by a person of
one-and-twenty; however, our readers will recollect, that, though young
in years, our heroine was old in life and experience, and we have
written to no purpose if they have not discovered that she was a very
clever woman.

The elder and younger son of the house of Crawley were, like the
gentleman and lady in the weather-box, never at home together---they
hated each other cordially: indeed, Rawdon Crawley, the dragoon, had a
great contempt for the establishment altogether, and seldom came
thither except when his aunt paid her annual visit.

The great good quality of this old lady has been mentioned.  She
possessed seventy thousand pounds, and had almost adopted Rawdon. She
disliked her elder nephew exceedingly, and despised him as a milksop.
In return he did not hesitate to state that her soul was irretrievably
lost, and was of opinion that his brother's chance in the next world
was not a whit better.  ``She is a godless woman of the world,'' would
Mr.\ Crawley say; ``she lives with atheists and Frenchmen.  My mind
shudders when I think of her awful, awful situation, and that, near as
she is to the grave, she should be so given up to vanity,
licentiousness, profaneness, and folly.'' In fact, the old lady declined
altogether to hear his hour's lecture of an evening; and when she came
to Queen's Crawley alone, he was obliged to pretermit his usual
devotional exercises.

``Shut up your sarmons, Pitt, when Miss Crawley comes down,'' said his
father; ``she has written to say that she won't stand the preachifying.''

``O, sir! consider the servants.''

``The servants be hanged,'' said Sir Pitt; and his son thought even worse
would happen were they deprived of the benefit of his instruction.

``Why, hang it, Pitt!'' said the father to his remonstrance. ``You
wouldn't be such a flat as to let three thousand a year go out of the
family?''

``What is money compared to our souls, sir?'' continued Mr.\ Crawley.

``You mean that the old lady won't leave the money to you?''---and who
knows but it was Mr.\ Crawley's meaning?

Old Miss Crawley was certainly one of the reprobate. She had a snug
little house in Park Lane, and, as she ate and drank a great deal too
much during the season in London, she went to Harrowgate or Cheltenham
for the summer.  She was the most hospitable and jovial of old vestals,
and had been a beauty in her day, she said. (All old women were
beauties once, we very well know.) She was a bel esprit, and a dreadful
Radical for those days.  She had been in France (where St.\ Just, they
say, inspired her with an unfortunate passion), and loved, ever after,
French novels, French cookery, and French wines.  She read Voltaire,
and had Rousseau by heart; talked very lightly about divorce, and most
energetically of the rights of women.  She had pictures of Mr.\ Fox in
every room in the house: when that statesman was in opposition, I am
not sure that she had not flung a main with him; and when he came into
office, she took great credit for bringing over to him Sir Pitt and his
colleague for Queen's Crawley, although Sir Pitt would have come over
himself, without any trouble on the honest lady's part.  It is needless
to say that Sir Pitt was brought to change his views after the death of
the great Whig statesman.

This worthy old lady took a fancy to Rawdon Crawley when a boy, sent
him to Cambridge (in opposition to his brother at Oxford), and, when
the young man was requested by the authorities of the first-named
University to quit after a residence of two years, she bought him his
commission in the Life Guards Green.

A perfect and celebrated ``blood,'' or dandy about town, was this young
officer.  Boxing, rat-hunting, the fives court, and four-in-hand
driving were then the fashion of our British aristocracy; and he was an
adept in all these noble sciences.  And though he belonged to the
household troops, who, as it was their duty to rally round the Prince
Regent, had not shown their valour in foreign service yet, Rawdon
Crawley had already (apropos of play, of which he was immoderately
fond) fought three bloody duels, in which he gave ample proofs of his
contempt for death.

``And for what follows after death,'' would Mr.\ Crawley observe, throwing
his gooseberry-coloured eyes up to the ceiling.  He was always thinking
of his brother's soul, or of the souls of those who differed with him
in opinion: it is a sort of comfort which many of the serious give
themselves.

Silly, romantic Miss Crawley, far from being horrified at the courage
of her favourite, always used to pay his debts after his duels; and
would not listen to a word that was whispered against his morality.
``He will sow his wild oats,'' she would say, ``and is worth far more than
that puling hypocrite of a brother of his.''



\chapter{Arcadian Simplicity}

Besides these honest folks at the Hall (whose simplicity and sweet
rural purity surely show the advantage of a country life over a town
one), we must introduce the reader to their relatives and neighbours at
the Rectory, Bute Crawley and his wife.

The Reverend Bute Crawley was a tall, stately, jolly, shovel-hatted
man, far more popular in his county than the Baronet his brother. At
college he pulled stroke-oar in the Christchurch boat, and had thrashed
all the best bruisers of the ``town.'' He carried his taste for boxing
and athletic exercises into private life; there was not a fight within
twenty miles at which he was not present, nor a race, nor a coursing
match, nor a regatta, nor a ball, nor an election, nor a visitation
dinner, nor indeed a good dinner in the whole county, but he found
means to attend it.  You might see his bay mare and gig-lamps a score
of miles away from his Rectory House, whenever there was any
dinner-party at Fuddleston, or at Roxby, or at Wapshot Hall, or at the
great lords of the county, with all of whom he was intimate.  He had a
fine voice; sang ``A southerly wind and a cloudy sky''; and gave the
``whoop'' in chorus with general applause.  He rode to hounds in a
pepper-and-salt frock, and was one of the best fishermen in the county.

Mrs.\ Crawley, the rector's wife, was a smart little body, who wrote
this worthy divine's sermons.  Being of a domestic turn, and keeping
the house a great deal with her daughters, she ruled absolutely within
the Rectory, wisely giving her husband full liberty without. He was
welcome to come and go, and dine abroad as many days as his fancy
dictated, for Mrs.\ Crawley was a saving woman and knew the price of
port wine.  Ever since Mrs.\ Bute carried off the young Rector of
Queen's Crawley (she was of a good family, daughter of the late
Lieut.-Colonel Hector McTavish, and she and her mother played for Bute
and won him at Harrowgate), she had been a prudent and thrifty wife to
him.  In spite of her care, however, he was always in debt.  It took
him at least ten years to pay off his college bills contracted during
his father's lifetime. In the year 179-, when he was just clear of
these incumbrances, he gave the odds of 100 to 1 (in twenties) against
Kangaroo, who won the Derby.  The Rector was obliged to take up the
money at a ruinous interest, and had been struggling ever since.  His
sister helped him with a hundred now and then, but of course his great
hope was in her death---when ``hang it'' (as he would say), ``Matilda must
leave me half her money.''

So that the Baronet and his brother had every reason which two brothers
possibly can have for being by the ears.  Sir Pitt had had the better
of Bute in innumerable family transactions.  Young Pitt not only did
not hunt, but set up a meeting house under his uncle's very nose.
Rawdon, it was known, was to come in for the bulk of Miss Crawley's
property.  These money transactions---these speculations in life and
death---these silent battles for reversionary spoil---make brothers very
loving towards each other in Vanity Fair.  I, for my part, have known a
five-pound note to interpose and knock up a half century's attachment
between two brethren; and can't but admire, as I think what a fine and
durable thing Love is among worldly people.

It cannot be supposed that the arrival of such a personage as Rebecca
at Queen's Crawley, and her gradual establishment in the good graces of
all people there, could be unremarked by Mrs.\ Bute Crawley.  Mrs.\ Bute,
who knew how many days the sirloin of beef lasted at the Hall; how much
linen was got ready at the great wash; how many peaches were on the
south wall; how many doses her ladyship took when she was ill---for such
points are matters of intense interest to certain persons in the
country---Mrs.\ Bute, I say, could not pass over the Hall governess
without making every inquiry respecting her history and character.
There was always the best understanding between the servants at the
Rectory and the Hall. There was always a good glass of ale in the
kitchen of the former place for the Hall people, whose ordinary drink
was very small---and, indeed, the Rector's lady knew exactly how much
malt went to every barrel of Hall beer---ties of relationship existed
between the Hall and Rectory domestics, as between their masters; and
through these channels each family was perfectly well acquainted with
the doings of the other.  That, by the way, may be set down as a
general remark.  When you and your brother are friends, his doings are
indifferent to you.  When you have quarrelled, all his outgoings and
incomings you know, as if you were his spy.

Very soon then after her arrival, Rebecca began to take a regular place
in Mrs.\ Crawley's bulletin from the Hall. It was to this effect: ``The
black porker's killed---weighed x stone---salted the sides---pig's pudding
and leg of pork for dinner.  Mr.\ Cramp from Mudbury, over with Sir Pitt
about putting John Blackmore in gaol---Mr.\ Pitt at meeting (with all the
names of the people who attended)---my lady as usual---the young ladies
with the governess.''

Then the report would come---the new governess be a rare manager---Sir
Pitt be very sweet on her---Mr.\ Crawley too---He be reading tracts to
her---``What an abandoned wretch!'' said little, eager, active,
black-faced Mrs.\ Bute Crawley.

Finally, the reports were that the governess had ``come round''
everybody, wrote Sir Pitt's letters, did his business, managed his
accounts---had the upper hand of the whole house, my lady, Mr.\ Crawley,
the girls and all---at which Mrs.\ Crawley declared she was an artful
hussy, and had some dreadful designs in view.  Thus the doings at the
Hall were the great food for conversation at the Rectory, and Mrs.\ %
Bute's bright eyes spied out everything that took place in the enemy's
camp---everything and a great deal besides.


Mrs.\ Bute Crawley to Miss Pinkerton, The Mall, Chiswick.

Rectory, Queen's Crawley, December---.

My Dear Madam,---Although it is so many years since I profited by your
delightful and invaluable instructions, yet I have ever retained the
FONDEST and most reverential regard for Miss Pinkerton, and DEAR
Chiswick.  I hope your health is GOOD.  The world and the cause of
education cannot afford to lose Miss Pinkerton for MANY MANY YEARS.
When my friend, Lady Fuddleston, mentioned that her dear girls required
an instructress (I am too poor to engage a governess for mine, but was
I not educated at Chiswick?)---``Who,'' I exclaimed, ``can we consult but
the excellent, the incomparable Miss Pinkerton?'' In a word, have you,
dear madam, any ladies on your list, whose services might be made
available to my kind friend and neighbour? I assure you she will take
no governess BUT OF YOUR CHOOSING.

My dear husband is pleased to say that he likes EVERYTHING WHICH COMES
FROM MISS PINKERTON'S SCHOOL.  How I wish I could present him and my
beloved girls to the friend of my youth, and the ADMIRED of the great
lexicographer of our country! If you ever travel into Hampshire, Mr.\ %
Crawley begs me to say, he hopes you will adorn our RURAL RECTORY with
your presence.  'Tis the humble but happy home of

Your affectionate Martha Crawley

P.S.  Mr.\ Crawley's brother, the baronet, with whom we are not, alas!
upon those terms of UNITY in which it BECOMES BRETHREN TO DWELL, has a
governess for his little girls, who, I am told, had the good fortune to
be educated at Chiswick.  I hear various reports of her; and as I have
the tenderest interest in my dearest little nieces, whom I wish, in
spite of family differences, to see among my own children---and as I
long to be attentive to ANY PUPIL OF YOURS---do, my dear Miss Pinkerton,
tell me the history of this young lady, whom, for YOUR SAKE, I am most
anxious to befriend.---M. C.


Miss Pinkerton to Mrs.\ Bute Crawley.

Johnson House, Chiswick, Dec. 18---.

Dear Madam,---I have the honour to acknowledge your polite
communication, to which I promptly reply. 'Tis most gratifying to one
in my most arduous position to find that my maternal cares have
elicited a responsive affection; and to recognize in the amiable Mrs.\ %
Bute Crawley my excellent pupil of former years, the sprightly and
accomplished Miss Martha MacTavish.  I am happy to have under my charge
now the daughters of many of those who were your contemporaries at my
establishment---what pleasure it would give me if your own beloved young
ladies had need of my instructive superintendence!

Presenting my respectful compliments to Lady Fuddleston, I have the
honour (epistolarily) to introduce to her ladyship my two friends, Miss
Tuffin and Miss Hawky.

Either of these young ladies is PERFECTLY QUALIFIED to instruct in
Greek, Latin, and the rudiments of Hebrew; in mathematics and history;
in Spanish, French, Italian, and geography; in music, vocal and
instrumental; in dancing, without the aid of a master; and in the
elements of natural sciences.  In the use of the globes both are
proficients.  In addition to these Miss Tuffin, who is daughter of the
late Reverend Thomas Tuffin (Fellow of Corpus College, Cambridge), can
instruct in the Syriac language, and the elements of Constitutional
law. But as she is only eighteen years of age, and of exceedingly
pleasing personal appearance, perhaps this young lady may be
objectionable in Sir Huddleston Fuddleston's family.

Miss Letitia Hawky, on the other hand, is not personally well-favoured.
She is twenty-nine; her face is much pitted with the small-pox.  She
has a halt in her gait, red hair, and a trifling obliquity of vision.
Both ladies are endowed with EVERY MORAL AND RELIGIOUS VIRTUE.  Their
terms, of course, are such as their accomplishments merit.  With my
most grateful respects to the Reverend Bute Crawley, I have the honour
to be,

Dear Madam,

Your most faithful and obedient servant, Barbara Pinkerton.

P.S.  The Miss Sharp, whom you mention as governess to Sir Pitt
Crawley, Bart., M.P., was a pupil of mine, and I have nothing to say in
her disfavour. Though her appearance is disagreeable, we cannot control
the operations of nature: and though her parents were disreputable (her
father being a painter, several times bankrupt, and her mother, as I
have since learned, with horror, a dancer at the Opera); yet her
talents are considerable, and I cannot regret that I received her OUT
OF CHARITY.  My dread is, lest the principles of the mother---who was
represented to me as a French Countess, forced to emigrate in the late
revolutionary horrors; but who, as I have since found, was a person of
the very lowest order and morals---should at any time prove to be
HEREDITARY in the unhappy young woman whom I took as AN OUTCAST.  But
her principles have hitherto been correct (I believe), and I am sure
nothing will occur to injure them in the elegant and refined circle of
the eminent Sir Pitt Crawley.


Miss Rebecca Sharp to Miss Amelia Sedley.

I have not written to my beloved Amelia for these many weeks past, for
what news was there to tell of the sayings and doings at Humdrum Hall,
as I have christened it; and what do you care whether the turnip crop
is good or bad; whether the fat pig weighed thirteen stone or fourteen;
and whether the beasts thrive well upon mangelwurzel? Every day since I
last wrote has been like its neighbour.  Before breakfast, a walk with
Sir Pitt and his spud; after breakfast studies (such as they are) in
the schoolroom; after schoolroom, reading and writing about lawyers,
leases, coal-mines, canals, with Sir Pitt (whose secretary I am
become); after dinner, Mr.\ Crawley's discourses on the baronet's
backgammon; during both of which amusements my lady looks on with equal
placidity.  She has become rather more interesting by being ailing of
late, which has brought a new visitor to the Hall, in the person of a
young doctor. Well, my dear, young women need never despair.  The young
doctor gave a certain friend of yours to understand that, if she chose
to be Mrs.\ Glauber, she was welcome to ornament the surgery! I told his
impudence that the gilt pestle and mortar was quite ornament enough; as
if I was born, indeed, to be a country surgeon's wife! Mr.\ Glauber went
home seriously indisposed at his rebuff, took a cooling draught, and is
now quite cured.  Sir Pitt applauded my resolution highly; he would be
sorry to lose his little secretary, I think; and I believe the old
wretch likes me as much as it is in his nature to like any one.  Marry,
indeed! and with a country apothecary, after---  No, no, one cannot so
soon forget old associations, about which I will talk no more.  Let us
return to Humdrum Hall.

For some time past it is Humdrum Hall no longer. My dear, Miss Crawley
has arrived with her fat horses, fat servants, fat spaniel---the great
rich Miss Crawley, with seventy thousand pounds in the five per cents.,
whom, or I had better say WHICH, her two brothers adore.  She looks
very apoplectic, the dear soul; no wonder her brothers are anxious
about her.  You should see them struggling to settle her cushions, or
to hand her coffee! ``When I come into the country,'' she says (for she
has a great deal of humour), ``I leave my toady, Miss Briggs, at home.
My brothers are my toadies here, my dear, and a pretty pair they are!''

When she comes into the country our hall is thrown open, and for a
month, at least, you would fancy old Sir Walpole was come to life
again.  We have dinner-parties, and drive out in the coach-and-four---the
footmen put on their newest canary-coloured liveries; we drink claret
and champagne as if we were accustomed to it every day.  We have wax
candles in the schoolroom, and fires to warm ourselves with.  Lady
Crawley is made to put on the brightest pea-green in her wardrobe, and
my pupils leave off their thick shoes and tight old tartan pelisses,
and wear silk stockings and muslin frocks, as fashionable baronets'
daughters should.  Rose came in yesterday in a sad plight---the
Wiltshire sow (an enormous pet of hers) ran her down, and destroyed a
most lovely flowered lilac silk dress by dancing over it---had this
happened a week ago, Sir Pitt would have sworn frightfully, have boxed
the poor wretch's ears, and put her upon bread and water for a month.
All he said was, ``I'll serve you out, Miss, when your aunt's gone,'' and
laughed off the accident as quite trivial.  Let us hope his wrath will
have passed away before Miss Crawley's departure.  I hope so, for Miss
Rose's sake, I am sure. What a charming reconciler and peacemaker money
is!

Another admirable effect of Miss Crawley and her seventy thousand
pounds is to be seen in the conduct of the two brothers Crawley.  I
mean the baronet and the rector, not OUR brothers---but the former, who
hate each other all the year round, become quite loving at Christmas.
I wrote to you last year how the abominable horse-racing rector was in
the habit of preaching clumsy sermons at us at church, and how Sir Pitt
snored in answer.  When Miss Crawley arrives there is no such thing as
quarrelling heard of---the Hall visits the Rectory, and vice versa---the
parson and the Baronet talk about the pigs and the poachers, and the
county business, in the most affable manner, and without quarrelling in
their cups, I believe---indeed Miss Crawley won't hear of their
quarrelling, and vows that she will leave her money to the Shropshire
Crawleys if they offend her.  If they were clever people, those
Shropshire Crawleys, they might have it all, I think; but the
Shropshire Crawley is a clergyman like his Hampshire cousin, and
mortally offended Miss Crawley (who had fled thither in a fit of rage
against her impracticable brethren) by some strait-laced notions of
morality.  He would have prayers in the house, I believe.

Our sermon books are shut up when Miss Crawley arrives, and Mr.\ Pitt,
whom she abominates, finds it convenient to go to town.  On the other
hand, the young dandy---``blood,'' I believe, is the term---Captain Crawley
makes his appearance, and I suppose you will like to know what sort of
a person he is.

Well, he is a very large young dandy.  He is six feet high, and speaks
with a great voice; and swears a great deal; and orders about the
servants, who all adore him nevertheless; for he is very generous of
his money, and the domestics will do anything for him. Last week the
keepers almost killed a bailiff and his man who came down from London
to arrest the Captain, and who were found lurking about the Park
wall---they beat them, ducked them, and were going to shoot them for
poachers, but the baronet interfered.

The Captain has a hearty contempt for his father, I can see, and calls
him an old PUT, an old SNOB, an old CHAW-BACON, and numberless other
pretty names.  He has a DREADFUL REPUTATION among the ladies. He brings
his hunters home with him, lives with the Squires of the county, asks
whom he pleases to dinner, and Sir Pitt dares not say no, for fear of
offending Miss Crawley, and missing his legacy when she dies of her
apoplexy. Shall I tell you a compliment the Captain paid me?  I must,
it is so pretty.  One evening we actually had a dance; there was Sir
Huddleston Fuddleston and his family, Sir Giles Wapshot and his young
ladies, and I don't know how many more.  Well, I heard him say---``By
Jove, she's a neat little filly!'' meaning your humble servant; and he
did me the honour to dance two country-dances with me.  He gets on
pretty gaily with the young Squires, with whom he drinks, bets, rides,
and talks about hunting and shooting; but he says the country girls are
BORES; indeed, I don't think he is far wrong. You should see the
contempt with which they look down on poor me! When they dance I sit
and play the piano very demurely; but the other night, coming in rather
flushed from the dining-room, and seeing me employed in this way, he
swore out loud that I was the best dancer in the room, and took a great
oath that he would have the fiddlers from Mudbury.

``I'll go and play a country-dance,'' said Mrs.\ Bute Crawley, very
readily (she is a little, black-faced old woman in a turban, rather
crooked, and with very twinkling eyes); and after the Captain and your
poor little Rebecca had performed a dance together, do you know she
actually did me the honour to compliment me upon my steps! Such a thing
was never heard of before; the proud Mrs.\ Bute Crawley, first cousin to
the Earl of Tiptoff, who won't condescend to visit Lady Crawley, except
when her sister is in the country.  Poor Lady Crawley! during most part
of these gaieties, she is upstairs taking pills.

Mrs.\ Bute has all of a sudden taken a great fancy to me.  ``My dear Miss
Sharp,'' she says, ``why not bring over your girls to the Rectory?---their
cousins will be so happy to see them.'' I know what she means.  Signor
Clementi did not teach us the piano for nothing; at which price Mrs.\ %
Bute hopes to get a professor for her children. I can see through her
schemes, as though she told them to me; but I shall go, as I am
determined to make myself agreeable---is it not a poor governess's duty,
who has not a friend or protector in the world? The Rector's wife paid
me a score of compliments about the progress my pupils made, and
thought, no doubt, to touch my heart---poor, simple, country soul!---as
if I cared a fig about my pupils!

Your India muslin and your pink silk, dearest Amelia, are said to
become me very well.  They are a good deal worn now; but, you know, we
poor girls can't afford des fraiches toilettes.  Happy, happy you! who
have but to drive to St.\ James's Street, and a dear mother who will
give you any thing you ask.  Farewell, dearest girl,

Your affectionate Rebecca.

P.S.---I wish you could have seen the faces of the Miss Blackbrooks
(Admiral Blackbrook's daughters, my dear), fine young ladies, with
dresses from London, when Captain Rawdon selected poor me for a partner!


When Mrs.\ Bute Crawley (whose artifices our ingenious Rebecca had so
soon discovered) had procured from Miss Sharp the promise of a visit,
she induced the all-powerful Miss Crawley to make the necessary
application to Sir Pitt, and the good-natured old lady, who loved to be
gay herself, and to see every one gay and happy round about her, was
quite charmed, and ready to establish a reconciliation and intimacy
between her two brothers. It was therefore agreed that the young people
of both families should visit each other frequently for the future, and
the friendship of course lasted as long as the jovial old mediatrix was
there to keep the peace.

``Why did you ask that scoundrel, Rawdon Crawley, to dine?'' said the
Rector to his lady, as they were walking home through the park.  ``I
don't want the fellow.  He looks down upon us country people as so many
blackamoors. He's never content unless he gets my yellow-sealed wine,
which costs me ten shillings a bottle, hang him! Besides, he's such an
infernal character---he's a gambler---he's a drunkard---he's a profligate
in every way.  He shot a man in a duel---he's over head and ears in
debt, and he's robbed me and mine of the best part of Miss Crawley's
fortune.  Waxy says she has him''---here the Rector shook his fist at the
moon, with something very like an oath, and added, in a melancholious
tone, ``---down in her will for fifty thousand; and there won't be above
thirty to divide.''

``I think she's going,'' said the Rector's wife.  ``She was very red in
the face when we left dinner.  I was obliged to unlace her.''

``She drank seven glasses of champagne,'' said the reverend gentleman, in
a low voice; ``and filthy champagne it is, too, that my brother poisons
us with---but you women never know what's what.''

``We know nothing,'' said Mrs.\ Bute Crawley.

``She drank cherry-brandy after dinner,'' continued his Reverence, ``and
took curacao with her coffee.  I wouldn't take a glass for a five-pound
note: it kills me with heartburn.  She can't stand it, Mrs.\ %
Crawley---she must go---flesh and blood won't bear it! and I lay five to
two, Matilda drops in a year.''

Indulging in these solemn speculations, and thinking about his debts,
and his son Jim at College, and Frank at Woolwich, and the four girls,
who were no beauties, poor things, and would not have a penny but what
they got from the aunt's expected legacy, the Rector and his lady
walked on for a while.

``Pitt can't be such an infernal villain as to sell the reversion of the
living.  And that Methodist milksop of an eldest son looks to
Parliament,'' continued Mr.\ Crawley, after a pause.

``Sir Pitt Crawley will do anything,'' said the Rector's wife.  ``We must
get Miss Crawley to make him promise it to James.''

``Pitt will promise anything,'' replied the brother.  ``He promised he'd
pay my college bills, when my father died; he promised he'd build the
new wing to the Rectory; he promised he'd let me have Jibb's field and
the Six-acre Meadow---and much he executed his promises! And it's to
this man's son---this scoundrel, gambler, swindler, murderer of a Rawdon
Crawley, that Matilda leaves the bulk of her money.  I say it's
un-Christian.  By Jove, it is. The infamous dog has got every vice
except hypocrisy, and that belongs to his brother.''

``Hush, my dearest love! we're in Sir Pitt's grounds,'' interposed his
wife.

``I say he has got every vice, Mrs.\ Crawley.  Don't Ma'am, bully me.
Didn't he shoot Captain Marker? Didn't he rob young Lord Dovedale at
the Cocoa-Tree? Didn't he cross the fight between Bill Soames and the
Cheshire Trump, by which I lost forty pound? You know he did; and as
for the women, why, you heard that before me, in my own magistrate's
room.''

``For heaven's sake, Mr.\ Crawley,'' said the lady, ``spare me the details.''

``And you ask this villain into your house!'' continued the exasperated
Rector.  ``You, the mother of a young family---the wife of a clergyman of
the Church of England.  By Jove!''

``Bute Crawley, you are a fool,'' said the Rector's wife scornfully.

``Well, Ma'am, fool or not---and I don't say, Martha, I'm so clever as
you are, I never did.  But I won't meet Rawdon Crawley, that's flat.
I'll go over to Huddleston, that I will, and see his black greyhound,
Mrs.\ Crawley; and I'll run Lancelot against him for fifty.  By Jove, I
will; or against any dog in England.  But I won't meet that beast
Rawdon Crawley.''

``Mr.\ Crawley, you are intoxicated, as usual,'' replied his wife.  And
the next morning, when the Rector woke, and called for small beer, she
put him in mind of his promise to visit Sir Huddleston Fuddleston on
Saturday, and as he knew he should have a wet night, it was agreed that
he might gallop back again in time for church on Sunday morning.  Thus
it will be seen that the parishioners of Crawley were equally happy in
their Squire and in their Rector.

Miss Crawley had not long been established at the Hall before Rebecca's
fascinations had won the heart of that good-natured London rake, as
they had of the country innocents whom we have been describing.  Taking
her accustomed drive, one day, she thought fit to order that ``that
little governess'' should accompany her to Mudbury. Before they had
returned Rebecca had made a conquest of her; having made her laugh four
times, and amused her during the whole of the little journey.

``Not let Miss Sharp dine at table!'' said she to Sir Pitt, who had
arranged a dinner of ceremony, and asked all the neighbouring baronets.
``My dear creature, do you suppose I can talk about the nursery with
Lady Fuddleston, or discuss justices' business with that goose, old Sir
Giles Wapshot? I insist upon Miss Sharp appearing.  Let Lady Crawley
remain upstairs, if there is no room. But little Miss Sharp! Why, she's
the only person fit to talk to in the county!''

Of course, after such a peremptory order as this, Miss Sharp, the
governess, received commands to dine with the illustrious company below
stairs.  And when Sir Huddleston had, with great pomp and ceremony,
handed Miss Crawley in to dinner, and was preparing to take his place
by her side, the old lady cried out, in a shrill voice, ``Becky Sharp!
Miss Sharp!  Come you and sit by me and amuse me; and let Sir
Huddleston sit by Lady Wapshot.''

When the parties were over, and the carriages had rolled away, the
insatiable Miss Crawley would say, ``Come to my dressing room, Becky,
and let us abuse the company''---which, between them, this pair of
friends did perfectly.  Old Sir Huddleston wheezed a great deal at
dinner; Sir Giles Wapshot had a particularly noisy manner of imbibing
his soup, and her ladyship a wink of the left eye; all of which Becky
caricatured to admiration; as well as the particulars of the night's
conversation; the politics; the war; the quarter-sessions; the famous
run with the H.H., and those heavy and dreary themes, about which
country gentlemen converse.  As for the Misses Wapshot's toilettes and
Lady Fuddleston's famous yellow hat, Miss Sharp tore them to tatters,
to the infinite amusement of her audience.

``My dear, you are a perfect trouvaille,'' Miss Crawley would say.  ``I
wish you could come to me in London, but I couldn't make a butt of you
as I do of poor Briggs no, no, you little sly creature; you are too
clever---Isn't she, Firkin?''

Mrs.\ Firkin (who was dressing the very small remnant of hair which
remained on Miss Crawley's pate), flung up her head and said, ``I think
Miss is very clever,'' with the most killing sarcastic air.  In fact,
Mrs.\ Firkin had that natural jealousy which is one of the main
principles of every honest woman.

After rebuffing Sir Huddleston Fuddleston, Miss Crawley ordered that
Rawdon Crawley should lead her in to dinner every day, and that Becky
should follow with her cushion---or else she would have Becky's arm and
Rawdon with the pillow.  ``We must sit together,'' she said. ``We're the
only three Christians in the county, my love''---in which case, it must
be confessed, that religion was at a very low ebb in the county of
Hants.

Besides being such a fine religionist, Miss Crawley was, as we have
said, an Ultra-liberal in opinions, and always took occasion to express
these in the most candid manner.

``What is birth, my dear!'' she would say to Rebecca---``Look at my brother
Pitt; look at the Huddlestons, who have been here since Henry II; look
at poor Bute at the parsonage---is any one of them equal to you in
intelligence or breeding? Equal to you---they are not even equal to poor
dear Briggs, my companion, or Bowls, my butler. You, my love, are a
little paragon---positively a little jewel---You have more brains than
half the shire---if merit had its reward you ought to be a Duchess---no,
there ought to be no duchesses at all---but you ought to have no
superior, and I consider you, my love, as my equal in every respect;
and---will you put some coals on the fire, my dear; and will you pick
this dress of mine, and alter it, you who can do it so well?'' So this
old philanthropist used to make her equal run of her errands, execute
her millinery, and read her to sleep with French novels, every night.

At this time, as some old readers may recollect, the genteel world had
been thrown into a considerable state of excitement by two events,
which, as the papers say, might give employment to the gentlemen of the
long robe. Ensign Shafton had run away with Lady Barbara Fitzurse, the
Earl of Bruin's daughter and heiress; and poor Vere Vane, a gentleman
who, up to forty, had maintained a most respectable character and
reared a numerous family, suddenly and outrageously left his home, for
the sake of Mrs.\ Rougemont, the actress, who was sixty-five years of
age.

``That was the most beautiful part of dear Lord Nelson's character,''
Miss Crawley said.  ``He went to the deuce for a woman.  There must be
good in a man who will do that.  I adore all impudent matches.---  What
I like best, is for a nobleman to marry a miller's daughter, as Lord
Flowerdale did---it makes all the women so angry---I wish some great man
would run away with you, my dear; I'm sure you're pretty enough.''

``Two post-boys!---Oh, it would be delightful!'' Rebecca owned.

``And what I like next best, is for a poor fellow to run away with a
rich girl.  I have set my heart on Rawdon running away with some one.''

``A rich some one, or a poor some one?''

``Why, you goose! Rawdon has not a shilling but what I give him.  He is
crible de dettes---he must repair his fortunes, and succeed in the
world.''

``Is he very clever?'' Rebecca asked.

``Clever, my love?---not an idea in the world beyond his horses, and his
regiment, and his hunting, and his play; but he must succeed---he's so
delightfully wicked.  Don't you know he has hit a man, and shot an
injured father through the hat only? He's adored in his regiment; and
all the young men at Wattier's and the Cocoa-Tree swear by him.''

When Miss Rebecca Sharp wrote to her beloved friend the account of the
little ball at Queen's Crawley, and the manner in which, for the first
time, Captain Crawley had distinguished her, she did not, strange to
relate, give an altogether accurate account of the transaction.  The
Captain had distinguished her a great number of times before.  The
Captain had met her in a half-score of walks. The Captain had lighted
upon her in a half-hundred of corridors and passages.  The Captain had
hung over her piano twenty times of an evening (my Lady was now
upstairs, being ill, and nobody heeded her) as Miss Sharp sang.  The
Captain had written her notes (the best that the great blundering
dragoon could devise and spell; but dulness gets on as well as any
other quality with women).  But when he put the first of the notes into
the leaves of the song she was singing, the little governess, rising
and looking him steadily in the face, took up the triangular missive
daintily, and waved it about as if it were a cocked hat, and she,
advancing to the enemy, popped the note into the fire, and made him a
very low curtsey, and went back to her place, and began to sing away
again more merrily than ever.

``What's that?'' said Miss Crawley, interrupted in her after-dinner doze
by the stoppage of the music.

``It's a false note,'' Miss Sharp said with a laugh; and Rawdon Crawley
fumed with rage and mortification.

Seeing the evident partiality of Miss Crawley for the new governess,
how good it was of Mrs.\ Bute Crawley not to be jealous, and to welcome
the young lady to the Rectory, and not only her, but Rawdon Crawley,
her husband's rival in the Old Maid's five per cents! They became very
fond of each other's society, Mrs.\ Crawley and her nephew.  He gave up
hunting; he declined entertainments at Fuddleston: he would not dine
with the mess of the depot at Mudbury: his great pleasure was to stroll
over to Crawley parsonage---whither Miss Crawley came too; and as their
mamma was ill, why not the children with Miss Sharp? So the children
(little dears!) came with Miss Sharp; and of an evening some of the
party would walk back together.  Not Miss Crawley---she preferred her
carriage---but the walk over the Rectory fields, and in at the little
park wicket, and through the dark plantation, and up the checkered
avenue to Queen's Crawley, was charming in the moonlight to two such
lovers of the picturesque as the Captain and Miss Rebecca.

``O those stars, those stars!'' Miss Rebecca would say, turning her
twinkling green eyes up towards them.  ``I feel myself almost a spirit
when I gaze upon them.''

``O---ah---Gad---yes, so do I exactly, Miss Sharp,'' the other enthusiast
replied.  ``You don't mind my cigar, do you, Miss Sharp?''  Miss Sharp
loved the smell of a cigar out of doors beyond everything in the
world---and she just tasted one too, in the prettiest way possible, and
gave a little puff, and a little scream, and a little giggle, and
restored the delicacy to the Captain, who twirled his moustache, and
straightway puffed it into a blaze that glowed quite red in the dark
plantation, and swore---``Jove---aw---Gad---aw---it's the finest segaw I ever
smoked in the world aw,'' for his intellect and conversation were alike
brilliant and becoming to a heavy young dragoon.

Old Sir Pitt, who was taking his pipe and beer, and talking to John
Horrocks about a ``ship'' that was to be killed, espied the pair so
occupied from his study-window, and with dreadful oaths swore that if
it wasn't for Miss Crawley, he'd take Rawdon and bundle un out of
doors, like a rogue as he was.

``He be a bad'n, sure enough,'' Mr.\ Horrocks remarked; ``and his man
Flethers is wuss, and have made such a row in the housekeeper's room
about the dinners and hale, as no lord would make---but I think Miss
Sharp's a match for'n, Sir Pitt,'' he added, after a pause.

And so, in truth, she was---for father and son too.



\chapter{Quite a Sentimental Chapter}

We must now take leave of Arcadia, and those amiable people practising
the rural virtues there, and travel back to London, to inquire what has
become of Miss Amelia.

``We don't care a fig for her,'' writes some unknown correspondent with a
pretty little handwriting and a pink seal to her note.  ``She is fade and
insipid,'' and adds some more kind remarks in this strain, which I should
never have repeated at all, but that they are in truth prodigiously
complimentary to the young lady whom they concern.

Has the beloved reader, in his experience of society, never heard
similar remarks by good-natured female friends; who always wonder what
you CAN see in Miss Smith that is so fascinating; or what COULD induce
Major Jones to propose for that silly insignificant simpering Miss
Thompson, who has nothing but her wax-doll face to recommend her? What
is there in a pair of pink cheeks and blue eyes forsooth? these dear
Moralists ask, and hint wisely that the gifts of genius, the
accomplishments of the mind, the mastery of Mangnall's Questions, and a
ladylike knowledge of botany and geology, the knack of making poetry,
the power of rattling sonatas in the Herz-manner, and so forth, are far
more valuable endowments for a female, than those fugitive charms which
a few years will inevitably tarnish.  It is quite edifying to hear
women speculate upon the worthlessness and the duration of beauty.

But though virtue is a much finer thing, and those hapless creatures
who suffer under the misfortune of good looks ought to be continually
put in mind of the fate which awaits them; and though, very likely, the
heroic female character which ladies admire is a more glorious and
beautiful object than the kind, fresh, smiling, artless, tender little
domestic goddess, whom men are inclined to worship---yet the latter and
inferior sort of women must have this consolation---that the men do
admire them after all; and that, in spite of all our kind friends'
warnings and protests, we go on in our desperate error and folly, and
shall to the end of the chapter. Indeed, for my own part, though I have
been repeatedly told by persons for whom I have the greatest respect,
that Miss Brown is an insignificant chit, and Mrs.\ White has nothing
but her petit minois chiffonne, and Mrs.\ Black has not a word to say
for herself; yet I know that I have had the most delightful
conversations with Mrs.\ Black (of course, my dear Madam, they are
inviolable): I see all the men in a cluster round Mrs.\ White's chair:
all the young fellows battling to dance with Miss Brown; and so I am
tempted to think that to be despised by her sex is a very great
compliment to a woman.

The young ladies in Amelia's society did this for her very
satisfactorily.  For instance, there was scarcely any point upon which
the Misses Osborne, George's sisters, and the Mesdemoiselles Dobbin
agreed so well as in their estimate of her very trifling merits: and
their wonder that their brothers could find any charms in her.  ``We are
kind to her,'' the Misses Osborne said, a pair of fine black-browed
young ladies who had had the best of governesses, masters, and
milliners; and they treated her with such extreme kindness and
condescension, and patronised her so insufferably, that the poor little
thing was in fact perfectly dumb in their presence, and to all outward
appearance as stupid as they thought her.  She made efforts to like
them, as in duty bound, and as sisters of her future husband.  She
passed ``long mornings'' with them---the most dreary and serious of
forenoons.  She drove out solemnly in their great family coach with
them, and Miss Wirt their governess, that raw-boned Vestal.  They took
her to the ancient concerts by way of a treat, and to the oratorio, and
to St.\ Paul's to see the charity children, where in such terror was she
of her friends, she almost did not dare be affected by the hymn the
children sang.  Their house was comfortable; their papa's table rich
and handsome; their society solemn and genteel; their self-respect
prodigious; they had the best pew at the Foundling: all their habits
were pompous and orderly, and all their amusements intolerably dull and
decorous. After every one of her visits (and oh how glad she was when
they were over!) Miss Osborne and Miss Maria Osborne, and Miss Wirt,
the vestal governess, asked each other with increased wonder, ``What
could George find in that creature?''

How is this? some carping reader exclaims.  How is it that Amelia, who
had such a number of friends at school, and was so beloved there, comes
out into the world and is spurned by her discriminating sex? My dear
sir, there were no men at Miss Pinkerton's establishment except the old
dancing-master; and you would not have had the girls fall out about
HIM? When George, their handsome brother, ran off directly after
breakfast, and dined from home half-a-dozen times a week, no wonder the
neglected sisters felt a little vexation.  When young Bullock (of the
firm of Hulker, Bullock \& Co., Bankers, Lombard Street), who had been
making up to Miss Maria the last two seasons, actually asked Amelia to
dance the cotillon, could you expect that the former young lady should
be pleased? And yet she said she was, like an artless forgiving
creature.  ``I'm so delighted you like dear Amelia,'' she said quite
eagerly to Mr.\ Bullock after the dance.  ``She's engaged to my brother
George; there's not much in her, but she's the best-natured and most
unaffected young creature: at home we're all so fond of her.'' Dear
girl! who can calculate the depth of affection expressed in that
enthusiastic SO?

Miss Wirt and these two affectionate young women so earnestly and
frequently impressed upon George Osborne's mind the enormity of the
sacrifice he was making, and his romantic generosity in throwing
himself away upon Amelia, that I'm not sure but that he really thought
he was one of the most deserving characters in the British army, and
gave himself up to be loved with a good deal of easy resignation.

Somehow, although he left home every morning, as was stated, and dined
abroad six days in the week, when his sisters believed the infatuated
youth to be at Miss Sedley's apron-strings: he was NOT always with
Amelia, whilst the world supposed him at her feet. Certain it is that
on more occasions than one, when Captain Dobbin called to look for his
friend, Miss Osborne (who was very attentive to the Captain, and
anxious to hear his military stories, and to know about the health of
his dear Mamma), would laughingly point to the opposite side of the
square, and say, ``Oh, you must go to the Sedleys' to ask for George; WE
never see him from morning till night.'' At which kind of speech the
Captain would laugh in rather an absurd constrained manner, and turn
off the conversation, like a consummate man of the world, to some topic
of general interest, such as the Opera, the Prince's last ball at
Carlton House, or the weather---that blessing to society.

``What an innocent it is, that pet of yours,'' Miss Maria would then say
to Miss Jane, upon the Captain's departure.  ``Did you see how he
blushed at the mention of poor George on duty?''

``It's a pity Frederick Bullock hadn't some of his modesty, Maria,''
replies the elder sister, with a toss of he head.

``Modesty!  Awkwardness you mean, Jane.  I don't want Frederick to
trample a hole in my muslin frock, as Captain Dobbin did in yours at
Mrs.\ Perkins'.''

``In YOUR frock, he, he!  How could he? Wasn't he dancing with Amelia?''

The fact is, when Captain Dobbin blushed so, and looked so awkward, he
remembered a circumstance of which he did not think it was necessary to
inform the young ladies, viz., that he had been calling at Mr.\ Sedley's
house already, on the pretence of seeing George, of course, and George
wasn't there, only poor little Amelia, with rather a sad wistful face,
seated near the drawing-room window, who, after some very trifling
stupid talk, ventured to ask, was there any truth in the report that
the regiment was soon to be ordered abroad; and had Captain Dobbin seen
Mr.\ Osborne that day?

The regiment was not ordered abroad as yet; and Captain Dobbin had not
seen George.  ``He was with his sister, most likely,'' the Captain said.
``Should he go and fetch the truant?''  So she gave him her hand kindly
and gratefully: and he crossed the square; and she waited and waited,
but George never came.

Poor little tender heart! and so it goes on hoping and beating, and
longing and trusting.  You see it is not much of a life to describe.
There is not much of what you call incident in it.  Only one feeling
all day---when will he come? only one thought to sleep and wake upon. I
believe George was playing billiards with Captain Cannon in Swallow
Street at the time when Amelia was asking Captain Dobbin about him; for
George was a jolly sociable fellow, and excellent in all games of skill.

Once, after three days of absence, Miss Amelia put on her bonnet, and
actually invaded the Osborne house. ``What! leave our brother to come to
us?'' said the young ladies.  ``Have you had a quarrel, Amelia? Do tell
us!'' No, indeed, there had been no quarrel.  ``Who could quarrel with
him?'' says she, with her eyes filled with tears. She only came over
to---to see her dear friends; they had not met for so long.  And this
day she was so perfectly stupid and awkward, that the Misses Osborne
and their governess, who stared after her as she went sadly away,
wondered more than ever what George could see in poor little Amelia.

Of course they did.  How was she to bare that timid little heart for
the inspection of those young ladies with their bold black eyes? It was
best that it should shrink and hide itself.  I know the Misses Osborne
were excellent critics of a Cashmere shawl, or a pink satin slip; and
when Miss Turner had hers dyed purple, and made into a spencer; and
when Miss Pickford had her ermine tippet twisted into a muff and
trimmings, I warrant you the changes did not escape the two intelligent
young women before mentioned.  But there are things, look you, of a
finer texture than fur or satin, and all Solomon's glories, and all the
wardrobe of the Queen of Sheba---things whereof the beauty escapes the
eyes of many connoisseurs.  And there are sweet modest little souls on
which you light, fragrant and blooming tenderly in quiet shady places;
and there are garden-ornaments, as big as brass warming-pans, that are
fit to stare the sun itself out of countenance.  Miss Sedley was not of
the sunflower sort; and I say it is out of the rules of all proportion
to draw a violet of the size of a double dahlia.

No, indeed; the life of a good young girl who is in the paternal nest
as yet, can't have many of those thrilling incidents to which the
heroine of romance commonly lays claim.  Snares or shot may take off
the old birds foraging without---hawks may be abroad, from which they
escape or by whom they suffer; but the young ones in the nest have a
pretty comfortable unromantic sort of existence in the down and the
straw, till it comes to their turn, too, to get on the wing. While
Becky Sharp was on her own wing in the country, hopping on all sorts of
twigs, and amid a multiplicity of traps, and pecking up her food quite
harmless and successful, Amelia lay snug in her home of Russell Square;
if she went into the world, it was under the guidance of the elders;
nor did it seem that any evil could befall her or that opulent cheery
comfortable home in which she was affectionately sheltered. Mamma had
her morning duties, and her daily drive, and the delightful round of
visits and shopping which forms the amusement, or the profession as you
may call it, of the rich London lady.  Papa conducted his mysterious
operations in the City---a stirring place in those days, when war was
raging all over Europe, and empires were being staked; when the
``Courier'' newspaper had tens of thousands of subscribers; when one day
brought you a battle of Vittoria, another a burning of Moscow, or a
newsman's horn blowing down Russell Square about dinner-time, announced
such a fact as---``Battle of Leipsic---six hundred thousand men
engaged---total defeat of the French---two hundred thousand killed.'' Old
Sedley once or twice came home with a very grave face; and no wonder,
when such news as this was agitating all the hearts and all the Stocks
of Europe.

Meanwhile matters went on in Russell Square, Bloomsbury, just as if
matters in Europe were not in the least disorganised.  The retreat from
Leipsic made no difference in the number of meals Mr.\ Sambo took in the
servants' hall; the allies poured into France, and the dinner-bell rang
at five o'clock just as usual.  I don't think poor Amelia cared
anything about Brienne and Montmirail, or was fairly interested in the
war until the abdication of the Emperor; when she clapped her hands and
said prayers---oh, how grateful! and flung herself into George Osborne's
arms with all her soul, to the astonishment of everybody who witnessed
that ebullition of sentiment. The fact is, peace was declared, Europe
was going to be at rest; the Corsican was overthrown, and Lieutenant
Osborne's regiment would not be ordered on service.  That was the way
in which Miss Amelia reasoned.  The fate of Europe was Lieutenant
George Osborne to her.  His dangers being over, she sang Te Deum.  He
was her Europe: her emperor: her allied monarchs and august prince
regent.  He was her sun and moon; and I believe she thought the grand
illumination and ball at the Mansion House, given to the sovereigns,
were especially in honour of George Osborne.

We have talked of shift, self, and poverty, as those dismal instructors
under whom poor Miss Becky Sharp got her education. Now, love was Miss
Amelia Sedley's last tutoress, and it was amazing what progress our
young lady made under that popular teacher.  In the course of fifteen
or eighteen months' daily and constant attention to this eminent
finishing governess, what a deal of secrets Amelia learned, which Miss
Wirt and the black-eyed young ladies over the way, which old Miss
Pinkerton of Chiswick herself, had no cognizance of!  As, indeed, how
should any of those prim and reputable virgins?  With Misses P. and W.
the tender passion is out of the question: I would not dare to breathe
such an idea regarding them.  Miss Maria Osborne, it is true, was
``attached'' to Mr.\ Frederick Augustus Bullock, of the firm of Hulker,
Bullock \& Bullock; but hers was a most respectable attachment, and she
would have taken Bullock Senior just the same, her mind being fixed---as
that of a well-bred young woman should be---upon a house in Park Lane, a
country house at Wimbledon, a handsome chariot, and two prodigious tall
horses and footmen, and a fourth of the annual profits of the eminent
firm of Hulker \& Bullock, all of which advantages were represented in
the person of Frederick Augustus. Had orange blossoms been invented
then (those touching emblems of female purity imported by us from
France, where people's daughters are universally sold in marriage),
Miss Maria, I say, would have assumed the spotless wreath, and stepped
into the travelling carriage by the side of gouty, old, bald-headed,
bottle-nosed Bullock Senior; and devoted her beautiful existence to his
happiness with perfect modesty---only the old gentleman was married
already; so she bestowed her young affections on the junior partner.
Sweet, blooming, orange flowers!  The other day I saw Miss Trotter
(that was), arrayed in them, trip into the travelling carriage at St.\ %
George's, Hanover Square, and Lord Methuselah hobbled in after. With
what an engaging modesty she pulled down the blinds of the chariot---the
dear innocent!  There were half the carriages of Vanity Fair at the
wedding.

This was not the sort of love that finished Amelia's education; and in
the course of a year turned a good young girl into a good young
woman---to be a good wife presently, when the happy time should come.
This young person (perhaps it was very imprudent in her parents to
encourage her, and abet her in such idolatry and silly romantic ideas)
loved, with all her heart, the young officer in His Majesty's service
with whom we have made a brief acquaintance.  She thought about him the
very first moment on waking; and his was the very last name mentioned
in her prayers.  She never had seen a man so beautiful or so clever:
such a figure on horseback: such a dancer: such a hero in general.
Talk of the Prince's bow! what was it to George's? She had seen Mr.\ %
Brummell, whom everybody praised so.  Compare such a person as that to
her George! Not amongst all the beaux at the Opera (and there were
beaux in those days with actual opera hats) was there any one to equal
him.  He was only good enough to be a fairy prince; and oh, what
magnanimity to stoop to such a humble Cinderella!  Miss Pinkerton would
have tried to check this blind devotion very likely, had she been
Amelia's confidante; but not with much success, depend upon it.  It is
in the nature and instinct of some women.  Some are made to scheme, and
some to love; and I wish any respected bachelor that reads this may
take the sort that best likes him.

While under this overpowering impression, Miss Amelia neglected her
twelve dear friends at Chiswick most cruelly, as such selfish people
commonly will do.  She had but this subject, of course, to think about;
and Miss Saltire was too cold for a confidante, and she couldn't bring
her mind to tell Miss Swartz, the woolly-haired young heiress from St.\ %
Kitt's.  She had little Laura Martin home for the holidays; and my
belief is, she made a confidante of her, and promised that Laura should
come and live with her when she was married, and gave Laura a great
deal of information regarding the passion of love, which must have been
singularly useful and novel to that little person.  Alas, alas!  I fear
poor Emmy had not a well-regulated mind.

What were her parents doing, not to keep this little heart from beating
so fast?  Old Sedley did not seem much to notice matters. He was graver
of late, and his City affairs absorbed him.  Mrs.\ Sedley was of so easy
and uninquisitive a nature that she wasn't even jealous.  Mr.\ Jos was
away, being besieged by an Irish widow at Cheltenham.  Amelia had the
house to herself---ah! too much to herself sometimes---not that she ever
doubted; for, to be sure, George must be at the Horse Guards; and he
can't always get leave from Chatham; and he must see his friends and
sisters, and mingle in society when in town (he, such an ornament to
every society!); and when he is with the regiment, he is too tired to
write long letters. I know where she kept that packet she had---and can
steal in and out of her chamber like Iachimo---like Iachimo?  No---that
is a bad part. I will only act Moonshine, and peep harmless into the
bed where faith and beauty and innocence lie dreaming.

But if Osborne's were short and soldierlike letters, it must be
confessed, that were Miss Sedley's letters to Mr.\ Osborne to be
published, we should have to extend this novel to such a multiplicity
of volumes as not the most sentimental reader could support; that she
not only filled sheets of large paper, but crossed them with the most
astonishing perverseness; that she wrote whole pages out of
poetry-books without the least pity; that she underlined words and
passages with quite a frantic emphasis; and, in fine, gave the usual
tokens of her condition.  She wasn't a heroine. Her letters were full
of repetition.  She wrote rather doubtful grammar sometimes, and in her
verses took all sorts of liberties with the metre.  But oh, mesdames,
if you are not allowed to touch the heart sometimes in spite of syntax,
and are not to be loved until you all know the difference between
trimeter and tetrameter, may all Poetry go to the deuce, and every
schoolmaster perish miserably!



\chapter{Sentimental and Otherwise}

I fear the gentleman to whom Miss Amelia's letters were addressed was
rather an obdurate critic.  Such a number of notes followed Lieutenant
Osborne about the country, that he became almost ashamed of the jokes
of his mess-room companions regarding them, and ordered his servant
never to deliver them except at his private apartment. He was seen
lighting his cigar with one, to the horror of Captain Dobbin, who, it
is my belief, would have given a bank-note for the document.

For some time George strove to keep the liaison a secret.  There was a
woman in the case, that he admitted. ``And not the first either,'' said
Ensign Spooney to Ensign Stubble.  ``That Osborne's a devil of a fellow.
There was a judge's daughter at Demerara went almost mad about him;
then there was that beautiful quadroon girl, Miss Pye, at St.\ %
Vincent's, you know; and since he's been home, they say he's a regular
Don Giovanni, by Jove.''

Stubble and Spooney thought that to be a ``regular Don Giovanni, by
Jove'' was one of the finest qualities a man could possess, and
Osborne's reputation was prodigious amongst the young men of the
regiment.  He was famous in field-sports, famous at a song, famous on
parade; free with his money, which was bountifully supplied by his
father.  His coats were better made than any man's in the regiment, and
he had more of them.  He was adored by the men.  He could drink more
than any officer of the whole mess, including old Heavytop, the
colonel.  He could spar better than Knuckles, the private (who would
have been a corporal but for his drunkenness, and who had been in the
prize-ring); and was the best batter and bowler, out and out, of the
regimental club. He rode his own horse, Greased Lightning, and won the
Garrison cup at Quebec races.  There were other people besides Amelia
who worshipped him.  Stubble and Spooney thought him a sort of Apollo;
Dobbin took him to be an Admirable Crichton; and Mrs.\ Major O'Dowd
acknowledged he was an elegant young fellow, and put her in mind of
Fitzjurld Fogarty, Lord Castlefogarty's second son.

Well, Stubble and Spooney and the rest indulged in most romantic
conjectures regarding this female correspondent of Osborne's---opining
that it was a Duchess in London who was in love with him---or that it
was a General's daughter, who was engaged to somebody else, and madly
attached to him---or that it was a Member of Parliament's lady, who
proposed four horses and an elopement---or that it was some other victim
of a passion delightfully exciting, romantic, and disgraceful to all
parties, on none of which conjectures would Osborne throw the least
light, leaving his young admirers and friends to invent and arrange
their whole history.

And the real state of the case would never have been known at all in
the regiment but for Captain Dobbin's indiscretion.  The Captain was
eating his breakfast one day in the mess-room, while Cackle, the
assistant-surgeon, and the two above-named worthies were speculating
upon Osborne's intrigue---Stubble holding out that the lady was a
Duchess about Queen Charlotte's court, and Cackle vowing she was an
opera-singer of the worst reputation. At this idea Dobbin became so
moved, that though his mouth was full of eggs and bread-and-butter at
the time, and though he ought not to have spoken at all, yet he
couldn't help blurting out, ``Cackle, you're a stupid fool. You're
always talking nonsense and scandal.  Osborne is not going to run off
with a Duchess or ruin a milliner. Miss Sedley is one of the most
charming young women that ever lived.  He's been engaged to her ever so
long; and the man who calls her names had better not do so in my
hearing.'' With which, turning exceedingly red, Dobbin ceased speaking,
and almost choked himself with a cup of tea.  The story was over the
regiment in half-an-hour; and that very evening Mrs.\ Major O'Dowd wrote
off to her sister Glorvina at O'Dowdstown not to hurry from
Dublin---young Osborne being prematurely engaged already.

She complimented the Lieutenant in an appropriate speech over a glass
of whisky-toddy that evening, and he went home perfectly furious to
quarrel with Dobbin (who had declined Mrs.\ Major O'Dowd's party, and
sat in his own room playing the flute, and, I believe, writing poetry
in a very melancholy manner)---to quarrel with Dobbin for betraying his
secret.

``Who the deuce asked you to talk about my affairs?'' Osborne shouted
indignantly.  ``Why the devil is all the regiment to know that I am
going to be married? Why is that tattling old harridan, Peggy O'Dowd,
to make free with my name at her d---d supper-table, and advertise my
engagement over the three kingdoms? After all, what right have you to
say I am engaged, or to meddle in my business at all, Dobbin?''

``It seems to me,'' Captain Dobbin began.

``Seems be hanged, Dobbin,'' his junior interrupted him.  ``I am under
obligations to you, I know it, a d---d deal too well too; but I won't be
always sermonised by you because you're five years my senior. I'm
hanged if I'll stand your airs of superiority and infernal pity and
patronage.  Pity and patronage! I should like to know in what I'm your
inferior?''

``Are you engaged?'' Captain Dobbin interposed.

``What the devil's that to you or any one here if I am?''

``Are you ashamed of it?'' Dobbin resumed.

``What right have you to ask me that question, sir? I should like to
know,'' George said.

``Good God, you don't mean to say you want to break off?'' asked Dobbin,
starting up.

``In other words, you ask me if I'm a man of honour,'' said Osborne,
fiercely; ``is that what you mean? You've adopted such a tone regarding
me lately that I'm --------- if I'll bear it any more.''

``What have I done? I've told you you were neglecting a sweet girl,
George.  I've told you that when you go to town you ought to go to her,
and not to the gambling-houses about St.\ James's.''

``You want your money back, I suppose,'' said George, with a sneer.

``Of course I do---I always did, didn't I?'' says Dobbin. ``You speak like
a generous fellow.''

``No, hang it, William, I beg your pardon''---here George interposed in a
fit of remorse; ``you have been my friend in a hundred ways, Heaven
knows.  You've got me out of a score of scrapes.  When Crawley of the
Guards won that sum of money of me I should have been done but for you:
I know I should.  But you shouldn't deal so hardly with me; you
shouldn't be always catechising me. I am very fond of Amelia; I adore
her, and that sort of thing.  Don't look angry.  She's faultless; I
know she is. But you see there's no fun in winning a thing unless you
play for it.  Hang it: the regiment's just back from the West Indies, I
must have a little fling, and then when I'm married I'll reform; I will
upon my honour, now.  And---I say---Dob---don't be angry with me, and
I'll give you a hundred next month, when I know my father will stand
something handsome; and I'll ask Heavytop for leave, and I'll go to
town, and see Amelia to-morrow---there now, will that satisfy you?''

``It is impossible to be long angry with you, George,'' said the
good-natured Captain; ``and as for the money, old boy, you know if I wanted
it you'd share your last shilling with me.''

``That I would, by Jove, Dobbin,'' George said, with the greatest
generosity, though by the way he never had any money to spare.

``Only I wish you had sown those wild oats of yours, George.  If you
could have seen poor little Miss Emmy's face when she asked me about
you the other day, you would have pitched those billiard-balls to the
deuce.  Go and comfort her, you rascal.  Go and write her a long
letter.  Do something to make her happy; a very little will.''

``I believe she's d---d fond of me,'' the Lieutenant said, with a
self-satisfied air; and went off to finish the evening with some jolly
fellows in the mess-room.

Amelia meanwhile, in Russell Square, was looking at the moon, which was
shining upon that peaceful spot, as well as upon the square of the
Chatham barracks, where Lieutenant Osborne was quartered, and thinking
to herself how her hero was employed.  Perhaps he is visiting the
sentries, thought she; perhaps he is bivouacking; perhaps he is
attending the couch of a wounded comrade, or studying the art of war up
in his own desolate chamber. And her kind thoughts sped away as if they
were angels and had wings, and flying down the river to Chatham and
Rochester, strove to peep into the barracks where George was. . . . All
things considered, I think it was as well the gates were shut, and the
sentry allowed no one to pass; so that the poor little white-robed
angel could not hear the songs those young fellows were roaring over
the whisky-punch.

The day after the little conversation at Chatham barracks, young
Osborne, to show that he would be as good as his word, prepared to go
to town, thereby incurring Captain Dobbin's applause.  ``I should have
liked to make her a little present,'' Osborne said to his friend in
confidence, ``only I am quite out of cash until my father tips up.'' But
Dobbin would not allow this good nature and generosity to be balked,
and so accommodated Mr.\ Osborne with a few pound notes, which the
latter took after a little faint scruple.

And I dare say he would have bought something very handsome for Amelia;
only, getting off the coach in Fleet Street, he was attracted by a
handsome shirt-pin in a jeweller's window, which he could not resist;
and having paid for that, had very little money to spare for indulging
in any further exercise of kindness.  Never mind: you may be sure it
was not his presents Amelia wanted.  When he came to Russell Square,
her face lighted up as if he had been sunshine.  The little cares,
fears, tears, timid misgivings, sleepless fancies of I don't know how
many days and nights, were forgotten, under one moment's influence of
that familiar, irresistible smile.  He beamed on her from the
drawing-room door---magnificent, with ambrosial whiskers, like a god.
Sambo, whose face as he announced Captain Osbin (having conferred a
brevet rank on that young officer) blazed with a sympathetic grin, saw
the little girl start, and flush, and jump up from her watching-place
in the window; and Sambo retreated: and as soon as the door was shut,
she went fluttering to Lieutenant George Osborne's heart as if it was
the only natural home for her to nestle in.  Oh, thou poor panting
little soul!  The very finest tree in the whole forest, with the
straightest stem, and the strongest arms, and the thickest foliage,
wherein you choose to build and coo, may be marked, for what you know,
and may be down with a crash ere long.  What an old, old simile that
is, between man and timber!

In the meanwhile, George kissed her very kindly on her forehead and
glistening eyes, and was very gracious and good; and she thought his
diamond shirt-pin (which she had not known him to wear before) the
prettiest ornament ever seen.

The observant reader, who has marked our young Lieutenant's previous
behaviour, and has preserved our report of the brief conversation which
he has just had with Captain Dobbin, has possibly come to certain
conclusions regarding the character of Mr.\ Osborne.  Some cynical
Frenchman has said that there are two parties to a love-transaction:
the one who loves and the other who condescends to be so treated.
Perhaps the love is occasionally on the man's side; perhaps on the
lady's. Perhaps some infatuated swain has ere this mistaken
insensibility for modesty, dulness for maiden reserve, mere vacuity for
sweet bashfulness, and a goose, in a word, for a swan. Perhaps some
beloved female subscriber has arrayed an ass in the splendour and glory
of her imagination; admired his dulness as manly simplicity; worshipped
his selfishness as manly superiority; treated his stupidity as majestic
gravity, and used him as the brilliant fairy Titania did a certain
weaver at Athens.  I think I have seen such comedies of errors going on
in the world.  But this is certain, that Amelia believed her lover to
be one of the most gallant and brilliant men in the empire: and it is
possible Lieutenant Osborne thought so too.

He was a little wild: how many young men are; and don't girls like a
rake better than a milksop?  He hadn't sown his wild oats as yet, but
he would soon: and quit the army now that peace was proclaimed; the
Corsican monster locked up at Elba; promotion by consequence over; and
no chance left for the display of his undoubted military talents and
valour: and his allowance, with Amelia's settlement, would enable them
to take a snug place in the country somewhere, in a good sporting
neighbourhood; and he would hunt a little, and farm a little; and they
would be very happy.  As for remaining in the army as a married man,
that was impossible. Fancy Mrs.\ George Osborne in lodgings in a county
town; or, worse still, in the East or West Indies, with a society of
officers, and patronized by Mrs.\ Major O'Dowd! Amelia died with
laughing at Osborne's stories about Mrs.\ Major O'Dowd.  He loved her
much too fondly to subject her to that horrid woman and her
vulgarities, and the rough treatment of a soldier's wife.  He didn't
care for himself---not he; but his dear little girl should take the
place in society to which, as his wife, she was entitled: and to these
proposals you may be sure she acceded, as she would to any other from
the same author.

Holding this kind of conversation, and building numberless castles in
the air (which Amelia adorned with all sorts of flower-gardens, rustic
walks, country churches, Sunday schools, and the like; while George had
his mind's eye directed to the stables, the kennel, and the cellar),
this young pair passed away a couple of hours very pleasantly; and as
the Lieutenant had only that single day in town, and a great deal of
most important business to transact, it was proposed that Miss Emmy
should dine with her future sisters-in-law. This invitation was
accepted joyfully.  He conducted her to his sisters; where he left her
talking and prattling in a way that astonished those ladies, who
thought that George might make something of her; and he then went off
to transact his business.

In a word, he went out and ate ices at a pastry-cook's shop in Charing
Cross; tried a new coat in Pall Mall; dropped in at the Old
Slaughters', and called for Captain Cannon; played eleven games at
billiards with the Captain, of which he won eight, and returned to
Russell Square half an hour late for dinner, but in very good humour.

It was not so with old Mr.\ Osborne.  When that gentleman came from the
City, and was welcomed in the drawing-room by his daughters and the
elegant Miss Wirt, they saw at once by his face---which was puffy,
solemn, and yellow at the best of times---and by the scowl and twitching
of his black eyebrows, that the heart within his large white waistcoat
was disturbed and uneasy.  When Amelia stepped forward to salute him,
which she always did with great trembling and timidity, he gave a surly
grunt of recognition, and dropped the little hand out of his great
hirsute paw without any attempt to hold it there.  He looked round
gloomily at his eldest daughter; who, comprehending the meaning of his
look, which asked unmistakably, ``Why the devil is she here?'' said at
once:

``George is in town, Papa; and has gone to the Horse Guards, and will be
back to dinner.''

``O he is, is he? I won't have the dinner kept waiting for him, Jane'';
with which this worthy man lapsed into his particular chair, and then
the utter silence in his genteel, well-furnished drawing-room was only
interrupted by the alarmed ticking of the great French clock.

When that chronometer, which was surmounted by a cheerful brass group
of the sacrifice of Iphigenia, tolled five in a heavy cathedral tone,
Mr.\ Osborne pulled the bell at his right hand---violently, and the
butler rushed up.

``Dinner!'' roared Mr.\ Osborne.

``Mr.\ George isn't come in, sir,'' interposed the man.

``Damn Mr.\ George, sir.  Am I master of the house? DINNER!'' Mr.\ Osborne
scowled.  Amelia trembled.  A telegraphic communication of eyes passed
between the other three ladies.  The obedient bell in the lower regions
began ringing the announcement of the meal.  The tolling over, the head
of the family thrust his hands into the great tail-pockets of his great
blue coat with brass buttons, and without waiting for a further
announcement strode downstairs alone, scowling over his shoulder at the
four females.

``What's the matter now, my dear?'' asked one of the other, as they rose
and tripped gingerly behind the sire.    ``I suppose the funds are
falling,'' whispered Miss Wirt; and so, trembling and in silence, this
hushed female company followed their dark leader.  They took their
places in silence.  He growled out a blessing, which sounded as gruffly
as a curse.  The great silver dish-covers were removed. Amelia trembled
in her place, for she was next to the awful Osborne, and alone on her
side of the table---the gap being occasioned by the absence of George.

``Soup?'' says Mr.\ Osborne, clutching the ladle, fixing his eyes on her,
in a sepulchral tone; and having helped her and the rest, did not speak
for a while.

``Take Miss Sedley's plate away,'' at last he said.  ``She can't eat the
soup---no more can I.  It's beastly.  Take away the soup, Hicks, and
to-morrow turn the cook out of the house, Jane.''

Having concluded his observations upon the soup, Mr.\ Osborne made a few
curt remarks respecting the fish, also of a savage and satirical
tendency, and cursed Billingsgate with an emphasis quite worthy of the
place. Then he lapsed into silence, and swallowed sundry glasses of
wine, looking more and more terrible, till a brisk knock at the door
told of George's arrival when everybody began to rally.

``He could not come before.  General Daguilet had kept him waiting at
the Horse Guards.  Never mind soup or fish.  Give him anything---he
didn't care what.  Capital mutton---capital everything.'' His good humour
contrasted with his father's severity; and he rattled on unceasingly
during dinner, to the delight of all---of one especially, who need not
be mentioned.

As soon as the young ladies had discussed the orange and the glass of
wine which formed the ordinary conclusion of the dismal banquets at Mr.\ %
Osborne's house, the signal to make sail for the drawing-room was
given, and they all arose and departed.  Amelia hoped George would soon
join them there.  She began playing some of his favourite waltzes (then
newly imported) at the great carved-legged, leather-cased grand piano
in the drawing-room overhead.  This little artifice did not bring him.
He was deaf to the waltzes; they grew fainter and fainter; the
discomfited performer left the huge instrument presently; and though
her three friends performed some of the loudest and most brilliant new
pieces of their repertoire, she did not hear a single note, but sate
thinking, and boding evil.  Old Osborne's scowl, terrific always, had
never before looked so deadly to her.  His eyes followed her out of the
room, as if she had been guilty of something. When they brought her
coffee, she started as though it were a cup of poison which Mr.\ Hicks,
the butler, wished to propose to her.  What mystery was there lurking?
Oh, those women! They nurse and cuddle their presentiments, and make
darlings of their ugliest thoughts, as they do of their deformed
children.

The gloom on the paternal countenance had also impressed George Osborne
with anxiety.  With such eyebrows, and a look so decidedly bilious, how
was he to extract that money from the governor, of which George was
consumedly in want? He began praising his father's wine.  That was
generally a successful means of cajoling the old gentleman.

``We never got such Madeira in the West Indies, sir, as yours. Colonel
Heavytop took off three bottles of that you sent me down, under his
belt the other day.''

``Did he?'' said the old gentleman.  ``It stands me in eight shillings a
bottle.''

``Will you take six guineas a dozen for it, sir?'' said George, with a
laugh.  ``There's one of the greatest men in the kingdom wants some.''

``Does he?'' growled the senior.  ``Wish he may get it.''

``When General Daguilet was at Chatham, sir, Heavytop gave him a
breakfast, and asked me for some of the wine.  The General liked it
just as well---wanted a pipe for the Commander-in-Chief.  He's his Royal
Highness's right-hand man.''

``It is devilish fine wine,'' said the Eyebrows, and they looked more
good-humoured; and George was going to take advantage of this
complacency, and bring the supply question on the mahogany, when the
father, relapsing into solemnity, though rather cordial in manner, bade
him ring the bell for claret.  ``And we'll see if that's as good as the
Madeira, George, to which his Royal Highness is welcome, I'm sure.  And
as we are drinking it, I'll talk to you about a matter of importance.''

Amelia heard the claret bell ringing as she sat nervously upstairs. She
thought, somehow, it was a mysterious and presentimental bell. Of the
presentiments which some people are always having, some surely must
come right.

``What I want to know, George,'' the old gentleman said, after slowly
smacking his first bumper---``what I want to know is, how you
and---ah---that little thing upstairs, are carrying on?''

``I think, sir, it is not hard to see,'' George said, with a
self-satisfied grin.  ``Pretty clear, sir.---What capital wine!''

``What d'you mean, pretty clear, sir?''

``Why, hang it, sir, don't push me too hard.  I'm a modest man.
I---ah---I don't set up to be a lady-killer; but I do own that she's as
devilish fond of me as she can be.  Anybody can see that with half an
eye.''

``And you yourself?''

``Why, sir, didn't you order me to marry her, and ain't I a good boy?
Haven't our Papas settled it ever so long?''

``A pretty boy, indeed.  Haven't I heard of your doings, sir, with Lord
Tarquin, Captain Crawley of the Guards, the Honourable Mr.\ Deuceace and
that set.  Have a care sir, have a care.''

The old gentleman pronounced these aristocratic names with the greatest
gusto.  Whenever he met a great man he grovelled before him, and
my-lorded him as only a free-born Briton can do.  He came home and
looked out his history in the Peerage: he introduced his name into his
daily conversation; he bragged about his Lordship to his daughters.  He
fell down prostrate and basked in him as a Neapolitan beggar does in
the sun.  George was alarmed when he heard the names.  He feared his
father might have been informed of certain transactions at play.  But
the old moralist eased him by saying serenely:

``Well, well, young men will be young men.  And the comfort to me is,
George, that living in the best society in England, as I hope you do;
as I think you do; as my means will allow you to do---''

``Thank you, sir,'' says George, making his point at once.  ``One can't
live with these great folks for nothing; and my purse, sir, look at
it''; and he held up a little token which had been netted by Amelia, and
contained the very last of Dobbin's pound notes.

``You shan't want, sir.  The British merchant's son shan't want, sir. My
guineas are as good as theirs, George, my boy; and I don't grudge 'em.
Call on Mr.\ Chopper as you go through the City to-morrow; he'll have
something for you.  I don't grudge money when I know you're in good
society, because I know that good society can never go wrong.  There's
no pride in me.  I was a humbly born man---but you have had advantages.
Make a good use of 'em.  Mix with the young nobility. There's many of
'em who can't spend a dollar to your guinea, my boy.  And as for the
pink bonnets (here from under the heavy eyebrows there came a knowing
and not very pleasing leer)---why boys will be boys.  Only there's one
thing I order you to avoid, which, if you do not, I'll cut you off with
a shilling, by Jove; and that's gambling.''

``Oh, of course, sir,'' said George.

``But to return to the other business about Amelia: why shouldn't you
marry higher than a stockbroker's daughter, George---that's what I want
to know?''

``It's a family business, sir,'' says George, cracking filberts.  ``You
and Mr.\ Sedley made the match a hundred years ago.''

``I don't deny it; but people's positions alter, sir.  I don't deny that
Sedley made my fortune, or rather put me in the way of acquiring, by my
own talents and genius, that proud position, which, I may say, I occupy
in the tallow trade and the City of London. I've shown my gratitude to
Sedley; and he's tried it of late, sir, as my cheque-book can show.
George!  I tell you in confidence I don't like the looks of Mr.\ %
Sedley's affairs.  My chief clerk, Mr.\ Chopper, does not like the looks
of 'em, and he's an old file, and knows 'Change as well as any man in
London.  Hulker \& Bullock are looking shy at him.  He's been dabbling
on his own account I fear. They say the Jeune Amelie was his, which was
taken by the Yankee privateer Molasses.  And that's flat---unless I see
Amelia's ten thousand down you don't marry her.  I'll have no lame
duck's daughter in my family.  Pass the wine, sir---or ring for coffee.''

With which Mr.\ Osborne spread out the evening paper, and George knew
from this signal that the colloquy was ended, and that his papa was
about to take a nap.

He hurried upstairs to Amelia in the highest spirits. What was it that
made him more attentive to her on that night than he had been for a
long time---more eager to amuse her, more tender, more brilliant in
talk?  Was it that his generous heart warmed to her at the prospect of
misfortune; or that the idea of losing the dear little prize made him
value it more?

She lived upon the recollections of that happy evening for many days
afterwards, remembering his words; his looks; the song he sang; his
attitude, as he leant over her or looked at her from a distance.  As it
seemed to her, no night ever passed so quickly at Mr.\ Osborne's house
before; and for once this young person was almost provoked to be angry
by the premature arrival of Mr.\ Sambo with her shawl.

George came and took a tender leave of her the next morning; and then
hurried off to the City, where he visited Mr.\ Chopper, his father's
head man, and received from that gentleman a document which he
exchanged at Hulker \& Bullock's for a whole pocketful of money. As
George entered the house, old John Sedley was passing out of the
banker's parlour, looking very dismal.  But his godson was much too
elated to mark the worthy stockbroker's depression, or the dreary eyes
which the kind old gentleman cast upon him.  Young Bullock did not come
grinning out of the parlour with him as had been his wont in former
years.

And as the swinging doors of Hulker, Bullock \& Co. closed upon Mr.\ %
Sedley, Mr.\ Quill, the cashier (whose benevolent occupation it is to
hand out crisp bank-notes from a drawer and dispense sovereigns out of
a copper shovel), winked at Mr.\ Driver, the clerk at the desk on his
right.  Mr.\ Driver winked again.

``No go,'' Mr.\ D. whispered.

``Not at no price,'' Mr.\ Q. said.  ``Mr.\ George Osborne,  sir, how will
you take it?'' George crammed eagerly a quantity of notes into his
pockets, and paid Dobbin fifty pounds that very evening at mess.

That very evening Amelia wrote him the tenderest of long letters. Her
heart was overflowing with tenderness, but it still foreboded evil.
What was the cause of Mr.\ Osborne's dark looks? she asked. Had any
difference arisen between him and her papa? Her poor papa returned so
melancholy from the City, that all were alarmed about him at home---in
fine, there were four pages of loves and fears and hopes and
forebodings.

``Poor little Emmy---dear little Emmy.  How fond she is of me,'' George
said, as he perused the missive---``and Gad, what a headache that mixed
punch has given me!'' Poor little Emmy, indeed.



\chapter{Miss Crawley at Home}

About this time there drove up to an exceedingly snug and
well-appointed house in Park Lane, a travelling chariot with a lozenge on
the panels, a discontented female in a green veil and crimped curls on
the rumble, and a large and confidential man on the box.  It was the
equipage of our friend Miss Crawley, returning from Hants.  The
carriage windows were shut; the fat spaniel, whose head and tongue
ordinarily lolled out of one of them, reposed on the lap of the
discontented female.  When the vehicle stopped, a large round bundle of
shawls was taken out of the carriage by the aid of various domestics
and a young lady who accompanied the heap of cloaks.  That bundle
contained Miss Crawley, who was conveyed upstairs forthwith, and put
into a bed and chamber warmed properly as for the reception of an
invalid.  Messengers went off for her physician and medical man.  They
came, consulted, prescribed, vanished.  The young companion of Miss
Crawley, at the conclusion of their interview, came in to receive their
instructions, and administered those antiphlogistic medicines which the
eminent men ordered.

Captain Crawley of the Life Guards rode up from Knightsbridge Barracks
the next day; his black charger pawed the straw before his invalid
aunt's door.  He was most affectionate in his inquiries regarding that
amiable relative.  There seemed to be much source of apprehension. He
found Miss Crawley's maid (the discontented female) unusually sulky and
despondent; he found Miss Briggs, her dame de compagnie, in tears alone
in the drawing-room.  She had hastened home, hearing of her beloved
friend's illness.  She wished to fly to her couch, that couch which
she, Briggs, had so often smoothed in the hour of sickness.  She was
denied admission to Miss Crawley's apartment.  A stranger was
administering her medicines---a stranger from the country---an odious
Miss ... ---tears choked the utterance of the dame de compagnie, and
she buried her crushed affections and her poor old red nose in her
pocket handkerchief.

Rawdon Crawley sent up his name by the sulky femme de chambre, and Miss
Crawley's new companion, coming tripping down from the sick-room, put
a little hand into his as he stepped forward eagerly to meet her, gave
a glance of great scorn at the bewildered Briggs, and beckoning the
young Guardsman out of the back drawing-room, led him downstairs into
that now desolate dining-parlour, where so many a good dinner had been
celebrated.

Here these two talked for ten minutes, discussing, no doubt, the
symptoms of the old invalid above stairs; at the end of which period
the parlour bell was rung briskly, and answered on that instant by Mr.\ %
Bowls, Miss Crawley's large confidential butler (who, indeed, happened
to be at the keyhole during the most part of the interview); and the
Captain coming out, curling his mustachios, mounted the black charger
pawing among the straw, to the admiration of the little blackguard boys
collected in the street.  He looked in at the dining-room window,
managing his horse, which curvetted and capered beautifully---for one
instant the young person might be seen at the window, when her figure
vanished, and, doubtless, she went upstairs again to resume the
affecting duties of benevolence.

Who could this young woman be, I wonder?  That evening a little dinner
for two persons was laid in the dining-room---when Mrs.\ Firkin, the
lady's maid, pushed into her mistress's apartment, and bustled about
there during the vacancy occasioned by the departure of the new
nurse---and the latter and Miss Briggs sat down to the neat little meal.

Briggs was so much choked by emotion that she could hardly take a
morsel of meat.  The young person carved a fowl with the utmost
delicacy, and asked so distinctly for egg-sauce, that poor Briggs,
before whom that delicious condiment was placed, started, made a great
clattering with the ladle, and once more fell back in the most gushing
hysterical state.

``Had you not better give Miss Briggs a glass of wine?'' said the person
to Mr.\ Bowls, the large confidential man. He did so.  Briggs seized it
mechanically, gasped it down convulsively, moaned a little, and began
to play with the chicken on her plate.

``I think we shall be able to help each other,'' said the person with
great suavity: ``and shall have no need of Mr.\ Bowls's kind services.
Mr.\ Bowls, if you please, we will ring when we want you.'' He went
downstairs, where, by the way, he vented the most horrid curses upon
the unoffending footman, his subordinate.

``It is a pity you take on so, Miss Briggs,'' the young lady said, with a
cool, slightly sarcastic, air.

``My dearest friend is so ill, and wo-o-on't see me,'' gurgled out Briggs
in an agony of renewed grief.

``She's not very ill any more.  Console yourself, dear Miss Briggs. She
has only overeaten herself---that is all. She is greatly better. She
will soon be quite restored again. She is weak from being cupped and
from medical treatment, but she will rally immediately.  Pray console
yourself, and take a little more wine.''

``But why, why won't she see me again?'' Miss Briggs bleated out. ``Oh,
Matilda, Matilda, after three-and-twenty years' tenderness! is this the
return to your poor, poor Arabella?''

``Don't cry too much, poor Arabella,'' the other said (with ever so
little of a grin); ``she only won't see you, because she says you don't
nurse her as well as I do. It's no pleasure to me to sit up all night.
I wish you might do it instead.''

``Have I not tended that dear couch for years?'' Arabella said, ``and
now---''

``Now she prefers somebody else.  Well, sick people have these fancies,
and must be humoured.  When she's well I shall go.''

``Never, never,'' Arabella exclaimed, madly inhaling her salts-bottle.

``Never be well or never go, Miss Briggs?'' the other said, with the same
provoking good-nature.  ``Pooh---she will be well in a fortnight, when I
shall go back to my little pupils at Queen's Crawley, and to their
mother, who is a great deal more sick than our friend.  You need not be
jealous about me, my dear Miss Briggs.  I am a poor little girl without
any friends, or any harm in me. I don't want to supplant you in Miss
Crawley's good graces.  She will forget me a week after I am gone: and
her affection for you has been the work of years.  Give me a little
wine if you please, my dear Miss Briggs, and let us be friends.  I'm
sure I want friends.''

The placable and soft-hearted Briggs speechlessly pushed out her hand
at this appeal; but she felt the desertion most keenly for all that,
and bitterly, bitterly moaned the fickleness of her Matilda. At the end
of half an hour, the meal over, Miss Rebecca Sharp (for such,
astonishing to state, is the name of her who has been described
ingeniously as ``the person'' hitherto), went upstairs again to her
patient's rooms, from which, with the most engaging politeness, she
eliminated poor Firkin. ``Thank you, Mrs.\ Firkin, that will quite do;
how nicely you make it! I will ring when anything is wanted.'' ``Thank
you''; and Firkin came downstairs in a tempest of jealousy, only the
more dangerous because she was forced to confine it in her own bosom.

Could it be the tempest which, as she passed the landing of the first
floor, blew open the drawing-room door? No; it was stealthily opened by
the hand of Briggs. Briggs had been on the watch. Briggs too well heard
the creaking Firkin descend the stairs, and the clink of the spoon and
gruel-basin the neglected female carried.

``Well, Firkin?'' says she, as the other entered the apartment. ``Well,
Jane?''

``Wuss and wuss, Miss B.,'' Firkin said, wagging her head.

``Is she not better then?''

``She never spoke but once, and I asked her if she felt a little more
easy, and she told me to hold my stupid tongue. Oh, Miss B., I never
thought to have seen this day!''  And the water-works again began to
play.

``What sort of a person is this Miss Sharp, Firkin? I little thought,
while enjoying my Christmas revels in the elegant home of my firm
friends, the Reverend Lionel Delamere and his amiable lady, to find a
stranger had taken my place in the affections of my dearest, my still
dearest Matilda!''  Miss Briggs, it will be seen by her language, was of
a literary and sentimental turn, and had once published a volume of
poems---``Trills of the Nightingale''---by subscription.

``Miss B., they are all infatyated about that young woman,'' Firkin
replied. ``Sir Pitt wouldn't have let her go, but he daredn't refuse
Miss Crawley anything. Mrs.\ Bute at the Rectory jist as bad---never
happy out of her sight. The Capting quite wild about her. Mr.\ Crawley
mortial jealous. Since Miss C. was took ill, she won't have nobody near
her but Miss Sharp, I can't tell for where nor for why; and I think
somethink has bewidged everybody.''

Rebecca passed that night in constant watching upon Miss Crawley; the
next night the old lady slept so comfortably, that Rebecca had time for
several hours' comfortable repose herself on the sofa, at the foot of
her patroness's bed; very soon, Miss Crawley was so well that she sat
up and laughed heartily at a perfect imitation of Miss Briggs and her
grief, which Rebecca described to her. Briggs' weeping snuffle, and her
manner of using the handkerchief, were so completely rendered that Miss
Crawley became quite cheerful, to the admiration of the doctors when
they visited her, who usually found this worthy woman of the world,
when the least sickness attacked her, under the most abject depression
and terror of death.

Captain Crawley came every day, and received bulletins from Miss
Rebecca respecting his aunt's health. This improved so rapidly, that
poor Briggs was allowed to see her patroness; and persons with tender
hearts may imagine the smothered emotions of that sentimental female,
and the affecting nature of the interview.

Miss Crawley liked to have Briggs in a good deal soon.  Rebecca used to
mimic her to her face with the most admirable gravity, thereby
rendering the imitation doubly piquant to her worthy patroness.

The causes which had led to the deplorable illness of Miss Crawley, and
her departure from her brother's house in the country, were of such an
unromantic nature that they are hardly fit to be explained in this
genteel and sentimental novel.  For how is it possible to hint of a
delicate female, living in good society, that she ate and drank too
much, and that a hot supper of lobsters profusely enjoyed at the
Rectory was the reason of an indisposition which Miss Crawley herself
persisted was solely attributable to the dampness of the weather?  The
attack was so sharp that Matilda---as his Reverence expressed it---was
very nearly ``off the hooks''; all the family were in a fever of
expectation regarding the will, and Rawdon Crawley was making sure of
at least forty thousand pounds before the commencement of the London
season.  Mr.\ Crawley sent over a choice parcel of tracts, to prepare
her for the change from Vanity Fair and Park Lane for another world;
but a good doctor from Southampton being called in in time, vanquished
the lobster which was so nearly fatal to her, and gave her sufficient
strength to enable her to return to London. The Baronet did not
disguise his exceeding mortification at the turn which affairs took.

While everybody was attending on Miss Crawley, and messengers every
hour from the Rectory were carrying news of her health to the
affectionate folks there, there was a lady in another part of the
house, being exceedingly ill, of whom no one took any notice at all;
and this was the lady of Crawley herself.  The good doctor shook his
head after seeing her; to which visit Sir Pitt consented, as it could
be paid without a fee; and she was left fading away in her lonely
chamber, with no more heed paid to her than to a weed in the park.

The young ladies, too, lost much of the inestimable benefit of their
governess's instruction, So affectionate a nurse was Miss Sharp, that
Miss Crawley would take her medicines from no other hand. Firkin had
been deposed long before her mistress's departure from the country.
That faithful attendant found a gloomy consolation on returning to
London, in seeing Miss Briggs suffer the same pangs of jealousy and
undergo the same faithless treatment to which she herself had been
subject.

Captain Rawdon got an extension of leave on his aunt's illness, and
remained dutifully at home.  He was always in her antechamber.  (She
lay sick in the state bedroom, into which you entered by the little
blue saloon.) His father was always meeting him there; or if he came
down the corridor ever so quietly, his father's door was sure to open,
and the hyena face of the old gentleman to glare out.  What was it set
one to watch the other so?  A generous rivalry, no doubt, as to which
should be most attentive to the dear sufferer in the state bedroom.
Rebecca used to come out and comfort both of them; or one or the other
of them rather.  Both of these worthy gentlemen were most anxious to
have news of the invalid from her little confidential messenger.

At dinner---to which meal she descended for half an hour---she kept the
peace between them: after which she disappeared for the night; when
Rawdon would ride over to the depot of the 150th at Mudbury, leaving
his papa to the society of Mr.\ Horrocks and his rum and water. She
passed as weary a fortnight as ever mortal spent in Miss Crawley's
sick-room; but her little nerves seemed to be of iron, as she was quite
unshaken by the duty and the tedium of the sick-chamber.

She never told until long afterwards how painful that duty was; how
peevish a patient was the jovial old lady; how angry; how sleepless; in
what horrors of death; during what long nights she lay moaning, and in
almost delirious agonies respecting that future world which she quite
ignored when she was in good health.---Picture to yourself, oh fair
young reader, a worldly, selfish, graceless, thankless, religionless
old woman, writhing in pain and fear, and without her wig.  Picture her
to yourself, and ere you be old, learn to love and pray!

Sharp watched this graceless bedside with indomitable patience. Nothing
escaped her; and, like a prudent steward, she found a use for
everything.  She told many a good story about Miss Crawley's illness in
after days---stories which made the lady blush through her artificial
carnations.  During the illness she was never out of temper; always
alert; she slept light, having a perfectly clear conscience; and could
take that refreshment at almost any minute's warning.  And so you saw
very few traces of fatigue in her appearance.  Her face might be a
trifle paler, and the circles round her eyes a little blacker than
usual; but whenever she came out from the sick-room she was always
smiling, fresh, and neat, and looked as trim in her little
dressing-gown and cap, as in her smartest evening suit.

The Captain thought so, and raved about her in uncouth convulsions. The
barbed shaft of love had penetrated his dull hide.  Six
weeks---appropinquity---opportunity---had victimised him completely.  He
made a confidante of his aunt at the Rectory, of all persons in the
world.  She rallied him about it; she had perceived his folly; she
warned him; she finished by owning that little Sharp was the most clever,
droll, odd, good-natured, simple, kindly creature in England.  Rawdon
must not trifle with her affections, though---dear Miss Crawley would
never pardon him for that; for she, too, was quite overcome by the little
governess, and loved Sharp like a daughter.  Rawdon must go away---go
back to his regiment and naughty London, and not play with a poor
artless girl's feelings.

Many and many a time this good-natured lady, compassionating the
forlorn life-guardsman's condition, gave him an opportunity of seeing
Miss Sharp at the Rectory, and of walking home with her, as we have
seen.  When men of a certain sort, ladies, are in love, though they see
the hook and the string, and the whole apparatus with which they are to
be taken, they gorge the bait nevertheless---they must come to it---they
must swallow it---and are presently struck and landed gasping.  Rawdon
saw there was a manifest intention on Mrs.\ Bute's part to captivate him
with Rebecca.  He was not very wise; but he was a man about town, and
had seen several seasons.  A light dawned upon his dusky soul, as he
thought, through a speech of Mrs.\ Bute's.

``Mark my words, Rawdon,'' she said.  ``You will have Miss Sharp one day
for your relation.''

``What relation---my cousin, hey, Mrs.\ Bute? James sweet on her, hey?''
inquired the waggish officer.

``More than that,'' Mrs.\ Bute said, with a flash from her black eyes.

``Not Pitt?  He sha'n't have her.  The sneak a'n't worthy of her. He's
booked to Lady Jane Sheepshanks.''

``You men perceive nothing.  You silly, blind creature---if anything
happens to Lady Crawley, Miss Sharp will be your mother-in-law; and
that's what will happen.''

Rawdon Crawley, Esquire, gave vent to a prodigious whistle, in token of
astonishment at this announcement. He couldn't deny it.  His father's
evident liking for Miss Sharp had not escaped him.  He knew the old
gentleman's character well; and a more unscrupulous old---whyou---he did
not conclude the sentence, but walked home, curling his mustachios, and
convinced he had found a clue to Mrs.\ Bute's mystery.

``By Jove, it's too bad,'' thought Rawdon, ``too bad, by Jove! I do
believe the woman wants the poor girl to be ruined, in order that she
shouldn't come into the family as Lady Crawley.''

When he saw Rebecca alone, he rallied her about his father's attachment
in his graceful way.  She flung up her head scornfully, looked him full
in the face, and said,

``Well, suppose he is fond of me.  I know he is, and others too.  You
don't think I am afraid of him, Captain Crawley?  You don't suppose I
can't defend my own honour,'' said the little woman, looking as stately
as a queen.

``Oh, ah, why---give you fair warning---look out, you know---that's all,''
said the mustachio-twiddler.

``You hint at something not honourable, then?'' said she, flashing out.

``O Gad---really---Miss Rebecca,'' the heavy dragoon interposed.

``Do you suppose I have no feeling of self-respect, because I am poor
and friendless, and because rich people have none?  Do you think,
because I am a governess, I have not as much sense, and feeling, and
good breeding as you gentlefolks in Hampshire? I'm a Montmorency. Do
you suppose a Montmorency is not as good as a Crawley?''

When Miss Sharp was agitated, and alluded to her maternal relatives,
she spoke with ever so slight a foreign accent, which gave a great
charm to her clear ringing voice.  ``No,'' she continued, kindling as she
spoke to the Captain; ``I can endure poverty, but not shame---neglect,
but not insult; and insult from---from you.''

Her feelings gave way, and she burst into tears.

``Hang it, Miss Sharp---Rebecca---by Jove---upon my soul, I wouldn't for a
thousand pounds.  Stop, Rebecca!''

She was gone.  She drove out with Miss Crawley that day.  It was before
the latter's illness.  At dinner she was unusually brilliant and
lively; but she would take no notice of the hints, or the nods, or the
clumsy expostulations of the humiliated, infatuated guardsman.
Skirmishes of this sort passed perpetually during the little
campaign---tedious to relate, and similar in result.  The Crawley heavy
cavalry was maddened by defeat, and routed every day.

If the Baronet of Queen's Crawley had not had the fear of losing his
sister's legacy before his eyes, he never would have permitted his dear
girls to lose the educational blessings which their invaluable
governess was conferring upon them.  The old house at home seemed a
desert without her, so useful and pleasant had Rebecca made herself
there.  Sir Pitt's letters were not copied and corrected; his books not
made up; his household business and manifold schemes neglected, now
that his little secretary was away.  And it was easy to see how
necessary such an amanuensis was to him, by the tenor and spelling of
the numerous letters which he sent to her, entreating her and
commanding her to return.  Almost every day brought a frank from the
Baronet, enclosing the most urgent prayers to Becky for her return, or
conveying pathetic statements to Miss Crawley, regarding the neglected
state of his daughters' education; of which documents Miss Crawley took
very little heed.

Miss Briggs was not formally dismissed, but her place as companion was
a sinecure and a derision; and her company was the fat spaniel in the
drawing-room, or occasionally the discontented Firkin in the
housekeeper's closet.  Nor though the old lady would by no means hear
of Rebecca's departure, was the latter regularly installed in office in
Park Lane.  Like many wealthy people, it was Miss Crawley's habit to
accept as much service as she could get from her inferiors; and
good-naturedly to take leave of them when she no longer found them
useful.  Gratitude among certain rich folks is scarcely natural or to
be thought of.  They take needy people's services as their due.  Nor
have you, O poor parasite and humble hanger-on, much reason to
complain!  Your friendship for Dives is about as sincere as the return
which it usually gets.  It is money you love, and not the man; and were
Croesus and his footman to change places you know, you poor rogue, who
would have the benefit of your allegiance.

And I am not sure that, in spite of Rebecca's simplicity and activity,
and gentleness and untiring good humour, the shrewd old London lady,
upon whom these treasures of friendship were lavished, had not a
lurking suspicion all the while of her affectionate nurse and friend.
It must have often crossed Miss Crawley's mind that nobody does
anything for nothing.  If she measured her own feeling towards the
world, she must have been pretty well able to gauge those of the world
towards herself; and perhaps she reflected that it is the ordinary lot
of people to have no friends if they themselves care for nobody.

Well, meanwhile Becky was the greatest comfort and convenience to her,
and she gave her a couple of new gowns, and an old necklace and shawl,
and showed her friendship by abusing all her intimate acquaintances to
her new confidante (than which there can't be a more touching proof of
regard), and meditated vaguely some great future benefit---to marry her
perhaps to Clump, the apothecary, or to settle her in some advantageous
way of life; or at any rate, to send her back to Queen's Crawley when
she had done with her, and the full London season had begun.

When Miss Crawley was convalescent and descended to the drawing-room,
Becky sang to her, and otherwise amused her; when she was well enough
to drive out, Becky accompanied her.  And amongst the drives which they
took, whither, of all places in the world, did Miss Crawley's admirable
good-nature and friendship actually induce her to penetrate, but to
Russell Square, Bloomsbury, and the house of John Sedley, Esquire.

Ere that event, many notes had passed, as may be imagined, between the
two dear friends.  During the months of Rebecca's stay in Hampshire,
the eternal friendship had (must it be owned?) suffered considerable
diminution, and grown so decrepit and feeble with old age as to
threaten demise altogether.  The fact is, both girls had their own real
affairs to think of: Rebecca her advance with her employers---Amelia her
own absorbing topic.  When the two girls met, and flew into each
other's arms with that impetuosity which distinguishes the behaviour of
young ladies towards each other, Rebecca performed her part of the
embrace with the most perfect briskness and energy.  Poor little Amelia
blushed as she kissed her friend, and thought she had been guilty of
something very like coldness towards her.

Their first interview was but a very short one.  Amelia was just ready
to go out for a walk.  Miss Crawley was waiting in her carriage below,
her people wondering at the locality in which they found themselves,
and gazing upon honest Sambo, the black footman of Bloomsbury, as one
of the queer natives of the place.  But when Amelia came down with her
kind smiling looks (Rebecca must introduce her to her friend, Miss
Crawley was longing to see her, and was too ill to leave her
carriage)---when, I say, Amelia came down, the Park Lane shoulder-knot
aristocracy wondered more and more that such a thing could come out of
Bloomsbury; and Miss Crawley was fairly captivated by the sweet
blushing face of the young lady who came forward so timidly and so
gracefully to pay her respects to the protector of her friend.

``What a complexion, my dear! What a sweet voice!'' Miss Crawley said, as
they drove away westward after the little interview.  ``My dear Sharp,
your young friend is charming.  Send for her to Park Lane, do you
hear?'' Miss Crawley had a good taste.  She liked natural manners---a
little timidity only set them off.  She liked pretty faces near her; as
she liked pretty pictures and nice china.  She talked of Amelia with
rapture half a dozen times that day.  She mentioned her to Rawdon
Crawley, who came dutifully to partake of his aunt's chicken.

Of course, on this Rebecca instantly stated that Amelia was engaged to
be married---to a Lieutenant Osborne---a very old flame.

``Is he a man in a line-regiment?'' Captain Crawley asked, remembering
after an effort, as became a guardsman, the number of the regiment,
the ---th.

Rebecca thought that was the regiment.  ``The Captain's name,'' she said,
``was Captain Dobbin.''

``A lanky gawky fellow,'' said Crawley, ``tumbles over everybody.  I know
him; and Osborne's a goodish-looking fellow, with large black whiskers?''

``Enormous,'' Miss Rebecca Sharp said, ``and enormously proud of them, I
assure you.''

Captain Rawdon Crawley burst into a horse-laugh by way of reply; and
being pressed by the ladies to explain, did so when the explosion of
hilarity was over.  ``He fancies he can play at billiards,'' said he. ``I
won two hundred of him at the Cocoa-Tree.  HE play, the young flat!
He'd have played for anything that day, but his friend Captain Dobbin
carried him off, hang him!''

``Rawdon, Rawdon, don't be so wicked,'' Miss Crawley remarked, highly
pleased.

``Why, ma'am, of all the young fellows I've seen out of the line, I
think this fellow's the greenest.  Tarquin and Deuceace get what money
they like out of him.  He'd go to the deuce to be seen with a lord.  He
pays their dinners at Greenwich, and they invite the company.''

``And very pretty company too, I dare say.''

``Quite right, Miss Sharp.  Right, as usual, Miss Sharp. Uncommon pretty
company---haw, haw!'' and the Captain laughed more and more, thinking he
had made a good joke.

``Rawdon, don't be naughty!'' his aunt exclaimed.

``Well, his father's a City man---immensely rich, they say.  Hang those
City fellows, they must bleed; and I've not done with him yet, I can
tell you.  Haw, haw!''

``Fie, Captain Crawley; I shall warn Amelia.  A gambling husband!''

``Horrid, ain't he, hey?'' the Captain said with great solemnity; and
then added, a sudden thought having struck him: ``Gad, I say, ma'am,
we'll have him here.''

``Is he a presentable sort of a person?'' the aunt inquired.

``Presentable?---oh, very well.  You wouldn't see any difference,''
Captain Crawley answered.  ``Do let's have him, when you begin to see a
few people; and his whatdyecallem---his inamorato---eh, Miss Sharp;
that's what you call it---comes.  Gad, I'll write him a note, and have
him; and I'll try if he can play piquet as well as billiards. Where
does he live, Miss Sharp?''

Miss Sharp told Crawley the Lieutenant's town address; and a few days
after this conversation, Lieutenant Osborne received a letter, in
Captain Rawdon's schoolboy hand, and enclosing a note of invitation
from Miss Crawley.

Rebecca despatched also an invitation to her darling Amelia, who, you
may be sure, was ready enough to accept it when she heard that George
was to be of the party.  It was arranged that Amelia was to spend the
morning with the ladies of Park Lane, where all were very kind to her.
Rebecca patronised her with calm superiority: she was so much the
cleverer of the two, and her friend so gentle and unassuming, that she
always yielded when anybody chose to command, and so took Rebecca's
orders with perfect meekness and good humour. Miss Crawley's
graciousness was also remarkable.  She continued her raptures about
little Amelia, talked about her before her face as if she were a doll,
or a servant, or a picture, and admired her with the most benevolent
wonder possible.  I admire that admiration which the genteel world
sometimes extends to the commonalty. There is no more agreeable object
in life than to see Mayfair folks condescending.  Miss Crawley's
prodigious benevolence rather fatigued poor little Amelia, and I am not
sure that of the three ladies in Park Lane she did not find honest Miss
Briggs the most agreeable.  She sympathised with Briggs as with all
neglected or gentle people: she wasn't what you call a woman of spirit.

George came to dinner---a repast en garcon with Captain Crawley.

The great family coach of the Osbornes transported him to Park Lane
from Russell Square; where the young ladies, who were not themselves
invited, and professed the greatest indifference at that slight,
nevertheless looked at Sir Pitt Crawley's name in the baronetage; and
learned everything which that work had to teach about the Crawley
family and their pedigree, and the Binkies, their relatives, \&c., \&c.
Rawdon Crawley received George Osborne with great frankness and
graciousness: praised his play at billiards: asked him when he would
have his revenge: was interested about Osborne's regiment: and would
have proposed piquet to him that very evening, but Miss Crawley
absolutely forbade any gambling in her house; so that the young
Lieutenant's purse was not lightened by his gallant patron, for that
day at least.  However, they made an engagement for the next,
somewhere: to look at a horse that Crawley had to sell, and to try him
in the Park; and to dine together, and to pass the evening with some
jolly fellows.  ``That is, if you're not on duty to that pretty Miss
Sedley,'' Crawley said, with a knowing wink. ``Monstrous nice girl, 'pon
my honour, though, Osborne,'' he was good enough to add.  ``Lots of tin,
I suppose, eh?''

Osborne wasn't on duty; he would join Crawley with pleasure: and the
latter, when they met the next day, praised his new friend's
horsemanship---as he might with perfect honesty---and introduced him to
three or four young men of the first fashion, whose acquaintance
immensely elated the simple young officer.

``How's little Miss Sharp, by-the-bye?'' Osborne inquired of his friend
over their wine, with a dandified air. ``Good-natured little girl that.
Does she suit you well at Queen's Crawley? Miss Sedley liked her a good
deal last year.''

Captain Crawley looked savagely at the Lieutenant out of his little
blue eyes, and watched him when he went up to resume his acquaintance
with the fair governess.  Her conduct must have relieved Crawley if
there was any jealousy in the bosom of that life-guardsman.

When the young men went upstairs, and after Osborne's introduction to
Miss Crawley, he walked up to Rebecca with a patronising, easy swagger.
He was going to be kind to her and protect her.  He would even shake
hands with her, as a friend of Amelia's; and saying, ``Ah, Miss Sharp!
how-dy-doo?'' held out his left hand towards her, expecting that she
would be quite confounded at the honour.

Miss Sharp put out her right forefinger, and gave him a little nod, so
cool and killing, that Rawdon Crawley, watching the operations from the
other room, could hardly restrain his laughter as he saw the
Lieutenant's entire discomfiture; the start he gave, the pause, and the
perfect clumsiness with which he at length condescended to take the
finger which was offered for his embrace.

``She'd beat the devil, by Jove!'' the Captain said, in a rapture; and
the Lieutenant, by way of beginning the conversation, agreeably asked
Rebecca how she liked her new place.

``My place?'' said Miss Sharp, coolly, ``how kind of you to remind me of
it!  It's a tolerably good place: the wages are pretty good---not so
good as Miss Wirt's, I believe, with your sisters in Russell Square.
How are those young ladies?---not that I ought to ask.''

``Why not?'' Mr.\ Osborne said, amazed.

``Why, they never condescended to speak to me, or to ask me into their
house, whilst I was staying with Amelia; but we poor governesses, you
know, are used to slights of this sort.''

``My dear Miss Sharp!'' Osborne ejaculated.

``At least in some families,'' Rebecca continued.  ``You can't think what
a difference there is though.  We are not so wealthy in Hampshire as
you lucky folks of the City. But then I am in a gentleman's
family---good old English stock.  I suppose you know Sir Pitt's father
refused a peerage.  And you see how I am treated.  I am pretty
comfortable.  Indeed it is rather a good place.  But how very good of
you to inquire!''

Osborne was quite savage.  The little governess patronised him and
persiffled him until this young British Lion felt quite uneasy; nor
could he muster sufficient presence of mind to find a pretext for
backing out of this most delectable conversation.

``I thought you liked the City families pretty well,'' he said, haughtily.

``Last year you mean, when I was fresh from that horrid vulgar school?
Of course I did.  Doesn't every girl like to come home for the
holidays?  And how was I to know any better?  But oh, Mr.\ Osborne, what
a difference eighteen months' experience makes! eighteen months spent,
pardon me for saying so, with gentlemen.  As for dear Amelia, she, I
grant you, is a pearl, and would be charming anywhere.  There now, I
see you are beginning to be in a good humour; but oh these queer odd
City people! And Mr.\ Jos---how is that wonderful Mr.\ Joseph?''

``It seems to me you didn't dislike that wonderful Mr.\ Joseph last
year,'' Osborne said kindly.

``How severe of you!  Well, entre nous, I didn't break my heart about
him; yet if he had asked me to do what you mean by your looks (and very
expressive and kind they are, too), I wouldn't have said no.''

Mr.\ Osborne gave a look as much as to say, ``Indeed, how very obliging!''

``What an honour to have had you for a brother-in-law, you are thinking?
To be sister-in-law to George Osborne, Esquire, son of John Osborne,
Esquire, son of---what was your grandpapa, Mr.\ Osborne? Well, don't be
angry.  You can't help your pedigree, and I quite agree with you that I
would have married Mr.\ Joe Sedley; for could a poor penniless girl do
better?  Now you know the whole secret.  I'm frank and open;
considering all things, it was very kind of you to allude to the
circumstance---very kind and polite.  Amelia dear, Mr.\ Osborne and I
were talking about your poor brother Joseph. How is he?''

Thus was George utterly routed.  Not that Rebecca was in the right; but
she had managed most successfully to put him in the wrong.  And he now
shamefully fled, feeling, if he stayed another minute, that he would
have been made to look foolish in the presence of Amelia.

Though Rebecca had had the better of him, George was above the meanness
of talebearing or revenge upon a lady---only he could not help cleverly
confiding to Captain Crawley, next day, some notions of his regarding
Miss Rebecca---that she was a sharp one, a dangerous one, a desperate
flirt, \&c.; in all of which opinions Crawley agreed laughingly, and
with every one of which Miss Rebecca was made acquainted before
twenty-four hours were over.  They added to her original regard for Mr.\ %
Osborne.  Her woman's instinct had told her that it was George who had
interrupted the success of her first love-passage, and she esteemed him
accordingly.

``I only just warn you,'' he said to Rawdon Crawley, with a knowing
look---he had bought the horse, and lost some score of guineas after
dinner, ``I just warn you---I know women, and counsel you to be on the
look-out.''

``Thank you, my boy,'' said Crawley, with a look of peculiar gratitude.
``You're wide awake, I see.'' And George went off, thinking Crawley was
quite right.

He told Amelia of what he had done, and how he had counselled Rawdon
Crawley---a devilish good, straightforward fellow---to be on his guard
against that little sly, scheming Rebecca.

``Against whom?'' Amelia cried.

``Your friend the governess.---Don't look so astonished.''

``O George, what have you done?'' Amelia said.  For her woman's eyes,
which Love had made sharp-sighted, had in one instant discovered a
secret which was invisible to Miss Crawley, to poor virgin Briggs, and
above all, to the stupid peepers of that young whiskered prig,
Lieutenant Osborne.

For as Rebecca was shawling her in an upper apartment, where these two
friends had an opportunity for a little of that secret talking and
conspiring which form the delight of female life, Amelia, coming up to
Rebecca, and taking her two little hands in hers, said, ``Rebecca, I see
it all.''

Rebecca kissed her.

And regarding this delightful secret, not one syllable more was said by
either of the young women.  But it was destined to come out before long.

Some short period after the above events, and Miss Rebecca Sharp still
remaining at her patroness's house in Park Lane, one more hatchment
might have been seen in Great Gaunt Street, figuring amongst the many
which usually ornament that dismal quarter.  It was over Sir Pitt
Crawley's house; but it did not indicate the worthy baronet's demise.
It was a feminine hatchment, and indeed a few years back had served as
a funeral compliment to Sir Pitt's old mother, the late dowager Lady
Crawley. Its period of service over, the hatchment had come down from
the front of the house, and lived in retirement somewhere in the back
premises of Sir Pitt's mansion. It reappeared now for poor Rose Dawson.
Sir Pitt was a widower again.  The arms quartered on the shield along
with his own were not, to be sure, poor Rose's. She had no arms.  But
the cherubs painted on the scutcheon answered as well for her as for
Sir Pitt's mother, and Resurgam was written under the coat, flanked by
the Crawley Dove and Serpent.  Arms and Hatchments, Resurgam.---Here is
an opportunity for moralising!

Mr.\ Crawley had tended that otherwise friendless bedside.  She went out
of the world strengthened by such words and comfort as he could give
her.  For many years his was the only kindness she ever knew; the only
friendship that solaced in any way that feeble, lonely soul. Her heart
was dead long before her body.  She had sold it to become Sir Pitt
Crawley's wife.  Mothers and daughters are making the same bargain
every day in Vanity Fair.

When the demise took place, her husband was in London attending to some
of his innumerable schemes, and busy with his endless lawyers. He had
found time, nevertheless, to call often in Park Lane, and to despatch
many notes to Rebecca, entreating her, enjoining her, commanding her to
return to her young pupils in the country, who were now utterly without
companionship during their mother's illness.  But Miss Crawley would
not hear of her departure; for though there was no lady of fashion in
London who would desert her friends more complacently as soon as she
was tired of their society, and though few tired of them sooner, yet as
long as her engoument lasted her attachment was prodigious, and she
clung still with the greatest energy to Rebecca.

The news of Lady Crawley's death provoked no more grief or comment than
might have been expected in Miss Crawley's family circle.  ``I suppose I
must put off my party for the 3rd,'' Miss Crawley said; and added, after
a pause, ``I hope my brother will have the decency not to marry again.''
``What a confounded rage Pitt will be in if he does,'' Rawdon remarked,
with his usual regard for his elder brother. Rebecca said nothing.  She
seemed by far the gravest and most impressed of the family.  She left
the room before Rawdon went away that day; but they met by chance
below, as he was going away after taking leave, and had a parley
together.

On the morrow, as Rebecca was gazing from the window, she startled Miss
Crawley, who was placidly occupied with a French novel, by crying out
in an alarmed tone, ``Here's Sir Pitt, Ma'am!'' and the Baronet's knock
followed this announcement.

``My dear, I can't see him.  I won't see him.  Tell Bowls not at home,
or go downstairs and say I'm too ill to receive any one.  My nerves
really won't bear my brother at this moment,'' cried out Miss Crawley,
and resumed the novel.

``She's too ill to see you, sir,'' Rebecca said, tripping down to Sir
Pitt, who was preparing to ascend.

``So much the better,'' Sir Pitt answered.  ``I want to see YOU, Miss
Becky.  Come along a me into the parlour,'' and they entered that
apartment together.

``I wawnt you back at Queen's Crawley, Miss,'' the baronet said, fixing
his eyes upon her, and taking off his black gloves and his hat with its
great crape hat-band. His eyes had such a strange look, and fixed upon
her so steadfastly, that Rebecca Sharp began almost to tremble.

``I hope to come soon,'' she said in a low voice, ``as soon as Miss
Crawley is better---and return to---to the dear children.''

``You've said so these three months, Becky,'' replied Sir Pitt, ``and
still you go hanging on to my sister, who'll fling you off like an old
shoe, when she's wore you out. I tell you I want you.  I'm going back
to the Vuneral. Will you come back?  Yes or no?''

``I daren't---I don't think---it would be right---to be alone---with you,
sir,'' Becky said, seemingly in great agitation.

``I say agin, I want you,'' Sir Pitt said, thumping the table.  ``I can't
git on without you.  I didn't see what it was till you went away.  The
house all goes wrong.  It's not the same place.  All my accounts has
got muddled agin. You MUST come back.  Do come back. Dear Becky, do
come.''

``Come---as what, sir?'' Rebecca gasped out.

``Come as Lady Crawley, if you like,'' the Baronet said, grasping his
crape hat.  ``There! will that zatusfy you? Come back and be my wife.
Your vit vor't.  Birth be hanged.  You're as good a lady as ever I see.
You've got more brains in your little vinger than any baronet's wife in
the county.  Will you come? Yes or no?''

``Oh, Sir Pitt!'' Rebecca said, very much moved.

``Say yes, Becky,'' Sir Pitt continued.  ``I'm an old man, but a good'n.
I'm good for twenty years.  I'll make you happy, zee if I don't.  You
shall do what you like; spend what you like; and 'ave it all your own
way.  I'll make you a zettlement.  I'll do everything reglar.  Look
year!'' and the old man fell down on his knees and leered at her like a
satyr.

Rebecca started back a picture of consternation.  In the course of this
history we have never seen her lose her presence of mind; but she did
now, and wept some of the most genuine tears that ever fell from her
eyes.

``Oh, Sir Pitt!'' she said.  ``Oh, sir---I---I'm married ALREADY.''



\chapter{In Which Rebecca's Husband Appears for a Short Time}

Every reader of a sentimental turn (and we desire no other) must have
been pleased with the tableau with which the last act of our little
drama concluded; for what can be prettier than an image of Love on his
knees before Beauty?

But when Love heard that awful confession from Beauty that she was
married already, he bounced up from his attitude of humility on the
carpet, uttering exclamations which caused poor little Beauty to be
more frightened than she was when she made her avowal.  ``Married;
you're joking,'' the Baronet cried, after the first explosion of rage
and wonder.  ``You're making vun of me, Becky.  Who'd ever go to marry
you without a shilling to your vortune?''

``Married! married!'' Rebecca said, in an agony of tears---her voice
choking with emotion, her handkerchief up to her ready eyes, fainting
against the mantelpiece a figure of woe fit to melt the most obdurate
heart.  ``O Sir Pitt, dear Sir Pitt, do not think me ungrateful for all
your goodness to me.  It is only your generosity that has extorted my
secret.''

``Generosity be hanged!'' Sir Pitt roared out.  ``Who is it tu, then,
you're married? Where was it?''

``Let me come back with you to the country, sir!  Let me watch over you
as faithfully as ever!  Don't, don't separate me from dear Queen's
Crawley!''

``The feller has left you, has he?'' the Baronet said, beginning, as he
fancied, to comprehend.  ``Well, Becky---come back if you like. You can't
eat your cake and have it.  Any ways I made you a vair offer.  Coom
back as governess---you shall have it all your own way.'' She held out
one hand.  She cried fit to break her heart; her ringlets fell over her
face, and over the marble mantelpiece where she laid it.

``So the rascal ran off, eh?''  Sir Pitt said, with a hideous attempt at
consolation.  ``Never mind, Becky, I'LL take care of 'ee.''

``Oh, sir! it would be the pride of my life to go back to Queen's
Crawley, and take care of the children, and of you as formerly, when
you said you were pleased with the services of your little Rebecca.
When I think of what you have just offered me, my heart fills with
gratitude indeed it does.  I can't be your wife, sir; let me---let me be
your daughter.''  Saying which, Rebecca went down on HER knees in a most
tragical way, and, taking Sir Pitt's horny black hand between her own
two (which were very pretty and white, and as soft as satin), looked up
in his face with an expression of exquisite pathos and confidence,
when---when the door opened, and Miss Crawley sailed in.

Mrs.\ Firkin and Miss Briggs, who happened by chance to be at the
parlour door soon after the Baronet and Rebecca entered the apartment,
had also seen accidentally, through the keyhole, the old gentleman
prostrate before the governess, and had heard the generous proposal
which he made her.  It was scarcely out of his mouth when Mrs.\ Firkin
and Miss Briggs had streamed up the stairs, had rushed into the
drawing-room where Miss Crawley was reading the French novel, and had
given that old lady the astounding intelligence that Sir Pitt was on
his knees, proposing to Miss Sharp.  And if you calculate the time for
the above dialogue to take place---the time for Briggs and Firkin to fly
to the drawing-room---the time for Miss Crawley to be astonished, and to
drop her volume of Pigault le Brun---and the time for her to come
downstairs---you will see how exactly accurate this history is, and how
Miss Crawley must have appeared at the very instant when Rebecca had
assumed the attitude of humility.

``It is the lady on the ground, and not the gentleman,'' Miss Crawley
said, with a look and voice of great scorn. ``They told me that YOU were
on your knees, Sir Pitt: do kneel once more, and let me see this pretty
couple!''

``I have thanked Sir Pitt Crawley, Ma'am,'' Rebecca said, rising, ``and
have told him that---that I never can become Lady Crawley.''

``Refused him!''  Miss Crawley said, more bewildered than ever. Briggs
and Firkin at the door opened the eyes of astonishment and the lips of
wonder.

``Yes---refused,'' Rebecca continued, with a sad, tearful voice.

``And am I to credit my ears that you absolutely proposed to her, Sir
Pitt?'' the old lady asked.

``Ees,'' said the Baronet, ``I did.''

``And she refused you as she says?''

``Ees,'' Sir Pitt said, his features on a broad grin.

``It does not seem to break your heart at any rate,'' Miss Crawley
remarked.

``Nawt a bit,'' answered Sir Pitt, with a coolness and good-humour which
set Miss Crawley almost mad with bewilderment.  That an old gentleman
of station should fall on his knees to a penniless governess, and burst
out laughing because she refused to marry him---that a penniless
governess should refuse a Baronet with four thousand a year---these were
mysteries which Miss Crawley could never comprehend.  It surpassed any
complications of intrigue in her favourite Pigault le Brun.

``I'm glad you think it good sport, brother,'' she continued, groping
wildly through this amazement.

``Vamous,'' said Sir Pitt.  ``Who'd ha' thought it! what a sly little
devil! what a little fox it waws!'' he muttered to himself, chuckling
with pleasure.

``Who'd have thought what?'' cries Miss Crawley, stamping with her foot.
``Pray, Miss Sharp, are you waiting for the Prince Regent's divorce,
that you don't think our family good enough for you?''

``My attitude,'' Rebecca said, ``when you came in, ma'am, did not look as
if I despised such an honour as this good---this noble man has deigned
to offer me.  Do you think I have no heart?  Have you all loved me, and
been so kind to the poor orphan---deserted---girl, and am I to feel
nothing?  O my friends!  O my benefactors! may not my love, my life, my
duty, try to repay the confidence you have shown me?  Do you grudge me
even gratitude, Miss Crawley?  It is too much---my heart is too full'';
and she sank down in a chair so pathetically, that most of the audience
present were perfectly melted with her sadness.

``Whether you marry me or not, you're a good little girl, Becky, and I'm
your vriend, mind,'' said Sir Pitt, and putting on his crape-bound hat,
he walked away---greatly to Rebecca's relief; for it was evident that
her secret was unrevealed to Miss Crawley, and she had the advantage of
a brief reprieve.

Putting her handkerchief to her eyes, and nodding away honest Briggs,
who would have followed her upstairs, she went up to her apartment;
while Briggs and Miss Crawley, in a high state of excitement, remained
to discuss the strange event, and Firkin, not less moved, dived down
into the kitchen regions, and talked of it with all the male and female
company there.  And so impressed was Mrs.\ Firkin with the news, that
she thought proper to write off by that very night's post, ``with her
humble duty to Mrs.\ Bute Crawley and the family at the Rectory, and Sir
Pitt has been and proposed for to marry Miss Sharp, wherein she has
refused him, to the wonder of all.''

The two ladies in the dining-room (where worthy Miss Briggs was
delighted to be admitted once more to confidential conversation with
her patroness) wondered to their hearts' content at Sir Pitt's offer,
and Rebecca's refusal; Briggs very acutely suggesting that there must
have been some obstacle in the shape of a previous attachment,
otherwise no young woman in her senses would ever have refused so
advantageous a proposal.

``You would have accepted it yourself, wouldn't you, Briggs?'' Miss
Crawley said, kindly.

``Would it not be a privilege to be Miss Crawley's sister?'' Briggs
replied, with meek evasion.

``Well, Becky would have made a good Lady Crawley, after all,'' Miss
Crawley remarked (who was mollified by the girl's refusal, and very
liberal and generous now there was no call for her sacrifices). ``She
has brains in plenty (much more wit in her little finger than you have,
my poor dear Briggs, in all your head).  Her manners are excellent, now
I have formed her.  She is a Montmorency, Briggs, and blood is
something, though I despise it for my part; and she would have held her
own amongst those pompous stupid Hampshire people much better than that
unfortunate ironmonger's daughter.''

Briggs coincided as usual, and the ``previous attachment'' was then
discussed in conjectures.  ``You poor friendless creatures are always
having some foolish tendre,'' Miss Crawley said.  ``You yourself, you
know, were in love with a writing-master (don't cry, Briggs---you're
always crying, and it won't bring him to life again), and I suppose
this unfortunate Becky has been silly and sentimental too---some
apothecary, or house-steward, or painter, or young curate, or something
of that sort.''

``Poor thing! poor thing!'' says Briggs (who was thinking of twenty-four
years back, and that hectic young writing-master whose lock of yellow
hair, and whose letters, beautiful in their illegibility, she cherished
in her old desk upstairs).  ``Poor thing, poor thing!'' says Briggs.
Once more she was a fresh-cheeked lass of eighteen; she was at evening
church, and the hectic writing-master and she were quavering out of the
same psalm-book.

``After such conduct on Rebecca's part,'' Miss Crawley said
enthusiastically, ``our family should do something. Find out who is the
objet, Briggs.  I'll set him up in a shop; or order my portrait of him,
you know; or speak to my cousin, the Bishop and I'll doter Becky, and
we'll have a wedding, Briggs, and you shall make the breakfast, and be
a bridesmaid.''

Briggs declared that it would be delightful, and vowed that her dear
Miss Crawley was always kind and generous, and went up to Rebecca's
bedroom to console her and prattle about the offer, and the refusal,
and the cause thereof; and to hint at the generous intentions of Miss
Crawley, and to find out who was the gentleman that had the mastery of
Miss Sharp's heart.

Rebecca was very kind, very affectionate and affected---responded to
Briggs's offer of tenderness with grateful fervour---owned there was a
secret attachment---a delicious mystery---what a pity Miss Briggs had not
remained half a minute longer at the keyhole!  Rebecca might, perhaps,
have told more: but five minutes after Miss Briggs's arrival in
Rebecca's apartment, Miss Crawley actually made her appearance
there---an unheard-of honour---her impatience had overcome her; she could
not wait for the tardy operations of her ambassadress: so she came in
person, and ordered Briggs out of the room. And expressing her approval
of Rebecca's conduct, she asked particulars of the interview, and the
previous transactions which had brought about the astonishing offer of
Sir Pitt.

Rebecca said she had long had some notion of the partiality with which
Sir Pitt honoured her (for he was in the habit of making his feelings
known in a very frank and unreserved manner) but, not to mention
private reasons with which she would not for the present trouble Miss
Crawley, Sir Pitt's age, station, and habits were such as to render a
marriage quite impossible; and could a woman with any feeling of
self-respect and any decency listen to proposals at such a moment, when
the funeral of the lover's deceased wife had not actually taken place?

``Nonsense, my dear, you would never have refused him had there not been
some one else in the case,'' Miss Crawley said, coming to her point at
once.  ``Tell me the private reasons; what are the private reasons?
There is some one; who is it that has touched your heart?''

Rebecca cast down her eyes, and owned there was. ``You have guessed
right, dear lady,'' she said, with a sweet simple faltering voice. ``You
wonder at one so poor and friendless having an attachment, don't you? I
have never heard that poverty was any safeguard against it.  I wish it
were.''

``My poor dear child,'' cried Miss Crawley, who was always quite ready to
be sentimental, ``is our passion unrequited, then?  Are we pining in
secret? Tell me all, and let me console you.''

``I wish you could, dear Madam,'' Rebecca said in the same tearful tone.
``Indeed, indeed, I need it.'' And she laid her head upon Miss Crawley's
shoulder and wept there so naturally that the old lady, surprised into
sympathy, embraced her with an almost maternal kindness, uttered many
soothing protests of regard and affection for her, vowed that she loved
her as a daughter, and would do everything in her power to serve her.
``And now who is it, my dear?  Is it that pretty Miss Sedley's brother?
You said something about an affair with him. I'll ask him here, my
dear.  And you shall have him: indeed you shall.''

``Don't ask me now,'' Rebecca said.  ``You shall know all soon.  Indeed
you shall.  Dear kind Miss Crawley---dear friend, may I say so?''

``That you may, my child,'' the old lady replied, kissing her.

``I can't tell you now,'' sobbed out Rebecca, ``I am very miserable. But
O! love me always---promise you will love me always.'' And in the midst
of mutual tears---for the emotions of the younger woman had awakened the
sympathies of the elder---this promise was solemnly given by Miss
Crawley, who left her little protege, blessing and admiring her as a
dear, artless, tender-hearted, affectionate, incomprehensible creature.

And now she was left alone to think over the sudden and wonderful
events of the day, and of what had been and what might have been. What
think you were the private feelings of Miss, no (begging her pardon) of
Mrs.\ Rebecca?  If, a few pages back, the present writer claimed the
privilege of peeping into Miss Amelia Sedley's bedroom, and
understanding with the omniscience of the novelist all the gentle pains
and passions which were tossing upon that innocent pillow, why should
he not declare himself to be Rebecca's confidante too, master of her
secrets, and seal-keeper of that young woman's conscience?

Well, then, in the first place, Rebecca gave way to some very sincere
and touching regrets that a piece of marvellous good fortune should
have been so near her, and she actually obliged to decline it.  In this
natural emotion every properly regulated mind will certainly share.
What good mother is there that would not commiserate a penniless
spinster, who might have been my lady, and have shared four thousand a
year?  What well-bred young person is there in all Vanity Fair, who
will not feel for a hard-working, ingenious, meritorious girl, who gets
such an honourable, advantageous, provoking offer, just at the very
moment when it is out of her power to accept it?  I am sure our friend
Becky's disappointment deserves and will command every sympathy.

I remember one night being in the Fair myself, at an evening party. I
observed old Miss Toady there also present, single out for her special
attentions and flattery little Mrs.\ Briefless, the barrister's wife,
who is of a good family certainly, but, as we all know, is as poor as
poor can be.

What, I asked in my own mind, can cause this obsequiousness on the part
of Miss Toady; has Briefless got a county court, or has his wife had a
fortune left her? Miss Toady explained presently, with that simplicity
which distinguishes all her conduct.  ``You know,'' she said, ``Mrs
Briefless is granddaughter of Sir John Redhand, who is so ill at
Cheltenham that he can't last six months.  Mrs.\ Briefless's papa
succeeds; so you see she will be a baronet's daughter.'' And Toady asked
Briefless and his wife to dinner the very next week.

If the mere chance of becoming a baronet's daughter can procure a lady
such homage in the world, surely, surely we may respect the agonies of
a young woman who has lost the opportunity of becoming a baronet's
wife.  Who would have dreamed of Lady Crawley dying so soon?  She was
one of those sickly women that might have lasted these ten
years---Rebecca thought to herself, in all the woes of repentance---and I
might have been my lady!  I might have led that old man whither I
would.  I might have thanked Mrs.\ Bute for her patronage, and Mr.\ Pitt
for his insufferable condescension.  I would have had the town-house
newly furnished and decorated.  I would have had the handsomest
carriage in London, and a box at the opera; and I would have been
presented next season.  All this might have been; and now---now all was
doubt and mystery.

But Rebecca was a young lady of too much resolution and energy of
character to permit herself much useless and unseemly sorrow for the
irrevocable past; so, having devoted only the proper portion of regret
to it, she wisely turned her whole attention towards the future, which
was now vastly more important to her.  And she surveyed her position,
and its hopes, doubts, and chances.

In the first place, she was MARRIED---that was a great fact.  Sir Pitt
knew it.  She was not so much surprised into the avowal, as induced to
make it by a sudden calculation. It must have come some day: and why
not now as at a later period? He who would have married her himself
must at least be silent with regard to her marriage. How Miss Crawley
would bear the news---was the great question. Misgivings Rebecca had;
but she remembered all Miss Crawley had said; the old lady's avowed
contempt for birth; her daring liberal opinions; her general romantic
propensities; her almost doting attachment to her nephew, and her
repeatedly expressed fondness for Rebecca herself.  She is so fond of
him, Rebecca thought, that she will forgive him anything: she is so
used to me that I don't think she could be comfortable without me: when
the eclaircissement comes there will be a scene, and hysterics, and a
great quarrel, and then a great reconciliation.  At all events, what
use was there in delaying? the die was thrown, and now or to-morrow the
issue must be the same.  And so, resolved that Miss Crawley should have
the news, the young person debated in her mind as to the best means of
conveying it to her; and whether she should face the storm that must
come, or fly and avoid it until its first fury was blown over.  In this
state of meditation she wrote the following letter:

Dearest Friend,

The great crisis which we have debated about so often is COME.  Half of
my secret is known, and I have thought and thought, until I am quite
sure that now is the time to reveal THE WHOLE OF THE MYSTERY. Sir Pitt
came to me this morning, and made---what do you think?---A DECLARATION IN
FORM.  Think of that!  Poor little me.  I might have been Lady Crawley.
How pleased Mrs.\ Bute would have been: and ma tante if I had taken
precedence of her! I might have been somebody's mamma, instead of---O, I
tremble, I tremble, when I think how soon we must tell all!

Sir Pitt knows I am married, and not knowing to whom, is not very much
displeased as yet.  Ma tante is ACTUALLY ANGRY that I should have
refused him.  But she is all kindness and graciousness.  She
condescends to say I would have made him a good wife; and vows that she
will be a mother to your little Rebecca.  She will be shaken when she
first hears the news.  But need we fear anything beyond a momentary
anger?  I think not: I AM SURE not.  She dotes upon you so (you
naughty, good-for-nothing man), that she would pardon you ANYTHING:
and, indeed, I believe, the next place in her heart is mine: and that
she would be miserable without me. Dearest! something TELLS ME we shall
conquer.  You shall leave that odious regiment: quit gaming, racing,
and BE A GOOD BOY; and we shall all live in Park Lane, and ma tante
shall leave us all her money.

I shall try and walk to-morrow at 3 in the usual place. If Miss B.
accompanies me, you must come to dinner, and bring an answer, and put
it in the third volume of Porteus's Sermons.  But, at all events, come
to your own

R.

To Miss Eliza Styles, At Mr.\ Barnet's, Saddler, Knightsbridge.

And I trust there is no reader of this little story who has not
discernment enough to perceive that the Miss Eliza Styles (an old
schoolfellow, Rebecca said, with whom she had resumed an active
correspondence of late, and who used to fetch these letters from the
saddler's), wore brass spurs, and large curling mustachios, and was
indeed no other than Captain Rawdon Crawley.



\chapter{The Letter on the Pincushion}

How they were married is not of the slightest consequence to anybody.
What is to hinder a Captain who is a major, and a young lady who is of
age, from purchasing a licence, and uniting themselves at any church in
this town?  Who needs to be told, that if a woman has a will she will
assuredly find a way?---My belief is that one day, when Miss Sharp had
gone to pass the forenoon with her dear friend Miss Amelia Sedley in
Russell Square, a lady very like her might have been seen entering a
church in the City, in company with a gentleman with dyed mustachios,
who, after a quarter of an hour's interval, escorted her back to the
hackney-coach in waiting, and that this was a quiet bridal party.

And who on earth, after the daily experience we have, can question the
probability of a gentleman marrying anybody? How many of the wise and
learned have married their cooks?  Did not Lord Eldon himself, the most
prudent of men, make a runaway match? Were not Achilles and Ajax both
in love with their servant maids? And are we to expect a heavy dragoon
with strong desires and small brains, who had never controlled a
passion in his life, to become prudent all of a sudden, and to refuse
to pay any price for an indulgence to which he had a mind?  If people
only made prudent marriages, what a stop to population there would be!

It seems to me, for my part, that Mr.\ Rawdon's marriage was one of the
honestest actions which we shall have to record in any portion of that
gentleman's biography which has to do with the present history.  No one
will say it is unmanly to be captivated by a woman, or, being
captivated, to marry her; and the admiration, the delight, the passion,
the wonder, the unbounded confidence, and frantic adoration with which,
by degrees, this big warrior got to regard the little Rebecca, were
feelings which the ladies at least will pronounce were not altogether
discreditable to him.  When she sang, every note thrilled in his dull
soul, and tingled through his huge frame.  When she spoke, he brought
all the force of his brains to listen and wonder. If she was jocular,
he used to revolve her jokes in his mind, and explode over them half an
hour afterwards in the street, to the surprise of the groom in the
tilbury by his side, or the comrade riding with him in Rotten Row. Her
words were oracles to him, her smallest actions marked by an infallible
grace and wisdom. ``How she sings,---how she paints,'' thought he.  ``How
she rode that kicking mare at Queen's Crawley!''  And he would say to
her in confidential moments, ``By Jove, Beck, you're fit to be
Commander-in-Chief, or Archbishop of Canterbury, by Jove.''  Is his
case a rare one? and don't we see every day in the world many an honest
Hercules at the apron-strings of Omphale, and great whiskered Samsons
prostrate in Delilah's lap?

When, then, Becky told him that the great crisis was near, and the time
for action had arrived, Rawdon expressed himself as ready to act under
her orders, as he would be to charge with his troop at the command of
his colonel.  There was no need for him to put his letter into the
third volume of Porteus.  Rebecca easily found a means to get rid of
Briggs, her companion, and met her faithful friend in ``the usual place''
on the next day.  She had thought over matters at night, and
communicated to Rawdon the result of her determinations. He agreed, of
course, to everything; was quite sure that it was all right: that what
she proposed was best; that Miss Crawley would infallibly relent, or
``come round,'' as he said, after a time.  Had Rebecca's resolutions been
entirely different, he would have followed them as implicitly.  ``You
have head enough for both of us, Beck,'' said he.  ``You're sure to get
us out of the scrape.  I never saw your equal, and I've met with some
clippers in my time too.'' And with this simple confession of faith, the
love-stricken dragoon left her to execute his part of the project which
she had formed for the pair.

It consisted simply in the hiring of quiet lodgings at Brompton, or in
the neighbourhood of the barracks, for Captain and Mrs.\ Crawley. For
Rebecca had determined, and very prudently, we think, to fly. Rawdon
was only too happy at her resolve; he had been entreating her to take
this measure any time for weeks past.  He pranced off to engage the
lodgings with all the impetuosity of love.  He agreed to pay two
guineas a week so readily, that the landlady regretted she had asked
him so little. He ordered in a piano, and half a nursery-house full of
flowers: and a heap of good things.  As for shawls, kid gloves, silk
stockings, gold French watches, bracelets and perfumery, he sent them
in with the profusion of blind love and unbounded credit.  And having
relieved his mind by this outpouring of generosity, he went and dined
nervously at the club, waiting until the great moment of his life
should come.

The occurrences of the previous day; the admirable conduct of
Rebecca in refusing an offer so advantageous to her, the secret
unhappiness preying upon her, the sweetness and silence with which she
bore her affliction, made Miss Crawley much more tender than usual.  An
event of this nature, a marriage, or a refusal, or a proposal, thrills
through a whole household of women, and sets all their hysterical
sympathies at work.  As an observer of human nature, I regularly
frequent St.\ George's, Hanover Square, during the genteel marriage
season; and though I have never seen the bridegroom's male friends give
way to tears, or the beadles and officiating clergy any way affected,
yet it is not at all uncommon to see women who are not in the least
concerned in the operations going on---old ladies who are long past
marrying, stout middle-aged females with plenty of sons and daughters,
let alone pretty young creatures in pink bonnets, who are on their
promotion, and may naturally take an interest in the ceremony---I say it
is quite common to see the women present piping, sobbing, sniffling;
hiding their little faces in their little useless pocket-handkerchiefs;
and heaving, old and young, with emotion.  When my friend, the
fashionable John Pimlico, married the lovely Lady Belgravia Green
Parker, the excitement was so general that even the little snuffy old
pew-opener who let me into the seat was in tears.  And wherefore? I
inquired of my own soul: she was not going to be married.

Miss Crawley and Briggs in a word, after the affair of Sir Pitt,
indulged in the utmost luxury of sentiment, and Rebecca became an
object of the most tender interest to them.  In her absence Miss
Crawley solaced herself with the most sentimental of the novels in her
library.  Little Sharp, with her secret griefs, was the heroine of the
day.

That night Rebecca sang more sweetly and talked more pleasantly than
she had ever been heard to do in Park Lane.  She twined herself round
the heart of Miss Crawley. She spoke lightly and laughingly of Sir
Pitt's proposal, ridiculed it as the foolish fancy of an old man; and
her eyes filled with tears, and Briggs's heart with unutterable pangs
of defeat, as she said she desired no other lot than to remain for ever
with her dear benefactress.  ``My dear little creature,'' the old lady
said, ``I don't intend to let you stir for years, that you may depend
upon it.  As for going back to that odious brother of mine after what
has passed, it is out of the question.  Here you stay with me and
Briggs.  Briggs wants to go to see her relations very often.  Briggs,
you may go when you like. But as for you, my dear, you must stay and
take care of the old woman.''

If Rawdon Crawley had been then and there present, instead of being at
the club nervously drinking claret, the pair might have gone down on
their knees before the old spinster, avowed all, and been forgiven in a
twinkling. But that good chance was denied to the young couple,
doubtless in order that this story might be written, in which numbers
of their wonderful adventures are narrated---adventures which could
never have occurred to them if they had been housed and sheltered under
the comfortable uninteresting forgiveness of Miss Crawley.

Under Mrs.\ Firkin's orders, in the Park Lane establishment, was a young
woman from Hampshire, whose business it was, among other duties, to
knock at Miss Sharp's door with that jug of hot water which Firkin
would rather have perished than have presented to the intruder.  This
girl, bred on the family estate, had a brother in Captain Crawley's
troop, and if the truth were known, I daresay it would come out that
she was aware of certain arrangements, which have a great deal to do
with this history. At any rate she purchased a yellow shawl, a pair of
green boots, and a light blue hat with a red feather with three guineas
which Rebecca gave her, and as little Sharp was by no means too liberal
with her money, no doubt it was for services rendered that Betty Martin
was so bribed.

On the second day after Sir Pitt Crawley's offer to Miss Sharp, the sun
rose as usual, and at the usual hour Betty Martin, the upstairs maid,
knocked at the door of the governess's bedchamber.

No answer was returned, and she knocked again.  Silence was still
uninterrupted; and Betty, with the hot water, opened the door and
entered the chamber.

The little white dimity bed was as smooth and trim as on the day
previous, when Betty's own hands had helped to make it.  Two little
trunks were corded in one end of the room; and on the table before the
window---on the pincushion---the great fat pincushion lined with pink
inside, and twilled like a lady's nightcap---lay a letter.  It had been
reposing there probably all night.

Betty advanced towards it on tiptoe, as if she were afraid to awake
it---looked at it, and round the room, with an air of great wonder and
satisfaction; took up the letter, and grinned intensely as she turned
it round and over, and finally carried it into Miss Briggs's room below.

How could Betty tell that the letter was for Miss Briggs, I should like
to know?  All the schooling Betty had had was at Mrs.\ Bute Crawley's
Sunday school, and she could no more read writing than Hebrew.

``La, Miss Briggs,'' the girl exclaimed, ``O, Miss, something must have
happened---there's nobody in Miss Sharp's room; the bed ain't been slep
in, and she've run away, and left this letter for you, Miss.''

``WHAT!'' cries Briggs, dropping her comb, the thin wisp of faded hair
falling over her shoulders; ``an elopement! Miss Sharp a fugitive! What,
what is this?'' and she eagerly broke the neat seal, and, as they say,
``devoured the contents'' of the letter addressed to her.

Dear Miss Briggs [the refugee wrote], the kindest heart in the world,
as yours is, will pity and sympathise with me and excuse me. With
tears, and prayers, and blessings, I leave the home where the poor
orphan has ever met with kindness and affection.  Claims even superior
to those of my benefactress call me hence.  I go to my duty---to my
HUSBAND.  Yes, I am married.  My husband COMMANDS me to seek the HUMBLE
HOME which we call ours.  Dearest Miss Briggs, break the news as your
delicate sympathy will know how to do it---to my dear, my beloved friend
and benefactress.  Tell her, ere I went, I shed tears on her dear
pillow---that pillow that I have so often soothed in sickness---that I
long AGAIN to watch---Oh, with what joy shall I return to dear Park
Lane! How I tremble for the answer which is to SEAL MY FATE! When Sir
Pitt deigned to offer me his hand, an honour of which my beloved Miss
Crawley said I was DESERVING (my blessings go with her for judging the
poor orphan worthy to be HER SISTER!) I told Sir Pitt that I was
already A WIFE.  Even he forgave me.  But my courage failed me, when I
should have told him all---that I could not be his wife, for I WAS HIS
DAUGHTER!  I am wedded to the best and most generous of men---Miss
Crawley's Rawdon is MY Rawdon. At his COMMAND I open my lips, and
follow him to our humble home, as I would THROUGH THE WORLD.  O, my
excellent and kind friend, intercede with my Rawdon's beloved aunt for
him and the poor girl to whom all HIS NOBLE RACE have shown such
UNPARALLELED AFFECTION.  Ask Miss Crawley to receive HER CHILDREN.  I
can say no more, but blessings, blessings on all in the dear house I
leave, prays

Your affectionate and GRATEFUL Rebecca Crawley. Midnight.

Just as Briggs had finished reading this affecting and interesting
document, which reinstated her in her position as first confidante of
Miss Crawley, Mrs.\ Firkin entered the room.  ``Here's Mrs.\ Bute Crawley
just arrived by the mail from Hampshire, and wants some tea; will you
come down and make breakfast, Miss?''

And to the surprise of Firkin, clasping her dressing-gown around her,
the wisp of hair floating dishevelled behind her, the little
curl-papers still sticking in bunches round her forehead, Briggs sailed
down to Mrs.\ Bute with the letter in her hand containing the wonderful
news.

``Oh, Mrs.\ Firkin,'' gasped Betty, ``sech a business.  Miss Sharp have a
gone and run away with the Capting, and they're off to Gretney Green!''
We would devote a chapter to describe the emotions of Mrs.\ Firkin, did
not the passions of her mistresses occupy our genteeler muse.

When Mrs.\ Bute Crawley, numbed with midnight travelling, and warming
herself at the newly crackling parlour fire, heard from Miss Briggs the
intelligence of the clandestine marriage, she declared it was quite
providential that she should have arrived at such a time to assist poor
dear Miss Crawley in supporting the shock---that Rebecca was an artful
little hussy of whom she had always had her suspicions; and that as for
Rawdon Crawley, she never could account for his aunt's infatuation
regarding him, and had long considered him a profligate, lost, and
abandoned being.  And this awful conduct, Mrs.\ Bute said, will have at
least this good effect, it will open poor dear Miss Crawley's eyes to
the real character of this wicked man.  Then Mrs.\ Bute had a
comfortable hot toast and tea; and as there was a vacant room in the
house now, there was no need for her to remain at the Gloster Coffee
House where the Portsmouth mail had set her down, and whence she
ordered Mr.\ Bowls's aide-de-camp the footman to bring away her trunks.

Miss Crawley, be it known, did not leave her room until near noon---taking
chocolate in bed in the morning, while Becky Sharp read the
Morning Post to her, or otherwise amusing herself or dawdling.  The
conspirators below agreed that they would spare the dear lady's
feelings until she appeared in her drawing-room: meanwhile it was
announced to her that Mrs.\ Bute Crawley had come up from Hampshire by
the mail, was staying at the Gloster, sent her love to Miss Crawley,
and asked for breakfast with Miss Briggs.  The arrival of Mrs.\ Bute,
which would not have caused any extreme delight at another period, was
hailed with pleasure now; Miss Crawley being pleased at the notion of a
gossip with her sister-in-law regarding the late Lady Crawley, the
funeral arrangements pending, and Sir Pitt's abrupt proposal to Rebecca.

It was not until the old lady was fairly ensconced in her usual
arm-chair in the drawing-room, and the preliminary embraces and inquiries
had taken place between the ladies, that the conspirators thought it
advisable to submit her to the operation.  Who has not admired the
artifices and delicate approaches with which women ``prepare'' their
friends for bad news?  Miss Crawley's two friends made such an
apparatus of mystery before they broke the intelligence to her, that
they worked her up to the necessary degree of doubt and alarm.

``And she refused Sir Pitt, my dear, dear Miss Crawley, prepare yourself
for it,'' Mrs.\ Bute said, ``because---because she couldn't help herself.''

``Of course there was a reason,'' Miss Crawley answered. ``She liked
somebody else.  I told Briggs so yesterday.''

``LIKES somebody else!'' Briggs gasped.  ``O my dear friend, she is
married already.''

``Married already,'' Mrs.\ Bute chimed in; and both sate with clasped
hands looking from each other at their victim.

``Send her to me, the instant she comes in.  The little sly wretch: how
dared she not tell me?'' cried out Miss Crawley.

``She won't come in soon.  Prepare yourself, dear friend---she's gone out
for a long time---she's---she's gone altogether.''

``Gracious goodness, and who's to make my chocolate? Send for her and
have her back; I desire that she come back,'' the old lady said.

``She decamped last night, Ma'am,'' cried Mrs.\ Bute.

``She left a letter for me,'' Briggs exclaimed.  ``She's married to---''

``Prepare her, for heaven's sake.  Don't torture her, my dear Miss
Briggs.''

``She's married to whom?'' cries the spinster in a nervous fury.

``To---to a relation of---''

``She refused Sir Pitt,'' cried the victim.  ``Speak at once. Don't drive
me mad.''

``O Ma'am---prepare her, Miss Briggs---she's married to Rawdon Crawley.''

``Rawdon married Rebecca---governess---nobod--- Get out of my house, you
fool, you idiot---you stupid old Briggs---how dare you? You're in the
plot---you made him marry, thinking that I'd leave my money from him---you
did, Martha,'' the poor old lady screamed in hysteric sentences.

``I, Ma'am, ask a member of this family to marry a drawing-master's
daughter?''

``Her mother was a Montmorency,'' cried out the old lady, pulling at the
bell with all her might.

``Her mother was an opera girl, and she has been on the stage or worse
herself,'' said Mrs.\ Bute.

Miss Crawley gave a final scream, and fell back in a faint.  They were
forced to take her back to the room which she had just quitted. One fit
of hysterics succeeded another.  The doctor was sent for---the
apothecary arrived. Mrs.\ Bute took up the post of nurse by her bedside.
``Her relations ought to be round about her,'' that amiable woman said.

She had scarcely been carried up to her room, when a new person arrived
to whom it was also necessary to break the news.  This was Sir Pitt.
``Where's Becky?'' he said, coming in.  ``Where's her traps? She's coming
with me to Queen's Crawley.''

``Have you not heard the astonishing intelligence regarding her
surreptitious union?'' Briggs asked.

``What's that to me?'' Sir Pitt asked.  ``I know she's married.  That
makes no odds.  Tell her to come down at once, and not keep me.''

``Are you not aware, sir,'' Miss Briggs asked, ``that she has left our
roof, to the dismay of Miss Crawley, who is nearly killed by the
intelligence of Captain Rawdon's union with her?''

When Sir Pitt Crawley heard that Rebecca was married to his son, he
broke out into a fury of language, which it would do no good to repeat
in this place, as indeed it sent poor Briggs shuddering out of the
room; and with her we will shut the door upon the figure of the
frenzied old man, wild with hatred and insane with baffled desire.

One day after he went to Queen's Crawley, he burst like a madman into
the room she had used when there---dashed open her boxes with his foot,
and flung about her papers, clothes, and other relics. Miss Horrocks,
the butler's daughter, took some of them.  The children dressed
themselves and acted plays in the others.  It was but a few days after
the poor mother had gone to her lonely burying-place; and was laid,
unwept and disregarded, in a vault full of strangers.

``Suppose the old lady doesn't come to,'' Rawdon said to his little wife,
as they sate together in the snug little Brompton lodgings. She had
been trying the new piano all the morning.  The new gloves fitted her
to a nicety; the new shawls became her wonderfully; the new rings
glittered on her little hands, and the new watch ticked at her waist;
``suppose she don't come round, eh, Becky?''

``I'LL make your fortune,'' she said; and Delilah patted Samson's cheek.

``You can do anything,'' he said, kissing the little hand. ``By Jove you
can; and we'll drive down to the Star and Garter, and dine, by Jove.''



\chapter{How Captain Dobbin Bought a Piano}

If there is any exhibition in all Vanity Fair which Satire and
Sentiment can visit arm in arm together; where you light on the
strangest contrasts laughable and tearful: where you may be gentle and
pathetic, or savage and cynical with perfect propriety: it is at one of
those public assemblies, a crowd of which are advertised every day in
the last page of the Times newspaper, and over which the late Mr.\ %
George Robins used to preside with so much dignity. There are very few
London people, as I fancy, who have not attended at these meetings, and
all with a taste for moralizing must have thought, with a sensation and
interest not a little startling and queer, of the day when their turn
shall come too, and Mr.\ Hammerdown will sell by the orders of Diogenes'
assignees, or will be instructed by the executors, to offer to public
competition, the library, furniture, plate, wardrobe, and choice cellar
of wines of Epicurus deceased.

Even with the most selfish disposition, the Vanity Fairian, as he
witnesses this sordid part of the obsequies of a departed friend, can't
but feel some sympathies and regret. My Lord Dives's remains are in the
family vault: the statuaries are cutting an inscription veraciously
commemorating his virtues, and the sorrows of his heir, who is
disposing of his goods.  What guest at Dives's table can pass the
familiar house without a sigh?---the familiar house of which the lights
used to shine so cheerfully at seven o'clock, of which the hall-doors
opened so readily, of which the obsequious servants, as you passed up
the comfortable stair, sounded your name from landing to landing, until
it reached the apartment where jolly old Dives welcomed his friends!
What a number of them he had; and what a noble way of entertaining
them.  How witty people used to be here who were morose when they got
out of the door; and how courteous and friendly men who slandered and
hated each other everywhere else!  He was pompous, but with such a cook
what would one not swallow? he was rather dull, perhaps, but would not
such wine make any conversation pleasant?  We must get some of his
Burgundy at any price, the mourners cry at his club.  ``I got this box
at old Dives's sale,'' Pincher says, handing it round, ``one of Louis
XV's mistresses---pretty thing, is it not?---sweet miniature,'' and they
talk of the way in which young Dives is dissipating his fortune.

How changed the house is, though!  The front is patched over with
bills, setting forth the particulars of the furniture in staring
capitals.  They have hung a shred of carpet out of an upstairs
window---a half dozen of porters are lounging on the dirty steps---the
hall swarms with dingy guests of oriental countenance, who thrust
printed cards into your hand, and offer to bid.  Old women and amateurs
have invaded the upper apartments, pinching the bed-curtains, poking
into the feathers, shampooing the mattresses, and clapping the wardrobe
drawers to and fro. Enterprising young housekeepers are measuring the
looking-glasses and hangings to see if they will suit the new menage
(Snob will brag for years that he has purchased this or that at Dives's
sale), and Mr.\ Hammerdown is sitting on the great mahogany
dining-tables, in the dining-room below, waving the ivory hammer, and
employing all the artifices of eloquence, enthusiasm, entreaty, reason,
despair; shouting to his people; satirizing Mr.\ Davids for his
sluggishness; inspiriting Mr.\ Moss into action; imploring, commanding,
bellowing, until down comes the hammer like fate, and we pass to the
next lot.  O Dives, who would ever have thought, as we sat round the
broad table sparkling with plate and spotless linen, to have seen such
a dish at the head of it as that roaring auctioneer?

It was rather late in the sale.  The excellent drawing-room furniture
by the best makers; the rare and famous wines selected, regardless of
cost, and with the well-known taste of the purchaser; the rich and
complete set of family plate had been sold on the previous days.
Certain of the best wines (which all had a great character among
amateurs in the neighbourhood) had been purchased for his master, who
knew them very well, by the butler of our friend John Osborne, Esquire,
of Russell Square.  A small portion of the most useful articles of the
plate had been bought by some young stockbrokers from the City.  And
now the public being invited to the purchase of minor objects, it
happened that the orator on the table was expatiating on the merits of
a picture, which he sought to recommend to his audience: it was by no
means so select or numerous a company as had attended the previous days
of the auction.

``No.\ 369,'' roared Mr.\ Hammerdown.  ``Portrait of a gentleman on an
elephant.  Who'll bid for the gentleman on the elephant?  Lift up the
picture, Blowman, and let the company examine this lot.'' A long, pale,
military-looking gentleman, seated demurely at the mahogany table,
could not help grinning as this valuable lot was shown by Mr.\ Blowman.
``Turn the elephant to the Captain, Blowman.  What shall we say, sir,
for the elephant?'' but the Captain, blushing in a very hurried and
discomfited manner, turned away his head.

``Shall we say twenty guineas for this work of art?---fifteen, five, name
your own price.  The gentleman without the elephant is worth five
pound.''

``I wonder it ain't come down with him,'' said a professional wag, ``he's
anyhow a precious big one''; at which (for the elephant-rider was
represented as of a very stout figure) there was a general giggle in
the room.

``Don't be trying to deprecate the value of the lot, Mr.\ Moss,'' Mr.\ %
Hammerdown said; ``let the company examine it as a work of art---the
attitude of the gallant animal quite according to natur'; the gentleman
in a nankeen jacket, his gun in his hand, is going to the chase; in the
distance a banyhann tree and a pagody, most likely resemblances of some
interesting spot in our famous Eastern possessions.  How much for this
lot? Come, gentlemen, don't keep me here all day.''

Some one bid five shillings, at which the military gentleman looked
towards the quarter from which this splendid offer had come, and there
saw another officer with a young lady on his arm, who both appeared to
be highly amused with the scene, and to whom, finally, this lot was
knocked down for half a guinea.  He at the table looked more surprised
and discomposed than ever when he spied this pair, and his head sank
into his military collar, and he turned his back upon them, so as to
avoid them altogether.

Of all the other articles which Mr.\ Hammerdown had the honour to offer
for public competition that day it is not our purpose to make mention,
save of one only, a little square piano, which came down from the upper
regions of the house (the state grand piano having been disposed of
previously); this the young lady tried with a rapid and skilful hand
(making the officer blush and start again), and for it, when its turn
came, her agent began to bid.

But there was an opposition here.  The Hebrew aide-de-camp in the
service of the officer at the table bid against the Hebrew gentleman
employed by the elephant purchasers, and a brisk battle ensued over
this little piano, the combatants being greatly encouraged by Mr.\ %
Hammerdown.

At last, when the competition had been prolonged for some time, the
elephant captain and lady desisted from the race; and the hammer coming
down, the auctioneer said:---``Mr.\ Lewis, twenty-five,'' and Mr.\ Lewis's
chief thus became the proprietor of the little square piano. Having
effected the purchase, he sate up as if he was greatly relieved, and
the unsuccessful competitors catching a glimpse of him at this moment,
the lady said to her friend,

``Why, Rawdon, it's Captain Dobbin.''

I suppose Becky was discontented with the new piano her husband had
hired for her, or perhaps the proprietors of that instrument had
fetched it away, declining farther credit, or perhaps she had a
particular attachment for the one which she had just tried to purchase,
recollecting it in old days, when she used to play upon it, in the
little sitting-room of our dear Amelia Sedley.

The sale was at the old house in Russell Square, where we passed some
evenings together at the beginning of this story.  Good old John Sedley
was a ruined man.  His name had been proclaimed as a defaulter on the
Stock Exchange, and his bankruptcy and commercial extermination had
followed.  Mr.\ Osborne's butler came to buy some of the famous port
wine to transfer to the cellars over the way. As for one dozen
well-manufactured silver spoons and forks at per oz., and one dozen
dessert ditto ditto, there were three young stockbrokers (Messrs. Dale,
Spiggot, and Dale, of Threadneedle Street, indeed), who, having had
dealings with the old man, and kindnesses from him in days when he was
kind to everybody with whom he dealt, sent this little spar out of the
wreck with their love to good Mrs.\ Sedley; and with respect to the
piano, as it had been Amelia's, and as she might miss it and want one
now, and as Captain William Dobbin could no more play upon it than he
could dance on the tight rope, it is probable that he did not purchase
the instrument for his own use.

In a word, it arrived that evening at a wonderful small cottage in a
street leading from the Fulham Road---one of those streets which have
the finest romantic names---(this was called St.\ Adelaide Villas,
Anna-Maria Road West), where the houses look like baby-houses; where
the people, looking out of the first-floor windows, must infallibly, as
you think, sit with their feet in the parlours; where the shrubs in the
little gardens in front bloom with a perennial display of little
children's pinafores, little red socks, caps, \&c. (polyandria
polygynia); whence you hear the sound of jingling spinets and women
singing; where little porter pots hang on the railings sunning
themselves; whither of evenings you see City clerks padding wearily:
here it was that Mr.\ Clapp, the clerk of Mr.\ Sedley, had his domicile,
and in this asylum the good old gentleman hid his head with his wife
and daughter when the crash came.

Jos Sedley had acted as a man of his disposition would, when the
announcement of the family misfortune reached him.  He did not come to
London, but he wrote to his mother to draw upon his agents for whatever
money was wanted, so that his kind broken-spirited old parents had no
present poverty to fear.  This done, Jos went on at the boarding-house
at Cheltenham pretty much as before.  He drove his curricle; he drank
his claret; he played his rubber; he told his Indian stories, and the
Irish widow consoled and flattered him as usual. His present of money,
needful as it was, made little impression on his parents; and I have
heard Amelia say that the first day on which she saw her father lift up
his head after the failure was on the receipt of the packet of forks
and spoons with the young stockbrokers' love, over which he burst out
crying like a child, being greatly more affected than even his wife, to
whom the present was addressed.  Edward Dale, the junior of the house,
who purchased the spoons for the firm, was, in fact, very sweet upon
Amelia, and offered for her in spite of all. He married Miss Louisa
Cutts (daughter of Higham and Cutts, the eminent cornfactors) with a
handsome fortune in 1820; and is now living in splendour, and with a
numerous family, at his elegant villa, Muswell Hill.  But we must not
let the recollections of this good fellow cause us to diverge from the
principal history.

I hope the reader has much too good an opinion of Captain and Mrs.\ %
Crawley to suppose that they ever would have dreamed of paying a visit
to so remote a district as Bloomsbury, if they thought the family whom
they proposed to honour with a visit were not merely out of fashion,
but out of money, and could be serviceable to them in no possible
manner.  Rebecca was entirely surprised at the sight of the comfortable
old house where she had met with no small kindness, ransacked by
brokers and bargainers, and its quiet family treasures given up to
public desecration and plunder.  A month after her flight, she had
bethought her of Amelia, and Rawdon, with a horse-laugh, had expressed
a perfect willingness to see young George Osborne again.  ``He's a very
agreeable acquaintance, Beck,'' the wag added.  ``I'd like to sell him
another horse, Beck.  I'd like to play a few more games at billiards
with him.  He'd be what I call useful just now, Mrs.\ C.---ha, ha!'' by
which sort of speech it is not to be supposed that Rawdon Crawley had a
deliberate desire to cheat Mr.\ Osborne at play, but only wished to take
that fair advantage of him which almost every sporting gentleman in
Vanity Fair considers to be his due from his neighbour.

The old aunt was long in ``coming-to.'' A month had elapsed.  Rawdon was
denied the door by Mr.\ Bowls; his servants could not get a lodgment in
the house at Park Lane; his letters were sent back unopened.  Miss
Crawley never stirred out---she was unwell---and Mrs.\ Bute remained still
and never left her.  Crawley and his wife both of them augured evil
from the continued presence of Mrs.\ Bute.

``Gad, I begin to perceive now why she was always bringing us together
at Queen's Crawley,'' Rawdon said.

``What an artful little woman!'' ejaculated Rebecca.

``Well, I don't regret it, if you don't,'' the Captain cried, still in an
amorous rapture with his wife, who rewarded him with a kiss by way of
reply, and was indeed not a little gratified by the generous confidence
of her husband.

``If he had but a little more brains,'' she thought to herself, ``I might
make something of him''; but she never let him perceive the opinion she
had of him; listened with indefatigable complacency to his stories of
the stable and the mess; laughed at all his jokes; felt the greatest
interest in Jack Spatterdash, whose cab-horse had come down, and Bob
Martingale, who had been taken up in a gambling-house, and Tom
Cinqbars, who was going to ride the steeplechase. When he came home she
was alert and happy: when he went out she pressed him to go: when he
stayed at home, she played and sang for him, made him good drinks,
superintended his dinner, warmed his slippers, and steeped his soul in
comfort.  The best of women (I have heard my grandmother say) are
hypocrites.  We don't know how much they hide from us: how watchful
they are when they seem most artless and confidential: how often those
frank smiles which they wear so easily, are traps to cajole or elude or
disarm---I don't mean in your mere coquettes, but your domestic models,
and paragons of female virtue. Who has not seen a woman hide the
dulness of a stupid husband, or coax the fury of a savage one?  We
accept this amiable slavishness, and praise a woman for it: we call
this pretty treachery truth.  A good housewife is of necessity a
humbug; and Cornelia's husband was hoodwinked, as Potiphar was---only in
a different way.

By these attentions, that veteran rake, Rawdon Crawley, found himself
converted into a very happy and submissive married man.  His former
haunts knew him not. They asked about him once or twice at his clubs,
but did not miss him much: in those booths of Vanity Fair people seldom
do miss each other.  His secluded wife ever smiling and cheerful, his
little comfortable lodgings, snug meals, and homely evenings, had all
the charms of novelty and secrecy.  The marriage was not yet declared
to the world, or published in the Morning Post.  All his creditors
would have come rushing on him in a body, had they known that he was
united to a woman without fortune. ``My relations won't cry fie upon
me,'' Becky said, with rather a bitter laugh; and she was quite
contented to wait until the old aunt should be reconciled, before she
claimed her place in society.  So she lived at Brompton, and meanwhile
saw no one, or only those few of her husband's male companions who were
admitted into her little dining-room.  These were all charmed with her.
The little dinners, the laughing and chatting, the music afterwards,
delighted all who participated in these enjoyments.  Major Martingale
never thought about asking to see the marriage licence, Captain
Cinqbars was perfectly enchanted with her skill in making punch.  And
young Lieutenant Spatterdash (who was fond of piquet, and whom Crawley
would often invite) was evidently and quickly smitten by Mrs.\ Crawley;
but her own circumspection and modesty never forsook her for a moment,
and Crawley's reputation as a fire-eating and jealous warrior was a
further and complete defence to his little wife.

There are gentlemen of very good blood and fashion in this city, who
never have entered a lady's drawing-room; so that though Rawdon
Crawley's marriage might be talked about in his county, where, of
course, Mrs.\ Bute had spread the news, in London it was doubted, or not
heeded, or not talked about at all.  He lived comfortably on credit.
He had a large capital of debts, which laid out judiciously, will carry
a man along for many years, and on which certain men about town
contrive to live a hundred times better than even men with ready money
can do.  Indeed who is there that walks London streets, but can point
out a half-dozen of men riding by him splendidly, while he is on foot,
courted by fashion, bowed into their carriages by tradesmen, denying
themselves nothing, and living on who knows what?  We see Jack
Thriftless prancing in the park, or darting in his brougham down Pall
Mall: we eat his dinners served on his miraculous plate.  ``How did this
begin,'' we say, ``or where will it end?'' ``My dear fellow,'' I heard Jack
once say, ``I owe money in every capital in Europe.''  The end must come
some day, but in the meantime Jack thrives as much as ever; people are
glad enough to shake him by the hand, ignore the little dark stories
that are whispered every now and then against him, and pronounce him a
good-natured, jovial, reckless fellow.

Truth obliges us to confess that Rebecca had married a gentleman of
this order.  Everything was plentiful in his house but ready money, of
which their menage pretty early felt the want; and reading the Gazette
one day, and coming upon the announcement of ``Lieutenant G. Osborne to
be Captain by purchase, vice Smith, who exchanges,'' Rawdon uttered that
sentiment regarding Amelia's lover, which ended in the visit to Russell
Square.

When Rawdon and his wife wished to communicate with Captain Dobbin at
the sale, and to know particulars of the catastrophe which had befallen
Rebecca's old acquaintances, the Captain had vanished; and such
information as they got was from a stray porter or broker at the
auction.

``Look at them with their hooked beaks,'' Becky said, getting into the
buggy, her picture under her arm, in great glee.  ``They're like
vultures after a battle.''

``Don't know.  Never was in action, my dear.  Ask Martingale; he was in
Spain, aide-de-camp to General Blazes.''

``He was a very kind old man, Mr.\ Sedley,'' Rebecca said; ``I'm really
sorry he's gone wrong.''

``O stockbrokers---bankrupts---used to it, you know,'' Rawdon replied,
cutting a fly off the horse's ear.

``I wish we could have afforded some of the plate, Rawdon,'' the wife
continued sentimentally.  ``Five-and-twenty guineas was monstrously dear
for that little piano. We chose it at Broadwood's for Amelia, when she
came from school.  It only cost five-and-thirty then.''

``What-d'-ye-call'em---'Osborne,' will cry off now, I suppose, since the
family is smashed.  How cut up your pretty little friend will be; hey,
Becky?''

``I daresay she'll recover it,'' Becky said with a smile---and they drove
on and talked about something else.



\chapter{Who Played on the Piano Captain Dobbin Bought}

Our surprised story now finds itself for a moment among very famous
events and personages, and hanging on to the skirts of history. When
the eagles of Napoleon Bonaparte, the Corsican upstart, were flying
from Provence, where they had perched after a brief sojourn in Elba,
and from steeple to steeple until they reached the towers of Notre
Dame, I wonder whether the Imperial birds had any eye for a little
corner of the parish of Bloomsbury, London, which you might have
thought so quiet, that even the whirring and flapping of those mighty
wings would pass unobserved there?

``Napoleon has landed at Cannes.''  Such news might create a panic at
Vienna, and cause Russia to drop his cards, and take Prussia into a
corner, and Talleyrand and Metternich to wag their heads together,
while Prince Hardenberg, and even the present Marquis of Londonderry,
were puzzled; but how was this intelligence to affect a young lady in
Russell Square, before whose door the watchman sang the hours when she
was asleep: who, if she strolled in the square, was guarded there by
the railings and the beadle:  who, if she walked ever so short a
distance to buy a ribbon in Southampton Row, was followed by Black
Sambo with an enormous cane:  who was always cared for, dressed, put to
bed, and watched over by ever so many guardian angels, with and without
wages?  Bon Dieu, I say, is it not hard that the fateful rush of the
great Imperial struggle can't take place without affecting a poor
little harmless girl of eighteen, who is occupied in billing and
cooing, or working muslin collars in Russell Square?  You too, kindly,
homely flower!---is the great roaring war tempest coming to sweep you
down, here, although cowering under the shelter of Holborn?  Yes;
Napoleon is flinging his last stake, and poor little Emmy Sedley's
happiness forms, somehow, part of it.

In the first place, her father's fortune was swept down with that fatal
news.  All his speculations had of late gone wrong with the luckless
old gentleman.  Ventures had failed; merchants had broken; funds had
risen when he calculated they would fall.  What need to particularize?
If success is rare and slow, everybody knows how quick and easy ruin
is.  Old Sedley had kept his own sad counsel. Everything seemed to go
on as usual in the quiet, opulent house; the good-natured mistress
pursuing, quite unsuspiciously, her bustling idleness, and daily easy
avocations; the daughter absorbed still in one selfish, tender thought,
and quite regardless of all the world besides, when that final crash
came, under which the worthy family fell.

One night Mrs.\ Sedley was writing cards for a party; the Osbornes had
given one, and she must not be behindhand; John Sedley, who had come
home very late from the City, sate silent at the chimney side, while
his wife was prattling to him; Emmy had gone up to her room ailing and
low-spirited.  ``She's not happy,'' the mother went on. ``George Osborne
neglects her.  I've no patience with the airs of those people.  The
girls have not been in the house these three weeks; and George has been
twice in town without coming.  Edward Dale saw him at the Opera.
Edward would marry her I'm sure: and there's Captain Dobbin who, I
think, would---only I hate all army men.  Such a dandy as George has
become.  With his military airs, indeed!  We must show some folks that
we're as good as they.  Only give Edward Dale any encouragement, and
you'll see.  We must have a party, Mr.\ S.  Why don't you speak, John?
Shall I say Tuesday fortnight? Why don't you answer? Good God, John,
what has happened?''

John Sedley sprang up out of his chair to meet his wife, who ran to
him.  He seized her in his arms, and said with a hasty voice, ``We're
ruined, Mary.  We've got the world to begin over again, dear.  It's
best that you should know all, and at once.''  As he spoke, he trembled
in every limb, and almost fell.  He thought the news would have
overpowered his wife---his wife, to whom he had never said a hard word.
But it was he that was the most moved, sudden as the shock was to her.
When he sank back into his seat, it was the wife that took the office
of consoler.  She took his trembling hand, and kissed it, and put it
round her neck: she called him her John---her dear John---her old
man---her kind old man; she poured out a hundred words of incoherent
love and tenderness; her faithful voice and simple caresses wrought
this sad heart up to an inexpressible delight and anguish, and cheered
and solaced his over-burdened soul.

Only once in the course of the long night as they sate together, and
poor Sedley opened his pent-up soul, and told the story of his losses
and embarrassments---the treason of some of his oldest friends, the
manly kindness of some, from whom he never could have expected it---in a
general confession---only once did the faithful wife give way to emotion.

``My God, my God, it will break Emmy's heart,'' she said.

The father had forgotten the poor girl.  She was lying, awake and
unhappy, overhead.  In the midst of friends, home, and kind parents,
she was alone.  To how many people can any one tell all?  Who will be
open where there is no sympathy, or has call to speak to those who
never can understand?  Our gentle Amelia was thus solitary.  She had no
confidante, so to speak, ever since she had anything to confide.  She
could not tell the old mother her doubts and cares; the would-be
sisters seemed every day more strange to her.  And she had misgivings
and fears which she dared not acknowledge to herself, though she was
always secretly brooding over them.

Her heart tried to persist in asserting that George Osborne was worthy
and faithful to her, though she knew otherwise.  How many a thing had
she said, and got no echo from him.  How many suspicions of selfishness
and indifference had she to encounter and obstinately overcome.  To
whom could the poor little martyr tell these daily struggles and
tortures?  Her hero himself only half understood her. She did not dare
to own that the man she loved was her inferior; or to feel that she had
given her heart away too soon.  Given once, the pure bashful maiden was
too modest, too tender, too trustful, too weak, too much woman to
recall it.  We are Turks with the affections of our women; and have
made them subscribe to our doctrine too.  We let their bodies go abroad
liberally enough, with smiles and ringlets and pink bonnets to disguise
them instead of veils and yakmaks.  But their souls must be seen by
only one man, and they obey not unwillingly, and consent to remain at
home as our slaves---ministering to us and doing drudgery for us.

So imprisoned and tortured was this gentle little heart, when in the
month of March, Anno Domini 1815, Napoleon landed at Cannes, and Louis
XVIII fled, and all Europe was in alarm, and the funds fell, and old
John Sedley was ruined.

We are not going to follow the worthy old stockbroker through those
last pangs and agonies of ruin through which he passed before his
commercial demise befell. They declared him at the Stock Exchange; he
was absent from his house of business: his bills were protested: his
act of bankruptcy formal.  The house and furniture of Russell Square
were seized and sold up, and he and his family were thrust away, as we
have seen, to hide their heads where they might.

John Sedley had not the heart to review the domestic establishment who
have appeared now and anon in our pages and of whom he was now forced
by poverty to take leave.  The wages of those worthy people were
discharged with that punctuality which men frequently show who only owe
in great sums---they were sorry to leave good places---but they did not
break their hearts at parting from their adored master and mistress.
Amelia's maid was profuse in condolences, but went off quite resigned
to better herself in a genteeler quarter of the town.  Black Sambo,
with the infatuation of his profession, determined on setting up a
public-house.  Honest old Mrs.\ Blenkinsop indeed, who had seen the
birth of Jos and Amelia, and the wooing of John Sedley and his wife,
was for staying by them without wages, having amassed a considerable
sum in their service: and she accompanied the fallen people into their
new and humble place of refuge, where she tended them and grumbled
against them for a while.

Of all Sedley's opponents in his debates with his creditors which now
ensued, and harassed the feelings of the humiliated old gentleman so
severely, that in six weeks he oldened more than he had done for
fifteen years before---the most determined and obstinate seemed to be
John Osborne, his old friend and neighbour---John Osborne, whom he had
set up in life---who was under a hundred obligations to him---and whose
son was to marry Sedley's daughter. Any one of these circumstances
would account for the bitterness of Osborne's opposition.

When one man has been under very remarkable obligations to another,
with whom he subsequently quarrels, a common sense of decency, as it
were, makes of the former a much severer enemy than a mere stranger
would be.  To account for your own hard-heartedness and ingratitude in
such a case, you are bound to prove the other party's crime.  It is not
that you are selfish, brutal, and angry at the failure of a
speculation---no, no---it is that your partner has led you into it by the
basest treachery and with the most sinister motives.  From a mere sense
of consistency, a persecutor is bound to show that the fallen man is a
villain---otherwise he, the persecutor, is a wretch himself.

And as a general rule, which may make all creditors who are inclined to
be severe pretty comfortable in their minds, no men embarrassed are
altogether honest, very likely.  They conceal something; they
exaggerate chances of good luck; hide away the real state of affairs;
say that things are flourishing when they are hopeless, keep a smiling
face (a dreary smile it is) upon the verge of bankruptcy---are ready to
lay hold of any pretext for delay or of any money, so as to stave off
the inevitable ruin a few days longer. ``Down with such dishonesty,''
says the creditor in triumph, and reviles his sinking enemy.  ``You
fool, why do you catch at a straw?'' calm good sense says to the man
that is drowning.  ``You villain, why do you shrink from plunging into
the irretrievable Gazette?'' says prosperity to the poor devil battling
in that black gulf.  Who has not remarked the readiness with which the
closest of friends and honestest of men suspect and accuse each other
of cheating when they fall out on money matters? Everybody does it.
Everybody is right, I suppose, and the world is a rogue.

Then Osborne had the intolerable sense of former benefits to goad and
irritate him: these are always a cause of hostility aggravated.
Finally, he had to break off the match between Sedley's daughter and
his son; and as it had gone very far indeed, and as the poor girl's
happiness and perhaps character were compromised, it was necessary to
show the strongest reasons for the rupture, and for John Osborne to
prove John Sedley to be a very bad character indeed.

At the meetings of creditors, then, he comported himself with a
savageness and scorn towards Sedley, which almost succeeded in breaking
the heart of that ruined bankrupt man.  On George's intercourse with
Amelia he put an instant veto---menacing the youth with maledictions if
he broke his commands, and vilipending the poor innocent girl as the
basest and most artful of vixens. One of the great conditions of anger
and hatred is, that you must tell and believe lies against the hated
object, in order, as we said, to be consistent.

When the great crash came---the announcement of ruin, and the departure
from Russell Square, and the declaration that all was over between her
and George---all over between her and love, her and happiness, her and
faith in the world---a brutal letter from John Osborne told her in a few
curt lines that her father's conduct had been of such a nature that all
engagements between the families were at an end---when the final award
came, it did not shock her so much as her parents, as her mother rather
expected (for John Sedley himself was entirely prostrate in the ruins
of his own affairs and shattered honour).  Amelia took the news very
palely and calmly. It was only the confirmation of the dark presages
which had long gone before.  It was the mere reading of the
sentence---of the crime she had long ago been guilty---the crime of
loving wrongly, too violently, against reason. She told no more of her
thoughts now than she had before.  She seemed scarcely more unhappy now
when convinced all hope was over, than before when she felt but dared
not confess that it was gone.  So she changed from the large house to
the small one without any mark or difference; remained in her little
room for the most part; pined silently; and died away day by day.  I do
not mean to say that all females are so.  My dear Miss Bullock, I do
not think your heart would break in this way.  You are a strong-minded
young woman with proper principles. I do not venture to say that mine
would; it has suffered, and, it must be confessed, survived. But there
are some souls thus gently constituted, thus frail, and delicate, and
tender.

Whenever old John Sedley thought of the affair between George and
Amelia, or alluded to it, it was with bitterness almost as great as Mr.\ %
Osborne himself had shown.  He cursed Osborne and his family as
heartless, wicked, and ungrateful.  No power on earth, he swore, would
induce him to marry his daughter to the son of such a villain, and he
ordered Emmy to banish George from her mind, and to return all the
presents and letters which she had ever had from him.

She promised acquiescence, and tried to obey.  She put up the two or
three trinkets: and, as for the letters, she drew them out of the place
where she kept them; and read them over---as if she did not know them by
heart already: but she could not part with them.  That effort was too
much for her; she placed them back in her bosom again---as you have seen
a woman nurse a child that is dead.  Young Amelia felt that she would
die or lose her senses outright, if torn away from this last
consolation. How she used to blush and lighten up when those letters
came!  How she used to trip away with a beating heart, so that she
might read unseen!  If they were cold, yet how perversely this fond
little soul interpreted them into warmth.  If they were short or
selfish, what excuses she found for the writer!

It was over these few worthless papers that she brooded and brooded.
She lived in her past life---every letter seemed to recall some
circumstance of it.  How well she remembered them all!  His looks and
tones, his dress, what he said and how---these relics and remembrances
of dead affection were all that were left her in the world. And the
business of her life, was---to watch the corpse of Love.

To death she looked with inexpressible longing.  Then, she thought, I
shall always be able to follow him.  I am not praising her conduct or
setting her up as a model for Miss Bullock to imitate.  Miss B. knows
how to regulate her feelings better than this poor little creature.
Miss B. would never have committed herself as that imprudent Amelia had
done; pledged her love irretrievably; confessed her heart away, and got
back nothing---only a brittle promise which was snapt and worthless in a
moment.  A long engagement is a partnership which one party is free to
keep or to break, but which involves all the capital of the other.

Be cautious then, young ladies; be wary how you engage.  Be shy of
loving frankly; never tell all you feel, or (a better way still), feel
very little.  See the consequences of being prematurely honest and
confiding, and mistrust yourselves and everybody.  Get yourselves
married as they do in France, where the lawyers are the bridesmaids and
confidantes.  At any rate, never have any feelings which may make you
uncomfortable, or make any promises which you cannot at any required
moment command and withdraw.  That is the way to get on, and be
respected, and have a virtuous character in Vanity Fair.

If Amelia could have heard the comments regarding her which were made
in the circle from which her father's ruin had just driven her, she
would have seen what her own crimes were, and how entirely her
character was jeopardised.  Such criminal imprudence Mrs.\ Smith never
knew of; such horrid familiarities Mrs.\ Brown had always condemned, and
the end might be a warning to HER daughters. ``Captain Osborne, of
course, could not marry a bankrupt's daughter,'' the Misses Dobbin said.
``It was quite enough to have been swindled by the father.  As for that
little Amelia, her folly had really passed all---''

``All what?'' Captain Dobbin roared out.  ``Haven't they been engaged ever
since they were children?  Wasn't it as good as a marriage? Dare any
soul on earth breathe a word against the sweetest, the purest, the
tenderest, the most angelical of young women?''

``La, William, don't be so highty-tighty with US.  We're not men.  We
can't fight you,'' Miss Jane said.  ``We've said nothing against Miss
Sedley: but that her conduct throughout was MOST IMPRUDENT, not to call
it by any worse name; and that her parents are people who certainly
merit their misfortunes.''

``Hadn't you better, now that Miss Sedley is free, propose for her
yourself, William?'' Miss Ann asked sarcastically.  ``It would be a most
eligible family connection.  He!  he!''

``I marry her!'' Dobbin said, blushing very much, and talking quick. ``If
you are so ready, young ladies, to chop and change, do you suppose that
she is?  Laugh and sneer at that angel.  She can't hear it; and she's
miserable and unfortunate, and deserves to be laughed at.  Go on
joking, Ann.  You're the wit of the family, and the others like to hear
it.''

``I must tell you again we're not in a barrack, William,'' Miss Ann
remarked.

``In a barrack, by Jove---I wish anybody in a barrack would say what you
do,'' cried out this uproused British lion.  ``I should like to hear a
man breathe a word against her, by Jupiter.  But men don't talk in this
way, Ann: it's only women, who get together and hiss, and shriek, and
cackle.  There, get away---don't begin to cry.  I only said you were a
couple of geese,'' Will Dobbin said, perceiving Miss Ann's pink eyes
were beginning to moisten as usual.  ``Well, you're not geese, you're
swans---anything you like, only do, do leave Miss Sedley alone.''

Anything like William's infatuation about that silly little flirting,
ogling thing was never known, the mamma and sisters agreed together in
thinking: and they trembled lest, her engagement being off with
Osborne, she should take up immediately her other admirer and Captain.
In which forebodings these worthy young women no doubt judged according
to the best of their experience; or rather (for as yet they had had no
opportunities of marrying or of jilting) according to their own notions
of right and wrong.

``It is a mercy, Mamma, that the regiment is ordered abroad,'' the girls
said.  ``THIS danger, at any rate, is spared our brother.''

Such, indeed, was the fact; and so it is that the French Emperor comes
in to perform a part in this domestic comedy of Vanity Fair which we
are now playing, and which would never have been enacted without the
intervention of this august mute personage.  It was he that ruined the
Bourbons and Mr.\ John Sedley.  It was he whose arrival in his capital
called up all France in arms to defend him there; and all Europe to
oust him. While the French nation and army were swearing fidelity round
the eagles in the Champ de Mars, four mighty European hosts were
getting in motion for the great chasse a l'aigle; and one of these was
a British army, of which two heroes of ours, Captain Dobbin and Captain
Osborne, formed a portion.

The news of Napoleon's escape and landing was received by the gallant
---th with a fiery delight and enthusiasm, which everybody can
understand who knows that famous corps.  From the colonel to the
smallest drummer in the regiment, all were filled with hope and
ambition and patriotic fury; and thanked the French Emperor as for a
personal kindness in coming to disturb the peace of Europe.  Now was
the time the ---th had so long panted for, to show their comrades in
arms that they could fight as well as the Peninsular veterans, and that
all the pluck and valour of the ---th had not been killed by the West
Indies and the yellow fever.  Stubble and Spooney looked to get their
companies without purchase. Before the end of the campaign (which she
resolved to share), Mrs.\ Major O'Dowd hoped to write herself Mrs.\ %
Colonel O'Dowd, C.B.  Our two friends (Dobbin and Osborne) were quite
as much excited as the rest: and each in his way---Mr.\ Dobbin very
quietly, Mr.\ Osborne very loudly and energetically---was bent upon doing
his duty, and gaining his share of honour and distinction.

The agitation thrilling through the country and army in consequence of
this news was so great, that private matters were little heeded: and
hence probably George Osborne, just gazetted to his company, busy with
preparations for the march, which must come inevitably, and panting for
further promotion---was not so much affected by other incidents which
would have interested him at a more quiet period. He was not, it must
be confessed, very much cast down by good old Mr.\ Sedley's catastrophe.
He tried his new uniform, which became him very handsomely, on the day
when the first meeting of the creditors of the unfortunate gentleman
took place. His father told him of the wicked, rascally, shameful
conduct of the bankrupt, reminded him of what he had said about Amelia,
and that their connection was broken off for ever; and gave him that
evening a good sum of money to pay for the new clothes and epaulets in
which he looked so well.  Money was always useful to this free-handed
young fellow, and he took it without many words. The bills were up in
the Sedley house, where he had passed so many, many happy hours.  He
could see them as he walked from home that night (to the Old
Slaughters', where he put up when in town) shining white in the moon.
That comfortable home was shut, then, upon Amelia and her parents:
where had they taken refuge? The thought of their ruin affected him not
a little.  He was very melancholy that night in the coffee-room at the
Slaughters'; and drank a good deal, as his comrades remarked there.

Dobbin came in presently, cautioned him about the drink, which he only
took, he said, because he was deuced low; but when his friend began to
put to him clumsy inquiries, and asked him for news in a significant
manner, Osborne declined entering into conversation with him, avowing,
however, that he was devilish disturbed and unhappy.

Three days afterwards, Dobbin found Osborne in his room at the
barracks---his head on the table, a number of papers about, the young
Captain evidently in a state of great despondency.  ``She---she's sent me
back some things I gave her---some damned trinkets.  Look here!'' There
was a little packet directed in the well-known hand to Captain George
Osborne, and some things lying about---a ring, a silver knife he had
bought, as a boy, for her at a fair; a gold chain, and a locket with
hair in it.  ``It's all over,'' said he, with a groan of sickening
remorse. ``Look, Will, you may read it if you like.''

There was a little letter of a few lines, to which he pointed, which
said:

My papa has ordered me to return to you these presents, which you made
in happier days to me; and I am to write to you for the last time.  I
think, I know you feel as much as I do the blow which has come upon us.
It is I that absolve you from an engagement which is impossible in our
present misery.  I am sure you had no share in it, or in the cruel
suspicions of Mr.\ Osborne, which are the hardest of all our griefs to
bear.  Farewell. Farewell.  I pray God to strengthen me to bear this
and other calamities, and to bless you always.     A.

I shall often play upon the piano---your piano.  It was like you to send
it.

Dobbin was very soft-hearted.  The sight of women and children in pain
always used to melt him.  The idea of Amelia broken-hearted and lonely
tore that good-natured soul with anguish.  And he broke out into an
emotion, which anybody who likes may consider unmanly. He swore that
Amelia was an angel, to which Osborne said aye with all his heart.  He,
too, had been reviewing the history of their lives---and had seen her
from her childhood to her present age, so sweet, so innocent, so
charmingly simple, and artlessly fond and tender.

What a pang it was to lose all that: to have had it and not prized it!
A thousand homely scenes and recollections crowded on him---in which he
always saw her good and beautiful.  And for himself, he blushed with
remorse and shame, as the remembrance of his own selfishness and
indifference contrasted with that perfect purity. For a while, glory,
war, everything was forgotten, and the pair of friends talked about her
only.

``Where are they?'' Osborne asked, after a long talk, and a long
pause---and, in truth, with no little shame at thinking that he had
taken no steps to follow her.  ``Where are they? There's no address to
the note.''

Dobbin knew.  He had not merely sent the piano; but had written a note
to Mrs.\ Sedley, and asked permission to come and see her---and he had
seen her, and Amelia too, yesterday, before he came down to Chatham;
and, what is more, he had brought that farewell letter and packet which
had so moved them.

The good-natured fellow had found Mrs.\ Sedley only too willing to
receive him, and greatly agitated by the arrival of the piano, which,
as she conjectured, MUST have come from George, and was a signal of
amity on his part.  Captain Dobbin did not correct this error of the
worthy lady, but listened to all her story of complaints and
misfortunes with great sympathy---condoled with her losses and
privations, and agreed in reprehending the cruel conduct of Mr.\ Osborne
towards his first benefactor. When she had eased her overflowing bosom
somewhat, and poured forth many of her sorrows, he had the courage to
ask actually to see Amelia, who was above in her room as usual, and
whom her mother led trembling downstairs.

Her appearance was so ghastly, and her look of despair so pathetic,
that honest William Dobbin was frightened as he beheld it; and read the
most fatal forebodings in that pale fixed face.  After sitting in his
company a minute or two, she put the packet into his hand, and said,
``Take this to Captain Osborne, if you please, and---and I hope he's
quite well---and it was very kind of you to come and see us---and we like
our new house very much. And I---I think I'll go upstairs, Mamma, for
I'm not very strong.'' And with this, and a curtsey and a smile, the
poor child went her way.  The mother, as she led her up, cast back
looks of anguish towards Dobbin.  The good fellow wanted no such
appeal.  He loved her himself too fondly for that.  Inexpressible
grief, and pity, and terror pursued him, and he came away as if he was
a criminal after seeing her.

When Osborne heard that his friend had found her, he made hot and
anxious inquiries regarding the poor child.  How was she?  How did she
look?  What did she say?  His comrade took his hand, and looked him in
the face.

``George, she's dying,'' William Dobbin said---and could speak no more.

There was a buxom Irish servant-girl, who performed all the duties of
the little house where the Sedley family had found refuge: and this
girl had in vain, on many previous days, striven to give Amelia aid or
consolation. Emmy was much too sad to answer, or even to be aware of
the attempts the other was making in her favour.

Four hours after the talk between Dobbin and Osborne, this servant-maid
came into Amelia's room, where she sate as usual, brooding
silently over her letters---her little treasures.  The girl, smiling,
and looking arch and happy, made many trials to attract poor Emmy's
attention, who, however, took no heed of her.

``Miss Emmy,'' said the girl.

``I'm coming,'' Emmy said, not looking round.

``There's a message,'' the maid went on.  ``There's
something---somebody---sure, here's a new letter for you---don't be reading
them old ones any more.'' And she gave her a letter, which Emmy took, and
read.

``I must see you,'' the letter said.  ``Dearest Emmy---dearest
love---dearest wife, come to me.''

George and her mother were outside, waiting until she had read the
letter.



\chapter{Miss Crawley at Nurse}

We have seen how Mrs.\ Firkin, the lady's maid, as soon as any event of
importance to the Crawley family came to her knowledge, felt bound to
communicate it to Mrs.\ Bute Crawley, at the Rectory; and have before
mentioned how particularly kind and attentive that good-natured lady
was to Miss Crawley's confidential servant. She had been a gracious
friend to Miss Briggs, the companion, also; and had secured the
latter's good-will by a number of those attentions and promises, which
cost so little in the making, and are yet so valuable and agreeable to
the recipient.  Indeed every good economist and manager of a household
must know how cheap and yet how amiable these professions are, and what
a flavour they give to the most homely dish in life.  Who was the
blundering idiot who said that ``fine words butter no parsnips''?  Half
the parsnips of society are served and rendered palatable with no other
sauce.  As the immortal Alexis Soyer can make more delicious soup for a
half-penny than an ignorant cook can concoct with pounds of vegetables
and meat, so a skilful artist will make a few simple and pleasing
phrases go farther than ever so much substantial benefit-stock in the
hands of a mere bungler. Nay, we know that substantial benefits often
sicken some stomachs; whereas, most will digest any amount of fine
words, and be always eager for more of the same food. Mrs.\ Bute had
told Briggs and Firkin so often of the depth of her affection for them;
and what she would do, if she had Miss Crawley's fortune, for friends
so excellent and attached, that the ladies in question had the deepest
regard for her; and felt as much gratitude and confidence as if Mrs.\ %
Bute had loaded them with the most expensive favours.

Rawdon Crawley, on the other hand, like a selfish heavy dragoon as he
was, never took the least trouble to conciliate his aunt's aides-de-camp,
showed his contempt for the pair with entire frankness---made
Firkin pull off his boots on one occasion---sent her out in the rain on
ignominious messages---and if he gave her a guinea, flung it to her as
if it were a box on the ear.  As his aunt, too, made a butt of Briggs,
the Captain followed the example, and levelled his jokes at her---jokes
about as delicate as a kick from his charger. Whereas, Mrs.\ Bute
consulted her in matters of taste or difficulty, admired her poetry,
and by a thousand acts of kindness and politeness, showed her
appreciation of Briggs; and if she made Firkin a twopenny-halfpenny
present, accompanied it with so many compliments, that the
twopence-half-penny was transmuted into gold in the heart of the
grateful waiting-maid, who, besides, was looking forwards quite
contentedly to some prodigious benefit which must happen to her on the
day when Mrs.\ Bute came into her fortune.

The different conduct of these two people is pointed out respectfully
to the attention of persons commencing the world. Praise everybody, I
say to such: never be squeamish, but speak out your compliment both
point-blank in a man's face, and behind his back, when you know there
is a reasonable chance of his hearing it again.  Never lose a chance of
saying a kind word.  As Collingwood never saw a vacant place in his
estate but he took an acorn out of his pocket and popped it in; so deal
with your compliments through life.  An acorn costs nothing; but it may
sprout into a prodigious bit of timber.

In a word, during Rawdon Crawley's prosperity, he was only obeyed with
sulky acquiescence; when his disgrace came, there was nobody to help or
pity him.  Whereas, when Mrs.\ Bute took the command at Miss Crawley's
house, the garrison there were charmed to act under such a leader,
expecting all sorts of promotion from her promises, her generosity, and
her kind words.

That he would consider himself beaten, after one defeat, and make no
attempt to regain the position he had lost, Mrs.\ Bute Crawley never
allowed herself to suppose. She knew Rebecca to be too clever and
spirited and desperate a woman to submit without a struggle; and felt
that she must prepare for that combat, and be incessantly watchful
against assault; or mine, or surprise.

In the first place, though she held the town, was she sure of the
principal inhabitant?  Would Miss Crawley herself hold out; and had she
not a secret longing to welcome back the ousted adversary?  The old
lady liked Rawdon, and Rebecca, who amused her.  Mrs.\ Bute could not
disguise from herself the fact that none of her party could so
contribute to the pleasures of the town-bred lady.  ``My girls' singing,
after that little odious governess's, I know is unbearable,'' the candid
Rector's wife owned to herself.  ``She always used to go to sleep when
Martha and Louisa played their duets. Jim's stiff college manners and
poor dear Bute's talk about his dogs and horses always annoyed her.  If
I took her to the Rectory, she would grow angry with us all, and fly, I
know she would; and might fall into that horrid Rawdon's clutches
again, and be the victim of that little viper of a Sharp.  Meanwhile,
it is clear to me that she is exceedingly unwell, and cannot move for
some weeks, at any rate; during which we must think of some plan to
protect her from the arts of those unprincipled people.''

In the very best of moments, if anybody told Miss Crawley that she was,
or looked ill, the trembling old lady sent off for her doctor; and I
daresay she was very unwell after the sudden family event, which might
serve to shake stronger nerves than hers.  At least, Mrs.\ Bute thought
it was her duty to inform the physician, and the apothecary, and the
dame-de-compagnie, and the domestics, that Miss Crawley was in a most
critical state, and that they were to act accordingly.  She had the
street laid knee-deep with straw; and the knocker put by with Mr.\ %
Bowls's plate.  She insisted that the Doctor should call twice a day;
and deluged her patient with draughts every two hours.  When anybody
entered the room, she uttered a shshshsh so sibilant and ominous, that
it frightened the poor old lady in her bed, from which she could not
look without seeing Mrs.\ Bute's beady eyes eagerly fixed on her, as the
latter sate steadfast in the arm-chair by the bedside.  They seemed to
lighten in the dark (for she kept the curtains closed) as she moved
about the room on velvet paws like a cat.  There Miss Crawley lay for
days---ever so many days---Mr.\ Bute reading books of devotion to her: for
nights, long nights, during which she had to hear the watchman sing,
the night-light sputter; visited at midnight, the last thing, by the
stealthy apothecary; and then left to look at Mrs.\ Bute's twinkling
eyes, or the flicks of yellow that the rushlight threw on the dreary
darkened ceiling.  Hygeia herself would have fallen sick under such a
regimen; and how much more this poor old nervous victim?  It has been
said that when she was in health and good spirits, this venerable
inhabitant of Vanity Fair had as free notions about religion and morals
as Monsieur de Voltaire himself could desire, but when illness overtook
her, it was aggravated by the most dreadful terrors of death, and an
utter cowardice took possession of the prostrate old sinner.

Sick-bed homilies and pious reflections are, to be sure, out of place
in mere story-books, and we are not going (after the fashion of some
novelists of the present day) to cajole the public into a sermon, when
it is only a comedy that the reader pays his money to witness.  But,
without preaching, the truth may surely be borne in mind, that the
bustle, and triumph, and laughter, and gaiety which Vanity Fair
exhibits in public, do not always pursue the performer into private
life, and that the most dreary depression of spirits and dismal
repentances sometimes overcome him.  Recollection of the best ordained
banquets will scarcely cheer sick epicures. Reminiscences of the most
becoming dresses and brilliant ball triumphs will go very little way to
console faded beauties.  Perhaps statesmen, at a particular period of
existence, are not much gratified at thinking over the most triumphant
divisions; and the success or the pleasure of yesterday becomes of very
small account when a certain (albeit uncertain) morrow is in view,
about which all of us must some day or other be speculating.  O brother
wearers of motley!  Are there not moments when one grows sick of
grinning and tumbling, and the jingling of cap and bells?  This, dear
friends and companions, is my amiable object---to walk with you through
the Fair, to examine the shops and the shows there; and that we should
all come home after the flare, and the noise, and the gaiety, and be
perfectly miserable in private.

``If that poor man of mine had a head on his shoulders,'' Mrs.\ Bute
Crawley thought to herself, ``how useful he might be, under present
circumstances, to this unhappy old lady!  He might make her repent of
her shocking free-thinking ways; he might urge her to do her duty, and
cast off that odious reprobate who has disgraced himself and his
family; and he might induce her to do justice to my dear girls and the
two boys, who require and deserve, I am sure, every assistance which
their relatives can give them.''

And, as the hatred of vice is always a progress towards virtue, Mrs.\ %
Bute Crawley endeavoured to instil her sister-in-law a proper
abhorrence for all Rawdon Crawley's manifold sins: of which his uncle's
wife brought forward such a catalogue as indeed would have served to
condemn a whole regiment of young officers.  If a man has committed
wrong in life, I don't know any moralist more anxious to point his
errors out to the world than his own relations; so Mrs.\ Bute showed a
perfect family interest and knowledge of Rawdon's history.  She had all
the particulars of that ugly quarrel with Captain Marker, in which
Rawdon, wrong from the beginning, ended in shooting the Captain.  She
knew how the unhappy Lord Dovedale, whose mamma had taken a house at
Oxford, so that he might be educated there, and who had never touched a
card in his life till he came to London, was perverted by Rawdon at the
Cocoa-Tree, made helplessly tipsy by this abominable seducer and
perverter of youth, and fleeced of four thousand pounds.  She described
with the most vivid minuteness the agonies of the country families whom
he had ruined---the sons whom he had plunged into dishonour and
poverty---the daughters whom he had inveigled into perdition.  She knew
the poor tradesmen who were bankrupt by his extravagance---the mean
shifts and rogueries with which he had ministered to it---the astounding
falsehoods by which he had imposed upon the most generous of aunts, and
the ingratitude and ridicule by which he had repaid her sacrifices.
She imparted these stories gradually to Miss Crawley; gave her the
whole benefit of them; felt it to be her bounden duty as a Christian
woman and mother of a family to do so; had not the smallest remorse or
compunction for the victim whom her tongue was immolating; nay, very
likely thought her act was quite meritorious, and plumed herself upon
her resolute manner of performing it.  Yes, if a man's character is to
be abused, say what you will, there's nobody like a relation to do the
business.  And one is bound to own, regarding this unfortunate wretch
of a Rawdon Crawley, that the mere truth was enough to condemn him, and
that all inventions of scandal were quite superfluous pains on his
friends' parts.

Rebecca, too, being now a relative, came in for the fullest share of
Mrs.\ Bute's kind inquiries.  This indefatigable pursuer of truth
(having given strict orders that the door was to be denied to all
emissaries or letters from Rawdon), took Miss Crawley's carriage, and
drove to her old friend Miss Pinkerton, at Minerva House, Chiswick
Mall, to whom she announced the dreadful intelligence of Captain
Rawdon's seduction by Miss Sharp, and from whom she got sundry strange
particulars regarding the ex-governess's birth and early history.  The
friend of the Lexicographer had plenty of information to give.  Miss
Jemima was made to fetch the drawing-master's receipts and letters.
This one was from a spunging-house: that entreated an advance: another
was full of gratitude for Rebecca's reception by the ladies of
Chiswick: and the last document from the unlucky artist's pen was that
in which, from his dying bed, he recommended his orphan child to Miss
Pinkerton's protection. There were juvenile letters and petitions from
Rebecca, too, in the collection, imploring aid for her father or
declaring her own gratitude.  Perhaps in Vanity Fair there are no
better satires than letters.  Take a bundle of your dear friend's of
ten years back---your dear friend whom you hate now.  Look at a file of
your sister's! how you clung to each other till you quarrelled about
the twenty-pound legacy!  Get down the round-hand scrawls of your son
who has half broken your heart with selfish undutifulness since; or a
parcel of your own, breathing endless ardour and love eternal, which
were sent back by your mistress when she married the Nabob---your
mistress for whom you now care no more than for Queen Elizabeth. Vows,
love, promises, confidences, gratitude, how queerly they read after a
while!  There ought to be a law in Vanity Fair ordering the destruction
of every written document (except receipted tradesmen's bills) after a
certain brief and proper interval.  Those quacks and misanthropes who
advertise indelible Japan ink should be made to perish along with their
wicked discoveries.  The best ink for Vanity Fair use would be one that
faded utterly in a couple of days, and left the paper clean and blank,
so that you might write on it to somebody else.

From Miss Pinkerton's the indefatigable Mrs.\ Bute followed the track of
Sharp and his daughter back to the lodgings in Greek Street, which the
defunct painter had occupied; and where portraits of the landlady in
white satin, and of the husband in brass buttons, done by Sharp in lieu
of a quarter's rent, still decorated the parlour walls.  Mrs.\ Stokes
was a communicative person, and quickly told all she knew about Mr.\ %
Sharp; how dissolute and poor he was; how good-natured and amusing;
how he was always hunted by bailiffs and duns; how, to the landlady's
horror, though she never could abide the woman, he did not marry his
wife till a short time before her death; and what a queer little wild
vixen his daughter was; how she kept them all laughing with her fun and
mimicry; how she used to fetch the gin from the public-house, and was
known in all the studios in the quarter---in brief, Mrs.\ Bute got such a
full account of her new niece's parentage, education, and behaviour as
would scarcely have pleased Rebecca, had the latter known that such
inquiries were being made concerning her.

Of all these industrious researches Miss Crawley had the full benefit.
Mrs.\ Rawdon Crawley was the daughter of an opera-girl. She had danced
herself.  She had been a model to the painters.  She was brought up as
became her mother's daughter.  She drank gin with her father, \&c. \&c.
It was a lost woman who was married to a lost man; and the moral to be
inferred from Mrs.\ Bute's tale was, that the knavery of the pair was
irremediable, and that no properly conducted person should ever notice
them again.

These were the materials which prudent Mrs.\ Bute gathered together in
Park Lane, the provisions and ammunition as it were with which she
fortified the house against the siege which she knew that Rawdon and
his wife would lay to Miss Crawley.

But if a fault may be found with her arrangements, it is this, that she
was too eager: she managed rather too well; undoubtedly she made Miss
Crawley more ill than was necessary; and though the old invalid
succumbed to her authority, it was so harassing and severe, that the
victim would be inclined to escape at the very first chance which fell
in her way.  Managing women, the ornaments of their sex---women who
order everything for everybody, and know so much better than any person
concerned what is good for their neighbours, don't sometimes speculate
upon the possibility of a domestic revolt, or upon other extreme
consequences resulting from their overstrained authority.

Thus, for instance, Mrs.\ Bute, with the best intentions no doubt in the
world, and wearing herself to death as she did by foregoing sleep,
dinner, fresh air, for the sake of her invalid sister-in-law, carried
her conviction of the old lady's illness so far that she almost managed
her into her coffin.  She pointed out her sacrifices and their results
one day to the constant apothecary, Mr.\ Clump.

``I am sure, my dear Mr.\ Clump,'' she said, ``no efforts of mine have been
wanting to restore our dear invalid, whom the ingratitude of her nephew
has laid on the bed of sickness.  I never shrink from personal
discomfort: I never refuse to sacrifice myself.''

``Your devotion, it must be confessed, is admirable,'' Mr.\ Clump says,
with a low bow; ``but---''

``I have scarcely closed my eyes since my arrival: I give up sleep,
health, every comfort, to my sense of duty. When my poor James was in
the smallpox, did I allow any hireling to nurse him?  No.''

``You did what became an excellent mother, my dear Madam---the best of
mothers; but---''

``As the mother of a family and the wife of an English clergyman, I
humbly trust that my principles are good,'' Mrs.\ Bute said, with a happy
solemnity of conviction; ``and, as long as Nature supports me, never,
never, Mr.\ Clump, will I desert the post of duty.  Others may bring
that grey head with sorrow to the bed of sickness (here Mrs.\ Bute,
waving her hand, pointed to one of old Miss Crawley's coffee-coloured
fronts, which was perched on a stand in the dressing-room), but I will
never quit it. Ah, Mr.\ Clump!  I fear, I know, that the couch needs
spiritual as well as medical consolation.''

``What I was going to observe, my dear Madam,''---here the resolute Clump
once more interposed with a bland air---``what I was going to observe
when you gave utterance to sentiments which do you so much honour, was
that I think you alarm yourself needlessly about our kind friend, and
sacrifice your own health too prodigally in her favour.''

``I would lay down my life for my duty, or for any member of my
husband's family,'' Mrs.\ Bute interposed.

``Yes, Madam, if need were; but we don't want Mrs Bute Crawley to be a
martyr,'' Clump said gallantly.  ``Dr Squills and myself have both
considered Miss Crawley's case with every anxiety and care, as you may
suppose.  We see her low-spirited and nervous; family events have
agitated her.''

``Her nephew will come to perdition,'' Mrs.\ Crawley cried.

``Have agitated her: and you arrived like a guardian angel, my dear
Madam, a positive guardian angel, I assure you, to soothe her under the
pressure of calamity. But Dr. Squills and I were thinking that our
amiable friend is not in such a state as renders confinement to her bed
necessary.  She is depressed, but this confinement perhaps adds to her
depression.  She should have change, fresh air, gaiety; the most
delightful remedies in the pharmacopoeia,'' Mr.\ Clump said, grinning and
showing his handsome teeth.  ``Persuade her to rise, dear Madam; drag
her from her couch and her low spirits; insist upon her taking little
drives.  They will restore the roses too to your cheeks, if I may so
speak to Mrs.\ Bute Crawley.''

``The sight of her horrid nephew casually in the Park, where I am told
the wretch drives with the brazen partner of his crimes,'' Mrs.\ Bute
said (letting the cat of selfishness out of the bag of secrecy), ``would
cause her such a shock, that we should have to bring her back to bed
again.  She must not go out, Mr.\ Clump.  She shall not go out as long
as I remain to watch over her; And as for my health, what matters it?
I give it cheerfully, sir.  I sacrifice it at the altar of my duty.''

``Upon my word, Madam,'' Mr.\ Clump now said bluntly, ``I won't answer for
her life if she remains locked up in that dark room.  She is so nervous
that we may lose her any day; and if you wish Captain Crawley to be her
heir, I warn you frankly, Madam, that you are doing your very best to
serve him.''

``Gracious mercy! is her life in danger?'' Mrs.\ Bute cried.  ``Why, why,
Mr.\ Clump, did you not inform me sooner?''

The night before, Mr.\ Clump and Dr. Squills had had a consultation
(over a bottle of wine at the house of Sir Lapin Warren, whose lady was
about to present him with a thirteenth blessing), regarding Miss
Crawley and her case.

``What a little harpy that woman from Hampshire is, Clump,'' Squills
remarked, ``that has seized upon old Tilly Crawley.  Devilish good
Madeira.''

``What a fool Rawdon Crawley has been,'' Clump replied, ``to go and marry
a governess!  There was something about the girl, too.''

``Green eyes, fair skin, pretty figure, famous frontal development,''
Squills remarked.  ``There is something about her; and Crawley was a
fool, Squills.''

``A d--------- fool---always was,'' the apothecary replied.

``Of course the old girl will fling him over,'' said the physician, and
after a pause added, ``She'll cut up well, I suppose.''

``Cut up,'' says Clump with a grin; ``I wouldn't have her cut up for two
hundred a year.''

``That Hampshire woman will kill her in two months, Clump, my boy, if
she stops about her,'' Dr. Squills said. ``Old woman; full feeder;
nervous subject; palpitation of the heart; pressure on the brain;
apoplexy; off she goes. Get her up, Clump; get her out: or I wouldn't
give many weeks' purchase for your two hundred a year.'' And it was
acting upon this hint that the worthy apothecary spoke with so much
candour to Mrs.\ Bute Crawley.

Having the old lady under her hand: in bed: with nobody near, Mrs.\ Bute
had made more than one assault upon her, to induce her to alter her
will.  But Miss Crawley's usual terrors regarding death increased
greatly when such dismal propositions were made to her, and Mrs.\ Bute
saw that she must get her patient into cheerful spirits and health
before she could hope to attain the pious object which she had in view.
Whither to take her was the next puzzle. The only place where she is
not likely to meet those odious Rawdons is at church, and that won't
amuse her, Mrs.\ Bute justly felt.  ``We must go and visit our beautiful
suburbs of London,'' she then thought.  ``I hear they are the most
picturesque in the world''; and so she had a sudden interest for
Hampstead, and Hornsey, and found that Dulwich had great charms for
her, and getting her victim into her carriage, drove her to those
rustic spots, beguiling the little journeys with conversations about
Rawdon and his wife, and telling every story to the old lady which
could add to her indignation against this pair of reprobates.

Perhaps Mrs.\ Bute pulled the string unnecessarily tight. For though she
worked up Miss Crawley to a proper dislike of her disobedient nephew,
the invalid had a great hatred and secret terror of her victimizer, and
panted to escape from her.  After a brief space, she rebelled against
Highgate and Hornsey utterly.  She would go into the Park.  Mrs.\ Bute
knew they would meet the abominable Rawdon there, and she was right.
One day in the ring, Rawdon's stanhope came in sight; Rebecca was
seated by him.  In the enemy's equipage Miss Crawley occupied her usual
place, with Mrs.\ Bute on her left, the poodle and Miss Briggs on the
back seat.  It was a nervous moment, and Rebecca's heart beat quick as
she recognized the carriage; and as the two vehicles crossed each other
in a line, she clasped her hands, and looked towards the spinster with
a face of agonized attachment and devotion. Rawdon himself trembled,
and his face grew purple behind his dyed mustachios.  Only old Briggs
was moved in the other carriage, and cast her great eyes nervously
towards her old friends.  Miss Crawley's bonnet was resolutely turned
towards the Serpentine.  Mrs.\ Bute happened to be in ecstasies with the
poodle, and was calling him a little darling, and a sweet little zoggy,
and a pretty pet.  The carriages moved on, each in his line.

``Done, by Jove,'' Rawdon said to his wife.

``Try once more, Rawdon,'' Rebecca answered.  ``Could not you lock your
wheels into theirs, dearest?''

Rawdon had not the heart for that manoeuvre.  When the carriages met
again, he stood up in his stanhope; he raised his hand ready to doff
his hat; he looked with all his eyes.  But this time Miss Crawley's
face was not turned away; she and Mrs.\ Bute looked him full in the
face, and cut their nephew pitilessly.  He sank back in his seat with
an oath, and striking out of the ring, dashed away desperately
homewards.

It was a gallant and decided triumph for Mrs.\ Bute. But she felt the
danger of many such meetings, as she saw the evident nervousness of
Miss Crawley; and she determined that it was most necessary for her
dear friend's health, that they should leave town for a while, and
recommended Brighton very strongly.



\chapter{In Which Captain Dobbin Acts as the Messenger of Hymen}

Without knowing how, Captain William Dobbin found himself the great
promoter, arranger, and manager of the match between George Osborne and
Amelia.  But for him it never would have taken place:  he could not but
confess as much to himself, and smiled rather bitterly as he thought
that he of all men in the world should be the person upon whom the care
of this marriage had fallen. But though indeed the conducting of this
negotiation was about as painful a task as could be set to him, yet
when he had a duty to perform, Captain Dobbin was accustomed to go
through it without many words or much hesitation: and, having made up
his mind completely, that if Miss Sedley was balked of her husband she
would die of the disappointment, he was determined to use all his best
endeavours to keep her alive.

I forbear to enter into minute particulars of the interview between
George and Amelia, when the former was brought back to the feet (or
should we venture to say the arms?) of his young mistress by the
intervention of his friend honest William.  A much harder heart than
George's would have melted at the sight of that sweet face so sadly
ravaged by grief and despair, and at the simple tender accents in which
she told her little broken-hearted story: but as she did not faint when
her mother, trembling, brought Osborne to her; and as she only gave
relief to her overcharged grief, by laying her head on her lover's
shoulder and there weeping for a while the most tender, copious, and
refreshing tears---old Mrs.\ Sedley, too greatly relieved, thought it was
best to leave the young persons to themselves; and so quitted Emmy
crying over George's hand, and kissing it humbly, as if he were her
supreme chief and master, and as if she were quite a guilty and
unworthy person needing every favour and grace from him.

This prostration and sweet unrepining obedience exquisitely touched and
flattered George Osborne.  He saw a slave before him in that simple
yielding faithful creature, and his soul within him thrilled secretly
somehow at the knowledge of his power.  He would be generous-minded,
Sultan as he was, and raise up this kneeling Esther and make a queen of
her:  besides, her sadness and beauty touched him as much as her
submission, and so he cheered her, and raised her up and forgave her,
so to speak.  All her hopes and feelings, which were dying and
withering, this her sun having been removed from her, bloomed again and
at once, its light being restored. You would scarcely have recognised
the beaming little face upon Amelia's pillow that night as the one that
was laid there the night before, so wan, so lifeless, so careless of
all round about.  The honest Irish maid-servant, delighted with the
change, asked leave to kiss the face that had grown all of a sudden so
rosy.  Amelia put her arms round the girl's neck and kissed her with
all her heart, like a child.  She was little more.  She had that night
a sweet refreshing sleep, like one---and what a spring of inexpressible
happiness as she woke in the morning sunshine!

``He will be here again to-day,'' Amelia thought.  ``He is the greatest
and best of men.''  And the fact is, that George thought he was one of
the generousest creatures alive: and that he was making a tremendous
sacrifice in marrying this young creature.

While she and Osborne were having their delightful tete-a-tete above
stairs, old Mrs.\ Sedley and Captain Dobbin were conversing below upon
the state of the affairs, and the chances and future arrangements of
the young people.  Mrs.\ Sedley having brought the two lovers together
and left them embracing each other with all their might, like a true
woman, was of opinion that no power on earth would induce Mr.\ Sedley to
consent to the match between his daughter and the son of a man who had
so shamefully, wickedly, and monstrously treated him.  And she told a
long story about happier days and their earlier splendours, when
Osborne lived in a very humble way in the New Road, and his wife was
too glad to receive some of Jos's little baby things, with which Mrs.\ %
Sedley accommodated her at the birth of one of Osborne's own children.
The fiendish ingratitude of that man, she was sure, had broken Mr.\ S.'s
heart: and as for a marriage, he would never, never, never, never
consent.

``They must run away together, Ma'am,'' Dobbin said, laughing, ``and
follow the example of Captain Rawdon Crawley, and Miss Emmy's friend
the little governess.'' Was it possible? Well she never!  Mrs.\ Sedley
was all excitement about this news.  She wished that Blenkinsop were
here to hear it:  Blenkinsop always mistrusted that Miss Sharp.--- What
an escape Jos had had! and she described the already well-known
love-passages between Rebecca and the Collector of Boggley Wollah.

It was not, however, Mr.\ Sedley's wrath which Dobbin feared, so much as
that of the other parent concerned, and he owned that he had a very
considerable doubt and anxiety respecting the behaviour of the
black-browed old tyrant of a Russia merchant in Russell Square.  He has
forbidden the match peremptorily, Dobbin thought. He knew what a savage
determined man Osborne was, and how he stuck by his word. ``The only
chance George has of reconcilement,'' argued his friend, ``is by
distinguishing himself in the coming campaign.  If he dies they both go
together.  If he fails in distinction---what then?  He has some money
from his mother, I have heard enough to purchase his majority---or he
must sell out and go and dig in Canada, or rough it in a cottage in the
country.'' With such a partner Dobbin thought he would not mind
Siberia---and, strange to say, this absurd and utterly imprudent young
fellow never for a moment considered that the want of means to keep a
nice carriage and horses, and of an income which should enable its
possessors to entertain their friends genteelly, ought to operate as
bars to the union of George and Miss Sedley.

It was these weighty considerations which made him think too that the
marriage should take place as quickly as possible.  Was he anxious
himself, I wonder, to have it over?---as people, when death has
occurred, like to press forward the funeral, or when a parting is
resolved upon, hasten it.  It is certain that Mr.\ Dobbin, having taken
the matter in hand, was most extraordinarily eager in the conduct of
it.  He urged on George the necessity of immediate action:  he showed
the chances of reconciliation with his father, which a favourable
mention of his name in the Gazette must bring about.  If need were he
would go himself and brave both the fathers in the business.  At all
events, he besought George to go through with it before the orders
came, which everybody expected, for the departure of the regiment from
England on foreign service.

Bent upon these hymeneal projects, and with the applause and consent of
Mrs.\ Sedley, who did not care to break the matter personally to her
husband, Mr.\ Dobbin went to seek John Sedley at his house of call in
the City, the Tapioca Coffee-house, where, since his own offices were
shut up, and fate had overtaken him, the poor broken-down old
gentleman used to betake himself daily, and write letters and receive
them, and tie them up into mysterious bundles, several of which he
carried in the flaps of his coat.  I don't know anything more dismal
than that business and bustle and mystery of a ruined man:  those
letters from the wealthy which he shows you:  those worn greasy
documents promising support and offering condolence which he places
wistfully before you, and on which he builds his hopes of restoration
and future fortune. My beloved reader has no doubt in the course of his
experience been waylaid by many such a luckless companion.  He takes
you into the corner; he has his bundle of papers out of his gaping coat
pocket; and the tape off, and the string in his mouth, and the
favourite letters selected and laid before you; and who does not know
the sad eager half-crazy look which he fixes on you with his hopeless
eyes?

Changed into a man of this sort, Dobbin found the once florid, jovial,
and prosperous John Sedley.  His coat, that used to be so glossy and
trim, was white at the seams, and the buttons showed the copper.  His
face had fallen in, and was unshorn; his frill and neckcloth hung limp
under his bagging waistcoat.  When he used to treat the boys in old
days at a coffee-house, he would shout and laugh louder than anybody
there, and have all the waiters skipping round him; it was quite
painful to see how humble and civil he was to John of the Tapioca, a
blear-eyed old attendant in dingy stockings and cracked pumps, whose
business it was to serve glasses of wafers, and bumpers of ink in
pewter, and slices of paper to the frequenters of this dreary house of
entertainment, where nothing else seemed to be consumed.  As for
William Dobbin, whom he had tipped repeatedly in his youth, and who had
been the old gentleman's butt on a thousand occasions, old Sedley gave
his hand to him in a very hesitating humble manner now, and called him
``Sir.'' A feeling of shame and remorse took possession of William Dobbin
as the broken old man so received and addressed him, as if he himself
had been somehow guilty of the misfortunes which had brought Sedley so
low.

``I am very glad to see you, Captain Dobbin, sir,'' says he, after a
skulking look or two at his visitor (whose lanky figure and military
appearance caused some excitement likewise to twinkle in the blear eyes
of the waiter in the cracked dancing pumps, and awakened the old lady
in black, who dozed among the mouldy old coffee-cups in the bar).  ``How
is the worthy alderman, and my lady, your excellent mother, sir?''  He
looked round at the waiter as he said, ``My lady,'' as much as to say,
``Hark ye, John, I have friends still, and persons of rank and
reputation, too.''  ``Are you come to do anything in my way, sir?  My
young friends Dale and Spiggot do all my business for me now, until my
new offices are ready; for I'm only here temporarily, you know,
Captain.  What can we do for you, sir?  Will you like to take anything?''

Dobbin, with a great deal of hesitation and stuttering, protested that
he was not in the least hungry or thirsty; that he had no business to
transact; that he only came to ask if Mr.\ Sedley was well, and to shake
hands with an old friend; and, he added, with a desperate perversion of
truth, ``My mother is very well---that is, she's been very unwell, and is
only waiting for the first fine day to go out and call upon Mrs.\ %
Sedley.  How is Mrs.\ Sedley, sir?  I hope she's quite well.''  And here
he paused, reflecting on his own consummate hypocrisy; for the day was
as fine, and the sunshine as bright as it ever is in Coffin Court,
where the Tapioca Coffee-house is situated: and Mr.\ Dobbin remembered
that he had seen Mrs.\ Sedley himself only an hour before, having driven
Osborne down to Fulham in his gig, and left him there tete-a-tete with
Miss Amelia.

``My wife will be very happy to see her ladyship,'' Sedley replied,
pulling out his papers.  ``I've a very kind letter here from your
father, sir, and beg my respectful compliments to him.  Lady D. will
find us in rather a smaller house than we were accustomed to receive
our friends in; but it's snug, and the change of air does good to my
daughter, who was suffering in town rather---you remember little Emmy,
sir?---yes, suffering a good deal.'' The old gentleman's eyes were
wandering as he spoke, and he was thinking of something else, as he
sate thrumming on his papers and fumbling at the worn red tape.

``You're a military man,'' he went on; ``I ask you, Bill Dobbin, could any
man ever have speculated upon the return of that Corsican scoundrel
from Elba?  When the allied sovereigns were here last year, and we gave
'em that dinner in the City, sir, and we saw the Temple of Concord, and
the fireworks, and the Chinese bridge in St.\ James's Park, could any
sensible man suppose that peace wasn't really concluded, after we'd
actually sung Te Deum for it, sir?  I ask you, William, could I suppose
that the Emperor of Austria was a damned traitor---a traitor, and
nothing more?  I don't mince words---a double-faced infernal traitor and
schemer, who meant to have his son-in-law back all along.  And I say
that the escape of Boney from Elba was a damned imposition and plot,
sir, in which half the powers of Europe were concerned, to bring the
funds down, and to ruin this country.  That's why I'm here, William.
That's why my name's in the Gazette.  Why, sir?---because I trusted the
Emperor of Russia and the Prince Regent.  Look here.  Look at my
papers.  Look what the funds were on the 1st of March---what the French
fives were when I bought for the count.  And what they're at now.
There was collusion, sir, or that villain never would have escaped.
Where was the English Commissioner who allowed him to get away?  He
ought to be shot, sir---brought to a court-martial, and shot, by Jove.''

``We're going to hunt Boney out, sir,'' Dobbin said, rather alarmed at
the fury of the old man, the veins of whose forehead began to swell,
and who sate drumming his papers with his clenched fist.  ``We are going
to hunt him out, sir---the Duke's in Belgium already, and we expect
marching orders every day.''

``Give him no quarter.  Bring back the villain's head, sir. Shoot the
coward down, sir,'' Sedley roared.  ``I'd enlist myself, by---; but I'm a
broken old man---ruined by that damned scoundrel---and by a parcel of
swindling thieves in this country whom I made, sir, and who are rolling
in their carriages now,'' he added, with a break in his voice.

Dobbin was not a little affected by the sight of this once kind old
friend, crazed almost with misfortune and raving with senile anger.
Pity the fallen gentleman: you to whom money and fair repute are the
chiefest good; and so, surely, are they in Vanity Fair.

``Yes,'' he continued, ``there are some vipers that you warm, and they
sting you afterwards.  There are some beggars that you put on
horseback, and they're the first to ride you down.  You know whom I
mean, William Dobbin, my boy.  I mean a purse-proud villain in Russell
Square, whom I knew without a shilling, and whom I pray and hope to see
a beggar as he was when I befriended him.''

``I have heard something of this, sir, from my friend George,'' Dobbin
said, anxious to come to his point.  ``The quarrel between you and his
father has cut him up a great deal, sir.  Indeed, I'm the bearer of a
message from him.''

``O, THAT'S your errand, is it?'' cried the old man, jumping up. ``What!
perhaps he condoles with me, does he? Very kind of him, the
stiff-backed prig, with his dandified airs and West End swagger. He's
hankering about my house, is he still?  If my son had the courage of a
man, he'd shoot him.  He's as big a villain as his father.  I won't
have his name mentioned in my house.  I curse the day that ever I let
him into it; and I'd rather see my daughter dead at my feet than
married to him.''

``His father's harshness is not George's fault, sir.  Your daughter's
love for him is as much your doing as his.  Who are you, that you are
to play with two young people's affections and break their hearts at
your will?''

``Recollect it's not his father that breaks the match off,'' old Sedley
cried out.  ``It's I that forbid it.  That family and mine are separated
for ever.  I'm fallen low, but not so low as that: no, no. And so you
may tell the whole race---son, and father and sisters, and all.''

``It's my belief, sir, that you have not the power or the right to
separate those two,'' Dobbin answered in a low voice; ``and that if you
don't give your daughter your consent it will be her duty to marry
without it.  There's no reason she should die or live miserably because
you are wrong-headed.  To my thinking, she's just as much married as if
the banns had been read in all the churches in London.  And what better
answer can there be to Osborne's charges against you, as charges there
are, than that his son claims to enter your family and marry your
daughter?''

A light of something like satisfaction seemed to break over old Sedley
as this point was put to him: but he still persisted that with his
consent the marriage between Amelia and George should never take place.

``We must do it without,'' Dobbin said, smiling, and told Mr.\ Sedley, as
he had told Mrs.\ Sedley in the day, before, the story of Rebecca's
elopement with Captain Crawley.  It evidently amused the old gentleman.
``You're terrible fellows, you Captains,'' said he, tying up his papers;
and his face wore something like a smile upon it, to the astonishment
of the blear-eyed waiter who now entered, and had never seen such an
expression upon Sedley's countenance since he had used the dismal
coffee-house.

The idea of hitting his enemy Osborne such a blow soothed, perhaps, the
old gentleman: and, their colloquy presently ending, he and Dobbin
parted pretty good friends.

``My sisters say she has diamonds as big as pigeons' eggs,'' George said,
laughing.  ``How they must set off her complexion!  A perfect
illumination it must be when her jewels are on her neck.  Her
jet-black hair is as curly as Sambo's.  I dare say she wore a nose ring
when she went to court; and with a plume of feathers in her top-knot
she would look a perfect Belle Sauvage.''

George, in conversation with Amelia, was rallying the appearance of a
young lady of whom his father and sisters had lately made the
acquaintance, and who was an object of vast respect to the Russell
Square family.  She was reported to have I don't know how many
plantations in the West Indies; a deal of money in the funds; and three
stars to her name in the East India stockholders' list.  She had a
mansion in Surrey, and a house in Portland Place. The name of the rich
West India heiress had been mentioned with applause in the Morning
Post.  Mrs.\ Haggistoun, Colonel Haggistoun's widow, her relative,
``chaperoned'' her, and kept her house.  She was just from school, where
she had completed her education, and George and his sisters had met her
at an evening party at old Hulker's house, Devonshire Place (Hulker,
Bullock, and Co. were long the correspondents of her house in the West
Indies), and the girls had made the most cordial advances to her, which
the heiress had received with great good humour. An orphan in her
position---with her money---so interesting! the Misses Osborne said.
They were full of their new friend when they returned from the Hulker
ball to Miss Wirt, their companion; they had made arrangements for
continually meeting, and had the carriage and drove to see her the very
next day.  Mrs.\ Haggistoun, Colonel Haggistoun's widow, a relation of
Lord Binkie, and always talking of him, struck the dear unsophisticated
girls as rather haughty, and too much inclined to talk about her great
relations: but Rhoda was everything they could wish---the frankest,
kindest, most agreeable creature---wanting a little polish, but so
good-natured.  The girls Christian-named each other at once.

``You should have seen her dress for court, Emmy,'' Osborne cried,
laughing.  ``She came to my sisters to show it off, before she was
presented in state by my Lady Binkie, the Haggistoun's kinswoman. She's
related to every one, that Haggistoun.  Her diamonds blazed out like
Vauxhall on the night we were there.  (Do you remember Vauxhall, Emmy,
and Jos singing to his dearest diddle diddle darling?)  Diamonds and
mahogany, my dear! think what an advantageous contrast---and the white
feathers in her hair---I mean in her wool.  She had earrings like
chandeliers; you might have lighted 'em up, by Jove---and a yellow satin
train that streeled after her like the tail of a cornet.''

``How old is she?'' asked Emmy, to whom George was rattling away
regarding this dark paragon, on the morning of their reunion---rattling
away as no other man in the world surely could.

``Why the Black Princess, though she has only just left school, must be
two or three and twenty.  And you should see the hand she writes! Mrs.\ %
Colonel Haggistoun usually writes her letters, but in a moment of
confidence, she put pen to paper for my sisters; she spelt satin
satting, and Saint James's, Saint Jams.''

``Why, surely it must be Miss Swartz, the parlour boarder,'' Emmy said,
remembering that good-natured young mulatto girl, who had been so
hysterically affected when Amelia left Miss Pinkerton's academy.

``The very name,'' George said.  ``Her father was a German Jew---a
slave-owner they say---connected with the Cannibal Islands in some way
or other.  He died last year, and Miss Pinkerton has finished her
education.  She can play two pieces on the piano; she knows three
songs; she can write when Mrs.\ Haggistoun is by to spell for her; and
Jane and Maria already have got to love her as a sister.''

``I wish they would have loved me,'' said Emmy, wistfully. ``They were
always very cold to me.''

``My dear child, they would have loved you if you had had two hundred
thousand pounds,'' George replied.  ``That is the way in which they have
been brought up.  Ours is a ready-money society.  We live among bankers
and City big-wigs, and be hanged to them, and every man, as he talks to
you, is jingling his guineas in his pocket.  There is that jackass Fred
Bullock is going to marry Maria---there's Goldmore, the East India
Director, there's Dipley, in the tallow trade---OUR trade,'' George said,
with an uneasy laugh and a blush.  ``Curse the whole pack of
money-grubbing vulgarians!  I fall asleep at their great heavy dinners.
I feel ashamed in my father's great stupid parties.  I've been
accustomed to live with gentlemen, and men of the world and fashion,
Emmy, not with a parcel of turtle-fed tradesmen.  Dear little woman,
you are the only person of our set who ever looked, or thought, or
spoke like a lady: and you do it because you're an angel and can't help
it.  Don't remonstrate.  You are the only lady. Didn't Miss Crawley
remark it, who has lived in the best company in Europe?  And as for
Crawley, of the Life Guards, hang it, he's a fine fellow: and I like
him for marrying the girl he had chosen.''

Amelia admired Mr.\ Crawley very much, too, for this; and trusted
Rebecca would be happy with him, and hoped (with a laugh) Jos would be
consoled.  And so the pair went on prattling, as in quite early days.
Amelia's confidence being perfectly restored to her, though she
expressed a great deal of pretty jealousy about Miss Swartz, and
professed to be dreadfully frightened---like a hypocrite as she was---lest
George should forget her for the heiress and her money and her
estates in Saint Kitt's.  But the fact is, she was a great deal too
happy to have fears or doubts or misgivings of any sort: and having
George at her side again, was not afraid of any heiress or beauty, or
indeed of any sort of danger.

When Captain Dobbin came back in the afternoon to these people---which
he did with a great deal of sympathy for them---it did his heart good to
see how Amelia had grown young again---how she laughed, and chirped, and
sang familiar old songs at the piano, which were only interrupted by
the bell from without proclaiming Mr.\ Sedley's return from the City,
before whom George received a signal to retreat.

Beyond the first smile of recognition---and even that was an hypocrisy,
for she thought his arrival rather provoking---Miss Sedley did not once
notice Dobbin during his visit.  But he was content, so that he saw her
happy; and thankful to have been the means of making her so.



\chapter{A Quarrel About an Heiress}

Love may be felt for any young lady endowed with such qualities as Miss
Swartz possessed; and a great dream of ambition entered into old Mr.\ %
Osborne's soul, which she was to realize.  He encouraged, with the
utmost enthusiasm and friendliness, his daughters' amiable attachment
to the young heiress, and protested that it gave him the sincerest
pleasure as a father to see the love of his girls so well disposed.

``You won't find,'' he would say to Miss Rhoda, ``that splendour and rank
to which you are accustomed at the West End, my dear Miss, at our
humble mansion in Russell Square.  My daughters are plain,
disinterested girls, but their hearts are in the right place, and
they've conceived an attachment for you which does them honour---I say,
which does them honour.  I'm a plain, simple, humble British
merchant---an honest one, as my respected friends Hulker and Bullock
will vouch, who were the correspondents of your late lamented father.
You'll find us a united, simple, happy, and I think I may say
respected, family---a plain table, a plain people, but a warm welcome,
my dear Miss Rhoda---Rhoda, let me say, for my heart warms to you, it
does really.  I'm a frank man, and I like you.  A glass of Champagne!
Hicks, Champagne to Miss Swartz.''

There is little doubt that old Osborne believed all he said, and that
the girls were quite earnest in their protestations of affection for
Miss Swartz.  People in Vanity Fair fasten on to rich folks quite
naturally.  If the simplest people are disposed to look not a little
kindly on great Prosperity (for I defy any member of the British public
to say that the notion of Wealth has not something awful and pleasing
to him; and you, if you are told that the man next you at dinner has
got half a million, not to look at him with a certain interest)---if the
simple look benevolently on money, how much more do your old worldlings
regard it!  Their affections rush out to meet and welcome money.  Their
kind sentiments awaken spontaneously towards the interesting possessors
of it.  I know some respectable people who don't consider themselves at
liberty to indulge in friendship for any individual who has not a
certain competency, or place in society.  They give a loose to their
feelings on proper occasions.  And the proof is, that the major part of
the Osborne family, who had not, in fifteen years, been able to get up
a hearty regard for Amelia Sedley, became as fond of Miss Swartz in the
course of a single evening as the most romantic advocate of friendship
at first sight could desire.

What a match for George she'd be (the sisters and Miss Wirt agreed),
and how much better than that insignificant little Amelia!  Such a
dashing young fellow as he is, with his good looks, rank, and
accomplishments, would be the very husband for her.  Visions of balls
in Portland Place, presentations at Court, and introductions to half
the peerage, filled the minds of the young ladies; who talked of
nothing but George and his grand acquaintances to their beloved new
friend.

Old Osborne thought she would be a great match, too, for his son. He
should leave the army; he should go into Parliament; he should cut a
figure in the fashion and in the state.  His blood boiled with honest
British exultation, as he saw the name of Osborne ennobled in the
person of his son, and thought that he might be the progenitor of a
glorious line of baronets.  He worked in the City and on 'Change, until
he knew everything relating to the fortune of the heiress, how her
money was placed, and where her estates lay.  Young Fred Bullock, one
of his chief informants, would have liked to make a bid for her himself
(it was so the young banker expressed it), only he was booked to Maria
Osborne.  But not being able to secure her as a wife, the disinterested
Fred quite approved of her as a sister-in-law.  ``Let George cut in
directly and win her,'' was his advice.  ``Strike while the iron's hot,
you know---while she's fresh to the town: in a few weeks some d---------
fellow from the West End will come in with a title and a rotten
rent-roll and cut all us City men out, as Lord Fitzrufus did last year
with Miss Grogram, who was actually engaged to Podder, of Podder \&
Brown's.  The sooner it is done the better, Mr.\ Osborne; them's my
sentiments,'' the wag said; though, when Osborne had left the bank
parlour, Mr.\ Bullock remembered Amelia, and what a pretty girl she was,
and how attached to George Osborne; and he gave up at least ten seconds
of his valuable time to regretting the misfortune which had befallen
that unlucky young woman.

While thus George Osborne's good feelings, and his good friend and
genius, Dobbin, were carrying back the truant to Amelia's feet,
George's parent and sisters were arranging this splendid match for him,
which they never dreamed he would resist.

When the elder Osborne gave what he called ``a hint,'' there was no
possibility for the most obtuse to mistake his meaning.  He called
kicking a footman downstairs a hint to the latter to leave his service.
With his usual frankness and delicacy he told Mrs.\ Haggistoun that he
would give her a cheque for five thousand pounds on the day his son was
married to her ward; and called that proposal a hint, and considered it
a very dexterous piece of diplomacy.  He gave George finally such
another hint regarding the heiress; and ordered him to marry her out of
hand, as he would have ordered his butler to draw a cork, or his clerk
to write a letter.

This imperative hint disturbed George a good deal.  He was in the very
first enthusiasm and delight of his second courtship of Amelia, which
was inexpressibly sweet to him.  The contrast of her manners and
appearance with those of the heiress, made the idea of a union with the
latter appear doubly ludicrous and odious.  Carriages and opera-boxes,
thought he; fancy being seen in them by the side of such a mahogany
charmer as that!  Add to all that the junior Osborne was quite as
obstinate as the senior: when he wanted a thing, quite as firm in his
resolution to get it; and quite as violent when angered, as his father
in his most stern moments.

On the first day when his father formally gave him the hint that he was
to place his affections at Miss Swartz's feet, George temporised with
the old gentleman.  ``You should have thought of the matter sooner,
sir,'' he said. ``It can't be done now, when we're expecting every day to
go on foreign service.  Wait till my return, if I do return''; and then
he represented, that the time when the regiment was daily expecting to
quit England, was exceedingly ill-chosen: that the few days or weeks
during which they were still to remain at home, must be devoted to
business and not to love-making: time enough for that when he came home
with his majority; ``for, I promise you,'' said he, with a satisfied air,
``that one way or other you shall read the name of George Osborne in the
Gazette.''

The father's reply to this was founded upon the information which he
had got in the City: that the West End chaps would infallibly catch
hold of the heiress if any delay took place: that if he didn't marry
Miss S., he might at least have an engagement in writing, to come into
effect when he returned to England; and that a man who could get ten
thousand a year by staying at home, was a fool to risk his life abroad.

``So that you would have me shown up as a coward, sir, and our name
dishonoured for the sake of Miss Swartz's money,'' George interposed.

This remark staggered the old gentleman; but as he had to reply to it,
and as his mind was nevertheless made up, he said, ``You will dine here
to-morrow, sir, and every day Miss Swartz comes, you will be here to
pay your respects to her.  If you want for money, call upon Mr.\ %
Chopper.'' Thus a new obstacle was in George's way, to interfere with
his plans regarding Amelia; and about which he and Dobbin had more than
one confidential consultation.  His friend's opinion respecting the
line of conduct which he ought to pursue, we know already.  And as for
Osborne, when he was once bent on a thing, a fresh obstacle or two only
rendered him the more resolute.

The dark object of the conspiracy into which the chiefs of the Osborne
family had entered, was quite ignorant of all their plans regarding her
(which, strange to say, her friend and chaperon did not divulge), and,
taking all the young ladies' flattery for genuine sentiment, and being,
as we have before had occasion to show, of a very warm and impetuous
nature, responded to their affection with quite a tropical ardour.  And
if the truth may be told, I dare say that she too had some selfish
attraction in the Russell Square house; and in a word, thought George
Osborne a very nice young man. His whiskers had made an impression upon
her, on the very first night she beheld them at the ball at Messrs.
Hulkers; and, as we know, she was not the first woman who had been
charmed by them. George had an air at once swaggering and melancholy,
languid and fierce.  He looked like a man who had passions, secrets,
and private harrowing griefs and adventures.  His voice was rich and
deep.  He would say it was a warm evening, or ask his partner to take
an ice, with a tone as sad and confidential as if he were breaking her
mother's death to her, or preluding a declaration of love.  He trampled
over all the young bucks of his father's circle, and was the hero among
those third-rate men.  Some few sneered at him and hated him. Some,
like Dobbin, fanatically admired him.  And his whiskers had begun to do
their work, and to curl themselves round the affections of Miss Swartz.

Whenever there was a chance of meeting him in Russell Square, that
simple and good-natured young woman was quite in a flurry to see her
dear Misses Osborne.  She went to great expenses in new gowns, and
bracelets, and bonnets, and in prodigious feathers.  She adorned her
person with her utmost skill to please the Conqueror, and exhibited all
her simple accomplishments to win his favour.  The girls would ask her,
with the greatest gravity, for a little music, and she would sing her
three songs and play her two little pieces as often as ever they asked,
and with an always increasing pleasure to herself.  During these
delectable entertainments, Miss Wirt and the chaperon sate by, and
conned over the peerage, and talked about the nobility.

The day after George had his hint from his father, and a short time
before the hour of dinner, he was lolling upon a sofa in the
drawing-room in a very becoming and perfectly natural attitude of
melancholy.  He had been, at his father's request, to Mr.\ Chopper in
the City (the old gentleman, though he gave great sums to his son,
would never specify any fixed allowance for him, and rewarded him only
as he was in the humour).  He had then been to pass three hours with
Amelia, his dear little Amelia, at Fulham; and he came home to find his
sisters spread in starched muslin in the drawing-room, the dowagers
cackling in the background, and honest Swartz in her favourite
amber-coloured satin, with turquoise bracelets, countless rings,
flowers, feathers, and all sorts of tags and gimcracks, about as
elegantly decorated as a she chimney-sweep on May-day.

The girls, after vain attempts to engage him in conversation, talked
about fashions and the last drawing-room until he was perfectly sick of
their chatter.  He contrasted their behaviour with little Emmy's---their
shrill voices with her tender ringing tones; their attitudes and their
elbows and their starch, with her humble soft movements and modest
graces.  Poor Swartz was seated in a place where Emmy had been
accustomed to sit. Her bejewelled hands lay sprawling in her amber
satin lap.  Her tags and ear-rings twinkled, and her big eyes rolled
about.  She was doing nothing with perfect contentment, and thinking
herself charming.  Anything so becoming as the satin the sisters had
never seen.

``Dammy,'' George said to a confidential friend, ``she looked like a China
doll, which has nothing to do all day but to grin and wag its head.  By
Jove, Will, it was all I I could do to prevent myself from throwing the
sofa-cushion at her.'' He restrained that exhibition of sentiment,
however.

The sisters began to play the Battle of Prague.  ``Stop that d---------
thing,'' George howled out in a fury from the sofa.  ``It makes me mad.
You play us something, Miss Swartz, do.  Sing something, anything but
the Battle of Prague.''

``Shall I sing 'Blue Eyed Mary' or the air from the Cabinet?'' Miss
Swartz asked.

``That sweet thing from the Cabinet,'' the sisters said.

``We've had that,'' replied the misanthrope on the sofa.

``I can sing 'Fluvy du Tajy,''' Swartz said, in a meek voice, ``if I had
the words.'' It was the last of the worthy young woman's collection.

``O, 'Fleuve du Tage,''' Miss Maria cried; ``we have the song,'' and went
off to fetch the book in which it was.

Now it happened that this song, then in the height of the fashion, had
been given to the young ladies by a young friend of theirs, whose name
was on the title, and Miss Swartz, having concluded the ditty with
George's applause (for he remembered that it was a favourite of
Amelia's), was hoping for an encore perhaps, and fiddling with the
leaves of the music, when her eye fell upon the title, and she saw
``Amelia Sedley'' written in the comer.

``Lor!'' cried Miss Swartz, spinning swiftly round on the music-stool,
``is it my Amelia?  Amelia that was at Miss P.'s at Hammersmith?  I know
it is.  It's her, and---  Tell me about her---where is she?''

``Don't mention her,'' Miss Maria Osborne said hastily.  ``Her family has
disgraced itself.  Her father cheated Papa, and as for her, she is
never to be mentioned HERE.'' This was Miss Maria's return for George's
rudeness about the Battle of Prague.

``Are you a friend of Amelia's?'' George said, bouncing up.  ``God bless
you for it, Miss Swartz.  Don't believe what the girls say. SHE'S not
to blame at any rate. She's the best---''

``You know you're not to speak about her, George,'' cried Jane.  ``Papa
forbids it.''

``Who's to prevent me?'' George cried out.  ``I will speak of her.  I say
she's the best, the kindest, the gentlest, the sweetest girl in
England; and that, bankrupt or no, my sisters are not fit to hold
candles to her.  If you like her, go and see her, Miss Swartz; she
wants friends now; and I say, God bless everybody who befriends her.
Anybody who speaks kindly of her is my friend; anybody who speaks
against her is my enemy.  Thank you, Miss Swartz''; and he went up and
wrung her hand.

``George! George!'' one of the sisters cried imploringly.

``I say,'' George said fiercely, ``I thank everybody who loves Amelia
Sed---'' He stopped.  Old Osborne was in the room with a face livid with
rage, and eyes like hot coals.

Though George had stopped in his sentence, yet, his blood being up, he
was not to be cowed by all the generations of Osborne; rallying
instantly, he replied to the bullying look of his father, with another
so indicative of resolution and defiance that the elder man quailed in
his turn, and looked away.  He felt that the tussle was coming.  ``Mrs.\ %
Haggistoun, let me take you down to dinner,'' he said. ``Give your arm to
Miss Swartz, George,'' and they marched.

``Miss Swartz, I love Amelia, and we've been engaged almost all our
lives,'' Osborne said to his partner; and during all the dinner, George
rattled on with a volubility which surprised himself, and made his
father doubly nervous for the fight which was to take place as soon as
the ladies were gone.

The difference between the pair was, that while the father was violent
and a bully, the son had thrice the nerve and courage of the parent,
and could not merely make an attack, but resist it; and finding that
the moment was now come when the contest between him and his father was
to be decided, he took his dinner with perfect coolness and appetite
before the engagement began.  Old Osborne, on the contrary, was
nervous, and drank much.  He floundered in his conversation with the
ladies, his neighbours: George's coolness only rendering him more
angry.  It made him half mad to see the calm way in which George,
flapping his napkin, and with a swaggering bow, opened the door for the
ladies to leave the room; and filling himself a glass of wine, smacked
it, and looked his father full in the face, as if to say, ``Gentlemen of
the Guard, fire first.'' The old man also took a supply of ammunition,
but his decanter clinked against the glass as he tried to fill it.

After giving a great heave, and with a purple choking face, he then
began.  ``How dare you, sir, mention that person's name before Miss
Swartz to-day, in my drawing-room? I ask you, sir, how dare you do it?''

``Stop, sir,'' says George, ``don't say dare, sir.  Dare isn't a word to
be used to a Captain in the British Army.''

``I shall say what I like to my son, sir.  I can cut him off with a
shilling if I like.  I can make him a beggar if I like. I WILL say what
I like,'' the elder said.

``I'm a gentleman though I AM your son, sir,'' George answered haughtily.
``Any communications which you have to make to me, or any orders which
you may please to give, I beg may be couched in that kind of language
which I am accustomed to hear.''

Whenever the lad assumed his haughty manner, it always created either
great awe or great irritation in the parent.  Old Osborne stood in
secret terror of his son as a better gentleman than himself; and
perhaps my readers may have remarked in their experience of this Vanity
Fair of ours, that there is no character which a low-minded man so much
mistrusts as that of a gentleman.

``My father didn't give me the education you have had, nor the
advantages you have had, nor the money you have had.  If I had kept the
company SOME FOLKS have had through MY MEANS, perhaps my son wouldn't
have any reason to brag, sir, of his SUPERIORITY and WEST END AIRS
(these words were uttered in the elder Osborne's most sarcastic tones).
But it wasn't considered the part of a gentleman, in MY time, for a man
to insult his father. If I'd done any such thing, mine would have
kicked me downstairs, sir.''

``I never insulted you, sir.  I said I begged you to remember your son
was a gentleman as well as yourself. I know very well that you give me
plenty of money,'' said George (fingering a bundle of notes which he had
got in the morning from Mr.\ Chopper).  ``You tell it me often enough,
sir.  There's no fear of my forgetting it.''

``I wish you'd remember other things as well, sir,'' the sire answered.
``I wish you'd remember that in this house---so long as you choose to
HONOUR it with your COMPANY, Captain---I'm the master, and that name,
and that that---that you---that I say---''

``That what, sir?'' George asked, with scarcely a sneer, filling another
glass of claret.

``---------!'' burst out his father with a screaming oath---``that the name of
those Sedleys never be mentioned here, sir---not one of the whole damned
lot of 'em, sir.''

``It wasn't I, sir, that introduced Miss Sedley's name.  It was my
sisters who spoke ill of her to Miss Swartz; and by Jove I'll defend
her wherever I go.  Nobody shall speak lightly of that name in my
presence.  Our family has done her quite enough injury already, I
think, and may leave off reviling her now she's down.  I'll shoot any
man but you who says a word against her.''

``Go on, sir, go on,'' the old gentleman said, his eyes starting out of
his head.

``Go on about what, sir? about the way in which we've treated that angel
of a girl?  Who told me to love her?  It was your doing.  I might have
chosen elsewhere, and looked higher, perhaps, than your society: but I
obeyed you.  And now that her heart's mine you give me orders to fling
it away, and punish her, kill her perhaps---for the faults of other
people.  It's a shame, by Heavens,'' said George, working himself up
into passion and enthusiasm as he proceeded, ``to play at fast and loose
with a young girl's affections---and with such an angel as that---one so
superior to the people amongst whom she lived, that she might have
excited envy, only she was so good and gentle, that it's a wonder
anybody dared to hate her. If I desert her, sir, do you suppose she
forgets me?''

``I ain't going to have any of this dam sentimental nonsense and humbug
here, sir,'' the father cried out.  ``There shall be no beggar-marriages
in my family.  If you choose to fling away eight thousand a year, which
you may have for the asking, you may do it: but by Jove you take your
pack and walk out of this house, sir.  Will you do as I tell you, once
for all, sir, or will you not?''

``Marry that mulatto woman?'' George said, pulling up his shirt-collars.
``I don't like the colour, sir.  Ask the black that sweeps opposite
Fleet Market, sir.  I'm not going to marry a Hottentot Venus.''

Mr.\ Osborne pulled frantically at the cord by which he was accustomed
to summon the butler when he wanted wine---and almost black in the face,
ordered that functionary to call a coach for Captain Osborne.

``I've done it,'' said George, coming into the Slaughters' an hour
afterwards, looking very pale.

``What, my boy?'' says Dobbin.

George told what had passed between his father and himself.

``I'll marry her to-morrow,'' he said with an oath.  ``I love her more
every day, Dobbin.''



\chapter{A Marriage and Part of a Honeymoon}

Enemies the most obstinate and courageous can't hold out against
starvation; so the elder Osborne felt himself pretty easy about his
adversary in the encounter we have just described; and as soon as
George's supplies fell short, confidently expected his unconditional
submission. It was unlucky, to be sure, that the lad should have
secured a stock of provisions on the very day when the first encounter
took place; but this relief was only temporary, old Osborne thought,
and would but delay George's surrender.  No communication passed
between father and son for some days.  The former was sulky at this
silence, but not disquieted; for, as he said, he knew where he could
put the screw upon George, and only waited the result of that
operation.  He told the sisters the upshot of the dispute between them,
but ordered them to take no notice of the matter, and welcome George on
his return as if nothing had happened.  His cover was laid as usual
every day, and perhaps the old gentleman rather anxiously expected him;
but he never came. Some one inquired at the Slaughters' regarding him,
where it was said that he and his friend Captain Dobbin had left town.

One gusty, raw day at the end of April---the rain whipping the pavement
of that ancient street where the old Slaughters' Coffee-house was once
situated---George Osborne came into the coffee-room, looking very
haggard and pale; although dressed rather smartly in a blue coat and
brass buttons, and a neat buff waistcoat of the fashion of those days.
Here was his friend Captain Dobbin, in blue and brass too, having
abandoned the military frock and French-grey trousers, which were the
usual coverings of his lanky person.

Dobbin had been in the coffee-room for an hour or more.  He had tried
all the papers, but could not read them.  He had looked at the clock
many scores of times; and at the street, where the rain was pattering
down, and the people as they clinked by in pattens, left long
reflections on the shining stone: he tattooed at the table: he bit his
nails most completely, and nearly to the quick (he was accustomed to
ornament his great big hands in this way): he balanced the tea-spoon
dexterously on the milk jug: upset it, \&c., \&c.; and in fact showed
those signs of disquietude, and practised those desperate attempts at
amusement, which men are accustomed to employ when very anxious, and
expectant, and perturbed in mind.

Some of his comrades, gentlemen who used the room, joked him about the
splendour of his costume and his agitation of manner.  One asked him if
he was going to be married?  Dobbin laughed, and said he would send his
acquaintance (Major Wagstaff of the Engineers) a piece of cake when
that event took place.  At length Captain Osborne made his appearance,
very smartly dressed, but very pale and agitated as we have said.  He
wiped his pale face with a large yellow bandanna pocket-handkerchief
that was prodigiously scented. He shook hands with Dobbin, looked at
the clock, and told John, the waiter, to bring him some curacao.  Of
this cordial he swallowed off a couple of glasses with nervous
eagerness. His friend asked with some interest about his health.

``Couldn't get a wink of sleep till daylight, Dob,'' said he. ``Infernal
headache and fever.  Got up at nine, and went down to the Hummums for a
bath.  I say, Dob, I feel just as I did on the morning I went out with
Rocket at Quebec.''

``So do I,'' William responded.  ``I was a deuced deal more nervous than
you were that morning.  You made a famous breakfast, I remember.  Eat
something now.''

``You're a good old fellow, Will.  I'll drink your health, old boy, and
farewell to---''

``No, no; two glasses are enough,'' Dobbin interrupted him.  ``Here, take
away the liqueurs, John.  Have some cayenne-pepper with your fowl.
Make haste though, for it is time we were there.''

It was about half an hour from twelve when this brief meeting and
colloquy took place between the two captains.  A coach, into which
Captain Osborne's servant put his master's desk and dressing-case, had
been in waiting for some time; and into this the two gentlemen hurried
under an umbrella, and the valet mounted on the box, cursing the rain
and the dampness of the coachman who was steaming beside him.  ``We
shall find a better trap than this at the church-door,'' says he;
``that's a comfort.'' And the carriage drove on, taking the road down
Piccadilly, where Apsley House and St.\ George's Hospital wore red
jackets still; where there were oil-lamps; where Achilles was not yet
born; nor the Pimlico arch raised; nor the hideous equestrian monster
which pervades it and the neighbourhood; and so they drove down by
Brompton to a certain chapel near the Fulham Road there.

A chariot was in waiting with four horses; likewise a coach of the kind
called glass coaches.  Only a very few idlers were collected on account
of the dismal rain.

``Hang it!'' said George, ``I said only a pair.''

``My master would have four,'' said Mr.\ Joseph Sedley's servant, who was
in waiting; and he and Mr.\ Osborne's man agreed as they followed George
and William into the church, that it was a ``reg'lar shabby turn hout;
and with scarce so much as a breakfast or a wedding faviour.''

``Here you are,'' said our old friend, Jos Sedley, coming forward.
``You're five minutes late, George, my boy. What a day, eh? Demmy, it's
like the commencement of the rainy season in Bengal.  But you'll find
my carriage is watertight.  Come along, my mother and Emmy are in the
vestry.''

Jos Sedley was splendid.  He was fatter than ever.  His shirt collars
were higher; his face was redder; his shirt-frill flaunted gorgeously
out of his variegated waistcoat. Varnished boots were not invented as
yet; but the Hessians on his beautiful legs shone so, that they must
have been the identical pair in which the gentleman in the old picture
used to shave himself; and on his light green coat there bloomed a fine
wedding favour, like a great white spreading magnolia.

In a word, George had thrown the great cast.  He was going to be
married.  Hence his pallor and nervousness---his sleepless night and
agitation in the morning.  I have heard people who have gone through
the same thing own to the same emotion.  After three or four
ceremonies, you get accustomed to it, no doubt; but the first dip,
everybody allows, is awful.

The bride was dressed in a brown silk pelisse (as Captain Dobbin has
since informed me), and wore a straw bonnet with a pink ribbon; over
the bonnet she had a veil of white Chantilly lace, a gift from Mr.\ %
Joseph Sedley, her brother.  Captain Dobbin himself had asked leave to
present her with a gold chain and watch, which she sported on this
occasion; and her mother gave her her diamond brooch---almost the only
trinket which was left to the old lady.  As the service went on, Mrs.\ %
Sedley sat and whimpered a great deal in a pew, consoled by the Irish
maid-servant and Mrs.\ Clapp from the lodgings. Old Sedley would not be
present.  Jos acted for his father, giving away the bride, whilst
Captain Dobbin stepped up as groomsman to his friend George.

There was nobody in the church besides the officiating persons and the
small marriage party and their attendants. The two valets sat aloof
superciliously.  The rain came rattling down on the windows. In the
intervals of the service you heard it, and the sobbing of old Mrs.\ %
Sedley in the pew.  The parson's tones echoed sadly through the empty
walls.  Osborne's ``I will'' was sounded in very deep bass. Emmy's
response came fluttering up to her lips from her heart, but was
scarcely heard by anybody except Captain Dobbin.

When the service was completed, Jos Sedley came forward and kissed his
sister, the bride, for the first time for many months---George's look of
gloom had gone, and he seemed quite proud and radiant. ``It's your turn,
William,'' says he, putting his hand fondly upon Dobbin's shoulder; and
Dobbin went up and touched Amelia on the cheek.

Then they went into the vestry and signed the register. ``God bless you,
Old Dobbin,'' George said, grasping him by the hand, with something very
like moisture glistening in his eyes.  William replied only by nodding
his head. His heart was too full to say much.

``Write directly, and come down as soon as you can, you know,'' Osborne
said.  After Mrs.\ Sedley had taken an hysterical adieu of her daughter,
the pair went off to the carriage.  ``Get out of the way, you little
devils,'' George cried to a small crowd of damp urchins, that were
hanging about the chapel-door.  The rain drove into the bride and
bridegroom's faces as they passed to the chariot. The postilions'
favours draggled on their dripping jackets. The few children made a
dismal cheer, as the carriage, splashing mud, drove away.

William Dobbin stood in the church-porch, looking at it, a queer
figure.  The small crew of spectators jeered him. He was not thinking
about them or their laughter.

``Come home and have some tiffin, Dobbin,'' a voice cried behind him; as
a pudgy hand was laid on his shoulder, and the honest fellow's reverie
was interrupted.  But the Captain had no heart to go a-feasting with
Jos Sedley. He put the weeping old lady and her attendants into the
carriage along with Jos, and left them without any farther words
passing.  This carriage, too, drove away, and the urchins gave another
sarcastical cheer.

``Here, you little beggars,'' Dobbin said, giving some sixpences amongst
them, and then went off by himself through the rain.  It was all over.
They were married, and happy, he prayed God.  Never since he was a boy
had he felt so miserable and so lonely.  He longed with a heart-sick
yearning for the first few days to be over, that he might see her again.

Some ten days after the above ceremony, three young men of our
acquaintance were enjoying that beautiful prospect of bow windows on
the one side and blue sea on the other, which Brighton affords to the
traveller. Sometimes it is towards the ocean---smiling with countless
dimples, speckled with white sails, with a hundred bathing-machines
kissing the skirt of his blue garment---that the Londoner looks
enraptured: sometimes, on the contrary, a lover of human nature rather
than of prospects of any kind, it is towards the bow windows that he
turns, and that swarm of human life which they exhibit.  From one issue
the notes of a piano, which a young lady in ringlets practises six
hours daily, to the delight of the fellow-lodgers: at another, lovely
Polly, the nurse-maid, may be seen dandling Master Omnium in her arms:
whilst Jacob, his papa, is beheld eating prawns, and devouring the
Times for breakfast, at the window below. Yonder are the Misses Leery,
who are looking out for the young officers of the Heavies, who are
pretty sure to be pacing the cliff; or again it is a City man, with a
nautical turn, and a telescope, the size of a six-pounder, who has his
instrument pointed seawards, so as to command every pleasure-boat,
herring-boat, or bathing-machine that comes to, or quits, the shore,
\&c., \&c.  But have we any leisure for a description of Brighton?---for
Brighton, a clean Naples with genteel lazzaroni---for Brighton, that
always looks brisk, gay, and gaudy, like a harlequin's jacket---for
Brighton, which used to be seven hours distant from London at the time
of our story; which is now only a hundred minutes off; and which may
approach who knows how much nearer, unless Joinville comes and untimely
bombards it?

``What a monstrous fine girl that is in the lodgings over the
milliner's,'' one of these three promenaders remarked to the other;
``Gad, Crawley, did you see what a wink she gave me as I passed?''

``Don't break her heart, Jos, you rascal,'' said another. ``Don't trifle
with her affections, you Don Juan!''

``Get away,'' said Jos Sedley, quite pleased, and leering up at the
maid-servant in question with a most killing ogle.  Jos was even more
splendid at Brighton than he had been at his sister's marriage. He had
brilliant under-waistcoats, any one of which would have set up a
moderate buck. He sported a military frock-coat, ornamented with frogs,
knobs, black buttons, and meandering embroidery. He had affected a
military appearance and habits of late; and he walked with his two
friends, who were of that profession, clinking his boot-spurs,
swaggering prodigiously, and shooting death-glances at all the servant
girls who were worthy to be slain.

``What shall we do, boys, till the ladies return?'' the buck asked. The
ladies were out to Rottingdean in his carriage on a drive.

``Let's have a game at billiards,'' one of his friends said---the tall
one, with lacquered mustachios.

``No, dammy; no, Captain,'' Jos replied, rather alarmed.  ``No billiards
to-day, Crawley, my boy; yesterday was enough.''

``You play very well,'' said Crawley, laughing.  ``Don't he, Osborne? How
well he made that five stroke, eh?''

``Famous,'' Osborne said.  ``Jos is a devil of a fellow at billiards, and
at everything else, too.  I wish there were any tiger-hunting about
here! we might go and kill a few before dinner.  (There goes a fine
girl! what an ankle, eh, Jos?) Tell us that story about the tiger-hunt,
and the way you did for him in the jungle---it's a wonderful story that,
Crawley.'' Here George Osborne gave a yawn. ``It's rather slow work,''
said he, ``down here; what shall we do?''

``Shall we go and look at some horses that Snaffler's just brought from
Lewes fair?'' Crawley said.

``Suppose we go and have some jellies at Dutton's,'' and the rogue Jos,
willing to kill two birds with one stone.  ``Devilish fine gal at
Dutton's.''

``Suppose we go and see the Lightning come in, it's just about time?''
George said.  This advice prevailing over the stables and the jelly,
they turned towards the coach-office to witness the Lightning's arrival.

As they passed, they met the carriage---Jos Sedley's open carriage, with
its magnificent armorial bearings---that splendid conveyance in which he
used to drive, about at Cheltenham, majestic and solitary, with his
arms folded, and his hat cocked; or, more happy, with ladies by his
side.

Two were in the carriage now: one a little person, with light hair, and
dressed in the height of the fashion; the other in a brown silk
pelisse, and a straw bonnet with pink ribbons, with a rosy, round,
happy face, that did you good to behold.  She checked the carriage as
it neared the three gentlemen, after which exercise of authority she
looked rather nervous, and then began to blush most absurdly. ``We have
had a delightful drive, George,'' she said, ``and---and we're so glad to
come back; and, Joseph, don't let him be late.''

``Don't be leading our husbands into mischief, Mr.\ Sedley, you wicked,
wicked man you,'' Rebecca said, shaking at Jos a pretty little finger
covered with the neatest French kid glove.  ``No billiards, no smoking,
no naughtiness!''

``My dear Mrs.\ Crawley---Ah now! upon my honour!'' was all Jos could
ejaculate by way of reply; but he managed to fall into a tolerable
attitude, with his head lying on his shoulder, grinning upwards at his
victim, with one hand at his back, which he supported on his cane, and
the other hand (the one with the diamond ring) fumbling in his
shirt-frill and among his under-waistcoats.  As the carriage drove off
he kissed the diamond hand to the fair ladies within.  He wished all
Cheltenham, all Chowringhee, all Calcutta, could see him in that
position, waving his hand to such a beauty, and in company with such a
famous buck as Rawdon Crawley of the Guards.

Our young bride and bridegroom had chosen Brighton as the place where
they would pass the first few days after their marriage; and having
engaged apartments at the Ship Inn, enjoyed themselves there in great
comfort and quietude, until Jos presently joined them.  Nor was he the
only companion they found there.  As they were coming into the hotel
from a sea-side walk one afternoon, on whom should they light but
Rebecca and her husband.  The recognition was immediate.  Rebecca flew
into the arms of her dearest friend. Crawley and Osborne shook hands
together cordially enough: and Becky, in the course of a very few
hours, found means to make the latter forget that little unpleasant
passage of words which had happened between them.  ``Do you remember the
last time we met at Miss Crawley's, when I was so rude to you, dear
Captain Osborne? I thought you seemed careless about dear Amelia.  It
was that made me angry: and so pert: and so unkind: and so ungrateful.
Do forgive me!'' Rebecca said, and she held out her hand with so frank
and winning a grace, that Osborne could not but take it.  By humbly and
frankly acknowledging yourself to be in the wrong, there is no knowing,
my son, what good you may do.  I knew once a gentleman and very worthy
practitioner in Vanity Fair, who used to do little wrongs to his
neighbours on purpose, and in order to apologise for them in an open
and manly way afterwards---and what ensued?  My friend Crocky Doyle was
liked everywhere, and deemed to be rather impetuous---but the honestest
fellow.  Becky's humility passed for sincerity with George Osborne.

These two young couples had plenty of tales to relate to each other.
The marriages of either were discussed; and their prospects in life
canvassed with the greatest frankness and interest on both sides.
George's marriage was to be made known to his father by his friend
Captain Dobbin; and young Osborne trembled rather for the result of
that communication.  Miss Crawley, on whom all Rawdon's hopes depended,
still held out.  Unable to make an entry into her house in Park Lane,
her affectionate nephew and niece had followed her to Brighton, where
they had emissaries continually planted at her door.

``I wish you could see some of Rawdon's friends who are always about our
door,'' Rebecca said, laughing.  ``Did you ever see a dun, my dear; or a
bailiff and his man? Two of the abominable wretches watched all last
week at the greengrocer's opposite, and we could not get away until
Sunday.  If Aunty does not relent, what shall we do?''

Rawdon, with roars of laughter, related a dozen amusing anecdotes of
his duns, and Rebecca's adroit treatment of them.  He vowed with a
great oath that there was no woman in Europe who could talk a creditor
over as she could.  Almost immediately after their marriage, her
practice had begun, and her husband found the immense value of such a
wife.  They had credit in plenty, but they had bills also in abundance,
and laboured under a scarcity of ready money. Did these
debt-difficulties affect Rawdon's good spirits?  No. Everybody in
Vanity Fair must have remarked how well those live who are comfortably
and thoroughly in debt: how they deny themselves nothing; how jolly and
easy they are in their minds.  Rawdon and his wife had the very best
apartments at the inn at Brighton; the landlord, as he brought in the
first dish, bowed before them as to his greatest customers: and Rawdon
abused the dinners and wine with an audacity which no grandee in the
land could surpass.  Long custom, a manly appearance, faultless boots
and clothes, and a happy fierceness of manner, will often help a man as
much as a great balance at the banker's.

The two wedding parties met constantly in each other's apartments.
After two or three nights the gentlemen of an evening had a little
piquet, as their wives sate and chatted apart.  This pastime, and the
arrival of Jos Sedley, who made his appearance in his grand open
carriage, and who played a few games at billiards with Captain Crawley,
replenished Rawdon's purse somewhat, and gave him the benefit of that
ready money for which the greatest spirits are sometimes at a
stand-still.

So the three gentlemen walked down to see the Lightning coach come in.
Punctual to the minute, the coach crowded inside and out, the guard
blowing his accustomed tune on the horn---the Lightning came tearing
down the street, and pulled up at the coach-office.

``Hullo! there's old Dobbin,'' George cried, quite delighted to see his
old friend perched on the roof; and whose promised visit to Brighton
had been delayed until now.  ``How are you, old fellow? Glad you're come
down. Emmy'll be delighted to see you,'' Osborne said, shaking his
comrade warmly by the hand as soon as his descent from the vehicle was
effected---and then he added, in a lower and agitated voice, ``What's the
news?  Have you been in Russell Square? What does the governor say?
Tell me everything.''

Dobbin looked very pale and grave.  ``I've seen your father,'' said he.
``How's Amelia---Mrs.\ George?  I'll tell you all the news presently: but
I've brought the great news of all: and that is---''

``Out with it, old fellow,'' George said.

``We're ordered to Belgium.  All the army goes---guards and all.
Heavytop's got the gout, and is mad at not being able to move. O'Dowd
goes in command, and we embark from Chatham next week.'' This news of
war could not but come with a shock upon our lovers, and caused all
these gentlemen to look very serious.



\chapter{Captain Dobbin Proceeds on His Canvass}

What is the secret mesmerism which friendship possesses, and under the
operation of which a person ordinarily sluggish, or cold, or timid,
becomes wise, active, and resolute, in another's behalf?  As Alexis,
after a few passes from Dr. Elliotson, despises pain, reads with the
back of his head, sees miles off, looks into next week, and performs
other wonders, of which, in his own private normal condition, he is
quite incapable; so you see, in the affairs of the world and under the
magnetism of friendships, the modest man becomes bold, the shy
confident, the lazy active, or the impetuous prudent and peaceful.
What is it, on the other hand, that makes the lawyer eschew his own
cause, and call in his learned brother as an adviser? And what causes
the doctor, when ailing, to send for his rival, and not sit down and
examine his own tongue in the chimney glass, or write his own
prescription at his study-table?  I throw out these queries for
intelligent readers to answer, who know, at once, how credulous we are,
and how sceptical, how soft and how obstinate, how firm for others and
how diffident about ourselves:  meanwhile, it is certain that our
friend William Dobbin, who was personally of so complying a disposition
that if his parents had pressed him much, it is probable he would have
stepped down into the kitchen and married the cook, and who, to further
his own interests, would have found the most insuperable difficulty in
walking across the street, found himself as busy and eager in the
conduct of George Osborne's affairs, as the most selfish tactician
could be in the pursuit of his own.

Whilst our friend George and his young wife were enjoying the first
blushing days of the honeymoon at Brighton, honest William was left as
George's plenipotentiary in London, to transact all the business part
of the marriage. His duty it was to call upon old Sedley and his wife,
and to keep the former in good humour:  to draw Jos and his
brother-in-law nearer together, so that Jos's position and dignity, as
collector of Boggley Wollah, might compensate for his father's loss of
station, and tend to reconcile old Osborne to the alliance:  and
finally, to communicate it to the latter in such a way as should least
irritate the old gentleman.

Now, before he faced the head of the Osborne house with the news which
it was his duty to tell, Dobbin bethought him that it would be politic
to make friends of the rest of the family, and, if possible, have the
ladies on his side. They can't be angry in their hearts, thought he.
No woman ever was really angry at a romantic marriage. A little crying
out, and they must come round to their brother; when the three of us
will lay siege to old Mr.\ Osborne.  So this Machiavellian captain of
infantry cast about him for some happy means or stratagem by which he
could gently and gradually bring the Misses Osborne to a knowledge of
their brother's secret.

By a little inquiry regarding his mother's engagements, he was pretty
soon able to find out by whom of her ladyship's friends parties were
given at that season; where he would be likely to meet Osborne's
sisters; and, though he had that abhorrence of routs and evening
parties which many sensible men, alas! entertain, he soon found one
where the Misses Osborne were to be present. Making his appearance at
the ball, where he danced a couple of sets with both of them, and was
prodigiously polite, he actually had the courage to ask Miss Osborne
for a few minutes' conversation at an early hour the next day, when he
had, he said, to communicate to her news of the very greatest interest.

What was it that made her start back, and gaze upon him for a moment,
and then on the ground at her feet, and make as if she would faint on
his arm, had he not by opportunely treading on her toes, brought the
young lady back to self-control?  Why was she so violently agitated at
Dobbin's request?  This can never be known. But when he came the next
day, Maria was not in the drawing-room with her sister, and Miss Wirt
went off for the purpose of fetching the latter, and the Captain and
Miss Osborne were left together. They were both so silent that the
ticktock of the Sacrifice of Iphigenia clock on the mantelpiece became
quite rudely audible.

``What a nice party it was last night,'' Miss Osborne at length began,
encouragingly; ``and---and how you're improved in your dancing, Captain
Dobbin.  Surely somebody has taught you,'' she added, with amiable
archness.

``You should see me dance a reel with Mrs.\ Major O'Dowd of ours; and a
jig---did you ever see a jig?  But I think anybody could dance with you,
Miss Osborne, who dance so well.''

``Is the Major's lady young and beautiful, Captain?'' the fair questioner
continued.  ``Ah, what a terrible thing it must be to be a soldier's
wife!  I wonder they have any spirits to dance, and in these dreadful
times of war, too! O Captain Dobbin, I tremble sometimes when I think
of our dearest George, and the dangers of the poor soldier. Are there
many married officers of the ---th, Captain Dobbin?''

``Upon my word, she's playing her hand rather too openly,'' Miss Wirt
thought; but this observation is merely parenthetic, and was not heard
through the crevice of the door at which the governess uttered it.

``One of our young men is just married,'' Dobbin said, now coming to the
point.  ``It was a very old attachment, and the young couple are as poor
as church mice.'' ``O, how delightful! O, how romantic!'' Miss Osborne
cried, as the Captain said ``old attachment'' and ``poor.'' Her sympathy
encouraged him.

``The finest young fellow in the regiment,'' he continued. ``Not a braver
or handsomer officer in the army; and such a charming wife! How you
would like her!  how you will like her when you know her, Miss
Osborne.''  The young lady thought the actual moment had arrived, and
that Dobbin's nervousness which now came on and was visible in many
twitchings of his face, in his manner of beating the ground with his
great feet, in the rapid buttoning and unbuttoning of his frock-coat,
\&c.---Miss Osborne, I say, thought that when he had given himself a
little air, he would unbosom himself entirely, and prepared eagerly to
listen.  And the clock, in the altar on which Iphigenia was situated,
beginning, after a preparatory convulsion, to toll twelve, the mere
tolling seemed as if it would last until one---so prolonged was the
knell to the anxious spinster.

``But it's not about marriage that I came to speak---that is that
marriage---that is---no, I mean---my dear Miss Osborne, it's about our
dear friend George,'' Dobbin said.

``About George?'' she said in a tone so discomfited that Maria and Miss
Wirt laughed at the other side of the door, and even that abandoned
wretch of a Dobbin felt inclined to smile himself; for he was not
altogether unconscious of the state of affairs:  George having often
bantered him gracefully and said, ``Hang it, Will, why don't you take
old Jane?  She'll have you if you ask her. I'll bet you five to two she
will.''

``Yes, about George, then,'' he continued.  ``There has been a difference
between him and Mr.\ Osborne.  And I regard him so much---for you know
we have been like brothers---that I hope and pray the quarrel may be
settled.  We must go abroad, Miss Osborne.  We may be ordered off at a
day's warning.  Who knows what may happen in the campaign?  Don't be
agitated, dear Miss Osborne; and those two at least should part
friends.''

``There has been no quarrel, Captain Dobbin, except a little usual scene
with Papa,'' the lady said.  ``We are expecting George back daily.  What
Papa wanted was only for his good.  He has but to come back, and I'm
sure all will be well; and dear Rhoda, who went away from here in sad
sad anger, I know will forgive him.  Woman forgives but too readily,
Captain.''

``Such an angel as YOU I am sure would,'' Mr.\ Dobbin said, with atrocious
astuteness.  ``And no man can pardon himself for giving a woman pain.
What would you feel, if a man were faithless to you?''

``I should perish---I should throw myself out of window---I should take
poison---I should pine and die.  I know I should,'' Miss cried, who had
nevertheless gone through one or two affairs of the heart without any
idea of suicide.

``And there are others,'' Dobbin continued, ``as true and as kind-hearted
as yourself.  I'm not speaking about the West Indian heiress, Miss
Osborne, but about a poor girl whom George once loved, and who was bred
from her childhood to think of nobody but him. I've seen her in her
poverty uncomplaining, broken-hearted, without a fault.  It is of Miss
Sedley I speak.  Dear Miss Osborne, can your generous heart quarrel
with your brother for being faithful to her? Could his own conscience
ever forgive him if he deserted her?  Be her friend---she always loved
you---and---and I am come here charged by George to tell you that he
holds his engagement to her as the most sacred duty he has; and to
entreat you, at least, to be on his side.''

When any strong emotion took possession of Mr.\ Dobbin, and after the
first word or two of hesitation, he could speak with perfect fluency,
and it was evident that his eloquence on this occasion made some
impression upon the lady whom he addressed.

``Well,'' said she, ``this is---most surprising---most painful---most
extraordinary---what will Papa say?---that George should fling away such
a superb establishment as was offered to him but at any rate he has
found a very brave champion in you, Captain Dobbin.  It is of no use,
however,'' she continued, after a pause; ``I feel for poor Miss Sedley,
most certainly---most sincerely, you know. We never thought the match a
good one, though we were always very kind to her here---very.  But Papa
will never consent, I am sure.  And a well brought up young woman, you
know---with a well-regulated mind, must---George must give her up, dear
Captain Dobbin, indeed he must.''

``Ought a man to give up the woman he loved, just when misfortune befell
her?'' Dobbin said, holding out his hand.  ``Dear Miss Osborne, is this
the counsel I hear from you?  My dear young lady! you must befriend
her. He can't give her up.  He must not give her up.  Would a man,
think you, give YOU up if you were poor?''

This adroit question touched the heart of Miss Jane Osborne not a
little.  ``I don't know whether we poor girls ought to believe what you
men say, Captain,'' she said. ``There is that in woman's tenderness which
induces her to believe too easily.  I'm afraid you are cruel, cruel
deceivers,''---and Dobbin certainly thought he felt a pressure of the
hand which Miss Osborne had extended to him.

He dropped it in some alarm.  ``Deceivers!'' said he. ``No, dear Miss
Osborne, all men are not; your brother is not; George has loved Amelia
Sedley ever since they were children; no wealth would make him marry
any but her.  Ought he to forsake her?  Would you counsel him to do so?''

What could Miss Jane say to such a question, and with her own peculiar
views?  She could not answer it, so she parried it by saying, ``Well, if
you are not a deceiver, at least you are very romantic''; and Captain
William let this observation pass without challenge.

At length when, by the help of farther polite speeches, he deemed that
Miss Osborne was sufficiently prepared to receive the whole news, he
poured it into her ear. ``George could not give up Amelia---George was
married to her''---and then he related the circumstances of the marriage
as we know them already:  how the poor girl would have died had not her
lover kept his faith:  how Old Sedley had refused all consent to the
match, and a licence had been got: and Jos Sedley had come from
Cheltenham to give away the bride: how they had gone to Brighton in
Jos's chariot-and-four to pass the honeymoon: and how George counted on
his dear kind sisters to befriend him with their father, as women---so
true and tender as they were---assuredly would do.  And so, asking
permission (readily granted) to see her again, and rightly conjecturing
that the news he had brought would be told in the next five minutes to
the other ladies, Captain Dobbin made his bow and took his leave.

He was scarcely out of the house, when Miss Maria and Miss Wirt rushed
in to Miss Osborne, and the whole wonderful secret was imparted to them
by that lady.  To do them justice, neither of the sisters was very much
displeased.  There is something about a runaway match with which few
ladies can be seriously angry, and Amelia rather rose in their
estimation, from the spirit which she had displayed in consenting to
the union.  As they debated the story, and prattled about it, and
wondered what Papa would do and say, came a loud knock, as of an
avenging thunder-clap, at the door, which made these conspirators
start.  It must be Papa, they thought. But it was not he.  It was only
Mr.\ Frederick Bullock, who had come from the City according to
appointment, to conduct the ladies to a flower-show.

This gentleman, as may be imagined, was not kept long in ignorance of
the secret.  But his face, when he heard it, showed an amazement which
was very different to that look of sentimental wonder which the
countenances of the sisters wore.  Mr.\ Bullock was a man of the world,
and a junior partner of a wealthy firm.  He knew what money was, and
the value of it: and a delightful throb of expectation lighted up his
little eyes, and caused him to smile on his Maria, as he thought that
by this piece of folly of Mr.\ George's she might be worth thirty
thousand pounds more than he had ever hoped to get with her.

``Gad!  Jane,'' said he, surveying even the elder sister with some
interest, ``Eels will be sorry he cried off.  You may be a fifty
thousand pounder yet.''

The sisters had never thought of the money question up to that moment,
but Fred Bullock bantered them with graceful gaiety about it during
their forenoon's excursion; and they had risen not a little in their
own esteem by the time when, the morning amusement over, they drove
back to dinner.  And do not let my respected reader exclaim against
this selfishness as unnatural.  It was but this present morning, as he
rode on the omnibus from Richmond; while it changed horses, this
present chronicler, being on the roof, marked three little children
playing in a puddle below, very dirty, and friendly, and happy.  To
these three presently came another little one. ``POLLY,'' says she, ``YOUR
SISTER'S GOT A PENNY.''  At which the children got up from the puddle
instantly, and ran off to pay their court to Peggy.  And as the omnibus
drove off I saw Peggy with the infantine procession at her tail,
marching with great dignity towards the stall of a neighbouring
lollipop-woman.



\chapter{In Which Mr.\ Osborne Takes Down the Family Bible}

So having prepared the sisters, Dobbin hastened away to the City to
perform the rest and more difficult part of the task which he had
undertaken.  The idea of facing old Osborne rendered him not a little
nervous, and more than once he thought of leaving the young ladies to
communicate the secret, which, as he was aware, they could not long
retain.  But he had promised to report to George upon the manner in
which the elder Osborne bore the intelligence; so going into the City
to the paternal counting-house in Thames Street, he despatched thence a
note to Mr.\ Osborne begging for a half-hour's conversation relative to
the affairs of his son George.  Dobbin's messenger returned from Mr.\ %
Osborne's house of business, with the compliments of the latter, who
would be very happy to see the Captain immediately, and away
accordingly Dobbin went to confront him.

The Captain, with a half-guilty secret to confess, and with the
prospect of a painful and stormy interview before him, entered Mr.\ %
Osborne's offices with a most dismal countenance and abashed gait, and,
passing through the outer room where Mr.\ Chopper presided, was greeted
by that functionary from his desk with a waggish air which farther
discomfited him.  Mr.\ Chopper winked and nodded and pointed his pen
towards his patron's door, and said, ``You'll find the governor all
right,'' with the most provoking good humour.

Osborne rose too, and shook him heartily by the hand, and said, ``How
do, my dear boy?'' with a cordiality that made poor George's ambassador
feel doubly guilty.  His hand lay as if dead in the old gentleman's
grasp.  He felt that he, Dobbin, was more or less the cause of all that
had happened.  It was he had brought back George to Amelia: it was he
had applauded, encouraged, transacted almost the marriage which he was
come to reveal to George's father:  and the latter was receiving him
with smiles of welcome; patting him on the shoulder, and calling him
``Dobbin, my dear boy.'' The envoy had indeed good reason to hang his
head.

Osborne fully believed that Dobbin had come to announce his son's
surrender.  Mr.\ Chopper and his principal were talking over the matter
between George and his father, at the very moment when Dobbin's
messenger arrived.  Both agreed that George was sending in his
submission.  Both had been expecting it for some days---and ``Lord!
Chopper, what a marriage we'll have!'' Mr.\ Osborne said to his clerk,
snapping his big fingers, and jingling all the guineas and shillings in
his great pockets as he eyed his subordinate with a look of triumph.

With similar operations conducted in both pockets, and a knowing jolly
air, Osborne from his chair regarded Dobbin seated blank and silent
opposite to him.  ``What a bumpkin he is for a Captain in the army,'' old
Osborne thought.  ``I wonder George hasn't taught him better manners.''

At last Dobbin summoned courage to begin.  ``Sir,'' said he, ``I've
brought you some very grave news.  I have been at the Horse Guards this
morning, and there's no doubt that our regiment will be ordered abroad,
and on its way to Belgium before the week is over.  And you know, sir,
that we shan't be home again before a tussle which may be fatal to many
of us.''   Osborne looked grave.  ``My s---, the regiment will do its
duty, sir, I daresay,'' he said.

``The French are very strong, sir,'' Dobbin went on. ``The Russians and
Austrians will be a long time before they can bring their troops down.
We shall have the first of the fight, sir; and depend on it Boney will
take care that it shall be a hard one.''

``What are you driving at, Dobbin?'' his interlocutor said, uneasy and
with a scowl.  ``I suppose no Briton's afraid of any d--------- Frenchman,
hey?''

``I only mean, that before we go, and considering the great and certain
risk that hangs over every one of us---if there are any differences
between you and George---it would be as well, sir, that---that you
should shake hands: wouldn't it?  Should anything happen to him, I
think you would never forgive yourself if you hadn't parted in charity.''

As he said this, poor William Dobbin blushed crimson, and felt and
owned that he himself was a traitor.  But for him, perhaps, this
severance need never have taken place.  Why had not George's marriage
been delayed? What call was there to press it on so eagerly?  He felt
that George would have parted from Amelia at any rate without a mortal
pang.  Amelia, too, MIGHT have recovered the shock of losing him.  It
was his counsel had brought about this marriage, and all that was to
ensue from it. And why was it? Because he loved her so much that he
could not bear to see her unhappy:  or because his own sufferings of
suspense were so unendurable that he was glad to crush them at once---as
we hasten a funeral after a death, or, when a separation from those we
love is imminent, cannot rest until the parting be over.

``You are a good fellow, William,'' said Mr.\ Osborne in a softened voice;
``and me and George shouldn't part in anger, that is true. Look here.
I've done for him as much as any father ever did.  He's had three times
as much money from me, as I warrant your father ever gave you.  But I
don't brag about that.  How I've toiled for him, and worked and
employed my talents and energy, I won't say.  Ask Chopper.  Ask
himself.  Ask the City of London.  Well, I propose to him such a
marriage as any nobleman in the land might be proud of---the only thing
in life I ever asked him---and he refuses me.  Am I wrong? Is the
quarrel of MY making?  What do I seek but his good, for which I've been
toiling like a convict ever since he was born? Nobody can say there's
anything selfish in me.  Let him come back. I say, here's my hand.  I
say, forget and forgive.  As for marrying now, it's out of the
question.  Let him and Miss S. make it up, and make out the marriage
afterwards, when he comes back a Colonel; for he shall be a Colonel, by
G--- he shall, if money can do it.  I'm glad you've brought him round.
I know it's you, Dobbin.  You've took him out of many a scrape before.
Let him come.  I shan't be hard.  Come along, and dine in Russell
Square to-day: both of you.  The old shop, the old hour.  You'll find a
neck of venison, and no questions asked.''

This praise and confidence smote Dobbin's heart very keenly.  Every
moment the colloquy continued in this tone, he felt more and more
guilty.  ``Sir,'' said he, ``I fear you deceive yourself.  I am sure you
do.  George is much too high-minded a man ever to marry for money.  A
threat on your part that you would disinherit him in case of
disobedience would only be followed by resistance on his.''

``Why, hang it, man, you don't call offering him eight or ten thousand a
year threatening him?'' Mr.\ Osborne said, with still provoking good
humour.  ``'Gad, if Miss S. will have me, I'm her man. I ain't
particular about a shade or so of tawny.'' And the old gentleman gave
his knowing grin and coarse laugh.

``You forget, sir, previous engagements into which Captain Osborne had
entered,'' the ambassador said, gravely.

``What engagements? What the devil do you mean? You don't mean,'' Mr.\ %
Osborne continued, gathering wrath and astonishment as the thought now
first came upon him; ``you don't mean that he's such a d--------- fool as to
be still hankering after that swindling old bankrupt's daughter? You've
not come here for to make me suppose that he wants to marry HER?  Marry
HER, that IS a good one.  My son and heir marry a beggar's girl out of
a gutter.  D--------- him, if he does, let him buy a broom and sweep a
crossing.  She was always dangling and ogling after him, I recollect
now; and I've no doubt she was put on by her old sharper of a father.''

``Mr.\ Sedley was your very good friend, sir,'' Dobbin interposed, almost
pleased at finding himself growing angry.  ``Time was you called him
better names than rogue and swindler.  The match was of your making.
George had no right to play fast and loose---''

``Fast and loose!'' howled out old Osborne.  ``Fast and loose!  Why, hang
me, those are the very words my gentleman used himself when he gave
himself airs, last Thursday was a fortnight, and talked about the
British army to his father who made him.  What, it's you who have been
a setting of him up---is it? and my service to you, CAPTAIN.  It's you
who want to introduce beggars into my family. Thank you for nothing,
Captain.  Marry HER indeed---he, he! why should he?  I warrant you she'd
go to him fast enough without.''

``Sir,'' said Dobbin, starting up in undisguised anger; ``no man shall
abuse that lady in my hearing, and you least of all.''

``O, you're a-going to call me out, are you?  Stop, let me ring the bell
for pistols for two.  Mr.\ George sent you here to insult his father,
did he?'' Osborne said, pulling at the bell-cord.

``Mr.\ Osborne,'' said Dobbin, with a faltering voice, ``it's you who are
insulting the best creature in the world. You had best spare her, sir,
for she's your son's wife.''

And with this, feeling that he could say no more, Dobbin went away,
Osborne sinking back in his chair, and looking wildly after him.  A
clerk came in, obedient to the bell; and the Captain was scarcely out
of the court where Mr.\ Osborne's offices were, when Mr.\ Chopper the
chief clerk came rushing hatless after him.

``For God's sake, what is it?'' Mr.\ Chopper said, catching the Captain by
the skirt.  ``The governor's in a fit. What has Mr.\ George been doing?''

``He married Miss Sedley five days ago,'' Dobbin replied. ``I was his
groomsman, Mr.\ Chopper, and you must stand his friend.''

The old clerk shook his head.  ``If that's your news, Captain, it's bad.
The governor will never forgive him.''

Dobbin begged Chopper to report progress to him at the hotel where he
was stopping, and walked off moodily westwards, greatly perturbed as to
the past and the future.

When the Russell Square family came to dinner that evening, they found
the father of the house seated in his usual place, but with that air of
gloom on his face, which, whenever it appeared there, kept the whole
circle silent. The ladies, and Mr.\ Bullock who dined with them, felt
that the news had been communicated to Mr.\ Osborne. His dark looks
affected Mr.\ Bullock so far as to render him still and quiet: but he
was unusually bland and attentive to Miss Maria, by whom he sat, and to
her sister presiding at the head of the table.

Miss Wirt, by consequence, was alone on her side of the board, a gap
being left between her and Miss Jane Osborne.  Now this was George's
place when he dined at home; and his cover, as we said, was laid for
him in expectation of that truant's return.  Nothing occurred during
dinner-time except smiling Mr.\ Frederick's flagging confidential
whispers, and the clinking of plate and china, to interrupt the silence
of the repast.  The servants went about stealthily doing their duty.
Mutes at funerals could not look more glum than the domestics of Mr.\ %
Osborne The neck of venison of which he had invited Dobbin to partake,
was carved by him in perfect silence; but his own share went away
almost untasted, though he drank much, and the butler assiduously
filled his glass.

At last, just at the end of the dinner, his eyes, which had been
staring at everybody in turn, fixed themselves for a while upon the
plate laid for George.  He pointed to it presently with his left hand.
His daughters looked at him and did not comprehend, or choose to
comprehend, the signal; nor did the servants at first understand it.

``Take that plate away,'' at last he said, getting up with an oath---and
with this pushing his chair back, he walked into his own room.

Behind Mr.\ Osborne's dining-room was the usual apartment which went in
his house by the name of the study; and was sacred to the master of the
house.  Hither Mr.\ Osborne would retire of a Sunday forenoon when not
minded to go to church; and here pass the morning in his crimson
leather chair, reading the paper.  A couple of glazed book-cases were
here, containing standard works in stout gilt bindings. The ``Annual
Register,'' the ``Gentleman's Magazine,'' ``Blair's Sermons,'' and ``Hume and
Smollett.'' From year's end to year's end he never took one of these
volumes from the shelf; but there was no member of the family that
would dare for his life to touch one of the books, except upon those
rare Sunday evenings when there was no dinner-party, and when the great
scarlet Bible and Prayer-book were taken out from the corner where they
stood beside his copy of the Peerage, and the servants being rung up to
the dining parlour, Osborne read the evening service to his family in a
loud grating pompous voice.  No member of the household, child, or
domestic, ever entered that room without a certain terror.  Here he
checked the housekeeper's accounts, and overhauled the butler's
cellar-book. Hence he could command, across the clean gravel
court-yard, the back entrance of the stables with which one of his
bells communicated, and into this yard the coachman issued from his
premises as into a dock, and Osborne swore at him from the study
window.  Four times a year Miss Wirt entered this apartment to get her
salary; and his daughters to receive their quarterly allowance.  George
as a boy had been horsewhipped in this room many times; his mother
sitting sick on the stair listening to the cuts of the whip.  The boy
was scarcely ever known to cry under the punishment; the poor woman
used to fondle and kiss him secretly, and give him money to soothe him
when he came out.

There was a picture of the family over the mantelpiece, removed thither
from the front room after Mrs.\ Osborne's death---George was on a pony,
the elder sister holding him up a bunch of flowers; the younger led by
her mother's hand; all with red cheeks and large red mouths, simpering
on each other in the approved family-portrait manner.  The mother lay
underground now, long since forgotten---the sisters and brother had a
hundred different interests of their own, and, familiar still, were
utterly estranged from each other.  Some few score of years afterwards,
when all the parties represented are grown old, what bitter satire
there is in those flaunting childish family-portraits, with their farce
of sentiment and smiling lies, and innocence so self-conscious and
self-satisfied.  Osborne's own state portrait, with that of his great
silver inkstand and arm-chair, had taken the place of honour in the
dining-room, vacated by the family-piece.

To this study old Osborne retired then, greatly to the relief of the
small party whom he left.  When the servants had withdrawn, they began
to talk for a while volubly but very low; then they went upstairs
quietly, Mr.\ Bullock accompanying them stealthily on his creaking
shoes.  He had no heart to sit alone drinking wine, and so close to the
terrible old gentleman in the study hard at hand.

An hour at least after dark, the butler, not having received any
summons, ventured to tap at his door and take him in wax candles and
tea.  The master of the house sate in his chair, pretending to read the
paper, and when the servant, placing the lights and refreshment on the
table by him, retired, Mr.\ Osborne got up and locked the door after
him.  This time there was no mistaking the matter; all the household
knew that some great catastrophe was going to happen which was likely
direly to affect Master George.

In the large shining mahogany escritoire Mr.\ Osborne had a drawer
especially devoted to his son's affairs and papers.  Here he kept all
the documents relating to him ever since he had been a boy: here were
his prize copy-books and drawing-books, all bearing George's hand, and
that of the master:  here were his first letters in large round-hand
sending his love to papa and mamma, and conveying his petitions for a
cake.  His dear godpapa Sedley was more than once mentioned in them.
Curses quivered on old Osborne's livid lips, and horrid hatred and
disappointment writhed in his heart, as looking through some of these
papers he came on that name. They were all marked and docketed, and
tied with red tape. It was---``From Georgy, requesting 5s., April 23,
18---; answered, April 25''---or ``Georgy about a pony, October 13''---and so
forth. In another packet were ``Dr. S.'s accounts''---``G.'s tailor's bills
and outfits, drafts on me by G. Osborne, jun.,'' \&c.---his letters from
the West Indies---his agent's letters, and the newspapers containing his
commissions: here was a whip he had when a boy, and in a paper a locket
containing his hair, which his mother used to wear.

Turning one over after another, and musing over these memorials, the
unhappy man passed many hours.  His dearest vanities, ambitious hopes,
had all been here.  What pride he had in his boy!  He was the
handsomest child ever seen.  Everybody said he was like a nobleman's
son.  A royal princess had remarked him, and kissed him, and asked his
name in Kew Gardens.  What City man could show such another? Could a
prince have been better cared for?  Anything that money could buy had
been his son's.  He used to go down on speech-days with four horses and
new liveries, and scatter new shillings among the boys at the school
where George was:  when he went with George to the depot of his
regiment, before the boy embarked for Canada, he gave the officers such
a dinner as the Duke of York might have sat down to.  Had he ever
refused a bill when George drew one? There they were---paid without a
word.  Many a general in the army couldn't ride the horses he had!  He
had the child before his eyes, on a hundred different days when he
remembered George after dinner, when he used to come in as bold as a
lord and drink off his glass by his father's side, at the head of the
table---on the pony at Brighton, when he cleared the hedge and kept up
with the huntsman---on the day when he was presented to the Prince
Regent at the levee, when all Saint James's couldn't produce a finer
young fellow.  And this, this was the end of all!---to marry a bankrupt
and fly in the face of duty and fortune!  What humiliation and fury:
what pangs of sickening rage, balked ambition and love; what wounds of
outraged vanity, tenderness even, had this old worldling now to suffer
under!

Having examined these papers, and pondered over this one and the other,
in that bitterest of all helpless woe, with which miserable men think
of happy past times---George's father took the whole of the documents
out of the drawer in which he had kept them so long, and locked them
into a writing-box, which he tied, and sealed with his seal.  Then he
opened the book-case, and took down the great red Bible we have spoken
of a pompous book, seldom looked at, and shining all over with gold.
There was a frontispiece to the volume, representing Abraham
sacrificing Isaac.  Here, according to custom, Osborne had recorded on
the fly-leaf, and in his large clerk-like hand, the dates of his
marriage and his wife's death, and the births and Christian names of
his children. Jane came first, then George Sedley Osborne, then Maria
Frances, and the days of the christening of each.  Taking a pen, he
carefully obliterated George's names from the page; and when the leaf
was quite dry, restored the volume to the place from which he had moved
it.  Then he took a document out of another drawer, where his own
private papers were kept; and having read it, crumpled it up and
lighted it at one of the candles, and saw it burn entirely away in the
grate.  It was his will; which being burned, he sate down and wrote off
a letter, and rang for his servant, whom he charged to deliver it in
the morning.  It was morning already: as he went up to bed, the whole
house was alight with the sunshine; and the birds were singing among
the fresh green leaves in Russell Square.

Anxious to keep all Mr.\ Osborne's family and dependants in good humour,
and to make as many friends as possible for George in his hour of
adversity, William Dobbin, who knew the effect which good dinners and
good wines have upon the soul of man, wrote off immediately on his
return to his inn the most hospitable of invitations to Thomas Chopper,
Esquire, begging that gentleman to dine with him at the Slaughters'
next day.  The note reached Mr.\ Chopper before he left the City, and
the instant reply was, that ``Mr.\ Chopper presents his respectful
compliments, and will have the honour and pleasure of waiting on
Captain D.''  The invitation and the rough draft of the answer were
shown to Mrs.\ Chopper and her daughters on his return to Somers' Town
that evening, and they talked about military gents and West End men
with great exultation as the family sate and partook of tea.  When the
girls had gone to rest, Mr.\ and Mrs.\ C. discoursed upon the strange
events which were occurring in the governor's family.  Never had the
clerk seen his principal so moved.  When he went in to Mr.\ Osborne,
after Captain Dobbin's departure, Mr.\ Chopper found his chief black in
the face, and all but in a fit: some dreadful quarrel, he was certain,
had occurred between Mr.\ O. and the young Captain.  Chopper had been
instructed to make out an account of all sums paid to Captain Osborne
within the last three years.  ``And a precious lot of money he has had
too,'' the chief clerk said, and respected his old and young master the
more, for the liberal way in which the guineas had been flung about.
The dispute was something about Miss Sedley.  Mrs.\ Chopper vowed and
declared she pitied that poor young lady to lose such a handsome young
fellow as the Capting. As the daughter of an unlucky speculator, who
had paid a very shabby dividend, Mr.\ Chopper had no great regard for
Miss Sedley.  He respected the house of Osborne before all others in
the City of London: and his hope and wish was that Captain George
should marry a nobleman's daughter. The clerk slept a great deal
sounder than his principal that night; and, cuddling his children after
breakfast (of which he partook with a very hearty appetite, though his
modest cup of life was only sweetened with brown sugar), he set off in
his best Sunday suit and frilled shirt for business, promising his
admiring wife not to punish Captain D.'s port too severely that evening.

Mr.\ Osborne's countenance, when he arrived in the City at his usual
time, struck those dependants who were accustomed, for good reasons, to
watch its expression, as peculiarly ghastly and worn.  At twelve
o'clock Mr.\ Higgs (of the firm of Higgs \& Blatherwick, solicitors,
Bedford Row) called by appointment, and was ushered into the governor's
private room, and closeted there for more than an hour. At about one
Mr.\ Chopper received a note brought by Captain Dobbin's man, and
containing an inclosure for Mr.\ Osborne, which the clerk went in and
delivered.  A short time afterwards Mr.\ Chopper and Mr.\ Birch, the next
clerk, were summoned, and requested to witness a paper.  ``I've been
making a new will,'' Mr.\ Osborne said, to which these gentlemen appended
their names accordingly.  No conversation passed.  Mr.\ Higgs looked
exceedingly grave as he came into the outer rooms, and very hard in Mr.\ %
Chopper's face; but there were not any explanations.  It was remarked
that Mr.\ Osborne was particularly quiet and gentle all day, to the
surprise of those who had augured ill from his darkling demeanour.  He
called no man names that day, and was not heard to swear once.  He left
business early; and before going away, summoned his chief clerk once
more, and having given him general instructions, asked him, after some
seeming hesitation and reluctance to speak, if he knew whether Captain
Dobbin was in town?

Chopper said he believed he was.  Indeed both of them knew the fact
perfectly.

Osborne took a letter directed to that officer, and giving it to the
clerk, requested the latter to deliver it into Dobbin's own hands
immediately.

``And now, Chopper,'' says he, taking his hat, and with a strange look,
``my mind will be easy.''  Exactly as the clock struck two (there was no
doubt an appointment between the pair) Mr.\ Frederick Bullock called,
and he and Mr.\ Osborne walked away together.

The Colonel of the ---th regiment, in which Messieurs Dobbin and Osborne
had companies, was an old General who had made his first campaign under
Wolfe at Quebec, and was long since quite too old and feeble for
command; but he took some interest in the regiment of which he was the
nominal head, and made certain of his young officers welcome at his
table, a kind of hospitality which I believe is not now common amongst
his brethren.  Captain Dobbin was an especial favourite of this old
General.  Dobbin was versed in the literature of his profession, and
could talk about the great Frederick, and the Empress Queen, and their
wars, almost as well as the General himself, who was indifferent to the
triumphs of the present day, and whose heart was with the tacticians of
fifty years back.  This officer sent a summons to Dobbin to come and
breakfast with him, on the morning when Mr.\ Osborne altered his will
and Mr.\ Chopper put on his best shirt frill, and then informed his
young favourite, a couple of days in advance, of that which they were
all expecting---a marching order to go to Belgium. The order for the
regiment to hold itself in readiness would leave the Horse Guards in a
day or two; and as transports were in plenty, they would get their
route before the week was over.  Recruits had come in during the stay
of the regiment at Chatham; and the old General hoped that the regiment
which had helped to beat Montcalm in Canada, and to rout Mr.\ Washington
on Long Island, would prove itself worthy of its historical reputation
on the oft-trodden battle-grounds of the Low Countries.  ``And so, my
good friend, if you have any affaire la,'' said the old General, taking a
pinch of snuff with his trembling white old hand, and then pointing to
the spot of his robe de chambre under which his heart was still feebly
beating, ``if you have any Phillis to console, or to bid farewell to
papa and mamma, or any will to make, I recommend you to set about your
business without delay.'' With which the General gave his young friend a
finger to shake, and a good-natured nod of his powdered and pigtailed
head; and the door being closed upon Dobbin, sate down to pen a poulet
(he was exceedingly vain of his French) to Mademoiselle Amenaide of His
Majesty's Theatre.

This news made Dobbin grave, and he thought of our friends at Brighton,
and then he was ashamed of himself that Amelia was always the first
thing in his thoughts (always before anybody---before father and mother,
sisters and duty---always at waking and sleeping indeed, and all day
long); and returning to his hotel, he sent off a brief note to Mr.\ %
Osborne acquainting him with the information which he had received, and
which might tend farther, he hoped, to bring about a reconciliation
with George.

This note, despatched by the same messenger who had carried the
invitation to Chopper on the previous day, alarmed the worthy clerk not
a little.  It was inclosed to him, and as he opened the letter he
trembled lest the dinner should be put off on which he was calculating.
His mind was inexpressibly relieved when he found that the envelope was
only a reminder for himself.  (``I shall expect you at half-past five,''
Captain Dobbin wrote.) He was very much interested about his employer's
family; but, que voulez-vous? a grand dinner was of more concern to him
than the affairs of any other mortal.

Dobbin was quite justified in repeating the General's information to
any officers of the regiment whom he should see in the course of his
peregrinations; accordingly he imparted it to Ensign Stubble, whom he
met at the agent's, and who---such was his military ardour---went off
instantly to purchase a new sword at the accoutrement-maker's. Here
this young fellow, who, though only seventeen years of age, and about
sixty-five inches high, with a constitution naturally rickety and much
impaired by premature brandy and water, had an undoubted courage and a
lion's heart, poised, tried, bent, and balanced a weapon such as he
thought would do execution amongst Frenchmen. Shouting ``Ha, ha!'' and
stamping his little feet with tremendous energy, he delivered the point
twice or thrice at Captain Dobbin, who parried the thrust laughingly
with his bamboo walking-stick.

Mr.\ Stubble, as may be supposed from his size and slenderness, was of
the Light Bobs.  Ensign Spooney, on the contrary, was a tall youth, and
belonged to (Captain Dobbin's) the Grenadier Company, and he tried on a
new bearskin cap, under which he looked savage beyond his years.  Then
these two lads went off to the Slaughters', and having ordered a famous
dinner, sate down and wrote off letters to the kind anxious parents at
home---letters full of love and heartiness, and pluck and bad spelling.
Ah! there were many anxious hearts beating through England at that
time; and mothers' prayers and tears flowing in many homesteads.

Seeing young Stubble engaged in composition at one of the coffee-room
tables at the Slaughters', and the tears trickling down his nose on to
the paper (for the youngster was thinking of his mamma, and that he
might never see her again), Dobbin, who was going to write off a letter
to George Osborne, relented, and locked up his desk.  ``Why should I?''
said he.  ``Let her have this night happy. I'll go and see my parents
early in the morning, and go down to Brighton myself to-morrow.''

So he went up and laid his big hand on young Stubble's shoulder, and
backed up that young champion, and told him if he would leave off
brandy and water he would be a good soldier, as he always was a
gentlemanly good-hearted fellow.  Young Stubble's eyes brightened up at
this, for Dobbin was greatly respected in the regiment, as the best
officer and the cleverest man in it.

``Thank you, Dobbin,'' he said, rubbing his eyes with his knuckles, ``I
was just---just telling her I would.  And, O Sir, she's so dam kind to
me.'' The water pumps were at work again, and I am not sure that the
soft-hearted Captain's eyes did not also twinkle.

The two ensigns, the Captain, and Mr.\ Chopper, dined together in the
same box.  Chopper brought the letter from Mr.\ Osborne, in which the
latter briefly presented his compliments to Captain Dobbin, and
requested him to forward the inclosed to Captain George Osborne.
Chopper knew nothing further; he described Mr.\ Osborne's appearance, it
is true, and his interview with his lawyer, wondered how the governor
had sworn at nobody, and---especially as the wine circled
round---abounded in speculations and conjectures.  But these grew more
vague with every glass, and at length became perfectly unintelligible.
At a late hour Captain Dobbin put his guest into a hackney coach, in a
hiccupping state, and swearing that he would be the kick---the
kick---Captain's friend for ever and ever.

When Captain Dobbin took leave of Miss Osborne we have said that he
asked leave to come and pay her another visit, and the spinster
expected him for some hours the next day, when, perhaps, had he come,
and had he asked her that question which she was prepared to answer,
she would have declared herself as her brother's friend, and a
reconciliation might have been effected between George and his angry
father.  But though she waited at home the Captain never came. He had
his own affairs to pursue; his own parents to visit and console; and at
an early hour of the day to take his place on the Lightning coach, and
go down to his friends at Brighton.  In the course of the day Miss
Osborne heard her father give orders that that meddling scoundrel,
Captain Dobbin, should never be admitted within his doors again, and
any hopes in which she may have indulged privately were thus abruptly
brought to an end.  Mr.\ Frederick Bullock came, and was particularly
affectionate to Maria, and attentive to the broken-spirited old
gentleman.  For though he said his mind would be easy, the means which
he had taken to secure quiet did not seem to have succeeded as yet, and
the events of the past two days had visibly shattered him.



\chapter{In Which All the Principal Personages Think Fit to Leave Brighton}

Conducted to the ladies, at the Ship Inn, Dobbin assumed a jovial and
rattling manner, which proved that this young officer was becoming a
more consummate hypocrite every day of his life.  He was trying to hide
his own private feelings, first upon seeing Mrs.\ George Osborne in her
new condition, and secondly to mask the apprehensions he entertained as
to the effect which the dismal news brought down by him would certainly
have upon her.

``It is my opinion, George,'' he said, ``that the French Emperor will be
upon us, horse and foot, before three weeks are over, and will give the
Duke such a dance as shall make the Peninsula appear mere child's play.
But you need not say that to Mrs.\ Osborne, you know. There mayn't be
any fighting on our side after all, and our business in Belgium may
turn out to be a mere military occupation.  Many persons think so; and
Brussels is full of fine people and ladies of fashion.'' So it was
agreed to represent the duty of the British army in Belgium in this
harmless light to Amelia.

This plot being arranged, the hypocritical Dobbin saluted Mrs.\ George
Osborne quite gaily, tried to pay her one or two compliments relative
to her new position as a bride (which compliments, it must be
confessed, were exceedingly clumsy and hung fire woefully), and then
fell to talking about Brighton, and the sea-air, and the gaieties of
the place, and the beauties of the road and the merits of the Lightning
coach and horses---all in a manner quite incomprehensible to Amelia, and
very amusing to Rebecca, who was watching the Captain, as indeed she
watched every one near whom she came.

Little Amelia, it must be owned, had rather a mean opinion of her
husband's friend, Captain Dobbin.  He lisped---he was very plain and
homely-looking: and exceedingly awkward and ungainly.  She liked him
for his attachment to her husband (to be sure there was very little
merit in that), and she thought George was most generous and kind in
extending his friendship to his brother officer. George had mimicked
Dobbin's lisp and queer manners many times to her, though to do him
justice, he always spoke most highly of his friend's good qualities. In
her little day of triumph, and not knowing him intimately as yet, she
made light of honest William---and he knew her opinions of him quite
well, and acquiesced in them very humbly.  A time came when she knew
him better, and changed her notions regarding him; but that was distant
as yet.

As for Rebecca, Captain Dobbin had not been two hours in the ladies'
company before she understood his secret perfectly.  She did not like
him, and feared him privately; nor was he very much prepossessed in her
favour.  He was so honest, that her arts and cajoleries did not affect
him, and he shrank from her with instinctive repulsion. And, as she was
by no means so far superior to her sex as to be above jealousy, she
disliked him the more for his adoration of Amelia.  Nevertheless, she
was very respectful and cordial in her manner towards him.  A friend to
the Osbornes! a friend to her dearest benefactors!  She vowed she
should always love him sincerely: she remembered him quite well on the
Vauxhall night, as she told Amelia archly, and she made a little fun of
him when the two ladies went to dress for dinner.  Rawdon Crawley paid
scarcely any attention to Dobbin, looking upon him as a good-natured
nincompoop and under-bred City man.  Jos patronised him with much
dignity.

When George and Dobbin were alone in the latter's room, to which George
had followed him, Dobbin took from his desk the letter which he had
been charged by Mr.\ Osborne to deliver to his son.  ``It's not in my
father's handwriting,'' said George, looking rather alarmed; nor was it:
the letter was from Mr.\ Osborne's lawyer, and to the following effect:

                                    ``Bedford Row, May 7, 1815.

``SIR,

``I am commissioned by Mr.\ Osborne to inform you, that he abides by the
determination which he before expressed to you, and that in consequence
of the marriage which you have been pleased to contract, he ceases to
consider you henceforth as a member of his family. This determination
is final and irrevocable.

``Although the monies expended upon you in your minority, and the bills
which you have drawn upon him so unsparingly of late years, far exceed
in amount the sum to which you are entitled in your own right (being
the third part of the fortune of your mother, the late Mrs.\ Osborne and
which reverted to you at her decease, and to Miss Jane Osborne and Miss
Maria Frances Osborne); yet I am instructed by Mr.\ Osborne to say, that
he waives all claim upon your estate, and that the sum of 2,000 pounds,
4 per cent. annuities, at the value of the day (being your one-third
share of the sum of 6,000 pounds), shall be paid over to yourself or
your agents upon your receipt for the same, by

                          ``Your obedient Servt.,
                                           ``S. HIGGS.

``P.S.---Mr.\ Osborne desires me to say, once for all, that he declines to
receive any messages, letters, or communications from you on this or
any other subject.

``A pretty way you have managed the affair,'' said George, looking
savagely at William Dobbin.  ``Look there, Dobbin,'' and he flung over to
the latter his parent's letter. ``A beggar, by Jove, and all in
consequence of my d---d sentimentality.  Why couldn't we have waited? A
ball might have done for me in the course of the war, and may still,
and how will Emmy be bettered by being left a beggar's widow? It was
all your doing.  You were never easy until you had got me married and
ruined.  What the deuce am I to do with two thousand pounds?  Such a
sum won't last two years.  I've lost a hundred and forty to Crawley at
cards and billiards since I've been down here. A pretty manager of a
man's matters YOU are, forsooth.''

``There's no denying that the position is a hard one,'' Dobbin replied,
after reading over the letter with a blank countenance; ``and as you
say, it is partly of my making. There are some men who wouldn't mind
changing with you,'' he added, with a bitter smile. ``How many captains
in the regiment have two thousand pounds to the fore, think you?  You
must live on your pay till your father relents, and if you die, you
leave your wife a hundred a year.''

``Do you suppose a man of my habits can live on his pay and a hundred a
year?'' George cried out in great anger.  ``You must be a fool to talk
so, Dobbin.  How the deuce am I to keep up my position in the world
upon such a pitiful pittance?  I can't change my habits.  I must have
my comforts.  I wasn't brought up on porridge, like MacWhirter, or on
potatoes, like old O'Dowd.  Do you expect my wife to take in soldiers'
washing, or ride after the regiment in a baggage waggon?''

``Well, well,'' said Dobbin, still good-naturedly, ``we'll get her a
better conveyance.  But try and remember that you are only a dethroned
prince now, George, my boy; and be quiet whilst the tempest lasts.  It
won't be for long.  Let your name be mentioned in the Gazette, and I'll
engage the old father relents towards you:''

``Mentioned in the Gazette!'' George answered.  ``And in what part of it?
Among the killed and wounded returns, and at the top of the list, very
likely.''

``Psha!  It will be time enough to cry out when we are hurt,'' Dobbin
said.  ``And if anything happens, you know, George, I have got a little,
and I am not a marrying man, and I shall not forget my godson in my
will,'' he added, with a smile.  Whereupon the dispute ended---as many
scores of  such conversations between Osborne and his friend had
concluded previously---by the former declaring there was no possibility
of being angry with Dobbin long, and forgiving him very generously
after abusing him without cause.

``I say, Becky,'' cried Rawdon Crawley out of his dressing-room, to his
lady, who was attiring herself for dinner in her own chamber.

``What?'' said Becky's shrill voice.  She was looking over her shoulder
in the glass.  She had put on the neatest and freshest white frock
imaginable, and with bare shoulders and a little necklace, and a light
blue sash, she looked the image of youthful innocence and girlish
happiness.

``I say, what'll Mrs.\ O. do, when O. goes out with the regiment?''
Crawley said coming into the room, performing a duet on his head with
two huge hair-brushes, and looking out from under his hair with
admiration on his pretty little wife.

``I suppose she'll cry her eyes out,'' Becky answered. ``She has been
whimpering half a dozen times, at the very notion of it, already to me.''

``YOU don't care, I suppose?'' Rawdon said, half angry at his wife's want
of feeling.

``You wretch! don't you know that I intend to go with you,'' Becky
replied.  ``Besides, you're different.  You go as General Tufto's
aide-de-camp.  We don't belong to the line,'' Mrs.\ Crawley said,
throwing up her head with an air that so enchanted her husband that he
stooped down and kissed it.

``Rawdon dear---don't you think---you'd better get that---money from Cupid,
before he goes?'' Becky continued, fixing on a killing bow. She called
George Osborne, Cupid.  She had flattered him about his good looks a
score of times already.  She watched over him kindly at ecarte of a
night when he would drop in to Rawdon's quarters for a half-hour before
bed-time.

She had often called him a horrid dissipated wretch, and threatened to
tell Emmy of his wicked ways and naughty extravagant habits.  She
brought his cigar and lighted it for him; she knew the effect of that
manoeuvre, having practised it in former days upon Rawdon Crawley. He
thought her gay, brisk, arch, distinguee, delightful. In their little
drives and dinners, Becky, of course, quite outshone poor Emmy, who
remained very mute and timid while Mrs.\ Crawley and her husband rattled
away together, and Captain Crawley (and Jos after he joined the young
married people) gobbled in silence.

Emmy's mind somehow misgave her about her friend. Rebecca's wit,
spirits, and accomplishments troubled her with a rueful disquiet. They
were only a week married, and here was George already suffering ennui,
and eager for others' society!  She trembled for the future. How shall
I be a companion for him, she thought---so clever and so brilliant, and
I such a humble foolish creature? How noble it was of him to marry
me---to give up everything and stoop down to me!  I ought to have
refused him, only I had not the heart.  I ought to have stopped at home
and taken care of poor Papa.  And her neglect of her parents (and
indeed there was some foundation for this charge which the poor child's
uneasy conscience brought against her) was now remembered for the first
time, and caused her to blush with humiliation.  Oh! thought she, I
have been very wicked and selfish---selfish in forgetting them in their
sorrows---selfish in forcing George to marry me.  I know I'm not worthy
of him---I know he would have been happy without me---and yet---I tried, I
tried to give him up.

It is hard when, before seven days of marriage are over, such thoughts
and confessions as these force themselves on a little bride's mind.
But so it was, and the night before Dobbin came to join these young
people---on a fine brilliant moonlight night of May---so warm and balmy
that the windows were flung open to the balcony, from which George and
Mrs.\ Crawley were gazing upon the calm ocean spread shining before
them, while Rawdon and Jos were engaged at backgammon within---Amelia
couched in a great chair quite neglected, and watching both these
parties, felt a despair and remorse such as were bitter companions for
that tender lonely soul.  Scarce a week was past, and it was come to
this! The future, had she regarded it, offered a dismal prospect; but
Emmy was too shy, so to speak, to look to that, and embark alone on
that wide sea, and unfit to navigate it without a guide and protector.
I know Miss Smith has a mean opinion of her.  But how many, my dear
Madam, are endowed with your prodigious strength of mind?

``Gad, what a fine night, and how bright the moon is!'' George said, with
a puff of his cigar, which went soaring up skywards.

``How delicious they smell in the open air!  I adore them.  Who'd think
the moon was two hundred and thirty-six thousand eight hundred and
forty-seven miles off?'' Becky added, gazing at that orb with a smile.
``Isn't it clever of me to remember that?  Pooh!  we learned it all at
Miss Pinkerton's!  How calm the sea is, and how clear everything.  I
declare I can almost see the coast of France!'' and her bright green
eyes streamed out, and shot into the night as if they could see through
it.

``Do you know what I intend to do one morning?'' she said; ``I find I can
swim beautifully, and some day, when my Aunt Crawley's companion---old
Briggs, you know---you remember her---that hook-nosed woman, with the
long wisps of hair---when Briggs goes out to bathe, I intend to dive
under her awning, and insist on a reconciliation in the water.  Isn't
that a stratagem?''

George burst out laughing at the idea of this aquatic meeting. ``What's
the row there, you two?'' Rawdon shouted out, rattling the box.  Amelia
was making a fool of herself in an absurd hysterical manner, and
retired to her own room to whimper in private.

Our history is destined in this chapter to go backwards and forwards in
a very irresolute manner seemingly, and having conducted our story to
to-morrow presently, we shall immediately again have occasion to step
back to yesterday, so that the whole of the tale may get a hearing. As
you behold at her Majesty's drawing-room, the ambassadors' and high
dignitaries' carriages whisk off from a private door, while Captain
Jones's ladies are waiting for their fly: as you see in the Secretary
of the Treasury's antechamber, a half-dozen of petitioners waiting
patiently for their audience, and called out one by one, when suddenly
an Irish member or some eminent personage enters the apartment, and
instantly walks into Mr.\ Under-Secretary over the heads of all the
people present: so in the conduct of a tale, the romancer is obliged to
exercise this most partial sort of justice.  Although all the little
incidents must be heard, yet they must be put off when the great events
make their appearance; and surely such a circumstance as that which
brought Dobbin to Brighton, viz., the ordering out of the Guards and
the line to Belgium, and the mustering of the allied armies in that
country under the command of his Grace the Duke of Wellington---such a
dignified circumstance as that, I say, was entitled to the pas over all
minor occurrences whereof this history is composed mainly, and hence a
little trifling disarrangement and disorder was excusable and becoming.
We have only now advanced in time so far beyond Chapter XXII as to have
got our various characters up into their dressing-rooms before the
dinner, which took place as usual on the day of Dobbin's arrival.

George was too humane or too much occupied with the tie of his
neckcloth to convey at once all the news to Amelia which his comrade
had brought with him from London.  He came into her room, however,
holding the attorney's letter in his hand, and with so solemn and
important an air that his wife, always ingeniously on the watch for
calamity, thought the worst was about to befall, and running up to her
husband, besought her dearest George to tell her everything---he was
ordered abroad; there would be a battle next week---she knew there would.

Dearest George parried the question about foreign service, and with a
melancholy shake of the head said, ``No, Emmy; it isn't that:  it's not
myself I care about: it's you.  I have had bad news from my father.  He
refuses any communication with me; he has flung us off; and leaves us
to poverty.  I can rough it well enough; but you, my dear, how will you
bear it? read here.'' And he handed her over the letter.

Amelia, with a look of tender alarm in her eyes, listened to her noble
hero as he uttered the above generous sentiments, and sitting down on
the bed, read the letter which George gave her with such a pompous
martyr-like air.  Her face cleared up as she read the document,
however. The idea of sharing poverty and privation in company with the
beloved object is, as we have before said, far from being disagreeable
to a warm-hearted woman. The notion was actually pleasant to little
Amelia.  Then, as usual, she was ashamed of herself for feeling happy
at such an indecorous moment, and checked her pleasure, saying
demurely, ``O, George, how your poor heart must bleed at the idea of
being separated from your papa!''

``It does,'' said George, with an agonised countenance.

``But he can't be angry with you long,'' she continued. ``Nobody could,
I'm sure.  He must forgive you, my dearest, kindest husband.  O, I
shall never forgive myself if he does not.''

``What vexes me, my poor Emmy, is not my misfortune, but yours,'' George
said.  ``I don't care for a little poverty; and I think, without vanity,
I've talents enough to make my own way.''

``That you have,'' interposed his wife, who thought that war should
cease, and her husband should be made a general instantly.

``Yes, I shall make my way as well as another,'' Osborne went on; ``but
you, my dear girl, how can I bear your being deprived of the comforts
and station in society which my wife had a right to expect? My dearest
girl in barracks; the wife of a soldier in a marching regiment; subject
to all sorts of annoyance and privation! It makes me miserable.''

Emmy, quite at ease, as this was her husband's only cause of disquiet,
took his hand, and with a radiant face and smile began to warble that
stanza from the favourite song of ``Wapping Old Stairs,'' in which the
heroine, after rebuking her Tom for inattention, promises ``his trousers
to mend, and his grog too to make,'' if he will be constant and kind,
and not forsake her.  ``Besides,'' she said, after a pause, during which
she looked as pretty and happy as any young woman need, ``isn't two
thousand pounds an immense deal of money, George?''

George laughed at her naivete; and finally they went down to dinner,
Amelia clinging to George's arm, still warbling the tune of ``Wapping
Old Stairs,'' and more pleased and light of mind than she had been for
some days past.

Thus the repast, which at length came off, instead of being dismal, was
an exceedingly brisk and merry one. The excitement of the campaign
counteracted in George's mind the depression occasioned by the
disinheriting letter. Dobbin still kept up his character of rattle.  He
amused the company with accounts of the army in Belgium; where nothing
but fetes and gaiety and fashion were going on.  Then, having a
particular end in view, this dexterous captain proceeded to describe
Mrs.\ Major O'Dowd packing her own and her Major's wardrobe, and how his
best epaulets had been stowed into a tea canister, whilst her own
famous yellow turban, with the bird of paradise wrapped in brown paper,
was locked up in the Major's tin cocked-hat case, and wondered what
effect it would have at the French king's court at Ghent, or the great
military balls at Brussels.

``Ghent! Brussels!'' cried out Amelia with a sudden shock and start. ``Is
the regiment ordered away, George---is it ordered away?'' A look of
terror came over the sweet smiling face, and she clung to George as by
an instinct.

``Don't be afraid, dear,'' he said good-naturedly; ``it is but a twelve
hours' passage.  It won't hurt you.  You shall go, too, Emmy.''

``I intend to go,'' said Becky.  ``I'm on the staff.  General Tufto is a
great flirt of mine.  Isn't he, Rawdon?'' Rawdon laughed out with his
usual roar.  William Dobbin flushed up quite red.  ``She can't go,'' he
said; ``think of the---of the danger,'' he was going to add; but had not
all his conversation during dinner-time tended to prove there was none?
He became very confused and silent.

``I must and will go,'' Amelia cried with the greatest spirit; and
George, applauding her resolution, patted her under the chin, and asked
all the persons present if they ever saw such a termagant of a wife,
and agreed that the lady should bear him company.  ``We'll have Mrs.\ %
O'Dowd to chaperon you,'' he said.  What cared she so long as her
husband was near her?  Thus somehow the bitterness of a parting was
juggled away.  Though war and danger were in store, war and danger
might not befall for months to come.  There was a respite at any rate,
which made the timid little Amelia almost as happy as a full reprieve
would have done, and which even Dobbin owned in his heart was very
welcome.  For, to be permitted to see her was now the greatest
privilege and hope of his life, and he thought with himself secretly
how he would watch and protect her.  I wouldn't have let her go if I
had been married to her, he thought.  But George was the master, and
his friend did not think fit to remonstrate.

Putting her arm round her friend's waist, Rebecca at length carried
Amelia off from the dinner-table where so much business of importance
had been discussed, and left the gentlemen in a highly exhilarated
state, drinking and talking very gaily.

In the course of the evening Rawdon got a little family-note from his
wife, which, although he crumpled it up and burnt it instantly in the
candle, we had the good luck to read over Rebecca's shoulder. ``Great
news,'' she wrote.  ``Mrs.\ Bute is gone.  Get the money from Cupid
tonight, as he'll be off to-morrow most likely.  Mind this.---R.'' So
when the little company was about adjourning to coffee in the women's
apartment, Rawdon touched Osborne on the elbow, and said gracefully, ``I
say, Osborne, my boy, if quite convenient, I'll trouble you for that
'ere small trifle.'' It was not quite convenient, but nevertheless
George gave him a considerable present instalment in bank-notes from
his pocket-book, and a bill on his agents at a week's date, for the
remaining sum.

This matter arranged, George, and Jos, and Dobbin, held a council of
war over their cigars, and agreed that a general move should be made
for London in Jos's open carriage the next day.  Jos, I think, would
have preferred staying until Rawdon Crawley quitted Brighton, but
Dobbin and George overruled him, and he agreed to carry the party to
town, and ordered four horses, as became his dignity.  With these they
set off in state, after breakfast, the next day.  Amelia had risen very
early in the morning, and packed her little trunks with the greatest
alacrity, while Osborne lay in bed deploring that she had not a maid to
help her.  She was only too glad, however, to perform this office for
herself.  A dim uneasy sentiment about Rebecca filled her mind already;
and although they kissed each other most tenderly at parting, yet we
know what jealousy is; and Mrs.\ Amelia possessed that among other
virtues of her sex.

Besides these characters who are coming and going away, we must
remember that there were some other old friends of ours at Brighton;
Miss Crawley, namely, and the suite in attendance upon her.  Now,
although Rebecca and her husband were but at a few stones' throw of the
lodgings which the invalid Miss Crawley occupied, the old lady's door
remained as pitilessly closed to them as it had been heretofore in
London.  As long as she remained by the side of her sister-in-law,
Mrs.\ Bute Crawley took care that her beloved Matilda should not be
agitated by a meeting with her nephew.  When the spinster took her
drive, the faithful Mrs.\ Bute sate beside her in the carriage. When
Miss Crawley took the air in a chair, Mrs.\ Bute marched on one side of
the vehicle, whilst honest Briggs occupied the other wing. And if they
met Rawdon and his wife by chance---although the former constantly and
obsequiously took off his hat, the Miss-Crawley party passed him by
with such a frigid and killing indifference, that Rawdon began to
despair.

``We might as well be in London as here,'' Captain Rawdon often said,
with a downcast air.

``A comfortable inn in Brighton is better than a spunging-house in
Chancery Lane,'' his wife answered, who was of a more cheerful
temperament.  ``Think of those two aides-de-camp of Mr.\ Moses, the
sheriff's-officer, who watched our lodging for a week.  Our friends
here are very stupid, but Mr.\ Jos and Captain Cupid are better
companions than Mr.\ Moses's men, Rawdon, my love.''

``I wonder the writs haven't followed me down here,'' Rawdon continued,
still desponding.

``When they do, we'll find means to give them the slip,'' said dauntless
little Becky, and further pointed out to her husband the great comfort
and advantage of meeting Jos and Osborne, whose acquaintance had
brought to Rawdon Crawley a most timely little supply of ready money.

``It will hardly be enough to pay the inn bill,'' grumbled the Guardsman.

``Why need we pay it?'' said the lady, who had an answer for everything.

Through Rawdon's valet, who still kept up a trifling acquaintance with
the male inhabitants of Miss Crawley's servants' hall, and was
instructed to treat the coachman to drink whenever they met, old Miss
Crawley's movements were pretty well known by our young couple; and
Rebecca luckily bethought herself of being unwell, and of calling in
the same apothecary who was in attendance upon the spinster, so that
their information was on the whole tolerably complete.  Nor was Miss
Briggs, although forced to adopt a hostile attitude, secretly inimical
to Rawdon and his wife.  She was naturally of a kindly and forgiving
disposition.  Now that the cause of jealousy was removed, her dislike
for Rebecca disappeared also, and she remembered the latter's
invariable good words and good humour.  And, indeed, she and Mrs.\ %
Firkin, the lady's-maid, and the whole of Miss Crawley's household,
groaned under the tyranny of the triumphant Mrs.\ Bute.

As often will be the case, that good but imperious woman pushed her
advantages too far, and her successes quite unmercifully.  She had in
the course of a few weeks brought the invalid to such a state of
helpless docility, that the poor soul yielded herself entirely to her
sister's orders, and did not even dare to complain of her slavery to
Briggs or Firkin.  Mrs.\ Bute measured out the glasses of wine which
Miss Crawley was daily allowed to take, with irresistible accuracy,
greatly to the annoyance of Firkin and the butler, who found themselves
deprived of control over even the sherry-bottle. She apportioned the
sweetbreads, jellies, chickens; their quantity and order. Night and
noon and morning she brought the abominable drinks ordained by the
Doctor, and made her patient swallow them with so affecting an
obedience that Firkin said ``my poor Missus du take her physic like a
lamb.'' She prescribed the drive in the carriage or the ride in the
chair, and, in a word, ground down the old lady in her convalescence in
such a way as only belongs to your proper-managing, motherly moral
woman.  If ever the patient faintly resisted, and pleaded for a little
bit more dinner or a little drop less medicine, the nurse threatened
her with instantaneous death, when Miss Crawley instantly gave in.
``She's no spirit left in her,'' Firkin remarked to Briggs; ``she ain't
ave called me a fool these three weeks.'' Finally, Mrs.\ Bute had made up
her mind to dismiss the aforesaid honest lady's-maid, Mr.\ Bowls the
large confidential man, and Briggs herself, and to send for her
daughters from the Rectory, previous to removing the dear invalid
bodily to Queen's Crawley, when an odious accident happened which
called her away from duties so pleasing.  The Reverend Bute Crawley,
her husband, riding home one night, fell with his horse and broke his
collar-bone.  Fever and inflammatory symptoms set in, and Mrs.\ Bute was
forced to leave Sussex for Hampshire.  As soon as ever Bute was
restored, she promised to return to her dearest friend, and departed,
leaving the strongest injunctions with the household regarding their
behaviour to their mistress; and as soon as she got into the
Southampton coach, there was such a jubilee and sense of relief in all
Miss Crawley's house, as the company of persons assembled there had not
experienced for many a week before.  That very day Miss Crawley left
off her afternoon dose of medicine:  that afternoon Bowls opened an
independent bottle of sherry for himself and Mrs.\ Firkin:  that night
Miss Crawley and Miss Briggs indulged in a game of piquet instead of
one of Porteus's sermons.  It was as in the old nursery-story, when
the stick forgot to beat the dog, and the whole course of events
underwent a peaceful and happy revolution.

At a very early hour in the morning, twice or thrice a week, Miss
Briggs used to betake herself to a bathing-machine, and disport in the
water in a flannel gown and an oilskin cap.  Rebecca, as we have seen,
was aware of this circumstance, and though she did not attempt to storm
Briggs as she had threatened, and actually dive into that lady's
presence and surprise her under the sacredness of the awning, Mrs.\ %
Rawdon determined to attack Briggs as she came away from her bath,
refreshed and invigorated by her dip, and likely to be in good humour.

So getting up very early the next morning, Becky brought the telescope
in their sitting-room, which faced the sea, to bear upon the
bathing-machines on the beach; saw Briggs arrive, enter her box; and
put out to sea; and was on the shore just as the nymph of whom she came
in quest stepped out of the little caravan on to the shingles.  It was
a pretty picture:  the beach; the bathing-women's faces; the long line
of rocks and building were blushing and bright in the sunshine.
Rebecca wore a kind, tender smile on her face, and was holding out her
pretty white hand as Briggs emerged from the box.  What could Briggs do
but accept the salutation?

``Miss Sh---Mrs.\ Crawley,'' she said.

Mrs.\ Crawley seized her hand, pressed it to her heart, and with a
sudden impulse, flinging her arms round Briggs, kissed her
affectionately.  ``Dear, dear friend!'' she said, with a touch of such
natural feeling, that Miss Briggs of course at once began to melt, and
even the bathing-woman was mollified.

Rebecca found no difficulty in engaging Briggs in a long, intimate, and
delightful conversation.  Everything that had passed since the morning
of Becky's sudden departure from Miss Crawley's house in Park Lane up
to the present day, and Mrs.\ Bute's happy retreat, was discussed and
described by Briggs.  All Miss Crawley's symptoms, and the particulars
of her illness and medical treatment, were narrated by the confidante
with that fulness and accuracy which women delight in.  About their
complaints and their doctors do ladies ever tire of talking to each
other?  Briggs did not on this occasion; nor did Rebecca weary of
listening.  She was thankful, truly thankful, that the dear kind
Briggs, that the faithful, the invaluable Firkin, had been permitted to
remain with their benefactress through her illness.  Heaven bless her!
though she, Rebecca, had seemed to act undutifully towards Miss
Crawley; yet was not her fault a natural and excusable one? Could she
help giving her hand to the man who had won her heart?  Briggs, the
sentimental, could only turn up her eyes to heaven at this appeal, and
heave a sympathetic sigh, and think that she, too, had given away her
affections long years ago, and own that Rebecca was no very great
criminal.

``Can I ever forget her who so befriended the friendless orphan?  No,
though she has cast me off,'' the latter said, ``I shall never cease to
love her, and I would devote my life to her service.  As my own
benefactress, as my beloved Rawdon's adored relative, I love and admire
Miss Crawley, dear Miss Briggs, beyond any woman in the world, and next
to her I love all those who are faithful to her.  I would never have
treated Miss Crawley's faithful friends as that odious designing Mrs.\ %
Bute has done.  Rawdon, who was all heart,'' Rebecca continued,
``although his outward manners might seem rough and careless, had said a
hundred times, with tears in his eyes, that he blessed Heaven for
sending his dearest Aunty two such admirable nurses as her attached
Firkin and her admirable Miss Briggs.  Should the machinations of the
horrible Mrs.\ Bute end, as she too much feared they would, in banishing
everybody that Miss Crawley loved from her side, and leaving that poor
lady a victim to those harpies at the Rectory, Rebecca besought her
(Miss Briggs) to remember that her own home, humble as it was, was
always open to receive Briggs. Dear friend,'' she exclaimed, in a
transport of enthusiasm, ``some hearts can never forget benefits; all
women are not Bute Crawleys! Though why should I complain of her,''
Rebecca added; ``though I have been her tool and the victim to her arts,
do I not owe my dearest Rawdon to her?''  And Rebecca unfolded to Briggs
all Mrs.\ Bute's conduct at Queen's Crawley, which, though
unintelligible to her then, was clearly enough explained by the events
now---now that the attachment had sprung up which Mrs.\ Bute had
encouraged by a thousand artifices---now that two innocent people had
fallen into the snares which she had laid for them, and loved and
married and been ruined through her schemes.

It was all very true.  Briggs saw the stratagems as clearly as
possible.  Mrs.\ Bute had made the match between Rawdon and Rebecca.
Yet, though the latter was a perfectly innocent victim, Miss Briggs
could not disguise from her friend her fear that Miss Crawley's
affections were hopelessly estranged from Rebecca, and that the old
lady would never forgive her nephew for making so imprudent a marriage.

On this point Rebecca had her own opinion, and still kept up a good
heart.  If Miss Crawley did not forgive them at present, she might at
least relent on a future day.  Even now, there was only that puling,
sickly Pitt Crawley between Rawdon and a baronetcy; and should anything
happen to the former, all would be well.  At all events, to have Mrs.\ %
Bute's designs exposed, and herself well abused, was a satisfaction,
and might be advantageous to Rawdon's interest; and Rebecca, after an
hour's chat with her recovered friend, left her with the most tender
demonstrations of regard, and quite assured that the conversation they
had had together would be reported to Miss Crawley before many hours
were over.

This interview ended, it became full time for Rebecca to return to her
inn, where all the party of the previous day were assembled at a
farewell breakfast.  Rebecca took such a tender leave of Amelia as
became two women who loved each other as sisters; and having used her
handkerchief plentifully, and hung on her friend's neck as if they were
parting for ever, and waved the handkerchief (which was quite dry, by
the way) out of window, as the carriage drove off, she came back to the
breakfast table, and ate some prawns with a good deal of appetite,
considering her emotion; and while she was munching these delicacies,
explained to Rawdon what had occurred in her morning walk between
herself and Briggs.  Her hopes were very high:  she made her husband
share them.  She generally succeeded in making her husband share all
her opinions, whether melancholy or cheerful.

``You will now, if you please, my dear, sit down at the writing-table
and pen me a pretty little letter to Miss Crawley, in which you'll say
that you are a good boy, and that sort of thing.''  So Rawdon sate down,
and wrote off, ``Brighton, Thursday,'' and ``My dear Aunt,'' with great
rapidity: but there the gallant officer's imagination failed him.  He
mumbled the end of his pen, and looked up in his wife's face.  She
could not help laughing at his rueful countenance, and marching up and
down the room with her hands behind her, the little woman began to
dictate a letter, which he took down.

``Before quitting the country and commencing a campaign, which very
possibly may be fatal.''

``What?'' said Rawdon, rather surprised, but took the humour of the
phrase, and presently wrote it down with a grin.

``Which very possibly may be fatal, I have come hither---''

``Why not say come here, Becky?  Come here's grammar,'' the dragoon
interposed.

``I have come hither,'' Rebecca insisted, with a stamp of her foot, ``to
say farewell to my dearest and earliest friend.  I beseech you before I
go, not perhaps to return, once more to let me press the hand from
which I have received nothing but kindnesses all my life.''

``Kindnesses all my life,'' echoed Rawdon, scratching down the words, and
quite amazed at his own facility of composition.

``I ask nothing from you but that we should part not in anger.  I have
the pride of my family on some points, though not on all.  I married a
painter's daughter, and am not ashamed of the union.''

``No, run me through the body if I am!'' Rawdon ejaculated.

``You old booby,'' Rebecca said, pinching his ear and looking over to see
that he made no mistakes in spelling---``beseech is not spelt with an a,
and earliest is.''  So he altered these words, bowing to the superior
knowledge of his little Missis.

``I thought that you were aware of the progress of my attachment,''
Rebecca continued:  ``I knew that Mrs.\ Bute Crawley confirmed and
encouraged it.  But I make no reproaches.  I married a poor woman, and
am content to abide by what I have done.  Leave your property, dear
Aunt, as you will.  I shall never complain of the way in which you
dispose of it.  I would have you believe that I love you for yourself,
and not for money's sake.  I want to be reconciled to you ere I leave
England.  Let me, let me see you before I go.  A few weeks or months
hence it may be too late, and I cannot bear the notion of quitting the
country without a kind word of farewell from you.''

``She won't recognise my style in that,'' said Becky.  ``I made the
sentences short and brisk on purpose.'' And this authentic missive was
despatched under cover to Miss Briggs.

Old Miss Crawley laughed when Briggs, with great mystery, handed her
over this candid and simple statement.  ``We may read it now Mrs.\ Bute
is away,'' she said.  ``Read it to me, Briggs.''

When Briggs had read the epistle out, her patroness laughed more.
``Don't you see, you goose,'' she said to Briggs, who professed to be
much touched by the honest affection which pervaded the composition,
``don't you see that Rawdon never wrote a word of it.  He never wrote to
me without asking for money in his life, and all his letters are full
of bad spelling, and dashes, and bad grammar.  It is that little
serpent of a governess who rules him.'' They are all alike, Miss Crawley
thought in her heart.  They all want me dead, and are hankering for my
money.

``I don't mind seeing Rawdon,'' she added, after a pause, and in a tone
of perfect indifference.  ``I had just as soon shake hands with him as
not.  Provided there is no scene, why shouldn't we meet?  I don't mind.
But human patience has its limits; and mind, my dear, I respectfully
decline to receive Mrs.\ Rawdon---I can't support that quite''---and Miss
Briggs was fain to be content with this half-message of conciliation;
and thought that the best method of bringing the old lady and her
nephew together, was to warn Rawdon to be in waiting on the Cliff, when
Miss Crawley went out for her air in her chair.  There they met.  I
don't know whether Miss Crawley had any private feeling of regard or
emotion upon seeing her old favourite; but she held out a couple of
fingers to him with as smiling and good-humoured an air, as if they had
met only the day before.  And as for Rawdon, he turned as red as
scarlet, and wrung off Briggs's hand, so great was his rapture and his
confusion at the meeting. Perhaps it was interest that moved him:  or
perhaps affection:  perhaps he was touched by the change which the
illness of the last weeks had wrought in his aunt.

``The old girl has always acted like a trump to me,'' he said to his
wife, as he narrated the interview, ``and I felt, you know, rather
queer, and that sort of thing.  I walked by the side of the
what-dy'e-call-'em, you know, and to her own door, where Bowls came to help
her in.  And I wanted to go in very much, only---''

``YOU DIDN'T GO IN, Rawdon!'' screamed his wife.

``No, my dear; I'm hanged if I wasn't afraid when it came to the point.''

``You fool! you ought to have gone in, and never come out again,''
Rebecca said.

``Don't call me names,'' said the big Guardsman, sulkily. ``Perhaps I WAS
a fool, Becky, but you shouldn't say so''; and he gave his wife a look,
such as his countenance could wear when angered, and such as was not
pleasant to face.

``Well, dearest, to-morrow you must be on the look-out, and go and see
her, mind, whether she asks you or no,'' Rebecca said, trying to soothe
her angry yoke-mate.  On which he replied, that he would do exactly as
he liked, and would just thank her to keep a civil tongue in her
head---and the wounded husband went away, and passed the forenoon at the
billiard-room, sulky, silent, and suspicious.

But before the night was over he was compelled to give in, and own, as
usual, to his wife's superior prudence and foresight, by the most
melancholy confirmation of the presentiments which she had regarding
the consequences of the mistake which he had made.  Miss Crawley must
have had some emotion upon seeing him and shaking hands with him after
so long a rupture.  She mused upon the meeting a considerable time.
``Rawdon is getting very fat and old, Briggs,'' she said to her
companion.  ``His nose has become red, and he is exceedingly coarse in
appearance.  His marriage to that woman has hopelessly vulgarised him.
Mrs.\ Bute always said they drank together; and I have no doubt they do.
Yes:  he smelt of gin abominably.  I remarked it.  Didn't you?''

In vain Briggs interposed that Mrs.\ Bute spoke ill of everybody: and,
as far as a person in her humble position could judge, was an---

``An artful designing woman?  Yes, so she is, and she does speak ill of
every one---but I am certain that woman has made Rawdon drink. All those
low people do---''

``He was very much affected at seeing you, ma'am,'' the companion said;
``and I am sure, when you remember that he is going to the field of
danger---''

``How much money has he promised you, Briggs?'' the old spinster cried
out, working herself into a nervous rage---``there now, of course you
begin to cry.  I hate scenes.  Why am I always to be worried?  Go and
cry up in your own room, and send Firkin to me---no, stop, sit down and
blow your nose, and leave off crying, and write a letter to Captain
Crawley.'' Poor Briggs went and placed herself obediently at the
writing-book.  Its leaves were blotted all over with relics of the
firm, strong, rapid handwriting of the spinster's late amanuensis, Mrs.\ %
Bute Crawley.

``Begin 'My dear sir,' or 'Dear sir,' that will be better, and say you
are desired by Miss Crawley---no, by Miss Crawley's medical man, by Mr.\ %
Creamer, to state that my health is such that all strong emotions would
be dangerous in my present delicate condition---and that I must decline
any family discussions or interviews whatever. And thank him for coming
to Brighton, and so forth, and beg him not to stay any longer on my
account.  And, Miss Briggs, you may add that I wish him a bon voyage,
and that if he will take the trouble to call upon my lawyer's in Gray's
Inn Square, he will find there a communication for him.  Yes, that will
do; and that will make him leave Brighton.'' The benevolent Briggs
penned this sentence with the utmost satisfaction.

``To seize upon me the very day after Mrs.\ Bute was gone,'' the old lady
prattled on; ``it was too indecent. Briggs, my dear, write to Mrs.\ %
Crawley, and say SHE needn't come back.  No---she needn't---and she
shan't---and I won't be a slave in my own house---and I won't be starved
and choked with poison.  They all want to kill me---all---all''---and with
this the lonely old woman burst into a scream of hysterical tears.

The last scene of her dismal Vanity Fair comedy was fast approaching;
the tawdry lamps were going out one by one; and the dark curtain was
almost ready to descend.

That final paragraph, which referred Rawdon to Miss Crawley's solicitor
in London, and which Briggs had written so good-naturedly, consoled the
dragoon and his wife somewhat, after their first blank disappointment,
on reading the spinster's refusal of a reconciliation.  And it effected
the purpose for which the old lady had caused it to be written, by
making Rawdon very eager to get to London.

Out of Jos's losings and George Osborne's bank-notes, he paid his bill
at the inn, the landlord whereof does not probably know to this day how
doubtfully his account once stood.  For, as a general sends his baggage
to the rear before an action, Rebecca had wisely packed up all their
chief valuables and sent them off under care of George's servant, who
went in charge of the trunks on the coach back to London.  Rawdon and
his wife returned by the same conveyance next day.

``I should have liked to see the old girl before we went,'' Rawdon said.
``She looks so cut up and altered that I'm sure she can't last long.  I
wonder what sort of a cheque I shall have at Waxy's.  Two hundred---it
can't be less than two hundred---hey, Becky?''

In consequence of the repeated visits of the aides-de-camp of the
Sheriff of Middlesex, Rawdon and his wife did not go back to their
lodgings at Brompton, but put up at an inn.  Early the next morning,
Rebecca had an opportunity of seeing them as she skirted that suburb on
her road to old Mrs.\ Sedley's house at Fulham, whither she went to look
for her dear Amelia and her Brighton friends.  They were all off to
Chatham, thence to Harwich, to take shipping for Belgium with the
regiment---kind old Mrs.\ Sedley very much depressed and tearful,
solitary.  Returning from this visit, Rebecca found her husband, who
had been off to Gray's Inn, and learnt his fate.  He came back furious.

``By Jove, Becky,'' says he, ``she's only given me twenty pound!''

Though it told against themselves, the joke was too good, and Becky
burst out laughing at Rawdon's discomfiture.



\chapter{Between London and Chatham}

On quitting Brighton, our friend George, as became a person of rank and
fashion travelling in a barouche with four horses, drove in state to a
fine hotel in Cavendish Square, where a suite of splendid rooms, and a
table magnificently furnished with plate and surrounded by a half-dozen
of black and silent waiters, was ready to receive the young gentleman
and his bride.  George did the honours of the place with a princely air
to Jos and Dobbin; and Amelia, for the first time, and with exceeding
shyness and timidity, presided at what George called her own table.

George pooh-poohed the wine and bullied the waiters royally, and Jos
gobbled the turtle with immense satisfaction. Dobbin helped him to it;
for the lady of the house, before whom the tureen was placed, was so
ignorant of the contents, that she was going to help Mr.\ Sedley without
bestowing upon him either calipash or calipee.

The splendour of the entertainment, and the apartments in which it was
given, alarmed Mr.\ Dobbin, who remonstrated after dinner, when Jos was
asleep in the great chair.  But in vain he cried out against the
enormity of turtle and champagne that was fit for an archbishop. ``I've
always been accustomed to travel like a gentleman,'' George said, ``and,
damme, my wife shall travel like a lady.  As long as there's a shot in
the locker, she shall want for nothing,'' said the generous fellow,
quite pleased with himself for his magnificence of spirit.  Nor did
Dobbin try and convince him that Amelia's happiness was not centred in
turtle-soup.

A while after dinner, Amelia timidly expressed a wish to go and see her
mamma, at Fulham: which permission George granted her with some
grumbling.  And she tripped away to her enormous bedroom, in the centre
of which stood the enormous funereal bed, ``that the Emperor
Halixander's sister slep in when the allied sufferings was here,'' and
put on her little bonnet and shawl with the utmost eagerness and
pleasure.  George was still drinking claret when she returned to the
dining-room, and made no signs of moving.  ``Ar'n't you coming with me,
dearest?'' she asked him.  No; the ``dearest'' had ``business'' that night.
His man should get her a coach and go with her.  And the coach being at
the door of the hotel, Amelia made George a little disappointed curtsey
after looking vainly into his face once or twice, and went sadly down
the great staircase, Captain Dobbin after, who handed her into the
vehicle, and saw it drive away to its destination. The very valet was
ashamed of mentioning the address to the hackney-coachman before the
hotel waiters, and promised to instruct him when they got further on.

Dobbin walked home to his old quarters and the Slaughters', thinking
very likely that it would be delightful to be in that hackney-coach,
along with Mrs.\ Osborne. George was evidently of quite a different
taste; for when he had taken wine enough, he went off to half-price at
the play, to see Mr.\ Kean perform in Shylock.  Captain Osborne was a
great lover of the drama, and had himself performed high-comedy
characters with great distinction in several garrison theatrical
entertainments.  Jos slept on until long after dark, when he woke up
with a start at the motions of his servant, who was removing and
emptying the decanters on the table; and the hackney-coach stand was
again put into requisition for a carriage to convey this stout hero to
his lodgings and bed.

Mrs.\ Sedley, you may be sure, clasped her daughter to her heart with
all maternal eagerness and affection, running out of the door as the
carriage drew up before the little garden-gate, to welcome the weeping,
trembling, young bride.  Old Mr.\ Clapp, who was in his shirt-sleeves,
trimming the garden-plot, shrank back alarmed.  The Irish servant-lass
rushed up from the kitchen and smiled a ``God bless you.''  Amelia could
hardly walk along the flags and up the steps into the parlour.

How the floodgates were opened, and mother and daughter wept, when they
were together embracing each other in this sanctuary, may readily be
imagined by every reader who possesses the least sentimental turn.
When don't ladies weep?  At what occasion of joy, sorrow, or other
business of life, and, after such an event as a marriage, mother and
daughter were surely at liberty to give way to a sensibility which is
as tender as it is refreshing. About a question of marriage I have seen
women who hate each other kiss and cry together quite fondly. How much
more do they feel when they love!  Good mothers are married over again
at their daughters' weddings: and as for subsequent events, who does
not know how ultra-maternal grandmothers are?---in fact a woman, until
she is a grandmother, does not often really know what to be a mother
is.  Let us respect Amelia and her mamma whispering and whimpering and
laughing and crying in the parlour and the twilight.  Old Mr.\ Sedley
did.  HE had not divined who was in the carriage when it drove up. He
had not flown out to meet his daughter, though he kissed her very
warmly when she entered the room (where he was occupied, as usual, with
his papers and tapes and statements of accounts), and after sitting
with the mother and daughter for a short time, he very wisely left the
little apartment in their possession.

George's valet was looking on in a very supercilious manner at Mr.\ %
Clapp in his shirt-sleeves, watering his rose-bushes.  He took off his
hat, however, with much condescension to Mr.\ Sedley, who asked news
about his son-in-law, and about Jos's carriage, and whether his horses
had been down to Brighton, and about that infernal traitor Bonaparty,
and the war; until the Irish maid-servant came with a plate and a
bottle of wine, from which the old gentleman insisted upon helping the
valet.  He gave him a half-guinea too, which the servant pocketed with
a mixture of wonder and contempt.  ``To the health of your master and
mistress, Trotter,'' Mr.\ Sedley said, ``and here's something to drink
your health when you get home, Trotter.''

There were but nine days past since Amelia had left that little cottage
and home---and yet how far off the time seemed since she had bidden it
farewell.  What a gulf lay between her and that past life. She could
look back to it from her present standing-place, and contemplate,
almost as another being, the young unmarried girl absorbed in her love,
having no eyes but for one special object, receiving parental affection
if not ungratefully, at least indifferently, and as if it were her
due---her whole heart and thoughts bent on the accomplishment of one
desire.  The review of those days, so lately gone yet so far away,
touched her with shame; and the aspect of the kind parents filled her
with tender remorse. Was the prize gained---the heaven of life---and the
winner still doubtful and unsatisfied?  As his hero and heroine pass
the matrimonial barrier, the novelist generally drops the curtain, as
if the drama were over then:  the doubts and struggles of life ended:
as if, once landed in the marriage country, all were green and pleasant
there:  and wife and husband had nothing to do but to link each other's
arms together, and wander gently downwards towards old age in happy and
perfect fruition.  But our little Amelia was just on the bank of her
new country, and was already looking anxiously back towards the sad
friendly figures waving farewell to her across the stream, from the
other distant shore.

In honour of the young bride's arrival, her mother thought it necessary
to prepare I don't know what festive entertainment, and after the first
ebullition of talk, took leave of Mrs.\ George Osborne for a while, and
dived down to the lower regions of the house to a sort of
kitchen-parlour (occupied by Mr.\ and Mrs.\ Clapp, and in the evening,
when her dishes were washed and her curl-papers removed, by Miss
Flannigan, the Irish servant), there to take measures for the preparing
of a magnificent ornamented tea.  All people have their ways of
expressing kindness, and it seemed to Mrs.\ Sedley that a muffin and a
quantity of orange marmalade spread out in a little cut-glass saucer
would be peculiarly agreeable refreshments to Amelia in her most
interesting situation.

While these delicacies were being transacted below, Amelia, leaving the
drawing-room, walked upstairs and found herself, she scarce knew how,
in the little room which she had occupied before her marriage, and in
that very chair in which she had passed so many bitter hours. She sank
back in its arms as if it were an old friend; and fell to thinking over
the past week, and the life beyond it.  Already to be looking sadly and
vaguely back: always to be pining for something which, when obtained,
brought doubt and sadness rather than pleasure; here was the lot of our
poor little creature and harmless lost wanderer in the great struggling
crowds of Vanity Fair.

Here she sate, and recalled to herself fondly that image of George to
which she had knelt before marriage.  Did she own to herself how
different the real man was from that superb young hero whom she had
worshipped?  It requires many, many years---and a man must be very bad
indeed---before a woman's pride and vanity will let her own to such a
confession.  Then Rebecca's twinkling green eyes and baleful smile
lighted upon her, and filled her with dismay.  And so she sate for
awhile indulging in her usual mood of selfish brooding, in that very
listless melancholy attitude in which the honest maid-servant had found
her, on the day when she brought up the letter in which George renewed
his offer of marriage.

She looked at the little white bed, which had been hers a few days
before, and thought she would like to sleep in it that night, and wake,
as formerly, with her mother smiling over her in the morning: Then she
thought with terror of the great funereal damask pavilion in the vast
and dingy state bedroom, which was awaiting her at the grand hotel in
Cavendish Square.  Dear little white bed! how many a long night had she
wept on its pillow! How she had despaired and hoped to die there; and
now were not all her wishes accomplished, and the lover of whom she had
despaired her own for ever?  Kind mother! how patiently and tenderly
she had watched round that bed! She went and knelt down by the bedside;
and there this wounded and timorous, but gentle and loving soul, sought
for consolation, where as yet, it must be owned, our little girl had
but seldom looked for it.  Love had been her faith hitherto; and the
sad, bleeding disappointed heart began to feel the want of another
consoler.

Have we a right to repeat or to overhear her prayers? These, brother,
are secrets, and out of the domain of Vanity Fair, in which our story
lies.

But this may be said, that when the tea was finally announced, our
young lady came downstairs a great deal more cheerful; that she did not
despond, or deplore her fate, or think about George's coldness, or
Rebecca's eyes, as she had been wont to do of late.  She went
downstairs, and kissed her father and mother, and talked to the old
gentleman, and made him more merry than he had been for many a day. She
sate down at the piano which Dobbin had bought for her, and sang over
all her father's favourite old songs.  She pronounced the tea to be
excellent, and praised the exquisite taste in which the marmalade was
arranged in the saucers.  And in determining to make everybody else
happy, she found herself so; and was sound asleep in the great funereal
pavilion, and only woke up with a smile when George arrived from the
theatre.

For the next day, George had more important ``business'' to transact than
that which took him to see Mr.\ Kean in Shylock.  Immediately on his
arrival in London he had written off to his father's solicitors,
signifying his royal pleasure that an interview should take place
between them on the morrow.  His hotel bill, losses at billiards and
cards to Captain Crawley had almost drained the young man's purse,
which wanted replenishing before he set out on his travels, and he had
no resource but to infringe upon the two thousand pounds which the
attorneys were commissioned to pay over to him.  He had a perfect
belief in his own mind that his father would relent before very long.
How could any parent be obdurate for a length of time against such a
paragon as he was?  If his mere past and personal merits did not
succeed in mollifying his father, George determined that he would
distinguish himself so prodigiously in the ensuing campaign that the
old gentleman must give in to him.  And if not? Bah! the world was
before him.  His luck might change at cards, and there was a deal of
spending in two thousand pounds.

So he sent off Amelia once more in a carriage to her mamma, with strict
orders and carte blanche to the two ladies to purchase everything
requisite for a lady of Mrs.\ George Osborne's fashion, who was going on
a foreign tour.  They had but one day to complete the outfit, and it
may be imagined that their business therefore occupied them pretty
fully.  In a carriage once more, bustling about from milliner to
linen-draper, escorted back to the carriage by obsequious shopmen or
polite owners, Mrs.\ Sedley was herself again almost, and sincerely
happy for the first time since their misfortunes.  Nor was Mrs.\ Amelia
at all above the pleasure of shopping, and bargaining, and seeing and
buying pretty things. (Would any man, the most philosophic, give
twopence for a woman who was?)  She gave herself a little treat,
obedient to her husband's orders, and purchased a quantity of lady's
gear, showing a great deal of taste and elegant discernment, as all the
shopfolks said.

And about the war that was ensuing, Mrs.\ Osborne was not much alarmed;
Bonaparty was to be crushed almost without a struggle. Margate packets
were sailing every day, filled with men of fashion and ladies of note,
on their way to Brussels and Ghent.  People were going not so much to a
war as to a fashionable tour.  The newspapers laughed the wretched
upstart and swindler to scorn.  Such a Corsican wretch as that
withstand the armies of Europe and the genius of the immortal
Wellington!  Amelia held him in utter contempt; for it needs not to be
said that this soft and gentle creature took her opinions from those
people who surrounded her, such fidelity being much too humble-minded
to think for itself. Well, in a word, she and her mother performed a
great day's shopping, and she acquitted herself with considerable
liveliness and credit on this her first appearance in the genteel world
of London.

George meanwhile, with his hat on one side, his elbows squared, and his
swaggering martial air, made for Bedford Row, and stalked into the
attorney's offices as if he was lord of every pale-faced clerk who was
scribbling there.  He ordered somebody to inform Mr.\ Higgs that Captain
Osborne was waiting, in a fierce and patronizing way, as if the pekin
of an attorney, who had thrice his brains, fifty times his money, and a
thousand times his experience, was a wretched underling who should
instantly leave all his business in life to attend on the Captain's
pleasure.  He did not see the sneer of contempt which passed all round
the room, from the first clerk to the articled gents, from the articled
gents to the ragged writers and white-faced runners, in clothes too
tight for them, as he sate there tapping his boot with his cane, and
thinking what a parcel of miserable poor devils these were.  The
miserable poor devils knew all about his affairs.  They talked about
them over their pints of beer at their public-house clubs to other
clerks of a night. Ye gods, what do not attorneys and attorneys' clerks
know in London! Nothing is hidden from their inquisition, and their
families mutely rule our city.

Perhaps George expected, when he entered Mr.\ Higgs's apartment, to find
that gentleman commissioned to give him some message of compromise or
conciliation from his father; perhaps his haughty and cold demeanour
was adopted as a sign of his spirit and resolution: but if so, his
fierceness was met by a chilling coolness and indifference on the
attorney's part, that rendered swaggering absurd.  He pretended to be
writing at a paper, when the Captain entered.  ``Pray, sit down, sir,''
said he, ``and I will attend to your little affair in a moment.  Mr.\ %
Poe, get the release papers, if you please''; and then he fell to
writing again.

Poe having produced those papers, his chief calculated the amount of
two thousand pounds stock at the rate of the day; and asked Captain
Osborne whether he would take the sum in a cheque upon the bankers, or
whether he should direct the latter to purchase stock to that amount.
``One of the late Mrs.\ Osborne's trustees is out of town,'' he said
indifferently, ``but my client wishes to meet your wishes, and have done
with the business as quick as possible.''

``Give me a cheque, sir,'' said the Captain very surlily. ``Damn the
shillings and halfpence, sir,'' he added, as the lawyer was making out
the amount of the draft; and, flattering himself that by this stroke of
magnanimity he had put the old quiz to the blush, he stalked out of the
office with the paper in his pocket.

``That chap will be in gaol in two years,'' Mr.\ Higgs said to Mr.\ Poe.

``Won't O. come round, sir, don't you think?''

``Won't the monument come round,'' Mr.\ Higgs replied.

``He's going it pretty fast,'' said the clerk.  ``He's only married a
week, and I saw him and some other military chaps handing Mrs.\ %
Highflyer to her carriage after the play.'' And then another case was
called, and Mr.\ George Osborne thenceforth dismissed from these worthy
gentlemen's memory.

The draft was upon our friends Hulker and Bullock of Lombard Street, to
whose house, still thinking he was doing business, George bent his way,
and from whom he received his money.  Frederick Bullock, Esq., whose
yellow face was over a ledger, at which sate a demure clerk, happened
to be in the banking-room when George entered. His yellow face turned
to a more deadly colour when he saw the Captain, and he slunk back
guiltily into the inmost parlour.  George was too busy gloating over
the money (for he had never had such a sum before), to mark the
countenance or flight of the cadaverous suitor of his sister.

Fred Bullock told old Osborne of his son's appearance and conduct. ``He
came in as bold as brass,'' said Frederick.  ``He has drawn out every
shilling.  How long will a few hundred pounds last such a chap as
that?'' Osborne swore with a great oath that he little cared when or how
soon he spent it.  Fred dined every day in Russell Square now.  But
altogether, George was highly pleased with his day's business.  All his
own baggage and outfit was put into a state of speedy preparation, and
he paid Amelia's purchases with cheques on his agents, and with the
splendour of a lord.



\chapter{In Which Amelia Joins Her Regiment}

When Jos's fine carriage drove up to the inn door at Chatham, the first
face which Amelia recognized was the friendly countenance of Captain
Dobbin, who had been pacing the street for an hour past in expectation
of his friends' arrival.  The Captain, with shells on his frockcoat,
and a crimson sash and sabre, presented a military appearance, which
made Jos quite proud to be able to claim such an acquaintance, and the
stout civilian hailed him with a cordiality very different from the
reception which Jos vouchsafed to his friend in Brighton and Bond
Street.

Along with the Captain was Ensign Stubble; who, as the barouche neared
the inn, burst out with an exclamation of ``By Jove! what a pretty
girl''; highly applauding Osborne's choice.  Indeed, Amelia dressed in
her wedding-pelisse and pink ribbons, with a flush in her face,
occasioned by rapid travel through the open air, looked so fresh and
pretty, as fully to justify the Ensign's compliment. Dobbin liked him
for making it.  As he stepped forward to help the lady out of the
carriage, Stubble saw what a pretty little hand she gave him, and what
a sweet pretty little foot came tripping down the step.  He blushed
profusely, and made the very best bow of which he was capable; to which
Amelia, seeing the number of the the regiment embroidered on the
Ensign's cap, replied with a blushing smile, and a curtsey on her part;
which finished the young Ensign on the spot. Dobbin took most kindly to
Mr.\ Stubble from that day, and encouraged him to talk about Amelia in
their private walks, and at each other's quarters.  It became the
fashion, indeed, among all the honest young fellows of the ---th to
adore and admire Mrs.\ Osborne.  Her simple artless behaviour, and
modest kindness of demeanour, won all their unsophisticated hearts; all
which simplicity and sweetness are quite impossible to describe in
print.  But who has not beheld these among women, and recognised the
presence of all sorts of qualities in them, even though they say no
more to you than that they are engaged to dance the next quadrille, or
that it is very hot weather? George, always the champion of his
regiment, rose immensely in the opinion of the youth of the corps, by
his gallantry in marrying this portionless young creature, and by his
choice of such a pretty kind partner.

In the sitting-room which was awaiting the travellers, Amelia, to her
surprise, found a letter addressed to Mrs.\ Captain Osborne.  It was a
triangular billet, on pink paper, and sealed with a dove and an olive
branch, and a profusion of light blue sealing wax, and it was written
in a very large, though undecided female hand.

``It's Peggy O'Dowd's fist,'' said George, laughing.  ``I know it by the
kisses on the seal.'' And in fact, it was a note from Mrs.\ Major O'Dowd,
requesting the pleasure of Mrs.\ Osborne's company that very evening to
a small friendly party.  ``You must go,'' George said. ``You will make
acquaintance with the regiment there.  O'Dowd goes in command of the
regiment, and Peggy goes in command.''

But they had not been for many minutes in the enjoyment of Mrs.\ %
O'Dowd's letter, when the door was flung open, and a stout jolly lady,
in a riding-habit, followed by a couple of officers of Ours, entered
the room.

``Sure, I couldn't stop till tay-time.  Present me, Garge, my dear
fellow, to your lady.  Madam, I'm deloighted to see ye; and to present
to you me husband, Meejor O'Dowd''; and with this, the jolly lady in the
riding-habit grasped Amelia's hand very warmly, and the latter knew at
once that the lady was before her whom her husband had so often laughed
at.  ``You've often heard of me from that husband of yours,'' said the
lady, with great vivacity.

``You've often heard of her,'' echoed her husband, the Major.

Amelia answered, smiling, ``that she had.''

``And small good he's told you of me,'' Mrs.\ O'Dowd replied; adding that
``George was a wicked divvle.''

``That I'll go bail for,'' said the Major, trying to look knowing, at
which George laughed; and Mrs.\ O'Dowd, with a tap of her whip, told the
Major to be quiet; and then requested to be presented in form to Mrs.\ %
Captain Osborne.

``This, my dear,'' said George with great gravity, ``is my very good,
kind, and excellent friend, Auralia Margaretta, otherwise called Peggy.''

``Faith, you're right,'' interposed the Major.

``Otherwise called Peggy, lady of Major Michael O'Dowd, of our regiment,
and daughter of Fitzjurld Ber'sford de Burgo Malony of Glenmalony,
County Kildare.''

``And Muryan Squeer, Doblin,'' said the lady with calm superiority.

``And Muryan Square, sure enough,'' the Major whispered.

``'Twas there ye coorted me, Meejor dear,'' the lady said; and the Major
assented to this as to every other proposition which was made generally
in company.

Major O'Dowd, who had served his sovereign in every quarter of the
world, and had paid for every step in his profession by some more than
equivalent act of daring and gallantry, was the most modest, silent,
sheep-faced and meek of little men, and as obedient to his wife as if
he had been her tay-boy.  At the mess-table he sat silently, and drank
a great deal.  When full of liquor, he reeled silently home.  When he
spoke, it was to agree with everybody on every conceivable point; and
he passed through life in perfect ease and good-humour.  The hottest
suns of India never heated his temper; and the Walcheren ague never
shook it.  He walked up to a battery with just as much indifference as
to a dinner-table; had dined on horse-flesh and turtle with equal
relish and appetite; and had an old mother, Mrs.\ O'Dowd of O'Dowdstown
indeed, whom he had never disobeyed but when he ran away and enlisted,
and when he persisted in marrying that odious Peggy Malony.

Peggy was one of five sisters, and eleven children of the noble house
of Glenmalony; but her husband, though her own cousin, was of the
mother's side, and so had not the inestimable advantage of being allied
to the Malonys, whom she believed to be the most famous family in the
world.  Having tried nine seasons at Dublin and two at Bath and
Cheltenham, and not finding a partner for life, Miss Malony ordered her
cousin Mick to marry her when she was about thirty-three years of age;
and the honest fellow obeying, carried her off to the West Indies, to
preside over the ladies of the ---th regiment, into which he had just
exchanged.

Before Mrs.\ O'Dowd was half an hour in Amelia's (or indeed in anybody
else's) company, this amiable lady told all her birth and pedigree to
her new friend.  ``My dear,'' said she, good-naturedly, ``it was my
intention that Garge should be a brother of my own, and my sister
Glorvina would have suited him entirely.  But as bygones are bygones,
and he was engaged to yourself, why, I'm determined to take you as a
sister instead, and to look upon you as such, and to love you as one of
the family.  Faith, you've got such a nice good-natured face and way
widg you, that I'm sure we'll agree; and that you'll be an addition to
our family anyway.''

``'Deed and she will,'' said O'Dowd, with an approving air, and Amelia
felt herself not a little amused and grateful to be thus suddenly
introduced to so large a party of relations.

``We're all good fellows here,'' the Major's lady continued. ``There's not
a regiment in the service where you'll find a more united society nor a
more agreeable mess-room.  There's no quarrelling, bickering,
slandthering, nor small talk amongst us.  We all love each other.''

``Especially Mrs.\ Magenis,'' said George, laughing.

``Mrs.\ Captain Magenis and me has made up, though her treatment of me
would bring me gray hairs with sorrow to the grave.''

``And you with such a beautiful front of black, Peggy, my dear,'' the
Major cried.

``Hould your tongue, Mick, you booby.  Them husbands are always in the
way, Mrs.\ Osborne, my dear; and as for my Mick, I often tell him he
should never open his mouth but to give the word of command, or to put
meat and drink into it.  I'll tell you about the regiment, and warn you
when we're alone.  Introduce me to your brother now; sure he's a mighty
fine man, and reminds me of me cousin, Dan Malony (Malony of
Ballymalony, my dear, you know who mar'ied Ophalia Scully, of
Oystherstown, own cousin to Lord Poldoody).  Mr.\ Sedley, sir, I'm
deloighted to be made known te ye.  I suppose you'll dine at the mess
to-day.  (Mind that divvle of a docther, Mick, and whatever ye du, keep
yourself sober for me party this evening.)''

``It's the 150th gives us a farewell dinner, my love,'' interposed the
Major, ``but we'll easy get a card for Mr.\ Sedley.''

``Run Simple (Ensign Simple, of Ours, my dear Amelia. I forgot to
introjuice him to ye).  Run in a hurry, with Mrs.\ Major O'Dowd's
compliments to Colonel Tavish, and Captain Osborne has brought his
brothernlaw down, and will bring him to the 150th mess at five o'clock
sharp---when you and I, my dear, will take a snack here, if you like.''
Before Mrs.\ O'Dowd's speech was concluded, the young Ensign was
trotting downstairs on his commission.

``Obedience is the soul of the army.  We will go to our duty while Mrs.\ %
O'Dowd will stay and enlighten you, Emmy,'' Captain Osborne said; and
the two gentlemen, taking each a wing of the Major, walked out with
that officer, grinning at each other over his head.

And, now having her new friend to herself, the impetuous Mrs.\ O'Dowd
proceeded to pour out such a quantity of information as no poor little
woman's memory could ever tax itself to bear.  She told Amelia a
thousand particulars relative to the very numerous family of which the
amazed young lady found herself a member.  ``Mrs.\ Heavytop, the
Colonel's wife, died in Jamaica of the yellow faver and a broken heart
comboined, for the horrud old Colonel, with a head as bald as a
cannon-ball, was making sheep's eyes at a half-caste girl there.  Mrs.\ %
Magenis, though without education, was a good woman, but she had the
divvle's tongue, and would cheat her own mother at whist.  Mrs.\ Captain
Kirk must turn up her lobster eyes forsooth at the idea of an honest
round game (wherein me fawther, as pious a man as ever went to church,
me uncle Dane Malony, and our cousin the Bishop, took a hand at loo, or
whist, every night of their lives).  Nayther of 'em's goin' with the
regiment this time,'' Mrs.\ O'Dowd added.  ``Fanny Magenis stops with her
mother, who sells small coal and potatoes, most likely, in
Islington-town, hard by London, though she's always bragging of her
father's ships, and pointing them out to us as they go up the river:
and Mrs.\ Kirk and her children will stop here in Bethesda Place, to be
nigh to her favourite preacher, Dr. Ramshorn.  Mrs.\ Bunny's in an
interesting situation---faith, and she always is, then---and has given
the Lieutenant seven already.  And Ensign Posky's wife, who joined two
months before you, my dear, has quarl'd with Tom Posky a score of
times, till you can hear'm all over the bar'ck (they say they're come
to broken pleets, and Tom never accounted for his black oi), and she'll
go back to her mother, who keeps a ladies' siminary at Richmond---bad
luck to her for running away from it!  Where did ye get your finishing,
my dear?  I had moin, and no expince spared, at Madame Flanahan's, at
Ilyssus Grove, Booterstown, near Dublin, wid a Marchioness to teach us
the true Parisian pronunciation, and a retired Mejor-General of the
French service to put us through the exercise.''

Of this incongruous family our astonished Amelia found herself all of a
sudden a member:  with Mrs.\ O'Dowd as an elder sister.  She was
presented to her other female relations at tea-time, on whom, as she
was quiet, good-natured, and not too handsome, she made rather an
agreeable impression until the arrival of the gentlemen from the mess
of the 150th, who all admired her so, that her sisters began, of
course, to find fault with her.

``I hope Osborne has sown his wild oats,'' said Mrs.\ Magenis to Mrs.\ %
Bunny.  ``If a reformed rake makes a good husband, sure it's she will
have the fine chance with Garge,'' Mrs.\ O'Dowd remarked to Posky, who
had lost her position as bride in the regiment, and was quite angry
with the usurper.  And as for Mrs.\ Kirk:  that disciple of Dr. Ramshorn
put one or two leading professional questions to Amelia, to see whether
she was awakened, whether she was a professing Christian and so forth,
and finding from the simplicity of Mrs.\ Osborne's replies that she was
yet in utter darkness, put into her hands three little penny books with
pictures, viz., the ``Howling Wilderness,'' the ``Washerwoman of
Wandsworth Common,'' and the ``British Soldier's best Bayonet,'' which,
bent upon awakening her before she slept, Mrs.\ Kirk begged Amelia to
read that night ere she went to bed.

But all the men, like good fellows as they were, rallied round their
comrade's pretty wife, and paid her their court with soldierly
gallantry.  She had a little triumph, which flushed her spirits and
made her eyes sparkle. George was proud of her popularity, and pleased
with the manner (which was very gay and graceful, though naive and a
little timid) with which she received the gentlemen's attentions, and
answered their compliments.  And he in his uniform---how much handsomer
he was than any man in the room!  She felt that he was affectionately
watching her, and glowed with pleasure at his kindness.  ``I will make
all his friends welcome,'' she resolved in her heart.  ``I will love all
as I love him.  I will always try and be gay and good-humoured and make
his home happy.''

The regiment indeed adopted her with acclamation. The Captains
approved, the Lieutenants applauded, the Ensigns admired.  Old Cutler,
the Doctor, made one or two jokes, which, being professional, need not
be repeated; and Cackle, the Assistant M.D. of Edinburgh, condescended
to examine her upon leeterature, and tried her with his three best
French quotations.  Young Stubble went about from man to man
whispering, ``Jove, isn't she a pretty gal?'' and never took his eyes off
her except when the negus came in.

As for Captain Dobbin, he never so much as spoke to her during the
whole evening.  But he and Captain Porter of the 150th took home Jos to
the hotel, who was in a very maudlin state, and had told his tiger-hunt
story with great effect, both at the mess-table and at the soiree, to
Mrs.\ O'Dowd in her turban and bird of paradise. Having put the
Collector into the hands of his servant, Dobbin loitered about, smoking
his cigar before the inn door. George had meanwhile very carefully
shawled his wife, and brought her away from Mrs.\ O'Dowd's after a
general handshaking from the young officers, who accompanied her to the
fly, and cheered that vehicle as it drove off.  So Amelia gave Dobbin
her little hand as she got out of the carriage, and rebuked him
smilingly for not having taken any notice of her all night.

The Captain continued that deleterious amusement of smoking, long after
the inn and the street were gone to bed.  He watched the lights vanish
from George's sitting-room windows, and shine out in the bedroom close
at hand.  It was almost morning when he returned to his own quarters.
He could hear the cheering from the ships in the river, where the
transports were already taking in their cargoes preparatory to dropping
down the Thames.



\chapter{In Which Amelia Invades the Low Countries}

The regiment with its officers was to be transported in ships provided
by His Majesty's government for the occasion:  and in two days after
the festive assembly at Mrs.\ O'Dowd's apartments, in the midst of
cheering from all the East India ships in the river, and the military
on shore, the band playing ``God Save the King,'' the officers waving
their hats, and the crews hurrahing gallantly, the transports went down
the river and proceeded under convoy to Ostend. Meanwhile the gallant
Jos had agreed to escort his sister and the Major's wife, the bulk of
whose goods and chattels, including the famous bird of paradise and
turban, were with the regimental baggage: so that our two heroines
drove pretty much unencumbered to Ramsgate, where there were plenty of
packets plying, in one of which they had a speedy passage to Ostend.

That period of Jos's life which now ensued was so full of incident,
that it served him for conversation for many years after, and even the
tiger-hunt story was put aside for more stirring narratives which he
had to tell about the great campaign of Waterloo.  As soon as he had
agreed to escort his sister abroad, it was remarked that he ceased
shaving his upper lip.  At Chatham he followed the parades and drills
with great assiduity.  He listened with the utmost attention to the
conversation of his brother officers (as he called them in after days
sometimes), and learned as many military names as he could. In these
studies the excellent Mrs.\ O'Dowd was of great assistance to him; and
on the day finally when they embarked on board the Lovely Rose, which
was to carry them to their destination, he made his appearance in a
braided frock-coat and duck trousers, with a foraging cap ornamented
with a smart gold band.  Having his carriage with him, and informing
everybody on board confidentially that he was going to join the Duke of
Wellington's army, folks mistook him for a great personage, a
commissary-general, or a government courier at the very least.

He suffered hugely on the voyage, during which the ladies were likewise
prostrate; but Amelia was brought to life again as the packet made
Ostend, by the sight of the transports conveying her regiment, which
entered the harbour almost at the same time with the Lovely Rose. Jos
went in a collapsed state to an inn, while Captain Dobbin escorted the
ladies, and then busied himself in freeing Jos's carriage and luggage
from the ship and the custom-house, for Mr.\ Jos was at present without
a servant, Osborne's man and his own pampered menial having conspired
together at Chatham, and refused point-blank to cross the water.  This
revolt, which came very suddenly, and on the last day, so alarmed Mr.\ %
Sedley, junior, that he was on the point of giving up the expedition,
but Captain Dobbin (who made himself immensely officious in the
business, Jos said), rated him and laughed at him soundly:  the
mustachios were grown in advance, and Jos finally was persuaded to
embark.  In place of the well-bred and well-fed London domestics, who
could only speak English, Dobbin procured for Jos's party a swarthy
little Belgian servant who could speak no language at all; but who, by
his bustling behaviour, and by invariably addressing Mr.\ Sedley as ``My
lord,'' speedily acquired that gentleman's favour.  Times are altered at
Ostend now; of the Britons who go thither, very few look like lords, or
act like those members of our hereditary aristocracy.  They seem for
the most part shabby in attire, dingy of linen, lovers of billiards and
brandy, and cigars and greasy ordinaries.

But it may be said as a rule, that every Englishman in the Duke of
Wellington's army paid his way.  The remembrance of such a fact surely
becomes a nation of shopkeepers.  It was a blessing for a
commerce-loving country to be overrun by such an army of customers: and
to have such creditable warriors to feed.  And the country which they
came to protect is not military.  For a long period of history they
have let other people fight there.  When the present writer went to
survey with eagle glance the field of Waterloo, we asked the conductor
of the diligence, a portly warlike-looking veteran, whether he had been
at the battle.  ``Pas si bete''---such an answer and sentiment as no
Frenchman would own to---was his reply.  But, on the other hand, the
postilion who drove us was a Viscount, a son of some bankrupt Imperial
General, who accepted a pennyworth of beer on the road.  The moral is
surely a good one.

This flat, flourishing, easy country never could have looked more rich
and prosperous than in that opening summer of 1815, when its green
fields and quiet cities were enlivened by multiplied red-coats: when
its wide chaussees swarmed with brilliant English equipages: when its
great canal-boats, gliding by rich pastures and pleasant quaint old
villages, by old chateaux lying amongst old trees, were all crowded
with well-to-do English travellers: when the soldier who drank at the
village inn, not only drank, but paid his score; and Donald, the
Highlander, billeted in the Flemish farm-house, rocked the baby's
cradle, while Jean and Jeannette were out getting in the hay.  As our
painters are bent on military subjects just now, I throw out this as a
good subject for the pencil, to illustrate the principle of an honest
English war.  All looked as brilliant and harmless as a Hyde Park
review.  Meanwhile, Napoleon screened behind his curtain of
frontier-fortresses, was preparing for the outbreak which was to drive
all these orderly people into fury and blood; and lay so many of them
low.

Everybody had such a perfect feeling of confidence in the leader (for
the resolute faith which the Duke of Wellington had inspired in the
whole English nation was as intense as that more frantic enthusiasm
with which at one time the French regarded Napoleon), the country
seemed in so perfect a state of orderly defence, and the help at hand
in case of need so near and overwhelming, that alarm was unknown, and
our travellers, among whom two were naturally of a very timid sort,
were, like all the other multiplied English tourists, entirely at ease.
The famous regiment, with so many of whose officers we have made
acquaintance, was drafted in canal boats to Bruges and Ghent, thence to
march to Brussels. Jos accompanied the ladies in the public boats; the
which all old travellers in Flanders must remember for the luxury and
accommodation they afforded.  So prodigiously good was the eating and
drinking on board these sluggish but most comfortable vessels, that
there are legends extant of an English traveller, who, coming to
Belgium for a week, and travelling in one of these boats, was so
delighted with the fare there that he went backwards and forwards from
Ghent to Bruges perpetually until the railroads were invented, when he
drowned himself on the last trip of the passage-boat.  Jos's death was
not to be of this sort, but his comfort was exceeding, and Mrs.\ O'Dowd
insisted that he only wanted her sister Glorvina to make his happiness
complete.  He sate on the roof of the cabin all day drinking Flemish
beer, shouting for Isidor, his servant, and talking gallantly to the
ladies.

His courage was prodigious.  ``Boney attack us!'' he cried.  ``My dear
creature, my poor Emmy, don't be frightened.  There's no danger. The
allies will be in Paris in two months, I tell you; when I'll take you
to dine in the Palais Royal, by Jove!  There are three hundred thousand
Rooshians, I tell you, now entering France by Mayence and the
Rhine---three hundred thousand under Wittgenstein and Barclay de Tolly,
my poor love.  You don't know military affairs, my dear.  I do, and I
tell you there's no infantry in France can stand against Rooshian
infantry, and no general of Boney's that's fit to hold a candle to
Wittgenstein.  Then there are the Austrians, they are five hundred
thousand if a man, and they are within ten marches of the frontier by
this time, under Schwartzenberg and Prince Charles.  Then there are the
Prooshians under the gallant Prince Marshal.  Show me a cavalry chief
like him now that Murat is gone. Hey, Mrs.\ O'Dowd?  Do you think our
little girl here need be afraid? Is there any cause for fear, Isidor?
Hey, sir?  Get some more beer.''

Mrs.\ O'Dowd said that her ``Glorvina was not afraid of any man alive,
let alone a Frenchman,'' and tossed off a glass of beer with a wink
which expressed her liking for the beverage.

Having frequently been in presence of the enemy, or, in other words,
faced the ladies at Cheltenham and Bath, our friend, the Collector, had
lost a great deal of his pristine timidity, and was now, especially
when fortified with liquor, as talkative as might be.  He was rather a
favourite with the regiment, treating the young officers with
sumptuosity, and amusing them by his military airs. And as there is one
well-known regiment of the army which travels with a goat heading the
column, whilst another is led by a deer, George said with respect to
his brother-in-law, that his regiment marched with an elephant.

Since Amelia's introduction to the regiment, George began to be rather
ashamed of some of the company to which he had been forced to present
her; and determined, as he told Dobbin (with what satisfaction to the
latter it need not be said), to exchange into some better regiment
soon, and to get his wife away from those damned vulgar women.  But
this vulgarity of being ashamed of one's society is much more common
among men than women (except very great ladies of fashion, who, to be
sure, indulge in it); and Mrs.\ Amelia, a natural and unaffected person,
had none of that artificial shamefacedness which her husband mistook
for delicacy on his own part.  Thus Mrs.\ O'Dowd had a cock's plume in
her hat, and a very large ``repayther'' on her stomach, which she used to
ring on all occasions, narrating how it had been presented to her by
her fawther, as she stipt into the car'ge after her mar'ge; and these
ornaments, with other outward peculiarities of the Major's wife, gave
excruciating agonies to Captain Osborne, when his wife and the Major's
came in contact; whereas Amelia was only amused by the honest lady's
eccentricities, and not in the least ashamed of her company.

As they made that well-known journey, which almost every Englishman of
middle rank has travelled since, there might have been more
instructive, but few more entertaining, companions than Mrs.\ Major
O'Dowd.  ``Talk about kenal boats; my dear!  Ye should see the kenal
boats between Dublin and Ballinasloe.  It's there the rapid travelling
is; and the beautiful cattle.  Sure me fawther got a goold medal (and
his Excellency himself eat a slice of it, and said never was finer mate
in his loif) for a four-year-old heifer, the like of which ye never saw
in this country any day.'' And Jos owned with a sigh, ``that for good
streaky beef, really mingled with fat and lean, there was no country
like England.''

``Except Ireland, where all your best mate comes from,'' said the Major's
lady; proceeding, as is not unusual with patriots of her nation, to
make comparisons greatly in favour of her own country. The idea of
comparing the market at Bruges with those of Dublin, although she had
suggested it herself, caused immense scorn and derision on her part.
``I'll thank ye tell me what they mean by that old gazabo on the top of
the market-place,'' said she, in a burst of ridicule fit to have brought
the old tower down.  The place was full of English soldiery as they
passed.  English bugles woke them in the morning; at nightfall they
went to bed to the note of the British fife and drum:  all the country
and Europe was in arms, and the greatest event of history pending:  and
honest Peggy O'Dowd, whom it concerned as well as another, went on
prattling about Ballinafad, and the horses in the stables at
Glenmalony, and the clar't drunk there; and Jos Sedley interposed about
curry and rice at Dumdum; and Amelia thought about her husband, and how
best she should show her love for him; as if these were the great
topics of the world.

Those who like to lay down the History-book, and to speculate upon what
MIGHT have happened in the world, but for the fatal occurrence of what
actually did take place (a most puzzling, amusing, ingenious, and
profitable kind of meditation), have no doubt often thought to
themselves what a specially bad time Napoleon took to come back from
Elba, and to let loose his eagle from Gulf San Juan to Notre Dame.  The
historians on our side tell us that the armies of the allied powers
were all providentially on a war-footing, and ready to bear down at a
moment's notice upon the Elban Emperor. The august jobbers assembled at
Vienna, and carving out the kingdoms of Europe according to their
wisdom, had such causes of quarrel among themselves as might have set
the armies which had overcome Napoleon to fight against each other, but
for the return of the object of unanimous hatred and fear.  This
monarch had an army in full force because he had jobbed to himself
Poland, and was determined to keep it:  another had robbed half Saxony,
and was bent upon maintaining his acquisition: Italy was the object of
a third's solicitude.  Each was protesting against the rapacity of the
other; and could the Corsican but have waited in prison until all these
parties were by the ears, he might have returned and reigned
unmolested.  But what would have become of our story and all our
friends, then?  If all the drops in it were dried up, what would become
of the sea?

In the meanwhile the business of life and living, and the pursuits of
pleasure, especially, went on as if no end were to be expected to them,
and no enemy in front. When our travellers arrived at Brussels, in
which their regiment was quartered, a great piece of good fortune, as
all said, they found themselves in one of the gayest and most brilliant
little capitals in Europe, and where all the Vanity Fair booths were
laid out with the most tempting liveliness and splendour.  Gambling was
here in profusion, and dancing in plenty:  feasting was there to fill
with delight that great gourmand of a Jos:  there was a theatre where a
miraculous Catalani was delighting all hearers:  beautiful rides, all
enlivened with martial splendour; a rare old city, with strange
costumes and wonderful architecture, to delight the eyes of little
Amelia, who had never before seen a foreign country, and fill her with
charming surprises: so that now and for a few weeks' space in a fine
handsome lodging, whereof the expenses were borne by Jos and Osborne,
who was flush of money and full of kind attentions to his wife---for
about a fortnight, I say, during which her honeymoon ended, Mrs.\ Amelia
was as pleased and happy as any little bride out of England.

Every day during this happy time there was novelty and amusement for
all parties.  There was a church to see, or a picture-gallery---there
was a ride, or an opera. The bands of the regiments were making music
at all hours.  The greatest folks of England walked in the Park---there
was a perpetual military festival.  George, taking out his wife to a
new jaunt or junket every night, was quite pleased with himself as
usual, and swore he was becoming quite a domestic character.  And a
jaunt or a junket with HIM!  Was it not enough to set this little heart
beating with joy?  Her letters home to her mother were filled with
delight and gratitude at this season.  Her husband bade her buy laces,
millinery, jewels, and gimcracks of all sorts.  Oh, he was the kindest,
best, and most generous of men!

The sight of the very great company of lords and ladies and fashionable
persons who thronged the town, and appeared in every public place,
filled George's truly British soul with intense delight.  They flung
off that happy frigidity and insolence of demeanour which occasionally
characterises the great at home, and appearing in numberless public
places, condescended to mingle with the rest of the company whom they
met there.  One night at a party given by the general of the division
to which George's regiment belonged, he had the honour of dancing with
Lady Blanche Thistlewood, Lord Bareacres' daughter; he bustled for ices
and refreshments for the two noble ladies; he pushed and squeezed for
Lady Bareacres' carriage; he bragged about the Countess when he got
home, in a way which his own father could not have surpassed.  He
called upon the ladies the next day; he rode by their side in the Park;
he asked their party to a great dinner at a restaurateur's, and was
quite wild with exultation when they agreed to come.  Old Bareacres,
who had not much pride and a large appetite, would go for a dinner
anywhere.

``I hope there will be no women besides our own party,'' Lady Bareacres
said, after reflecting upon the invitation which had been made, and
accepted with too much precipitancy.

``Gracious Heaven, Mamma---you don't suppose the man would bring his
wife,'' shrieked Lady Blanche, who had been languishing in George's arms
in the newly imported waltz for hours the night before.  ``The men are
bearable, but their women---''

``Wife, just married, dev'lish pretty woman, I hear,'' the old Earl said.

``Well, my dear Blanche,'' said the mother, ``I suppose, as Papa wants to
go, we must go; but we needn't know them in England, you know.'' And so,
determined to cut their new acquaintance in Bond Street, these great
folks went to eat his dinner at Brussels, and condescending to make him
pay for their pleasure, showed their dignity by making his wife
uncomfortable, and carefully excluding her from the conversation.  This
is a species of dignity in which the high-bred British female reigns
supreme.  To watch the behaviour of a fine lady to other and humbler
women, is a very good sport for a philosophical frequenter of Vanity
Fair.

This festival, on which honest George spent a great deal of money, was
the very dismallest of all the entertainments which Amelia had in her
honeymoon.  She wrote the most piteous accounts of the feast home to
her mamma:  how the Countess of Bareacres would not answer when spoken
to; how Lady Blanche stared at her with her eye-glass; and what a rage
Captain Dobbin was in at their behaviour; and how my lord, as they came
away from the feast, asked to see the bill, and pronounced it a d---------
bad dinner, and d--------- dear.  But though Amelia told all these stories,
and wrote home regarding her guests' rudeness, and her own
discomfiture, old Mrs.\ Sedley was mightily pleased nevertheless, and
talked about Emmy's friend, the Countess of Bareacres, with such
assiduity that the news how his son was entertaining peers and
peeresses actually came to Osborne's ears in the City.

Those who know the present Lieutenant-General Sir George Tufto, K.C.B.,
and have seen him, as they may on most days in the season, padded and
in stays, strutting down Pall Mall with a rickety swagger on his
high-heeled lacquered boots, leering under the bonnets of passers-by,
or riding a showy chestnut, and ogling broughams in the Parks---those
who know the present Sir George Tufto would hardly recognise the daring
Peninsular and Waterloo officer.  He has thick curling brown hair and
black eyebrows now, and his whiskers are of the deepest purple.  He was
light-haired and bald in 1815, and stouter in the person and in the
limbs, which especially have shrunk very much of late.  When he was
about seventy years of age (he is now nearly eighty), his hair, which
was very scarce and quite white, suddenly grew thick, and brown, and
curly, and his whiskers and eyebrows took their present colour.
Ill-natured people say that his chest is all wool, and that his hair,
because it never grows, is a wig.  Tom Tufto, with whose father he
quarrelled ever so many years ago, declares that Mademoiselle de
Jaisey, of the French theatre, pulled his grandpapa's hair off in the
green-room; but Tom is notoriously spiteful and jealous; and the
General's wig has nothing to do with our story.

One day, as some of our friends of the ---th were sauntering in the
flower-market of Brussels, having been to see the Hotel de Ville, which
Mrs.\ Major O'Dowd declared was not near so large or handsome as her
fawther's mansion of Glenmalony, an officer of rank, with an orderly
behind him, rode up to the market, and descending from his horse, came
amongst the flowers, and selected the very finest bouquet which money
could buy. The beautiful bundle being tied up in a paper, the officer
remounted, giving the nosegay into the charge of his military groom,
who carried it with a grin, following his chief, who rode away in great
state and self-satisfaction.

``You should see the flowers at Glenmalony,'' Mrs.\ O'Dowd was remarking.
``Me fawther has three Scotch garners with nine helpers. We have an acre
of hot-houses, and pines as common as pays in the sayson.  Our greeps
weighs six pounds every bunch of 'em, and upon me honour and conscience
I think our magnolias is as big as taykettles.''

Dobbin, who never used to ``draw out'' Mrs.\ O'Dowd as that wicked Osborne
delighted in doing (much to Amelia's terror, who implored him to spare
her), fell back in the crowd, crowing and sputtering until he reached a
safe distance, when he exploded amongst the astonished market-people
with shrieks of yelling laughter.

``Hwhat's that gawky guggling about?'' said Mrs.\ O'Dowd.  ``Is it his nose
bleedn?  He always used to say 'twas his nose bleedn, till he must have
pomped all the blood out of 'um.  An't the magnolias at Glenmalony as
big as taykettles, O'Dowd?''

``'Deed then they are, and bigger, Peggy,'' the Major said.  When the
conversation was interrupted in the manner stated by the arrival of the
officer who purchased the bouquet.

``Devlish fine horse---who is it?'' George asked.

``You should see me brother Molloy Malony's horse, Molasses, that won
the cop at the Curragh,'' the Major's wife was exclaiming, and was
continuing the family history, when her husband interrupted her by
saying---

``It's General Tufto, who commands the --------- cavalry division''; adding
quietly, ``he and I were both shot in the same leg at Talavera.''

``Where you got your step,'' said George with a laugh. ``General Tufto!
Then, my dear, the Crawleys are come.''

Amelia's heart fell---she knew not why.  The sun did not seem to shine
so bright.  The tall old roofs and gables looked less picturesque all
of a sudden, though it was a brilliant sunset, and one of the brightest
and most beautiful days at the end of May.



\chapter{Brussels}

Mr.\ Jos had hired a pair of horses for his open carriage, with which
cattle, and the smart London vehicle, he made a very tolerable figure
in the drives about Brussels. George purchased a horse for his private
riding, and he and Captain Dobbin would often accompany the carriage in
which Jos and his sister took daily excursions of pleasure.  They went
out that day in the park for their accustomed diversion, and there,
sure enough, George's remark with regard to the arrival of Rawdon
Crawley and his wife proved to be correct.  In the midst of a little
troop of horsemen, consisting of some of the very greatest persons in
Brussels, Rebecca was seen in the prettiest and tightest of
riding-habits, mounted on a beautiful little Arab, which she rode to
perfection (having acquired the art at Queen's Crawley, where the
Baronet, Mr.\ Pitt, and Rawdon himself had given her many lessons), and
by the side of the gallant General Tufto.

``Sure it's the Juke himself,'' cried Mrs.\ Major O'Dowd to Jos, who began
to blush violently; ``and that's Lord Uxbridge on the bay.  How elegant
he looks!  Me brother, Molloy Malony, is as like him as two pays.''

Rebecca did not make for the carriage; but as soon as she perceived her
old acquaintance Amelia seated in it, acknowledged her presence by a
gracious nod and smile, and by kissing and shaking her fingers
playfully in the direction of the vehicle.  Then she resumed her
conversation with General Tufto, who asked ``who the fat officer was in
the gold-laced cap?'' on which Becky replied, ``that he was an officer in
the East Indian service.'' But Rawdon Crawley rode out of the ranks of
his company, and came up and shook hands heartily with Amelia, and said
to Jos, ``Well, old boy, how are you?'' and stared in Mrs.\ O'Dowd's face
and at the black cock's feathers until she began to think she had made
a conquest of him.

George, who had been delayed behind, rode up almost immediately with
Dobbin, and they touched their caps to the august personages, among
whom Osborne at once perceived Mrs.\ Crawley.  He was delighted to see
Rawdon leaning over his carriage familiarly and talking to Amelia, and
met the aide-de-camp's cordial greeting with more than corresponding
warmth.  The nods between Rawdon and Dobbin were of the very faintest
specimens of politeness.

Crawley told George where they were stopping with General Tufto at the
Hotel du Parc, and George made his friend promise to come speedily to
Osborne's own residence.  ``Sorry I hadn't seen you three days ago,''
George said.  ``Had a dinner at the Restaurateur's---rather a nice thing.
Lord Bareacres, and the Countess, and Lady Blanche, were good enough to
dine with us---wish we'd had you.'' Having thus let his friend know his
claims to be a man of fashion, Osborne parted from Rawdon, who followed
the august squadron down an alley into which they cantered, while
George and Dobbin resumed their places, one on each side of Amelia's
carriage.

``How well the Juke looked,'' Mrs.\ O'Dowd remarked. ``The Wellesleys and
Malonys are related; but, of course, poor I would never dream of
introjuicing myself unless his Grace thought proper to remember our
family-tie.''

``He's a great soldier,'' Jos said, much more at ease now the great man
was gone.  ``Was there ever a battle won like Salamanca?  Hey, Dobbin?
But where was it he learnt his art?  In India, my boy!  The jungle's
the school for a general, mark me that.  I knew him myself, too, Mrs.\ %
O'Dowd:  we both of us danced the same evening with Miss Cutler,
daughter of Cutler of the Artillery, and a devilish fine girl, at
Dumdum.''

The apparition of the great personages held them all in talk during the
drive; and at dinner; and until the hour came when they were all to go
to the Opera.

It was almost like Old England.  The house was filled with familiar
British faces, and those toilettes for which the British female has
long been celebrated.  Mrs.\ O'Dowd's was not the least splendid amongst
these, and she had a curl on her forehead, and a set of Irish diamonds
and Cairngorms, which outshone all the decorations in the house, in her
notion.  Her presence used to excruciate Osborne; but go she would upon
all parties of pleasure on which she heard her young friends were bent.
It never entered into her thought but that they must be charmed with
her company.

``She's been useful to you, my dear,'' George said to his wife, whom he
could leave alone with less scruple when she had this society. ``But
what a comfort it is that Rebecca's come:  you will have her for a
friend, and we may get rid now of this damn'd Irishwoman.''  To this
Amelia did not answer, yes or no:  and how do we know what her thoughts
were?

The coup d'oeil of the Brussels opera-house did not strike Mrs.\ O'Dowd
as being so fine as the theatre in Fishamble Street, Dublin, nor was
French music at all equal, in her opinion, to the melodies of her
native country. She favoured her friends with these and other opinions
in a very loud tone of voice, and tossed about a great clattering fan
she sported, with the most splendid complacency.

``Who is that wonderful woman with Amelia, Rawdon, love?'' said a lady in
an opposite box (who, almost always civil to her husband in private,
was more fond than ever of him in company).

``Don't you see that creature with a yellow thing in her turban, and a
red satin gown, and a great watch?''

``Near the pretty little woman in white?'' asked a middle-aged gentleman
seated by the querist's side, with orders in his button, and several
under-waistcoats, and a great, choky, white stock.

``That pretty woman in white is Amelia, General:  you are remarking all
the pretty women, you naughty man.''

``Only one, begad, in the world!'' said the General, delighted, and the
lady gave him a tap with a large bouquet which she had.

``Bedad it's him,'' said Mrs.\ O'Dowd; ``and that's the very bokay he
bought in the Marshy aux Flures!'' and when Rebecca, having caught her
friend's eye, performed the little hand-kissing operation once more,
Mrs.\ Major O'D., taking the compliment to herself, returned the salute
with a gracious smile, which sent that unfortunate Dobbin shrieking out
of the box again.

At the end of the act, George was out of the box in a moment, and he
was even going to pay his respects to Rebecca in her loge.  He met
Crawley in the lobby, however, where they exchanged a few sentences
upon the occurrences of the last fortnight.

``You found my cheque all right at the agent's? George said, with a
knowing air.

``All right, my boy,'' Rawdon answered.  ``Happy to give you your revenge.
Governor come round?''

``Not yet,'' said George, ``but he will; and you know I've some private
fortune through my mother.  Has Aunty relented?''

``Sent me twenty pound, damned old screw.  When shall we have a meet?
The General dines out on Tuesday. Can't you come Tuesday?  I say, make
Sedley cut off his moustache.  What the devil does a civilian mean with
a moustache and those infernal frogs to his coat!  By-bye. Try and come
on Tuesday''; and Rawdon was going-off with two brilliant young
gentlemen of fashion, who were, like himself, on the staff of a general
officer.

George was only half pleased to be asked to dinner on that particular
day when the General was not to dine.  ``I will go in and pay my
respects to your wife,'' said he; at which Rawdon said, ``Hm, as you
please,'' looking very glum, and at which the two young officers
exchanged knowing glances.  George parted from them and strutted down
the lobby to the General's box, the number of which he had carefully
counted.

``Entrez,'' said a clear little voice, and our friend found himself in
Rebecca's presence; who jumped up, clapped her hands together, and held
out both of them to George, so charmed was she to see him.  The
General, with the orders in his button, stared at the newcomer with a
sulky scowl, as much as to say, who the devil are you?

``My dear Captain George!'' cried little Rebecca in an ecstasy.  ``How
good of you to come.  The General and I were moping together tete-a-tete.
General, this is my Captain George of whom you heard me talk.''

``Indeed,'' said the General, with a very small bow; ``of what regiment is
Captain George?''

George mentioned the ---th:  how he wished he could have said it was a
crack cavalry corps.

``Come home lately from the West Indies, I believe. Not seen much
service in the late war.  Quartered here, Captain George?''---the General
went on with killing haughtiness.

``Not Captain George, you stupid man; Captain Osborne,'' Rebecca said.
The General all the while was looking savagely from one to the other.

``Captain Osborne, indeed! Any relation to the L--------- Osbornes?''

``We bear the same arms,'' George said, as indeed was the fact; Mr.\ %
Osborne having consulted with a herald in Long Acre, and picked the
L--------- arms out of the peerage, when he set up his carriage fifteen
years before.  The General made no reply to this announcement; but took
up his opera-glass---the double-barrelled lorgnon was not invented in
those days---and pretended to examine the house; but Rebecca saw that
his disengaged eye was working round in her direction, and shooting out
bloodshot glances at her and George.

She redoubled in cordiality.  ``How is dearest Amelia? But I needn't
ask: how pretty she looks!  And who is that nice good-natured looking
creature with her---a flame of yours?  O, you wicked men! And there is
Mr.\ Sedley eating ice, I declare: how he seems to enjoy it!  General,
why have we not had any ices?''

``Shall I go and fetch you some?'' said the General, bursting with wrath.

``Let ME go, I entreat you,'' George said.

``No, I will go to Amelia's box.  Dear, sweet girl!  Give me your arm,
Captain George''; and so saying, and with a nod to the General, she
tripped into the lobby.  She gave George the queerest, knowingest look,
when they were together, a look which might have been interpreted,
``Don't you see the state of affairs, and what a fool I'm making of
him?''  But he did not perceive it.  He was thinking of his own plans,
and lost in pompous admiration of his own irresistible powers of
pleasing.

The curses to which the General gave a low utterance, as soon as
Rebecca and her conqueror had quitted him, were so deep, that I am sure
no compositor would venture to print them were they written down.  They
came from the General's heart; and a wonderful thing it is to think
that the human heart is capable of generating such produce, and can
throw out, as occasion demands, such a supply of lust and fury, rage
and hatred.

Amelia's gentle eyes, too, had been fixed anxiously on the pair, whose
conduct had so chafed the jealous General; but when Rebecca entered her
box, she flew to her friend with an affectionate rapture which showed
itself, in spite of the publicity of the place; for she embraced her
dearest friend in the presence of the whole house, at least in full
view of the General's glass, now brought to bear upon the Osborne
party.  Mrs.\ Rawdon saluted Jos, too, with the kindliest greeting: she
admired Mrs.\ O'Dowd's large Cairngorm brooch and superb Irish diamonds,
and wouldn't believe that they were not from Golconda direct. She
bustled, she chattered, she turned and twisted, and smiled upon one,
and smirked on another, all in full view of the jealous opera-glass
opposite.  And when the time for the ballet came (in which there was no
dancer that went through her grimaces or performed her comedy of action
better), she skipped back to her own box, leaning on Captain Dobbin's
arm this time.  No, she would not have George's: he must stay and talk
to his dearest, best, little Amelia.

``What a humbug that woman is!'' honest old Dobbin mumbled to George,
when he came back from Rebecca's box, whither he had conducted her in
perfect silence, and with a countenance as glum as an undertaker's.
``She writhes and twists about like a snake.  All the time she was here,
didn't you see, George, how she was acting at the General over the way?''

``Humbug---acting!  Hang it, she's the nicest little woman in England,''
George replied, showing his white teeth, and giving his ambrosial
whiskers a twirl.  ``You ain't a man of the world, Dobbin. Dammy, look
at her now, she's talked over Tufto in no time.  Look how he's
laughing!  Gad, what a shoulder she has!  Emmy, why didn't you have a
bouquet?  Everybody has a bouquet.''

``Faith, then, why didn't you BOY one?'' Mrs.\ O'Dowd said; and both
Amelia and William Dobbin thanked her for this timely observation. But
beyond this neither of the ladies rallied.  Amelia was overpowered by
the flash and the dazzle and the fashionable talk of her worldly rival.
Even the O'Dowd was silent and subdued after Becky's brilliant
apparition, and scarcely said a word more about Glenmalony all the
evening.

``When do you intend to give up play, George, as you have promised me,
any time these hundred years?'' Dobbin said to his friend a few days
after the night at the Opera.  ``When do you intend to give up
sermonising?'' was the other's reply.  ``What the deuce, man, are you
alarmed about?  We play low; I won last night.  You don't suppose
Crawley cheats?  With fair play it comes to pretty much the same thing
at the year's end.''

``But I don't think he could pay if he lost,'' Dobbin said; and his
advice met with the success which advice usually commands.  Osborne and
Crawley were repeatedly together now.  General Tufto dined abroad
almost constantly. George was always welcome in the apartments (very
close indeed to those of the General) which the aide-de-camp and his
wife occupied in the hotel.

Amelia's manners were such when she and George visited Crawley and his
wife at these quarters, that they had very nearly come to their first
quarrel; that is, George scolded his wife violently for her evident
unwillingness to go, and the high and mighty manner in which she
comported herself towards Mrs.\ Crawley, her old friend; and Amelia did
not say one single word in reply; but with her husband's eye upon her,
and Rebecca scanning her as she felt, was, if possible, more bashful
and awkward on the second visit which she paid to Mrs.\ Rawdon, than on
her first call.

Rebecca was doubly affectionate, of course, and would not take notice,
in the least, of her friend's coolness.  ``I think Emmy has become
prouder since her father's name was in the---since Mr.\ Sedley's
MISFORTUNES,'' Rebecca said, softening the phrase charitably for
George's ear.

``Upon my word, I thought when we were at Brighton she was doing me the
honour to be jealous of me; and now I suppose she is scandalised
because Rawdon, and I, and the General live together.  Why, my dear
creature, how could we, with our means, live at all, but for a friend
to share expenses?  And do you suppose that Rawdon is not big enough to
take care of my honour?  But I'm very much obliged to Emmy, very,'' Mrs.\ %
Rawdon said.

``Pooh, jealousy!'' answered George, ``all women are jealous.''

``And all men too.  Weren't you jealous of General Tufto, and the
General of you, on the night of the Opera? Why, he was ready to eat me
for going with you to visit that foolish little wife of yours; as if I
care a pin for either of you,'' Crawley's wife said, with a pert toss of
her head.  ``Will you dine here?  The dragon dines with the
Commander-in-Chief.  Great news is stirring.  They say the French have
crossed the frontier.  We shall have a quiet dinner.''

George accepted the invitation, although his wife was a little ailing.
They were now not quite six weeks married. Another woman was laughing
or sneering at her expense, and he not angry.  He was not even angry
with himself, this good-natured fellow.  It is a shame, he owned to
himself; but hang it, if a pretty woman WILL throw herself in your way,
why, what can a fellow do, you know?  I AM rather free about women, he
had often said, smiling and nodding knowingly to Stubble and Spooney,
and other comrades of the mess-table; and they rather respected him
than otherwise for this prowess.  Next to conquering in war, conquering
in love has been a source of pride, time out of mind, amongst men in
Vanity Fair, or how should schoolboys brag of their amours, or Don Juan
be popular?

So Mr.\ Osborne, having a firm conviction in his own mind that he was a
woman-killer and destined to conquer, did not run counter to his fate,
but yielded himself up to it quite complacently.  And as Emmy did not
say much or plague him with her jealousy, but merely became unhappy and
pined over it miserably in secret, he chose to fancy that she was not
suspicious of what all his acquaintance were perfectly aware---namely,
that he was carrying on a desperate flirtation with Mrs.\ Crawley.  He
rode with her whenever she was free.  He pretended regimental business
to Amelia (by which falsehood she was not in the least deceived), and
consigning his wife to solitude or her brother's society, passed his
evenings in the Crawleys' company; losing money to the husband and
flattering himself that the wife was dying of love for him. It is very
likely that this worthy couple never absolutely conspired and agreed
together in so many words:  the one to cajole the young gentleman,
whilst the other won his money at cards: but they understood each other
perfectly well, and Rawdon let Osborne come and go with entire good
humour.

George was so occupied with his new acquaintances that he and William
Dobbin were by no means so much together as formerly. George avoided
him in public and in the regiment, and, as we see, did not like those
sermons which his senior was disposed to inflict upon him. If some
parts of his conduct made Captain Dobbin exceedingly grave and cool; of
what use was it to tell George that, though his whiskers were large,
and his own opinion of his knowingness great, he was as green as a
schoolboy? that Rawdon was making a victim of him as he had done of
many before, and as soon as he had used him would fling him off with
scorn?  He would not listen:  and so, as Dobbin, upon those days when
he visited the Osborne house, seldom had the advantage of meeting his
old friend, much painful and unavailing talk between them was spared.
Our friend George was in the full career of the pleasures of Vanity
Fair.

There never was, since the days of Darius, such a brilliant train of
camp-followers as hung round the Duke of Wellington's army in the Low
Countries, in 1815; and led it dancing and feasting, as it were, up to
the very brink of battle.  A certain ball which a noble Duchess gave at
Brussels on the 15th of June in the above-named year is historical.
All Brussels had been in a state of excitement about it, and I have
heard from ladies who were in that town at the period, that the talk
and interest of persons of their own sex regarding the ball was much
greater even than in respect of the enemy in their front. The
struggles, intrigues, and prayers to get tickets were such as only
English ladies will employ, in order to gain admission to the society
of the great of their own nation.

Jos and Mrs.\ O'Dowd, who were panting to be asked, strove in vain to
procure tickets; but others of our friends were more lucky.  For
instance, through the interest of my Lord Bareacres, and as a set-off
for the dinner at the restaurateur's, George got a card for Captain and
Mrs.\ Osborne; which circumstance greatly elated him. Dobbin, who was a
friend of the General commanding the division in which their regiment
was, came laughing one day to Mrs.\ Osborne, and displayed a similar
invitation, which made Jos envious, and George wonder how the deuce he
should be getting into society.  Mr.\ and Mrs.\ Rawdon, finally, were of
course invited; as became the friends of a General commanding a cavalry
brigade.

On the appointed night, George, having commanded new dresses and
ornaments of all sorts for Amelia, drove to the famous ball, where his
wife did not know a single soul.  After looking about for Lady
Bareacres, who cut him, thinking the card was quite enough---and after
placing Amelia on a bench, he left her to her own cogitations there,
thinking, on his own part, that he had behaved very handsomely in
getting her new clothes, and bringing her to the ball, where she was
free to amuse herself as she liked.  Her thoughts were not of the
pleasantest, and nobody except honest Dobbin came to disturb them.

Whilst her appearance was an utter failure (as her husband felt with a
sort of rage), Mrs.\ Rawdon Crawley's debut was, on the contrary, very
brilliant.  She arrived very late.  Her face was radiant; her dress
perfection.  In the midst of the great persons assembled, and the
eye-glasses directed to her, Rebecca seemed to be as cool and collected
as when she used to marshal Miss Pinkerton's little girls to church.
Numbers of the men she knew already, and the dandies thronged round
her.  As for the ladies, it was whispered among them that Rawdon had
run away with her from out of a convent, and that she was a relation of
the Montmorency family.  She spoke French so perfectly that there might
be some truth in this report, and it was agreed that her manners were
fine, and her air distingue.  Fifty would-be partners thronged round
her at once, and pressed to have the honour to dance with her.  But she
said she was engaged, and only going to dance very little; and made her
way at once to the place where Emmy sate quite unnoticed, and dismally
unhappy.  And so, to finish the poor child at once, Mrs.\ Rawdon ran and
greeted affectionately her dearest Amelia, and began forthwith to
patronise her. She found fault with her friend's dress, and her
hairdresser, and wondered how she could be so chaussee, and vowed that
she must send her corsetiere the next morning.  She vowed that it was a
delightful ball; that there was everybody that every one knew, and only
a VERY few nobodies in the whole room.  It is a fact, that in a
fortnight, and after three dinners in general society, this young woman
had got up the genteel jargon so well, that a native could not speak it
better; and it was only from her French being so good, that you could
know she was not a born woman of fashion.

George, who had left Emmy on her bench on entering the ball-room, very
soon found his way back when Rebecca was by her dear friend's side.
Becky was just lecturing Mrs.\ Osborne upon the follies which her
husband was committing.  ``For God's sake, stop him from gambling, my
dear,'' she said, ``or he will ruin himself. He and Rawdon are playing at
cards every night, and you know he is very poor, and Rawdon will win
every shilling from him if he does not take care.  Why don't you
prevent him, you little careless creature? Why don't you come to us of
an evening, instead of moping at home with that Captain Dobbin?  I dare
say he is tres aimable; but how could one love a man with feet of such
size? Your husband's feet are darlings---Here he comes.  Where have you
been, wretch?  Here is Emmy crying her eyes out for you.  Are you
coming to fetch me for the quadrille?'' And she left her bouquet and
shawl by Amelia's side, and tripped off with George to dance.  Women
only know how to wound so. There is a poison on the tips of their
little shafts, which stings a thousand times more than a man's blunter
weapon.  Our poor Emmy, who had never hated, never sneered all her
life, was powerless in the hands of her remorseless little enemy.

George danced with Rebecca twice or thrice---how many times Amelia
scarcely knew.  She sat quite unnoticed in her corner, except when
Rawdon came up with some words of clumsy conversation:  and later in
the evening, when Captain Dobbin made so bold as to bring her
refreshments and sit beside her.  He did not like to ask her why she
was so sad; but as a pretext for the tears which were filling in her
eyes, she told him that Mrs.\ Crawley had alarmed her by telling her
that George would go on playing.

``It is curious, when a man is bent upon play, by what clumsy rogues he
will allow himself to be cheated,'' Dobbin said; and Emmy said,
``Indeed.'' She was thinking of something else.  It was not the loss of
the money that grieved her.

At last George came back for Rebecca's shawl and flowers.  She was
going away.  She did not even condescend to come back and say good-bye
to Amelia.  The poor girl let her husband come and go without saying a
word, and her head fell on her breast.  Dobbin had been called away,
and was whispering deep in conversation with the General of the
division, his friend, and had not seen this last parting.  George went
away then with the bouquet; but when he gave it to the owner, there lay
a note, coiled like a snake among the flowers.  Rebecca's eye caught it
at once.  She had been used to deal with notes in early life.  She put
out her hand and took the nosegay.  He saw by her eyes as they met,
that she was aware what she should find there.  Her husband hurried her
away, still too intent upon his own thoughts, seemingly, to take note
of any marks of recognition which might pass between his friend and his
wife. These were, however, but trifling.  Rebecca gave George her hand
with one of her usual quick knowing glances, and made a curtsey and
walked away.  George bowed over the hand, said nothing in reply to a
remark of Crawley's, did not hear it even, his brain was so throbbing
with triumph and excitement, and allowed them to go away without a word.

His wife saw the one part at least of the bouquet-scene. It was quite
natural that George should come at Rebecca's request to get her her
scarf and flowers:  it was no more than he had done twenty times before
in the course of the last few days; but now it was too much for her.
``William,'' she said, suddenly clinging to Dobbin, who was near her,
``you've always been very kind to me---I'm---I'm not well.  Take me home.''
She did not know she called him by his Christian name, as George was
accustomed to do.  He went away with her quickly.  Her lodgings were
hard by; and they threaded through the crowd without, where everything
seemed to be more astir than even in the ball-room within.

George had been angry twice or thrice at finding his wife up on his
return from the parties which he frequented:  so she went straight to
bed now; but although she did not sleep, and although the din and
clatter, and the galloping of horsemen were incessant, she never heard
any of these noises, having quite other disturbances to keep her awake.

Osborne meanwhile, wild with elation, went off to a play-table, and
began to bet frantically.  He won repeatedly. ``Everything succeeds with
me to-night,'' he said. But his luck at play even did not cure him of
his restlessness, and he started up after awhile, pocketing his
winnings, and went to a buffet, where he drank off many bumpers of wine.

Here, as he was rattling away to the people around, laughing loudly and
wild with spirits, Dobbin found him. He had been to the card-tables to
look there for his friend.  Dobbin looked as pale and grave as his
comrade was flushed and jovial.

``Hullo, Dob!  Come and drink, old Dob!  The Duke's wine is famous. Give
me some more, you sir''; and he held out a trembling glass for the
liquor.

``Come out, George,'' said Dobbin, still gravely; ``don't drink.''

``Drink!  there's nothing like it.  Drink yourself, and light up your
lantern jaws, old boy.  Here's to you.''

Dobbin went up and whispered something to him, at which George, giving
a start and a wild hurray, tossed off his glass, clapped it on the
table, and walked away speedily on his friend's arm.  ``The enemy has
passed the Sambre,'' William said, ``and our left is already engaged.
Come away.  We are to march in three hours.''

Away went George, his nerves quivering with excitement at the news so
long looked for, so sudden when it came.  What were love and intrigue
now?  He thought about a thousand things but these in his rapid walk to
his quarters---his past life and future chances---the fate which might be
before him---the wife, the child perhaps, from whom unseen he might be
about to part.  Oh, how he wished that night's work undone!  and that
with a clear conscience at least he might say farewell to the tender
and guileless being by whose love he had set such little store!

He thought over his brief married life.  In those few weeks he had
frightfully dissipated his little capital.  How wild and reckless he
had been!  Should any mischance befall him:  what was then left for
her?  How unworthy he was of her.  Why had he married her?  He was not
fit for marriage.  Why had he disobeyed his father, who had been always
so generous to him?  Hope, remorse, ambition, tenderness, and selfish
regret filled his heart.  He sate down and wrote to his father,
remembering what he had said once before, when he was engaged to fight
a duel. Dawn faintly streaked the sky as he closed this farewell
letter.  He sealed it, and kissed the superscription. He thought how he
had deserted that generous father, and of the thousand kindnesses which
the stern old man had done him.

He had looked into Amelia's bedroom when he entered; she lay quiet, and
her eyes seemed closed, and he was glad that she was asleep.  On
arriving at his quarters from the ball, he had found his regimental
servant already making preparations for his departure:  the man had
understood his signal to be still, and these arrangements were very
quickly and silently made.  Should he go in and wake Amelia, he
thought, or leave a note for her brother to break the news of departure
to her?  He went in to look at her once again.

She had been awake when he first entered her room, but had kept her
eyes closed, so that even her wakefulness should not seem to reproach
him.  But when he had returned, so soon after herself, too, this timid
little heart had felt more at ease, and turning towards him as he stept
softly out of the room, she had fallen into a light sleep.  George came
in and looked at her again, entering still more softly.  By the pale
night-lamp he could see her sweet, pale face---the purple eyelids were
fringed and closed, and one round arm, smooth and white, lay outside of
the coverlet.  Good God!  how pure she was; how gentle, how tender, and
how friendless!  and he, how selfish, brutal, and black with crime!
Heart-stained, and shame-stricken, he stood at the bed's foot, and
looked at the sleeping girl.  How dared he---who was he, to pray for one
so spotless!  God bless her!  God bless her!  He came to the bedside,
and looked at the hand, the little soft hand, lying asleep; and he bent
over the pillow noiselessly towards the gentle pale face.

Two fair arms closed tenderly round his neck as he stooped down.  ``I am
awake, George,'' the poor child said, with a sob fit to break the little
heart that nestled so closely by his own.  She was awake, poor soul,
and to what?  At that moment a bugle from the Place of Arms began
sounding clearly, and was taken up through the town; and amidst the
drums of the infantry, and the shrill pipes of the Scotch, the whole
city awoke.



\chapter{``The Girl I Left Behind Me''}

We do not claim to rank among the military novelists. Our place is with
the non-combatants.  When the decks are cleared for action we go below
and wait meekly.  We should only be in the way of the manoeuvres that
the gallant fellows are performing overhead.  We shall go no farther
with the ---th than to the city gate:  and leaving Major O'Dowd to his
duty, come back to the Major's wife, and the ladies and the baggage.

Now the Major and his lady, who had not been invited to the ball at
which in our last chapter other of our friends figured, had much more
time to take their wholesome natural rest in bed, than was accorded to
people who wished to enjoy pleasure as well as to do duty.  ``It's my
belief, Peggy, my dear,'' said he, as he placidly pulled his nightcap
over his ears, ``that there will be such a ball danced in a day or two
as some of 'em has never heard the chune of''; and he was much more
happy to retire to rest after partaking of a quiet tumbler, than to
figure at any other sort of amusement. Peggy, for her part, would have
liked to have shown her turban and bird of paradise at the ball, but
for the information which her husband had given her, and which made her
very grave.

``I'd like ye wake me about half an hour before the assembly beats,'' the
Major said to his lady.  ``Call me at half-past one, Peggy dear, and see
me things is ready.  May be I'll not come back to breakfast, Mrs.\ O'D.''
With which words, which signified his opinion that the regiment would
march the next morning, the Major ceased talking, and fell asleep.

Mrs.\ O'Dowd, the good housewife, arrayed in curl papers and a camisole,
felt that her duty was to act, and not to sleep, at this juncture.
``Time enough for that,'' she said, ``when Mick's gone''; and so she packed
his travelling valise ready for the march, brushed his cloak, his cap,
and other warlike habiliments, set them out in order for him; and
stowed away in the cloak pockets a light package of portable
refreshments, and a wicker-covered flask or pocket-pistol, containing
near a pint of a remarkably sound Cognac brandy, of which she and the
Major approved very much; and as soon as the hands of the ``repayther''
pointed to half-past one, and its interior arrangements (it had a tone
quite equal to a cathaydral, its fair owner considered) knelled forth
that fatal hour, Mrs.\ O'Dowd woke up her Major, and had as comfortable
a cup of coffee prepared for him as any made that morning in Brussels.
And who is there will deny that this worthy lady's preparations
betokened affection as much as the fits of tears and hysterics by which
more sensitive females exhibited their love, and that their partaking
of this coffee, which they drank together while the bugles were
sounding the turn-out and the drums beating in the various quarters of
the town, was not more useful and to the purpose than the outpouring of
any mere sentiment could be?  The consequence was, that the Major
appeared on parade quite trim, fresh, and alert, his well-shaved rosy
countenance, as he sate on horseback, giving cheerfulness and
confidence to the whole corps.  All the officers saluted her when the
regiment marched by the balcony on which this brave woman stood, and
waved them a cheer as they passed; and I daresay it was not from want
of courage, but from a sense of female delicacy and propriety, that she
refrained from leading the gallant---th personally into action.

On Sundays, and at periods of a solemn nature, Mrs.\ O'Dowd used to read
with great gravity out of a large volume of her uncle the Dean's
sermons.  It had been of great comfort to her on board the transport as
they were coming home, and were very nearly wrecked, on their return
from the West Indies.  After the regiment's departure she betook
herself to this volume for meditation; perhaps she did not understand
much of what she was reading, and her thoughts were elsewhere:  but the
sleep project, with poor Mick's nightcap there on the pillow, was quite
a vain one.  So it is in the world.  Jack or Donald marches away to
glory with his knapsack on his shoulder, stepping out briskly to the
tune of ``The Girl I Left Behind Me.'' It is she who remains and
suffers---and has the leisure to think, and brood, and remember.

Knowing how useless regrets are, and how the indulgence of sentiment
only serves to make people more miserable, Mrs.\ Rebecca wisely
determined to give way to no vain feelings of sorrow, and bore the
parting from her husband with quite a Spartan equanimity.  Indeed
Captain Rawdon himself was much more affected at the leave-taking than
the resolute little woman to whom he bade farewell.  She had mastered
this rude coarse nature; and he loved and worshipped her with all his
faculties of regard and admiration.  In all his life he had never been
so happy, as, during the past few months, his wife had made him.  All
former delights of turf, mess, hunting-field, and gambling-table; all
previous loves and courtships of milliners, opera-dancers, and the like
easy triumphs of the clumsy military Adonis, were quite insipid when
compared to the lawful matrimonial pleasures which of late he had
enjoyed.  She had known perpetually how to divert him; and he had found
his house and her society a thousand times more pleasant than any place
or company which he had ever frequented from his childhood until now.
And he cursed his past follies and extravagances, and bemoaned his vast
outlying debts above all, which must remain for ever as obstacles to
prevent his wife's advancement in the world.  He had often groaned over
these in midnight conversations with Rebecca, although as a bachelor
they had never given him any disquiet.  He himself was struck with this
phenomenon.  ``Hang it,'' he would say (or perhaps use a still stronger
expression out of his simple vocabulary), ``before I was married I
didn't care what bills I put my name to, and so long as Moses would
wait or Levy would renew for three months, I kept on never minding.
But since I'm married, except renewing, of course, I give you my honour
I've not touched a bit of stamped paper.''

Rebecca always knew how to conjure away these moods of melancholy.
``Why, my stupid love,'' she would say, ``we have not done with your aunt
yet.  If she fails us, isn't there what you call the Gazette? or, stop,
when your uncle Bute's life drops, I have another scheme. The living
has always belonged to the younger brother, and why shouldn't you sell
out and go into the Church?''  The idea of this conversion set Rawdon
into roars of laughter: you might have heard the explosion through the
hotel at midnight, and the haw-haws of the great dragoon's voice.
General Tufto heard him from his quarters on the first floor above
them; and Rebecca acted the scene with great spirit, and preached
Rawdon's first sermon, to the immense delight of the General at
breakfast.

But these were mere by-gone days and talk.  When the final news arrived
that the campaign was opened, and the troops were to march, Rawdon's
gravity became such that Becky rallied him about it in a manner which
rather hurt the feelings of the Guardsman.  ``You don't suppose I'm
afraid, Becky, I should think,'' he said, with a tremor in his voice.
``But I'm a pretty good mark for a shot, and you see if it brings me
down, why I leave one and perhaps two behind me whom I should wish to
provide for, as I brought 'em into the scrape.  It is no laughing
matter that, Mrs.\ C., anyways.''

Rebecca by a hundred caresses and kind words tried to soothe the
feelings of the wounded lover.  It was only when her vivacity and sense
of humour got the better of this sprightly creature (as they would do
under most circumstances of life indeed) that she would break out with
her satire, but she could soon put on a demure face. ``Dearest love,''
she said, ``do you suppose I feel nothing?'' and hastily dashing
something from her eyes, she looked up in her husband's face with a
smile.

``Look here,'' said he.  ``If I drop, let us see what there is for you. I
have had a pretty good run of luck here, and here's two hundred and
thirty pounds.  I have got ten Napoleons in my pocket.  That is as much
as I shall want; for the General pays everything like a prince; and if
I'm hit, why you know I cost nothing.  Don't cry, little woman; I may
live to vex you yet.  Well, I shan't take either of my horses, but
shall ride the General's grey charger:  it's cheaper, and I told him
mine was lame.  If I'm done, those two ought to fetch you something.
Grigg offered ninety for the mare yesterday, before this confounded
news came, and like a fool I wouldn't let her go under the two o's.
Bullfinch will fetch his price any day, only you'd better sell him in
this country, because the dealers have so many bills of mine, and so
I'd rather he shouldn't go back to England.  Your little mare the
General gave you will fetch something, and there's no d---d livery
stable bills here as there are in London,'' Rawdon added, with a laugh.
``There's that dressing-case cost me two hundred---that is, I owe two for
it; and the gold tops and bottles must be worth thirty or forty.
Please to put THAT up the spout, ma'am, with my pins, and rings, and
watch and chain, and things.  They cost a precious lot of money.  Miss
Crawley, I know, paid a hundred down for the chain and ticker.  Gold
tops and bottles, indeed!  dammy, I'm sorry I didn't take more now.
Edwards pressed on me a silver-gilt boot-jack, and I might have had a
dressing-case fitted up with a silver warming-pan, and a service of
plate.  But we must make the best of what we've got, Becky, you know.''

And so, making his last dispositions, Captain Crawley, who had seldom
thought about anything but himself, until the last few months of his
life, when Love had obtained the mastery over the dragoon, went through
the various items of his little catalogue of effects, striving to see
how they might be turned into money for his wife's benefit, in case any
accident should befall him.  He pleased himself by noting down with a
pencil, in his big schoolboy handwriting, the various items of his
portable property which might be sold for his widow's advantage as, for
example, ``My double-barril by Manton, say 40 guineas; my driving cloak,
lined with sable fur, 50 pounds; my duelling pistols in rosewood case
(same which I shot Captain Marker), 20 pounds; my regulation
saddle-holsters and housings; my Laurie ditto,'' and so forth, over all
of which articles he made Rebecca the mistress.

Faithful to his plan of economy, the Captain dressed himself in his
oldest and shabbiest uniform and epaulets, leaving the newest behind,
under his wife's (or it might be his widow's) guardianship. And this
famous dandy of Windsor and Hyde Park went off on his campaign with a
kit as modest as that of a sergeant, and with something like a prayer
on his lips for the woman he was leaving. He took her up from the
ground, and held her in his arms for a minute, tight pressed against
his strong-beating heart.  His face was purple and his eyes dim, as he
put her down and left her.  He rode by his General's side, and smoked
his cigar in silence as they hastened after the troops of the General's
brigade, which preceded them; and it was not until they were some miles
on their way that he left off twirling his moustache and broke silence.

And Rebecca, as we have said, wisely determined not to give way to
unavailing sentimentality on her husband's departure.  She waved him an
adieu from the window, and stood there for a moment looking out after
he was gone. The cathedral towers and the full gables of the quaint old
houses were just beginning to blush in the sunrise. There had been no
rest for her that night.  She was still in her pretty ball-dress, her
fair hair hanging somewhat out of curl on her neck, and the circles
round her eyes dark with watching.  ``What a fright I seem,'' she said,
examining herself in the glass, ``and how pale this pink makes one
look!''  So she divested herself of this pink raiment; in doing which a
note fell out from her corsage, which she picked up with a smile, and
locked into her dressing-box. And then she put her bouquet of the ball
into a glass of water, and went to bed, and slept very comfortably.

The town was quite quiet when she woke up at ten o'clock, and partook
of coffee, very requisite and comforting after the exhaustion and grief
of the morning's occurrences.

This meal over, she resumed honest Rawdon's calculations of the night
previous, and surveyed her position. Should the worst befall, all
things considered, she was pretty well to do.  There were her own
trinkets and trousseau, in addition to those which her husband had left
behind. Rawdon's generosity, when they were first married, has already
been described and lauded.  Besides these, and the little mare, the
General, her slave and worshipper, had made her many very handsome
presents, in the shape of cashmere shawls bought at the auction of a
bankrupt French general's lady, and numerous tributes from the
jewellers' shops, all of which betokened her admirer's taste and
wealth.  As for ``tickers,'' as poor Rawdon called watches, her
apartments were alive with their clicking. For, happening to mention
one night that hers, which Rawdon had given to her, was of English
workmanship, and went ill, on the very next morning there came to her a
little bijou marked Leroy, with a chain and cover charmingly set with
turquoises, and another signed Brequet, which was covered with pearls,
and yet scarcely bigger than a half-crown.  General Tufto had bought
one, and Captain Osborne had gallantly presented the other.  Mrs.\ %
Osborne had no watch, though, to do George justice, she might have had
one for the asking, and the Honourable Mrs.\ Tufto in England had an old
instrument of her mother's that might have served for the plate-warming
pan which Rawdon talked about.  If Messrs. Howell and James were to
publish a list of the purchasers of all the trinkets which they sell,
how surprised would some families be: and if all these ornaments went
to gentlemen's lawful wives and daughters, what a profusion of
jewellery there would be exhibited in the genteelest homes of Vanity
Fair!

Every calculation made of these valuables Mrs.\ Rebecca found, not
without a pungent feeling of triumph and self-satisfaction, that should
circumstances occur, she might reckon on six or seven hundred pounds at
the very least, to begin the world with; and she passed the morning
disposing, ordering, looking out, and locking up her properties in the
most agreeable manner.  Among the notes in Rawdon's pocket-book was a
draft for twenty pounds on Osborne's banker.  This made her think about
Mrs.\ Osborne.  ``I will go and get the draft cashed,'' she said, ``and pay
a visit afterwards to poor little Emmy.'' If this is a novel without a
hero, at least let us lay claim to a heroine.  No man in the British
army which has marched away, not the great Duke himself, could be more
cool or collected in the presence of doubts and difficulties, than the
indomitable little aide-de-camp's wife.

And there was another of our acquaintances who was also to be left
behind, a non-combatant, and whose emotions and behaviour we have
therefore a right to know. This was our friend the ex-collector of
Boggley Wollah, whose rest was broken, like other people's, by the
sounding of the bugles in the early morning.  Being a great sleeper,
and fond of his bed, it is possible he would have snoozed on until his
usual hour of rising in the forenoon, in spite of all the drums,
bugles, and bagpipes in the British army, but for an interruption,
which did not come from George Osborne, who shared Jos's quarters with
him, and was as usual occupied too much with his own affairs or with
grief at parting with his wife, to think of taking leave of his
slumbering brother-in-law---it was not George, we say, who interposed
between Jos Sedley and sleep, but Captain Dobbin, who came and roused
him up, insisting on shaking hands with him before his departure.

``Very kind of you,'' said Jos, yawning, and wishing the Captain at the
deuce.

``I---I didn't like to go off without saying good-bye, you know,'' Dobbin
said in a very incoherent manner; ``because you know some of us mayn't
come back again, and I like to see you all well, and---and that sort of
thing, you know.''

``What do you mean?'' Jos asked, rubbing his eyes.  The Captain did not
in the least hear him or look at the stout gentleman in the nightcap,
about whom he professed to have such a tender interest. The hypocrite
was looking and listening with all his might in the direction of
George's apartments, striding about the room, upsetting the chairs,
beating the tattoo, biting his nails, and showing other signs of great
inward emotion.

Jos had always had rather a mean opinion of the Captain, and now began
to think his courage was somewhat equivocal.  ``What is it I can do for
you, Dobbin?'' he said, in a sarcastic tone.

``I tell you what you can do,'' the Captain replied, coming up to the
bed; ``we march in a quarter of an hour, Sedley, and neither George nor
I may ever come back. Mind you, you are not to stir from this town
until you ascertain how things go.  You are to stay here and watch over
your sister, and comfort her, and see that no harm comes to her.  If
anything happens to George, remember she has no one but you in the
world to look to.  If it goes wrong with the army, you'll see her safe
back to England; and you will promise me on your word that you will
never desert her.  I know you won't:  as far as money goes, you were
always free enough with that.  Do you want any? I mean, have you enough
gold to take you back to England in case of a misfortune?''

``Sir,'' said Jos, majestically, ``when I want money, I know where to ask
for it.  And as for my sister, you needn't tell me how I ought to
behave to her.''

``You speak like a man of spirit, Jos,'' the other answered good-naturedly,
``and I am glad that George can leave her in such good hands.
So I may give him your word of honour, may I, that in case of extremity
you will stand by her?''

``Of course, of course,'' answered Mr.\ Jos, whose generosity in money
matters Dobbin estimated quite correctly.

``And you'll see her safe out of Brussels in the event of a defeat?''

``A defeat! D--------- it, sir, it's impossible.  Don't try and frighten ME,''
the hero cried from his bed; and Dobbin's mind was thus perfectly set
at ease now that Jos had spoken out so resolutely respecting his
conduct to his sister.  ``At least,'' thought the Captain, ``there will be
a retreat secured for her in case the worst should ensue.''

If Captain Dobbin expected to get any personal comfort and satisfaction
from having one more view of Amelia before the regiment marched away,
his selfishness was punished just as such odious egotism deserved to
be.  The door of Jos's bedroom opened into the sitting-room which was
common to the family party, and opposite this door was that of Amelia's
chamber.  The bugles had wakened everybody:  there was no use in
concealment now.  George's servant was packing in this room:  Osborne
coming in and out of the contiguous bedroom, flinging to the man such
articles as he thought fit to carry on the campaign. And presently
Dobbin had the opportunity which his heart coveted, and he got sight of
Amelia's face once more.  But what a face it was!  So white, so wild
and despair-stricken, that the remembrance of it haunted him afterwards
like a crime, and the sight smote him with inexpressible pangs of
longing and pity.

She was wrapped in a white morning dress, her hair falling on her
shoulders, and her large eyes fixed and without light.  By way of
helping on the preparations for the departure, and showing that she too
could be useful at a moment so critical, this poor soul had taken up a
sash of George's from the drawers whereon it lay, and followed him to
and fro with the sash in her hand, looking on mutely as his packing
proceeded.  She came out and stood, leaning at the wall, holding this
sash against her bosom, from which the heavy net of crimson dropped
like a large stain of blood.  Our gentle-hearted Captain felt a guilty
shock as he looked at her.  ``Good God,'' thought he, ``and is it grief
like this I dared to pry into?'' And there was no help:  no means to
soothe and comfort this helpless, speechless misery.  He stood for a
moment and looked at her, powerless and torn with pity, as a parent
regards an infant in pain.

At last, George took Emmy's hand, and led her back into the bedroom,
from whence he came out alone.  The parting had taken place in that
moment, and he was gone.

``Thank Heaven that is over,'' George thought, bounding down the stair,
his sword under his arm, as he ran swiftly to the alarm ground, where
the regiment was mustered, and whither trooped men and officers
hurrying from their billets; his pulse was throbbing and his cheeks
flushed:  the great game of war was going to be played, and he one of
the players.  What a fierce excitement of doubt, hope, and pleasure!
What tremendous hazards of loss or gain!  What were all the games of
chance he had ever played compared to this one? Into all contests
requiring athletic skill and courage, the young man, from his boyhood
upwards, had flung himself with all his might. The champion of his
school and his regiment, the bravos of his companions had followed him
everywhere; from the boys' cricket-match to the garrison-races, he had
won a hundred of triumphs; and wherever he went women and men had
admired and envied him.  What qualities are there for which a man gets
so speedy a return of applause, as those of bodily superiority,
activity, and valour? Time out of mind strength and courage have been
the theme of bards and romances; and from the story of Troy down to
to-day, poetry has always chosen a soldier for a hero.  I wonder is it
because men are cowards in heart that they admire bravery so much, and
place military valour so far beyond every other quality for reward and
worship?

So, at the sound of that stirring call to battle, George jumped away
from the gentle arms in which he had been dallying; not without a
feeling of shame (although his wife's hold on him had been but feeble),
that he should have been detained there so long.  The same feeling of
eagerness and excitement was amongst all those friends of his of whom
we have had occasional glimpses, from the stout senior Major, who led
the regiment into action, to little Stubble, the Ensign, who was to
bear its colours on that day.

The sun was just rising as the march began---it was a gallant sight---the
band led the column, playing the regimental march---then came the
Major in command, riding upon Pyramus, his stout charger---then marched
the grenadiers, their Captain at their head; in the centre were the
colours, borne by the senior and junior Ensigns---then George came
marching at the head of his company. He looked up, and smiled at
Amelia, and passed on; and even the sound of the music died away.



\chapter{In Which Jos Sedley Takes Care of His Sister}

Thus all the superior officers being summoned on duty elsewhere, Jos
Sedley was left in command of the little colony at Brussels, with
Amelia invalided, Isidor, his Belgian servant, and the bonne, who was
maid-of-all-work for the establishment, as a garrison under him. Though
he was disturbed in spirit, and his rest destroyed by Dobbin's
interruption and the occurrences of the morning, Jos nevertheless
remained for many hours in bed, wakeful and rolling about there until
his usual hour of rising had arrived.  The sun was high in the heavens,
and our gallant friends of the ---th miles on their march, before the
civilian appeared in his flowered dressing-gown at breakfast.

About George's absence, his brother-in-law was very easy in mind.
Perhaps Jos was rather pleased in his heart that Osborne was gone, for
during George's presence, the other had played but a very secondary
part in the household, and Osborne did not scruple to show his contempt
for the stout civilian.  But Emmy had always been good and attentive to
him.  It was she who ministered to his comforts, who superintended the
dishes that he liked, who walked or rode with him (as she had many, too
many, opportunities of doing, for where was George?) and who interposed
her sweet face between his anger and her husband's scorn.  Many timid
remonstrances had she uttered to George in behalf of her brother, but
the former in his trenchant way cut these entreaties short. ``I'm an
honest man,'' he said, ``and if I have a feeling I show it, as an honest
man will.  How the deuce, my dear, would you have me behave
respectfully to such a fool as your brother?''  So Jos was pleased with
George's absence.  His plain hat, and gloves on a sideboard, and the
idea that the owner was away, caused Jos I don't know what secret
thrill of pleasure.  ``HE won't be troubling me this morning,'' Jos
thought, ``with his dandified airs and his impudence.''

``Put the Captain's hat into the ante-room,'' he said to Isidor, the
servant.

``Perhaps he won't want it again,'' replied the lackey, looking knowingly
at his master.  He hated George too, whose insolence towards him was
quite of the English sort.

``And ask if Madame is coming to breakfast,'' Mr.\ Sedley said with great
majesty, ashamed to enter with a servant upon the subject of his
dislike for George.  The truth is, he had abused his brother to the
valet a score of times before.

Alas!  Madame could not come to breakfast, and cut the tartines that
Mr.\ Jos liked.  Madame was a great deal too ill, and had been in a
frightful state ever since her husband's departure, so her bonne said.
Jos showed his sympathy by pouring her out a large cup of tea It was
his way of exhibiting kindness:  and he improved on this; he not only
sent her breakfast, but he bethought him what delicacies she would most
like for dinner.

Isidor, the valet, had looked on very sulkily, while Osborne's servant
was disposing of his master's baggage previous to the Captain's
departure:  for in the first place he hated Mr.\ Osborne, whose conduct
to him, and to all inferiors, was generally overbearing (nor does the
continental domestic like to be treated with insolence as our own
better-tempered servants do), and secondly, he was angry that so many
valuables should be removed from under his hands, to fall into other
people's possession when the English discomfiture should arrive.  Of
this defeat he and a vast number of other persons in Brussels and
Belgium did not make the slightest doubt.  The almost universal belief
was, that the Emperor would divide the Prussian and English armies,
annihilate one after the other, and march into Brussels before three
days were over: when all the movables of his present masters, who would
be killed, or fugitives, or prisoners, would lawfully become the
property of Monsieur Isidor.

As he helped Jos through his toilsome and complicated daily toilette,
this faithful servant would calculate what he should do with the very
articles with which he was decorating his master's person.  He would
make a present of the silver essence-bottles and toilet knicknacks to a
young lady of whom he was fond; and keep the English cutlery and the
large ruby pin for himself.  It would look very smart upon one of the
fine frilled shirts, which, with the gold-laced cap and the frogged
frock coat, that might easily be cut down to suit his shape, and the
Captain's gold-headed cane, and the great double ring with the rubies,
which he would have made into a pair of beautiful earrings, he
calculated would make a perfect Adonis of himself, and render
Mademoiselle Reine an easy prey.  ``How those sleeve-buttons will suit
me!'' thought he, as he fixed a pair on the fat pudgy wrists of Mr.\ %
Sedley.  ``I long for sleeve-buttons; and the Captain's boots with brass
spurs, in the next room, corbleu! what an effect they will make in the
Allee Verte!'' So while Monsieur Isidor with bodily fingers was holding
on to his master's nose, and shaving the lower part of Jos's face, his
imagination was rambling along the Green Avenue, dressed out in a
frogged coat and lace, and in company with Mademoiselle Reine; he was
loitering in spirit on the banks, and examining the barges sailing
slowly under the cool shadows of the trees by the canal, or refreshing
himself with a mug of Faro at the bench of a beer-house on the road to
Laeken.

But Mr.\ Joseph Sedley, luckily for his own peace, no more knew what was
passing in his domestic's mind than the respected reader, and I suspect
what John or Mary, whose wages we pay, think of ourselves. What our
servants think of us!---Did we know what our intimates and dear
relations thought of us, we should live in a world that we should be
glad to quit, and in a frame of mind and a constant terror, that would
be perfectly unbearable. So Jos's man was marking his victim down, as
you see one of Mr.\ Paynter's assistants in Leadenhall Street ornament
an unconscious turtle with a placard on which is written, ``Soup
to-morrow.''

Amelia's attendant was much less selfishly disposed. Few dependents
could come near that kind and gentle creature without paying their
usual tribute of loyalty and affection to her sweet and affectionate
nature.  And it is a fact that Pauline, the cook, consoled her mistress
more than anybody whom she saw on this wretched morning; for when she
found how Amelia remained for hours, silent, motionless, and haggard,
by the windows in which she had placed herself to watch the last
bayonets of the column as it marched away, the honest girl took the
lady's hand, and said, Tenez, Madame, est-ce qu'il n'est pas aussi a
l'armee, mon homme a moi?  with which she burst into tears, and Amelia
falling into her arms, did likewise, and so each pitied and soothed the
other.

Several times during the forenoon Mr.\ Jos's Isidor went from his
lodgings into the town, and to the gates of the hotels and lodging-houses
round about the Parc, where the English were congregated, and
there mingled with other valets, couriers, and lackeys, gathered such
news as was abroad, and brought back bulletins for his master's
information.  Almost all these gentlemen were in heart partisans of the
Emperor, and had their opinions about the speedy end of the campaign.
The Emperor's proclamation from Avesnes had been distributed everywhere
plentifully in Brussels.  ``Soldiers!''  it said, ``this is the
anniversary of Marengo and Friedland, by which the destinies of Europe
were twice decided.  Then, as after Austerlitz, as after Wagram, we
were too generous.  We believed in the oaths and promises of princes
whom we suffered to remain upon their thrones.  Let us march once more
to meet them.  We and they, are we not still the same men?  Soldiers!
these same Prussians who are so arrogant to-day, were three to one
against you at Jena, and six to one at Montmirail.  Those among you who
were prisoners in England can tell their comrades what frightful
torments they suffered on board the English hulks.  Madmen!  a moment
of prosperity has blinded them, and if they enter into France it will
be to find a grave there!''  But the partisans of the French prophesied
a more speedy extermination of the Emperor's enemies than this; and it
was agreed on all hands that Prussians and British would never return
except as prisoners in the rear of the conquering army.

These opinions in the course of the day were brought to operate upon
Mr.\ Sedley.  He was told that the Duke of Wellington had gone to try
and rally his army, the advance of which had been utterly crushed the
night before.

``Crushed, psha!'' said Jos, whose heart was pretty stout at
breakfast-time.  ``The Duke has gone to beat the Emperor as he has
beaten all his generals before.''

``His papers are burned, his effects are removed, and his quarters are
being got ready for the Duke of Dalmatia,'' Jos's informant replied.  ``I
had it from his own maitre d'hotel.  Milor Duc de Richemont's people
are packing up everything.  His Grace has fled already, and the Duchess
is only waiting to see the plate packed to join the King of France at
Ostend.''

``The King of France is at Ghent, fellow,'' replied Jos, affecting
incredulity.

``He fled last night to Bruges, and embarks today from Ostend.  The Duc
de Berri is taken prisoner.  Those who wish to be safe had better go
soon, for the dykes will be opened to-morrow, and who can fly when the
whole country is under water?''

``Nonsense, sir, we are three to one, sir, against any force Boney can
bring into the field,'' Mr.\ Sedley objected; ``the Austrians and the
Russians are on their march.  He must, he shall be crushed,'' Jos said,
slapping his hand on the table.

``The Prussians were three to one at Jena, and he took their army and
kingdom in a week.  They were six to one at Montmirail, and he
scattered them like sheep. The Austrian army is coming, but with the
Empress and the King of Rome at its head; and the Russians, bah! the
Russians will withdraw.  No quarter is to be given to the English, on
account of their cruelty to our braves on board the infamous pontoons.
Look here, here it is in black and white.  Here's the proclamation of
his Majesty the Emperor and King,'' said the now declared partisan of
Napoleon, and taking the document from his pocket, Isidor sternly
thrust it into his master's face, and already looked upon the frogged
coat and valuables as his own spoil.

Jos was, if not seriously alarmed as yet, at least considerably
disturbed in mind.  ``Give me my coat and cap, sir,'' said he, ``and follow
me.  I will go myself and learn the truth of these reports.'' Isidor was
furious as Jos put on the braided frock.  ``Milor had better not wear
that military coat,'' said he; ``the Frenchmen have sworn not to give
quarter to a single British soldier.''

``Silence, sirrah!'' said Jos, with a resolute countenance still, and
thrust his arm into the sleeve with indomitable resolution, in the
performance of which heroic act he was found by Mrs.\ Rawdon Crawley,
who at this juncture came up to visit Amelia, and entered without
ringing at the antechamber door.

Rebecca was dressed very neatly and smartly, as usual: her quiet sleep
after Rawdon's departure had refreshed her, and her pink smiling cheeks
were quite pleasant to look at, in a town and on a day when everybody
else's countenance wore the appearance of the deepest anxiety and
gloom.  She laughed at the attitude in which Jos was discovered, and
the struggles and convulsions with which the stout gentleman thrust
himself into the braided coat.

``Are you preparing to join the army, Mr.\ Joseph?'' she said.  ``Is there
to be nobody left in Brussels to protect us poor women?''  Jos succeeded
in plunging into the coat, and came forward blushing and stuttering out
excuses to his fair visitor.  ``How was she after the events of the
morning---after the fatigues of the ball the night before?''  Monsieur
Isidor disappeared into his master's adjacent bedroom, bearing off the
flowered dressing-gown.

``How good of you to ask,'' said she, pressing one of his hands in both
her own.  ``How cool and collected you look when everybody else is
frightened!  How is our dear little Emmy?  It must have been an awful,
awful parting.''

``Tremendous,'' Jos said.

``You men can bear anything,'' replied the lady.  ``Parting or danger are
nothing to you.  Own now that you were going to join the army and leave
us to our fate. I know you were---something tells me you were.  I was so
frightened, when the thought came into my head (for I do sometimes
think of you when I am alone, Mr.\ Joseph), that I ran off immediately
to beg and entreat you not to fly from us.''

This speech might be interpreted, ``My dear sir, should an accident
befall the army, and a retreat be necessary, you have a very
comfortable carriage, in which I propose to take a seat.'' I don't know
whether Jos understood the words in this sense.  But he was profoundly
mortified by the lady's inattention to him during their stay at
Brussels.  He had never been presented to any of Rawdon Crawley's great
acquaintances:  he had scarcely been invited to Rebecca's parties; for
he was too timid to play much, and his presence bored George and Rawdon
equally, who neither of them, perhaps, liked to have a witness of the
amusements in which the pair chose to indulge.  ``Ah!'' thought Jos, ``now
she wants me she comes to me.  When there is nobody else in the way she
can think about old Joseph Sedley!''  But besides these doubts he felt
flattered at the idea Rebecca expressed of his courage.

He blushed a good deal, and put on an air of importance. ``I should like
to see the action,'' he said.  ``Every man of any spirit would, you know.
I've seen a little service in India, but nothing on this grand scale.''

``You men would sacrifice anything for a pleasure,'' Rebecca answered.
``Captain Crawley left me this morning as gay as if he were going to a
hunting party.  What does he care?  What do any of you care for the
agonies and tortures of a poor forsaken woman?  (I wonder whether he
could really have been going to the troops, this great lazy gourmand?)
Oh!  dear Mr.\ Sedley, I have come to you for comfort---for consolation.
I have been on my knees all the morning. I tremble at the frightful
danger into which our husbands, our friends, our brave troops and
allies, are rushing.  And I come here for shelter, and find another of
my friends---the last remaining to me---bent upon plunging into the
dreadful scene!''

``My dear madam,'' Jos replied, now beginning to be quite soothed, ``don't
be alarmed.  I only said I should like to go---what Briton would not?
But my duty keeps me here:  I can't leave that poor creature in the
next room.'' And he pointed with his finger to the door of the chamber
in which Amelia was.

``Good noble brother!'' Rebecca said, putting her handkerchief to her
eyes, and smelling the eau-de-cologne with which it was scented.  ``I
have done you injustice: you have got a heart.  I thought you had not.''

``O, upon my honour!'' Jos said, making a motion as if he would lay his
hand upon the spot in question.  ``You do me injustice, indeed you
do---my dear Mrs.\ Crawley.''

``I do, now your heart is true to your sister.  But I remember two years
ago---when it was false to me!'' Rebecca said, fixing her eyes upon him
for an instant, and then turning away into the window.

Jos blushed violently.  That organ which he was accused by Rebecca of
not possessing began to thump tumultuously.  He recalled the days when
he had fled from her, and the passion which had once inflamed him---the
days when he had driven her in his curricle:  when she had knit the
green purse for him:  when he had sate enraptured gazing at her white
arms and bright eyes.

``I know you think me ungrateful,'' Rebecca continued, coming out of the
window, and once more looking at him and addressing him in a low
tremulous voice.  ``Your coldness, your averted looks, your manner when
we have met of late---when I came in just now, all proved it to me.  But
were there no reasons why I should avoid you? Let your own heart answer
that question.  Do you think my husband was too much inclined to
welcome you? The only unkind words I have ever had from him (I will do
Captain Crawley that justice) have been about you---and most cruel,
cruel words they were.''

``Good gracious! what have I done?'' asked Jos in a flurry of pleasure
and perplexity; ``what have I done---to---to---?''

``Is jealousy nothing?'' said Rebecca.  ``He makes me miserable about you.
And whatever it might have been once---my heart is all his.  I am
innocent now.  Am I not, Mr.\ Sedley?''

All Jos's blood tingled with delight, as he surveyed this victim to his
attractions.  A few adroit words, one or two knowing tender glances of
the eyes, and his heart was inflamed again and his doubts and
suspicions forgotten.  From Solomon downwards, have not wiser men than
he been cajoled and befooled by women?  ``If the worst comes to the
worst,'' Becky thought, ``my retreat is secure; and I have a right-hand
seat in the barouche.''

There is no knowing into what declarations of love and ardour the
tumultuous passions of Mr.\ Joseph might have led him, if Isidor the
valet had not made his reappearance at this minute, and begun to busy
himself about the domestic affairs.  Jos, who was just going to gasp
out an avowal, choked almost with the emotion that he was obliged to
restrain.  Rebecca too bethought her that it was time she should go in
and comfort her dearest Amelia.  ``Au revoir,'' she said, kissing her
hand to Mr.\ Joseph, and tapped gently at the door of his sister's
apartment.  As she entered and closed the door on herself, he sank down
in a chair, and gazed and sighed and puffed portentously.  ``That coat
is very tight for Milor,'' Isidor said, still having his eye on the
frogs; but his master heard him not: his thoughts were elsewhere:  now
glowing, maddening, upon the contemplation of the enchanting Rebecca:
anon shrinking guiltily before the vision of the jealous Rawdon
Crawley, with his curling, fierce mustachios, and his terrible duelling
pistols loaded and cocked.

Rebecca's appearance struck Amelia with terror, and made her shrink
back.  It recalled her to the world and the remembrance of yesterday.
In the overpowering fears about to-morrow she had forgotten
Rebecca---jealousy---everything except that her husband was gone and was
in danger.  Until this dauntless worldling came in and broke the spell,
and lifted the latch, we too have forborne to enter into that sad
chamber.  How long had that poor girl been on her knees!  what hours of
speechless prayer and bitter prostration had she passed there!  The
war-chroniclers who write brilliant stories of fight and triumph
scarcely tell us of these.  These are too mean parts of the pageant:
and you don't hear widows' cries or mothers' sobs in the midst of the
shouts and jubilation in the great Chorus of Victory.  And yet when was
the time that such have not cried out: heart-broken, humble
protestants, unheard in the uproar of the triumph!

After the first movement of terror in Amelia's mind---when Rebecca's
green eyes lighted upon her, and rustling in her fresh silks and
brilliant ornaments, the latter tripped up with extended arms to
embrace her---a feeling of anger succeeded, and from being deadly pale
before, her face flushed up red, and she returned Rebecca's look after
a moment with a steadiness which surprised and somewhat abashed her
rival.

``Dearest Amelia, you are very unwell,'' the visitor said, putting forth
her hand to take Amelia's.  ``What is it? I could not rest until I knew
how you were.''

Amelia drew back her hand---never since her life began had that gentle
soul refused to believe or to answer any demonstration of good-will or
affection.  But she drew back her hand, and trembled all over.  ``Why
are you here, Rebecca?'' she said, still looking at her solemnly with
her large eyes.  These glances troubled her visitor.

``She must have seen him give me the letter at the ball,'' Rebecca
thought.  ``Don't be agitated, dear Amelia,'' she said, looking down. ``I
came but to see if I could---if you were well.''

``Are you well?'' said Amelia.  ``I dare say you are. You don't love your
husband.  You would not be here if you did.  Tell me, Rebecca, did I
ever do you anything but kindness?''

``Indeed, Amelia, no,'' the other said, still hanging down her head.

``When you were quite poor, who was it that befriended you?  Was I not a
sister to you?  You saw us all in happier days before he married me.  I
was all in all then to him; or would he have given up his fortune, his
family, as he nobly did to make me happy?  Why did you come between my
love and me?  Who sent you to separate those whom God joined, and take
my darling's heart from me---my own husband? Do you think you could
love him as I did?  His love was everything to me. You knew it, and
wanted to rob me of it.  For shame, Rebecca; bad and wicked
woman---false friend and false wife.''

``Amelia, I protest before God, I have done my husband no wrong,''
Rebecca said, turning from her.

``Have you done me no wrong, Rebecca?  You did not succeed, but you
tried.  Ask your heart if you did not.''

She knows nothing, Rebecca thought.

``He came back to me.  I knew he would.  I knew that no falsehood, no
flattery, could keep him from me long. I knew he would come.  I prayed
so that he should.''

The poor girl spoke these words with a spirit and volubility which
Rebecca had never before seen in her, and before which the latter was
quite dumb.  ``But what have I done to you,'' she continued in a more
pitiful tone, ``that you should try and take him from me?  I had him but
for six weeks.  You might have spared me those, Rebecca. And yet, from
the very first day of our wedding, you came and blighted it.  Now he is
gone, are you come to see how unhappy I am?'' she continued.  ``You made
me wretched enough for the past fortnight: you might have spared me
to-day.''

``I---I never came here,'' interposed Rebecca, with unlucky truth.

``No.  You didn't come.  You took him away.  Are you come to fetch him
from me?'' she continued in a wilder tone.  ``He was here, but he is gone
now.  There on that very sofa he sate.  Don't touch it.  We sate and
talked there.  I was on his knee, and my arms were round his neck, and
we said 'Our Father.' Yes, he was here:  and they came and took him
away, but he promised me to come back.''

``He will come back, my dear,'' said Rebecca, touched in spite of herself.

``Look,'' said Amelia, ``this is his sash---isn't it a pretty colour?'' and
she took up the fringe and kissed it.  She had tied it round her waist
at some part of the day.  She had forgotten her anger, her jealousy,
the very presence of her rival seemingly.  For she walked silently and
almost with a smile on her face, towards the bed, and began to smooth
down George's pillow.

Rebecca walked, too, silently away.  ``How is Amelia?'' asked Jos, who
still held his position in the chair.

``There should be somebody with her,'' said Rebecca. ``I think she is very
unwell'':  and she went away with a very grave face, refusing Mr.\ %
Sedley's entreaties that she would stay and partake of the early dinner
which he had ordered.

Rebecca was of a good-natured and obliging disposition; and she liked
Amelia rather than otherwise.  Even her hard words, reproachful as they
were, were complimentary---the groans of a person stinging under defeat.
Meeting Mrs.\ O'Dowd, whom the Dean's sermons had by no means comforted,
and who was walking very disconsolately in the Parc, Rebecca accosted
the latter, rather to the surprise of the Major's wife, who was not
accustomed to such marks of politeness from Mrs.\ Rawdon Crawley, and
informing her that poor little Mrs.\ Osborne was in a desperate
condition, and almost mad with grief, sent off the good-natured
Irishwoman straight to see if she could console her young favourite.

``I've cares of my own enough,'' Mrs.\ O'Dowd said, gravely, ``and I
thought poor Amelia would be little wanting for company this day. But
if she's so bad as you say, and you can't attend to her, who used to be
so fond of her, faith I'll see if I can be of service. And so good
marning to ye, Madam''; with which speech and a toss of her head, the
lady of the repayther took a farewell of Mrs.\ Crawley, whose company
she by no means courted.

Becky watched her marching off, with a smile on her lip.  She had the
keenest sense of humour, and the Parthian look which the retreating
Mrs.\ O'Dowd flung over her shoulder almost upset Mrs.\ Crawley's
gravity. ``My service to ye, me fine Madam, and I'm glad to see ye so
cheerful,'' thought Peggy.  ``It's not YOU that will cry your eyes out
with grief, anyway.'' And with this she passed on, and speedily found
her way to Mrs.\ Osborne's lodgings.

The poor soul was still at the bedside, where Rebecca had left her, and
stood almost crazy with grief.  The Major's wife, a stronger-minded
woman, endeavoured her best to comfort her young friend. ``You must bear
up, Amelia, dear,'' she said kindly, ``for he mustn't find you ill when
he sends for you after the victory.  It's not you are the only woman
that are in the hands of God this day.''

``I know that.  I am very wicked, very weak,'' Amelia said.  She knew her
own weakness well enough.  The presence of the more resolute friend
checked it, however; and she was the better of this control and
company.  They went on till two o'clock; their hearts were with the
column as it marched farther and farther away.  Dreadful doubt and
anguish---prayers and fears and griefs unspeakable---followed the
regiment.  It was the women's tribute to the war.  It taxes both alike,
and takes the blood of the men, and the tears of the women.

At half-past two, an event occurred of daily importance to Mr.\ Joseph:
the dinner-hour arrived.  Warriors may fight and perish, but he must
dine.  He came into Amelia's room to see if he could coax her to share
that meal.  ``Try,'' said he; ``the soup is very good.  Do try, Emmy,'' and
he kissed her hand.  Except when she was married, he had not done so
much for years before.  ``You are very good and kind, Joseph,'' she said.
``Everybody is, but, if you please, I will stay in my room to-day.''

The savour of the soup, however, was agreeable to Mrs.\ O'Dowd's
nostrils: and she thought she would bear Mr.\ Jos company.  So the two
sate down to their meal. ``God bless the meat,'' said the Major's wife,
solemnly: she was thinking of her honest Mick, riding at the head of
his regiment:  ``'Tis but a bad dinner those poor boys will get to-day,''
she said, with a sigh, and then, like a philosopher, fell to.

Jos's spirits rose with his meal.  He would drink the regiment's
health; or, indeed, take any other excuse to indulge in a glass of
champagne.  ``We'll drink to O'Dowd and the brave ---th,'' said he, bowing
gallantly to his guest.  ``Hey, Mrs.\ O'Dowd?  Fill Mrs.\ O'Dowd's glass,
Isidor.''

But all of a sudden, Isidor started, and the Major's wife laid down her
knife and fork.  The windows of the room were open, and looked
southward, and a dull distant sound came over the sun-lighted roofs
from that direction.  ``What is it?'' said Jos.  ``Why don't you pour, you
rascal?''

``Cest le feu!'' said Isidor, running to the balcony.

``God defend us; it's cannon!'' Mrs.\ O'Dowd cried, starting up, and
followed too to the window.  A thousand pale and anxious faces might
have been seen looking from other casements.  And presently it seemed
as if the whole population of the city rushed into the streets.



\chapter{In Which Jos Takes Flight, and the War Is Brought to a Close}

We of peaceful London City have never beheld---and please God never
shall witness---such a scene of hurry and alarm, as that which Brussels
presented.  Crowds rushed to the Namur gate, from which direction the
noise proceeded, and many rode along the level chaussee, to be in
advance of any intelligence from the army.  Each man asked his
neighbour for news; and even great English lords and ladies
condescended to speak to persons whom they did not know.  The friends
of the French went abroad, wild with excitement, and prophesying the
triumph of their Emperor.  The merchants closed their shops, and came
out to swell the general chorus of alarm and clamour.  Women rushed to
the churches, and crowded the chapels, and knelt and prayed on the
flags and steps.  The dull sound of the cannon went on rolling,
rolling.  Presently carriages with travellers began to leave the town,
galloping away by the Ghent barrier.  The prophecies of the French
partisans began to pass for facts.  ``He has cut the armies in two,'' it
was said.  ``He is marching straight on Brussels.  He will overpower the
English, and be here to-night.'' ``He will overpower the English,''
shrieked Isidor to his master, ``and will be here to-night.'' The man
bounded in and out from the lodgings to the street, always returning
with some fresh particulars of disaster.  Jos's face grew paler and
paler. Alarm began to take entire possession of the stout civilian.
All the champagne he drank brought no courage to him.  Before sunset he
was worked up to such a pitch of nervousness as gratified his friend
Isidor to behold, who now counted surely upon the spoils of the owner
of the laced coat.

The women were away all this time.  After hearing the firing for a
moment, the stout Major's wife bethought her of her friend in the next
chamber, and ran in to watch, and if possible to console, Amelia.  The
idea that she had that helpless and gentle creature to protect, gave
additional strength to the natural courage of the honest Irishwoman.
She passed five hours by her friend's side, sometimes in remonstrance,
sometimes talking cheerfully, oftener in silence and terrified mental
supplication.  ``I never let go her hand once,'' said the stout lady
afterwards, ``until after sunset, when the firing was over.'' Pauline,
the bonne, was on her knees at church hard by, praying for son homme a
elle.

When the noise of the cannonading was over, Mrs.\ O'Dowd issued out of
Amelia's room into the parlour adjoining, where Jos sate with two
emptied flasks, and courage entirely gone.  Once or twice he had
ventured into his sister's bedroom, looking very much alarmed, and as
if he would say something.  But the Major's wife kept her place, and he
went away without disburthening himself of his speech.  He was ashamed
to tell her that he wanted to fly.

But when she made her appearance in the dining-room, where he sate in
the twilight in the cheerless company of his empty champagne bottles,
he began to open his mind to her.

``Mrs.\ O'Dowd,'' he said, ``hadn't you better get Amelia ready?''

``Are you going to take her out for a walk?'' said the Major's lady;
``sure she's too weak to stir.''

``I---I've ordered the carriage,'' he said, ``and---and post-horses; Isidor
is gone for them,'' Jos continued.

``What do you want with driving to-night?'' answered the lady.  ``Isn't
she better on her bed?  I've just got her to lie down.''

``Get her up,'' said Jos; ``she must get up, I say'':  and he stamped his
foot energetically.  ``I say the horses are ordered---yes, the horses are
ordered.  It's all over, and---''

``And what?'' asked Mrs.\ O'Dowd.

``I'm off for Ghent,'' Jos answered.  ``Everybody is going; there's a
place for you!  We shall start in half-an-hour.''

The Major's wife looked at him with infinite scorn.  ``I don't move till
O'Dowd gives me the route,'' said she. ``You may go if you like, Mr.\ %
Sedley; but, faith, Amelia and I stop here.''

``She SHALL go,'' said Jos, with another stamp of his foot.  Mrs.\ O'Dowd
put herself with arms akimbo before the bedroom door.

``Is it her mother you're going to take her to?'' she said; ``or do you
want to go to Mamma yourself, Mr.\ Sedley?  Good marning---a pleasant
journey to ye, sir. Bon voyage, as they say, and take my counsel, and
shave off them mustachios, or they'll bring you into mischief.''

``D---n!'' yelled out Jos, wild with fear, rage, and mortification; and
Isidor came in at this juncture, swearing in his turn.  ``Pas de
chevaux, sacre bleu!'' hissed out the furious domestic.  All the horses
were gone.  Jos was not the only man in Brussels seized with panic that
day.

But Jos's fears, great and cruel as they were already, were destined to
increase to an almost frantic pitch before the night was over. It has
been mentioned how Pauline, the bonne, had son homme a elle also in the
ranks of the army that had gone out to meet the Emperor Napoleon.  This
lover was a native of Brussels, and a Belgian hussar.  The troops of
his nation signalised themselves in this war for anything but courage,
and young Van Cutsum, Pauline's admirer, was too good a soldier to
disobey his Colonel's orders to run away. Whilst in garrison at
Brussels young Regulus (he had been born in the revolutionary times)
found his great comfort, and passed almost all his leisure moments, in
Pauline's kitchen; and it was with pockets and holsters crammed full of
good things from her larder, that he had take leave of his weeping
sweetheart, to proceed upon the campaign a few days before.

As far as his regiment was concerned, this campaign was over now. They
had formed a part of the division under the command of his Sovereign
apparent, the Prince of Orange, and as respected length of swords and
mustachios, and the richness of uniform and equipments, Regulus and his
comrades looked to be as gallant a body of men as ever trumpet sounded
for.

When Ney dashed upon the advance of the allied troops, carrying one
position after the other, until the arrival of the great body of the
British army from Brussels changed the aspect of the combat of Quatre
Bras, the squadrons among which Regulus rode showed the greatest
activity in retreating before the French, and were dislodged from one
post and another which they occupied with perfect alacrity on their
part.  Their movements were only checked by the advance of the British
in their rear.  Thus forced to halt, the enemy's cavalry (whose
bloodthirsty obstinacy cannot be too severely reprehended) had at
length an opportunity of coming to close quarters with the brave
Belgians before them; who preferred to encounter the British rather
than the French, and at once turning tail rode through the English
regiments that were behind them, and scattered in all directions.  The
regiment in fact did not exist any more.  It was nowhere.  It had no
head-quarters.  Regulus found himself galloping many miles from the
field of action, entirely alone; and whither should he fly for refuge
so naturally as to that kitchen and those faithful arms in which
Pauline had so often welcomed him?

At some ten o'clock the clinking of a sabre might have been heard up
the stair of the house where the Osbornes occupied a story in the
continental fashion.  A knock might have been heard at the kitchen
door; and poor Pauline, come back from church, fainted almost with
terror as she opened it and saw before her her haggard hussar.  He
looked as pale as the midnight dragoon who came to disturb Leonora.
Pauline would have screamed, but that her cry would have called her
masters, and discovered her friend.  She stifled her scream, then, and
leading her hero into the kitchen, gave him beer, and the choice bits
from the dinner, which Jos had not had the heart to taste.  The hussar
showed he was no ghost by the prodigious quantity of flesh and beer
which he devoured---and during the mouthfuls he told his tale of
disaster.

His regiment had performed prodigies of courage, and had withstood for
a while the onset of the whole French army.  But they were overwhelmed
at last, as was the whole British army by this time. Ney destroyed each
regiment as it came up.  The Belgians in vain interposed to prevent the
butchery of the English.  The Brunswickers were routed and had
fled---their Duke was killed.  It was a general debacle.  He sought to
drown his sorrow for the defeat in floods of beer.

Isidor, who had come into the kitchen, heard the conversation and
rushed out to inform his master.  ``It is all over,'' he shrieked to Jos.
``Milor Duke is a prisoner; the Duke of Brunswick is killed; the British
army is in full flight; there is only one man escaped, and he is in the
kitchen now---come and hear him.'' So Jos tottered into that apartment
where Regulus still sate on the kitchen table, and clung fast to his
flagon of beer.  In the best French which he could muster, and which
was in sooth of a very ungrammatical sort, Jos besought the hussar to
tell his tale.  The disasters deepened as Regulus spoke.  He was the
only man of his regiment not slain on the field. He had seen the Duke
of Brunswick fall, the black hussars fly, the Ecossais pounded down by
the cannon. ``And the ---th?'' gasped Jos.

``Cut in pieces,'' said the hussar---upon which Pauline cried out, ``O my
mistress, ma bonne petite dame,'' went off fairly into hysterics, and
filled the house with her screams.

Wild with terror, Mr.\ Sedley knew not how or where to seek for safety.
He rushed from the kitchen back to the sitting-room, and cast an
appealing look at Amelia's door, which Mrs.\ O'Dowd had closed and
locked in his face; but he remembered how scornfully the latter had
received him, and after pausing and listening for a brief space at the
door, he left it, and resolved to go into the street, for the first
time that day.  So, seizing a candle, he looked about for his
gold-laced cap, and found it lying in its usual place, on a
console-table, in the anteroom, placed before a mirror at which Jos
used to coquet, always giving his side-locks a twirl, and his cap the
proper cock over his eye, before he went forth to make appearance in
public.  Such is the force of habit, that even in the midst of his
terror he began mechanically to twiddle with his hair, and arrange the
cock of his hat.  Then he looked amazed at the pale face in the glass
before him, and especially at his mustachios, which had attained a rich
growth in the course of near seven weeks, since they had come into the
world.  They WILL mistake me for a military man, thought he,
remembering Isidor's warning as to the massacre with which all the
defeated British army was threatened; and staggering back to his
bedchamber, he began wildly pulling the bell which summoned his valet.

Isidor answered that summons.  Jos had sunk in a chair---he had torn off
his neckcloths, and turned down his collars, and was sitting with both
his hands lifted to his throat.

``Coupez-moi, Isidor,'' shouted he; ``vite!  Coupez-moi!''

Isidor thought for a moment he had gone mad, and that he wished his
valet to cut his throat.

``Les moustaches,'' gasped Joe; ``les moustaches---coupy, rasy, vite!''---his
French was of this sort---voluble, as we have said, but not
remarkable for grammar.

Isidor swept off the mustachios in no time with the razor, and heard
with inexpressible delight his master's orders that he should fetch a
hat and a plain coat.  ``Ne porty ploo---habit militair---bonn---bonny a
voo, prenny dehors''---were Jos's words---the coat and cap were at last
his property.

This gift being made, Jos selected a plain black coat and waistcoat
from his stock, and put on a large white neckcloth, and a plain beaver.
If he could have got a shovel hat he would have worn it. As it was, you
would have fancied he was a flourishing, large parson of the Church of
England.

``Venny maintenong,'' he continued, ``sweevy---ally---party---dong la roo.''
And so having said, he plunged swiftly down the stairs of the house,
and passed into the street.

Although Regulus had vowed that he was the only man of his regiment or
of the allied army, almost, who had escaped being cut to pieces by Ney,
it appeared that his statement was incorrect, and that a good number
more of the supposed victims had survived the massacre. Many scores of
Regulus's comrades had found their way back to Brussels, and all
agreeing that they had run away---filled the whole town with an idea of
the defeat of the allies.  The arrival of the French was expected
hourly; the panic continued, and preparations for flight went on
everywhere.  No horses!  thought Jos, in terror. He made Isidor inquire
of scores of persons, whether they had any to lend or sell, and his
heart sank within him, at the negative answers returned everywhere.
Should he take the journey on foot?  Even fear could not render that
ponderous body so active.

Almost all the hotels occupied by the English in Brussels face the
Parc, and Jos wandered irresolutely about in this quarter, with crowds
of other people, oppressed as he was by fear and curiosity. Some
families he saw more happy than himself, having discovered a team of
horses, and rattling through the streets in retreat; others again there
were whose case was like his own, and who could not for any bribes or
entreaties procure the necessary means of flight. Amongst these
would-be fugitives, Jos remarked the Lady Bareacres and her daughter,
who sate in their carriage in the porte-cochere of their hotel, all
their imperials packed, and the only drawback to whose flight was the
same want of motive power which kept Jos stationary.

Rebecca Crawley occupied apartments in this hotel; and had before this
period had sundry hostile meetings with the ladies of the Bareacres
family.  My Lady Bareacres cut Mrs.\ Crawley on the stairs when they met
by chance; and in all places where the latter's name was mentioned,
spoke perseveringly ill of her neighbour.  The Countess was shocked at
the familiarity of General Tufto with the aide-de-camp's wife.  The
Lady Blanche avoided her as if she had been an infectious disease.
Only the Earl himself kept up a sly occasional acquaintance with her,
when out of the jurisdiction of his ladies.

Rebecca had her revenge now upon these insolent enemies.  If became
known in the hotel that Captain Crawley's horses had been left behind,
and when the panic began, Lady Bareacres condescended to send her maid
to the Captain's wife with her Ladyship's compliments, and a desire to
know the price of Mrs.\ Crawley's horses.  Mrs.\ Crawley returned a note
with her compliments, and an intimation that it was not her custom to
transact bargains with ladies' maids.

This curt reply brought the Earl in person to Becky's apartment; but he
could get no more success than the first ambassador.  ``Send a lady's
maid to ME!'' Mrs.\ Crawley cried in great anger; ``why didn't my Lady
Bareacres tell me to go and saddle the horses!  Is it her Ladyship that
wants to escape, or her Ladyship's femme de chambre?'' And this was all
the answer that the Earl bore back to his Countess.

What will not necessity do?  The Countess herself actually came to wait
upon Mrs.\ Crawley on the failure of her second envoy.  She entreated
her to name her own price; she even offered to invite Becky to
Bareacres House, if the latter would but give her the means of
returning to that residence.  Mrs.\ Crawley sneered at her.

``I don't want to be waited on by bailiffs in livery,'' she said; ``you
will never get back though most probably---at least not you and your
diamonds together.  The French will have those. They will be here in two
hours, and I shall be half way to Ghent by that time.  I would not sell
you my horses, no, not for the two largest diamonds that your Ladyship
wore at the ball.''  Lady Bareacres trembled with rage and terror.  The
diamonds were sewed into her habit, and secreted in my Lord's padding
and boots. ``Woman, the diamonds are at the banker's, and I WILL have
the horses,'' she said.  Rebecca laughed in her face. The infuriate
Countess went below, and sate in her carriage; her maid, her courier,
and her husband were sent once more through the town, each to look for
cattle; and woe betide those who came last!  Her Ladyship was resolved
on departing the very instant the horses arrived from any quarter---with
her husband or without him.

Rebecca had the pleasure of seeing her Ladyship in the horseless
carriage, and keeping her eyes fixed upon her, and bewailing, in the
loudest tone of voice, the Countess's perplexities.  ``Not to be able to
get horses!'' she said, ``and to have all those diamonds sewed into the
carriage cushions!  What a prize it will be for the French when they
come!---the carriage and the diamonds, I mean; not the lady!'' She gave
this information to the landlord, to the servants, to the guests, and
the innumerable stragglers about the courtyard.  Lady Bareacres could
have shot her from the carriage window.

It was while enjoying the humiliation of her enemy that Rebecca caught
sight of Jos, who made towards her directly he perceived her.

That altered, frightened, fat face, told his secret well enough.  He
too wanted to fly, and was on the look-out for the means of escape. ``HE
shall buy my horses,'' thought Rebecca, ``and I'll ride the mare.''

Jos walked up to his friend, and put the question for the hundredth
time during the past hour, ``Did she know where horses were to be had?''

``What, YOU fly?'' said Rebecca, with a laugh.  ``I thought you were the
champion of all the ladies, Mr.\ Sedley.''

``I---I'm not a military man,'' gasped he.

``And Amelia?---Who is to protect that poor little sister of yours?''
asked Rebecca.  ``You surely would not desert her?''

``What good can I do her, suppose---suppose the enemy arrive?'' Jos
answered.  ``They'll spare the women; but my man tells me that they have
taken an oath to give no quarter to the men---the dastardly cowards.''

``Horrid!'' cried Rebecca, enjoying his perplexity.

``Besides, I don't want to desert her,'' cried the brother. ``She SHAN'T
be deserted.  There is a seat for her in my carriage, and one for you,
dear Mrs.\ Crawley, if you will come; and if we can get horses---'' sighed
he---

``I have two to sell,'' the lady said.  Jos could have flung himself into
her arms at the news.  ``Get the carriage, Isidor,'' he cried; ``we've
found them---we have found them.''

``My horses never were in harness,'' added the lady. ``Bullfinch would kick
the carriage to pieces, if you put him in the traces.''

``But he is quiet to ride?'' asked the civilian.

``As quiet as a lamb, and as fast as a hare,'' answered Rebecca.

``Do you think he is up to my weight?'' Jos said.  He was already on his
back, in imagination, without ever so much as a thought for poor
Amelia.  What person who loved a horse-speculation could resist such a
temptation?

In reply, Rebecca asked him to come into her room, whither he followed
her quite breathless to conclude the bargain.  Jos seldom spent a
half-hour in his life which cost him so much money. Rebecca, measuring
the value of the goods which she had for sale by Jos's eagerness to
purchase, as well as by the scarcity of the article, put upon her
horses a price so prodigious as to make even the civilian draw back.
``She would sell both or neither,'' she said, resolutely.  Rawdon had
ordered her not to part with them for a price less than that which she
specified. Lord Bareacres below would give her the same money---and with
all her love and regard for the Sedley family, her dear Mr.\ Joseph must
conceive that poor people must live---nobody, in a word, could be more
affectionate, but more firm about the matter of business.

Jos ended by agreeing, as might be supposed of him. The sum he had to
give her was so large that he was obliged to ask for time; so large as
to be a little fortune to Rebecca, who rapidly calculated that with
this sum, and the sale of the residue of Rawdon's effects, and her
pension as a widow should he fall, she would now be absolutely
independent of the world, and might look her weeds steadily in the face.

Once or twice in the day she certainly had herself thought about
flying.  But her reason gave her better counsel.  ``Suppose the French
do come,'' thought Becky, ``what can they do to a poor officer's widow?
Bah!  the times of sacks and sieges are over.  We shall be let to go
home quietly, or I may live pleasantly abroad with a snug little
income.''

Meanwhile Jos and Isidor went off to the stables to inspect the newly
purchased cattle.  Jos bade his man saddle the horses at once. He would
ride away that very night, that very hour.  And he left the valet busy
in getting the horses ready, and went homewards himself to prepare for
his departure.  It must be secret.  He would go to his chamber by the
back entrance.  He did not care to face Mrs.\ O'Dowd and Amelia, and own
to them that he was about to run.

By the time Jos's bargain with Rebecca was completed, and his horses
had been visited and examined, it was almost morning once more.  But
though midnight was long passed, there was no rest for the city; the
people were up, the lights in the houses flamed, crowds were still
about the doors, and the streets were busy.  Rumours of various natures
went still from mouth to mouth:  one report averred that the Prussians
had been utterly defeated; another that it was the English who had been
attacked and conquered:  a third that the latter had held their ground.
This last rumour gradually got strength.  No Frenchmen had made their
appearance.  Stragglers had come in from the army bringing reports more
and more favourable:  at last an aide-de-camp actually reached Brussels
with despatches for the Commandant of the place, who placarded
presently through the town an official announcement of the success of
the allies at Quatre Bras, and the entire repulse of the French under
Ney after a six hours' battle.  The aide-de-camp must have arrived
sometime while Jos and Rebecca were making their bargain together, or
the latter was inspecting his purchase.  When he reached his own hotel,
he found a score of its numerous inhabitants on the threshold
discoursing of the news; there was no doubt as to its truth.  And he
went up to communicate it to the ladies under his charge. He did not
think it was necessary to tell them how he had intended to take leave
of them, how he had bought horses, and what a price he had paid for
them.

But success or defeat was a minor matter to them, who had only thought
for the safety of those they loved. Amelia, at the news of the victory,
became still more agitated even than before.  She was for going that
moment to the army.  She besought her brother with tears to conduct her
thither.  Her doubts and terrors reached their paroxysm; and the poor
girl, who for many hours had been plunged into stupor, raved and ran
hither and thither in hysteric insanity---a piteous sight.  No man
writhing in pain on the hard-fought field fifteen miles off, where lay,
after their struggles, so many of the brave---no man suffered more
keenly than this poor harmless victim of the war.  Jos could not bear
the sight of her pain.  He left his sister in the charge of her stouter
female companion, and descended once more to the threshold of the
hotel, where everybody still lingered, and talked, and waited for more
news.

It grew to be broad daylight as they stood here, and fresh news began
to arrive from the war, brought by men who had been actors in the
scene.  Wagons and long country carts laden with wounded came rolling
into the town; ghastly groans came from within them, and haggard faces
looked up sadly from out of the straw.  Jos Sedley was looking at one
of these carriages with a painful curiosity---the moans of the people
within were frightful---the wearied horses could hardly pull the cart.
``Stop!  stop!'' a feeble voice cried from the straw, and the carriage
stopped opposite Mr.\ Sedley's hotel.

``It is George, I know it is!'' cried Amelia, rushing in a moment to the
balcony, with a pallid face and loose flowing hair.  It was not George,
however, but it was the next best thing:  it was news of him.

It was poor Tom Stubble, who had marched out of Brussels so gallantly
twenty-four hours before, bearing the colours of the regiment, which he
had defended very gallantly upon the field.  A French lancer had
speared the young ensign in the leg, who fell, still bravely holding to
his flag.  At the conclusion of the engagement, a place had been found
for the poor boy in a cart, and he had been brought back to Brussels.

``Mr.\ Sedley, Mr.\ Sedley!'' cried the boy, faintly, and Jos came up
almost frightened at the appeal.  He had not at first distinguished who
it was that called him.

Little Tom Stubble held out his hot and feeble hand. ``I'm to be taken
in here,'' he said.  ``Osborne---and---and Dobbin said I was; and you are
to give the man two napoleons: my mother will pay you.'' This young
fellow's thoughts, during the long feverish hours passed in the cart,
had been wandering to his father's parsonage which he had quitted only
a few months before, and he had sometimes forgotten his pain in that
delirium.

The hotel was large, and the people kind, and all the inmates of the
cart were taken in and placed on various couches.  The young ensign was
conveyed upstairs to Osborne's quarters.  Amelia and the Major's wife
had rushed down to him, when the latter had recognised him from the
balcony.  You may fancy the feelings of these women when they were told
that the day was over, and both their husbands were safe; in what mute
rapture Amelia fell on her good friend's neck, and embraced her; in
what a grateful passion of prayer she fell on her knees, and thanked
the Power which had saved her husband.

Our young lady, in her fevered and nervous condition, could have had no
more salutary medicine prescribed for her by any physician than that
which chance put in her way.  She and Mrs.\ O'Dowd watched incessantly
by the wounded lad, whose pains were very severe, and in the duty thus
forced upon her, Amelia had not time to brood over her personal
anxieties, or to give herself up to her own fears and forebodings after
her wont.  The young patient told in his simple fashion the events of
the day, and the actions of our friends of the gallant ---th.  They had
suffered severely.  They had lost very many officers and men.  The
Major's horse had been shot under him as the regiment charged, and they
all thought that O'Dowd was gone, and that Dobbin had got his majority,
until on their return from the charge to their old ground, the Major
was discovered seated on Pyramus's carcase, refreshing him-self from a
case-bottle.  It was Captain Osborne that cut down the French lancer
who had speared the ensign. Amelia turned so pale at the notion, that
Mrs.\ O'Dowd stopped the young ensign in this story.  And it was Captain
Dobbin who at the end of the day, though wounded himself, took up the
lad in his arms and carried him to the surgeon, and thence to the cart
which was to bring him back to Brussels.  And it was he who promised
the driver two louis if he would make his way to Mr.\ Sedley's hotel in
the city; and tell Mrs.\ Captain Osborne that the action was over, and
that her husband was unhurt and well.

``Indeed, but he has a good heart that William Dobbin,'' Mrs.\ O'Dowd
said, ``though he is always laughing at me.''

Young Stubble vowed there was not such another officer in the army, and
never ceased his praises of the senior captain, his modesty, his
kindness, and his admirable coolness in the field.  To these parts of
the conversation, Amelia lent a very distracted attention:  it was only
when George was spoken of that she listened, and when he was not
mentioned, she thought about him.

In tending her patient, and in thinking of the wonderful escapes of the
day before, her second day passed away not too slowly with Amelia.
There was only one man in the army for her:  and as long as he was
well, it must be owned that its movements interested her little. All
the reports which Jos brought from the streets fell very vaguely on her
ears; though they were sufficient to give that timorous gentleman, and
many other people then in Brussels, every disquiet.  The French had
been repulsed certainly, but it was after a severe and doubtful
struggle, and with only a division of the French army. The Emperor,
with the main body, was away at Ligny, where he had utterly annihilated
the Prussians, and was now free to bring his whole force to bear upon
the allies. The Duke of Wellington was retreating upon the capital, and
a great battle must be fought under its walls probably, of which the
chances were more than doubtful. The Duke of Wellington had but twenty
thousand British troops on whom he could rely, for the Germans were raw
militia, the Belgians disaffected, and with this handful his Grace had
to resist a hundred and fifty thousand men that had broken into Belgium
under Napoleon.  Under Napoleon!  What warrior was there, however
famous and skilful, that could fight at odds with him?

Jos thought of all these things, and trembled.  So did all the rest of
Brussels---where people felt that the fight of the day before was but
the prelude to the greater combat which was imminent.  One of the
armies opposed to the Emperor was scattered to the winds already.  The
few English that could be brought to resist him would perish at their
posts, and the conqueror would pass over their bodies into the city.
Woe be to those whom he found there! Addresses were prepared, public
functionaries assembled and debated secretly, apartments were got
ready, and tricoloured banners and triumphal emblems manufactured, to
welcome the arrival of His Majesty the Emperor and King.

The emigration still continued, and wherever families could find means
of departure, they fled.  When Jos, on the afternoon of the 17th of
June, went to Rebecca's hotel, he found that the great Bareacres'
carriage had at length rolled away from the porte-cochere.  The Earl
had procured a pair of horses somehow, in spite of Mrs.\ Crawley, and
was rolling on the road to Ghent.  Louis the Desired was getting ready
his portmanteau in that city, too.  It seemed as if Misfortune was
never tired of worrying into motion that unwieldy exile.

Jos felt that the delay of yesterday had been only a respite, and that
his dearly bought horses must of a surety be put into requisition.  His
agonies were very severe all this day.  As long as there was an English
army between Brussels and Napoleon, there was no need of immediate
flight; but he had his horses brought from their distant stables, to
the stables in the court-yard of the hotel where he lived; so that they
might be under his own eyes, and beyond the risk of violent abduction.
Isidor watched the stable-door constantly, and had the horses saddled,
to be ready for the start. He longed intensely for that event.

After the reception of the previous day, Rebecca did not care to come
near her dear Amelia.  She clipped the bouquet which George had brought
her, and gave fresh water to the flowers, and read over the letter
which he had sent her.  ``Poor wretch,'' she said, twirling round the
little bit of paper in her fingers, ``how I could crush her with
this!---and it is for a thing like this that she must break her heart,
forsooth---for a man who is stupid---a coxcomb---and who does not care for
her.  My poor good Rawdon is worth ten of this creature.'' And then she
fell to thinking what she should do if---if anything happened to poor
good Rawdon, and what a great piece of luck it was that he had left his
horses behind.

In the course of this day too, Mrs.\ Crawley, who saw not without anger
the Bareacres party drive off, bethought her of the precaution which
the Countess had taken, and did a little needlework for her own
advantage; she stitched away the major part of her trinkets, bills, and
bank-notes about her person, and so prepared, was ready for any
event---to fly if she thought fit, or to stay and welcome the conqueror,
were he Englishman or Frenchman.  And I am not sure that she did not
dream that night of becoming a duchess and Madame la Marechale, while
Rawdon wrapped in his cloak, and making his bivouac under the rain at
Mount Saint John, was thinking, with all the force of his heart, about
the little wife whom he had left behind him.

The next day was a Sunday.  And Mrs.\ Major O'Dowd had the satisfaction
of seeing both her patients refreshed in health and spirits by some
rest which they had taken during the night.  She herself had slept on a
great chair in Amelia's room, ready to wait upon her poor friend or the
ensign, should either need her nursing. When morning came, this robust
woman went back to the house where she and her Major had their billet;
and here performed an elaborate and splendid toilette, befitting the
day.  And it is very possible that whilst alone in that chamber, which
her husband had inhabited, and where his cap still lay on the pillow,
and his cane stood in the corner, one prayer at least was sent up to
Heaven for the welfare of the brave soldier, Michael O'Dowd.

When she returned she brought her prayer-book with her, and her uncle
the Dean's famous book of sermons, out of which she never failed to
read every Sabbath; not understanding all, haply, not pronouncing many
of the words aright, which were long and abstruse---for the Dean was a
learned man, and loved long Latin words---but with great gravity, vast
emphasis, and with tolerable correctness in the main.  How often has my
Mick listened to these sermons, she thought, and me reading in the
cabin of a calm!  She proposed to resume this exercise on the present
day, with Amelia and the wounded ensign for a congregation.  The same
service was read on that day in twenty thousand churches at the same
hour; and millions of British men and women, on their knees, implored
protection of the Father of all.

They did not hear the noise which disturbed our little congregation at
Brussels.  Much louder than that which had interrupted them two days
previously, as Mrs.\ O'Dowd was reading the service in her best voice,
the cannon of Waterloo began to roar.

When Jos heard that dreadful sound, he made up his mind that he would
bear this perpetual recurrence of terrors no longer, and would fly at
once.  He rushed into the sick man's room, where our three friends had
paused in their prayers, and further interrupted them by a passionate
appeal to Amelia.

``I can't stand it any more, Emmy,'' he said; ``I won't stand it; and you
must come with me.  I have bought a horse for you---never mind at what
price---and you must dress and come with me, and ride behind Isidor.''

``God forgive me, Mr.\ Sedley, but you are no better than a coward,'' Mrs.\ %
O'Dowd said, laying down the book.

``I say come, Amelia,'' the civilian went on; ``never mind what she says;
why are we to stop here and be butchered by the Frenchmen?''

``You forget the ---th, my boy,'' said the little Stubble, the wounded
hero, from his bed---``and and you won't leave me, will you, Mrs.\ O'Dowd?''

``No, my dear fellow,'' said she, going up and kissing the boy.  ``No harm
shall come to you while I stand by. I don't budge till I get the word
from Mick.  A pretty figure I'd be, wouldn't I, stuck behind that chap
on a pillion?''

This image caused the young patient to burst out laughing in his bed,
and even made Amelia smile.  ``I don't ask her,'' Jos shouted out---``I
don't ask that---that Irishwoman, but you Amelia; once for all, will you
come?''

``Without my husband, Joseph?'' Amelia said, with a look of wonder, and
gave her hand to the Major's wife. Jos's patience was exhausted.

``Good-bye, then,'' he said, shaking his fist in a rage, and slamming the
door by which he retreated.  And this time he really gave his order for
march:  and mounted in the court-yard.  Mrs.\ O'Dowd heard the
clattering hoofs of the horses as they issued from the gate; and
looking on, made many scornful remarks on poor Joseph as he rode down
the street with Isidor after him in the laced cap.  The horses, which
had not been exercised for some days, were lively, and sprang about the
street.  Jos, a clumsy and timid horseman, did not look to advantage in
the saddle.  ``Look at him, Amelia dear, driving into the parlour
window.  Such a bull in a china-shop I never saw.'' And presently the
pair of riders disappeared at a canter down the street leading in the
direction of the Ghent road, Mrs.\ O'Dowd pursuing them with a fire of
sarcasm so long as they were in sight.

All that day from morning until past sunset, the cannon never ceased to
roar.  It was dark when the cannonading stopped all of a sudden.

All of us have read of what occurred during that interval.  The tale is
in every Englishman's mouth; and you and I, who were children when the
great battle was won and lost, are never tired of hearing and
recounting the history of that famous action.  Its remembrance rankles
still in the bosoms of millions of the countrymen of those brave men
who lost the day.  They pant for an opportunity of revenging that
humiliation; and if a contest, ending in a victory on their part,
should ensue, elating them in their turn, and leaving its cursed legacy
of hatred and rage behind to us, there is no end to the so-called glory
and shame, and to the alternations of successful and unsuccessful
murder, in which two high-spirited nations might engage.  Centuries
hence, we Frenchmen and Englishmen might be boasting and killing each
other still, carrying out bravely the Devil's code of honour.

All our friends took their share and fought like men in the great
field.  All day long, whilst the women were praying ten miles away, the
lines of the dauntless English infantry were receiving and repelling
the furious charges of the French horsemen.  Guns which were heard at
Brussels were ploughing up their ranks, and comrades falling, and the
resolute survivors closing in.  Towards evening, the attack of the
French, repeated and resisted so bravely, slackened in its fury.  They
had other foes besides the British to engage, or were preparing for a
final onset.  It came at last:  the columns of the Imperial Guard
marched up the hill of Saint Jean, at length and at once to sweep the
English from the height which they had maintained all day, and spite of
all:  unscared by the thunder of the artillery, which hurled death from
the English line---the dark rolling column pressed on and up the hill.
It seemed almost to crest the eminence, when it began to wave and
falter.  Then it stopped, still facing the shot.  Then at last the
English troops rushed from the post from which no enemy had been able
to dislodge them, and the Guard turned and fled.

No more firing was heard at Brussels---the pursuit rolled miles away.
Darkness came down on the field and city:  and Amelia was praying for
George, who was lying on his face, dead, with a bullet through his
heart.



\chapter{In Which Miss Crawley's Relations Are Very Anxious About Her}

The kind reader must please to remember---while the army is marching
from Flanders, and, after its heroic actions there, is advancing to
take the fortifications on the frontiers of France, previous to an
occupation of that country---that there are a number of persons living
peaceably in England who have to do with the history at present in
hand, and must come in for their share of the chronicle. During the
time of these battles and dangers, old Miss Crawley was living at
Brighton, very moderately moved by the great events that were going on.
The great events rendered the newspapers rather interesting, to be
sure, and Briggs read out the Gazette, in which Rawdon Crawley's
gallantry was mentioned with honour, and his promotion was presently
recorded.

``What a pity that young man has taken such an irretrievable step in the
world!'' his aunt said; ``with his rank and distinction he might have
married a brewer's daughter with a quarter of a million---like Miss
Grains; or have looked to ally himself with the best families in
England. He would have had my money some day or other; or his children
would---for I'm not in a hurry to go, Miss Briggs, although you may be
in a hurry to be rid of me; and instead of that, he is a doomed pauper,
with a dancing-girl for a wife.''

``Will my dear Miss Crawley not cast an eye of compassion upon the
heroic soldier, whose name is inscribed in the annals of his country's
glory?'' said Miss Briggs, who was greatly excited by the Waterloo
proceedings, and loved speaking romantically when there was an
occasion.  ``Has not the Captain---or the Colonel as I may now style
him---done deeds which make the name of Crawley illustrious?''

``Briggs, you are a fool,'' said Miss Crawley: ``Colonel Crawley has
dragged the name of Crawley through the mud, Miss Briggs.  Marry a
drawing-master's daughter, indeed!---marry a dame de compagnie---for she
was no better, Briggs; no, she was just what you are---only younger, and
a great deal prettier and cleverer.  Were you an accomplice of that
abandoned wretch, I wonder, of whose vile arts he became a victim, and
of whom you used to be such an admirer?  Yes, I daresay you were an
accomplice. But you will find yourself disappointed in my will, I can
tell you:  and you will have the goodness to write to Mr.\ Waxy, and say
that I desire to see him immediately.'' Miss Crawley was now in the
habit of writing to Mr.\ Waxy her solicitor almost every day in the
week, for her arrangements respecting her property were all revoked,
and her perplexity was great as to the future disposition of her money.

The spinster had, however, rallied considerably; as was proved by the
increased vigour and frequency of her sarcasms upon Miss Briggs, all
which attacks the poor companion bore with meekness, with cowardice,
with a resignation that was half generous and half hypocritical---with
the slavish submission, in a word, that women of her disposition and
station are compelled to show.  Who has not seen how women bully women?
What tortures have men to endure, comparable to those daily repeated
shafts of scorn and cruelty with which poor women are riddled by the
tyrants of their sex?  Poor victims!  But we are starting from our
proposition, which is, that Miss Crawley was always particularly
annoying and savage when she was rallying from illness---as they say
wounds tingle most when they are about to heal.

While thus approaching, as all hoped, to convalescence, Miss Briggs was
the only victim admitted into the presence of the invalid; yet Miss
Crawley's relatives afar off did not forget their beloved kinswoman,
and by a number of tokens, presents, and kind affectionate messages,
strove to keep themselves alive in her recollection.

In the first place, let us mention her nephew, Rawdon Crawley.  A few
weeks after the famous fight of Waterloo, and after the Gazette had
made known to her the promotion and gallantry of that distinguished
officer, the Dieppe packet brought over to Miss Crawley at Brighton, a
box containing presents, and a dutiful letter, from the Colonel her
nephew.  In the box were a pair of French epaulets, a Cross of the
Legion of Honour, and the hilt of a sword---relics from the field of
battle:  and the letter described with a good deal of humour how the
latter belonged to a commanding officer of the Guard, who having sworn
that ``the Guard died, but never surrendered,'' was taken prisoner the
next minute by a private soldier, who broke the Frenchman's sword with
the butt of his musket, when Rawdon made himself master of the
shattered weapon.  As for the cross and epaulets, they came from a
Colonel of French cavalry, who had fallen under the aide-de-camp's arm
in the battle: and Rawdon Crawley did not know what better to do with
the spoils than to send them to his kindest and most affectionate old
friend. Should he continue to write to her from Paris, whither the army
was marching?  He might be able to give her interesting news from that
capital, and of some of Miss Crawley's old friends of the emigration,
to whom she had shown so much kindness during their distress.

The spinster caused Briggs to write back to the Colonel a gracious and
complimentary letter, encouraging him to continue his correspondence.
His first letter was so excessively lively and amusing that she should
look with pleasure for its successors.---``Of course, I know,'' she
explained to Miss Briggs, ``that Rawdon could not write such a good
letter any more than you could, my poor Briggs, and that it is that
clever little wretch of a Rebecca, who dictates every word to him; but
that is no reason why my nephew should not amuse me; and so I wish to
let him understand that I am in high good humour.''

I wonder whether she knew that it was not only Becky who wrote the
letters, but that Mrs.\ Rawdon actually took and sent home the trophies
which she bought for a few francs, from one of the innumerable pedlars
who immediately began to deal in relics of the war.  The novelist, who
knows everything, knows this also.  Be this, however, as it may, Miss
Crawley's gracious reply greatly encouraged our young friends, Rawdon
and his lady, who hoped for the best from their aunt's evidently
pacified humour:  and they took care to entertain her with many
delightful letters from Paris, whither, as Rawdon said, they had the
good luck to go in the track of the conquering army.

To the rector's lady, who went off to tend her husband's broken
collar-bone at the Rectory at Queen's Crawley, the spinster's
communications were by no means so gracious.  Mrs.\ Bute, that brisk,
managing, lively, imperious woman, had committed the most fatal of all
errors with regard to her sister-in-law.  She had not merely oppressed
her and her household---she had bored Miss Crawley; and if poor Miss
Briggs had been a woman of any spirit, she might have been made happy
by the commission which her principal gave her to write a letter to
Mrs.\ Bute Crawley, saying that Miss Crawley's health was greatly
improved since Mrs.\ Bute had left her, and begging the latter on no
account to put herself to trouble, or quit her family for Miss
Crawley's sake.  This triumph over a lady who had been very haughty and
cruel in her behaviour to Miss Briggs, would have rejoiced most women;
but the truth is, Briggs was a woman of no spirit at all, and the
moment her enemy was discomfited, she began to feel compassion in her
favour.

``How silly I was,'' Mrs.\ Bute thought, and with reason, ``ever to hint
that I was coming, as I did, in that foolish letter when we sent Miss
Crawley the guinea-fowls. I ought to have gone without a word to the
poor dear doting old creature, and taken her out of the hands of that
ninny Briggs, and that harpy of a femme de chambre.  Oh! Bute, Bute,
why did you break your collar-bone?''

Why, indeed?  We have seen how Mrs.\ Bute, having the game in her hands,
had really played her cards too well. She had ruled over Miss Crawley's
household utterly and completely, to be utterly and completely routed
when a favourable opportunity for rebellion came. She and her
household, however, considered that she had been the victim of horrible
selfishness and treason, and that her sacrifices in Miss Crawley's
behalf had met with the most savage ingratitude. Rawdon's promotion,
and the honourable mention made of his name in the Gazette, filled this
good Christian lady also with alarm.  Would his aunt relent towards him
now that he was a Lieutenant-Colonel and a C.B.? and would that odious
Rebecca once more get into favour? The Rector's wife wrote a sermon for
her husband about the vanity of military glory and the prosperity of
the wicked, which the worthy parson read in his best voice and without
understanding one syllable of it.  He had Pitt Crawley for one of his
auditors---Pitt, who had come with his two half-sisters to church, which
the old Baronet could now by no means be brought to frequent.

Since the departure of Becky Sharp, that old wretch had given himself
up entirely to his bad courses, to the great scandal of the county and
the mute horror of his son.  The ribbons in Miss Horrocks's cap became
more splendid than ever.  The polite families fled the hall and its
owner in terror.  Sir Pitt went about tippling at his tenants' houses;
and drank rum-and-water with the farmers at Mudbury and the
neighbouring places on market-days.  He drove the family coach-and-four
to Southampton with Miss Horrocks inside:  and the county people
expected, every week, as his son did in speechless agony, that his
marriage with her would be announced in the provincial paper.  It was
indeed a rude burthen for Mr.\ Crawley to bear.  His eloquence was
palsied at the missionary meetings, and other religious assemblies in
the neighbourhood, where he had been in the habit of presiding, and of
speaking for hours; for he felt, when he rose, that the audience said,
``That is the son of the old reprobate Sir Pitt, who is very likely
drinking at the public house at this very moment.'' And once when he was
speaking of the benighted condition of the king of Timbuctoo, and the
number of his wives who were likewise in darkness, some gipsy miscreant
from the crowd asked, ``How many is there at Queen's Crawley, Young
Squaretoes?'' to the surprise of the platform, and the ruin of Mr.\ %
Pitt's speech. And the two daughters of the house of Queen's Crawley
would have been allowed to run utterly wild (for Sir Pitt swore that no
governess should ever enter into his doors again), had not Mr.\ Crawley,
by threatening the old gentleman, forced the latter to send them to
school.

Meanwhile, as we have said, whatever individual differences there might
be between them all, Miss Crawley's dear nephews and nieces were
unanimous in loving her and sending her tokens of affection. Thus Mrs.\ %
Bute sent guinea-fowls, and some remarkably fine cauliflowers, and a
pretty purse or pincushion worked by her darling girls, who begged to
keep a LITTLE place in the recollection of their dear aunt, while Mr.\ %
Pitt sent peaches and grapes and venison from the Hall.  The
Southampton coach used to carry these tokens of affection to Miss
Crawley at Brighton:  it used sometimes to convey Mr.\ Pitt thither too:
for his differences with Sir Pitt caused Mr.\ Crawley to absent himself
a good deal from home now:  and besides, he had an attraction at
Brighton in the person of the Lady Jane Sheepshanks, whose engagement
to Mr.\ Crawley has been formerly mentioned in this history. Her
Ladyship and her sisters lived at Brighton with their mamma, the
Countess Southdown, that strong-minded woman so favourably known in
the serious world.

A few words ought to be said regarding her Ladyship and her noble
family, who are bound by ties of present and future relationship to the
house of Crawley. Respecting the chief of the Southdown family, Clement
William, fourth Earl of Southdown, little need be told, except that his
Lordship came into Parliament (as Lord Wolsey) under the auspices of
Mr.\ Wilberforce, and for a time was a credit to his political sponsor,
and decidedly a serious young man.  But words cannot describe the
feelings of his admirable mother, when she learned, very shortly after
her noble husband's demise, that her son was a member of several
worldly clubs, had lost largely at play at Wattier's and the Cocoa
Tree; that he had raised money on post-obits, and encumbered the
family estate; that he drove four-in-hand, and patronised the ring; and
that he actually had an opera-box, where he entertained the most
dangerous bachelor company.  His name was only mentioned with groans in
the dowager's circle.

The Lady Emily was her brother's senior by many years; and took
considerable rank in the serious world as author of some of the
delightful tracts before mentioned, and of many hymns and spiritual
pieces.  A mature spinster, and having but faint ideas of marriage, her
love for the blacks occupied almost all her feelings.  It is to her, I
believe, we owe that beautiful poem.

  Lead us to some sunny isle,
  Yonder in the western deep;
  Where the skies for ever smile,
  And the blacks for ever weep, \&c.

She had correspondences with clerical gentlemen in most of our East and
West India possessions; and was secretly attached to the Reverend Silas
Hornblower, who was tattooed in the South Sea Islands.

As for the Lady Jane, on whom, as it has been said, Mr.\ Pitt Crawley's
affection had been placed, she was gentle, blushing, silent, and timid.
In spite of his falling away, she wept for her brother, and was quite
ashamed of loving him still.  Even yet she used to send him little
hurried smuggled notes, and pop them into the post in private. The one
dreadful secret which weighed upon her life was, that she and the old
housekeeper had been to pay Southdown a furtive visit at his chambers
in the Albany; and found him---O the naughty dear abandoned
wretch!---smoking a cigar with a bottle of Curacao before him.  She
admired her sister, she adored her mother, she thought Mr.\ Crawley the
most delightful and accomplished of men, after Southdown, that fallen
angel:  and her mamma and sister, who were ladies of the most superior
sort, managed everything for her, and regarded her with that amiable
pity, of which your really superior woman always has such a share to
give away.  Her mamma ordered her dresses, her books, her bonnets, and
her ideas for her. She was made to take pony-riding, or piano-exercise,
or any other sort of bodily medicament, according as my Lady Southdown
saw meet; and her ladyship would have kept her daughter in pinafores up
to her present age of six-and-twenty, but that they were thrown off
when Lady Jane was presented to Queen Charlotte.

When these ladies first came to their house at Brighton, it was to them
alone that Mr.\ Crawley paid his personal visits, contenting himself by
leaving a card at his aunt's house, and making a modest inquiry of Mr.\ %
Bowls or his assistant footman, with respect to the health of the
invalid.  When he met Miss Briggs coming home from the library with a
cargo of novels under her arm, Mr.\ Crawley blushed in a manner quite
unusual to him, as he stepped forward and shook Miss Crawley's
companion by the hand.  He introduced Miss Briggs to the lady with whom
he happened to be walking, the Lady Jane Sheepshanks, saying, ``Lady
Jane, permit me to introduce to you my aunt's kindest friend and most
affectionate companion, Miss Briggs, whom you know under another title,
as authoress of the delightful 'Lyrics of the Heart,' of which you are
so fond.''  Lady Jane blushed too as she held out a kind little hand to
Miss Briggs, and said something very civil and incoherent about mamma,
and proposing to call on Miss Crawley, and being glad to be made known
to the friends and relatives of Mr.\ Crawley; and with soft dove-like
eyes saluted Miss Briggs as they separated, while Pitt Crawley treated
her to a profound courtly bow, such as he had used to H.H. the Duchess
of Pumpernickel, when he was attache at that court.

The artful diplomatist and disciple of the Machiavellian Binkie!  It
was he who had given Lady Jane that copy of poor Briggs's early poems,
which he remembered to have seen at Queen's Crawley, with a dedication
from the poetess to his father's late wife; and he brought the volume
with him to Brighton, reading it in the Southampton coach and marking
it with his own pencil, before he presented it to the gentle Lady Jane.

It was he, too, who laid before Lady Southdown the great advantages
which might occur from an intimacy between her family and Miss
Crawley---advantages both worldly and spiritual, he said:  for Miss
Crawley was now quite alone; the monstrous dissipation and alliance of
his brother Rawdon had estranged her affections from that reprobate
young man; the greedy tyranny and avarice of Mrs.\ Bute Crawley had
caused the old lady to revolt against the exorbitant pretensions of
that part of the family; and though he himself had held off all his
life from cultivating Miss Crawley's friendship, with perhaps an
improper pride, he thought now that every becoming means should be
taken, both to save her soul from perdition, and to secure her fortune
to himself as the head of the house of Crawley.

The strong-minded Lady Southdown quite agreed in both proposals of her
son-in-law, and was for converting Miss Crawley off-hand.  At her own
home, both at Southdown and at Trottermore Castle, this tall and awful
missionary of the truth rode about the country in her barouche with
outriders, launched packets of tracts among the cottagers and tenants,
and would order Gaffer Jones to be converted, as she would order Goody
Hicks to take a James's powder, without appeal, resistance, or benefit
of clergy.  My Lord Southdown, her late husband, an epileptic and
simple-minded nobleman, was in the habit of approving of everything
which his Matilda did and thought. So that whatever changes her own
belief might undergo (and it accommodated itself to a prodigious
variety of opinion, taken from all sorts of doctors among the
Dissenters) she had not the least scruple in ordering all her tenants
and inferiors to follow and believe after her.  Thus whether she
received the Reverend Saunders McNitre, the Scotch divine; or the
Reverend Luke Waters, the mild Wesleyan; or the Reverend Giles Jowls,
the illuminated Cobbler, who dubbed himself Reverend as Napoleon
crowned himself Emperor---the household, children, tenantry of my Lady
Southdown were expected to go down on their knees with her Ladyship,
and say Amen to the prayers of either Doctor.  During these exercises
old Southdown, on account of his invalid condition, was allowed to sit
in his own room, and have negus and the paper read to him.  Lady Jane
was the old Earl's favourite daughter, and tended him and loved him
sincerely:  as for Lady Emily, the authoress of the ``Washerwoman of
Finchley Common,'' her denunciations of future punishment (at this
period, for her opinions modified afterwards) were so awful that they
used to frighten the timid old gentleman her father, and the physicians
declared his fits always occurred after one of her Ladyship's sermons.

``I will certainly call,'' said Lady Southdown then, in reply to the
exhortation of her daughter's pretendu, Mr.\ Pitt Crawley---``Who is Miss
Crawley's medical man?''

Mr.\ Crawley mentioned the name of Mr.\ Creamer.

``A most dangerous and ignorant practitioner, my dear Pitt.  I have
providentially been the means of removing him from several houses:
though in one or two instances I did not arrive in time.  I could not
save poor dear General Glanders, who was dying under the hands of that
ignorant man---dying.  He rallied a little under the Podgers' pills
which I administered to him; but alas!  it was too late.  His death was
delightful, however; and his change was only for the better; Creamer,
my dear Pitt, must leave your aunt.''

Pitt expressed his perfect acquiescence.  He, too, had been carried
along by the energy of his noble kinswoman, and future mother-in-law.
He had been made to accept Saunders McNitre, Luke Waters, Giles Jowls,
Podgers' Pills, Rodgers' Pills, Pokey's Elixir, every one of her
Ladyship's remedies spiritual or temporal.  He never left her house
without carrying respectfully away with him piles of her quack theology
and medicine.  O, my dear brethren and fellow-sojourners in Vanity
Fair, which among you does not know and suffer under such benevolent
despots?  It is in vain you say to them, ``Dear Madam, I took Podgers'
specific at your orders last year, and believe in it.  Why, why am I to
recant and accept the Rodgers' articles now?''  There is no help for it;
the faithful proselytizer, if she cannot convince by argument, bursts
into tears, and the refusant finds himself, at the end of the contest,
taking down the bolus, and saying, ``Well, well, Rodgers' be it.''

``And as for her spiritual state,'' continued the Lady, ``that of course
must be looked to immediately:  with Creamer about her, she may go off
any day:  and in what a condition, my dear Pitt, in what a dreadful
condition! I will send the Reverend Mr.\ Irons to her instantly.  Jane,
write a line to the Reverend Bartholomew Irons, in the third person,
and say that I desire the pleasure of his company this evening at tea
at half-past six.  He is an awakening man; he ought to see Miss Crawley
before she rests this night.  And Emily, my love, get ready a packet of
books for Miss Crawley.  Put up 'A Voice from the Flames,' 'A
Trumpet-warning to Jericho,' and the 'Fleshpots Broken; or, the
Converted Cannibal.'''

``And the 'Washerwoman of Finchley Common,' Mamma,'' said Lady Emily. ``It
is as well to begin soothingly at first.''

``Stop, my dear ladies,'' said Pitt, the diplomatist. ``With every
deference to the opinion of my beloved and respected Lady Southdown, I
think it would be quite unadvisable to commence so early upon serious
topics with Miss Crawley.  Remember her delicate condition, and how
little, how very little accustomed she has hitherto been to
considerations connected with her immortal welfare.''

``Can we then begin too early, Pitt?'' said Lady Emily, rising with six
little books already in her hand.

``If you begin abruptly, you will frighten her altogether. I know my
aunt's worldly nature so well as to be sure that any abrupt attempt at
conversion will be the very worst means that can be employed for the
welfare of that unfortunate lady.  You will only frighten and annoy
her. She will very likely fling the books away, and refuse all
acquaintance with the givers.''

``You are as worldly as Miss Crawley, Pitt,'' said Lady Emily, tossing
out of the room, her books in her hand.

``And I need not tell you, my dear Lady Southdown,'' Pitt continued, in a
low voice, and without heeding the interruption, ``how fatal a little
want of gentleness and caution may be to any hopes which we may
entertain with regard to the worldly possessions of my aunt. Remember
she has seventy thousand pounds; think of her age, and her highly
nervous and delicate condition; I know that she has destroyed the will
which was made in my brother's (Colonel Crawley's) favour: it is by
soothing that wounded spirit that we must lead it into the right path,
and not by frightening it; and so I think you will agree with me
that---that---'

``Of course, of course,'' Lady Southdown remarked. ``Jane, my love, you
need not send that note to Mr.\ Irons. If her health is such that
discussions fatigue her, we will wait her amendment.  I will call upon
Miss Crawley tomorrow.''

``And if I might suggest, my sweet lady,'' Pitt said in a bland tone, ``it
would be as well not to take our precious Emily, who is too
enthusiastic; but rather that you should be accompanied by our sweet
and dear Lady Jane.''

``Most certainly, Emily would ruin everything,'' Lady Southdown said; and
this time agreed to forego her usual practice, which was, as we have
said, before she bore down personally upon any individual whom she
proposed to subjugate, to fire in a quantity of tracts upon the menaced
party (as a charge of the French was always preceded by a furious
cannonade).  Lady Southdown, we say, for the sake of the invalid's
health, or for the sake of her soul's ultimate welfare, or for the sake
of her money, agreed to temporise.

The next day, the great Southdown female family carriage, with the
Earl's coronet and the lozenge (upon which the three lambs trottant
argent upon the field vert of the Southdowns, were quartered with sable
on a bend or, three snuff-mulls gules, the cognizance of the house of
Binkie), drove up in state to Miss Crawley's door, and the tall serious
footman handed in to Mr.\ Bowls her Ladyship's cards for Miss Crawley,
and one likewise for Miss Briggs.  By way of compromise, Lady Emily
sent in a packet in the evening for the latter lady, containing copies
of the ``Washerwoman,'' and other mild and favourite tracts for Miss B.'s
own perusal; and a few for the servants' hall, viz.:  ``Crumbs from the
Pantry,'' ``The Frying Pan and the Fire,'' and ``The Livery of Sin,'' of a
much stronger kind.



\chapter{James Crawley's Pipe Is Put Out}

The amiable behaviour of Mr.\ Crawley, and Lady Jane's kind reception of
her, highly flattered Miss Briggs, who was enabled to speak a good word
for the latter, after the cards of the Southdown family had been
presented to Miss Crawley.  A Countess's card left personally too for
her, Briggs, was not a little pleasing to the poor friendless
companion.  ``What could Lady Southdown mean by leaving a card upon you,
I wonder, Miss Briggs?'' said the republican Miss Crawley; upon which
the companion meekly said ``that she hoped there could be no harm in a
lady of rank taking notice of a poor gentlewoman,'' and she put away
this card in her work-box amongst her most cherished personal
treasures.  Furthermore, Miss Briggs explained how she had met Mr.\ %
Crawley walking with his cousin and long affianced bride the day
before:  and she told how kind and gentle-looking the lady was, and
what a plain, not to say common, dress she had, all the articles of
which, from the bonnet down to the boots, she described and estimated
with female accuracy.

Miss Crawley allowed Briggs to prattle on without interrupting her too
much.  As she got well, she was pining for society.  Mr.\ Creamer, her
medical man, would not hear of her returning to her old haunts and
dissipation in London.  The old spinster was too glad to find any
companionship at Brighton, and not only were the cards acknowledged the
very next day, but Pitt Crawley was graciously invited to come and see
his aunt.  He came, bringing with him Lady Southdown and her daughter.
The dowager did not say a word about the state of Miss Crawley's soul;
but talked with much discretion about the weather:  about the war and
the downfall of the monster Bonaparte:  and above all, about doctors,
quacks, and the particular merits of Dr. Podgers, whom she then
patronised.

During their interview Pitt Crawley made a great stroke, and one which
showed that, had his diplomatic career not been blighted by early
neglect, he might have risen to a high rank in his profession. When the
Countess Dowager of Southdown fell foul of the Corsican upstart, as the
fashion was in those days, and showed that he was a monster stained
with every conceivable crime, a coward and a tyrant not fit to live,
one whose fall was predicted, \&c., Pitt Crawley suddenly took up the
cudgels in favour of the man of Destiny.  He described the First Consul
as he saw him at Paris at the peace of Amiens; when he, Pitt Crawley,
had the gratification of making the acquaintance of the great and good
Mr.\ Fox, a statesman whom, however much he might differ with him, it
was impossible not to admire fervently---a statesman who had always had
the highest opinion of the Emperor Napoleon.  And he spoke in terms of
the strongest indignation of the faithless conduct of the allies
towards this dethroned monarch, who, after giving himself generously up
to their mercy, was consigned to an ignoble and cruel banishment, while
a bigoted Popish rabble was tyrannising over France in his stead.

This orthodox horror of Romish superstition saved Pitt Crawley in Lady
Southdown's opinion, whilst his admiration for Fox and Napoleon raised
him immeasurably in Miss Crawley's eyes.  Her friendship with that
defunct British statesman was mentioned when we first introduced her in
this history.  A true Whig, Miss Crawley had been in opposition all
through the war, and though, to be sure, the downfall of the Emperor
did not very much agitate the old lady, or his ill-treatment tend to
shorten her life or natural rest, yet Pitt spoke to her heart when he
lauded both her idols; and by that single speech made immense progress
in her favour.

``And what do you think, my dear?'' Miss Crawley said to the young lady,
for whom she had taken a liking at first sight, as she always did for
pretty and modest young people; though it must be owned her affections
cooled as rapidly as they rose.

Lady Jane blushed very much, and said ``that she did not understand
politics, which she left to wiser heads than hers; but though Mamma
was, no doubt, correct, Mr.\ Crawley had spoken beautifully.'' And when
the ladies were retiring at the conclusion of their visit, Miss Crawley
hoped ``Lady Southdown would be so kind as to send her Lady Jane
sometimes, if she could be spared to come down and console a poor sick
lonely old woman.'' This promise was graciously accorded, and they
separated upon great terms of amity.

``Don't let Lady Southdown come again, Pitt,'' said the old lady. ``She is
stupid and pompous, like all your mother's family, whom I never could
endure.  But bring that nice good-natured little Jane as often as ever
you please.'' Pitt promised that he would do so.  He did not tell the
Countess of Southdown what opinion his aunt had formed of her Ladyship,
who, on the contrary, thought that she had made a most delightful and
majestic impression on Miss Crawley.

And so, nothing loth to comfort a sick lady, and perhaps not sorry in
her heart to be freed now and again from the dreary spouting of the
Reverend Bartholomew Irons, and the serious toadies who gathered round
the footstool of the pompous Countess, her mamma, Lady Jane became a
pretty constant visitor to Miss Crawley, accompanied her in her drives,
and solaced many of her evenings.  She was so naturally good and soft,
that even Firkin was not jealous of her; and the gentle Briggs thought
her friend was less cruel to her when kind Lady Jane was by.  Towards
her Ladyship Miss Crawley's manners were charming.  The old spinster
told her a thousand anecdotes about her youth, talking to her in a very
different strain from that in which she had been accustomed to converse
with the godless little Rebecca; for there was that in Lady Jane's
innocence which rendered light talking impertinence before her, and
Miss Crawley was too much of a gentlewoman to offend such purity.  The
young lady herself had never received kindness except from this old
spinster, and her brother and father:  and she repaid Miss Crawley's
engoument by artless sweetness and friendship.

In the autumn evenings (when Rebecca was flaunting at Paris, the gayest
among the gay conquerors there, and our Amelia, our dear wounded
Amelia, ah! where was she?) Lady Jane would be sitting in Miss
Crawley's drawing-room singing sweetly to her, in the twilight, her
little simple songs and hymns, while the sun was setting and the sea
was roaring on the beach.  The old spinster used to wake up when these
ditties ceased, and ask for more.  As for Briggs, and the quantity of
tears of happiness which she now shed as she pretended to knit, and
looked out at the splendid ocean darkling before the windows, and the
lamps of heaven beginning more brightly to shine---who, I say can
measure the happiness and sensibility of Briggs?

Pitt meanwhile in the dining-room, with a pamphlet on the Corn Laws or
a Missionary Register by his side, took that kind of recreation which
suits romantic and unromantic men after dinner.  He sipped Madeira:
built castles in the air:  thought himself a fine fellow: felt himself
much more in love with Jane than he had been any time these seven
years, during which their liaison had lasted without the slightest
impatience on Pitt's part---and slept a good deal.  When the time for
coffee came, Mr.\ Bowls used to enter in a noisy manner, and summon
Squire Pitt, who would be found in the dark very busy with his pamphlet.

``I wish, my love, I could get somebody to play piquet with me,'' Miss
Crawley said one night when this functionary made his appearance with
the candles and the coffee. ``Poor Briggs can no more play than an owl,
she is so stupid'' (the spinster always took an opportunity of abusing
Briggs before the servants); ``and I think I should sleep better if I
had my game.''

At this Lady Jane blushed to the tips of her little ears, and down to
the ends of her pretty fingers; and when Mr.\ Bowls had quitted the
room, and the door was quite shut, she said:

``Miss Crawley, I can play a little.  I used to---to play a little with
poor dear papa.''

``Come and kiss me.  Come and kiss me this instant, you dear good little
soul,'' cried Miss Crawley in an ecstasy: and in this picturesque and
friendly occupation Mr.\ Pitt found the old lady and the young one, when
he came upstairs with him pamphlet in his hand. How she did blush all
the evening, that poor Lady Jane!

It must not be imagined that Mr.\ Pitt Crawley's artifices escaped the
attention of his dear relations at the Rectory at Queen's Crawley.
Hampshire and Sussex lie very close together, and Mrs.\ Bute had friends
in the latter county who took care to inform her of all, and a great
deal more than all, that passed at Miss Crawley's house at Brighton.
Pitt was there more and more.  He did not come for months together to
the Hall, where his abominable old father abandoned himself completely
to rum-and-water, and the odious society of the Horrocks family. Pitt's
success rendered the Rector's family furious, and Mrs.\ Bute regretted
more (though she confessed less) than ever her monstrous fault in so
insulting Miss Briggs, and in being so haughty and parsimonious to
Bowls and Firkin, that she had not a single person left in Miss
Crawley's household to give her information of what took place there.
``It was all Bute's collar-bone,'' she persisted in saying; ``if that had
not broke, I never would have left her.  I am a martyr to duty and to
your odious unclerical habit of hunting, Bute.''

``Hunting; nonsense!  It was you that frightened her, Barbara,'' the
divine interposed.  ``You're a clever woman, but you've got a devil of a
temper; and you're a screw with your money, Barbara.''

``You'd have been screwed in gaol, Bute, if I had not kept your money.''

``I know I would, my dear,'' said the Rector, good-naturedly. ``You ARE a
clever woman, but you manage too well, you know'':  and the pious man
consoled himself with a big glass of port.

``What the deuce can she find in that spooney of a Pitt Crawley?'' he
continued.  ``The fellow has not pluck enough to say Bo to a goose. I
remember when Rawdon, who is a man, and be hanged to him, used to flog
him round the stables as if he was a whipping-top:  and Pitt would go
howling home to his ma---ha, ha!  Why, either of my boys would whop him
with one hand.  Jim says he's remembered at Oxford as Miss Crawley
still---the spooney.

``I say, Barbara,'' his reverence continued, after a pause.

``What?'' said Barbara, who was biting her nails, and drumming the table.

``I say, why not send Jim over to Brighton to see if he can do anything
with the old lady.  He's very near getting his degree, you know.  He's
only been plucked twice---so was I---but he's had the advantages of
Oxford and a university education.  He knows some of the best chaps
there. He pulls stroke in the Boniface boat.  He's a handsome feller.
D--------- it, ma'am, let's put him on the old woman, hey, and tell him to
thrash Pitt if he says anything. Ha, ha, ha!

``Jim might go down and see her, certainly,'' the housewife said; adding
with a sigh, ``If we could but get one of the girls into the house; but
she could never endure them, because they are not pretty!''  Those
unfortunate and well-educated women made themselves heard from the
neighbouring drawing-room, where they were thrumming away, with hard
fingers, an elaborate music-piece on the piano-forte, as their mother
spoke; and indeed, they were at music, or at backboard, or at
geography, or at history, the whole day long.  But what avail all these
accomplishments, in Vanity Fair, to girls who are short, poor, plain,
and have a bad complexion?  Mrs.\ Bute could think of nobody but the
Curate to take one of them off her hands; and Jim coming in from the
stable at this minute, through the parlour window, with a short pipe
stuck in his oilskin cap, he and his father fell to talking about odds
on the St.\ Leger, and the colloquy between the Rector and his wife
ended.

Mrs.\ Bute did not augur much good to the cause from the sending of her
son James as an ambassador, and saw him depart in rather a despairing
mood.  Nor did the young fellow himself, when told what his mission was
to be, expect much pleasure or benefit from it; but he was consoled by
the thought that possibly the old lady would give him some handsome
remembrance of her, which would pay a few of his most pressing bills at
the commencement of the ensuing Oxford term, and so took his place by
the coach from Southampton, and was safely landed at Brighton on the
same evening with his portmanteau, his favourite bull-dog Towzer, and
an immense basket of farm and garden produce, from the dear Rectory
folks to the dear Miss Crawley. Considering it was too late to disturb
the invalid lady on the first night of his arrival, he put up at an
inn, and did not wait upon Miss Crawley until a late hour in the noon
of next day.

James Crawley, when his aunt had last beheld him, was a gawky lad, at
that uncomfortable age when the voice varies between an unearthly
treble and a preternatural bass; when the face not uncommonly blooms
out with appearances for which Rowland's Kalydor is said to act as a
cure; when boys are seen to shave furtively with their sister's
scissors, and the sight of other young women produces intolerable
sensations of terror in them; when the great hands and ankles protrude
a long way from garments which have grown too tight for them; when
their presence after dinner is at once frightful to the ladies, who are
whispering in the twilight in the drawing-room, and inexpressibly
odious to the gentlemen over the mahogany, who are restrained from
freedom of intercourse and delightful interchange of wit by the
presence of that gawky innocence; when, at the conclusion of the second
glass, papa says, ``Jack, my boy, go out and see if the evening holds
up,'' and the youth, willing to be free, yet hurt at not being yet a
man, quits the incomplete banquet.  James, then a hobbadehoy, was now
become a young man, having had the benefits of a university education,
and acquired the inestimable polish which is gained by living in a fast
set at a small college, and contracting debts, and being rusticated,
and being plucked.

He was a handsome lad, however, when he came to present himself to his
aunt at Brighton, and good looks were always a title to the fickle old
lady's favour.  Nor did his blushes and awkwardness take away from it:
she was pleased with these healthy tokens of the young gentleman's
ingenuousness.

He said ``he had come down for a couple of days to see a man of his
college, and---and to pay my respects to you, Ma'am, and my father's and
mother's, who hope you are well.''

Pitt was in the room with Miss Crawley when the lad was announced, and
looked very blank when his name was mentioned.  The old lady had plenty
of humour, and enjoyed her correct nephew's perplexity.  She asked
after all the people at the Rectory with great interest; and said she
was thinking of paying them a visit.  She praised the lad to his face,
and said he was well-grown and very much improved, and that it was a
pity his sisters had not some of his good looks; and finding, on
inquiry, that he had taken up his quarters at an hotel, would not hear
of his stopping there, but bade Mr.\ Bowls send for Mr.\ James Crawley's
things instantly; ``and hark ye, Bowls,'' she added, with great
graciousness, ``you will have the goodness to pay Mr.\ James's bill.''

She flung Pitt a look of arch triumph, which caused that diplomatist
almost to choke with envy.  Much as he had ingratiated himself with his
aunt, she had never yet invited him to stay under her roof, and here
was a young whipper-snapper, who at first sight was made welcome there.

``I beg your pardon, sir,'' says Bowls, advancing with a profound bow;
``what 'otel, sir, shall Thomas fetch the luggage from?''

``O, dam,'' said young James, starting up, as if in some alarm, ``I'll go.''

``What!'' said Miss Crawley.

``The Tom Cribb's Arms,'' said James, blushing deeply.

Miss Crawley burst out laughing at this title.  Mr.\ Bowls gave one
abrupt guffaw, as a confidential servant of the family, but choked the
rest of the volley; the diplomatist only smiled.

``I---I didn't know any better,'' said James, looking down. ``I've never
been here before; it was the coachman told me.'' The young story-teller!
The fact is, that on the Southampton coach, the day previous,
James Crawley had met the Tutbury Pet, who was coming to Brighton to
make a match with the Rottingdean Fibber; and enchanted by the Pet's
conversation, had passed the evening in company with that scientific
man and his friends, at the inn in question.

``I---I'd best go and settle the score,'' James continued. ``Couldn't think
of asking you, Ma'am,'' he added, generously.

This delicacy made his aunt laugh the more.

``Go and settle the bill, Bowls,'' she said, with a wave of her hand,
``and bring it to me.''

Poor lady, she did not know what she had done!  ``There---there's a
little dawg,'' said James, looking frightfully guilty.  ``I'd best go for
him.  He bites footmen's calves.''

All the party cried out with laughing at this description; even Briggs
and Lady Jane, who was sitting mute during the interview between Miss
Crawley and her nephew:  and Bowls, without a word, quitted the room.

Still, by way of punishing her elder nephew, Miss Crawley persisted in
being gracious to the young Oxonian. There were no limits to her
kindness or her compliments when they once began.  She told Pitt he
might come to dinner, and insisted that James should accompany her in
her drive, and paraded him solemnly up and down the cliff, on the back
seat of the barouche.  During all this excursion, she condescended to
say civil things to him: she quoted Italian and French poetry to the
poor bewildered lad, and persisted that he was a fine scholar, and was
perfectly sure he would gain a gold medal, and be a Senior Wrangler.

``Haw, haw,'' laughed James, encouraged by these compliments; ``Senior
Wrangler, indeed; that's at the other shop.''

``What is the other shop, my dear child?'' said the lady.

``Senior Wranglers at Cambridge, not Oxford,'' said the scholar, with a
knowing air; and would probably have been more confidential, but that
suddenly there appeared on the cliff in a tax-cart, drawn by a bang-up
pony, dressed in white flannel coats, with mother-of-pearl buttons, his
friends the Tutbury Pet and the Rottingdean Fibber, with three other
gentlemen of their acquaintance, who all saluted poor James there in
the carriage as he sate.  This incident damped the ingenuous youth's
spirits, and no word of yea or nay could he be induced to utter during
the rest of the drive.

On his return he found his room prepared, and his portmanteau ready,
and might have remarked that Mr.\ Bowls's countenance, when the latter
conducted him to his apartments, wore a look of gravity, wonder, and
compassion.  But the thought of Mr.\ Bowls did not enter his head.  He
was deploring the dreadful predicament in which he found himself, in a
house full of old women, jabbering French and Italian, and talking
poetry to him. ``Reglarly up a tree, by jingo!'' exclaimed the modest
boy, who could not face the gentlest of her sex---not even Briggs---when
she began to talk to him; whereas, put him at Iffley Lock, and he could
out-slang the boldest bargeman.

At dinner, James appeared choking in a white neckcloth, and had the
honour of handing my Lady Jane downstairs, while Briggs and Mr.\ Crawley
followed afterwards, conducting the old lady, with her apparatus of
bundles, and shawls, and cushions.  Half of Briggs's time at dinner was
spent in superintending the invalid's comfort, and in cutting up
chicken for her fat spaniel.  James did not talk much, but he made a
point of asking all the ladies to drink wine, and accepted Mr.\ %
Crawley's challenge, and consumed the greater part of a bottle of
champagne which Mr.\ Bowls was ordered to produce in his honour.  The
ladies having withdrawn, and the two cousins being left together, Pitt,
the ex-diplomatist, he came very communicative and friendly.  He asked
after James's career at college---what his prospects in life were---hoped
heartily he would get on; and, in a word, was frank and amiable.
James's tongue unloosed with the port, and he told his cousin his life,
his prospects, his debts, his troubles at the little-go, and his rows
with the proctors, filling rapidly from the bottles before him, and
flying from Port to Madeira with joyous activity.

``The chief pleasure which my aunt has,'' said Mr.\ Crawley, filling his
glass, ``is that people should do as they like in her house. This is
Liberty Hall, James, and you can't do Miss Crawley a greater kindness
than to do as you please, and ask for what you will.  I know you have
all sneered at me in the country for being a Tory. Miss Crawley is
liberal enough to suit any fancy.  She is a Republican in principle,
and despises everything like rank or title.''

``Why are you going to marry an Earl's daughter?'' said James.

``My dear friend, remember it is not poor Lady Jane's fault that she is
well born,'' Pitt replied, with a courtly air.  ``She cannot help being a
lady.  Besides, I am a Tory, you know.''

``Oh, as for that,'' said Jim, ``there's nothing like old blood; no,
dammy, nothing like it.  I'm none of your radicals.  I know what it is
to be a gentleman, dammy. See the chaps in a boat-race; look at the
fellers in a fight; aye, look at a dawg killing rats---which is it wins?
the good-blooded ones.  Get some more port, Bowls, old boy, whilst I
buzz this bottle here.  What was I asaying?''

``I think you were speaking of dogs killing rats,'' Pitt remarked mildly,
handing his cousin the decanter to ``buzz.''

``Killing rats was I? Well, Pitt, are you a sporting man? Do you want to
see a dawg as CAN kill a rat? If you do, come down with me to Tom
Corduroy's, in Castle Street Mews, and I'll show you such a bull-terrier
as---Pooh! gammon,'' cried James, bursting out laughing at his
own absurdity---``YOU don't care about a dawg or rat; it's all nonsense.
I'm blest if I think you know the difference between a dog and a duck.''

``No; by the way,'' Pitt continued with increased blandness, ``it was
about blood you were talking, and the personal advantages which people
derive from patrician birth.  Here's the fresh bottle.''

``Blood's the word,'' said James, gulping the ruby fluid down. ``Nothing
like blood, sir, in hosses, dawgs, AND men.  Why, only last term, just
before I was rusticated, that is, I mean just before I had the measles,
ha, ha---there was me and Ringwood of Christchurch, Bob Ringwood, Lord
Cinqbars' son, having our beer at the Bell at Blenheim, when the
Banbury bargeman offered to fight either of us for a bowl of punch.  I
couldn't.  My arm was in a sling; couldn't even take the drag down---a
brute of a mare of mine had fell with me only two days before, out with
the Abingdon, and I thought my arm was broke. Well, sir, I couldn't
finish him, but Bob had his coat off at once---he stood up to the
Banbury man for three minutes, and polished him off in four rounds
easy.  Gad, how he did drop, sir, and what was it? Blood, sir, all
blood.''

``You don't drink, James,'' the ex-attache continued. ``In my time at
Oxford, the men passed round the bottle a little quicker than you young
fellows seem to do.''

``Come, come,'' said James, putting his hand to his nose and winking at
his cousin with a pair of vinous eyes, ``no jokes, old boy; no trying it
on on me.  You want to trot me out, but it's no go.  In vino veritas,
old boy.  Mars, Bacchus, Apollo virorum, hey? I wish my aunt would send
down some of this to the governor; it's a precious good tap.''

``You had better ask her,'' Machiavel continued, ``or make the best of
your time now.  What says the bard? 'Nunc vino pellite curas, Cras
ingens iterabimus aequor,''' and the Bacchanalian, quoting the above
with a House of Commons air, tossed off nearly a thimbleful of wine
with an immense flourish of his glass.

At the Rectory, when the bottle of port wine was opened after dinner,
the young ladies had each a glass from a bottle of currant wine.  Mrs.\ %
Bute took one glass of port, honest James had a couple commonly, but as
his father grew very sulky if he made further inroads on the bottle,
the good lad generally refrained from trying for more, and subsided
either into the currant wine, or to some private gin-and-water in the
stables, which he enjoyed in the company of the coachman and his pipe.
At Oxford, the quantity of wine was unlimited, but the quality was
inferior:  but when quantity and quality united as at his aunt's house,
James showed that he could appreciate them indeed; and hardly needed
any of his cousin's encouragement in draining off the second bottle
supplied by Mr.\ Bowls.

When the time for coffee came, however, and for a return to the ladies,
of whom he stood in awe, the young gentleman's agreeable frankness left
him, and he relapsed into his usual surly timidity; contenting himself
by saying yes and no, by scowling at Lady Jane, and by upsetting one
cup of coffee during the evening.

If he did not speak he yawned in a pitiable manner, and his presence
threw a damp upon the modest proceedings of the evening, for Miss
Crawley and Lady Jane at their piquet, and Miss Briggs at her work,
felt that his eyes were wildly fixed on them, and were uneasy under
that maudlin look.

``He seems a very silent, awkward, bashful lad,'' said Miss Crawley to
Mr.\ Pitt.

``He is more communicative in men's society than with ladies,'' Machiavel
dryly replied:  perhaps rather disappointed that the port wine had not
made Jim speak more.

He had spent the early part of the next morning in writing home to his
mother a most flourishing account of his reception by Miss Crawley.
But ah! he little knew what evils the day was bringing for him, and how
short his reign of favour was destined to be.  A circumstance which Jim
had forgotten---a trivial but fatal circumstance---had taken place at the
Cribb's Arms on the night before he had come to his aunt's house.  It
was no other than this---Jim, who was always of a generous disposition,
and when in his cups especially hospitable, had in the course of the
night treated the Tutbury champion and the Rottingdean man, and their
friends, twice or thrice to the refreshment of gin-and-water---so that
no less than eighteen glasses of that fluid at eightpence per glass
were charged in Mr.\ James Crawley's bill.  It was not the amount of
eightpences, but the quantity of gin which told fatally against poor
James's character, when his aunt's butler, Mr.\ Bowls, went down at his
mistress's request to pay the young gentleman's bill.  The landlord,
fearing lest the account should be refused altogether, swore solemnly
that the young gent had consumed personally every farthing's worth of
the liquor:  and Bowls paid the bill finally, and showed it on his
return home to Mrs.\ Firkin, who was shocked at the frightful
prodigality of gin; and took the bill to Miss Briggs as
accountant-general; who thought it her duty to mention the circumstance
to her principal, Miss Crawley.

Had he drunk a dozen bottles of claret, the old spinster could have
pardoned him.  Mr.\ Fox and Mr.\ Sheridan drank claret.  Gentlemen drank
claret.  But eighteen glasses of gin consumed among boxers in an
ignoble pot-house---it was an odious crime and not to be pardoned
readily.  Everything went against the lad:  he came home perfumed from
the stables, whither he had been to pay his dog Towzer a visit---and
whence he was going to take his friend out for an airing, when he met
Miss Crawley and her wheezy Blenheim spaniel, which Towzer would have
eaten up had not the Blenheim fled squealing to the protection of Miss
Briggs, while the atrocious master of the bull-dog stood laughing at
the horrible persecution.

This day too the unlucky boy's modesty had likewise forsaken him. He
was lively and facetious at dinner. During the repast he levelled one
or two jokes against Pitt Crawley:  he drank as much wine as upon the
previous day; and going quite unsuspiciously to the drawing-room, began
to entertain the ladies there with some choice Oxford stories.  He
described the different pugilistic qualities of Molyneux and Dutch Sam,
offered playfully to give Lady Jane the odds upon the Tutbury Pet
against the Rottingdean man, or take them, as her Ladyship chose: and
crowned the pleasantry by proposing to back himself against his cousin
Pitt Crawley, either with or without the gloves.  ``And that's a fair
offer, my buck,'' he said, with a loud laugh, slapping Pitt on the
shoulder, ``and my father told me to make it too, and he'll go halves in
the bet, ha, ha!'' So saying, the engaging youth nodded knowingly at
poor Miss Briggs, and pointed his thumb over his shoulder at Pitt
Crawley in a jocular and exulting manner.

Pitt was not pleased altogether perhaps, but still not unhappy in the
main.  Poor Jim had his laugh out:  and staggered across the room with
his aunt's candle, when the old lady moved to retire, and offered to
salute her with the blandest tipsy smile:  and he took his own leave
and went upstairs to his bedroom perfectly satisfied with himself, and
with a pleased notion that his aunt's money would be left to him in
preference to his father and all the rest of the family.

Once up in the bedroom, one would have thought he could not make
matters worse; and yet this unlucky boy did.  The moon was shining very
pleasantly out on the sea, and Jim, attracted to the window by the
romantic appearance of the ocean and the heavens, thought he would
further enjoy them while smoking.  Nobody would smell the tobacco, he
thought, if he cunningly opened the window and kept his head and pipe
in the fresh air. This he did:  but being in an excited state, poor Jim
had forgotten that his door was open all this time, so that the breeze
blowing inwards and a fine thorough draught being established, the
clouds of tobacco were carried downstairs, and arrived with quite
undiminished fragrance to Miss Crawley and Miss Briggs.

The pipe of tobacco finished the business:  and the Bute-Crawleys never
knew how many thousand pounds it cost them.  Firkin rushed downstairs
to Bowls who was reading out the ``Fire and the Frying Pan'' to his
aide-de-camp in a loud and ghostly voice.  The dreadful secret was told
to him by Firkin with so frightened a look, that for the first moment
Mr.\ Bowls and his young man thought that robbers were in the house, the
legs of whom had probably been discovered by the woman under Miss
Crawley's bed.  When made aware of the fact, however---to rush upstairs
at three steps at a time to enter the unconscious James's apartment,
calling out, ``Mr.\ James,'' in a voice stifled with alarm, and to cry,
``For Gawd's sake, sir, stop that 'ere pipe,'' was the work of a minute
with Mr.\ Bowls.  ``O, Mr.\ James, what 'AVE you done!'' he said in a voice
of the deepest pathos, as he threw the implement out of the window.
``What 'ave you done, sir! Missis can't abide 'em.''

``Missis needn't smoke,'' said James with a frantic misplaced laugh, and
thought the whole matter an excellent joke.  But his feelings were very
different in the morning, when Mr.\ Bowls's young man, who operated upon
Mr.\ James's boots, and brought him his hot water to shave that beard
which he was so anxiously expecting, handed a note in to Mr.\ James in
bed, in the handwriting of Miss Briggs.

``Dear sir,'' it said, ``Miss Crawley has passed an exceedingly disturbed
night, owing to the shocking manner in which the house has been
polluted by tobacco; Miss Crawley bids me say she regrets that she is
too unwell to see you before you go---and above all that she ever
induced you to remove from the ale-house, where she is sure you will be
much more comfortable during the rest of your stay at Brighton.''

And herewith honest James's career as a candidate for his aunt's favour
ended.  He had in fact, and without knowing it, done what he menaced to
do.  He had fought his cousin Pitt with the gloves.

Where meanwhile was he who had been once first favourite for this race
for money? Becky and Rawdon, as we have seen, were come together after
Waterloo, and were passing the winter of 1815 at Paris in great
splendour and gaiety.  Rebecca was a good economist, and the price poor
Jos Sedley had paid for her two horses was in itself sufficient to keep
their little establishment afloat for a year, at the least; there was
no occasion to turn into money ``my pistols, the same which I shot
Captain Marker,'' or the gold dressing-case, or the cloak lined with
sable.  Becky had it made into a pelisse for herself, in which she rode
in the Bois de Boulogne to the admiration of all:  and you should have
seen the scene between her and her delighted husband, whom she rejoined
after the army had entered Cambray, and when she unsewed herself, and
let out of her dress all those watches, knick-knacks, bank-notes,
cheques, and valuables, which she had secreted in the wadding, previous
to her meditated flight from Brussels!  Tufto was charmed, and Rawdon
roared with delighted laughter, and swore that she was better than any
play he ever saw, by Jove. And the way in which she jockeyed Jos, and
which she described with infinite fun, carried up his delight to a
pitch of quite insane enthusiasm.  He believed in his wife as much as
the French soldiers in Napoleon.

Her success in Paris was remarkable.  All the French ladies voted her
charming.  She spoke their language admirably.  She adopted at once
their grace, their liveliness, their manner.  Her husband was stupid
certainly---all English are stupid---and, besides, a dull husband at
Paris is always a point in a lady's favour.  He was the heir of the
rich and spirituelle Miss Crawley, whose house had been open to so many
of the French noblesse during the emigration.  They received the
colonel's wife in their own hotels---``Why,'' wrote a great lady to Miss
Crawley, who had bought her lace and trinkets at the Duchess's own
price, and given her many a dinner during the pinching times after the
Revolution---``Why does not our dear Miss come to her nephew and niece,
and her attached friends in Paris? All the world raffoles of the
charming Mistress and her espiegle beauty. Yes, we see in her the
grace, the charm, the wit of our dear friend Miss Crawley! The King
took notice of her yesterday at the Tuileries, and we are all jealous
of the attention which Monsieur pays her.  If you could have seen the
spite of a certain stupid Miladi Bareacres (whose eagle-beak and toque
and feathers may be seen peering over the heads of all assemblies) when
Madame, the Duchess of Angouleme, the august daughter and companion of
kings, desired especially to be presented to Mrs.\ Crawley, as your dear
daughter and protegee, and thanked her in the name of France, for all
your benevolence towards our unfortunates during their exile! She is of
all the societies, of all the balls---of the balls---yes---of the dances,
no; and yet how interesting and pretty this fair creature looks
surrounded by the homage of the men, and so soon to be a mother!  To
hear her speak of you, her protectress, her mother, would bring tears
to the eyes of ogres.  How she loves you! how we all love our
admirable, our respectable Miss Crawley!''

It is to be feared that this letter of the Parisian great lady did not
by any means advance Mrs.\ Becky's interest with her admirable, her
respectable, relative.  On the contrary, the fury of the old spinster
was beyond bounds, when she found what was Rebecca's situation, and how
audaciously she had made use of Miss Crawley's name, to get an entree
into Parisian society.  Too much shaken in mind and body to compose a
letter in the French language in reply to that of her correspondent,
she dictated to Briggs a furious answer in her own native tongue,
repudiating Mrs.\ Rawdon Crawley altogether, and warning the public to
beware of her as a most artful and dangerous person.  But as Madame the
Duchess of X---had only been twenty years in England, she did not
understand a single word of the language, and contented herself by
informing Mrs.\ Rawdon Crawley at their next meeting, that she had
received a charming letter from that chere Mees, and that it was full
of benevolent things for Mrs.\ Crawley, who began seriously to have
hopes that the spinster would relent.

Meanwhile, she was the gayest and most admired of Englishwomen:  and
had a little European congress on her reception-night.  Prussians and
Cossacks, Spanish and English---all the world was at Paris during this
famous winter:  to have seen the stars and cordons in Rebecca's humble
saloon would have made all Baker Street pale with envy. Famous warriors
rode by her carriage in the Bois, or crowded her modest little box at
the Opera. Rawdon was in the highest spirits. There were no duns in
Paris as yet:  there were parties every day at Very's or Beauvilliers';
play was plentiful and his luck good. Tufto perhaps was sulky.  Mrs.\ %
Tufto had come over to Paris at her own invitation, and besides this
contretemps, there were a score of generals now round Becky's chair,
and she might take her choice of a dozen bouquets when she went to the
play.  Lady Bareacres and the chiefs of the English society, stupid and
irreproachable females, writhed with anguish at the success of the
little upstart Becky, whose poisoned jokes quivered and rankled in
their chaste breasts. But she had all the men on her side.  She fought
the women with indomitable courage, and they could not talk scandal in
any tongue but their own.

So in fetes, pleasures, and prosperity, the winter of 1815-16 passed
away with Mrs.\ Rawdon Crawley, who accommodated herself to polite life
as if her ancestors had been people of fashion for centuries past---and
who from her wit, talent, and energy, indeed merited a place of honour
in Vanity Fair.  In the early spring of 1816, Galignani's Journal
contained the following announcement in an interesting corner of the
paper:  ``On the 26th of March---the Lady of Lieutenant-Colonel Crawley,
of the Life Guards Green---of a son and heir.''

This event was copied into the London papers, out of which Miss Briggs
read the statement to Miss Crawley, at breakfast, at Brighton.  The
intelligence, expected as it might have been, caused a crisis in the
affairs of the Crawley family.  The spinster's rage rose to its height,
and sending instantly for Pitt, her nephew, and for the Lady Southdown,
from Brunswick Square, she requested an immediate celebration of the
marriage which had been so long pending between the two families.  And
she announced that it was her intention to allow the young couple a
thousand a year during her lifetime, at the expiration of which the
bulk of her property would be settled upon her nephew and her dear
niece, Lady Jane Crawley. Waxy came down to ratify the deeds---Lord
Southdown gave away his sister---she was married by a Bishop, and not by
the Rev. Bartholomew Irons---to the disappointment of the irregular
prelate.

When they were married, Pitt would have liked to take a hymeneal tour
with his bride, as became people of their condition.  But the affection
of the old lady towards Lady Jane had grown so strong, that she fairly
owned she could not part with her favourite.  Pitt and his wife came
therefore and lived with Miss Crawley: and (greatly to the annoyance of
poor Pitt, who conceived himself a most injured character---being
subject to the humours of his aunt on one side, and of his
mother-in-law on the other). Lady Southdown, from her neighbouring
house, reigned over the whole family---Pitt, Lady Jane, Miss Crawley,
Briggs, Bowls, Firkin, and all.  She pitilessly dosed them with her
tracts and her medicine, she dismissed Creamer, she installed Rodgers,
and soon stripped Miss Crawley of even the semblance of authority.  The
poor soul grew so timid that she actually left off bullying Briggs any
more, and clung to her niece, more fond and terrified every day.  Peace
to thee, kind and selfish, vain and generous old heathen!---We shall see
thee no more.  Let us hope that Lady Jane supported her kindly, and led
her with gentle hand out of the busy struggle of Vanity Fair.



\chapter{Widow and Mother}

The news of the great fights of Quatre Bras and Waterloo reached
England at the same time.  The Gazette first published the result of
the two battles; at which glorious intelligence all England thrilled
with triumph and fear. Particulars then followed; and after the
announcement of the victories came the list of the wounded and the
slain. Who can tell the dread with which that catalogue was opened and
read!  Fancy, at every village and homestead almost through the three
kingdoms, the great news coming of the battles in Flanders, and the
feelings of exultation and gratitude, bereavement and sickening dismay,
when the lists of the regimental losses were gone through, and it
became known whether the dear friend and relative had escaped or
fallen.  Anybody who will take the trouble of looking back to a file of
the newspapers of the time, must, even now, feel at second-hand this
breathless pause of expectation.  The lists of casualties are carried
on from day to day:  you stop in the midst as in a story which is to be
continued in our next.  Think what the feelings must have been as those
papers followed each other fresh from the press; and if such an
interest could be felt in our country, and about a battle where but
twenty thousand of our people were engaged, think of the condition of
Europe for twenty years before, where people were fighting, not by
thousands, but by millions; each one of whom as he struck his enemy
wounded horribly some other innocent heart far away.

The news which that famous Gazette brought to the Osbornes gave a
dreadful shock to the family and its chief. The girls indulged
unrestrained in their grief.  The gloom-stricken old father was still
more borne down by his fate and sorrow.  He strove to think that a
judgment was on the boy for his disobedience.  He dared not own that
the severity of the sentence frightened him, and that its fulfilment
had come too soon upon his curses.  Sometimes a shuddering terror
struck him, as if he had been the author of the doom which he had
called down on his son.  There was a chance before of reconciliation.
The boy's wife might have died; or he might have come back and said,
Father I have sinned.  But there was no hope now.  He stood on the
other side of the gulf impassable, haunting his parent with sad eyes.
He remembered them once before so in a fever, when every one thought
the lad was dying, and he lay on his bed speechless, and gazing with a
dreadful gloom.  Good God! how the father clung to the doctor then, and
with what a sickening anxiety he followed him:  what a weight of grief
was off his mind when, after the crisis of the fever, the lad
recovered, and looked at his father once more with eyes that recognised
him. But now there was no help or cure, or chance of reconcilement:
above all, there were no humble words to soothe vanity outraged and
furious, or bring to its natural flow the poisoned, angry blood.  And
it is hard to say which pang it was that tore the proud father's heart
most keenly---that his son should have gone out of the reach of his
forgiveness, or that the apology which his own pride expected should
have escaped him.

Whatever his sensations might have been, however, the stern old man
would have no confidant.  He never mentioned his son's name to his
daughters; but ordered the elder to place all the females of the
establishment in mourning; and desired that the male servants should be
similarly attired in deep black.  All parties and entertainments, of
course, were to be put off.  No communications were made to his future
son-in-law, whose marriage-day had been fixed:  but there was enough in
Mr.\ Osborne's appearance to prevent Mr.\ Bullock from making any
inquiries, or in any way pressing forward that ceremony. He and the
ladies whispered about it under their voices in the drawing-room
sometimes, whither the father never came.  He remained constantly in
his own study; the whole front part of the house being closed until
some time after the completion of the general mourning.

About three weeks after the 18th of June, Mr.\ Osborne's acquaintance,
Sir William Dobbin, called at Mr.\ Osborne's house in Russell Square,
with a very pale and agitated face, and insisted upon seeing that
gentleman. Ushered into his room, and after a few words, which neither
the speaker nor the host understood, the former produced from an
inclosure a letter sealed with a large red seal. ``My son, Major
Dobbin,'' the Alderman said, with some hesitation, ``despatched me a
letter by an officer of the ---th, who arrived in town to-day.  My son's
letter contains one for you, Osborne.'' The Alderman placed the letter
on the table, and Osborne stared at him for a moment or two in silence.
His looks frightened the ambassador, who after looking guiltily for a
little time at the grief-stricken man, hurried away without another
word.

The letter was in George's well-known bold handwriting. It was that one
which he had written before daybreak on the 16th of June, and just
before he took leave of Amelia.  The great red seal was emblazoned with
the sham coat of arms which Osborne had assumed from the Peerage, with
``Pax in bello'' for a motto; that of the ducal house with which the vain
old man tried to fancy himself connected. The hand that signed it would
never hold pen or sword more.  The very seal that sealed it had been
robbed from George's dead body as it lay on the field of battle.  The
father knew nothing of this, but sat and looked at the letter in
terrified vacancy.  He almost fell when he went to open it.

Have you ever had a difference with a dear friend? How his letters,
written in the period of love and confidence, sicken and rebuke you!
What a dreary mourning it is to dwell upon those vehement protests of
dead affection!  What lying epitaphs they make over the corpse of love!
What dark, cruel comments upon Life and Vanities! Most of us have got
or written drawers full of them. They are closet-skeletons which we
keep and shun. Osborne trembled long before the letter from his dead
son.

The poor boy's letter did not say much.  He had been too proud to
acknowledge the tenderness which his heart felt.  He only said, that on
the eve of a great battle, he wished to bid his father farewell, and
solemnly to implore his good offices for the wife---it might be for the
child---whom he left behind him.  He owned with contrition that his
irregularities and his extravagance had already wasted a large part of
his mother's little fortune.  He thanked his father for his former
generous conduct; and he promised him that if he fell on the field or
survived it, he would act in a manner worthy of the name of George
Osborne.

His English habit, pride, awkwardness perhaps, had prevented him from
saying more.  His father could not see the kiss George had placed on
the superscription of his letter.  Mr.\ Osborne dropped it with the
bitterest, deadliest pang of balked affection and revenge. His son was
still beloved and unforgiven.

About two months afterwards, however, as the young ladies of the family
went to church with their father, they remarked how he took a different
seat from that which he usually occupied when he chose to attend divine
worship; and that from his cushion opposite, he looked up at the wall
over their heads.  This caused the young women likewise to gaze in the
direction towards which their father's gloomy eyes pointed:  and they
saw an elaborate monument upon the wall, where Britannia was
represented weeping over an urn, and a broken sword and a couchant lion
indicated that the piece of sculpture had been erected in honour of a
deceased warrior.  The sculptors of those days had stocks of such
funereal emblems in hand; as you may see still on the walls of St.\ %
Paul's, which are covered with hundreds of these braggart heathen
allegories.  There was a constant demand for them during the first
fifteen years of the present century.

Under the memorial in question were emblazoned the well-known and
pompous Osborne arms; and the inscription said, that the monument was
``Sacred to the memory of George Osborne, Junior, Esq., late a Captain
in his Majesty's ---th regiment of foot, who fell on the 18th of June,
1815, aged 28 years, while fighting for his king and country in the
glorious victory of Waterloo. Dulce et decorum est pro patria mori.''

The sight of that stone agitated the nerves of the sisters so much,
that Miss Maria was compelled to leave the church.  The congregation
made way respectfully for those sobbing girls clothed in deep black,
and pitied the stern old father seated opposite the memorial of the
dead soldier.  ``Will he forgive Mrs.\ George?'' the girls said to
themselves as soon as their ebullition of grief was over. Much
conversation passed too among the acquaintances of the Osborne family,
who knew of the rupture between the son and father caused by the
former's marriage, as to the chance of a reconciliation with the young
widow. There were bets among the gentlemen both about Russell Square
and in the City.

If the sisters had any anxiety regarding the possible recognition of
Amelia as a daughter of the family, it was increased presently, and
towards the end of the autumn, by their father's announcement that he
was going abroad.  He did not say whither, but they knew at once that
his steps would be turned towards Belgium, and were aware that George's
widow was still in Brussels.  They had pretty accurate news indeed of
poor Amelia from Lady Dobbin and her daughters.  Our honest Captain had
been promoted in consequence of the death of the second Major of the
regiment on the field; and the brave O'Dowd, who had distinguished
himself greatly here as upon all occasions where he had a chance to
show his coolness and valour, was a Colonel and Companion of the Bath.

Very many of the brave ---th, who had suffered severely upon both days
of action, were still at Brussels in the autumn, recovering of their
wounds.  The city was a vast military hospital for months after the
great battles; and as men and officers began to rally from their hurts,
the gardens and places of public resort swarmed with maimed warriors,
old and young, who, just rescued out of death, fell to gambling, and
gaiety, and love-making, as people of Vanity Fair will do.  Mr.\ Osborne
found out some of the ---th easily.  He knew their uniform quite well,
and had been used to follow all the promotions and exchanges in the
regiment, and loved to talk about it and its officers as if he had been
one of the number.  On the day after his arrival at Brussels, and as he
issued from his hotel, which faced the park, he saw a soldier in the
well-known facings, reposing on a stone bench in the garden, and went
and sate down trembling by the wounded convalescent man.

``Were you in Captain Osborne's company?'' he said, and added, after a
pause, ``he was my son, sir.''

The man was not of the Captain's company, but he lifted up his
unwounded arm and touched his cap sadly and respectfully to the haggard
broken-spirited gentleman who questioned him.  ``The whole army didn't
contain a finer or a better officer,'' the soldier said. ``The Sergeant
of the Captain's company (Captain Raymond had it now), was in town,
though, and was just well of a shot in the shoulder. His honour might
see him if he liked, who could tell him anything he wanted to know
about---about the ---th's actions.  But his honour had seen Major Dobbin,
no doubt, the brave Captain's great friend; and Mrs.\ Osborne, who was
here too, and had been very bad, he heard everybody say.  They say she
was out of her mind like for six weeks or more.  But your honour knows
all about that---and asking your pardon''---the man added.

Osborne put a guinea into the soldier's hand, and told him he should
have another if he would bring the Sergeant to the Hotel du Parc; a
promise which very soon brought the desired officer to Mr.\ Osborne's
presence. And the first soldier went away; and after telling a comrade
or two how Captain Osborne's father was arrived, and what a free-handed
generous gentleman he was, they went and made good cheer with drink and
feasting, as long as the guineas lasted which had come from the proud
purse of the mourning old father.

In the Sergeant's company, who was also just convalescent, Osborne made
the journey of Waterloo and Quatre Bras, a journey which thousands of
his countrymen were then taking.  He took the Sergeant with him in his
carriage, and went through both fields under his guidance.  He saw the
point of the road where the regiment marched into action on the 16th,
and the slope down which they drove the French cavalry who were
pressing on the retreating Belgians.  There was the spot where the
noble Captain cut down the French officer who was grappling with the
young Ensign for the colours, the Colour-Sergeants having been shot
down.  Along this road they retreated on the next day, and here was the
bank at which the regiment bivouacked under the rain of the night of
the seventeenth.  Further on was the position which they took and held
during the day, forming time after time to receive the charge of the
enemy's horsemen and lying down under the shelter of the bank from the
furious French cannonade. And it was at this declivity when at evening
the whole English line received the order to advance, as the enemy fell
back after his last charge, that the Captain, hurraying and rushing
down the hill waving his sword, received a shot and fell dead.  ``It was
Major Dobbin who took back the Captain's body to Brussels,'' the
Sergeant said, in a low voice, ``and had him buried, as your honour
knows.'' The peasants and relic-hunters about the place were screaming
round the pair, as the soldier told his story, offering for sale all
sorts of mementoes of the fight, crosses, and epaulets, and shattered
cuirasses, and eagles.

Osborne gave a sumptuous reward to the Sergeant when he parted with
him, after having visited the scenes of his son's last exploits. His
burial-place he had already seen.  Indeed, he had driven thither
immediately after his arrival at Brussels.  George's body lay in the
pretty burial-ground of Laeken, near the city; in which place, having
once visited it on a party of pleasure, he had lightly expressed a wish
to have his grave made.  And there the young officer was laid by his
friend, in the unconsecrated corner of the garden, separated by a
little hedge from the temples and towers and plantations of flowers and
shrubs, under which the Roman Catholic dead repose.  It seemed a
humiliation to old Osborne to think that his son, an English gentleman,
a captain in the famous British army, should not be found worthy to lie
in ground where mere foreigners were buried.  Which of us is there can
tell how much vanity lurks in our warmest regard for others, and how
selfish our love is? Old Osborne did not speculate much upon the
mingled nature of his feelings, and how his instinct and selfishness
were combating together.  He firmly believed that everything he did was
right, that he ought on all occasions to have his own way---and like the
sting of a wasp or serpent his hatred rushed out armed and poisonous
against anything like opposition.  He was proud of his hatred as of
everything else.  Always to be right, always to trample forward, and
never to doubt, are not these the great qualities with which dullness
takes the lead in the world?

As after the drive to Waterloo, Mr.\ Osborne's carriage was nearing the
gates of the city at sunset, they met another open barouche, in which
were a couple of ladies and a gentleman, and by the side of which an
officer was riding.  Osborne gave a start back, and the Sergeant,
seated with him, cast a look of surprise at his neighbour, as he
touched his cap to the officer, who mechanically returned his salute.
It was Amelia, with the lame young Ensign by her side, and opposite to
her her faithful friend Mrs.\ O'Dowd.  It was Amelia, but how changed
from the fresh and comely girl Osborne knew.  Her face was white and
thin.  Her pretty brown hair was parted under a widow's cap---the poor
child.  Her eyes were fixed, and looking nowhere.  They stared blank in
the face of Osborne, as the carriages crossed each other, but she did
not know him; nor did he recognise her, until looking up, he saw Dobbin
riding by her:  and then he knew who it was.  He hated her.  He did not
know how much until he saw her there.  When her carriage had passed on,
he turned and stared at the Sergeant, with a curse and defiance in his
eye cast at his companion, who could not help looking at him---as much
as to say ``How dare you look at me? Damn you!  I do hate her.  It is
she who has tumbled my hopes and all my pride down.'' ``Tell the
scoundrel to drive on quick,'' he shouted with an oath, to the lackey on
the box. A minute afterwards, a horse came clattering over the pavement
behind Osborne's carriage, and Dobbin rode up.  His thoughts had been
elsewhere as the carriages passed each other, and it was not until he
had ridden some paces forward, that he remembered it was Osborne who
had just passed him.  Then he turned to examine if the sight of her
father-in-law had made any impression on Amelia, but the poor girl did
not know who had passed.  Then William, who daily used to accompany her
in his drives, taking out his watch, made some excuse about an
engagement which he suddenly recollected, and so rode off.  She did not
remark that either:  but sate looking before her, over the homely
landscape towards the woods in the distance, by which George marched
away.

``Mr.\ Osborne, Mr.\ Osborne!'' cried Dobbin, as he rode up and held out
his hand.  Osborne made no motion to take it, but shouted out once more
and with another curse to his servant to drive on.

Dobbin laid his hand on the carriage side.  ``I will see you, sir,'' he
said.  ``I have a message for you.''

``From that woman?'' said Osborne, fiercely.

``No,'' replied the other, ``from your son''; at which Osborne fell back
into the corner of his carriage, and Dobbin allowing it to pass on,
rode close behind it, and so through the town until they reached Mr.\ %
Osborne's hotel, and without a word.  There he followed Osborne up to
his apartments.  George had often been in the rooms; they were the
lodgings which the Crawleys had occupied during their stay in Brussels.

``Pray, have you any commands for me, Captain Dobbin, or, I beg your
pardon, I should say MAJOR Dobbin, since better men than you are dead,
and you step into their SHOES?'' said Mr.\ Osborne, in that sarcastic
tone which he sometimes was pleased to assume.

``Better men ARE dead,'' Dobbin replied.  ``I want to speak to you about
one.''

``Make it short, sir,'' said the other with an oath, scowling at his
visitor.

``I am here as his closest friend,'' the Major resumed, ``and the executor
of his will.  He made it before he went into action.  Are you aware how
small his means are, and of the straitened circumstances of his widow?''

``I don't know his widow, sir,'' Osborne said.  ``Let her go back to her
father.'' But the gentleman whom he addressed was determined to remain
in good temper, and went on without heeding the interruption.

``Do you know, sir, Mrs.\ Osborne's condition? Her life and her reason
almost have been shaken by the blow which has fallen on her.  It is
very doubtful whether she will rally.  There is a chance left for her,
however, and it is about this I came to speak to you.  She will be a
mother soon.  Will you visit the parent's offence upon the child's
head? or will you forgive the child for poor George's sake?''

Osborne broke out into a rhapsody of self-praise and imprecations;---by
the first, excusing himself to his own conscience for his conduct; by
the second, exaggerating the undutifulness of George. No father in all
England could have behaved more generously to a son, who had rebelled
against him wickedly.  He had died without even so much as confessing
he was wrong.  Let him take the consequences of his undutifulness and
folly.  As for himself, Mr.\ Osborne, he was a man of his word.  He had
sworn never to speak to that woman, or to recognize her as his son's
wife.  ``And that's what you may tell her,'' he concluded with an oath;
``and that's what I will stick to to the last day of my life.''

There was no hope from that quarter then.  The widow must live on her
slender pittance, or on such aid as Jos could give her.  ``I might tell
her, and she would not heed it,'' thought Dobbin, sadly: for the poor
girl's thoughts were not here at all since her catastrophe, and,
stupefied under the pressure of her sorrow, good and evil were alike
indifferent to her.

So, indeed, were even friendship and kindness.  She received them both
uncomplainingly, and having accepted them, relapsed into her grief.

Suppose some twelve months after the above conversation took place to
have passed in the life of our poor Amelia.  She has spent the first
portion of that time in a sorrow so profound and pitiable, that we who
have been watching and describing some of the emotions of that weak and
tender heart, must draw back in the presence of the cruel grief under
which it is bleeding.  Tread silently round the hapless couch of the
poor prostrate soul. Shut gently the door of the dark chamber wherein
she suffers, as those kind people did who nursed her through the first
months of her pain, and never left her until heaven had sent her
consolation.  A day came---of almost terrified delight and wonder---when
the poor widowed girl pressed a child upon her breast---a child, with
the eyes of George who was gone---a little boy, as beautiful as a
cherub.  What a miracle it was to hear its first cry!  How she laughed
and wept over it---how love, and hope, and prayer woke again in her
bosom as the baby nestled there.  She was safe.  The doctors who
attended her, and had feared for her life or for her brain, had waited
anxiously for this crisis before they could pronounce that either was
secure.  It was worth the long months of doubt and dread which the
persons who had constantly been with her had passed, to see her eyes
once more beaming tenderly upon them.

Our friend Dobbin was one of them.  It was he who brought her back to
England and to her mother's house; when Mrs.\ O'Dowd, receiving a
peremptory summons from her Colonel, had been forced to quit her
patient. To see Dobbin holding the infant, and to hear Amelia's laugh
of triumph as she watched him, would have done any man good who had a
sense of humour.  William was the godfather of the child, and exerted
his ingenuity in the purchase of cups, spoons, pap-boats, and corals
for this little Christian.

How his mother nursed him, and dressed him, and lived upon him; how she
drove away all nurses, and would scarce allow any hand but her own to
touch him; how she considered that the greatest favour she could confer
upon his godfather, Major Dobbin, was to allow the Major occasionally
to dandle him, need not be told here.  This child was her being.  Her
existence was a maternal caress.  She enveloped the feeble and
unconscious creature with love and worship.  It was her life which the
baby drank in from her bosom.  Of nights, and when alone, she had
stealthy and intense raptures of motherly love, such as God's
marvellous care has awarded to the female instinct---joys how far
higher and lower than reason---blind beautiful devotions which only
women's hearts know.  It was William Dobbin's task to muse upon these
movements of Amelia's, and to watch her heart; and if his love made him
divine almost all the feelings which agitated it, alas! he could see
with a fatal perspicuity that there was no place there for him.  And
so, gently, he bore his fate, knowing it, and content to bear it.

I suppose Amelia's father and mother saw through the intentions of the
Major, and were not ill-disposed to encourage him; for Dobbin visited
their house daily, and stayed for hours with them, or with Amelia, or
with the honest landlord, Mr.\ Clapp, and his family.  He brought, on
one pretext or another, presents to everybody, and almost every day;
and went, with the landlord's little girl, who was rather a favourite
with Amelia, by the name of Major Sugarplums.  It was this little child
who commonly acted as mistress of the ceremonies to introduce him to
Mrs.\ Osborne.  She laughed one day when Major Sugarplums' cab drove up
to Fulham, and he descended from it, bringing out a wooden horse, a
drum, a trumpet, and other warlike toys, for little Georgy, who was
scarcely six months old, and for whom the articles in question were
entirely premature.

The child was asleep.  ``Hush,'' said Amelia, annoyed, perhaps, at the
creaking of the Major's boots; and she held out her hand; smiling
because William could not take it until he had rid himself of his cargo
of toys.  ``Go downstairs, little Mary,'' said he presently to the child,
``I want to speak to Mrs.\ Osborne.'' She looked up rather astonished, and
laid down the infant on its bed.

``I am come to say good-bye, Amelia,'' said he, taking her slender little
white hand gently.

``Good-bye? and where are you going?'' she said, with a smile.

``Send the letters to the agents,'' he said; ``they will forward them; for
you will write to me, won't you? I shall be away a long time.''

``I'll write to you about Georgy,'' she said.  ``Dear William, how good
you have been to him and to me.  Look at him.  Isn't he like an angel?''

The little pink hands of the child closed mechanically round the honest
soldier's finger, and Amelia looked up in his face with bright maternal
pleasure.  The cruellest looks could not have wounded him more than
that glance of hopeless kindness.  He bent over the child and mother.
He could not speak for a moment.  And it was only with all his strength
that he could force himself to say a God bless you.  ``God bless you,''
said Amelia, and held up her face and kissed him.

``Hush!  Don't wake Georgy!'' she added, as William Dobbin went to the
door with heavy steps.  She did not hear the noise of his cab-wheels as
he drove away:  she was looking at the child, who was laughing in his
sleep.



\chapter{How to Live Well on Nothing a Year}

I suppose there is no man in this Vanity Fair of ours so little
observant as not to think sometimes about the worldly affairs of his
acquaintances, or so extremely charitable as not to wonder how his
neighbour Jones, or his neighbour Smith, can make both ends meet at the
end of the year.  With the utmost regard for the family, for instance
(for I dine with them twice or thrice in the season), I cannot but own
that the appearance of the Jenkinses in the park, in the large barouche
with the grenadier-footmen, will surprise and mystify me to my dying
day:  for though I know the equipage is only jobbed, and all the
Jenkins people are on board wages, yet those three men and the carriage
must represent an expense of six hundred a year at the very least---and
then there are the splendid dinners, the two boys at Eton, the prize
governess and masters for the girls, the trip abroad, or to Eastbourne
or Worthing, in the autumn, the annual ball with a supper from Gunter's
(who, by the way, supplies most of the first-rate dinners which J.
gives, as I know very well, having been invited to one of them to fill
a vacant place, when I saw at once that these repasts are very superior
to the common run of entertainments for which the humbler sort of J.'s
acquaintances get cards)---who, I say, with the most good-natured
feelings in the world, can help wondering how the Jenkinses make out
matters? What is Jenkins? We all know---Commissioner of the Tape and
Sealing Wax Office, with 1200 pounds a year for a salary.  Had his wife
a private fortune? Pooh!---Miss Flint---one of eleven children of a small
squire in Buckinghamshire.  All she ever gets from her family is a
turkey at Christmas, in exchange for which she has to board two or
three of her sisters in the off season, and lodge and feed her brothers
when they come to town.  How does Jenkins balance his income? I say, as
every friend of his must say, How is it that he has not been outlawed
long since, and that he ever came back (as he did to the surprise of
everybody) last year from Boulogne?

``I'' is here introduced to personify the world in general---the Mrs.\ %
Grundy of each respected reader's private circle---every one of whom can
point to some families of his acquaintance who live nobody knows how.
Many a glass of wine have we all of us drunk, I have very little doubt,
hob-and-nobbing with the hospitable giver and wondering how the deuce
he paid for it.

Some three or four years after his stay in Paris, when Rawdon Crawley
and his wife were established in a very small comfortable house in
Curzon Street, May Fair, there was scarcely one of the numerous friends
whom they entertained at dinner that did not ask the above question
regarding them.  The novelist, it has been said before, knows
everything, and as I am in a situation to be able to tell the public
how Crawley and his wife lived without any income, may I entreat the
public newspapers which are in the habit of extracting portions of the
various periodical works now published not to reprint the following
exact narrative and calculations---of which I ought, as the discoverer
(and at some expense, too), to have the benefit? My son, I would say,
were I blessed with a child---you may by deep inquiry and constant
intercourse with him learn how a man lives comfortably on nothing a
year.  But it is best not to be intimate with gentlemen of this
profession and to take the calculations at second hand, as you do
logarithms, for to work them yourself, depend upon it, will cost you
something considerable.

On nothing per annum then, and during a course of some two or three
years, of which we can afford to give but a very brief history, Crawley
and his wife lived very happily and comfortably at Paris. It was in
this period that he quitted the Guards and sold out of the army.  When
we find him again, his mustachios and the title of Colonel on his card
are the only relics of his military profession.

It has been mentioned that Rebecca, soon after her arrival in Paris,
took a very smart and leading position in the society of that capital,
and was welcomed at some of the most distinguished houses of the
restored French nobility.  The English men of fashion in Paris courted
her, too, to the disgust of the ladies their wives, who could not bear
the parvenue.  For some months the salons of the Faubourg St.\ Germain,
in which her place was secured, and the splendours of the new Court,
where she was received with much distinction, delighted and perhaps a
little intoxicated Mrs.\ Crawley, who may have been disposed during this
period of elation to slight the people---honest young military men
mostly---who formed her husband's chief society.

But the Colonel yawned sadly among the Duchesses and great ladies of
the Court.  The old women who played ecarte made such a noise about a
five-franc piece that it was not worth Colonel Crawley's while to sit
down at a card-table.  The wit of their conversation he could not
appreciate, being ignorant of their language. And what good could his
wife get, he urged, by making curtsies every night to a whole circle of
Princesses? He left Rebecca presently to frequent these parties alone,
resuming his own simple pursuits and amusements amongst the amiable
friends of his own choice.

The truth is, when we say of a gentleman that he lives elegantly on
nothing a year, we use the word ``nothing'' to signify something unknown;
meaning, simply, that we don't know how the gentleman in question
defrays the expenses of his establishment.  Now, our friend the Colonel
had a great aptitude for all games of chance: and exercising himself,
as he continually did, with the cards, the dice-box, or the cue, it is
natural to suppose that he attained a much greater skill in the use of
these articles than men can possess who only occasionally handle them.
To use a cue at billiards well is like using a pencil, or a German
flute, or a small-sword---you cannot master any one of these implements
at first, and it is only by repeated study and perseverance, joined to
a natural taste, that a man can excel in the handling of either. Now
Crawley, from being only a brilliant amateur, had grown to be a
consummate master of billiards.  Like a great General, his genius used
to rise with the danger, and when the luck had been unfavourable to him
for a whole game, and the bets were consequently against him, he would,
with consummate skill and boldness, make some prodigious hits which
would restore the battle, and come in a victor at the end, to the
astonishment of everybody---of everybody, that is, who was a stranger to
his play.  Those who were accustomed to see it were cautious how they
staked their money against a man of such sudden resources and brilliant
and overpowering skill.

At games of cards he was equally skilful; for though he would
constantly lose money at the commencement of an evening, playing so
carelessly and making such blunders, that newcomers were often inclined
to think meanly of his talent; yet when roused to action and awakened
to caution by repeated small losses, it was remarked that Crawley's
play became quite different, and that he was pretty sure of beating his
enemy thoroughly before the night was over. Indeed, very few men could
say that they ever had the better of him. His successes were so
repeated that no wonder the envious and the vanquished spoke sometimes
with bitterness regarding them.  And as the French say of the Duke of
Wellington, who never suffered a defeat, that only an astonishing
series of lucky accidents enabled him to be an invariable winner; yet
even they allow that he cheated at Waterloo, and was enabled to win the
last great trick:  so it was hinted at headquarters in England that
some foul play must have taken place in order to account for the
continuous successes of Colonel Crawley.

Though Frascati's and the Salon were open at that time in Paris, the
mania for play was so widely spread that the public gambling-rooms did
not suffice for the general ardour, and gambling went on in private
houses as much as if there had been no public means for gratifying the
passion.  At Crawley's charming little \foreign{r\'eunions} of an evening this
fatal amusement commonly was practised---much to good-natured little
Mrs.\ Crawley's annoyance. She spoke about her husband's passion for
dice with the deepest grief; she bewailed it to everybody who came to
her house.  She besought the young fellows never, never to touch a box;
and when young Green, of the Rifles, lost a very considerable sum of
money, Rebecca passed a whole night in tears, as the servant told the
unfortunate young gentleman, and actually went on her knees to her
husband to beseech him to remit the debt, and burn the acknowledgement.
How could he? He had lost just as much himself to Blackstone of the
Hussars, and Count Punter of the Hanoverian Cavalry.  Green might have
any decent time; but pay?---of course he must pay; to talk of burning
IOU's was child's play.

Other officers, chiefly young---for the young fellows gathered round
Mrs.\ Crawley---came from her parties with long faces, having dropped
more or less money at her fatal card-tables.  Her house began to have
an unfortunate reputation.  The old hands warned the less experienced
of their danger.  Colonel O'Dowd, of the ---th regiment, one of those
occupying in Paris, warned Lieutenant Spooney of that corps.  A loud
and violent fracas took place between the infantry Colonel and his
lady, who were dining at the Cafe de Paris, and Colonel and Mrs.\ %
Crawley; who were also taking their meal there. The ladies engaged on
both sides.  Mrs.\ O'Dowd snapped her fingers in Mrs.\ Crawley's face and
called her husband ``no betther than a black-leg.'' Colonel Crawley
challenged Colonel O'Dowd, C.B.  The Commander-in-Chief hearing of the
dispute sent for Colonel Crawley, who was getting ready the same
pistols ``which he shot Captain Marker,'' and had such a conversation
with him that no duel took place.  If Rebecca had not gone on her knees
to General Tufto, Crawley would have been sent back to England; and he
did not play, except with civilians, for some weeks after.

But, in spite of Rawdon's undoubted skill and constant successes, it
became evident to Rebecca, considering these things, that their
position was but a precarious one, and that, even although they paid
scarcely anybody, their little capital would end one day by dwindling
into zero.  ``Gambling,'' she would say, ``dear, is good to help your
income, but not as an income itself.  Some day people may be tired of
play, and then where are we?'' Rawdon acquiesced in the justice of her
opinion; and in truth he had remarked that after a few nights of his
little suppers, \&c., gentlemen were tired of play with him, and, in
spite of Rebecca's charms, did not present themselves very eagerly.

Easy and pleasant as their life at Paris was, it was after all only an
idle dalliance and amiable trifling; and Rebecca saw that she must push
Rawdon's fortune in their own country.  She must get him a place or
appointment at home or in the colonies, and she determined to make a
move upon England as soon as the way could be cleared for her.  As a
first step she had made Crawley sell out of the Guards and go on
half-pay.  His function as aide-de-camp to General Tufto had ceased
previously. Rebecca laughed in all companies at that officer, at his
toupee (which he mounted on coming to Paris), at his waistband, at his
false teeth, at his pretensions to be a lady-killer above all, and his
absurd vanity in fancying every woman whom he came near was in love
with him.  It was to Mrs.\ Brent, the beetle-browed wife of Mr.\ %
Commissary Brent, to whom the general transferred his attentions
now---his bouquets, his dinners at the restaurateurs', his opera-boxes,
and his knick-knacks.  Poor Mrs.\ Tufto was no more happy than before,
and had still to pass long evenings alone with her daughters, knowing
that her General was gone off scented and curled to stand behind Mrs.\ %
Brent's chair at the play.  Becky had a dozen admirers in his place, to
be sure, and could cut her rival to pieces with her wit.  But, as we
have said, she was growing tired of this idle social life:
opera-boxes and restaurateur dinners palled upon her:  nosegays could
not be laid by as a provision for future years:  and she could not live
upon knick-knacks, laced handkerchiefs, and kid gloves.  She felt the
frivolity of pleasure and longed for more substantial benefits.

At this juncture news arrived which was spread among the many creditors
of the Colonel at Paris, and which caused them great satisfaction.
Miss Crawley, the rich aunt from whom he expected his immense
inheritance, was dying; the Colonel must haste to her bedside.  Mrs.\ %
Crawley and her child would remain behind until he came to reclaim
them.  He departed for Calais, and having reached that place in safety,
it might have been supposed that he went to Dover; but instead he took
the diligence to Dunkirk, and thence travelled to Brussels, for which
place he had a former predilection. The fact is, he owed more money at
London than at Paris; and he preferred the quiet little Belgian city to
either of the more noisy capitals.

Her aunt was dead.  Mrs.\ Crawley ordered the most intense mourning for
herself and little Rawdon.  The Colonel was busy arranging the affairs
of the inheritance.  They could take the premier now, instead of the
little entresol of the hotel which they occupied. Mrs.\ Crawley and the
landlord had a consultation about the new hangings, an amicable wrangle
about the carpets, and a final adjustment of everything except the
bill.  She went off in one of his carriages; her French bonne with her;
the child by her side; the admirable landlord and landlady smiling
farewell to her from the gate.  General Tufto was furious when he heard
she was gone, and Mrs.\ Brent furious with him for being furious;
Lieutenant Spooney was cut to the heart; and the landlord got ready his
best apartments previous to the return of the fascinating little woman
and her husband.  He \foreign{serr\'ed}
the trunks which she left in his charge 
with the greatest care.  They had been especially recommended to him by
Madame Crawley.  They were not, however, found to be particularly
valuable when opened some time after.

But before she went to join her husband in the Belgic capital, Mrs.\ %
Crawley made an expedition into England, leaving behind her her little
son upon the continent, under the care of her French maid.

The parting between Rebecca and the little Rawdon did not cause either
party much pain.  She had not, to say truth, seen much of the young
gentleman since his birth. After the amiable fashion of French mothers,
she had placed him out at nurse in a village in the neighbourhood of
Paris, where little Rawdon passed the first months of his life, not
unhappily, with a numerous family of foster-brothers in wooden shoes.
His father would ride over many a time to see him here, and the elder
Rawdon's paternal heart glowed to see him rosy and dirty, shouting
lustily, and happy in the making of mud-pies under the superintendence
of the gardener's wife, his nurse.

Rebecca did not care much to go and see the son and heir.  Once he
spoiled a new dove-coloured pelisse of hers.  He preferred his nurse's
caresses to his mamma's, and when finally he quitted that jolly nurse
and almost parent, he cried loudly for hours.  He was only consoled by
his mother's promise that he should return to his nurse the next day;
indeed the nurse herself, who probably would have been pained at the
parting too, was told that the child would immediately be restored to
her, and for some time awaited quite anxiously his return.

In fact, our friends may be said to have been among the first of that
brood of hardy English adventurers who have subsequently invaded the
Continent and swindled in all the capitals of Europe. The respect in
those happy days of 1817-18 was very great for the wealth and honour of
Britons.  They had not then learned, as I am told, to haggle for
bargains with the pertinacity which now distinguishes them.  The great
cities of Europe had not been as yet open to the enterprise of our
rascals. And whereas there is now hardly a town of France or Italy in
which you shall not see some noble countryman of our own, with that
happy swagger and insolence of demeanour which we carry everywhere,
swindling inn-landlords, passing fictitious cheques upon credulous
bankers, robbing coach-makers of their carriages, goldsmiths of their
trinkets, easy travellers of their money at cards, even public
libraries of their books---thirty years ago you needed but to be a Milor
Anglais, travelling in a private carriage, and credit was at your hand
wherever you chose to seek it, and gentlemen, instead of cheating, were
cheated.  It was not for some weeks after the Crawleys' departure that
the landlord of the hotel which they occupied during their residence at
Paris found out the losses which he had sustained:  not until Madame
Marabou, the milliner, made repeated visits with her little bill for
articles supplied to Madame Crawley; not until Monsieur Didelot from
Boule d'Or in the Palais Royal had asked half a dozen times whether
cette charmante Miladi who had bought watches and bracelets of him was
de retour. It is a fact that even the poor gardener's wife, who had
nursed madame's child, was never paid after the first six months for
that supply of the milk of human kindness with which she had furnished
the lusty and healthy little Rawdon.  No, not even the nurse was
paid---the Crawleys were in too great a hurry to remember their trifling
debt to her.  As for the landlord of the hotel, his curses against the
English nation were violent for the rest of his natural life.  He asked
all travellers whether they knew a certain Colonel Lor Crawley---avec sa
femme une petite dame, tres spirituelle.  ``Ah, Monsieur!'' he would
add---``ils m'ont affreusement vole.'' It was melancholy to hear his
accents as he spoke of that catastrophe.

Rebecca's object in her journey to London was to effect a kind of
compromise with her husband's numerous creditors, and by offering them
a dividend of ninepence or a shilling in the pound, to secure a return
for him into his own country.  It does not become us to trace the steps
which she took in the conduct of this most difficult negotiation; but,
having shown them to their satisfaction that the sum which she was
empowered to offer was all her husband's available capital, and having
convinced them that Colonel Crawley would prefer a perpetual retirement
on the Continent to a residence in this country with his debts
unsettled; having proved to them that there was no possibility of money
accruing to him from other quarters, and no earthly chance of their
getting a larger dividend than that which she was empowered to offer,
she brought the Colonel's creditors unanimously to accept her
proposals, and purchased with fifteen hundred pounds of ready money
more than ten times that amount of debts.

Mrs.\ Crawley employed no lawyer in the transaction. The matter was so
simple, to have or to leave, as she justly observed, that she made the
lawyers of the creditors themselves do the business.  And Mr.\ Lewis
representing Mr.\ Davids, of Red Lion Square, and Mr.\ Moss acting for
Mr.\ Manasseh of Cursitor Street (chief creditors of the Colonel's),
complimented his lady upon the brilliant way in which she did business,
and declared that there was no professional man who could beat her.

Rebecca received their congratulations with perfect modesty; ordered a
bottle of sherry and a bread cake to the little dingy lodgings where
she dwelt, while conducting the business, to treat the enemy's lawyers:
shook hands with them at parting, in excellent good humour, and
returned straightway to the Continent, to rejoin her husband and son
and acquaint the former with the glad news of his entire liberation.
As for the latter, he had been considerably neglected during his
mother's absence by Mademoiselle Genevieve, her French maid; for that
young woman, contracting an attachment for a soldier in the garrison of
Calais, forgot her charge in the society of this militaire, and little
Rawdon very narrowly escaped drowning on Calais sands at this period,
where the absent Genevieve had left and lost him.

And so, Colonel and Mrs.\ Crawley came to London: and it is at their
house in Curzon Street, May Fair, that they really showed the skill
which must be possessed by those who would live on the resources above
named.



\chapter{The Subject Continued}

In the first place, and as a matter of the greatest necessity, we are
bound to describe how a house may be got for nothing a year. These
mansions are to be had either unfurnished, where, if you have credit
with Messrs.  Gillows or Bantings, you can get them splendidly montees
and decorated entirely according to your own fancy; or they are to be
let furnished, a less troublesome and complicated arrangement to most
parties.  It was so that Crawley and his wife preferred to hire their
house.

Before Mr.\ Bowls came to preside over Miss Crawley's house and cellar
in Park Lane, that lady had had for a butler a Mr.\ Raggles, who was
born on the family estate of Queen's Crawley, and indeed was a younger
son of a gardener there.  By good conduct, a handsome person and
calves, and a grave demeanour, Raggles rose from the knife-board to the
footboard of the carriage; from the footboard to the butler's pantry.
When he had been a certain number of years at the head of Miss
Crawley's establishment, where he had had good wages, fat perquisites,
and plenty of opportunities of saving, he announced that he was about
to contract a matrimonial alliance with a late cook of Miss Crawley's,
who had subsisted in an honourable manner by the exercise of a mangle,
and the keeping of a small greengrocer's shop in the neighbourhood.
The truth is, that the ceremony had been clandestinely performed some
years back; although the news of Mr.\ Raggles' marriage was first
brought to Miss Crawley by a little boy and girl of seven and eight
years of age, whose continual presence in the kitchen had attracted the
attention of Miss Briggs.

Mr.\ Raggles then retired and personally undertook the superintendence
of the small shop and the greens.  He added milk and cream, eggs and
country-fed pork to his stores, contenting himself whilst other retired
butlers were vending spirits in public houses, by dealing in the
simplest country produce.  And having a good connection amongst the
butlers in the neighbourhood, and a snug back parlour where he and Mrs.\ %
Raggles received them, his milk, cream, and eggs got to be adopted by
many of the fraternity, and his profits increased every year.  Year
after year he quietly and modestly amassed money, and when at length
that snug and complete bachelor's residence at No.\ 201, Curzon Street,
May Fair, lately the residence of the Honourable Frederick Deuceace,
gone abroad, with its rich and appropriate furniture by the first
makers, was brought to the hammer, who should go in and purchase the
lease and furniture of the house but Charles Raggles? A part of the
money he borrowed, it is true, and at rather a high interest, from a
brother butler, but the chief part he paid down, and it was with no
small pride that Mrs.\ Raggles found herself sleeping in a bed of carved
mahogany, with silk curtains, with a prodigious cheval glass opposite
to her, and a wardrobe which would contain her, and Raggles, and all
the family.

Of course, they did not intend to occupy permanently an apartment so
splendid.  It was in order to let the house again that Raggles
purchased it.  As soon as a tenant was found, he subsided into the
greengrocer's shop once more; but a happy thing it was for him to walk
out of that tenement and into Curzon Street, and there survey his
house---his own house---with geraniums in the window and a carved bronze
knocker.  The footman occasionally lounging at the area railing,
treated him with respect; the cook took her green stuff at his house
and called him Mr.\ Landlord, and there was not one thing the tenants
did, or one dish which they had for dinner, that Raggles might not know
of, if he liked.

He was a good man; good and happy.  The house brought him in so
handsome a yearly income that he was determined to send his children to
good schools, and accordingly, regardless of expense, Charles was sent
to boarding at Dr. Swishtail's, Sugar-cane Lodge, and little Matilda to
Miss Peckover's, Laurentinum House, Clapham.

Raggles loved and adored the Crawley family as the author of all his
prosperity in life.  He had a silhouette of his mistress in his back
shop, and a drawing of the Porter's Lodge at Queen's Crawley, done by
that spinster herself in India ink---and the only addition he made to
the decorations of the Curzon Street House was a print of Queen's
Crawley in Hampshire, the seat of Sir Walpole Crawley, Baronet, who was
represented in a gilded car drawn by six white horses, and passing by a
lake covered with swans, and barges containing ladies in hoops, and
musicians with flags and penwigs.  Indeed Raggles thought there was no
such palace in all the world, and no such august family.

As luck would have it, Raggles' house in Curzon Street was to let when
Rawdon and his wife returned to London. The Colonel knew it and its
owner quite well; the latter's connection with the Crawley family had
been kept up constantly, for Raggles helped Mr.\ Bowls whenever Miss
Crawley received friends.  And the old man not only let his house to
the Colonel but officiated as his butler whenever he had company; Mrs.\ %
Raggles operating in the kitchen below and sending up dinners of which
old Miss Crawley herself might have approved.  This was the way, then,
Crawley got his house for nothing; for though Raggles had to pay taxes
and rates, and the interest of the mortgage to the brother butler; and
the insurance of his life; and the charges for his children at school;
and the value of the meat and drink which his own family---and for a
time that of Colonel Crawley too---consumed; and though the poor wretch
was utterly ruined by the transaction, his children being flung on the
streets, and himself driven into the Fleet Prison:  yet somebody must
pay even for gentlemen who live for nothing a year---and so it was this
unlucky Raggles was made the representative of Colonel Crawley's
defective capital.

I wonder how many families are driven to roguery and to ruin by great
practitioners in Crawley's way?---how many great noblemen rob their petty
tradesmen, condescend to swindle their poor retainers out of wretched
little sums and cheat for a few shillings? When we read that a noble
nobleman has left for the Continent, or that another noble nobleman has
an execution in his house---and that one or other owes six or seven
millions, the defeat seems glorious even, and we respect the victim in
the vastness of his ruin.  But who pities a poor barber who can't get
his money for powdering the footmen's heads; or a poor carpenter who
has ruined himself by fixing up ornaments and pavilions for my lady's
dejeuner; or the poor devil of a tailor whom the steward patronizes,
and who has pledged all he is worth, and more, to get the liveries
ready, which my lord has done him the honour to bespeak? When the great
house tumbles down, these miserable wretches fall under it unnoticed:
as they say in the old legends, before a man goes to the devil himself,
he sends plenty of other souls thither.

Rawdon and his wife generously gave their patronage to all such of Miss
Crawley's tradesmen and purveyors as chose to serve them.  Some were
willing enough, especially the poor ones.  It was wonderful to see the
pertinacity with which the washerwoman from Tooting brought the cart
every Saturday, and her bills week after week. Mr.\ Raggles himself had
to supply the greengroceries.  The bill for servants' porter at the
Fortune of War public house is a curiosity in the chronicles of beer.
Every servant also was owed the greater part of his wages, and thus
kept up perforce an interest in the house. Nobody in fact was paid.
Not the blacksmith who opened the lock; nor the glazier who mended the
pane; nor the jobber who let the carriage; nor the groom who drove it;
nor the butcher who provided the leg of mutton; nor the coals which
roasted it; nor the cook who basted it; nor the servants who ate it:
and this I am given to understand is not unfrequently the way in which
people live elegantly on nothing a year.

In a little town such things cannot be done without remark.  We know
there the quantity of milk our neighbour takes and espy the joint or
the fowls which are going in for his dinner.  So, probably, 200 and 202
in Curzon Street might know what was going on in the house between
them, the servants communicating through the area-railings; but Crawley
and his wife and his friends did not know 200 and 202. When you came to
201 there was a hearty welcome, a kind smile, a good dinner, and a
jolly shake of the hand from the host and hostess there, just for all
the world as if they had been undisputed masters of three or four
thousand a year---and so they were, not in money, but in produce and
labour---if they did not pay for the mutton, they had it:  if they did
not give bullion in exchange for their wine, how should we know? Never
was better claret at any man's table than at honest Rawdon's; dinners
more gay and neatly served.   His drawing-rooms were the prettiest,
little, modest salons conceivable: they were decorated with the
greatest taste, and a thousand knick-knacks from Paris, by Rebecca:
and when she sat at her piano trilling songs with a lightsome heart,
the stranger voted himself in a little paradise of domestic comfort and
agreed that, if the husband was rather stupid, the wife was charming,
and the dinners the pleasantest in the world.

Rebecca's wit, cleverness, and flippancy made her speedily the vogue in
London among a certain class.  You saw demure chariots at her door, out
of which stepped very great people.  You beheld her carriage in the
park, surrounded by dandies of note.  The little box in the third tier
of the opera was crowded with heads constantly changing; but it must be
confessed that the ladies held aloof from her, and that their doors
were shut to our little adventurer.

With regard to the world of female fashion and its customs, the present
writer of course can only speak at second hand.  A man can no more
penetrate or understand those mysteries than he can know what the
ladies talk about when they go upstairs after dinner.  It is only by
inquiry and perseverance that one sometimes gets hints of those
secrets; and by a similar diligence every person who treads the Pall
Mall pavement and frequents the clubs of this metropolis knows, either
through his own experience or through some acquaintance with whom he
plays at billiards or shares the joint, something about the genteel
world of London, and how, as there are men (such as Rawdon Crawley,
whose position we mentioned before) who cut a good figure to the eyes
of the ignorant world and to the apprentices in the park, who behold
them consorting with the most notorious dandies there, so there are
ladies, who may be called men's women, being welcomed entirely by all
the gentlemen and cut or slighted by all their wives.  Mrs.\ Firebrace
is of this sort; the lady with the beautiful fair ringlets whom you see
every day in Hyde Park, surrounded by the greatest and most famous
dandies of this empire.  Mrs.\ Rockwood is another, whose parties are
announced laboriously in the fashionable newspapers and with whom you
see that all sorts of ambassadors and great noblemen dine; and many
more might be mentioned had they to do with the history at present in
hand.  But while simple folks who are out of the world, or country
people with a taste for the genteel, behold these ladies in their
seeming glory in public places, or envy them from afar off, persons who
are better instructed could inform them that these envied ladies have
no more chance of establishing themselves in ``society,'' than the
benighted squire's wife in Somersetshire who reads of their doings in
the Morning Post. Men living about London are aware of these awful
truths. You hear how pitilessly many ladies of seeming rank and wealth
are excluded from this ``society.'' The frantic efforts which they make
to enter this circle, the meannesses to which they submit, the insults
which they undergo, are matters of wonder to those who take human or
womankind for a study; and the pursuit of fashion under difficulties
would be a fine theme for any very great person who had the wit, the
leisure, and the knowledge of the English language necessary for the
compiling of such a history.

Now the few female acquaintances whom Mrs.\ Crawley had known abroad not
only declined to visit her when she came to this side of the Channel,
but cut her severely when they met in public places.  It was curious to
see how the great ladies forgot her, and no doubt not altogether a
pleasant study to Rebecca.  When Lady Bareacres met her in the
waiting-room at the opera, she gathered her daughters about her as if
they would be contaminated by a touch of Becky, and retreating a step
or two, placed herself in front of them, and stared at her little
enemy. To stare Becky out of countenance required a severer glance than
even the frigid old Bareacres could shoot out of her dismal eyes.  When
Lady de la Mole, who had ridden a score of times by Becky's side at
Brussels, met Mrs.\ Crawley's open carriage in Hyde Park, her Ladyship
was quite blind, and could not in the least recognize her former
friend.  Even Mrs.\ Blenkinsop, the banker's wife, cut her at church.
Becky went regularly to church now; it was edifying to see her enter
there with Rawdon by her side, carrying a couple of large gilt
prayer-books, and afterwards going through the ceremony with the
gravest resignation.

Rawdon at first felt very acutely the slights which were passed upon
his wife, and was inclined to be gloomy and savage.  He talked of
calling out the husbands or brothers of every one of the insolent women
who did not pay a proper respect to his wife; and it was only by the
strongest commands and entreaties on her part that he was brought into
keeping a decent behaviour.  ``You can't shoot me into society,'' she
said good-naturedly.  ``Remember, my dear, that I was but a governess,
and you, you poor silly old man, have the worst reputation for debt,
and dice, and all sorts of wickedness.  We shall get quite as many
friends as we want by and by, and in the meanwhile you must be a good
boy and obey your schoolmistress in everything she tells you to do.
When we heard that your aunt had left almost everything to Pitt and his
wife, do you remember what a rage you were in? You would have told all
Paris, if I had not made you keep your temper, and where would you have
been now?---in prison at Ste.  Pelagie for debt, and not established in
London in a handsome house, with every comfort about you---you were in
such a fury you were ready to murder your brother, you wicked Cain you,
and what good would have come of remaining angry? All the rage in the
world won't get us your aunt's money; and it is much better that we
should be friends with your brother's family than enemies, as those
foolish Butes are.  When your father dies, Queen's Crawley will be a
pleasant house for you and me to pass the winter in.  If we are ruined,
you can carve and take charge of the stable, and I can be a governess
to Lady Jane's children.  Ruined! fiddlede-dee!  I will get you a good
place before that; or Pitt and his little boy will die, and we will be
Sir Rawdon and my lady.  While there is life, there is hope, my dear,
and I intend to make a man of you yet.  Who sold your horses for you?
Who paid your debts for you?'' Rawdon was obliged to confess that he
owed all these benefits to his wife, and to trust himself to her
guidance for the future.

Indeed, when Miss Crawley quitted the world, and that money for which
all her relatives had been fighting so eagerly was finally left to
Pitt, Bute Crawley, who found that only five thousand pounds had been
left to him instead of the twenty upon which he calculated, was in such
a fury at his disappointment that he vented it in savage abuse upon his
nephew; and the quarrel always rankling between them ended in an utter
breach of intercourse.  Rawdon Crawley's conduct, on the other hand,
who got but a hundred pounds, was such as to astonish his brother and
delight his sister-in-law, who was disposed to look kindly upon all the
members of her husband's family.  He wrote to his brother a very frank,
manly, good-humoured letter from Paris.  He was aware, he said, that by
his own marriage he had forfeited his aunt's favour; and though he did
not disguise his disappointment that she should have been so entirely
relentless towards him, he was glad that the money was still kept in
their branch of the family, and heartily congratulated his brother on
his good fortune.  He sent his affectionate remembrances to his sister,
and hoped to have her good-will for Mrs.\ Rawdon; and the letter
concluded with a postscript to Pitt in the latter lady's own
handwriting.  She, too, begged to join in her husband's
congratulations.  She should ever remember Mr.\ Crawley's kindness to
her in early days when she was a friendless orphan, the instructress of
his little sisters, in whose welfare she still took the tenderest
interest.  She wished him every happiness in his married life, and,
asking his permission to offer her remembrances to Lady Jane (of whose
goodness all the world informed her), she hoped that one day she might
be allowed to present her little boy to his uncle and aunt, and begged
to bespeak for him their good-will and protection.

Pitt Crawley received this communication very graciously---more
graciously than Miss Crawley had received some of Rebecca's previous
compositions in Rawdon's handwriting; and as for Lady Jane, she was so
charmed with the letter that she expected her husband would instantly
divide his aunt's legacy into two equal portions and send off one-half
to his brother at Paris.

To her Ladyship's surprise, however, Pitt declined to accommodate his
brother with a cheque for thirty thousand pounds.  But he made Rawdon a
handsome offer of his hand whenever the latter should come to England
and choose to take it; and, thanking Mrs.\ Crawley for her good opinion
of himself and Lady Jane, he graciously pronounced his willingness to
take any opportunity to serve her little boy.

Thus an almost reconciliation was brought about between the brothers.
When Rebecca came to town Pitt and his wife were not in London.  Many a
time she drove by the old door in Park Lane to see whether they had
taken possession of Miss Crawley's house there. But the new family did
not make its appearance; it was only through Raggles that she heard of
their movements---how Miss Crawley's domestics had been dismissed with
decent gratuities, and how Mr.\ Pitt had only once made his appearance
in London, when he stopped for a few days at the house, did business
with his lawyers there, and sold off all Miss Crawley's French novels
to a bookseller out of Bond Street.  Becky had reasons of her own which
caused her to long for the arrival of her new relation. ``When Lady Jane
comes,'' thought she, ``she shall be my sponsor in London society; and as
for the women! bah! the women will ask me when they find the men want
to see me.''

An article as necessary to a lady in this position as her brougham or
her bouquet is her companion.  I have always admired the way in which
the tender creatures, who cannot exist without sympathy, hire an
exceedingly plain friend of their own sex from whom they are almost
inseparable.  The sight of that inevitable woman in her faded gown
seated behind her dear friend in the opera-box, or occupying the back
seat of the barouche, is always a wholesome and moral one to me, as
jolly a reminder as that of the Death's-head which figured in the
repasts of Egyptian bon-vivants, a strange sardonic memorial of Vanity
Fair.  What? even battered, brazen, beautiful, conscienceless,
heartless, Mrs.\ Firebrace, whose father died of her shame:  even
lovely, daring Mrs.\ Mantrap, who will ride at any fence which any man
in England will take, and who drives her greys in the park, while her
mother keeps a huckster's stall in Bath still---even those who are so
bold, one might fancy they could face anything, dare not face the world
without a female friend.  They must have somebody to cling to, the
affectionate creatures!  And you will hardly see them in any public
place without a shabby companion in a dyed silk, sitting somewhere in
the shade close behind them.

``Rawdon,'' said Becky, very late one night, as a party of gentlemen were
seated round her crackling drawing-room fire (for the men came to her
house to finish the night; and she had ice and coffee for them, the
best in London):  ``I must have a sheep-dog.''

``A what?'' said Rawdon, looking up from an ecarte table.

``A sheep-dog!'' said young Lord Southdown.  ``My dear Mrs.\ Crawley, what
a fancy!  Why not have a Danish dog? I know of one as big as a
camel-leopard, by Jove. It would almost pull your brougham.  Or a
Persian greyhound, eh? (I propose, if you please); or a little pug that
would go into one of Lord Steyne's snuff-boxes? There's a man at
Bayswater got one with such a nose that you might---I mark the king and
play---that you might hang your hat on it.''

``I mark the trick,'' Rawdon gravely said.  He attended to his game
commonly and didn't much meddle with the conversation, except when it
was about horses and betting.

``What CAN you want with a shepherd's dog?'' the lively little Southdown
continued.

``I mean a MORAL shepherd's dog,'' said Becky, laughing and looking up at
Lord Steyne.

``What the devil's that?'' said his Lordship.

``A dog to keep the wolves off me,'' Rebecca continued. ``A companion.''

``Dear little innocent lamb, you want one,'' said the marquis; and his
jaw thrust out, and he began to grin hideously, his little eyes leering
towards Rebecca.

The great Lord of Steyne was standing by the fire sipping coffee. The
fire crackled and blazed pleasantly. There was a score of candles
sparkling round the mantel piece, in all sorts of quaint sconces, of
gilt and bronze and porcelain.  They lighted up Rebecca's figure to
admiration, as she sat on a sofa covered with a pattern of gaudy
flowers.  She was in a pink dress that looked as fresh as a rose; her
dazzling white arms and shoulders were half-covered with a thin hazy
scarf through which they sparkled; her hair hung in curls round her
neck; one of her little feet peeped out from the fresh crisp folds of
the silk:  the prettiest little foot in the prettiest little sandal in
the finest silk stocking in the world.

The candles lighted up Lord Steyne's shining bald head, which was
fringed with red hair.  He had thick bushy eyebrows, with little
twinkling bloodshot eyes, surrounded by a thousand wrinkles.  His jaw
was underhung, and when he laughed, two white buck-teeth protruded
themselves and glistened savagely in the midst of the grin. He had been
dining with royal personages, and wore his garter and ribbon.  A short
man was his Lordship, broad-chested and bow-legged, but proud of the
fineness of his foot and ankle, and always caressing his garter-knee.

``And so the shepherd is not enough,'' said he, ``to defend his lambkin?''

``The shepherd is too fond of playing at cards and going to his clubs,''
answered Becky, laughing.

``'Gad, what a debauched Corydon!'' said my lord---``what a mouth for a
pipe!''

``I take your three to two,'' here said Rawdon, at the card-table.

``Hark at Meliboeus,'' snarled the noble marquis; ``he's pastorally
occupied too:  he's shearing a Southdown. What an innocent mutton, hey?
Damme, what a snowy fleece!''

Rebecca's eyes shot out gleams of scornful humour. ``My lord,'' she said,
``you are a knight of the Order.'' He had the collar round his neck,
indeed---a gift of the restored princes of Spain.

Lord Steyne in early life had been notorious for his daring and his
success at play.  He had sat up two days and two nights with Mr.\ Fox at
hazard.  He had won money of the most august personages of the realm:
he had won his marquisate, it was said, at the gaming-table; but he did
not like an allusion to those bygone fredaines.  Rebecca saw the scowl
gathering over his heavy brow.

She rose up from her sofa and went and took his coffee cup out of his
hand with a little curtsey.  ``Yes,'' she said, ``I must get a watchdog.
But he won't bark at YOU.'' And, going into the other drawing-room, she
sat down to the piano and began to sing little French songs in such a
charming, thrilling voice that the mollified nobleman speedily followed
her into that chamber, and might be seen nodding his head and bowing
time over her.

Rawdon and his friend meanwhile played ecarte until they had enough.
The Colonel won; but, say that he won ever so much and often, nights
like these, which occurred many times in the week---his wife having all
the talk and all the admiration, and he sitting silent without the
circle, not comprehending a word of the jokes, the allusions, the
mystical language within---must have been rather wearisome to the
ex-dragoon.

``How is Mrs.\ Crawley's husband?'' Lord Steyne used to say to him by way
of a good day when they met; and indeed that was now his avocation in
life.  He was Colonel Crawley no more.  He was Mrs.\ Crawley's husband.

About the little Rawdon, if nothing has been said all this while, it is
because he is hidden upstairs in a garret somewhere, or has crawled
below into the kitchen for companionship.  His mother scarcely ever
took notice of him.  He passed the days with his French bonne as long
as that domestic remained in Mr.\ Crawley's family, and when the
Frenchwoman went away, the little fellow, howling in the loneliness of
the night, had compassion taken on him by a housemaid, who took him out
of his solitary nursery into her bed in the garret hard by and
comforted him.

Rebecca, my Lord Steyne, and one or two more were in the drawing-room
taking tea after the opera, when this shouting was heard overhead.
``It's my cherub crying for his nurse,'' she said.  She did not offer to
move to go and see the child.  ``Don't agitate your feelings by going to
look for him,'' said Lord Steyne sardonically. ``Bah!'' replied the other,
with a sort of blush, ``he'll cry himself to sleep''; and they fell to
talking about the opera.

Rawdon had stolen off though, to look after his son and heir; and came
back to the company when he found that honest Dolly was consoling the
child.  The Colonel's dressing-room was in those upper regions.  He
used to see the boy there in private.  They had interviews together
every morning when he shaved; Rawdon minor sitting on a box by his
father's side and watching the operation with never-ceasing pleasure.
He and the sire were great friends. The father would bring him
sweetmeats from the dessert and hide them in a certain old epaulet box,
where the child went to seek them, and laughed with joy on discovering
the treasure; laughed, but not too loud:  for mamma was below asleep
and must not be disturbed.  She did not go to rest till very late and
seldom rose till after noon.

Rawdon bought the boy plenty of picture-books and crammed his nursery
with toys.  Its walls were covered with pictures pasted up by the
father's own hand and purchased by him for ready money.  When he was
off duty with Mrs.\ Rawdon in the park, he would sit up here, passing
hours with the boy; who rode on his chest, who pulled his great
mustachios as if they were driving-reins, and spent days with him in
indefatigable gambols.  The room was a low room, and once, when the
child was not five years old, his father, who was tossing him wildly up
in his arms, hit the poor little chap's skull so violently against the
ceiling that he almost dropped the child, so terrified was he at the
disaster.

Rawdon minor had made up his face for a tremendous howl---the severity
of the blow indeed authorized that indulgence; but just as he was going
to begin, the father interposed.

``For God's sake, Rawdy, don't wake Mamma,'' he cried.  And the child,
looking in a very hard and piteous way at his father, bit his lips,
clenched his hands, and didn't cry a bit.  Rawdon told that story at
the clubs, at the mess, to everybody in town.  ``By Gad, sir,'' he
explained to the public in general, ``what a good plucked one that boy
of mine is---what a trump he is!  I half-sent his head through the
ceiling, by Gad, and he wouldn't cry for fear of disturbing his mother.''

Sometimes---once or twice in a week---that lady visited the upper regions
in which the child lived.  She came like a vivified figure out of the
Magasin des Modes---blandly smiling in the most beautiful new clothes
and little gloves and boots.  Wonderful scarfs, laces, and jewels
glittered about her.  She had always a new bonnet on, and flowers
bloomed perpetually in it, or else magnificent curling ostrich
feathers, soft and snowy as camellias.  She nodded twice or thrice
patronizingly to the little boy, who looked up from his dinner or from
the pictures of soldiers he was painting.  When she left the room, an
odour of rose, or some other magical fragrance, lingered about the
nursery. She was an unearthly being in his eyes, superior to his
father---to all the world:  to be worshipped and admired at a distance.
To drive with that lady in the carriage was an awful rite:  he sat up
in the back seat and did not dare to speak:  he gazed with all his eyes
at the beautifully dressed Princess opposite to him.  Gentlemen on
splendid prancing horses came up and smiled and talked with her. How
her eyes beamed upon all of them!  Her hand used to quiver and wave
gracefully as they passed.  When he went out with her he had his new
red dress on.  His old brown holland was good enough when he stayed at
home. Sometimes, when she was away, and Dolly his maid was making his
bed, he came into his mother's room.  It was as the abode of a fairy to
him---a mystic chamber of splendour and delights.  There in the wardrobe
hung those wonderful robes---pink and blue and many-tinted.  There was
the jewel-case, silver-clasped, and the wondrous bronze hand on the
dressing-table, glistening all over with a hundred rings.  There was
the cheval-glass, that miracle of art, in which he could just see his
own wondering head and the reflection of Dolly (queerly distorted, and
as if up in the ceiling), plumping and patting the pillows of the bed.
Oh, thou poor lonely little benighted boy! Mother is the name for God
in the lips and hearts of little children; and here was one who was
worshipping a stone!

Now Rawdon Crawley, rascal as the Colonel was, had certain manly
tendencies of affection in his heart and could love a child and a woman
still.  For Rawdon minor he had a great secret tenderness then, which
did not escape Rebecca, though she did not talk about it to her
husband.  It did not annoy her:  she was too good-natured. It only
increased her scorn for him.  He felt somehow ashamed of this paternal
softness and hid it from his wife---only indulging in it when alone with
the boy.

He used to take him out of mornings when they would go to the stables
together and to the park.  Little Lord Southdown, the best-natured of
men, who would make you a present of the hat from his head, and whose
main occupation in life was to buy knick-knacks that he might give them
away afterwards, bought the little chap a pony not much bigger than a
large rat, the donor said, and on this little black Shetland pygmy
young Rawdon's great father was pleased to mount the boy, and to walk
by his side in the park.  It pleased him to see his old quarters, and
his old fellow-guardsmen at Knightsbridge: he had begun to think of his
bachelorhood with something like regret.  The old troopers were glad to
recognize their ancient officer and dandle the little colonel. Colonel
Crawley found dining at mess and with his brother-officers very
pleasant. ``Hang it, I ain't clever enough for her---I know it.  She
won't miss me,'' he used to say:  and he was right, his wife did not
miss him.

Rebecca was fond of her husband.  She was always perfectly good-humoured
and kind to him.  She did not even show her scorn much for
him; perhaps she liked him the better for being a fool.  He was her
upper servant and maitre d'hotel.  He went on her errands; obeyed her
orders without question; drove in the carriage in the ring with her
without repining; took her to the opera-box, solaced himself at his
club during the performance, and came punctually back to fetch her when
due.  He would have liked her to be a little fonder of the boy, but
even to that he reconciled himself.  ``Hang it, you know she's so
clever,'' he said, ``and I'm not literary and that, you know.'' For, as we
have said before, it requires no great wisdom to be able to win at
cards and billiards, and Rawdon made no pretensions to any other sort
of skill.

When the companion came, his domestic duties became very light.  His
wife encouraged him to dine abroad:  she would let him off duty at the
opera.  ``Don't stay and stupefy yourself at home to-night, my dear,''
she would say.  ``Some men are coming who will only bore you. I would
not ask them, but you know it's for your good, and now I have a
sheep-dog, I need not be afraid to be alone.''

``A sheep-dog---a companion!  Becky Sharp with a companion!  Isn't it
good fun?'' thought Mrs.\ Crawley to herself.  The notion tickled hugely
her sense of humour.

One Sunday morning, as Rawdon Crawley, his little son, and the pony
were taking their accustomed walk in the park, they passed by an old
acquaintance of the Colonel's, Corporal Clink, of the regiment, who was
in conversation with a friend, an old gentleman, who held a boy in his
arms about the age of little Rawdon.  This other youngster had seized
hold of the Waterloo medal which the Corporal wore, and was examining
it with delight.

``Good morning, your Honour,'' said Clink, in reply to the ``How do,
Clink?'' of the Colonel.  ``This ere young gentleman is about the little
Colonel's age, sir,'' continued the corporal.

``His father was a Waterloo man, too,'' said the old gentleman, who
carried the boy.  ``Wasn't he, Georgy?''

``Yes,'' said Georgy.  He and the little chap on the pony were looking at
each other with all their might---solemnly scanning each other as
children do.

``In a line regiment,'' Clink said with a patronizing air.

``He was a Captain in the ---th regiment,'' said the old gentleman rather
pompously.  ``Captain George Osborne, sir---perhaps you knew him.  He
died the death of a hero, sir, fighting against the Corsican tyrant.''
Colonel Crawley blushed quite red.  ``I knew him very well, sir,'' he
said, ``and his wife, his dear little wife, sir---how is she?''

``She is my daughter, sir,'' said the old gentleman, putting down the boy
and taking out a card with great solemnity, which he handed to the
Colonel.  On it written---

``Mr.\ Sedley, Sole Agent for the Black Diamond and Anti-Cinder Coal
Association, Bunker's Wharf, Thames Street, and Anna-Maria Cottages,
Fulham Road West.''

Little Georgy went up and looked at the Shetland pony.

``Should you like to have a ride?'' said Rawdon minor from the saddle.

``Yes,'' said Georgy.  The Colonel, who had been looking at him with some
interest, took up the child and put him on the pony behind Rawdon minor.

``Take hold of him, Georgy,'' he said---``take my little boy round the
waist---his name is Rawdon.'' And both the children began to laugh.

``You won't see a prettier pair I think, THIS summer's day, sir,'' said
the good-natured Corporal; and the Colonel, the Corporal, and old Mr.\ %
Sedley with his umbrella, walked by the side of the children.



\chapter{A Family in a Very Small Way}

We must suppose little George Osborne has ridden from Knightsbridge
towards Fulham, and will stop and make inquiries at that village
regarding some friends whom we have left there.  How is Mrs.\ Amelia
after the storm of Waterloo? Is she living and thriving? What has come
of Major Dobbin, whose cab was always hankering about her premises? And
is there any news of the Collector of Boggley Wollah? The facts
concerning the latter are briefly these:

Our worthy fat friend Joseph Sedley returned to India not long after
his escape from Brussels.  Either his furlough was up, or he dreaded to
meet any witnesses of his Waterloo flight.  However it might be, he
went back to his duties in Bengal very soon after Napoleon had taken up
his residence at St.\ Helena, where Jos saw the ex-Emperor. To hear Mr.\ %
Sedley talk on board ship you would have supposed that it was not the
first time he and the Corsican had met, and that the civilian had
bearded the French General at Mount St.\ John.  He had a thousand
anecdotes about the famous battles; he knew the position of every
regiment and the loss which each had incurred.  He did not deny that he
had been concerned in those victories---that he had been with the army
and carried despatches for the Duke of Wellington. And he described
what the Duke did and said on every conceivable moment of the day of
Waterloo, with such an accurate knowledge of his Grace's sentiments and
proceedings that it was clear he must have been by the conqueror's side
throughout the day; though, as a non-combatant, his name was not
mentioned in the public documents relative to the battle.  Perhaps he
actually worked himself up to believe that he had been engaged with the
army; certain it is that he made a prodigious sensation for some time
at Calcutta, and was called Waterloo Sedley during the whole of his
subsequent stay in Bengal.

The bills which Jos had given for the purchase of those unlucky horses
were paid without question by him and his agents.  He never was heard
to allude to the bargain, and nobody knows for a certainty what became
of the horses, or how he got rid of them, or of Isidor, his Belgian
servant, who sold a grey horse, very like the one which Jos rode, at
Valenciennes sometime during the autumn of 1815.

Jos's London agents had orders to pay one hundred and twenty pounds
yearly to his parents at Fulham.  It was the chief support of the old
couple; for Mr.\ Sedley's speculations in life subsequent to his
bankruptcy did not by any means retrieve the broken old gentleman's
fortune.  He tried to be a wine-merchant, a coal-merchant, a commission
lottery agent, \&c., \&c.  He sent round prospectuses to his friends
whenever he took a new trade, and ordered a new brass plate for the
door, and talked pompously about making his fortune still.  But Fortune
never came back to the feeble and stricken old man.  One by one his
friends dropped off, and were weary of buying dear coals and bad wine
from him; and there was only his wife in all the world who fancied,
when he tottered off to the City of a morning, that he was still doing
any business there.  At evening he crawled slowly back; and he used to
go of nights to a little club at a tavern, where he disposed of the
finances of the nation. It was wonderful to hear him talk about
millions, and agios, and discounts, and what Rothschild was doing, and
Baring Brothers.  He talked of such vast sums that the gentlemen of the
club (the apothecary, the undertaker, the great carpenter and builder,
the parish clerk, who was allowed to come stealthily, and Mr.\ Clapp,
our old acquaintance,) respected the old gentleman.  ``I was better off
once, sir,'' he did not fail to tell everybody who ``used the room.'' ``My
son, sir, is at this minute chief magistrate of Ramgunge in the
Presidency of Bengal, and touching his four thousand rupees per mensem.
My daughter might be a Colonel's lady if she liked.  I might draw upon
my son, the first magistrate, sir, for two thousand pounds to-morrow,
and Alexander would cash my bill, down sir, down on the counter, sir.
But the Sedleys were always a proud family.'' You and I, my dear reader,
may drop into this condition one day: for have not many of our friends
attained it? Our luck may fail: our powers forsake us:  our place on
the boards be taken by better and younger mimes---the chance of life
roll away and leave us shattered and stranded.  Then men will walk
across the road when they meet you---or, worse still, hold you out a
couple of fingers and patronize you in a pitying way---then you will
know, as soon as your back is turned, that your friend begins with a
``Poor devil, what imprudences he has committed, what chances that chap
has thrown away!'' Well, well---a carriage and three thousand a year is
not the summit of the reward nor the end of God's judgment of men.  If
quacks prosper as often as they go to the wall---if zanies succeed and
knaves arrive at fortune, and, vice versa, sharing ill luck and
prosperity for all the world like the ablest and most honest amongst
us---I say, brother, the gifts and pleasures of Vanity Fair cannot be
held of any great account, and that it is probable . . . but we are
wandering out of the domain of the story.

Had Mrs.\ Sedley been a woman of energy, she would have exerted it after
her husband's ruin and, occupying a large house, would have taken in
boarders.  The broken Sedley would have acted well as the
boarding-house landlady's husband; the Munoz of private life; the
titular lord and master:  the carver, house-steward, and humble husband
of the occupier of the dingy throne.  I have seen men of good brains
and breeding, and of good hopes and vigour once, who feasted squires
and kept hunters in their youth, meekly cutting up legs of mutton for
rancorous old harridans and pretending to preside over their dreary
tables---but Mrs.\ Sedley, we say, had not spirit enough to bustle about
for ``a few select inmates to join a cheerful musical family,'' such as
one reads of in the Times.  She was content to lie on the shore where
fortune had stranded her---and you could see that the career of this old
couple was over.

I don't think they were unhappy.  Perhaps they were a little prouder in
their downfall than in their prosperity. Mrs.\ Sedley was always a great
person for her landlady, Mrs.\ Clapp, when she descended and passed many
hours with her in the basement or ornamented kitchen. The Irish maid
Betty Flanagan's bonnets and ribbons, her sauciness, her idleness, her
reckless prodigality of kitchen candles, her consumption of tea and
sugar, and so forth occupied and amused the old lady almost as much as
the doings of her former household, when she had Sambo and the
coachman, and a groom, and a footboy, and a housekeeper with a regiment
of female domestics---her former household, about which the good lady
talked a hundred times a day. And besides Betty Flanagan, Mrs.\ Sedley
had all the maids-of-all-work in the street to superintend. She knew
how each tenant of the cottages paid or owed his little rent.  She
stepped aside when Mrs.\ Rougemont the actress passed with her dubious
family.  She flung up her head when Mrs.\ Pestler, the apothecary's
lady, drove by in her husband's professional one-horse chaise.  She had
colloquies with the greengrocer about the pennorth of turnips which Mr.\ %
Sedley loved; she kept an eye upon the milkman and the baker's boy; and
made visitations to the butcher, who sold hundreds of oxen very likely
with less ado than was made about Mrs.\ Sedley's loin of mutton:  and
she counted the potatoes under the joint on Sundays, on which days,
dressed in her best, she went to church twice and read Blair's Sermons
in the evening.

On that day, for ``business'' prevented him on weekdays from taking such
a pleasure, it was old Sedley's delight to take out his little grandson
Georgy to the neighbouring parks or Kensington Gardens, to see the
soldiers or to feed the ducks.  Georgy loved the redcoats, and his
grandpapa told him how his father had been a famous soldier, and
introduced him to many sergeants and others with Waterloo medals on
their breasts, to whom the old grandfather pompously presented the
child as the son of Captain Osborne of the ---th, who died gloriously on
the glorious eighteenth.  He has been known to treat some of these
non-commissioned gentlemen to a glass of porter, and, indeed, in their
first Sunday walks was disposed to spoil little Georgy, sadly gorging
the boy with apples and parliament, to the detriment of his
health---until Amelia declared that George should never go out with his
grandpapa unless the latter promised solemnly, and on his honour, not
to give the child any cakes, lollipops, or stall produce whatever.

Between Mrs.\ Sedley and her daughter there was a sort of coolness about
this boy, and a secret jealousy---for one evening in George's very early
days, Amelia, who had been seated at work in their little parlour
scarcely remarking that the old lady had quitted the room, ran upstairs
instinctively to the nursery at the cries of the child, who had been
asleep until that moment---and there found Mrs.\ Sedley in the act of
surreptitiously administering Daffy's Elixir to the infant.  Amelia,
the gentlest and sweetest of everyday mortals, when she found this
meddling with her maternal authority, thrilled and trembled all over
with anger.  Her cheeks, ordinarily pale, now flushed up, until they
were as red as they used to be when she was a child of twelve years
old.  She seized the baby out of her mother's arms and then grasped at
the bottle, leaving the old lady gaping at her, furious, and holding
the guilty tea-spoon.

Amelia flung the bottle crashing into the fire-place. ``I will NOT have
baby poisoned, Mamma,'' cried Emmy, rocking the infant about violently
with both her arms round him and turning with flashing eyes at her
mother.

``Poisoned, Amelia!'' said the old lady; ``this language to me?''

``He shall not have any medicine but that which Mr.\ Pestler sends for
him.  He told me that Daffy's Elixir was poison.''

``Very good:  you think I'm a murderess then,'' replied Mrs.\ Sedley.
``This is the language you use to your mother. I have met with
misfortunes:  I have sunk low in life:  I have kept my carriage, and
now walk on foot:  but I did not know I was a murderess before, and
thank you for the NEWS.''

``Mamma,'' said the poor girl, who was always ready for tears---``you
shouldn't be hard upon me.  I---I didn't mean---I mean, I did not wish to
say you would do any wrong to this dear child, only---''

``Oh, no, my love,---only that I was a murderess; in which case I had
better go to the Old Bailey.  Though I didn't poison YOU, when you were
a child, but gave you the best of education and the most expensive
masters money could procure.  Yes; I've nursed five children and buried
three; and the one I loved the best of all, and tended through croup,
and teething, and measles, and hooping-cough, and brought up with
foreign masters, regardless of expense, and with accomplishments at
Minerva House---which I never had when I was a girl---when I was too glad
to honour my father and mother, that I might live long in the land, and
to be useful, and not to mope all day in my room and act the fine
lady---says I'm a murderess.  Ah, Mrs.\ Osborne! may YOU never nourish a
viper in your bosom, that's MY prayer.''

``Mamma, Mamma!'' cried the bewildered girl; and the child in her arms
set up a frantic chorus of shouts. ``A murderess, indeed!  Go down on
your knees and pray to God to cleanse your wicked ungrateful heart,
Amelia, and may He forgive you as I do.'' And Mrs.\ Sedley tossed out of
the room, hissing out the word poison once more, and so ending her
charitable benediction.

Till the termination of her natural life, this breach between Mrs.\ %
Sedley and her daughter was never thoroughly mended.  The quarrel gave
the elder lady numberless advantages which she did not fail to turn to
account with female ingenuity and perseverance.  For instance, she
scarcely spoke to Amelia for many weeks afterwards. She warned the
domestics not to touch the child, as Mrs.\ Osborne might be offended.
She asked her daughter to see and satisfy herself that there was no
poison prepared in the little daily messes that were concocted for
Georgy. When neighbours asked after the boy's health, she referred them
pointedly to Mrs.\ Osborne.  SHE never ventured to ask whether the baby
was well or not.  SHE would not touch the child although he was her
grandson, and own precious darling, for she was not USED to children,
and might kill it.  And whenever Mr.\ Pestler came upon his healing
inquisition, she received the doctor with such a sarcastic and scornful
demeanour, as made the surgeon declare that not Lady Thistlewood
herself, whom he had the honour of attending professionally, could give
herself greater airs than old Mrs.\ Sedley, from whom he never took a
fee.  And very likely Emmy was jealous too, upon her own part, as what
mother is not, of those who would manage her children for her, or
become candidates for the first place in their affections.  It is
certain that when anybody nursed the child, she was uneasy, and that
she would no more allow Mrs.\ Clapp or the domestic to dress or tend him
than she would have let them wash her husband's miniature which hung up
over her little bed---the same little bed from which the poor girl had
gone to his; and to which she retired now for many long, silent,
tearful, but happy years.

In this room was all Amelia's heart and treasure.  Here it was that she
tended her boy and watched him through the many ills of childhood, with
a constant passion of love.  The elder George returned in him somehow,
only improved, and as if come back from heaven.  In a hundred little
tones, looks, and movements, the child was so like his father that the
widow's heart thrilled as she held him to it; and he would often ask
the cause of her tears.  It was because of his likeness to his father,
she did not scruple to tell him.  She talked constantly to him about
this dead father, and spoke of her love for George to the innocent and
wondering child; much more than she ever had done to George himself, or
to any confidante of her youth.  To her parents she never talked about
this matter, shrinking from baring her heart to them.  Little George
very likely could understand no better than they, but into his ears she
poured her sentimental secrets unreservedly, and into his only.  The
very joy of this woman was a sort of grief, or so tender, at least,
that its expression was tears.  Her sensibilities were so weak and
tremulous that perhaps they ought not to be talked about in a book. I
was told by Dr. Pestler (now a most flourishing lady's physician, with
a sumptuous dark green carriage, a prospect of speedy knighthood, and a
house in Manchester Square) that her grief at weaning the child was a
sight that would have unmanned a Herod.  He was very soft-hearted many
years ago, and his wife was mortally jealous of Mrs.\ Amelia, then and
long afterwards.

Perhaps the doctor's lady had good reason for her jealousy:  most women
shared it, of those who formed the small circle of Amelia's
acquaintance, and were quite angry at the enthusiasm with which the
other sex regarded her.  For almost all men who came near her loved
her; though no doubt they would be at a loss to tell you why.  She was
not brilliant, nor witty, nor wise over much, nor extraordinarily
handsome.  But wherever she went she touched and charmed every one of
the male sex, as invariably as she awakened the scorn and incredulity
of her own sisterhood.  I think it was her weakness which was her
principal charm---a kind of sweet submission and softness, which seemed
to appeal to each man she met for his sympathy and protection.  We have
seen how in the regiment, though she spoke but to few of George's
comrades there, all the swords of the young fellows at the mess-table
would have leapt from their scabbards to fight round her; and so it was
in the little narrow lodging-house and circle at Fulham, she interested
and pleased everybody.  If she had been Mrs.\ Mango herself, of the
great house of Mango, Plantain, and Co., Crutched Friars, and the
magnificent proprietress of the Pineries, Fulham, who gave summer
dejeuners frequented by Dukes and Earls, and drove about the parish
with magnificent yellow liveries and bay horses, such as the royal
stables at Kensington themselves could not turn out---I say had she been
Mrs.\ Mango herself, or her son's wife, Lady Mary Mango (daughter of the
Earl of Castlemouldy, who condescended to marry the head of the firm),
the tradesmen of the neighbourhood could not pay her more honour than
they invariably showed to the gentle young widow, when she passed by
their doors, or made her humble purchases at their shops.

Thus it was not only Mr.\ Pestler, the medical man, but Mr.\ Linton the
young assistant, who doctored the servant maids and small tradesmen,
and might be seen any day reading the Times in the surgery, who openly
declared himself the slave of Mrs.\ Osborne.  He was a personable young
gentleman, more welcome at Mrs.\ Sedley's lodgings than his principal;
and if anything went wrong with Georgy, he would drop in twice or
thrice in the day to see the little chap, and without so much as the
thought of a fee.  He would abstract lozenges, tamarinds, and other
produce from the surgery-drawers for little Georgy's benefit, and
compounded draughts and mixtures for him of miraculous sweetness, so
that it was quite a pleasure to the child to be ailing.  He and
Pestler, his chief, sat up two whole nights by the boy in that
momentous and awful week when Georgy had the measles; and when you
would have thought, from the mother's terror, that there had never been
measles in the world before. Would they have done as much for other
people? Did they sit up for the folks at the Pineries, when Ralph
Plantagenet, and Gwendoline, and Guinever Mango had the same juvenile
complaint? Did they sit up for little Mary Clapp, the landlord's
daughter, who actually caught the disease of little Georgy? Truth
compels one to say, no. They slept quite undisturbed, at least as far
as she was concerned---pronounced hers to be a slight case, which would
almost cure itself, sent her in a draught or two, and threw in bark
when the child rallied, with perfect indifference, and just for form's
sake.

Again, there was the little French chevalier opposite, who gave lessons
in his native tongue at various schools in the neighbourhood, and who
might be heard in his apartment of nights playing tremulous old
gavottes and minuets on a wheezy old fiddle. Whenever this powdered and
courteous old man, who never missed a Sunday at the convent chapel at
Hammersmith, and who was in all respects, thoughts, conduct, and
bearing utterly unlike the bearded savages of his nation, who curse
perfidious Albion, and scowl at you from over their cigars, in the
Quadrant arcades at the present day---whenever the old Chevalier de
Talonrouge spoke of Mistress Osborne, he would first finish his pinch
of snuff, flick away the remaining particles of dust with a graceful
wave of his hand, gather up his fingers again into a bunch, and,
bringing them up to his mouth, blow them open with a kiss, exclaiming,
Ah! la divine creature!  He vowed and protested that when Amelia walked
in the Brompton Lanes flowers grew in profusion under her feet.  He
called little Georgy Cupid, and asked him news of Venus, his mamma; and
told the astonished Betty Flanagan that she was one of the Graces, and
the favourite attendant of the Reine des Amours.

Instances might be multiplied of this easily gained and unconscious
popularity.  Did not Mr.\ Binny, the mild and genteel curate of the
district chapel, which the family attended, call assiduously upon the
widow, dandle the little boy on his knee, and offer to teach him Latin,
to the anger of the elderly virgin, his sister, who kept house for him?
``There is nothing in her, Beilby,'' the latter lady would say.  ``When
she comes to tea here she does not speak a word during the whole
evening.  She is but a poor lackadaisical creature, and it is my belief
has no heart at all.  It is only her pretty face which all you
gentlemen admire so.  Miss Grits, who has five thousand pounds, and
expectations besides, has twice as much character, and is a thousand
times more agreeable to my taste; and if she were good-looking I know
that you would think her perfection.''

Very likely Miss Binny was right to a great extent.  It IS the pretty
face which creates sympathy in the hearts of men, those wicked rogues.
A woman may possess the wisdom and chastity of Minerva, and we give no
heed to her, if she has a plain face.  What folly will not a pair of
bright eyes make pardonable? What dulness may not red lips and sweet
accents render pleasant? And so, with their usual sense of justice,
ladies argue that because a woman is handsome, therefore she is a fool.
O ladies, ladies! there are some of you who are neither handsome nor
wise.

These are but trivial incidents to recount in the life of our heroine.
Her tale does not deal in wonders, as the gentle reader has already no
doubt perceived; and if a journal had been kept of her proceedings
during the seven years after the birth of her son, there would be found
few incidents more remarkable in it than that of the measles, recorded
in the foregoing page.  Yes, one day, and greatly to her wonder, the
Reverend Mr.\ Binny, just mentioned, asked her to change her name of
Osborne for his own; when, with deep blushes and tears in her eyes and
voice, she thanked him for his regard for her, expressed gratitude for
his attentions to her and to her poor little boy, but said that she
never, never could think of any but---but the husband whom she had lost.

On the twenty-fifth of April, and the eighteenth of June, the days of
marriage and widowhood, she kept her room entirely, consecrating them
(and we do not know how many hours of solitary night-thought, her
little boy sleeping in his crib by her bedside) to the memory of that
departed friend.  During the day she was more active. She had to teach
George to read and to write and a little to draw.  She read books, in
order that she might tell him stories from them.  As his eyes opened
and his mind expanded under the influence of the outward nature round
about him, she taught the child, to the best of her humble power, to
acknowledge the Maker of all, and every night and every morning he and
she---(in that awful and touching communion which I think must bring a
thrill to the heart of every man who witnesses or who remembers
it)---the mother and the little boy---prayed to Our Father together, the
mother pleading with all her gentle heart, the child lisping after her
as she spoke.  And each time they prayed to God to bless dear Papa, as
if he were alive and in the room with them.  To wash and dress this
young gentleman---to take him for a run of the mornings, before
breakfast, and the retreat of grandpapa for ``business''---to make for him
the most wonderful and ingenious dresses, for which end the thrifty
widow cut up and altered every available little bit of finery which she
possessed out of her wardrobe during her marriage---for Mrs.\ Osborne
herself (greatly to her mother's vexation, who preferred fine clothes,
especially since her misfortunes) always wore a black gown and a straw
bonnet with a black ribbon---occupied her many hours of the day.  Others
she had to spare, at the service of her mother and her old father.  She
had taken the pains to learn, and used to play cribbage with this
gentleman on the nights when he did not go to his club.  She sang for
him when he was so minded, and it was a good sign, for he invariably
fell into a comfortable sleep during the music.  She wrote out his
numerous memorials, letters, prospectuses, and projects.  It was in her
handwriting that most of the old gentleman's former acquaintances were
informed that he had become an agent for the Black Diamond and
Anti-Cinder Coal Company and could supply his friends and the public
with the best coals at ---s. per chaldron.  All he did was to sign the
circulars with his flourish and signature, and direct them in a shaky,
clerklike hand.  One of these papers was sent to Major Dobbin,---Regt.,
care of Messrs.  Cox and Greenwood; but the Major being in Madras at
the time, had no particular call for coals.  He knew, though, the hand
which had written the prospectus.  Good God! what would he not have
given to hold it in his own!  A second prospectus came out, informing
the Major that J.  Sedley and Company, having established agencies at
Oporto, Bordeaux, and St.\ Mary's, were enabled to offer to their
friends and the public generally the finest and most celebrated growths
of ports, sherries, and claret wines at reasonable prices and under
extraordinary advantages. Acting upon this hint, Dobbin furiously
canvassed the governor, the commander-in-chief, the judges, the
regiments, and everybody whom he knew in the Presidency, and sent home
to Sedley and Co.  orders for wine which perfectly astonished Mr.\ %
Sedley and Mr.\ Clapp, who was the Co.  in the business.  But no more
orders came after that first burst of good fortune, on which poor old
Sedley was about to build a house in the City, a regiment of clerks, a
dock to himself, and correspondents all over the world.  The old
gentleman's former taste in wine had gone:  the curses of the mess-room
assailed Major Dobbin for the vile drinks he had been the means of
introducing there; and he bought back a great quantity of the wine and
sold it at public outcry, at an enormous loss to himself. As for Jos,
who was by this time promoted to a seat at the Revenue Board at
Calcutta, he was wild with rage when the post brought him out a bundle
of these Bacchanalian prospectuses, with a private note from his
father, telling Jos that his senior counted upon him in this
enterprise, and had consigned a quantity of select wines to him, as per
invoice, drawing bills upon him for the amount of the same.  Jos, who
would no more have it supposed that his father, Jos Sedley's father, of
the Board of Revenue, was a wine merchant asking for orders, than that
he was Jack Ketch, refused the bills with scorn, wrote back
contumeliously to the old gentleman, bidding him to mind his own
affairs; and the protested paper coming back, Sedley and Co.  had to
take it up, with the profits which they had made out of the Madras
venture, and with a little portion of Emmy's savings.

Besides her pension of fifty pounds a year, there had been five hundred
pounds, as her husband's executor stated, left in the agent's hands at
the time of Osborne's demise, which sum, as George's guardian, Dobbin
proposed to put out at 8 per cent in an Indian house of agency.  Mr.\ %
Sedley, who thought the Major had some roguish intentions of his own
about the money, was strongly against this plan; and he went to the
agents to protest personally against the employment of the money in
question, when he learned, to his surprise, that there had been no such
sum in their hands, that all the late Captain's assets did not amount
to a hundred pounds, and that the five hundred pounds in question must
be a separate sum, of which Major Dobbin knew the particulars. More
than ever convinced that there was some roguery, old Sedley pursued the
Major.  As his daughter's nearest friend, he demanded with a high hand
a statement of the late Captain's accounts.  Dobbin's stammering,
blushing, and awkwardness added to the other's convictions that he had
a rogue to deal with, and in a majestic tone he told that officer a
piece of his mind, as he called it, simply stating his belief that the
Major was unlawfully detaining his late son-in-law's money.

Dobbin at this lost all patience, and if his accuser had not been so
old and so broken, a quarrel might have ensued between them at the
Slaughters' Coffee-house, in a box of which place of entertainment the
gentlemen had their colloquy.  ``Come upstairs, sir,'' lisped out the
Major.  ``I insist on your coming up the stairs, and I will show which
is the injured party, poor George or I''; and, dragging the old
gentleman up to his bedroom, he produced from his desk Osborne's
accounts, and a bundle of IOU's which the latter had given, who, to do
him justice, was always ready to give an IOU.  ``He paid his bills in
England,'' Dobbin added, ``but he had not a hundred pounds in the world
when he fell.  I and one or two of his brother officers made up the
little sum, which was all that we could spare, and you dare tell us
that we are trying to cheat the widow and the orphan.'' Sedley was very
contrite and humbled, though the fact is that William Dobbin had told a
great falsehood to the old gentleman; having himself given every
shilling of the money, having buried his friend, and paid all the fees
and charges incident upon the calamity and removal of poor Amelia.

About these expenses old Osborne had never given himself any trouble to
think, nor any other relative of Amelia, nor Amelia herself, indeed.
She trusted to Major Dobbin as an accountant, took his somewhat
confused calculations for granted, and never once suspected how much
she was in his debt.

Twice or thrice in the year, according to her promise, she wrote him
letters to Madras, letters all about little Georgy.  How he treasured
these papers!  Whenever Amelia wrote he answered, and not until then.
But he sent over endless remembrances of himself to his godson and to
her.  He ordered and sent a box of scarfs and a grand ivory set of
chess-men from China.  The pawns were little green and white men, with
real swords and shields; the knights were on horseback, the castles
were on the backs of elephants.  ``Mrs.\ Mango's own set at the Pineries
was not so fine,'' Mr.\ Pestler remarked.  These chess-men were the
delight of Georgy's life, who printed his first letter in
acknowledgement of this gift of his godpapa.  He sent over preserves
and pickles, which latter the young gentleman tried surreptitiously in
the sideboard and half-killed himself with eating.  He thought it was a
judgement upon him for stealing, they were so hot.  Emmy wrote a
comical little account of this mishap to the Major:  it pleased him to
think that her spirits were rallying and that she could be merry
sometimes now.  He sent over a pair of shawls, a white one for her and
a black one with palm-leaves for her mother, and a pair of red scarfs,
as winter wrappers, for old Mr.\ Sedley and George. The shawls were
worth fifty guineas apiece at the very least, as Mrs.\ Sedley knew.  She
wore hers in state at church at Brompton, and was congratulated by her
female friends upon the splendid acquisition.  Emmy's, too, became
prettily her modest black gown.  ``What a pity it is she won't think of
him!'' Mrs.\ Sedley remarked to Mrs.\ Clapp and to all her friends of
Brompton.  ``Jos never sent us such presents, I am sure, and grudges us
everything.  It is evident that the Major is over head and ears in love
with her; and yet, whenever I so much as hint it, she turns red and
begins to cry and goes and sits upstairs with her miniature.  I'm sick
of that miniature.  I wish we had never seen those odious purse-proud
Osbornes.''

Amidst such humble scenes and associates George's early youth was
passed, and the boy grew up delicate, sensitive, imperious,
woman-bred---domineering the gentle mother whom he loved with passionate
affection.  He ruled all the rest of the little world round about him.
As he grew, the elders were amazed at his haughty manner and his
constant likeness to his father.  He asked questions about everything,
as inquiring youth will do.  The profundity of his remarks and
interrogatories astonished his old grandfather, who perfectly bored the
club at the tavern with stories about the little lad's learning and
genius.  He suffered his grandmother with a good-humoured
indifference.  The small circle round about him believed that the equal
of the boy did not exist upon the earth.  Georgy inherited his father's
pride, and perhaps thought they were not wrong.

When he grew to be about six years old, Dobbin began to write to him
very much.  The Major wanted to hear that Georgy was going to a school
and hoped he would acquit himself with credit there:  or would he have
a good tutor at home? It was time that he should begin to learn; and
his godfather and guardian hinted that he hoped to be allowed to defray
the charges of the boy's education, which would fall heavily upon his
mother's straitened income.  The Major, in a word, was always thinking
about Amelia and her little boy, and by orders to his agents kept the
latter provided with picture-books, paint-boxes, desks, and all
conceivable implements of amusement and instruction.  Three days before
George's sixth birthday a gentleman in a gig, accompanied by a servant,
drove up to Mr.\ Sedley's house and asked to see Master George Osborne:
it was Mr.\ Woolsey, military tailor, of Conduit Street, who came at the
Major's order to measure the young gentleman for a suit of clothes.  He
had had the honour of making for the Captain, the young gentleman's
father. Sometimes, too, and by the Major's desire no doubt, his
sisters, the Misses Dobbin, would call in the family carriage to take
Amelia and the little boy to drive if they were so inclined.  The
patronage and kindness of these ladies was very uncomfortable to
Amelia, but she bore it meekly enough, for her nature was to yield;
and, besides, the carriage and its splendours gave little Georgy
immense pleasure. The ladies begged occasionally that the child might
pass a day with them, and he was always glad to go to that fine
garden-house at Denmark Hill, where they lived, and where there were
such fine grapes in the hot-houses and peaches on the walls.

One day they kindly came over to Amelia with news which they were SURE
would delight her---something VERY interesting about their dear William.

``What was it:  was he coming home?'' she asked with pleasure beaming in
her eyes.

``Oh, no---not the least---but they had very good reason to believe that
dear William was about to be married---and to a relation of a very dear
friend of Amelia's---to Miss Glorvina O'Dowd, Sir Michael O'Dowd's
sister, who had gone out to join Lady O'Dowd at Madras---a very
beautiful and accomplished girl, everybody said.''

Amelia said ``Oh!'' Amelia was very VERY happy indeed. But she supposed
Glorvina could not be like her old acquaintance, who was most
kind---but---but she was very happy indeed.  And by some impulse of which
I cannot explain the meaning, she took George in her arms and kissed
him with an extraordinary tenderness.  Her eyes were quite moist when
she put the child down; and she scarcely spoke a word during the whole
of the drive---though she was so very happy indeed.



\chapter{A Cynical Chapter}

Our duty now takes us back for a brief space to some old Hampshire
acquaintances of ours, whose hopes respecting the disposal of their
rich kinswoman's property were so woefully disappointed.  After
counting upon thirty thousand pounds from his sister, it was a heavy
blow to Bute Crawley to receive but five; out of which sum, when he
had paid his own debts and those of Jim, his son at college, a very
small fragment remained to portion off his four plain daughters.  Mrs.\ %
Bute never knew, or at least never acknowledged, how far her own
tyrannous behaviour had tended to ruin her husband. All that woman
could do, she vowed and protested she had done.  Was it her fault if
she did not possess those sycophantic arts which her hypocritical
nephew, Pitt Crawley, practised? She wished him all the happiness which
he merited out of his ill-gotten gains.  ``At least the money will
remain in the family,'' she said charitably.  ``Pitt will never spend it,
my dear, that is quite certain; for a greater miser does not exist in
England, and he is as odious, though in a different way, as his
spendthrift brother, the abandoned Rawdon.''

So Mrs.\ Bute, after the first shock of rage and disappointment, began
to accommodate herself as best she could to her altered fortunes and to
save and retrench with all her might.  She instructed her daughters how
to bear poverty cheerfully, and invented a thousand notable methods to
conceal or evade it.  She took them about to balls and public places in
the neighbourhood, with praiseworthy energy; nay, she entertained her
friends in a hospitable comfortable manner at the Rectory, and much
more frequently than before dear Miss Crawley's legacy had fallen in.
From her outward bearing nobody would have supposed that the family had
been disappointed in their expectations, or have guessed from her
frequent appearance in public how she pinched and starved at home.  Her
girls had more milliners' furniture than they had ever enjoyed before.
They appeared perseveringly at the Winchester and Southampton
assemblies; they penetrated to Cowes for the race-balls and
regatta-gaieties there; and their carriage, with the horses taken from
the plough, was at work perpetually, until it began almost to be
believed that the four sisters had had fortunes left them by their
aunt, whose name the family never mentioned in public but with the most
tender gratitude and regard.  I know no sort of lying which is more
frequent in Vanity Fair than this, and it may be remarked how people
who practise it take credit to themselves for their hypocrisy, and
fancy that they are exceedingly virtuous and praiseworthy, because they
are able to deceive the world with regard to the extent of their means.

Mrs.\ Bute certainly thought herself one of the most virtuous women in
England, and the sight of her happy family was an edifying one to
strangers.  They were so cheerful, so loving, so well-educated, so
simple!  Martha painted flowers exquisitely and furnished half the
charity bazaars in the county.  Emma was a regular County Bulbul, and
her verses in the Hampshire Telegraph were the glory of its Poet's
Corner.  Fanny and Matilda sang duets together, Mamma playing the
piano, and the other two sisters sitting with their arms round each
other's waists and listening affectionately.  Nobody saw the poor girls
drumming at the duets in private.  No one saw Mamma drilling them
rigidly hour after hour.  In a word, Mrs.\ Bute put a good face against
fortune and kept up appearances in the most virtuous manner.

Everything that a good and respectable mother could do Mrs.\ Bute did.
She got over yachting men from Southampton, parsons from the Cathedral
Close at Winchester, and officers from the barracks there. She tried to
inveigle the young barristers at assizes and encouraged Jim to bring
home friends with whom he went out hunting with the H. H.  What will
not a mother do for the benefit of her beloved ones?

Between such a woman and her brother-in-law, the odious Baronet at the
Hall, it is manifest that there could be very little in common. The
rupture between Bute and his brother Sir Pitt was complete; indeed,
between Sir Pitt and the whole county, to which the old man was a
scandal.  His dislike for respectable society increased with age, and
the lodge-gates had not opened to a gentleman's carriage-wheels since
Pitt and Lady Jane came to pay their visit of duty after their marriage.

That was an awful and unfortunate visit, never to be thought of by the
family without horror.  Pitt begged his wife, with a ghastly
countenance, never to speak of it, and it was only through Mrs.\ Bute
herself, who still knew everything which took place at the Hall, that
the circumstances of Sir Pitt's reception of his son and
daughter-in-law were ever known at all.

As they drove up the avenue of the park in their neat and
well-appointed carriage, Pitt remarked with dismay and wrath great gaps
among the trees---his trees---which the old Baronet was felling entirely
without license.  The park wore an aspect of utter dreariness and ruin.
The drives were ill kept, and the neat carriage splashed and floundered
in muddy pools along the road.  The great sweep in front of the terrace
and entrance stair was black and covered with mosses; the once trim
flower-beds rank and weedy. Shutters were up along almost the whole
line of the house; the great hall-door was unbarred after much ringing
of the bell; an individual in ribbons was seen flitting up the black
oak stair, as Horrocks at length admitted the heir of Queen's Crawley
and his bride into the halls of their fathers.  He led the way into Sir
Pitt's ``Library,'' as it was called, the fumes of tobacco growing
stronger as Pitt and Lady Jane approached that apartment, ``Sir Pitt
ain't very well,'' Horrocks remarked apologetically and hinted that his
master was afflicted with lumbago.

The library looked out on the front walk and park. Sir Pitt had opened
one of the windows, and was bawling out thence to the postilion and
Pitt's servant, who seemed to be about to take the baggage down.

``Don't move none of them trunks,'' he cried, pointing with a pipe which
he held in his hand.  ``It's only a morning visit, Tucker, you fool.
Lor, what cracks that off hoss has in his heels!  Ain't there no one at
the King's Head to rub 'em a little? How do, Pitt? How do, my dear?
Come to see the old man, hay? 'Gad---you've a pretty face, too. You
ain't like that old horse-godmother, your mother. Come and give old
Pitt a kiss, like a good little gal.''

The embrace disconcerted the daughter-in-law somewhat, as the caresses
of the old gentleman, unshorn and perfumed with tobacco, might well do.
But she remembered that her brother Southdown had mustachios, and
smoked cigars, and submitted to the Baronet with a tolerable grace.

``Pitt has got vat,'' said the Baronet, after this mark of affection.
``Does he read ee very long zermons, my dear? Hundredth Psalm, Evening
Hymn, hay Pitt? Go and get a glass of Malmsey and a cake for my Lady
Jane, Horrocks, you great big booby, and don't stand stearing there
like a fat pig.  I won't ask you to stop, my dear; you'll find it too
stoopid, and so should I too along a Pitt.  I'm an old man now, and
like my own ways, and my pipe and backgammon of a night.''

``I can play at backgammon, sir,'' said Lady Jane, laughing.  ``I used to
play with Papa and Miss Crawley, didn't I, Mr.\ Crawley?''

``Lady Jane can play, sir, at the game to which you state that you are
so partial,'' Pitt said haughtily.

``But she wawn't stop for all that.  Naw, naw, goo back to Mudbury and
give Mrs.\ Rincer a benefit; or drive down to the Rectory and ask Buty
for a dinner.  He'll be charmed to see you, you know; he's so much
obliged to you for gettin' the old woman's money.  Ha, ha! Some of it
will do to patch up the Hall when I'm gone.''

``I perceive, sir,'' said Pitt with a heightened voice, ``that your people
will cut down the timber.''

``Yees, yees, very fine weather, and seasonable for the time of year,''
Sir Pitt answered, who had suddenly grown deaf.  ``But I'm gittin' old,
Pitt, now.  Law bless you, you ain't far from fifty yourself.  But he
wears well, my pretty Lady Jane, don't he? It's all godliness,
sobriety, and a moral life.  Look at me, I'm not very fur from
fowr-score---he, he''; and he laughed, and took snuff, and leered at her
and pinched her hand.

Pitt once more brought the conversation back to the timber, but the
Baronet was deaf again in an instant.

``I'm gittin' very old, and have been cruel bad this year with the
lumbago.  I shan't be here now for long; but I'm glad ee've come,
daughter-in-law.  I like your face, Lady Jane:  it's got none of the
damned high-boned Binkie look in it; and I'll give ee something pretty,
my dear, to go to Court in.'' And he shuffled across the room to a
cupboard, from which he took a little old case containing jewels of
some value.  ``Take that,'' said he, ``my dear; it belonged to my mother,
and afterwards to the first Lady Binkie. Pretty pearls---never gave 'em
the ironmonger's daughter. No, no.  Take 'em and put 'em up quick,''
said he, thrusting the case into his daughter's hand, and clapping the
door of the cabinet to, as Horrocks entered with a salver and
refreshments.

``What have you a been and given Pitt's wife?'' said the individual in
ribbons, when Pitt and Lady Jane had taken leave of the old gentleman.
It was Miss Horrocks, the butler's daughter---the cause of the scandal
throughout the county---the lady who reigned now almost supreme at
Queen's Crawley.

The rise and progress of those Ribbons had been marked with dismay by
the county and family.  The Ribbons opened an account at the Mudbury
Branch Savings Bank; the Ribbons drove to church, monopolising the
pony-chaise, which was for the use of the servants at the Hall.  The
domestics were dismissed at her pleasure. The Scotch gardener, who
still lingered on the premises, taking a pride in his walls and
hot-houses, and indeed making a pretty good livelihood by the garden,
which he farmed, and of which he sold the produce at Southampton, found
the Ribbons eating peaches on a sunshiny morning at the south-wall, and
had his ears boxed when he remonstrated about this attack on his
property.  He and his Scotch wife and his Scotch children, the only
respectable inhabitants of Queen's Crawley, were forced to migrate,
with their goods and their chattels, and left the stately comfortable
gardens to go to waste, and the flower-beds to run to seed.  Poor Lady
Crawley's rose-garden became the dreariest wilderness.  Only two or
three domestics shuddered in the bleak old servants' hall.  The stables
and offices were vacant, and shut up, and half ruined.  Sir Pitt lived
in private, and boozed nightly with Horrocks, his butler or house-steward
(as he now began to be called), and the abandoned Ribbons. The
times were very much changed since the period when she drove to Mudbury
in the spring-cart and called the small tradesmen ``Sir.'' It may have
been shame, or it may have been dislike of his neighbours, but the old
Cynic of Queen's Crawley hardly issued from his park-gates at all now.
He quarrelled with his agents and screwed his tenants by letter.  His
days were passed in conducting his own correspondence; the lawyers and
farm-bailiffs who had to do business with him could not reach him but
through the Ribbons, who received them at the door of the housekeeper's
room, which commanded the back entrance by which they were admitted;
and so the Baronet's daily perplexities increased, and his
embarrassments multiplied round him.

The horror of Pitt Crawley may be imagined, as these reports of his
father's dotage reached the most exemplary and correct of gentlemen. He
trembled daily lest he should hear that the Ribbons was proclaimed his
second legal mother-in-law.  After that first and last visit, his
father's name was never mentioned in Pitt's polite and genteel
establishment.  It was the skeleton in his house, and all the family
walked by it in terror and silence.  The Countess Southdown kept on
dropping per coach at the lodge-gate the most exciting tracts, tracts
which ought to frighten the hair off your head.  Mrs.\ Bute at the
parsonage nightly looked out to see if the sky was red over the elms
behind which the Hall stood, and the mansion was on fire.  Sir G.
Wapshot and Sir H.  Fuddlestone, old friends of the house, wouldn't sit
on the bench with Sir Pitt at Quarter Sessions, and cut him dead in the
High Street of Southampton, where the reprobate stood offering his
dirty old hands to them.  Nothing had any effect upon him; he put his
hands into his pockets, and burst out laughing, as he scrambled into
his carriage and four; he used to burst out laughing at Lady
Southdown's tracts; and he laughed at his sons, and at the world, and
at the Ribbons when she was angry, which was not seldom.

Miss Horrocks was installed as housekeeper at Queen's Crawley, and
ruled all the domestics there with great majesty and rigour.  All the
servants were instructed to address her as ``Mum,'' or ``Madam''---and
there was one little maid, on her promotion, who persisted in calling
her ``My Lady,'' without any rebuke on the part of the housekeeper.
``There has been better ladies, and there has been worser, Hester,'' was
Miss Horrocks' reply to this compliment of her inferior; so she ruled,
having supreme power over all except her father, whom, however, she
treated with considerable haughtiness, warning him not to be too
familiar in his behaviour to one ``as was to be a Baronet's lady.''
Indeed, she rehearsed that exalted part in life with great satisfaction
to herself, and to the amusement of old Sir Pitt, who chuckled at her
airs and graces, and would laugh by the hour together at her
assumptions of dignity and imitations of genteel life. He swore it was
as good as a play to see her in the character of a fine dame, and he
made her put on one of the first Lady Crawley's court-dresses, swearing
(entirely to Miss Horrocks' own concurrence) that the dress became her
prodigiously, and threatening to drive her off that very instant to
Court in a coach-and-four.  She had the ransacking of the wardrobes of
the two defunct ladies, and cut and hacked their posthumous finery so
as to suit her own tastes and figure.  And she would have liked to take
possession of their jewels and trinkets too; but the old Baronet had
locked them away in his private cabinet; nor could she coax or wheedle
him out of the keys.  And it is a fact, that some time after she left
Queen's Crawley a copy-book belonging to this lady was discovered,
which showed that she had taken great pains in private to learn the art
of writing in general, and especially of writing her own name as Lady
Crawley, Lady Betsy Horrocks, Lady Elizabeth Crawley, \&c.

Though the good people of the Parsonage never went to the Hall and
shunned the horrid old dotard its owner, yet they kept a strict
knowledge of all that happened there, and were looking out every day
for the catastrophe for which Miss Horrocks was also eager.  But Fate
intervened enviously and prevented her from receiving the reward due to
such immaculate love and virtue.

One day the Baronet surprised ``her ladyship,'' as he jocularly called
her, seated at that old and tuneless piano in the drawing-room, which
had scarcely been touched since Becky Sharp played quadrilles upon
it---seated at the piano with the utmost gravity and squalling to the
best of her power in imitation of the music which she had sometimes
heard.  The little kitchen-maid on her promotion was standing at her
mistress's side, quite delighted during the operation, and wagging her
head up and down and crying, ``Lor, Mum, 'tis bittiful''---just like a
genteel sycophant in a real drawing-room.

This incident made the old Baronet roar with laughter, as usual.  He
narrated the circumstance a dozen times to Horrocks in the course of
the evening, and greatly to the discomfiture of Miss Horrocks.  He
thrummed on the table as if it had been a musical instrument, and
squalled in imitation of her manner of singing.  He vowed that such a
beautiful voice ought to be cultivated and declared she ought to have
singing-masters, in which proposals she saw nothing ridiculous. He was
in great spirits that night, and drank with his friend and butler an
extraordinary quantity of rum-and-water---at a very late hour the
faithful friend and domestic conducted his master to his bedroom.

Half an hour afterwards there was a great hurry and bustle in the
house.  Lights went about from window to window in the lonely desolate
old Hall, whereof but two or three rooms were ordinarily occupied by
its owner. Presently, a boy on a pony went galloping off to Mudbury, to
the Doctor's house there.  And in another hour (by which fact we
ascertain how carefully the excellent Mrs.\ Bute Crawley had always kept
up an understanding with the great house), that lady in her clogs and
calash, the Reverend Bute Crawley, and James Crawley, her son, had
walked over from the Rectory through the park, and had entered the
mansion by the open hall-door.

They passed through the hall and the small oak parlour, on the table of
which stood the three tumblers and the empty rum-bottle which had
served for Sir Pitt's carouse, and through that apartment into Sir
Pitt's study, where they found Miss Horrocks, of the guilty ribbons,
with a wild air, trying at the presses and escritoires with a bunch of
keys.  She dropped them with a scream of terror, as little Mrs.\ Bute's
eyes flashed out at her from under her black calash.

``Look at that, James and Mr.\ Crawley,'' cried Mrs.\ Bute, pointing at the
scared figure of the black-eyed, guilty wench.

``He gave 'em me; he gave 'em me!'' she cried.

``Gave them you, you abandoned creature!'' screamed Mrs.\ Bute.  ``Bear
witness, Mr.\ Crawley, we found this good-for-nothing woman in the act
of stealing your brother's property; and she will be hanged, as I
always said she would.''

Betsy Horrocks, quite daunted, flung herself down on her knees,
bursting into tears.  But those who know a really good woman are aware
that she is not in a hurry to forgive, and that the humiliation of an
enemy is a triumph to her soul.

``Ring the bell, James,'' Mrs.\ Bute said.  ``Go on ringing it till the
people come.'' The three or four domestics resident in the deserted old
house came presently at that jangling and continued summons.

``Put that woman in the strong-room,'' she said.  ``We caught her in the
act of robbing Sir Pitt.  Mr.\ Crawley, you'll make out her
committal---and, Beddoes, you'll drive her over in the spring cart, in
the morning, to Southampton Gaol.''

``My dear,'' interposed the Magistrate and Rector---``she's only---''

``Are there no handcuffs?'' Mrs.\ Bute continued, stamping in her clogs.
``There used to be handcuffs. Where's the creature's abominable father?''

``He DID give 'em me,'' still cried poor Betsy; ``didn't he, Hester? You
saw Sir Pitt---you know you did---give 'em me, ever so long ago---the day
after Mudbury fair:  not that I want 'em.  Take 'em if you think they
ain't mine.'' And here the unhappy wretch pulled out from her pocket a
large pair of paste shoe-buckles which had excited her admiration, and
which she had just appropriated out of one of the bookcases in the
study, where they had lain.

``Law, Betsy, how could you go for to tell such a wicked story!'' said
Hester, the little kitchen-maid late on her promotion---``and to Madame
Crawley, so good and kind, and his Rev'rince (with a curtsey), and you
may search all MY boxes, Mum, I'm sure, and here's my keys as I'm an
honest girl, though of pore parents and workhouse bred---and if you find
so much as a beggarly bit of lace or a silk stocking out of all the
gownds as YOU'VE had the picking of, may I never go to church agin.''

``Give up your keys, you hardened hussy,'' hissed out the virtuous little
lady in the calash.

``And here's a candle, Mum, and if you please, Mum, I can show you her
room, Mum, and the press in the housekeeper's room, Mum, where she
keeps heaps and heaps of things, Mum,'' cried out the eager little
Hester with a profusion of curtseys.

``Hold your tongue, if you please.  I know the room which the creature
occupies perfectly well.  Mrs.\ Brown, have the goodness to come with
me, and Beddoes don't you lose sight of that woman,'' said Mrs.\ Bute,
seizing the candle.  ``Mr.\ Crawley, you had better go upstairs and see
that they are not murdering your unfortunate brother''---and the calash,
escorted by Mrs.\ Brown, walked away to the apartment which, as she said
truly, she knew perfectly well.

Bute went upstairs and found the Doctor from Mudbury, with the
frightened Horrocks over his master in a chair.  They were trying to
bleed Sir Pitt Crawley.

With the early morning an express was sent off to Mr.\ Pitt Crawley by
the Rector's lady, who assumed the command of everything, and had
watched the old Baronet through the night.  He had been brought back to
a sort of life; he could not speak, but seemed to recognize people.
Mrs.\ Bute kept resolutely by his bedside.  She never seemed to want to
sleep, that little woman, and did not close her fiery black eyes once,
though the Doctor snored in the arm-chair. Horrocks made some wild
efforts to assert his authority and assist his master; but Mrs.\ Bute
called him a tipsy old wretch and bade him never show his face again in
that house, or he should be transported like his abominable daughter.

Terrified by her manner, he slunk down to the oak parlour where Mr.\ %
James was, who, having tried the bottle standing there and found no
liquor in it, ordered Mr.\ Horrocks to get another bottle of rum, which
he fetched, with clean glasses, and to which the Rector and his son sat
down, ordering Horrocks to put down the keys at that instant and never
to show his face again.

Cowed by this behaviour, Horrocks gave up the keys, and he and his
daughter slunk off silently through the night and gave up possession of
the house of Queen's Crawley.



\chapter{In Which Becky Is Recognized by the Family}

The heir of Crawley arrived at home, in due time, after this
catastrophe, and henceforth may be said to have reigned in Queen's
Crawley.  For though the old Baronet survived many months, he never
recovered the use of his intellect or his speech completely, and the
government of the estate devolved upon his elder son.  In a strange
condition Pitt found it.  Sir Pitt was always buying and mortgaging; he
had twenty men of business, and quarrels with each; quarrels with all
his tenants, and lawsuits with them; lawsuits with the lawyers;
lawsuits with the Mining and Dock Companies in which he was proprietor;
and with every person with whom he had business.  To unravel these
difficulties and to set the estate clear was a task worthy of the
orderly and persevering diplomatist of Pumpernickel, and he set himself
to work with prodigious assiduity.  His whole family, of course, was
transported to Queen's Crawley, whither Lady Southdown, of course, came
too; and she set about converting the parish under the Rector's nose,
and brought down her irregular clergy to the dismay of the angry Mrs
Bute.  Sir Pitt had concluded no bargain for the sale of the living of
Queen's Crawley; when it should drop, her Ladyship proposed to take the
patronage into her own hands and present a young protege to the
Rectory, on which subject the diplomatic Pitt said nothing.

Mrs.\ Bute's intentions with regard to Miss Betsy Horrocks were not
carried into effect, and she paid no visit to Southampton Gaol.  She
and her father left the Hall when the latter took possession of the
Crawley Arms in the village, of which he had got a lease from Sir Pitt.
The ex-butler had obtained a small freehold there likewise, which gave
him a vote for the borough.  The Rector had another of these votes, and
these and four others formed the representative body which returned the
two members for Queen's Crawley.

There was a show of courtesy kept up between the Rectory and the Hall
ladies, between the younger ones at least, for Mrs.\ Bute and Lady
Southdown never could meet without battles, and gradually ceased seeing
each other.  Her Ladyship kept her room when the ladies from the
Rectory visited their cousins at the Hall.  Perhaps Mr.\ Pitt was not
very much displeased at these occasional absences of his mamma-in-law.
He believed the Binkie family to be the greatest and wisest and most
interesting in the world, and her Ladyship and his aunt had long held
ascendency over him; but sometimes he felt that she commanded him too
much.  To be considered young was complimentary, doubtless, but at
six-and-forty to be treated as a boy was sometimes mortifying.  Lady
Jane yielded up everything, however, to her mother.  She was only fond
of her children in private, and it was lucky for her that Lady
Southdown's multifarious business, her conferences with ministers, and
her correspondence with all the missionaries of Africa, Asia, and
Australasia, \&c., occupied the venerable Countess a great deal, so that
she had but little time to devote to her granddaughter, the little
Matilda, and her grandson, Master Pitt Crawley. The latter was a feeble
child, and it was only by prodigious quantities of calomel that Lady
Southdown was able to keep him in life at all.

As for Sir Pitt he retired into those very apartments where Lady
Crawley had been previously extinguished, and here was tended by Miss
Hester, the girl upon her promotion, with constant care and assiduity.
What love, what fidelity, what constancy is there equal to that of a
nurse with good wages? They smooth pillows; and make arrowroot; they
get up at nights; they bear complaints and querulousness; they see the
sun shining out of doors and don't want to go abroad; they sleep on
arm-chairs and eat their meals in solitude; they pass long long
evenings doing nothing, watching the embers, and the patient's drink
simmering in the jug; they read the weekly paper the whole week
through; and Law's Serious Call or the Whole Duty of Man suffices them
for literature for the year---and we quarrel with them because, when
their relations come to see them once a week, a little gin is smuggled
in in their linen basket. Ladies, what man's love is there that would
stand a year's nursing of the object of his affection? Whereas a nurse
will stand by you for ten pounds a quarter, and we think her too highly
paid.  At least Mr.\ Crawley grumbled a good deal about paying half as
much to Miss Hester for her constant attendance upon the Baronet his
father.

Of sunshiny days this old gentleman was taken out in a chair on the
terrace---the very chair which Miss Crawley had had at Brighton, and
which had been transported thence with a number of Lady Southdown's
effects to Queen's Crawley.  Lady Jane always walked by the old man,
and was an evident favourite with him.  He used to nod many times to
her and smile when she came in, and utter inarticulate deprecatory
moans when she was going away.  When the door shut upon her he would
cry and sob---whereupon Hester's face and manner, which was always
exceedingly bland and gentle while her lady was present, would change
at once, and she would make faces at him and clench her fist and scream
out ``Hold your tongue, you stoopid old fool,'' and twirl away his chair
from the fire which he loved to look at---at which he would cry more.
For this was all that was left after more than seventy years of
cunning, and struggling, and drinking, and scheming, and sin and
selfishness---a whimpering old idiot put in and out of bed and cleaned
and fed like a baby.

At last a day came when the nurse's occupation was over.  Early one
morning, as Pitt Crawley was at his steward's and bailiff's books in
the study, a knock came to the door, and Hester presented herself,
dropping a curtsey, and said,

``If you please, Sir Pitt, Sir Pitt died this morning, Sir Pitt.  I was
a-making of his toast, Sir Pitt, for his gruel, Sir Pitt, which he took
every morning regular at six, Sir Pitt, and---I thought I heard a
moan-like, Sir Pitt---and---and---and---'' She dropped another curtsey.

What was it that made Pitt's pale face flush quite red? Was it because
he was Sir Pitt at last, with a seat in Parliament, and perhaps future
honours in prospect? ``I'll clear the estate now with the ready money,''
he thought and rapidly calculated its incumbrances and the improvements
which he would make.  He would not use his aunt's money previously lest
Sir Pitt should recover and his outlay be in vain.

All the blinds were pulled down at the Hall and Rectory: the church
bell was tolled, and the chancel hung in black; and Bute Crawley didn't
go to a coursing meeting, but went and dined quietly at Fuddleston,
where they talked about his deceased brother and young Sir Pitt over
their port.  Miss Betsy, who was by this time married to a saddler at
Mudbury, cried a good deal. The family surgeon rode over and paid his
respectful compliments, and inquiries for the health of their
ladyships.  The death was talked about at Mudbury and at the Crawley
Arms, the landlord whereof had become reconciled with the Rector of
late, who was occasionally known to step into the parlour and taste Mr.\ %
Horrocks' mild beer.

``Shall I write to your brother---or will you?'' asked Lady Jane of her
husband, Sir Pitt.

``I will write, of course,'' Sir Pitt said, ``and invite him to the
funeral:  it will be but becoming.''

``And---and---Mrs.\ Rawdon,'' said Lady Jane timidly.

``Jane!'' said Lady Southdown, ``how can you think of such a thing?''

``Mrs.\ Rawdon must of course be asked,'' said Sir Pitt, resolutely.

``Not whilst I am in the house!'' said Lady Southdown.

``Your Ladyship will be pleased to recollect that I am the head of this
family,'' Sir Pitt replied.  ``If you please, Lady Jane, you will write a
letter to Mrs.\ Rawdon Crawley, requesting her presence upon this
melancholy occasion.''

``Jane, I forbid you to put pen to paper!'' cried the Countess.

``I believe I am the head of this family,'' Sir Pitt repeated; ``and
however much I may regret any circumstance which may lead to your
Ladyship quitting this house, must, if you please, continue to govern
it as I see fit.''

Lady Southdown rose up as magnificent as Mrs.\ Siddons in Lady Macbeth
and ordered that horses might be put to her carriage.  If her son and
daughter turned her out of their house, she would hide her sorrows
somewhere in loneliness and pray for their conversion to better
thoughts.

``We don't turn you out of our house, Mamma,'' said the timid Lady Jane
imploringly.

``You invite such company to it as no Christian lady should meet, and I
will have my horses to-morrow morning.''

``Have the goodness to write, Jane, under my dictation,'' said Sir Pitt,
rising and throwing himself into an attitude of command, like the
portrait of a Gentleman in the Exhibition, ``and begin.  'Queen's
Crawley, September 14, 1822.---My dear brother---'''

Hearing these decisive and terrible words, Lady Macbeth, who had been
waiting for a sign of weakness or vacillation on the part of her
son-in-law, rose and, with a scared look, left the library. Lady Jane
looked up to her husband as if she would fain follow and soothe her
mamma, but Pitt forbade his wife to move.

``She won't go away,'' he said.  ``She has let her house at Brighton and
has spent her last half-year's dividends. A Countess living at an inn
is a ruined woman.  I have been waiting long for an opportunity---to
take this---this decisive step, my love; for, as you must perceive, it
is impossible that there should be two chiefs in a family: and now, if
you please, we will resume the dictation.  'My dear brother, the
melancholy intelligence which it is my duty to convey to my family must
have been long anticipated by,''' \&c.

In a word, Pitt having come to his kingdom, and having by good luck, or
desert rather, as he considered, assumed almost all the fortune which
his other relatives had expected, was determined to treat his family
kindly and respectably and make a house of Queen's Crawley once more.
It pleased him to think that he should be its chief.  He proposed to
use the vast influence that his commanding talents and position must
speedily acquire for him in the county to get his brother placed and
his cousins decently provided for, and perhaps had a little sting of
repentance as he thought that he was the proprietor of all that they
had hoped for.  In the course of three or four days' reign his bearing
was changed and his plans quite fixed:  he determined to rule justly
and honestly, to depose Lady Southdown, and to be on the friendliest
possible terms with all the relations of his blood.

So he dictated a letter to his brother Rawdon---a solemn and elaborate
letter, containing the profoundest observations, couched in the longest
words, and filling with wonder the simple little secretary, who wrote
under her husband's order.  ``What an orator this will be,'' thought she,
``when he enters the House of Commons'' (on which point, and on the
tyranny of Lady Southdown, Pitt had sometimes dropped hints to his wife
in bed); ``how wise and good, and what a genius my husband is!  I
fancied him a little cold; but how good, and what a genius!''

The fact is, Pitt Crawley had got every word of the letter by heart and
had studied it, with diplomatic secrecy, deeply and perfectly, long
before he thought fit to communicate it to his astonished wife.

This letter, with a huge black border and seal, was accordingly
despatched by Sir Pitt Crawley to his brother the Colonel, in London.
Rawdon Crawley was but half-pleased at the receipt of it. ``What's the
use of going down to that stupid place?'' thought he.  ``I can't stand
being alone with Pitt after dinner, and horses there and back will cost
us twenty pound.''

He carried the letter, as he did all difficulties, to Becky, upstairs
in her bedroom---with her chocolate, which he always made and took to
her of a morning.

He put the tray with the breakfast and the letter on the dressing-table,
before which Becky sat combing her yellow hair.  She took up the
black-edged missive, and having read it, she jumped up from the chair,
crying ``Hurray!'' and waving the note round her head.

``Hurray?'' said Rawdon, wondering at the little figure capering about in
a streaming flannel dressing-gown, with tawny locks dishevelled. ``He's
not left us anything, Becky.  I had my share when I came of age.''

``You'll never be of age, you silly old man,'' Becky replied.  ``Run out
now to Madam Brunoy's, for I must have some mourning:  and get a crape
on your hat, and a black waistcoat---I don't think you've got one; order
it to be brought home to-morrow, so that we may be able to start on
Thursday.''

``You don't mean to go?'' Rawdon interposed.

``Of course I mean to go.  I mean that Lady Jane shall present me at
Court next year.  I mean that your brother shall give you a seat in
Parliament, you stupid old creature.  I mean that Lord Steyne shall
have your vote and his, my dear, old silly man; and that you shall be
an Irish Secretary, or a West Indian Governor:  or a Treasurer, or a
Consul, or some such thing.''

``Posting will cost a dooce of a lot of money,'' grumbled Rawdon.

``We might take Southdown's carriage, which ought to be present at the
funeral, as he is a relation of the family:  but, no---I intend that we
shall go by the coach. They'll like it better.  It seems more humble---''

``Rawdy goes, of course?'' the Colonel asked.

``No such thing; why pay an extra place? He's too big to travel bodkin
between you and me.  Let him stay here in the nursery, and Briggs can
make him a black frock.  Go you, and do as I bid you. And you had best
tell Sparks, your man, that old Sir Pitt is dead and that you will come
in for something considerable when the affairs are arranged.  He'll
tell this to Raggles, who has been pressing for money, and it will
console poor Raggles.'' And so Becky began sipping her chocolate.

When the faithful Lord Steyne arrived in the evening, he found Becky
and her companion, who was no other than our friend Briggs, busy
cutting, ripping, snipping, and tearing all sorts of black stuffs
available for the melancholy occasion.

``Miss Briggs and I are plunged in grief and despondency for the death
of our Papa,'' Rebecca said.  ``Sir Pitt Crawley is dead, my lord.  We
have been tearing our hair all the morning, and now we are tearing up
our old clothes.''

``Oh, Rebecca, how can you---'' was all that Briggs could say as she
turned up her eyes.

``Oh, Rebecca, how can you---'' echoed my Lord.  ``So that old scoundrel's
dead, is he? He might have been a Peer if he had played his cards
better.  Mr.\ Pitt had very nearly made him; but he ratted always at the
wrong time.  What an old Silenus it was!''

``I might have been Silenus's widow,'' said Rebecca. ``Don't you remember,
Miss Briggs, how you peeped in at the door and saw old Sir Pitt on his
knees to me?'' Miss Briggs, our old friend, blushed very much at this
reminiscence, and was glad when Lord Steyne ordered her to go
downstairs and make him a cup of tea.

Briggs was the house-dog whom Rebecca had provided as guardian of her
innocence and reputation.  Miss Crawley had left her a little annuity.
She would have been content to remain in the Crawley family with Lady
Jane, who was good to her and to everybody; but Lady Southdown
dismissed poor Briggs as quickly as decency permitted; and Mr.\ Pitt
(who thought himself much injured by the uncalled-for generosity of his
deceased relative towards a lady who had only been Miss Crawley's
faithful retainer a score of years) made no objection to that exercise
of the dowager's authority. Bowls and Firkin likewise received their
legacies and their dismissals, and married and set up a lodging-house,
according to the custom of their kind.

Briggs tried to live with her relations in the country, but found that
attempt was vain after the better society to which she had been
accustomed.  Briggs's friends, small tradesmen, in a country town,
quarrelled over Miss Briggs's forty pounds a year as eagerly and more
openly than Miss Crawley's kinsfolk had for that lady's inheritance.
Briggs's brother, a radical hatter and grocer, called his sister a
purse-proud aristocrat, because she would not advance a part of her
capital to stock his shop; and she would have done so most likely, but
that their sister, a dissenting shoemaker's lady, at variance with the
hatter and grocer, who went to another chapel, showed how their brother
was on the verge of bankruptcy, and took possession of Briggs for a
while.  The dissenting shoemaker wanted Miss Briggs to send his son to
college and make a gentleman of him. Between them the two families got
a great portion of her private savings out of her, and finally she fled
to London followed by the anathemas of both, and determined to seek for
servitude again as infinitely less onerous than liberty.  And
advertising in the papers that a ``Gentlewoman of agreeable manners, and
accustomed to the best society, was anxious to,'' \&c., she took up her
residence with Mr.\ Bowls in Half Moon Street, and waited the result of
the advertisement.

So it was that she fell in with Rebecca.  Mrs.\ Rawdon's dashing little
carriage and ponies was whirling down the street one day, just as Miss
Briggs, fatigued, had reached Mr.\ Bowls's door, after a weary walk to
the Times Office in the City to insert her advertisement for the sixth
time.  Rebecca was driving, and at once recognized the gentlewoman with
agreeable manners, and being a perfectly good-humoured woman, as we
have seen, and having a regard for Briggs, she pulled up the ponies at
the doorsteps, gave the reins to the groom, and jumping out, had hold
of both Briggs's hands, before she of the agreeable manners had
recovered from the shock of seeing an old friend.

Briggs cried, and Becky laughed a great deal and kissed the gentlewoman
as soon as they got into the passage; and thence into Mrs.\ Bowls's
front parlour, with the red moreen curtains, and the round
looking-glass, with the chained eagle above, gazing upon the back of
the ticket in the window which announced ``Apartments to Let.''

Briggs told all her history amidst those perfectly uncalled-for sobs
and ejaculations of wonder with which women of her soft nature salute
an old acquaintance, or regard a rencontre in the street; for though
people meet other people every day, yet some there are who insist upon
discovering miracles; and women, even though they have disliked each
other, begin to cry when they meet, deploring and remembering the time
when they last quarrelled.  So, in a word, Briggs told all her history,
and Becky gave a narrative of her own life, with her usual artlessness
and candour.

Mrs.\ Bowls, late Firkin, came and listened grimly in the passage to the
hysterical sniffling and giggling which went on in the front parlour.
Becky had never been a favourite of hers.  Since the establishment of
the married couple in London they had frequented their former friends
of the house of Raggles, and did not like the latter's account of the
Colonel's menage.  ``I wouldn't trust him, Ragg, my boy,'' Bowls
remarked; and his wife, when Mrs.\ Rawdon issued from the parlour, only
saluted the lady with a very sour curtsey; and her fingers were like so
many sausages, cold and lifeless, when she held them out in deference
to Mrs.\ Rawdon, who persisted in shaking hands with the retired lady's
maid.  She whirled away into Piccadilly, nodding with the sweetest of
smiles towards Miss Briggs, who hung nodding at the window close under
the advertisement-card, and at the next moment was in the park with a
half-dozen of dandies cantering after her carriage.

When she found how her friend was situated, and how having a snug
legacy from Miss Crawley, salary was no object to our gentlewoman,
Becky instantly formed some benevolent little domestic plans concerning
her.  This was just such a companion as would suit her establishment,
and she invited Briggs to come to dinner with her that very evening,
when she should see Becky's dear little darling Rawdon.

Mrs.\ Bowls cautioned her lodger against venturing into the lion's den,
``wherein you will rue it, Miss B., mark my words, and as sure as my
name is Bowls.'' And Briggs promised to be very cautious.  The upshot of
which caution was that she went to live with Mrs.\ Rawdon the next week,
and had lent Rawdon Crawley six hundred pounds upon annuity before six
months were over.



\chapter{In Which Becky Revisits the Halls of Her Ancestors}

So the mourning being ready, and Sir Pitt Crawley warned of their
arrival, Colonel Crawley and his wife took a couple of places in the
same old High-flyer coach by which Rebecca had travelled in the defunct
Baronet's company, on her first journey into the world some nine years
before.  How well she remembered the Inn Yard, and the ostler to whom
she refused money, and the insinuating Cambridge lad who wrapped her in
his coat on the journey!  Rawdon took his place outside, and would have
liked to drive, but his grief forbade him. He sat by the coachman and
talked about horses and the road the whole way; and who kept the inns,
and who horsed the coach by which he had travelled so many a time, when
he and Pitt were boys going to Eton.  At Mudbury a carriage and a pair
of horses received them, with a coachman in black.  ``It's the old drag,
Rawdon,'' Rebecca said as they got in.  ``The worms have eaten the cloth
a good deal---there's the stain which Sir Pitt---ha!  I see Dawson the
Ironmonger has his shutters up---which Sir Pitt made such a noise about.
It was a bottle of cherry brandy he broke which we went to fetch for
your aunt from Southampton.  How time flies, to be sure!  That can't be
Polly Talboys, that bouncing girl standing by her mother at the cottage
there.  I remember her a mangy little urchin picking weeds in the
garden.''

``Fine gal,'' said Rawdon, returning the salute which the cottage gave
him, by two fingers applied to his crape hatband.  Becky bowed and
saluted, and recognized people here and there graciously.  These
recognitions were inexpressibly pleasant to her.  It seemed as if she
was not an imposter any more, and was coming to the home of her
ancestors.  Rawdon was rather abashed and cast down, on the other hand.
What recollections of boyhood and innocence might have been flitting
across his brain? What pangs of dim remorse and doubt and shame?

``Your sisters must be young women now,'' Rebecca said, thinking of those
girls for the first time perhaps since she had left them.

``Don't know, I'm shaw,'' replied the Colonel.  ``Hullo! here's old Mother
Lock.  How-dy-do, Mrs.\ Lock? Remember me, don't you? Master Rawdon,
hey? Dammy how those old women last; she was a hundred when I was a
boy.''

They were going through the lodge-gates kept by old Mrs.\ Lock, whose
hand Rebecca insisted upon shaking, as she flung open the creaking old
iron gate, and the carriage passed between the two moss-grown pillars
surmounted by the dove and serpent.

``The governor has cut into the timber,'' Rawdon said, looking about, and
then was silent---so was Becky.  Both of them were rather agitated, and
thinking of old times. He about Eton, and his mother, whom he
remembered, a frigid demure woman, and a sister who died, of whom he
had been passionately fond; and how he used to thrash Pitt; and about
little Rawdy at home.  And Rebecca thought about her own youth and the
dark secrets of those early tainted days; and of her entrance into life
by yonder gates; and of Miss Pinkerton, and Joe, and Amelia.

The gravel walk and terrace had been scraped quite clean.  A grand
painted hatchment was already over the great entrance, and two very
solemn and tall personages in black flung open each a leaf of the door
as the carriage pulled up at the familiar steps.  Rawdon turned red,
and Becky somewhat pale, as they passed through the old hall, arm in
arm.  She pinched her husband's arm as they entered the oak parlour,
where Sir Pitt and his wife were ready to receive them. Sir Pitt in
black, Lady Jane in black, and my Lady Southdown with a large black
head-piece of bugles and feathers, which waved on her Ladyship's head
like an undertaker's tray.

Sir Pitt had judged correctly, that she would not quit the premises.
She contented herself by preserving a solemn and stony silence, when in
company of Pitt and his rebellious wife, and by frightening the
children in the nursery by the ghastly gloom of her demeanour. Only a
very faint bending of the head-dress and plumes welcomed Rawdon and his
wife, as those prodigals returned to their family.

To say the truth, they were not affected very much one way or other by
this coolness.  Her Ladyship was a person only of secondary
consideration in their minds just then---they were intent upon the
reception which the reigning brother and sister would afford them.

Pitt, with rather a heightened colour, went up and shook his brother by
the hand, and saluted Rebecca with a hand-shake and a very low bow.
But Lady Jane took both the hands of her sister-in-law and kissed her
affectionately. The embrace somehow brought tears into the eyes of the
little adventuress---which ornaments, as we know, she wore very seldom.
The artless mark of kindness and confidence touched and pleased her;
and Rawdon, encouraged by this demonstration on his sister's part,
twirled up his mustachios and took leave to salute Lady Jane with a
kiss, which caused her Ladyship to blush exceedingly.

``Dev'lish nice little woman, Lady Jane,'' was his verdict, when he and
his wife were together again.  ``Pitt's got fat, too, and is doing the
thing handsomely.'' ``He can afford it,'' said Rebecca and agreed in her
husband's farther opinion ``that the mother-in-law was a tremendous old
Guy---and that the sisters were rather well-looking young women.''

They, too, had been summoned from school to attend the funeral
ceremonies.  It seemed Sir Pitt Crawley, for the dignity of the house
and family, had thought right to have about the place as many persons
in black as could possibly be assembled.  All the men and maids of the
house, the old women of the Alms House, whom the elder Sir Pitt had
cheated out of a great portion of their due, the parish clerk's family,
and the special retainers of both Hall and Rectory were habited in
sable; added to these, the undertaker's men, at least a score, with
crapes and hatbands, and who made goodly show when the great burying
show took place---but these are mute personages in our drama; and having
nothing to do or say, need occupy a very little space here.

With regard to her sisters-in-law Rebecca did not attempt to forget her
former position of Governess towards them, but recalled it frankly and
kindly, and asked them about their studies with great gravity, and told
them that she had thought of them many and many a day, and longed to
know of their welfare.  In fact you would have supposed that ever since
she had left them she had not ceased to keep them uppermost in her
thoughts and to take the tenderest interest in their welfare.  So
supposed Lady Crawley herself and her young sisters.

``She's hardly changed since eight years,'' said Miss Rosalind to Miss
Violet, as they were preparing for dinner.

``Those red-haired women look wonderfully well,'' replied the other.

``Hers is much darker than it was; I think she must dye it,'' Miss
Rosalind added.  ``She is stouter, too, and altogether improved,''
continued Miss Rosalind, who was disposed to be very fat.

``At least she gives herself no airs and remembers that she was our
Governess once,'' Miss Violet said, intimating that it befitted all
governesses to keep their proper place, and forgetting altogether that
she was granddaughter not only of Sir Walpole Crawley, but of Mr.\ %
Dawson of Mudbury, and so had a coal-scuttle in her scutcheon. There
are other very well-meaning people whom one meets every day in Vanity
Fair who are surely equally oblivious.

``It can't be true what the girls at the Rectory said, that her mother
was an opera-dancer---''

``A person can't help their birth,'' Rosalind replied with great
liberality.  ``And I agree with our brother, that as she is in the
family, of course we are bound to notice her. I am sure Aunt Bute need
not talk; she wants to marry Kate to young Hooper, the wine-merchant,
and absolutely asked him to come to the Rectory for orders.''

``I wonder whether Lady Southdown will go away, she looked very glum
upon Mrs.\ Rawdon,'' the other said.

``I wish she would.  I won't read the Washerwoman of Finchley Common,''
vowed Violet; and so saying, and avoiding a passage at the end of which
a certain coffin was placed with a couple of watchers, and lights
perpetually burning in the closed room, these young women came down to
the family dinner, for which the bell rang as usual.

But before this, Lady Jane conducted Rebecca to the apartments prepared
for her, which, with the rest of the house, had assumed a very much
improved appearance of order and comfort during Pitt's regency, and
here beholding that Mrs.\ Rawdon's modest little trunks had arrived, and
were placed in the bedroom and dressing-room adjoining, helped her to
take off her neat black bonnet and cloak, and asked her sister-in-law
in what more she could be useful.

``What I should like best,'' said Rebecca, ``would be to go to the nursery
and see your dear little children.'' On which the two ladies looked very
kindly at each other and went to that apartment hand in hand.

Becky admired little Matilda, who was not quite four years old, as the
most charming little love in the world; and the boy, a little fellow of
two years---pale, heavy-eyed, and large-headed---she pronounced to be a
perfect prodigy in point of size, intelligence, and beauty.

``I wish Mamma would not insist on giving him so much medicine,'' Lady
Jane said with a sigh.  ``I often think we should all be better without
it.'' And then Lady Jane and her new-found friend had one of those
confidential medical conversations about the children, which all
mothers, and most women, as I am given to understand, delight in. Fifty
years ago, and when the present writer, being an interesting little
boy, was ordered out of the room with the ladies after dinner, I
remember quite well that their talk was chiefly about their ailments;
and putting this question directly to two or three since, I have always
got from them the acknowledgement that times are not changed.  Let my
fair readers remark for themselves this very evening when they quit the
dessert-table and assemble to celebrate the drawing-room mysteries.
Well---in half an hour Becky and Lady Jane were close and intimate
friends---and in the course of the evening her Ladyship informed Sir
Pitt that she thought her new sister-in-law was a kind, frank,
unaffected, and affectionate young woman.

And so having easily won the daughter's good-will, the indefatigable
little woman bent herself to conciliate the august Lady Southdown. As
soon as she found her Ladyship alone, Rebecca attacked her on the
nursery question at once and said that her own little boy was saved,
actually saved, by calomel, freely administered, when all the
physicians in Paris had given the dear child up.  And then she
mentioned how often she had heard of Lady Southdown from that excellent
man the Reverend Lawrence Grills, Minister of the chapel in May Fair,
which she frequented; and how her views were very much changed by
circumstances and misfortunes; and how she hoped that a past life spent
in worldliness and error might not incapacitate her from more serious
thought for the future. She described how in former days she had been
indebted to Mr.\ Crawley for religious instruction, touched upon the
Washerwoman of Finchley Common, which she had read with the greatest
profit, and asked about Lady Emily, its gifted author, now Lady Emily
Hornblower, at Cape Town, where her husband had strong hopes of
becoming Bishop of Caffraria.

But she crowned all, and confirmed herself in Lady Southdown's favour,
by feeling very much agitated and unwell after the funeral and
requesting her Ladyship's medical advice, which the Dowager not only
gave, but, wrapped up in a bed-gown and looking more like Lady Macbeth
than ever, came privately in the night to Becky's room with a parcel of
favourite tracts, and a medicine of her own composition, which she
insisted that Mrs.\ Rawdon should take.

Becky first accepted the tracts and began to examine them with great
interest, engaging the Dowager in a conversation concerning them and
the welfare of her soul, by which means she hoped that her body might
escape medication.  But after the religious topics were exhausted, Lady
Macbeth would not quit Becky's chamber until her cup of night-drink was
emptied too; and poor Mrs.\ Rawdon was compelled actually to assume a
look of gratitude, and to swallow the medicine under the unyielding old
Dowager's nose, who left her victim finally with a benediction.

It did not much comfort Mrs.\ Rawdon; her countenance was very queer
when Rawdon came in and heard what had happened; and his explosions
of laughter were as loud as usual, when Becky, with a fun which she
could not disguise, even though it was at her own expense, described
the occurrence and how she had been victimized by Lady Southdown.  Lord
Steyne, and her son in London, had many a laugh over the story when
Rawdon and his wife returned to their quarters in May Fair.  Becky
acted the whole scene for them.  She put on a night-cap and gown.  She
preached a great sermon in the true serious manner; she lectured on the
virtue of the medicine which she pretended to administer, with a
gravity of imitation so perfect that you would have thought it was the
Countess's own Roman nose through which she snuffled. ``Give us Lady
Southdown and the black dose,'' was a constant cry amongst the folks in
Becky's little drawing-room in May Fair.  And for the first time in her
life the Dowager Countess of Southdown was made amusing.

Sir Pitt remembered the testimonies of respect and veneration which
Rebecca had paid personally to himself in early days, and was tolerably
well disposed towards her.  The marriage, ill-advised as it was, had
improved Rawdon very much---that was clear from the Colonel's altered
habits and demeanour---and had it not been a lucky union as regarded
Pitt himself? The cunning diplomatist smiled inwardly as he owned that
he owed his fortune to it, and acknowledged that he at least ought not
to cry out against it.  His satisfaction was not removed by Rebecca's
own statements, behaviour, and conversation.

She doubled the deference which before had charmed him, calling out his
conversational powers in such a manner as quite to surprise Pitt
himself, who, always inclined to respect his own talents, admired them
the more when Rebecca pointed them out to him.  With her sister-in-law,
Rebecca was satisfactorily able to prove that it was Mrs.\ Bute Crawley
who brought about the marriage which she afterwards so calumniated;
that it was Mrs.\ Bute's avarice---who hoped to gain all Miss Crawley's
fortune and deprive Rawdon of his aunt's favour---which caused and
invented all the wicked reports against Rebecca.  ``She succeeded in
making us poor,'' Rebecca said with an air of angelical patience; ``but
how can I be angry with a woman who has given me one of the best
husbands in the world? And has not her own avarice been sufficiently
punished by the ruin of her own hopes and the loss of the property by
which she set so much store? Poor!'' she cried.  ``Dear Lady Jane, what
care we for poverty? I am used to it from childhood, and I am often
thankful that Miss Crawley's money has gone to restore the splendour of
the noble old family of which I am so proud to be a member.  I am sure
Sir Pitt will make a much better use of it than Rawdon would.''

All these speeches were reported to Sir Pitt by the most faithful of
wives, and increased the favourable impression which Rebecca made; so
much so that when, on the third day after the funeral, the family party
were at dinner, Sir Pitt Crawley, carving fowls at the head of the
table, actually said to Mrs.\ Rawdon, ``Ahem!  Rebecca, may I give you a
wing?''---a speech which made the little woman's eyes sparkle with
pleasure.

While Rebecca was prosecuting the above schemes and hopes, and Pitt
Crawley arranging the funeral ceremonial and other matters connected
with his future progress and dignity, and Lady Jane busy with her
nursery, as far as her mother would let her, and the sun rising and
setting, and the clock-tower bell of the Hall ringing to dinner and to
prayers as usual, the body of the late owner of Queen's Crawley lay in
the apartment which he had occupied, watched unceasingly by the
professional attendants who were engaged for that rite.  A woman or
two, and three or four undertaker's men, the best whom Southampton
could furnish, dressed in black, and of a proper stealthy and tragical
demeanour, had charge of the remains which they watched turn about,
having the housekeeper's room for their place of rendezvous when off
duty, where they played at cards in privacy and drank their beer.

The members of the family and servants of the house kept away from the
gloomy spot, where the bones of the descendant of an ancient line of
knights and gentlemen lay, awaiting their final consignment to the
family crypt. No regrets attended them, save those of the poor woman
who had hoped to be Sir Pitt's wife and widow and who had fled in
disgrace from the Hall over which she had so nearly been a ruler.
Beyond her and a favourite old pointer he had, and between whom and
himself an attachment subsisted during the period of his imbecility,
the old man had not a single friend to mourn him, having indeed, during
the whole course of his life, never taken the least pains to secure
one.  Could the best and kindest of us who depart from the earth have
an opportunity of revisiting it, I suppose he or she (assuming that any
Vanity Fair feelings subsist in the sphere whither we are bound) would
have a pang of mortification at finding how soon our survivors were
consoled.  And so Sir Pitt was forgotten---like the kindest and best of
us---only a few weeks sooner.

Those who will may follow his remains to the grave, whither they were
borne on the appointed day, in the most becoming manner, the family in
black coaches, with their handkerchiefs up to their noses, ready for
the tears which did not come; the undertaker and his gentlemen in deep
tribulation; the select tenantry mourning out of compliment to the new
landlord; the neighbouring gentry's carriages at three miles an hour,
empty, and in profound affliction; the parson speaking out the formula
about ``our dear brother departed.'' As long as we have a man's body, we
play our Vanities upon it, surrounding it with humbug and ceremonies,
laying it in state, and packing it up in gilt nails and velvet; and we
finish our duty by placing over it a stone, written all over with lies.
Bute's curate, a smart young fellow from Oxford, and Sir Pitt Crawley
composed between them an appropriate Latin epitaph for the late
lamented Baronet, and the former preached a classical sermon, exhorting
the survivors not to give way to grief and informing them in the most
respectful terms that they also would be one day called upon to pass
that gloomy and mysterious portal which had just closed upon the
remains of their lamented brother. Then the tenantry mounted on
horseback again, or stayed and refreshed themselves at the Crawley
Arms.  Then, after a lunch in the servants' hall at Queen's Crawley,
the gentry's carriages wheeled off to their different destinations:
then the undertaker's men, taking the ropes, palls, velvets, ostrich
feathers, and other mortuary properties, clambered up on the roof of
the hearse and rode off to Southampton.  Their faces relapsed into a
natural expression as the horses, clearing the lodge-gates, got into a
brisker trot on the open road; and squads of them might have been seen,
speckling with black the public-house entrances, with pewter-pots
flashing in the sunshine.  Sir Pitt's invalid chair was wheeled away
into a tool-house in the garden; the old pointer used to howl sometimes
at first, but these were the only accents of grief which were heard in
the Hall of which Sir Pitt Crawley, Baronet, had been master for some
threescore years.

As the birds were pretty plentiful, and partridge shooting is as it
were the duty of an English gentleman of statesmanlike propensities,
Sir Pitt Crawley, the first shock of grief over, went out a little and
partook of that diversion in a white hat with crape round it. The sight
of those fields of stubble and turnips, now his own, gave him many
secret joys.  Sometimes, and with an exquisite humility, he took no
gun, but went out with a peaceful bamboo cane; Rawdon, his big brother,
and the keepers blazing away at his side.  Pitt's money and acres had a
great effect upon his brother.  The penniless Colonel became quite
obsequious and respectful to the head of his house, and despised the
milksop Pitt no longer.  Rawdon listened with sympathy to his senior's
prospects of planting and draining, gave his advice about the stables
and cattle, rode over to Mudbury to look at a mare, which he thought
would carry Lady Jane, and offered to break her, \&c.:  the rebellious
dragoon was quite humbled and subdued, and became a most creditable
younger brother.  He had constant bulletins from Miss Briggs in London
respecting little Rawdon, who was left behind there, who sent messages
of his own.  ``I am very well,'' he wrote.  ``I hope you are very well.  I
hope Mamma is very well.  The pony is very well.  Grey takes me to ride
in the park. I can canter.  I met the little boy who rode before.  He
cried when he cantered.  I do not cry.'' Rawdon read these letters to
his brother and Lady Jane, who was delighted with them.  The Baronet
promised to take charge of the lad at school, and his kind-hearted wife
gave Rebecca a bank-note, begging her to buy a present with it for her
little nephew.

One day followed another, and the ladies of the house passed their life
in those calm pursuits and amusements which satisfy country ladies.
Bells rang to meals and to prayers.  The young ladies took exercise on
the pianoforte every morning after breakfast, Rebecca giving them the
benefit of her instruction.  Then they put on thick shoes and walked in
the park or shrubberies, or beyond the palings into the village,
descending upon the cottages, with Lady Southdown's medicine and tracts
for the sick people there.  Lady Southdown drove out in a pony-chaise,
when Rebecca would take her place by the Dowager's side and listen to
her solemn talk with the utmost interest.  She sang Handel and Haydn to
the family of evenings, and engaged in a large piece of worsted work,
as if she had been born to the business and as if this kind of life was
to continue with her until she should sink to the grave in a polite old
age, leaving regrets and a great quantity of consols behind her---as if
there were not cares and duns, schemes, shifts, and poverty waiting
outside the park gates, to pounce upon her when she issued into the
world again.

``It isn't difficult to be a country gentleman's wife,'' Rebecca thought.
``I think I could be a good woman if I had five thousand a year.  I
could dawdle about in the nursery and count the apricots on the wall.
I could water plants in a green-house and pick off dead leaves from the
geraniums.  I could ask old women about their rheumatisms and order
half-a-crown's worth of soup for the poor.  I shouldn't miss it much,
out of five thousand a year.  I could even drive out ten miles to dine
at a neighbour's, and dress in the fashions of the year before last. I
could go to church and keep awake in the great family pew, or go to
sleep behind the curtains, with my veil down, if I only had practice.
I could pay everybody, if I had but the money.  This is what the
conjurors here pride themselves upon doing.  They look down with pity
upon us miserable sinners who have none.  They think themselves
generous if they give our children a five-pound note, and us
contemptible if we are without one.'' And who knows but Rebecca was
right in her speculations---and that it was only a question of money and
fortune which made the difference between her and an honest woman? If
you take temptations into account, who is to say that he is better than
his neighbour? A comfortable career of prosperity, if it does not make
people honest, at least keeps them so.  An alderman coming from a
turtle feast will not step out of his carriage to steal a leg of mutton;
but put him to starve, and see if he will not purloin a loaf.  Becky
consoled herself by so balancing the chances and equalizing the
distribution of good and evil in the world.

The old haunts, the old fields and woods, the copses, ponds, and
gardens, the rooms of the old house where she had spent a couple of
years seven years ago, were all carefully revisited by her.  She had
been young there, or comparatively so, for she forgot the time when she
ever WAS young---but she remembered her thoughts and feelings seven
years back and contrasted them with those which she had at present, now
that she had seen the world, and lived with great people, and raised
herself far beyond her original humble station.

``I have passed beyond it, because I have brains,'' Becky thought, ``and
almost all the rest of the world are fools. I could not go back and
consort with those people now, whom I used to meet in my father's
studio.  Lords come up to my door with stars and garters, instead of
poor artists with screws of tobacco in their pockets.  I have a
gentleman for my husband, and an Earl's daughter for my sister, in the
very house where I was little better than a servant a few years ago.
But am I much better to do now in the world than I was when I was the
poor painter's daughter and wheedled the grocer round the corner for
sugar and tea? Suppose I had married Francis who was so fond of me---I
couldn't have been much poorer than I am now.  Heigho!  I wish I could
exchange my position in society, and all my relations for a snug sum in
the Three Per Cent.  Consols''; for so it was that Becky felt the Vanity
of human affairs, and it was in those securities that she would have
liked to cast anchor.

It may, perhaps, have struck her that to have been honest and humble,
to have done her duty, and to have marched straightforward on her way,
would have brought her as near happiness as that path by which she was
striving to attain it.  But---just as the children at Queen's Crawley
went round the room where the body of their father lay---if ever Becky
had these thoughts, she was accustomed to walk round them and not look
in.  She eluded them and despised them---or at least she was committed
to the other path from which retreat was now impossible.  And for my
part I believe that remorse is the least active of all a man's moral
senses---the very easiest to be deadened when wakened, and in some never
wakened at all.  We grieve at being found out and at the idea of shame
or punishment, but the mere sense of wrong makes very few people
unhappy in Vanity Fair.

So Rebecca, during her stay at Queen's Crawley, made as many friends of
the Mammon of Unrighteousness as she could possibly bring under
control.  Lady Jane and her husband bade her farewell with the warmest
demonstrations of good-will.  They looked forward with pleasure to the
time when, the family house in Gaunt Street being repaired and
beautified, they were to meet again in London.  Lady Southdown made her
up a packet of medicine and sent a letter by her to the Rev. Lawrence
Grills, exhorting that gentleman to save the brand who ``honoured'' the
letter from the burning.  Pitt accompanied them with four horses in the
carriage to Mudbury, having sent on their baggage in a cart previously,
accompanied with loads of game.

``How happy you will be to see your darling little boy again!'' Lady
Crawley said, taking leave of her kinswoman.

``Oh so happy!'' said Rebecca, throwing up the green eyes. She was
immensely happy to be free of the place, and yet loath to go. Queen's
Crawley was abominably stupid, and yet the air there was somehow purer
than that which she had been accustomed to breathe. Everybody had been
dull, but had been kind in their way.  ``It is all the influence of a
long course of Three Per Cents,'' Becky said to herself, and was right
very likely.

However, the London lamps flashed joyfully as the stage rolled into
Piccadilly, and Briggs had made a beautiful fire in Curzon Street, and
little Rawdon was up to welcome back his papa and mamma.



\chapter{Which Treats of the Osborne Family}

Considerable time has elapsed since we have seen our respectable
friend, old Mr.\ Osborne of Russell Square.  He has not been the
happiest of mortals since last we met him. Events have occurred which
have not improved his temper, and in more instances than one he has
not been allowed to have his own way.  To be thwarted in this
reasonable desire was always very injurious to the old gentleman; and
resistance became doubly exasperating when gout, age, loneliness, and
the force of many disappointments combined to weigh him down.  His
stiff black hair began to grow quite white soon after his son's death;
his face grew redder; his hands trembled more and more as he poured out
his glass of port wine.  He led his clerks a dire life in the City:
his family at home were not much happier.  I doubt if Rebecca, whom we
have seen piously praying for Consols, would have exchanged her poverty
and the dare-devil excitement and chances of her life for Osborne's
money and the humdrum gloom which enveloped him.  He had proposed for
Miss Swartz, but had been rejected scornfully by the partisans of that
lady, who married her to a young sprig of Scotch nobility.  He was a
man to have married a woman out of low life and bullied her dreadfully
afterwards; but no person presented herself suitable to his taste, and,
instead, he tyrannized over his unmarried daughter, at home.  She had a
fine carriage and fine horses and sat at the head of a table loaded
with the grandest plate.  She had a cheque-book, a prize footman to
follow her when she walked, unlimited credit, and bows and compliments
from all the tradesmen, and all the appurtenances of an heiress; but
she spent a woeful time. The little charity-girls at the Foundling, the
sweeperess at the crossing, the poorest under-kitchen-maid in the
servants' hall, was happy compared to that unfortunate and now
middle-aged young lady.

Frederick Bullock, Esq., of the house of Bullock, Hulker, and Bullock,
had married Maria Osborne, not without a great deal of difficulty and
grumbling on Mr.\ Bullock's part. George being dead and cut out of his
father's will, Frederick insisted that the half of the old gentleman's
property should be settled upon his Maria, and indeed, for a long time,
refused, ``to come to the scratch'' (it was Mr.\ Frederick's own
expression) on any other terms.  Osborne said Fred had agreed to take
his daughter with twenty thousand, and he should bind himself to no
more.  ``Fred might take it, and welcome, or leave it, and go and be
hanged.'' Fred, whose hopes had been raised when George had been
disinherited, thought himself infamously swindled by the old merchant,
and for some time made as if he would break off the match altogether.
Osborne withdrew his account from Bullock and Hulker's, went on 'Change
with a horsewhip which he swore he would lay across the back of a
certain scoundrel that should be nameless, and demeaned himself in his
usual violent manner.  Jane Osborne condoled with her sister Maria
during this family feud.  ``I always told you, Maria, that it was your
money he loved and not you,'' she said, soothingly.

``He selected me and my money at any rate; he didn't choose you and
yours,'' replied Maria, tossing up her head.

The rapture was, however, only temporary.  Fred's father and senior
partners counselled him to take Maria, even with the twenty thousand
settled, half down, and half at the death of Mr.\ Osborne, with the
chances of the further division of the property.  So he ``knuckled
down,'' again to use his own phrase, and sent old Hulker with peaceable
overtures to Osborne.  It was his father, he said, who would not hear
of the match, and had made the difficulties; he was most anxious to
keep the engagement.  The excuse was sulkily accepted by Mr.\ Osborne.
Hulker and Bullock were a high family of the City aristocracy, and
connected with the ``nobs'' at the West End. It was something for the old
man to be able to say, ``My son, sir, of the house of Hulker, Bullock,
and Co., sir; my daughter's cousin, Lady Mary Mango, sir, daughter of
the Right Hon.  The Earl of Castlemouldy.'' In his imagination he saw
his house peopled by the ``nobs.'' So he forgave young Bullock and
consented that the marriage should take place.

It was a grand affair---the bridegroom's relatives giving the breakfast,
their habitations being near St.\ George's, Hanover Square, where the
business took place.  The ``nobs of the West End'' were invited, and many
of them signed the book.  Mr.\ Mango and Lady Mary Mango were there,
with the dear young Gwendoline and Guinever Mango as bridesmaids;
Colonel Bludyer of the Dragoon Guards (eldest son of the house of
Bludyer Brothers, Mincing Lane), another cousin of the bridegroom, and
the Honourable Mrs.\ Bludyer; the Honourable George Boulter, Lord
Levant's son, and his lady, Miss Mango that was; Lord Viscount
Castletoddy; Honourable James McMull and Mrs.\ McMull (formerly Miss
Swartz); and a host of fashionables, who have all married into Lombard
Street and done a great deal to ennoble Cornhill.

The young couple had a house near Berkeley Square and a small villa at
Roehampton, among the banking colony there.  Fred was considered to
have made rather a mesalliance by the ladies of his family, whose
grandfather had been in a Charity School, and who were allied through
the husbands with some of the best blood in England.  And Maria was
bound, by superior pride and great care in the composition of her
visiting-book, to make up for the defects of birth, and felt it her
duty to see her father and sister as little as possible.

That she should utterly break with the old man, who had still so many
scores of thousand pounds to give away, is absurd to suppose. Fred
Bullock would never allow her to do that.  But she was still young and
incapable of hiding her feelings; and by inviting her papa and sister
to her third-rate parties, and behaving very coldly to them when they
came, and by avoiding Russell Square, and indiscreetly begging her
father to quit that odious vulgar place, she did more harm than all
Frederick's diplomacy could repair, and perilled her chance of her
inheritance like a giddy heedless creature as she was.

``So Russell Square is not good enough for Mrs.\ Maria, hay?'' said the
old gentleman, rattling up the carriage windows as he and his daughter
drove away one night from Mrs.\ Frederick Bullock's, after dinner.  ``So
she invites her father and sister to a second day's dinner (if those
sides, or ontrys, as she calls 'em, weren't served yesterday, I'm
d---d), and to meet City folks and littery men, and keeps the Earls and
the Ladies, and the Honourables to herself. Honourables? Damn
Honourables.  I am a plain British merchant I am, and could buy the
beggarly hounds over and over.  Lords, indeed!---why, at one of her
swarreys I saw one of 'em speak to a dam fiddler---a fellar I despise.
And they won't come to Russell Square, won't they? Why, I'll lay my
life I've got a better glass of wine, and pay a better figure for it,
and can show a handsomer service of silver, and can lay a better dinner
on my mahogany, than ever they see on theirs---the cringing, sneaking,
stuck-up fools.  Drive on quick, James:  I want to get back to Russell
Square---ha, ha!'' and he sank back into the corner with a furious laugh.
With such reflections on his own superior merit, it was the custom of
the old gentleman not unfrequently to console himself.

Jane Osborne could not but concur in these opinions respecting her
sister's conduct; and when Mrs.\ Frederick's first-born, Frederick
Augustus Howard Stanley Devereux Bullock, was born, old Osborne, who
was invited to the christening and to be godfather, contented himself
with sending the child a gold cup, with twenty guineas inside it for
the nurse.  ``That's more than any of your Lords will give, I'LL
warrant,'' he said and refused to attend at the ceremony.

The splendour of the gift, however, caused great satisfaction to the
house of Bullock.  Maria thought that her father was very much pleased
with her, and Frederick augured the best for his little son and heir.

One can fancy the pangs with which Miss Osborne in her solitude in
Russell Square read the Morning Post, where her sister's name occurred
every now and then, in the articles headed ``Fashionable Reunions,'' and
where she had an opportunity of reading a description of Mrs.\ F.
Bullock's costume, when presented at the drawing room by Lady Frederica
Bullock.  Jane's own life, as we have said, admitted of no such
grandeur.  It was an awful existence. She had to get up of black
winter's mornings to make breakfast for her scowling old father, who
would have turned the whole house out of doors if his tea had not been
ready at half-past eight.  She remained silent opposite to him,
listening to the urn hissing, and sitting in tremor while the parent
read his paper and consumed his accustomed portion of muffins and tea.
At half-past nine he rose and went to the City, and she was almost free
till dinner-time, to make visitations in the kitchen and to scold the
servants; to drive abroad and descend upon the tradesmen, who were
prodigiously respectful; to leave her cards and her papa's at the great
glum respectable houses of their City friends; or to sit alone in the
large drawing-room, expecting visitors; and working at a huge piece of
worsted by the fire, on the sofa, hard by the great Iphigenia clock,
which ticked and tolled with mournful loudness in the dreary room.  The
great glass over the mantelpiece, faced by the other great console
glass at the opposite end of the room, increased and multiplied between
them the brown Holland bag in which the chandelier hung, until you saw
these brown Holland bags fading away in endless perspectives, and this
apartment of Miss Osborne's seemed the centre of a system of
drawing-rooms. When she removed the cordovan leather from the grand
piano and ventured to play a few notes on it, it sounded with a
mournful sadness, startling the dismal echoes of the house.  George's
picture was gone, and laid upstairs in a lumber-room in the garret; and
though there was a consciousness of him, and father and daughter often
instinctively knew that they were thinking of him, no mention was ever
made of the brave and once darling son.

At five o'clock Mr.\ Osborne came back to his dinner, which he and his
daughter took in silence (seldom broken, except when he swore and was
savage, if the cooking was not to his liking), or which they shared
twice in a month with a party of dismal friends of Osborne's rank and
age.  Old Dr. Gulp and his lady from Bloomsbury Square; old Mr.\ %
Frowser, the attorney, from Bedford Row, a very great man, and from his
business, hand-in-glove with the ``nobs at the West End''; old Colonel
Livermore, of the Bombay Army, and Mrs.\ Livermore, from Upper Bedford
Place; old Sergeant Toffy and Mrs.\ Toffy; and sometimes old Sir Thomas
Coffin and Lady Coffin, from Bedford Square.  Sir Thomas was celebrated
as a hanging judge, and the particular tawny port was produced when he
dined with Mr.\ Osborne.

These people and their like gave the pompous Russell Square merchant
pompous dinners back again.  They had solemn rubbers of whist, when
they went upstairs after drinking, and their carriages were called at
half past ten. Many rich people, whom we poor devils are in the habit
of envying, lead contentedly an existence like that above described.
Jane Osborne scarcely ever met a man under sixty, and almost the only
bachelor who appeared in their society was Mr.\ Smirk, the celebrated
ladies' doctor.

I can't say that nothing had occurred to disturb the monotony of this
awful existence:  the fact is, there had been a secret in poor Jane's
life which had made her father more savage and morose than even nature,
pride, and over-feeding had made him.  This secret was connected with
Miss Wirt, who had a cousin an artist, Mr.\ Smee, very celebrated since
as a portrait-painter and R.A., but who once was glad enough to give
drawing lessons to ladies of fashion.  Mr.\ Smee has forgotten where
Russell Square is now, but he was glad enough to visit it in the year
1818, when Miss Osborne had instruction from him.

Smee (formerly a pupil of Sharpe of Frith Street, a dissolute,
irregular, and unsuccessful man, but a man with great knowledge of his
art) being the cousin of Miss Wirt, we say, and introduced by her to
Miss Osborne, whose hand and heart were still free after various
incomplete love affairs, felt a great attachment for this lady, and it
is believed inspired one in her bosom.  Miss Wirt was the confidante of
this intrigue.  I know not whether she used to leave the room where the
master and his pupil were painting, in order to give them an
opportunity for exchanging those vows and sentiments which cannot be
uttered advantageously in the presence of a third party; I know not
whether she hoped that should her cousin succeed in carrying off the
rich merchant's daughter, he would give Miss Wirt a portion of the
wealth which she had enabled him to win---all that is certain is that
Mr.\ Osborne got some hint of the transaction, came back from the City
abruptly, and entered the drawing-room with his bamboo cane; found the
painter, the pupil, and the companion all looking exceedingly pale
there; turned the former out of doors with menaces that he would break
every bone in his skin, and half an hour afterwards dismissed Miss Wirt
likewise, kicking her trunks down the stairs, trampling on her
bandboxes, and shaking his fist at her hackney coach as it bore her
away.

Jane Osborne kept her bedroom for many days.  She was not allowed to
have a companion afterwards.  Her father swore to her that she should
not have a shilling of his money if she made any match without his
concurrence; and as he wanted a woman to keep his house, he did not
choose that she should marry, so that she was obliged to give up all
projects with which Cupid had any share. During her papa's life, then,
she resigned herself to the manner of existence here described, and was
content to be an old maid.  Her sister, meanwhile, was having children
with finer names every year and the intercourse between the two grew
fainter continually.  ``Jane and I do not move in the same sphere of
life,'' Mrs.\ Bullock said.  ``I regard her as a sister, of course''---which
means---what does it mean when a lady says that she regards Jane as a
sister?

It has been described how the Misses Dobbin lived with their father at
a fine villa at Denmark Hill, where there were beautiful graperies and
peach-trees which delighted little Georgy Osborne. The Misses Dobbin,
who drove often to Brompton to see our dear Amelia, came sometimes to
Russell Square too, to pay a visit to their old acquaintance Miss
Osborne.  I believe it was in consequence of the commands of their
brother the Major in India (for whom their papa had a prodigious
respect), that they paid attention to Mrs.\ George; for the Major, the
godfather and guardian of Amelia's little boy, still hoped that the
child's grandfather might be induced to relent towards him and
acknowledge him for the sake of his son.  The Misses Dobbin kept Miss
Osborne acquainted with the state of Amelia's affairs; how she was
living with her father and mother; how poor they were; how they
wondered what men, and such men as their brother and dear Captain
Osborne, could find in such an insignificant little chit; how she was
still, as heretofore, a namby-pamby milk-and-water affected
creature---but how the boy was really the noblest little boy ever
seen---for the hearts of all women warm towards young children, and the
sourest spinster is kind to them.

One day, after great entreaties on the part of the Misses Dobbin,
Amelia allowed little George to go and pass a day with them at Denmark
Hill---a part of which day she spent herself in writing to the Major in
India.  She congratulated him on the happy news which his sisters had
just conveyed to her.  She prayed for his prosperity and that of the
bride he had chosen.  She thanked him for a thousand thousand kind
offices and proofs of steadfast friendship to her in her affliction.
She told him the last news about little Georgy, and how he was gone to
spend that very day with his sisters in the country.  She underlined
the letter a great deal, and she signed herself affectionately his
friend, Amelia Osborne.  She forgot to send any message of kindness to
Lady O'Dowd, as her wont was---and did not mention Glorvina by name, and
only in italics, as the Major's BRIDE, for whom she begged blessings.
But the news of the marriage removed the reserve which she had kept up
towards him.  She was glad to be able to own and feel how warmly and
gratefully she regarded him---and as for the idea of being jealous of
Glorvina (Glorvina, indeed!), Amelia would have scouted it, if an angel
from heaven had hinted it to her.  That night, when Georgy came back in
the pony-carriage in which he rejoiced, and in which he was driven by
Sir Wm. Dobbin's old coachman, he had round his neck a fine gold chain
and watch.  He said an old lady, not pretty, had given it him, who
cried and kissed him a great deal.  But he didn't like her.  He liked
grapes very much.  And he only liked his mamma.  Amelia shrank and
started; the timid soul felt a presentiment of terror when she heard
that the relations of the child's father had seen him.

Miss Osborne came back to give her father his dinner.  He had made a
good speculation in the City, and was rather in a good humour that day,
and chanced to remark the agitation under which she laboured. ``What's
the matter, Miss Osborne?'' he deigned to say.

The woman burst into tears.  ``Oh, sir,'' she said, ``I've seen little
George.  He is as beautiful as an angel---and so like him!'' The old man
opposite to her did not say a word, but flushed up and began to tremble
in every limb.



\chapter{In Which the Reader Has to Double the Cape}

The astonished reader must be called upon to transport himself ten
thousand miles to the military station of Bundlegunge, in the Madras
division of our Indian empire, where our gallant old friends of the
---th regiment are quartered under the command of the brave Colonel, Sir
Michael O'Dowd.  Time has dealt kindly with that stout officer, as it
does ordinarily with men who have good stomachs and good tempers and
are not perplexed over much by fatigue of the brain. The Colonel plays
a good knife and fork at tiffin and resumes those weapons with great
success at dinner.  He smokes his hookah after both meals and puffs as
quietly while his wife scolds him as he did under the fire of the
French at Waterloo.  Age and heat have not diminished the activity or
the eloquence of the descendant of the Malonys and the Molloys.  Her
Ladyship, our old acquaintance, is as much at home at Madras as at
Brussels in the cantonment as under the tents.  On the march you saw
her at the head of the regiment seated on a royal elephant, a noble
sight. Mounted on that beast, she has been into action with tigers in
the jungle, she has been received by native princes, who have welcomed
her and Glorvina into the recesses of their zenanas and offered her
shawls and jewels which it went to her heart to refuse.  The sentries
of all arms salute her wherever she makes her appearance, and she
touches her hat gravely to their salutation.  Lady O'Dowd is one of the
greatest ladies in the Presidency of Madras---her quarrel with Lady
Smith, wife of Sir Minos Smith the puisne judge, is still remembered by
some at Madras, when the Colonel's lady snapped her fingers in the
Judge's lady's face and said SHE'D never walk behind ever a beggarly
civilian.  Even now, though it is five-and-twenty years ago, people
remember Lady O'Dowd performing a jig at Government House, where she
danced down two Aides-de-Camp, a Major of Madras cavalry, and two
gentlemen of the Civil Service; and, persuaded by Major Dobbin, C.B.,
second in command of the ---th, to retire to the supper-room, lassata
nondum satiata recessit.

Peggy O'Dowd is indeed the same as ever, kind in act and thought;
impetuous in temper; eager to command; a tyrant over her Michael; a
dragon amongst all the ladies of the regiment; a mother to all the
young men, whom she tends in their sickness, defends in all their
scrapes, and with whom Lady Peggy is immensely popular.  But the
Subalterns' and Captains' ladies (the Major is unmarried) cabal against
her a good deal.  They say that Glorvina gives herself airs and that
Peggy herself is intolerably domineering.  She interfered with a
little congregation which Mrs.\ Kirk had got up and laughed the young
men away from her sermons, stating that a soldier's wife had no
business to be a parson---that Mrs.\ Kirk would be much better mending
her husband's clothes; and, if the regiment wanted sermons, that she
had the finest in the world, those of her uncle, the Dean. She abruptly
put a termination to a flirtation which Lieutenant Stubble of the
regiment had commenced with the Surgeon's wife, threatening to come
down upon Stubble for the money which he had borrowed from her (for the
young fellow was still of an extravagant turn) unless he broke off at
once and went to the Cape on sick leave.  On the other hand, she housed
and sheltered Mrs.\ Posky, who fled from her bungalow one night, pursued
by her infuriate husband, wielding his second brandy bottle, and
actually carried Posky through the delirium tremens and broke him of
the habit of drinking, which had grown upon that officer, as all evil
habits will grow upon men.  In a word, in adversity she was the best of
comforters, in good fortune the most troublesome of friends, having a
perfectly good opinion of herself always and an indomitable resolution
to have her own way.

Among other points, she had made up her mind that Glorvina should marry
our old friend Dobbin.  Mrs.\ O'Dowd knew the Major's expectations and
appreciated his good qualities and the high character which he enjoyed
in his profession.  Glorvina, a very handsome, fresh-coloured,
black-haired, blue-eyed young lady, who could ride a horse, or play a
sonata with any girl out of the County Cork, seemed to be the very
person destined to insure Dobbin's happiness---much more than that poor
good little weak-spur'ted Amelia, about whom he used to take on
so.---``Look at Glorvina enter a room,'' Mrs.\ O'Dowd would say, ``and
compare her with that poor Mrs.\ Osborne, who couldn't say boo to a
goose.  She'd be worthy of you, Major---you're a quiet man yourself, and
want some one to talk for ye.  And though she does not come of such
good blood as the Malonys or Molloys, let me tell ye, she's of an
ancient family that any nobleman might be proud to marry into.''

But before she had come to such a resolution and determined to
subjugate Major Dobbin by her endearments, it must be owned that
Glorvina had practised them a good deal elsewhere.  She had had a
season in Dublin, and who knows how many in Cork, Killarney, and
Mallow? She had flirted with all the marriageable officers whom the
depots of her country afforded, and all the bachelor squires who seemed
eligible.  She had been engaged to be married a half-score times in
Ireland, besides the clergyman at Bath who used her so ill. She had
flirted all the way to Madras with the Captain and chief mate of the
Ramchunder East Indiaman, and had a season at the Presidency with her
brother and Mrs.\ O'Dowd, who was staying there, while the Major of the
regiment was in command at the station. Everybody admired her there;
everybody danced with her; but no one proposed who was worth the
marrying---one or two exceedingly young subalterns sighed after her, and
a beardless civilian or two, but she rejected these as beneath her
pretensions---and other and younger virgins than Glorvina were married
before her.  There are women, and handsome women too, who have this
fortune in life.  They fall in love with the utmost generosity; they
ride and walk with half the Army-list, though they draw near to forty,
and yet the Misses O'Grady are the Misses O'Grady still:  Glorvina
persisted that but for Lady O'Dowd's unlucky quarrel with the Judge's
lady, she would have made a good match at Madras, where old Mr.\ %
Chutney, who was at the head of the civil service (and who afterwards
married Miss Dolby, a young lady only thirteen years of age who had
just arrived from school in Europe), was just at the point of proposing
to her.

Well, although Lady O'Dowd and Glorvina quarrelled a great number of
times every day, and upon almost every conceivable subject---indeed, if
Mick O'Dowd had not possessed the temper of an angel two such women
constantly about his ears would have driven him out of his senses---yet
they agreed between themselves on this point, that Glorvina should
marry Major Dobbin, and were determined that the Major should have no
rest until the arrangement was brought about. Undismayed by forty or
fifty previous defeats, Glorvina laid siege to him.  She sang Irish
melodies at him unceasingly.  She asked him so frequently and
pathetically, Will ye come to the bower? that it is a wonder how any
man of feeling could have resisted the invitation.  She was never tired
of inquiring, if Sorrow had his young days faded, and was ready to
listen and weep like Desdemona at the stories of his dangers and his
campaigns.  It has been said that our honest and dear old friend used
to perform on the flute in private; Glorvina insisted upon having duets
with him, and Lady O'Dowd would rise and artlessly quit the room when
the young couple were so engaged. Glorvina forced the Major to ride
with her of mornings.  The whole cantonment saw them set out and
return.  She was constantly writing notes over to him at his house,
borrowing his books, and scoring with her great pencil-marks such
passages of sentiment or humour as awakened her sympathy.  She borrowed
his horses, his servants, his spoons, and palanquin---no wonder that
public rumour assigned her to him, and that the Major's sisters in
England should fancy they were about to have a sister-in-law.

Dobbin, who was thus vigorously besieged, was in the meanwhile in a
state of the most odious tranquillity.  He used to laugh when the young
fellows of the regiment joked him about Glorvina's manifest attentions
to him. ``Bah!'' said he, ``she is only keeping her hand in---she
practises upon me as she does upon Mrs.\ Tozer's piano, because it's the
most handy instrument in the station.  I am much too battered and old
for such a fine young lady as Glorvina.'' And so he went on riding with
her, and copying music and verses into her albums, and playing at chess
with her very submissively; for it is with these simple amusements that
some officers in India are accustomed to while away their leisure
moments, while others of a less domestic turn hunt hogs, and shoot
snipes, or gamble and smoke cheroots, and betake themselves to
brandy-and-water.  As for Sir Michael O'Dowd, though his lady and her
sister both urged him to call upon the Major to explain himself and not
keep on torturing a poor innocent girl in that shameful way, the old
soldier refused point-blank to have anything to do with the conspiracy.
``Faith, the Major's big enough to choose for himself,'' Sir Michael
said; ``he'll ask ye when he wants ye''; or else he would turn the matter
off jocularly, declaring that ``Dobbin was too young to keep house, and
had written home to ask lave of his mamma.'' Nay, he went farther, and
in private communications with his Major would caution and rally him,
crying, ``Mind your oi, Dob, my boy, them girls is bent on mischief---me
Lady has just got a box of gowns from Europe,  and there's a pink satin
for Glorvina, which will finish ye,  Dob, if it's in the power of woman
or satin to move ye.''

But the truth is, neither beauty nor fashion could conquer him.  Our
honest friend had but one idea of a woman in his head, and that one did
not in the least resemble Miss Glorvina O'Dowd in pink satin.  A gentle
little woman in black, with large eyes and brown hair, seldom speaking,
save when spoken to, and then in a voice not the least resembling Miss
Glorvina's---a soft young mother tending an infant and beckoning the
Major up with a smile to look at him---a rosy-cheeked lass coming
singing into the room in Russell Square or hanging on George Osborne's
arm, happy and loving---there was but this image that filled our honest
Major's mind, by day and by night, and reigned over it always.  Very
likely Amelia was not like the portrait the Major had formed of her:
there was a figure in a book of fashions which his sisters had in
England, and with which William had made away privately, pasting it
into the lid of his desk, and fancying he saw some resemblance to Mrs.\ %
Osborne in the print, whereas I have seen it, and can vouch that it is
but the picture of a high-waisted gown with an impossible doll's face
simpering over it---and, perhaps, Mr.\ Dobbin's sentimental Amelia was no
more like the real one than this absurd little print which he
cherished.  But what man in love, of us, is better informed?---or is he
much happier when he sees and owns his delusion? Dobbin was under this
spell.  He did not bother his friends and the public much about his
feelings, or indeed lose his natural rest or appetite on account of
them.  His head has grizzled since we saw him last, and a line or two
of silver may be seen in the soft brown hair likewise.  But his
feelings are not in the least changed or oldened, and his love remains
as fresh as a man's recollections of boyhood are.

We have said how the two Misses Dobbin and Amelia, the Major's
correspondents in Europe, wrote him letters from England, Mrs.\ Osborne
congratulating him with great candour and cordiality upon his
approaching nuptials with Miss O'Dowd. ``Your sister has just kindly
visited me,'' Amelia wrote in her letter, ``and informed me of an
INTERESTING EVENT, upon which I beg to offer my MOST SINCERE
CONGRATULATIONS. I hope the young lady to whom I hear you are to be
UNITED will in every respect prove worthy of one who is himself all
kindness and goodness.  The poor widow has only her prayers to offer
and her cordial cordial wishes for YOUR PROSPERITY!  Georgy sends his
love to HIS DEAR GODPAPA and hopes that you will not forget him. I tell
him that you are about to form OTHER TIES, with one who I am sure
merits ALL YOUR AFFECTION, but that, although such ties must of course
be the strongest and most sacred, and supersede ALL OTHERS, yet that I
am sure the widow and the child whom you have ever protected and loved
will always HAVE A CORNER IN YOUR HEART.'' The letter, which has been
before alluded to, went on in this strain, protesting throughout as to
the extreme satisfaction of the writer.

This letter, which arrived by the very same ship which brought out
Lady O'Dowd's box of millinery from London (and which you may be sure
Dobbin opened before any one of the other packets which the mail
brought him), put the receiver into such a state of mind that Glorvina,
and her pink satin, and everything belonging to her became perfectly
odious to him.  The Major cursed the talk of women, and the sex in
general.  Everything annoyed him that day---the parade was insufferably
hot and wearisome.  Good heavens! was a man of intellect to waste his
life, day after day, inspecting cross-belts and putting fools through
their manoeuvres? The senseless chatter of the young men at mess was
more than ever jarring. What cared he, a man on the high road to forty,
to know how many snipes Lieutenant Smith had shot, or what were the
performances of Ensign Brown's mare? The jokes about the table filled
him with shame.  He was too old to listen to the banter of the
assistant surgeon and the slang of the youngsters, at which old O'Dowd,
with his bald head and red face, laughed quite easily.  The old man had
listened to those jokes any time these thirty years---Dobbin himself had
been fifteen years hearing them.  And after the boisterous dulness of
the mess-table, the quarrels and scandal of the ladies of the regiment!
It was unbearable, shameful.  ``O Amelia, Amelia,'' he thought, ``you to
whom I have been so faithful---you reproach me!  It is because you
cannot feel for me that I drag on this wearisome life.  And you reward
me after years of devotion by giving me your blessing upon my marriage,
forsooth, with this flaunting Irish girl!'' Sick and sorry felt poor
William; more than ever wretched and lonely.  He would like to have
done with life and its vanity altogether---so bootless and
unsatisfactory the struggle, so cheerless and dreary the prospect
seemed to him.  He lay all that night sleepless, and yearning to go
home.  Amelia's letter had fallen as a blank upon him.  No fidelity, no
constant truth and passion, could move her into warmth.  She would not
see that he loved her.  Tossing in his bed, he spoke out to her. ``Good
God, Amelia!'' he said, ``don't you know that I only love you in the
world---you, who are a stone to me---you, whom I tended through months
and months of illness and grief, and who bade me farewell with a smile
on your face, and forgot me before the door shut between us!'' The
native servants lying outside his verandas beheld with wonder the
Major, so cold and quiet ordinarily, at present so passionately moved
and cast down.  Would she have pitied him had she seen him? He read
over and over all the letters which he ever had from her---letters of
business relative to the little property which he had made her believe
her husband had left to her---brief notes of invitation---every scrap of
writing that she had ever sent to him---how cold, how kind, how
hopeless, how selfish they were!

Had there been some kind gentle soul near at hand who could read and
appreciate this silent generous heart, who knows but that the reign of
Amelia might have been over, and that friend William's love might have
flowed into a kinder channel? But there was only Glorvina of the jetty
ringlets with whom his intercourse was familiar, and this dashing young
woman was not bent upon loving the Major, but rather on making the
Major admire HER---a most vain and hopeless task, too, at least
considering the means that the poor girl possessed to carry it out.
She curled her hair and showed her shoulders at him, as much as to say,
did ye ever see such jet ringlets and such a complexion? She grinned at
him so that he might see that every tooth in her head was sound---and he
never heeded all these charms.  Very soon after the arrival of the box
of millinery, and perhaps indeed in honour of it, Lady O'Dowd and the
ladies of the King's Regiment gave a ball to the Company's Regiments
and the civilians at the station.  Glorvina sported the killing pink
frock, and the Major, who attended the party and walked very ruefully
up and down the rooms, never so much as perceived the pink garment.
Glorvina danced past him in a fury with all the young subalterns of the
station, and the Major was not in the least jealous of her performance,
or angry because Captain Bangles of the Cavalry handed her to supper.
It was not jealousy, or frocks, or shoulders that could move him, and
Glorvina had nothing more.

So these two were each exemplifying the Vanity of this life, and each
longing for what he or she could not get. Glorvina cried with rage at
the failure.  She had set her mind on the Major ``more than on any of
the others,'' she owned, sobbing.  ``He'll break my heart, he will,
Peggy,'' she would whimper to her sister-in-law when they were good
friends; ``sure every one of me frocks must be taken in---it's such a
skeleton I'm growing.'' Fat or thin, laughing or melancholy, on
horseback or the music-stool, it was all the same to the Major.  And
the Colonel, puffing his pipe and listening to these complaints, would
suggest that Glory should have some black frocks out in the next box
from London, and told a mysterious story of a lady in Ireland who died
of grief for the loss of her husband before she got ere a one.

While the Major was going on in this tantalizing way, not proposing,
and declining to fall in love, there came another ship from Europe
bringing letters on board, and amongst them some more for the heartless
man.  These were home letters bearing an earlier postmark than that of
the former packets, and as Major Dobbin recognized among his the
handwriting of his sister, who always crossed and recrossed her letters
to her brother---gathered together all the possible bad news which she
could collect, abused him and read him lectures with sisterly
frankness, and always left him miserable for the day after ``dearest
William'' had achieved the perusal of one of her epistles---the truth
must be told that dearest William did not hurry himself to break the
seal of Miss Dobbin's letter, but waited for a particularly favourable
day and mood for doing so.  A fortnight before, moreover, he had
written to scold her for telling those absurd stories to Mrs.\ Osborne,
and had despatched a letter in reply to that lady, undeceiving her with
respect to the reports concerning him and assuring her that ``he had no
sort of present intention of altering his condition.''

Two or three nights after the arrival of the second package of letters,
the Major had passed the evening pretty cheerfully at Lady O'Dowd's
house, where Glorvina thought that he listened with rather more
attention than usual to the Meeting of the Wathers, the Minsthrel Boy,
and one or two other specimens of song with which she favoured him (the
truth is, he was no more listening to Glorvina than to the howling of
the jackals in the moonlight outside, and the delusion was hers as
usual), and having played his game at chess with her (cribbage with the
surgeon was Lady O'Dowd's favourite evening pastime), Major Dobbin took
leave of the Colonel's family at his usual hour and retired to his own
house.

There on his table, his sister's letter lay reproaching him.  He took
it up, ashamed rather of his negligence regarding it, and prepared
himself for a disagreeable hour's communing with that crabbed-handed
absent relative. . . . It may have been an hour after the Major's
departure from the Colonel's house---Sir Michael was sleeping the sleep
of the just; Glorvina had arranged her black ringlets in the
innumerable little bits of paper, in which it was her habit to confine
them; Lady O'Dowd, too, had gone to her bed in the nuptial chamber, on
the ground-floor, and had tucked her musquito curtains round her fair
form, when the guard at the gates of the Commanding-Officer's compound
beheld Major Dobbin, in the moonlight, rushing towards the house with a
swift step and a very agitated countenance, and he passed the sentinel
and went up to the windows of the Colonel's bedchamber.

``O'Dowd---Colonel!'' said Dobbin and kept up a great shouting.

``Heavens, Meejor!'' said Glorvina of the curl-papers, putting out her
head too, from her window.

``What is it, Dob, me boy?'' said the Colonel, expecting there was a fire
in the station, or that the route had come from headquarters.

``I---I must have leave of absence.  I must go to England---on the most
urgent private affairs,'' Dobbin said.

``Good heavens, what has happened!'' thought Glorvina, trembling with all
the papillotes.

``I want to be off---now---to-night,'' Dobbin continued; and the Colonel
getting up, came out to parley with him.

In the postscript of Miss Dobbin's cross-letter, the Major had just
come upon a paragraph, to the following effect:---``I drove yesterday to
see your old ACQUAINTANCE, Mrs.\ Osborne.  The wretched place they live
at, since they were bankrupts, you know---Mr.\ S., to judge from a BRASS
PLATE on the door of his hut (it is little better) is a coal-merchant.
The little boy, your godson, is certainly a fine child, though forward,
and inclined to be saucy and self-willed. But we have taken notice of
him as you wish it, and have introduced him to his aunt, Miss O., who
was rather pleased with him.  Perhaps his grandpapa, not the bankrupt
one, who is almost doting, but Mr.\ Osborne, of Russell Square, may be
induced to relent towards the child of your friend, HIS ERRING AND
SELF-WILLED SON.  And Amelia will not be ill-disposed to give him up.
The widow is CONSOLED, and is about to marry a reverend gentleman, the
Rev. Mr.\ Binny, one of the curates of Brompton.  A poor match.  But
Mrs.\ O. is getting old, and I saw a great deal of grey in her hair---she
was in very good spirits:  and your little godson overate himself at
our house. Mamma sends her love with that of your affectionate, Ann
Dobbin.''



\chapter{A Round-about Chapter between London and Hampshire}

Our old friends the Crawleys' family house, in Great Gaunt Street,
still bore over its front the hatchment which had been placed there as
a token of mourning for Sir Pitt Crawley's demise, yet this heraldic
emblem was in itself a very splendid and gaudy piece of furniture, and
all the rest of the mansion became more brilliant than it had ever been
during the late baronet's reign.  The black outer-coating of the bricks
was removed, and they appeared with a cheerful, blushing face streaked
with white: the old bronze lions of the knocker were gilt handsomely,
the railings painted, and the dismallest house in Great Gaunt Street
became the smartest in the whole quarter, before the green leaves in
Hampshire had replaced those yellowing ones which were on the trees in
Queen's Crawley Avenue when old Sir Pitt Crawley passed under them for
the last time.

A little woman, with a carriage to correspond, was perpetually seen
about this mansion; an elderly spinster, accompanied by a little boy,
also might be remarked coming thither daily.  It was Miss Briggs and
little Rawdon, whose business it was to see to the inward renovation of
Sir Pitt's house, to superintend the female band engaged in stitching
the blinds and hangings, to poke and rummage in the drawers and
cupboards crammed with the dirty relics and congregated trumperies of a
couple of generations of Lady Crawleys, and to take inventories of the
china, the glass, and other properties in the closets and store-rooms.

Mrs.\ Rawdon Crawley was general-in-chief over these arrangements, with
full orders from Sir Pitt to sell, barter, confiscate, or purchase
furniture, and she enjoyed herself not a little in an occupation which
gave full scope to her taste and ingenuity.  The renovation of the
house was determined upon when Sir Pitt came to town in November to see
his lawyers, and when he passed nearly a week in Curzon Street, under
the roof of his affectionate brother and sister.

He had put up at an hotel at first, but, Becky, as soon as she heard of
the Baronet's arrival, went off alone to greet him, and returned in an
hour to Curzon Street with Sir Pitt in the carriage by her side.  It
was impossible sometimes to resist this artless little creature's
hospitalities, so kindly were they pressed, so frankly and amiably
offered.  Becky seized Pitt's hand in a transport of gratitude when he
agreed to come.  ``Thank you,'' she said, squeezing it and looking into
the Baronet's eyes, who blushed a good deal; ``how happy this will make
Rawdon!'' She bustled up to Pitt's bedroom, leading on the servants, who
were carrying his trunks thither.  She came in herself laughing, with a
coal-scuttle out of her own room.

A fire was blazing already in Sir Pitt's apartment (it was Miss
Briggs's room, by the way, who was sent upstairs to sleep with the
maid).  ``I knew I should bring you,'' she said with pleasure beaming in
her glance.  Indeed, she was really sincerely happy at having him for a
guest.

Becky made Rawdon dine out once or twice on business, while Pitt stayed
with them, and the Baronet passed the happy evening alone with her and
Briggs.  She went downstairs to the kitchen and actually cooked little
dishes for him.  ``Isn't it a good salmi?'' she said; ``I made it for you.
I can make you better dishes than that, and will when you come to see
me.''

``Everything you do, you do well,'' said the Baronet gallantly.  ``The
salmi is excellent indeed.''

``A poor man's wife,'' Rebecca replied gaily, ``must make herself useful,
you know''; on which her brother-in-law vowed that ``she was fit to be
the wife of an Emperor, and that to be skilful in domestic duties was
surely one of the most charming of woman's qualities.'' And Sir Pitt
thought, with something like mortification, of Lady Jane at home, and
of a certain pie which she had insisted on making, and serving to him
at dinner---a most abominable pie.

Besides the salmi, which was made of Lord Steyne's pheasants from his
lordship's cottage of Stillbrook, Becky gave her brother-in-law a
bottle of white wine, some that Rawdon had brought with him from
France, and had picked up for nothing, the little story-teller said;
whereas the liquor was, in truth, some White Hermitage from the Marquis
of Steyne's famous cellars, which brought fire into the Baronet's
pallid cheeks and a glow into his feeble frame.

Then when he had drunk up the bottle of petit vin blanc, she gave him
her hand, and took him up to the drawing-room, and made him snug on the
sofa by the fire, and let him talk as she listened with the tenderest
kindly interest, sitting by him, and hemming a shirt for her dear
little boy.  Whenever Mrs.\ Rawdon wished to be particularly humble and
virtuous, this little shirt used to come out of her work-box.  It had
got to be too small for Rawdon long before it was finished.

Well, Rebecca listened to Pitt, she talked to him, she sang to him, she
coaxed him, and cuddled him, so that he found himself more and more
glad every day to get back from the lawyer's at Gray's Inn, to the
blazing fire in Curzon Street---a gladness in which the men of law
likewise participated, for Pitt's harangues were of the longest---and so
that when he went away he felt quite a pang at departing. How pretty
she looked kissing her hand to him from the carriage and waving her
handkerchief when he had taken his place in the mail! She put the
handkerchief to her eyes once.  He pulled his sealskin cap over his, as
the coach drove away, and, sinking back, he thought to himself how she
respected him and how he deserved it, and how Rawdon was a foolish dull
fellow who didn't half appreciate his wife; and how mum and stupid his
own wife was compared to that brilliant little Becky.  Becky had hinted
every one of these things herself, perhaps, but so delicately and
gently that you hardly knew when or where.  And, before they parted, it
was agreed that the house in London should be redecorated for the next
season, and that the brothers' families should meet again in the
country at Christmas.

``I wish you could have got a little money out of him,'' Rawdon said to
his wife moodily when the Baronet was gone.  ``I should like to give
something to old Raggles, hanged if I shouldn't.  It ain't right, you
know, that the old fellow should be kept out of all his money.  It may
be inconvenient, and he might let to somebody else besides us, you
know.''

``Tell him,'' said Becky, ``that as soon as Sir Pitt's affairs are
settled, everybody will be paid, and give him a little something on
account.  Here's a cheque that Pitt left for the boy,'' and she took
from her bag and gave her husband a paper which his brother had handed
over to her, on behalf of the little son and heir of the younger branch
of the Crawleys.

The truth is, she had tried personally the ground on which her husband
expressed a wish that she should venture---tried it ever so delicately,
and found it unsafe. Even at a hint about embarrassments, Sir Pitt
Crawley was off and alarmed.  And he began a long speech, explaining
how straitened he himself was in money matters; how the tenants would
not pay; how his father's affairs, and the expenses attendant upon the
demise of the old gentleman, had involved him; how he wanted to pay off
incumbrances; and how the bankers and agents were overdrawn; and Pitt
Crawley ended by making a compromise with his sister-in-law and giving
her a very small sum for the benefit of her little boy.

Pitt knew how poor his brother and his brother's family must be.  It
could not have escaped the notice of such a cool and experienced old
diplomatist that Rawdon's family had nothing to live upon, and that
houses and carriages are not to be kept for nothing.  He knew very well
that he was the proprietor or appropriator of the money, which,
according to all proper calculation, ought to have fallen to his
younger brother, and he had, we may be sure, some secret pangs of
remorse within him, which warned him that he ought to perform some act
of justice, or, let us say, compensation, towards these disappointed
relations.  A just, decent man, not without brains, who said his
prayers, and knew his catechism, and did his duty outwardly through
life, he could not be otherwise than aware that something was due to
his brother at his hands, and that morally he was Rawdon's debtor.

But, as one reads in the columns of the Times newspaper every now and
then, queer announcements from the Chancellor of the Exchequer,
acknowledging the receipt of 50 pounds from A. B., or 10 pounds from
W. T., as conscience-money, on account of taxes due by the said A. B.
or W. T., which payments the penitents beg the Right Honourable
gentleman to acknowledge through the medium of the public press---so is
the Chancellor no doubt, and the reader likewise, always perfectly sure
that the above-named A. B.  and W. T. are only paying a very small
instalment of what they really owe, and that the man who sends up a
twenty-pound note has very likely hundreds or thousands more for which
he ought to account.  Such, at least, are my feelings, when I see
A. B. or W. T.'s insufficient acts of repentance.  And I have no doubt
that Pitt Crawley's contrition, or kindness if you will, towards his
younger brother, by whom he had so much profited, was only a very small
dividend upon the capital sum in which he was indebted to Rawdon. Not
everybody is willing to pay even so much.  To part with money is a
sacrifice beyond almost all men endowed with a sense of order.  There
is scarcely any man alive who does not think himself meritorious for
giving his neighbour five pounds.  Thriftless gives, not from a
beneficent pleasure in giving, but from a lazy delight in spending. He
would not deny himself one enjoyment; not his opera-stall, not his
horse, not his dinner, not even the pleasure of giving Lazarus the five
pounds.  Thrifty, who is good, wise, just, and owes no man a penny,
turns from a beggar, haggles with a hackney-coachman, or denies a poor
relation, and I doubt which is the most selfish of the two.  Money has
only a different value in the eyes of each.

So, in a word, Pitt Crawley thought he would do something for his
brother, and then thought that he would think about it some other time.

And with regard to Becky, she was not a woman who expected too much
from the generosity of her neighbours, and so was quite content with
all that Pitt Crawley had done for her.  She was acknowledged by the
head of the family.  If Pitt would not give her anything, he would get
something for her some day.  If she got no money from her
brother-in-law, she got what was as good as money---credit.  Raggles was
made rather easy in his mind by the spectacle of the union between the
brothers, by a small payment on the spot, and by the promise of a much
larger sum speedily to be assigned to him.  And Rebecca told Miss
Briggs, whose Christmas dividend upon the little sum lent by her Becky
paid with an air of candid joy, and as if her exchequer was brimming
over with gold---Rebecca, we say, told Miss Briggs, in strict confidence
that she had conferred with Sir Pitt, who was famous as a financier, on
Briggs's special behalf, as to the most profitable investment of Miss
B.'s remaining capital; that Sir Pitt, after much consideration, had
thought of a most safe and advantageous way in which Briggs could lay
out her money; that, being especially interested in her as an attached
friend of the late Miss Crawley, and of the whole family, and that long
before he left town, he had recommended that she should be ready with
the money at a moment's notice, so as to purchase at the most
favourable opportunity the shares which Sir Pitt had in his eye.  Poor
Miss Briggs was very grateful for this mark of Sir Pitt's attention---it
came so unsolicited, she said, for she never should have thought of
removing the money from the funds---and the delicacy enhanced the
kindness of the office; and she promised to see her man of business
immediately and be ready with her little cash at the proper hour.

And this worthy woman was so grateful for the kindness of Rebecca in
the matter, and for that of her generous benefactor, the Colonel, that
she went out and spent a great part of her half-year's dividend in the
purchase of a black velvet coat for little Rawdon, who, by the way, was
grown almost too big for black velvet now, and was of a size and age
befitting him for the assumption of the virile jacket and pantaloons.

He was a fine open-faced boy, with blue eyes and waving flaxen hair,
sturdy in limb, but generous and soft in heart, fondly attaching
himself to all who were good to him---to the pony---to Lord Southdown,
who gave him the horse (he used to blush and glow all over when he saw
that kind young nobleman)---to the groom who had charge of the pony---to
Molly, the cook, who crammed him with ghost stories at night, and with
good things from the dinner---to Briggs, whom he plagued and laughed
at---and to his father especially, whose attachment towards the lad was
curious too to witness.  Here, as he grew to be about eight years old,
his attachments may be said to have ended.  The beautiful mother-vision
had faded away after a while.  During near two years she had scarcely
spoken to the child. She disliked him.  He had the measles and the
hooping-cough.  He bored her.  One day when he was standing at the
landing-place, having crept down from the upper regions, attracted by
the sound of his mother's voice, who was singing to Lord Steyne, the
drawing room door opening suddenly, discovered the little spy, who but
a moment before had been rapt in delight, and listening to the music.

His mother came out and struck him violently a couple of boxes on the
ear.  He heard a laugh from the Marquis in the inner room (who was
amused by this free and artless exhibition of Becky's temper) and fled
down below to his friends of the kitchen, bursting in an agony of grief.

``It is not because it hurts me,'' little Rawdon gasped
out---``only---only''---sobs and tears wound up the sentence in a storm.  It
was the little boy's heart that was bleeding.  ``Why mayn't I hear her
singing?  Why don't she ever sing to me---as she does to that baldheaded
man with the large teeth?'' He gasped out at various intervals these
exclamations of rage and grief.  The cook looked at the housemaid, the
housemaid looked knowingly at the footman---the awful kitchen inquisition
which sits in judgement in every house and knows everything---sat on
Rebecca at that moment.

After this incident, the mother's dislike increased to hatred; the
consciousness that the child was in the house was a reproach and a pain
to her.  His very sight annoyed her.  Fear, doubt, and resistance
sprang up, too, in the boy's own bosom.  They were separated from that
day of the boxes on the ear.

Lord Steyne also heartily disliked the boy.  When they met by
mischance, he made sarcastic bows or remarks to the child, or glared at
him with savage-looking eyes. Rawdon used to stare him in the face and
double his little fists in return.  He knew his enemy, and this
gentleman, of all who came to the house, was the one who angered him
most.  One day the footman found him squaring his fists at Lord
Steyne's hat in the hall.  The footman told the circumstance as a good
joke to Lord Steyne's coachman; that officer imparted it to Lord
Steyne's gentleman, and to the servants' hall in general. And very soon
afterwards, when Mrs.\ Rawdon Crawley made her appearance at Gaunt
House, the porter who unbarred the gates, the servants of all uniforms
in the hall, the functionaries in white waistcoats, who bawled out from
landing to landing the names of Colonel and Mrs.\ Rawdon Crawley, knew
about her, or fancied they did. The man who brought her refreshment and
stood behind her chair, had talked her character over with the large
gentleman in motley-coloured clothes at his side.  Bon Dieu! it is
awful, that servants' inquisition!  You see a woman in a great party in
a splendid saloon, surrounded by faithful admirers, distributing
sparkling glances, dressed to perfection, curled, rouged, smiling and
happy---Discovery walks respectfully up to her, in the shape of a huge
powdered man with large calves and a tray of ices---with Calumny (which
is as fatal as truth) behind him, in the shape of the hulking fellow
carrying the wafer-biscuits.  Madam, your secret will be talked over by
those men at their club at the public-house to-night.  Jeames will tell
Chawles his notions about you over their pipes and pewter beer-pots.
Some people ought to have mutes for servants in Vanity Fair---mutes who
could not write. If you are guilty, tremble.  That fellow behind your
chair may be a Janissary with a bow-string in his plush breeches
pocket.  If you are not guilty, have a care of appearances, which are
as ruinous as guilt.

``Was Rebecca guilty or not?'' the Vehmgericht of the servants' hall had
pronounced against her.

And, I shame to say, she would not have got credit had they not
believed her to be guilty.  It was the sight of the Marquis of Steyne's
carriage-lamps at her door, contemplated by Raggles, burning in the
blackness of midnight, ``that kep him up,'' as he afterwards said, that
even more than Rebecca's arts and coaxings.

And so---guiltless very likely---she was writhing and pushing onward
towards what they call ``a position in society,'' and the servants were
pointing at her as lost and ruined.  So you see Molly, the housemaid,
of a morning, watching a spider in the doorpost lay his thread and
laboriously crawl up it, until, tired of the sport, she raises her
broom and sweeps away the thread and the artificer.

A day or two before Christmas, Becky, her husband and her son made
ready and went to pass the holidays at the seat of their ancestors at
Queen's Crawley.  Becky would have liked to leave the little brat
behind, and would have done so but for Lady Jane's urgent invitations
to the youngster, and the symptoms of revolt and discontent which
Rawdon manifested at her neglect of her son.  ``He's the finest boy in
England,'' the father said in a tone of reproach to her, ``and you don't
seem to care for him, Becky, as much as you do for your spaniel.  He
shan't bother you much; at home he will be away from you in the
nursery, and he shall go outside on the coach with me.''

``Where you go yourself because you want to smoke those filthy cigars,''
replied Mrs.\ Rawdon.

``I remember when you liked 'em though,'' answered the husband.

Becky laughed; she was almost always good-humoured. ``That was when I
was on my promotion, Goosey,'' she said.  ``Take Rawdon outside with you
and give him a cigar too if you like.''

Rawdon did not warm his little son for the winter's journey in this
way, but he and Briggs wrapped up the child in shawls and comforters,
and he was hoisted respectfully onto the roof of the coach in the dark
morning, under the lamps of the White Horse Cellar; and with no small
delight he watched the dawn rise and made his first journey to the
place which his father still called home. It was a journey of infinite
pleasure to the boy, to whom the incidents of the road afforded endless
interest, his father answering to him all questions connected with it
and telling him who lived in the great white house to the right, and
whom the park belonged to.  His mother, inside the vehicle, with her
maid and her furs, her wrappers, and her scent bottles, made such a
to-do that you would have thought she never had been in a stage-coach
before---much less, that she had been turned out of this very one to
make room for a paying passenger on a certain journey performed some
half-score years ago.

It was dark again when little Rawdon was wakened up to enter his
uncle's carriage at Mudbury, and he sat and looked out of it wondering
as the great iron gates flew open, and at the white trunks of the limes
as they swept by, until they stopped, at length, before the light
windows of the Hall, which were blazing and comfortable with Christmas
welcome.  The hall-door was flung open---a big fire was burning in the
great old fire-place---a carpet was down over the chequered black
flags---``It's the old Turkey one that used to be in the Ladies'
Gallery,'' thought Rebecca, and the next instant was kissing Lady Jane.

She and Sir Pitt performed the same salute with great gravity; but
Rawdon, having been smoking, hung back rather from his sister-in-law,
whose two children came up to their cousin; and, while Matilda held out
her hand and kissed him, Pitt Binkie Southdown, the son and heir, stood
aloof rather and examined him as a little dog does a big dog.

Then the kind hostess conducted her guests to the snug apartments
blazing with cheerful fires.  Then the young ladies came and knocked at
Mrs.\ Rawdon's door, under the pretence that they were desirous to be
useful, but in reality to have the pleasure of inspecting the contents
of her band and bonnet-boxes, and her dresses which, though black, were
of the newest London fashion.  And they told her how much the Hall was
changed for the better, and how old Lady Southdown was gone, and how
Pitt was taking his station in the county, as became a Crawley in fact.
Then the great dinner-bell having rung, the family assembled at dinner,
at which meal Rawdon Junior was placed by his aunt, the good-natured
lady of the house, Sir Pitt being uncommonly attentive to his
sister-in-law at his own right hand.

Little Rawdon exhibited a fine appetite and showed a gentlemanlike
behaviour.

``I like to dine here,'' he said to his aunt when he had completed his
meal, at the conclusion of which, and after a decent grace by Sir Pitt,
the younger son and heir was introduced, and was perched on a high
chair by the Baronet's side, while the daughter took possession of the
place and the little wine-glass prepared for her near her mother.  ``I
like to dine here,'' said Rawdon Minor, looking up at his relation's
kind face.

``Why?'' said the good Lady Jane.

``I dine in the kitchen when I am at home,'' replied Rawdon Minor, ``or
else with Briggs.'' But Becky was so engaged with the Baronet, her host,
pouring out a flood of compliments and delights and raptures, and
admiring young Pitt Binkie, whom she declared to be the most beautiful,
intelligent, noble-looking little creature, and so like his father,
that she did not hear the remarks of her own flesh and blood at the
other end of the broad shining table.

As a guest, and it being the first night of his arrival, Rawdon the
Second was allowed to sit up until the hour when tea being over, and a
great gilt book being laid on the table before Sir Pitt, all the
domestics of the family streamed in, and Sir Pitt read prayers.  It was
the first time the poor little boy had ever witnessed or heard of such
a ceremonial.

The house had been much improved even since the Baronet's brief reign,
and was pronounced by Becky to be perfect, charming, delightful, when
she surveyed it in his company.  As for little Rawdon, who examined it
with the children for his guides, it seemed to him a perfect palace of
enchantment and wonder.  There were long galleries, and ancient state
bedrooms, there were pictures and old China, and armour.  There were
the rooms in which Grandpapa died, and by which the children walked
with terrified looks.  ``Who was Grandpapa?'' he asked; and they told him
how he used to be very old, and used to be wheeled about in a
garden-chair, and they showed him the garden-chair one day rotting in
the out-house in which it had lain since the old gentleman had been
wheeled away yonder to the church, of which the spire was glittering
over the park elms.

The brothers had good occupation for several mornings in examining the
improvements which had been effected by Sir Pitt's genius and economy.
And as they walked or rode, and looked at them, they could talk without
too much boring each other.  And Pitt took care to tell Rawdon what a
heavy outlay of money these improvements had occasioned, and that a man
of landed and funded property was often very hard pressed for twenty
pounds. ``There is that new lodge-gate,'' said Pitt, pointing to it
humbly with the bamboo cane, ``I can no more pay for it before the
dividends in January than I can fly.''

``I can lend you, Pitt, till then,'' Rawdon answered rather ruefully; and
they went in and looked at the restored lodge, where the family arms
were just new scraped in stone, and where old Mrs.\ Lock, for the first
time these many long years, had tight doors, sound roofs, and whole
windows.



\chapter{Between Hampshire and London}

Sir Pitt Crawley had done more than repair fences and restore
dilapidated lodges on the Queen's Crawley estate. Like a wise man he
had set to work to rebuild the injured popularity of his house and stop
up the gaps and ruins in which his name had been left by his
disreputable and thriftless old predecessor.  He was elected for the
borough speedily after his father's demise; a magistrate, a member of
parliament, a county magnate and representative of an ancient family,
he made it his duty to show himself before the Hampshire public,
subscribed handsomely to the county charities, called assiduously upon
all the county folk, and laid himself out in a word to take that
position in Hampshire, and in the Empire afterwards, to which he
thought his prodigious talents justly entitled him.  Lady Jane was
instructed to be friendly with the Fuddlestones, and the Wapshots, and
the other famous baronets, their neighbours.  Their carriages might
frequently be seen in the Queen's Crawley avenue now; they dined pretty
frequently at the Hall (where the cookery was so good that it was clear
Lady Jane very seldom had a hand in it), and in return Pitt and his
wife most energetically dined out in all sorts of weather and at all
sorts of distances.  For though Pitt did not care for joviality, being
a frigid man of poor hearth and appetite, yet he considered that to be
hospitable and condescending was quite incumbent on his station, and
every time that he got a headache from too long an after-dinner
sitting, he felt that he was a martyr to duty.  He talked about crops,
corn-laws, politics, with the best country gentlemen. He (who had been
formerly inclined to be a sad free-thinker on these points) entered
into poaching and game preserving with ardour.  He didn't hunt; he
wasn't a hunting man; he was a man of books and peaceful habits; but he
thought that the breed of horses must be kept up in the country, and
that the breed of foxes must therefore be looked to, and for his part,
if his friend, Sir Huddlestone Fuddlestone, liked to draw his country
and meet as of old the F.  hounds used to do at Queen's Crawley, he
should be happy to see him there, and the gentlemen of the Fuddlestone
hunt.  And to Lady Southdown's dismay too he became more orthodox in
his tendencies every day; gave up preaching in public and attending
meeting-houses; went stoutly to church; called on the Bishop and all
the Clergy at Winchester; and made no objection when the Venerable
Archdeacon Trumper asked for a game of whist.  What pangs must have
been those of Lady Southdown, and what an utter castaway she must have
thought her son-in-law for permitting such a godless diversion!  And
when, on the return of the family from an oratorio at Winchester, the
Baronet announced to the young ladies that he should next year very
probably take them to the ``county balls,'' they worshipped him for his
kindness.  Lady Jane was only too obedient, and perhaps glad herself to
go.  The Dowager wrote off the direst descriptions of her daughter's
worldly behaviour to the authoress of the Washerwoman of Finchley
Common at the Cape; and her house in Brighton being about this time
unoccupied, returned to that watering-place, her absence being not very
much deplored by her children. We may suppose, too, that Rebecca, on
paying a second visit to Queen's Crawley, did not feel particularly
grieved at the absence of the lady of the medicine chest; though she
wrote a Christmas letter to her Ladyship, in which she respectfully
recalled herself to Lady Southdown's recollection, spoke with gratitude
of the delight which her Ladyship's conversation had given her on the
former visit, dilated on the kindness with which her Ladyship had
treated her in sickness, and declared that everything at Queen's
Crawley reminded her of her absent friend.

A great part of the altered demeanour and popularity of Sir Pitt
Crawley might have been traced to the counsels of that astute little
lady of Curzon Street.  ``You remain a Baronet---you consent to be a mere
country gentleman,'' she said to him, while he had been her guest in
London. ``No, Sir Pitt Crawley, I know you better.  I know your talents
and your ambition.  You fancy you hide them both, but you can conceal
neither from me.  I showed Lord Steyne your pamphlet on malt.  He was
familiar with it, and said it was in the opinion of the whole Cabinet
the most masterly thing that had appeared on the subject. The Ministry
has its eye upon you, and I know what you want.  You want to
distinguish yourself in Parliament; every one says you are the finest
speaker in England (for your speeches at Oxford are still remembered).
You want to be Member for the County, where, with your own vote and
your borough at your back, you can command anything.  And you want to
be Baron Crawley of Queen's Crawley, and will be before you die.  I saw
it all.  I could read your heart, Sir Pitt.  If I had a husband who
possessed your intellect as he does your name, I sometimes think I
should not be unworthy of him---but---but I am your kinswoman now,'' she
added with a laugh.  ``Poor little penniless, I have got a little
interest---and who knows, perhaps the mouse may be able to aid the
lion.'' Pitt Crawley was amazed and enraptured with her speech.  ``How
that woman comprehends me!'' he said. ``I never could get Jane to read
three pages of the malt pamphlet.  She has no idea that I have
commanding talents or secret ambition.  So they remember my speaking at
Oxford, do they? The rascals!  Now that I represent my borough and may
sit for the county, they begin to recollect me!  Why, Lord Steyne cut
me at the levee last year; they are beginning to find out that Pitt
Crawley is some one at last.  Yes, the man was always the same whom
these people neglected:  it was only the opportunity that was wanting,
and I will show them now that I can speak and act as well as write.
Achilles did not declare himself until they gave him the sword.  I hold
it now, and the world shall yet hear of Pitt Crawley.''

Therefore it was that this roguish diplomatist has grown so hospitable;
that he was so civil to oratorios and hospitals; so kind to Deans and
Chapters; so generous in giving and accepting dinners; so uncommonly
gracious to farmers on market-days; and so much interested about county
business; and that the Christmas at the Hall was the gayest which had
been known there for many a long day.

On Christmas Day a great family gathering took place. All the Crawleys
from the Rectory came to dine.  Rebecca was as frank and fond of Mrs.\ %
Bute as if the other had never been her enemy; she was affectionately
interested in the dear girls, and surprised at the progress which they
had made in music since her time, and insisted upon encoring one of the
duets out of the great song-books which Jim, grumbling, had been forced
to bring under his arm from the Rectory.  Mrs.\ Bute, perforce, was
obliged to adopt a decent demeanour towards the little adventuress---of
course being free to discourse with her daughters afterwards about the
absurd respect with which Sir Pitt treated his sister-in-law.  But Jim,
who had sat next to her at dinner, declared she was a trump, and one
and all of the Rector's family agreed that the little Rawdon was a fine
boy. They respected a possible baronet in the boy, between whom and the
title there was only the little sickly pale Pitt Binkie.

The children were very good friends.  Pitt Binkie was too little a dog
for such a big dog as Rawdon to play with; and Matilda being only a
girl, of course not fit companion for a young gentleman who was near
eight years old, and going into jackets very soon.  He took the command
of this small party at once---the little girl and the little boy
following him about with great reverence at such times as he
condescended to sport with them.  His happiness and pleasure in the
country were extreme.  The kitchen garden pleased him hugely, the
flowers moderately, but the pigeons and the poultry, and the stables
when he was allowed to visit them, were delightful objects to him.  He
resisted being kissed by the Misses Crawley, but he allowed Lady Jane
sometimes to embrace him, and it was by her side that he liked to sit
when, the signal to retire to the drawing-room being given, the ladies
left the gentlemen to their claret---by her side rather than by his
mother.  For Rebecca, seeing that tenderness was the fashion, called
Rawdon to her one evening and stooped down and kissed him in the
presence of all the ladies.

He looked her full in the face after the operation, trembling and
turning very red, as his wont was when moved.  ``You never kiss me at
home, Mamma,'' he said, at which there was a general silence and
consternation and a by no means pleasant look in Becky's eyes.

Rawdon was fond of his sister-in-law, for her regard for his son. Lady
Jane and Becky did not get on quite so well at this visit as on
occasion of the former one, when the Colonel's wife was bent upon
pleasing.  Those two speeches of the child struck rather a chill.
Perhaps Sir Pitt was rather too attentive to her.

But Rawdon, as became his age and size, was fonder of the society of
the men than of the women, and never wearied of accompanying his sire
to the stables, whither the Colonel retired to smoke his cigar---Jim,
the Rector's son, sometimes joining his cousin in that and other
amusements. He and the Baronet's keeper were very close friends, their
mutual taste for ``dawgs'' bringing them much together. On one day, Mr.\ %
James, the Colonel, and Horn, the keeper, went and shot pheasants,
taking little Rawdon with them.  On another most blissful morning,
these four gentlemen partook of the amusement of rat-hunting in a barn,
than which sport Rawdon as yet had never seen anything more noble.
They stopped up the ends of certain drains in the barn, into the other
openings of which ferrets were inserted, and then stood silently aloof,
with uplifted stakes in their hands, and an anxious little terrier (Mr.\ %
James's celebrated ``dawg'' Forceps, indeed) scarcely breathing from
excitement, listening motionless on three legs, to the faint squeaking
of the rats below. Desperately bold at last, the persecuted animals
bolted above-ground---the terrier accounted for one, the keeper for
another; Rawdon, from flurry and excitement, missed his rat, but on the
other hand he half-murdered a ferret.

But the greatest day of all was that on which Sir Huddlestone
Fuddlestone's hounds met upon the lawn at Queen's Crawley.

That was a famous sight for little Rawdon.  At half-past ten, Tom
Moody, Sir Huddlestone Fuddlestone's huntsman, was seen trotting up the
avenue, followed by the noble pack of hounds in a compact body---the
rear being brought up by the two whips clad in stained scarlet
frocks---light hard-featured lads on well-bred lean horses, possessing
marvellous dexterity in casting the points of their long heavy whips at
the thinnest part of any dog's skin who dares to straggle from the main
body, or to take the slightest notice, or even so much as wink, at the
hares and rabbits starting under their noses.

Next comes boy Jack, Tom Moody's son, who weighs five stone, measures
eight-and-forty inches, and will never be any bigger.  He is perched on
a large raw-boned hunter, half-covered by a capacious saddle.  This
animal is Sir Huddlestone Fuddlestone's favourite horse the Nob. Other
horses, ridden by other small boys, arrive from time to time, awaiting
their masters, who will come cantering on anon.

Tom Moody rides up to the door of the Hall, where he is welcomed by the
butler, who offers him drink, which he declines.  He and his pack then
draw off into a sheltered corner of the lawn, where the dogs roll on
the grass, and play or growl angrily at one another, ever and anon
breaking out into furious fight speedily to be quelled by Tom's voice,
unmatched at rating, or the snaky thongs of the whips.

Many young gentlemen canter up on thoroughbred hacks, spatter-dashed to
the knee, and enter the house to drink cherry-brandy and pay their
respects to the ladies, or, more modest and sportsmanlike, divest
themselves of their mud-boots, exchange their hacks for their hunters,
and warm their blood by a preliminary gallop round the lawn.  Then they
collect round the pack in the corner and talk with Tom Moody of past
sport, and the merits of Sniveller and Diamond, and of the state of the
country and of the wretched breed of foxes.

Sir Huddlestone presently appears mounted on a clever cob and rides up
to the Hall, where he enters and does the civil thing by the ladies,
after which, being a man of few words, he proceeds to business.  The
hounds are drawn up to the hall-door, and little Rawdon descends
amongst them, excited yet half-alarmed by the caresses which they
bestow upon him, at the thumps he receives from their waving tails, and
at their canine bickerings, scarcely restrained by Tom Moody's tongue
and lash.

Meanwhile, Sir Huddlestone has hoisted himself unwieldily on the Nob:
``Let's try Sowster's Spinney, Tom,'' says the Baronet, ``Farmer Mangle
tells me there are two foxes in it.'' Tom blows his horn and trots off,
followed by the pack, by the whips, by the young gents from Winchester,
by the farmers of the neighbourhood, by the labourers of the parish on
foot, with whom the day is a great holiday, Sir Huddlestone bringing up
the rear with Colonel Crawley, and the whole cortege disappears down
the avenue.

The Reverend Bute Crawley (who has been too modest to appear at the
public meet before his nephew's windows), whom Tom Moody remembers
forty years back a slender divine riding the wildest horses, jumping
the widest brooks, and larking over the newest gates in the country---his
Reverence, we say, happens to trot out from the Rectory Lane on his
powerful black horse just as Sir Huddlestone passes; he joins the
worthy Baronet.  Hounds and horsemen disappear, and little Rawdon
remains on the doorsteps, wondering and happy.

During the progress of this memorable holiday, little Rawdon, if he had
got no special liking for his uncle, always awful and cold and locked
up in his study, plunged in justice-business and surrounded by bailiffs
and farmers---has gained the good graces of his married and maiden
aunts, of the two little folks of the Hall, and of Jim of the Rectory,
whom Sir Pitt is encouraging to pay his addresses to one of the young
ladies, with an understanding doubtless that he shall be presented to
the living when it shall be vacated by his fox-hunting old sire.  Jim
has given up that sport himself and confines himself to a little
harmless duck- or snipe-shooting, or a little quiet trifling with the
rats during the Christmas holidays, after which he will return to the
University and try and not be plucked, once more.  He has already
eschewed green coats, red neckcloths, and other worldly ornaments, and
is preparing himself for a change in his condition.  In this cheap and
thrifty way Sir Pitt tries to pay off his debt to his family.

Also before this merry Christmas was over, the Baronet had screwed up
courage enough to give his brother another draft on his bankers, and
for no less a sum than a hundred pounds, an act which caused Sir Pitt
cruel pangs at first, but which made him glow afterwards to think
himself one of the most generous of men.  Rawdon and his son went away
with the utmost heaviness of heart.  Becky and the ladies parted with
some alacrity, however, and our friend returned to London to commence
those avocations with which we find her occupied when this chapter
begins. Under her care the Crawley House in Great Gaunt Street was
quite rejuvenescent and ready for the reception of Sir Pitt and his
family, when the Baronet came to London to attend his duties in
Parliament and to assume that position in the country for which his
vast genius fitted him.

For the first session, this profound dissembler hid his projects and
never opened his lips but to present a petition from Mudbury.  But he
attended assiduously in his place and learned thoroughly the routine
and business of the House.  At home he gave himself up to the perusal
of Blue Books, to the alarm and wonder of Lady Jane, who thought he was
killing himself by late hours and intense application.  And he made
acquaintance with the ministers, and the chiefs of his party,
determining to rank as one of them before many years were over.

Lady Jane's sweetness and kindness had inspired Rebecca with such a
contempt for her ladyship as the little woman found no small difficulty
in concealing.  That sort of goodness and simplicity which Lady Jane
possessed annoyed our friend Becky, and it was impossible for her at
times not to show, or to let the other divine, her scorn. Her presence,
too, rendered Lady Jane uneasy.  Her husband talked constantly with
Becky.  Signs of intelligence seemed to pass between them, and Pitt
spoke with her on subjects on which he never thought of discoursing
with Lady Jane.  The latter did not understand them, to be sure, but it
was mortifying to remain silent; still more mortifying to know that you
had nothing to say, and hear that little audacious Mrs.\ Rawdon dashing
on from subject to subject, with a word for every man, and a joke
always pat; and to sit in one's own house alone, by the fireside, and
watching all the men round your rival.

In the country, when Lady Jane was telling stories to the children, who
clustered about her knees (little Rawdon into the bargain, who was very
fond of her), and Becky came into the room, sneering with green
scornful eyes, poor Lady Jane grew silent under those baleful glances.
Her simple little fancies shrank away tremulously, as fairies in the
story-books, before a superior bad angel.  She could not go on,
although Rebecca, with the smallest inflection of sarcasm in her voice,
besought her to continue that charming story.  And on her side gentle
thoughts and simple pleasures were odious to Mrs.\ Becky; they discorded
with her; she hated people for liking them; she spurned children and
children-lovers.  ``I have no taste for bread and butter,'' she would
say, when caricaturing Lady Jane and her ways to my Lord Steyne.

``No more has a certain person for holy water,'' his lordship replied
with a bow and a grin and a great jarring laugh afterwards.

So these two ladies did not see much of each other except upon those
occasions when the younger brother's wife, having an object to gain
from the other, frequented her.  They my-loved and my-deared each other
assiduously, but kept apart generally, whereas Sir Pitt, in the midst
of his multiplied avocations, found daily time to see his sister-in-law.

On the occasion of his first Speaker's dinner, Sir Pitt took the
opportunity of appearing before his sister-in-law in his uniform---that
old diplomatic suit which he had worn when attache to the Pumpernickel
legation.

Becky complimented him upon that dress and admired him almost as much
as his own wife and children, to whom he displayed himself before he
set out.  She said that it was only the thoroughbred gentleman who
could wear the Court suit with advantage:  it was only your men of
ancient race whom the culotte courte became.  Pitt looked down with
complacency at his legs, which had not, in truth, much more symmetry or
swell than the lean Court sword which dangled by his side---looked down
at his legs, and thought in his heart that he was killing.

When he was gone, Mrs.\ Becky made a caricature of his figure, which she
showed to Lord Steyne when he arrived.  His lordship carried off the
sketch, delighted with the accuracy of the resemblance.  He had done
Sir Pitt Crawley the honour to meet him at Mrs.\ Becky's house and had
been most gracious to the new Baronet and member.  Pitt was struck too
by the deference with which the great Peer treated his sister-in-law,
by her ease and sprightliness in the conversation, and by the delight
with which the other men of the party listened to her talk. Lord Steyne
made no doubt but that the Baronet had only commenced his career in
public life, and expected rather anxiously to hear him as an orator; as
they were neighbours (for Great Gaunt Street leads into Gaunt Square,
whereof Gaunt House, as everybody knows, forms one side) my lord hoped
that as soon as Lady Steyne arrived in London she would have the honour
of making the acquaintance of Lady Crawley.  He left a card upon his
neighbour in the course of a day or two, having never thought fit to
notice his predecessor, though they had lived near each other for near
a century past.

In the midst of these intrigues and fine parties and wise and brilliant
personages Rawdon felt himself more and more isolated every day.  He
was allowed to go to the club more; to dine abroad with bachelor
friends; to come and go when he liked, without any questions being
asked.  And he and Rawdon the younger many a time would walk to Gaunt
Street and sit with the lady and the children there while Sir Pitt was
closeted with Rebecca, on his way to the House, or on  his return from
it.

The ex-Colonel would sit for hours in his brother's house very silent,
and thinking and doing as little as possible.  He was glad to be
employed of an errand; to go and make inquiries about a horse or a
servant, or to carve the roast mutton for the dinner of the children.
He was beat and cowed into laziness and submission. Delilah had
imprisoned him and cut his hair off, too.  The bold and reckless young
blood of ten-years back was subjugated and was turned into a torpid,
submissive, middle-aged, stout gentleman.

And poor Lady Jane was aware that Rebecca had captivated her husband,
although she and Mrs.\ Rawdon my-deared and my-loved each other every
day they met.



\chapter{Struggles and Trials}

Our friends at Brompton were meanwhile passing their Christmas after
their fashion and in a manner by no means too cheerful.

Out of the hundred pounds a year, which was about the amount of her
income, the Widow Osborne had been in the habit of giving up nearly
three-fourths to her father and mother, for the expenses of herself and
her little boy.  With \pounds 120 more, supplied by Jos, this family of four
people, attended by a single Irish servant who also did for Clapp and
his wife, might manage to live in decent comfort through the year, and
hold up their heads yet, and be able to give a friend a dish of tea
still, after the storms and disappointments of their early life. Sedley
still maintained his ascendency over the family of Mr.\ Clapp, his
ex-clerk.  Clapp remembered the time when, sitting on the edge of the
chair, he tossed off a bumper to the health of ``Mrs.\ S---------, Miss Emmy,
and Mr.\ Joseph in India,'' at the merchant's rich table in Russell
Square.  Time magnified the splendour of those recollections in the
honest clerk's bosom.  Whenever he came up from the kitchen-parlour to
the drawing-room and partook of tea or gin-and-water with Mr.\ Sedley,
he would say, ``This was not what you was accustomed to once, sir,'' and
as gravely and reverentially drink the health of the ladies as he had
done in the days of their utmost prosperity.  He thought Miss 'Melia's
playing the divinest music ever performed, and her the finest lady. He
never would sit down before Sedley at the club even, nor would he have
that gentleman's character abused by any member of the society.  He had
seen the first men in London shaking hands with Mr.\ S---------; he said, ``He'd
known him in times when Rothschild might be seen on 'Change with him
any day, and he owed him personally everythink.''

Clapp, with the best of characters and handwritings, had been able very
soon after his master's disaster to find other employment for himself.
``Such a little fish as me can swim in any bucket,'' he used to remark,
and a member of the house from which old Sedley had seceded was very
glad to make use of Mr.\ Clapp's services and to reward them with a
comfortable salary.  In fine, all Sedley's wealthy friends had dropped
off one by one, and this poor ex-dependent still remained faithfully
attached to him.

Out of the small residue of her income which Amelia kept back for
herself, the widow had need of all the thrift and care possible in
order to enable her to keep her darling boy dressed in such a manner as
became George Osborne's son, and to defray the expenses of the little
school to which, after much misgiving and reluctance and many secret
pangs and fears on her own part, she had been induced to send the lad.
She had sat up of nights conning lessons and spelling over crabbed
grammars and geography books in order to teach them to Georgy.  She had
worked even at the Latin accidence, fondly hoping that she might be
capable of instructing him in that language.  To part with him all day,
to send him out to the mercy of a schoolmaster's cane and his
schoolfellows' roughness, was almost like weaning him over again to
that weak mother, so tremulous and full of sensibility.  He, for his
part, rushed off to the school with the utmost happiness.  He was
longing for the change. That childish gladness wounded his mother, who
was herself so grieved to part with him.  She would rather have had him
more sorry, she thought, and then was deeply repentant within herself
for daring to be so selfish as to wish her own son to be unhappy.

Georgy made great progress in the school, which was kept by a friend of
his mother's constant admirer, the Rev. Mr.\ Binny.  He brought home
numberless prizes and testimonials of ability.  He told his mother
countless stories every night about his school-companions: and what a
fine fellow Lyons was, and what a sneak Sniffin was, and how Steel's
father actually supplied the meat for the establishment, whereas
Golding's mother came in a carriage to fetch him every Saturday, and
how Neat had straps to his trowsers---might he have straps?---and how
Bull Major was so strong (though only in Eutropius) that it was
believed he could lick the Usher, Mr.\ Ward, himself.  So Amelia learned
to know every one of the boys in that school as well as Georgy himself,
and of nights she used to help him in his exercises and puzzle her
little head over his lessons as eagerly as if she was herself going in
the morning into the presence of the master. Once, after a certain
combat with Master Smith, George came home to his mother with a black
eye, and bragged prodigiously to his parent and his delighted old
grandfather about his valour in the fight, in which, if the truth was
known he did not behave with particular heroism, and in which he
decidedly had the worst.  But Amelia has never forgiven that Smith to
this day, though he is now a peaceful apothecary near Leicester Square.

In these quiet labours and harmless cares the gentle widow's life was
passing away, a silver hair or two marking the progress of time on her
head and a line deepening ever so little on her fair forehead.  She
used to smile at these marks of time.  ``What matters it,'' she asked,
``For an old woman like me?'' All she hoped for was to live to see her
son great, famous, and glorious, as he deserved to be.  She kept his
copy-books, his drawings, and compositions, and showed them about in
her little circle as if they were miracles of genius.  She confided
some of these specimens to Miss Dobbin, to show them to Miss Osborne,
George's aunt, to show them to Mr.\ Osborne himself---to make that old
man repent of his cruelty and ill feeling towards him who was gone.
All her husband's faults and foibles she had buried in the grave with
him: she only remembered the lover, who had married her at all
sacrifices, the noble husband, so brave and beautiful, in whose arms
she had hung on the morning when he had gone away to fight, and die
gloriously for his king. From heaven the hero must be smiling down upon
that paragon of a boy whom he had left to comfort and console her. We
have seen how one of George's grandfathers (Mr.\ Osborne), in his easy
chair in Russell Square, daily grew more violent and moody, and how his
daughter, with her fine carriage, and her fine horses, and her name on
half the public charity-lists of the town, was a lonely, miserable,
persecuted old maid.  She thought again and again of the beautiful
little boy, her brother's son, whom she had seen.  She longed to be
allowed to drive in the fine carriage to the house in which he lived,
and she used to look out day after day as she took her solitary drive
in the park, in hopes that she might see him.  Her sister, the banker's
lady, occasionally condescended to pay her old home and companion a
visit in Russell Square.  She brought a couple of sickly children
attended by a prim nurse, and in a faint genteel giggling tone cackled
to her sister about her fine acquaintance, and how her little Frederick
was the image of Lord Claud Lollypop and her sweet Maria had been
noticed by the Baroness as they were driving in their donkey-chaise at
Roehampton.  She urged her to make her papa do something for the
darlings. Frederick she had determined should go into the Guards; and
if they made an elder son of him (and Mr.\ Bullock was positively
ruining and pinching himself to death to buy land), how was the darling
girl to be provided for? ``I expect YOU, dear,'' Mrs.\ Bullock would say,
``for of course my share of our Papa's property must go to the head of
the house, you know.  Dear Rhoda McMull will disengage the whole of the
Castletoddy property as soon as poor dear Lord Castletoddy dies, who is
quite epileptic; and little Macduff McMull will be Viscount
Castletoddy.  Both the Mr.\ Bludyers of Mincing Lane have settled their
fortunes on Fanny Bludyer's little boy.  My darling Frederick must
positively be an eldest son; and---and do ask Papa to bring us back his
account in Lombard Street, will you, dear? It doesn't look well, his
going to Stumpy and Rowdy's.'' After which kind of speeches, in which
fashion and the main chance were blended together, and after a kiss,
which was like the contact of an oyster---Mrs.\ Frederick Bullock would
gather her starched nurslings and simper back into her carriage.

Every visit which this leader of ton paid to her family was more
unlucky for her.  Her father paid more money into Stumpy and Rowdy's.
Her patronage became more and more insufferable.  The poor widow in the
little cottage at Brompton, guarding her treasure there, little knew
how eagerly some people coveted it.

On that night when Jane Osborne had told her father that she had seen
his grandson, the old man had made her no reply, but he had shown no
anger---and had bade her good-night on going himself to his room in
rather a kindly voice.  And he must have meditated on what she said and
have made some inquiries of the Dobbin family regarding her visit, for
a fortnight after it took place, he asked her where was her little
French watch and chain she used to wear?

``I bought it with my money, sir,'' she said in a great fright.

``Go and order another like it, or a better if you can get it,'' said the
old gentleman and lapsed again into silence.

Of late the Misses Dobbin more than once repeated their entreaties to
Amelia, to allow George to visit them. His aunt had shown her
inclination; perhaps his grandfather himself, they hinted, might be
disposed to be reconciled to him.  Surely, Amelia could not refuse such
advantageous chances for the boy.  Nor could she, but she acceded to
their overtures with a very heavy and suspicious heart, was always
uneasy during the child's absence from her, and welcomed him back as if
he was rescued out of some danger.  He brought back money and toys, at
which the widow looked with alarm and jealousy; she asked him always if
he had seen any gentleman---``Only old Sir William, who drove him about
in the four-wheeled chaise, and Mr.\ Dobbin, who arrived on the
beautiful bay horse in the afternoon---in the green coat and pink
neck-cloth, with the gold-headed whip, who promised to show him the
Tower of London and take him out with the Surrey hounds.'' At last, he
said, ``There was an old gentleman, with thick eyebrows, and a broad
hat, and large chain and seals.'' He came one day as the coachman was
lunging Georgy round the lawn on the gray pony.  ``He looked at me very
much.  He shook very much.  I said 'My name is Norval' after dinner.
My aunt began to cry.  She is always crying.'' Such was George's report
on that night.

Then Amelia knew that the boy had seen his grandfather; and looked out
feverishly for a proposal which she was sure would follow, and which
came, in fact, in a few days afterwards.  Mr.\ Osborne formally offered
to take the boy and make him heir to the fortune which he had intended
that his father should inherit.  He would make Mrs.\ George Osborne an
allowance, such as to assure her a decent competency.  If Mrs.\ George
Osborne proposed to marry again, as Mr.\ O.  heard was her intention, he
would not withdraw that allowance. But it must be understood that the
child would live entirely with his grandfather in Russell Square, or at
whatever other place Mr.\ O. should select, and that he would be
occasionally permitted to see Mrs.\ George Osborne at her own residence.
This message was brought or read to her in a letter one day, when her
mother was from home and her father absent as usual in the City.

She was never seen angry but twice or thrice in her life, and it was in
one of these moods that Mr.\ Osborne's attorney had the fortune to
behold her.  She rose up trembling and flushing very much as soon as,
after reading the letter, Mr.\ Poe handed it to her, and she tore the
paper into a hundred fragments, which she trod on.  ``I marry again!  I
take money to part from my child!  Who dares insult me by proposing
such a thing? Tell Mr.\ Osborne it is a cowardly letter, sir---a cowardly
letter---I will not answer it.  I wish you good morning, sir---and she
bowed me out of the room like a tragedy Queen,'' said the lawyer who
told the story.

Her parents never remarked her agitation on that day, and she never
told them of the interview.  They had their own affairs to interest
them, affairs which deeply interested this innocent and unconscious
lady.  The old gentleman, her father, was always dabbling in
speculation. We have seen how the wine company and the coal company had
failed him.  But, prowling about the City always eagerly and restlessly
still, he lighted upon some other scheme, of which he thought so well
that he embarked in it in spite of the remonstrances of Mr.\ Clapp, to
whom indeed he never dared to tell how far he had engaged himself in
it.  And as it was always Mr.\ Sedley's maxim not to talk about money
matters before women, they had no inkling of the misfortunes that were
in store for them until the unhappy old gentleman was forced to make
gradual confessions.

The bills of the little household, which had been settled weekly, first
fell into arrear.  The remittances had not arrived from India, Mr.\ %
Sedley told his wife with a disturbed face.  As she had paid her bills
very regularly hitherto, one or two of the tradesmen to whom the poor
lady was obliged to go round asking for time were very angry at a delay
to which they were perfectly used from more irregular customers.
Emmy's contribution, paid over cheerfully without any questions, kept
the little company in half-rations however.  And the first six months
passed away pretty easily, old Sedley still keeping up with the notion
that his shares must rise and that all would be well.

No sixty pounds, however, came to help the household at the end of the
half year, and it fell deeper and deeper into trouble---Mrs.\ Sedley, who
was growing infirm and was much shaken, remained silent or wept a great
deal with Mrs.\ Clapp in the kitchen.  The butcher was particularly
surly, the grocer insolent:  once or twice little Georgy had grumbled
about the dinners, and Amelia, who still would have been satisfied with
a slice of bread for her own dinner, could not but perceive that her
son was neglected and purchased little things out of her private purse
to keep the boy in health.

At last they told her, or told her such a garbled story as people in
difficulties tell.  One day, her own money having been received, and
Amelia about to pay it over, she, who had kept an account of the moneys
expended by her, proposed to keep a certain portion back out of her
dividend, having contracted engagements for a new suit for Georgy.

Then it came out that Jos's remittances were not paid, that the house
was in difficulties, which Amelia ought to have seen before, her mother
said, but she cared for nothing or nobody except Georgy. At this she
passed all her money across the table, without a word, to her mother,
and returned to her room to cry her eyes out. She had a great access of
sensibility too that day, when obliged to go and countermand the
clothes, the darling clothes on which she had set her heart for
Christmas Day, and the cut and fashion of which she had arranged in
many conversations with a small milliner, her friend.

Hardest of all, she had to break the matter to Georgy, who made a loud
outcry.  Everybody had new clothes at Christmas.  The others would
laugh at him.  He would have new clothes.  She had promised them to
him.  The poor widow had only kisses to give him.  She darned the old
suit in tears.  She cast about among her little ornaments to see if she
could sell anything to procure the desired novelties.  There was her
India shawl that Dobbin had sent her.  She remembered in former days
going with her mother to a fine India shop on Ludgate Hill, where the
ladies had all sorts of dealings and bargains in these articles.  Her
cheeks flushed and her eyes shone with pleasure as she thought of this
resource, and she kissed away George to school in the morning, smiling
brightly after him.  The boy felt that there was good news in her look.

Packing up her shawl in a handkerchief (another of the gifts of the
good Major), she hid them under her cloak and walked flushed and eager
all the way to Ludgate Hill, tripping along by the park wall and
running over the crossings, so that many a man turned as she hurried by
him and looked after her rosy pretty face.  She calculated how she
should spend the proceeds of her shawl---how, besides the clothes, she
would buy the books that he longed for, and pay his half-year's
schooling; and how she would buy a cloak for her father instead of that
old great-coat which he wore.  She was not mistaken as to the value of
the Major's gift.  It was a very fine and beautiful web, and the
merchant made a very good bargain when he gave her twenty guineas for
her shawl.

She ran on amazed and flurried with her riches to Darton's shop, in St.\ %
Paul's Churchyard, and there purchased the Parents' Assistant and the
Sandford and Merton Georgy longed for, and got into the coach there
with her parcel, and went home exulting.  And she pleased herself by
writing in the fly-leaf in her neatest little hand, ``George Osborne, A
Christmas gift from his affectionate mother.'' The books are extant to
this day, with the fair delicate superscription.

She was going from her own room with the books in her hand to place
them on George's table, where he might find them on his return from
school, when in the passage, she and her mother met.  The gilt bindings
of the seven handsome little volumes caught the old lady's eye.

``What are those?'' she said.

``Some books for Georgy,'' Amelia replied---``I---I promised them to him at
Christmas.''

``Books!'' cried the elder lady indignantly, ``Books, when the whole house
wants bread!  Books, when to keep you and your son in luxury, and your
dear father out of gaol, I've sold every trinket I had, the India shawl
from my back even down to the very spoons, that our tradesmen mightn't
insult us, and that Mr.\ Clapp, which indeed he is justly entitled,
being not a hard landlord, and a civil man, and a father, might have
his rent.  Oh, Amelia! you break my heart with your books and that boy
of yours, whom you are ruining, though part with him you will not.  Oh,
Amelia, may God send you a more dutiful child than I have had!  There's
Jos, deserts his father in his old age; and there's George, who might
be provided for, and who might be rich, going to school like a lord,
with a gold watch and chain round his neck---while my dear, dear old man
is without a sh---shilling.'' Hysteric sobs and cries ended Mrs.\ Sedley's
speech---it echoed through every room in the small house, whereof the
other female inmates heard every word of the colloquy.

``Oh, Mother, Mother!'' cried poor Amelia in reply. ``You told me
nothing---I---I promised him the books. I---I only sold my shawl this
morning.  Take the money---take everything''---and with quivering hands
she took out her silver, and her sovereigns---her precious golden
sovereigns, which she thrust into the hands of her mother, whence they
overflowed and tumbled, rolling down the stairs.

And then she went into her room, and sank down in despair and utter
misery.  She saw it all now.  Her selfishness was sacrificing the boy.
But for her he might have wealth, station, education, and his father's
place, which the elder George had forfeited for her sake. She had but
to speak the words, and her father was restored to competency and the
boy raised to fortune.  Oh, what a conviction it was to that tender and
stricken heart!



\chapter{Gaunt House}

All the world knows that Lord Steyne's town palace stands in Gaunt
Square, out of which Great Gaunt Street leads, whither we first
conducted Rebecca, in the time of the departed Sir Pitt Crawley.
Peering over the railings and through the black trees into the garden
of the Square, you see a few miserable governesses with wan-faced
pupils wandering round and round it, and round the dreary grass-plot in
the centre of which rises the statue of Lord Gaunt, who fought at
Minden, in a three-tailed wig, and otherwise habited like a Roman
Emperor.  Gaunt House occupies nearly a side of the Square. The
remaining three sides are composed of mansions that have passed away
into dowagerism---tall, dark houses, with window-frames of stone, or
picked out of a lighter red.  Little light seems to be behind those
lean, comfortless casements now, and hospitality to have passed away
from those doors as much as the laced lacqueys and link-boys of old
times, who used to put out their torches in the blank iron
extinguishers that still flank the lamps over the steps. Brass plates
have penetrated into the square---Doctors, the Diddlesex Bank Western
Branch---the English and European Reunion, \&c.---it has a dreary
look---nor is my Lord Steyne's palace less dreary.  All I have ever seen
of it is the vast wall in front, with the rustic columns at the great
gate, through which an old porter peers sometimes with a fat and gloomy
red face---and over the wall the garret and bedroom windows, and the
chimneys, out of which there seldom comes any smoke now.  For the
present Lord Steyne lives at Naples, preferring the view of the Bay and
Capri and Vesuvius to the dreary aspect of the wall in Gaunt Square.

A few score yards down New Gaunt Street, and leading into Gaunt Mews
indeed, is a little modest back door, which you would not remark from
that of any of the other stables.  But many a little close carriage has
stopped at that door, as my informant (little Tom Eaves, who knows
everything, and who showed me the place) told me. ``The Prince and
Perdita have been in and out of that door, sir,'' he had often told me;
``Marianne Clarke has entered it with the Duke of ---------. It conducts to
the famous petits appartements of Lord Steyne---one, sir, fitted up all
in ivory and white satin, another in ebony and black velvet; there is a
little banqueting-room taken from Sallust's house at Pompeii, and
painted by Cosway---a little private kitchen, in which every saucepan
was silver and all the spits were gold.  It was there that Egalite
Orleans roasted partridges on the night when he and the Marquis of
Steyne won a hundred thousand from a great personage at ombre.  Half of
the money went to the French Revolution, half to purchase Lord Gaunt's
Marquisate and Garter---and the remainder---'' but it forms no part of our
scheme to tell what became of the remainder, for every shilling of
which, and a great deal more, little Tom Eaves, who knows everybody's
affairs, is ready to account.

Besides his town palace, the Marquis had castles and palaces in various
quarters of the three kingdoms, whereof the descriptions may be found
in the road-books---Castle Strongbow, with its woods, on the Shannon
shore; Gaunt Castle, in Carmarthenshire, where Richard II was taken
prisoner---Gauntly Hall in Yorkshire, where I have been informed there
were two hundred silver teapots for the breakfasts of the guests of the
house, with everything to correspond in splendour; and Stillbrook in
Hampshire, which was my lord's farm, an humble place of residence, of
which we all remember the wonderful furniture which was sold at my
lord's demise by a late celebrated auctioneer.

The Marchioness of Steyne was of the renowned and ancient family of the
Caerlyons, Marquises of Camelot, who have preserved the old faith ever
since the conversion of the venerable Druid, their first ancestor, and
whose pedigree goes far beyond the date of the arrival of King Brute in
these islands.  Pendragon is the title of the eldest son of the house.
The sons have been called Arthurs, Uthers, and Caradocs, from
immemorial time. Their heads have fallen in many a loyal conspiracy.
Elizabeth chopped off the head of the Arthur of her day, who had been
Chamberlain to Philip and Mary, and carried letters between the Queen
of Scots and her uncles the Guises.  A cadet of the house was an
officer of the great Duke and distinguished in the famous Saint
Bartholomew conspiracy.  During the whole of Mary's confinement, the
house of Camelot conspired in her behalf. It was as much injured by its
charges in fitting out an armament against the Spaniards, during the
time of the Armada, as by the fines and confiscations levied on it by
Elizabeth for harbouring of priests, obstinate recusancy, and popish
misdoings.  A recreant of James's time was momentarily perverted from
his religion by the arguments of that great theologian, and the
fortunes of the family somewhat restored by his timely weakness.  But
the Earl of Camelot, of the reign of Charles, returned to the old creed
of his family, and they continued to fight for it, and ruin themselves
for it, as long as there was a Stuart left to head or to instigate a
rebellion.

Lady Mary Caerlyon was brought up at a Parisian convent; the Dauphiness
Marie Antoinette was her godmother.  In the pride of her beauty she had
been married---sold, it was said---to Lord Gaunt, then at Paris, who won
vast sums from the lady's brother at some of Philip of Orleans's
banquets.  The Earl of Gaunt's famous duel with the Count de la Marche,
of the Grey Musqueteers, was attributed by common report to the
pretensions of that officer (who had been a page, and remained a
favourite of the Queen) to the hand of the beautiful Lady Mary
Caerlyon.  She was married to Lord Gaunt while the Count lay ill of his
wound, and came to dwell at Gaunt House, and to figure for a short time
in the splendid Court of the Prince of Wales.  Fox had toasted her.
Morris and Sheridan had written songs about her.  Malmesbury had made
her his best bow; Walpole had pronounced her charming; Devonshire had
been almost jealous of her; but she was scared by the wild pleasures
and gaieties of the society into which she was flung, and after she had
borne a couple of sons, shrank away into a life of devout seclusion.
No wonder that my Lord Steyne, who liked pleasure and cheerfulness, was
not often seen after their marriage by the side of this trembling,
silent, superstitious, unhappy lady.

The before-mentioned Tom Eaves (who has no part in this history, except
that he knew all the great folks in London, and the stories and
mysteries of each family) had further information regarding my Lady
Steyne, which may or may not be true.  ``The humiliations,'' Tom used to
say, ``which that woman has been made to undergo, in her own house, have
been frightful; Lord Steyne has made her sit down to table with women
with whom I would rather die than allow Mrs.\ Eaves to associate---with
Lady Crackenbury, with Mrs.\ Chippenham, with Madame de la Cruchecassee,
the French secretary's wife (from every one of which ladies Tom
Eaves---who would have sacrificed his wife for knowing them---was too
glad to get a bow or a dinner) with the REIGNING FAVOURITE in a word.
And do you suppose that that woman, of that family, who are as proud as
the Bourbons, and to whom the Steynes are but lackeys, mushrooms of
yesterday (for after all, they are not of the Old Gaunts, but of a
minor and doubtful branch of the house); do you suppose, I say (the
reader must bear in mind that it is always Tom Eaves who speaks) that
the Marchioness of Steyne, the haughtiest woman in England, would bend
down to her husband so submissively if there were not some cause? Pooh!
I tell you there are secret reasons. I tell you that, in the
emigration, the Abbe de la Marche who was here and was employed in the
Quiberoon business with Puisaye and Tinteniac, was the same Colonel of
Mousquetaires Gris with whom Steyne fought in the year '86---that he and
the Marchioness met again---that it was after the Reverend Colonel was
shot in Brittany that Lady Steyne took to those extreme practices of
devotion which she carries on now; for she is closeted with her
director every day---she is at service at Spanish Place, every morning,
I've watched her there---that is, I've happened to be passing there---and
depend on it, there's a mystery in her case. People are not so unhappy
unless they have something to repent of,'' added Tom Eaves with a
knowing wag of his head; ``and depend on it, that woman would not be so
submissive as she is if the Marquis had not some sword to hold over
her.''

So, if Mr.\ Eaves's information be correct, it is very likely that this
lady, in her high station, had to submit to many a private indignity
and to hide many secret griefs under a calm face.  And let us, my
brethren who have not our names in the Red Book, console ourselves by
thinking comfortably how miserable our betters may be, and that
Damocles, who sits on satin cushions and is served on gold plate, has
an awful sword hanging over his head in the shape of a bailiff, or an
hereditary disease, or a family secret, which peeps out every now and
then from the embroidered arras in a ghastly manner, and will be sure
to drop one day or the other in the right place.

In comparing, too, the poor man's situation with that of the great,
there is (always according to Mr.\ Eaves) another source of comfort for
the former.  You who have little or no patrimony to bequeath or to
inherit, may be on good terms with your father or your son, whereas the
heir of a great prince, such as my Lord Steyne, must naturally be angry
at being kept out of his kingdom, and eye the occupant of it with no
very agreeable glances. ``Take it as a rule,'' this sardonic old Eaves
would say, ``the fathers and elder sons of all great families hate each
other.  The Crown Prince is always in opposition to the crown or
hankering after it.  Shakespeare knew the world, my good sir, and when
he describes Prince Hal (from whose family the Gaunts pretend to be
descended, though they are no more related to John of Gaunt than you
are) trying on his father's coronet, he gives you a natural description
of all heirs apparent. If you were heir to a dukedom and a thousand
pounds a day, do you mean to say you would not wish for possession?
Pooh!  And it stands to reason that every great man, having experienced
this feeling towards his father, must be aware that his son entertains
it towards himself; and so they can't but be suspicious and hostile.

``Then again, as to the feeling of elder towards younger sons.  My dear
sir, you ought to know that every elder brother looks upon the cadets
of the house as his natural enemies, who deprive him of so much ready
money which ought to be his by right.  I have often heard George Mac
Turk, Lord Bajazet's eldest son, say that if he had his will when he
came to the title, he would do what the sultans do, and clear the
estate by chopping off all his younger brothers' heads at once; and so
the case is, more or less, with them all.  I tell you they are all
Turks in their hearts.  Pooh! sir, they know the world.'' And here,
haply, a great man coming up, Tom Eaves's hat would drop off his head,
and he would rush forward with a bow and a grin, which showed that he
knew the world too---in the Tomeavesian way, that is.  And having laid
out every shilling of his fortune on an annuity, Tom could afford to
bear no malice to his nephews and nieces, and to have no other feeling
with regard to his betters but a constant and generous desire to dine
with them.

Between the Marchioness and the natural and tender regard of mother for
children, there was that cruel barrier placed of difference of faith.
The very love which she might feel for her sons only served to render
the timid and pious lady more fearful and unhappy.  The gulf which
separated them was fatal and impassable.  She could not stretch her
weak arms across it, or draw her children over to that side away from
which her belief told her there was no safety. During the youth of his
sons, Lord Steyne, who was a good scholar and amateur casuist, had no
better sport in the evening after dinner in the country than in setting
the boys' tutor, the Reverend Mr.\ Trail (now my Lord Bishop of Ealing)
on her ladyship's director, Father Mole, over their wine, and in
pitting Oxford against St.\ Acheul.  He cried ``Bravo, Latimer!  Well
said, Loyola!'' alternately; he promised Mole a bishopric if he would
come over, and vowed he would use all his influence to get Trail a
cardinal's hat if he would secede.  Neither divine allowed himself to
be conquered, and though the fond mother hoped that her youngest and
favourite son would be reconciled to her church---his mother church---a
sad and awful disappointment awaited the devout lady---a disappointment
which seemed to be a judgement upon her for the sin of her marriage.

My Lord Gaunt married, as every person who frequents the Peerage knows,
the Lady Blanche Thistlewood, a daughter of the noble house of
Bareacres, before mentioned in this veracious history.  A wing of Gaunt
House was assigned to this couple; for the head of the family chose to
govern it, and while he reigned to reign supreme; his son and heir,
however, living little at home, disagreeing with his wife, and
borrowing upon post-obits such moneys as he required beyond the very
moderate sums which his father was disposed to allow him.  The Marquis
knew every shilling of his son's debts. At his lamented demise, he was
found himself to be possessor of many of his heir's bonds, purchased
for their benefit, and devised by his Lordship to the children of his
younger son.

As, to my Lord Gaunt's dismay, and the chuckling delight of his natural
enemy and father, the Lady Gaunt had no children---the Lord George Gaunt
was desired to return from Vienna, where he was engaged in waltzing and
diplomacy, and to contract a matrimonial alliance with the Honourable
Joan, only daughter of John Johnes, First Baron Helvellyn, and head of
the firm of Jones, Brown, and Robinson, of Threadneedle Street,
Bankers; from which union sprang several sons and daughters, whose
doings do not appertain to this story.

The marriage at first was a happy and prosperous one. My Lord George
Gaunt could not only read, but write pretty correctly.  He spoke French
with considerable fluency; and was one of the finest waltzers in
Europe.  With these talents, and his interest at home, there was little
doubt that his lordship would rise to the highest dignities in his
profession.  The lady, his wife, felt that courts were her sphere, and
her wealth enabled her to receive splendidly in those continental towns
whither her husband's diplomatic duties led him. There was talk of
appointing him minister, and bets were laid at the Travellers' that he
would be ambassador ere long, when of a sudden, rumours arrived of the
secretary's extraordinary behaviour. At a grand diplomatic dinner given
by his chief, he had started up and declared that a pate de foie gras
was poisoned.  He went to a ball at the hotel of the Bavarian envoy,
the Count de Springbock-Hohenlaufen, with his head shaved and dressed
as a Capuchin friar. It was not a masked ball, as some folks wanted to
persuade you.  It was something queer, people whispered.  His
grandfather was so.  It was in the family.

His wife and family returned to this country and took up their abode at
Gaunt House.  Lord George gave up his post on the European continent,
and was gazetted to Brazil.  But people knew better; he never returned
from that Brazil expedition---never died there---never lived there---never
was there at all.  He was nowhere; he was gone out altogether.
``Brazil,'' said one gossip to another, with a grin---``Brazil is St.\ %
John's Wood.  Rio de Janeiro is a cottage surrounded by four walls, and
George Gaunt is accredited to a keeper, who has invested him with the
order of the Strait-Waistcoat.'' These are the kinds of epitaphs which
men pass over one another in Vanity Fair.

Twice or thrice in a week, in the earliest morning, the poor mother
went for her sins and saw the poor invalid. Sometimes he laughed at her
(and his laughter was more pitiful than to hear him cry); sometimes she
found the brilliant dandy diplomatist of the Congress of Vienna
dragging about a child's toy, or nursing the keeper's baby's doll.
Sometimes he knew her and Father Mole, her director and companion;
oftener he forgot her, as he had done wife, children, love, ambition,
vanity.  But he remembered his dinner-hour, and used to cry if his
wine-and-water was not strong enough.

It was the mysterious taint of the blood; the poor mother had brought
it from her own ancient race.  The evil had broken out once or twice in
the father's family, long before Lady Steyne's sins had begun, or her
fasts and tears and penances had been offered in their expiation.  The
pride of the race was struck down as the first-born of Pharaoh.  The
dark mark of fate and doom was on the threshold---the tall old
threshold surmounted by coronets and caned heraldry.

The absent lord's children meanwhile prattled and grew on quite
unconscious that the doom was over them too.  First they talked of
their father and devised plans against his return.  Then the name of
the living dead man was less frequently in their mouth---then not
mentioned at all.  But the stricken old grandmother trembled to think
that these too were the inheritors of their father's shame as well as
of his honours, and watched sickening for the day when the awful
ancestral curse should come down on them.

This dark presentiment also haunted Lord Steyne.  He tried to lay the
horrid bedside ghost in Red Seas of wine and jollity, and lost sight of
it sometimes in the crowd and rout of his pleasures.  But it always
came back to him when alone, and seemed to grow more threatening with
years.  ``I have taken your son,'' it said, ``why not you? I may shut you
up in a prison some day like your son George.  I may tap you on the
head to-morrow, and away go pleasure and honours, feasts and beauty,
friends, flatterers, French cooks, fine horses and houses---in exchange
for a prison, a keeper, and a straw mattress like George Gaunt's.'' And
then my lord would defy the ghost which threatened him, for he knew of
a remedy by which he could baulk his enemy.

So there was splendour and wealth, but no great happiness perchance,
behind the tall caned portals of Gaunt House with its smoky coronets
and ciphers.  The feasts there were of the grandest in London, but
there was not overmuch content therewith, except among the guests who
sat at my lord's table.  Had he not been so great a Prince very few
possibly would have visited him; but in Vanity Fair the sins of very
great personages are looked at indulgently.  ``Nous regardons a deux
fois'' (as the French lady said) before we condemn a person of my lord's
undoubted quality.  Some notorious carpers and squeamish moralists
might be sulky with Lord Steyne, but they were glad enough to come when
he asked them.

``Lord Steyne is really too bad,'' Lady Slingstone said, ``but everybody
goes, and of course I shall see that my girls come to no harm.'' ``His
lordship is a man to whom I owe much, everything in life,'' said the
Right Reverend Doctor Trail, thinking that the Archbishop was rather
shaky, and Mrs.\ Trail and the young ladies would as soon have missed
going to church as to one of his lordship's parties.  ``His morals are
bad,'' said little Lord Southdown to his sister, who meekly
expostulated, having heard terrific legends from her mamma with respect
to the doings at Gaunt House; ``but hang it, he's got the best dry
Sillery in Europe!'' And as for Sir Pitt Crawley, Bart.---Sir Pitt that
pattern of decorum, Sir Pitt who had led off at missionary meetings---he
never for one moment thought of not going too.  ``Where you see such
persons as the Bishop of Ealing and the Countess of Slingstone, you may
be pretty sure, Jane,'' the Baronet would say, ``that we cannot be wrong.
The great rank and station of Lord Steyne put him in a position to
command people in our station in life.  The Lord Lieutenant of a
County, my dear, is a respectable man.  Besides, George Gaunt and I
were intimate in early life; he was my junior when we were attaches at
Pumpernickel together.''

In a word everybody went to wait upon this great man---everybody who was
asked, as you the reader (do not say nay) or I the writer hereof would
go if we had an invitation.



\chapter{In Which the Reader Is Introduced to the Very Best of Company}

At last Becky's kindness and attention to the chief of her husband's
family were destined to meet with an exceeding great reward, a reward
which, though certainly somewhat unsubstantial, the little woman
coveted with greater eagerness than more positive benefits. If she did
not wish to lead a virtuous life, at least she desired to enjoy a
character for virtue, and we know that no lady in the genteel world can
possess this desideratum, until she has put on a train and feathers and
has been presented to her Sovereign at Court. From that august
interview they come out stamped as honest women. The Lord Chamberlain
gives them a certificate of virtue.  And as dubious goods or letters
are passed through an oven at quarantine, sprinkled with aromatic
vinegar, and then pronounced clean, many a lady, whose reputation would
be doubtful otherwise and liable to give infection, passes through the
wholesome ordeal of the Royal presence and issues from it free from all
taint.

It might be very well for my Lady Bareacres, my Lady Tufto, Mrs.\ Bute
Crawley in the country, and other ladies who had come into contact with
Mrs.\ Rawdon Crawley to cry fie at the idea of the odious little
adventuress making her curtsey before the Sovereign, and to declare
that, if dear good Queen Charlotte had been alive, she never would have
admitted such an extremely ill-regulated personage into her chaste
drawing-room.  But when we consider that it was the First Gentleman in
Europe in whose high presence Mrs.\ Rawdon passed her examination, and
as it were, took her degree in reputation, it surely must be flat
disloyalty to doubt any more about her virtue.  I, for my part, look
back with love and awe to that Great Character in history.  Ah, what a
high and noble appreciation of Gentlewomanhood there must have been in
Vanity Fair, when that revered and august being was invested, by the
universal acclaim of the refined and educated portion of this empire,
with the title of Premier Gentilhomme of his Kingdom.  Do you remember,
dear M---, oh friend of my youth, how one blissful night five-and-twenty
years since, the ``Hypocrite'' being acted, Elliston being manager,
Dowton and Liston performers, two boys had leave from their loyal
masters to go out from Slaughter-House School where they were educated
and to appear on Drury Lane stage, amongst a crowd which assembled
there to greet the king.  THE KING? There he was. Beefeaters were
before the august box; the Marquis of Steyne (Lord of the Powder
Closet) and other great officers of state were behind the chair on
which he sat, HE sat---florid of face, portly of person, covered with
orders, and in a rich curling head of hair---how we sang God save him!
How the house rocked and shouted with that magnificent music.  How they
cheered, and cried, and waved handkerchiefs.  Ladies wept; mothers
clasped their children; some fainted with emotion.  People were
suffocated in the pit, shrieks and groans rising up amidst the writhing
and shouting mass there of his people who were, and indeed showed
themselves almost to be, ready to die for him.  Yes, we saw him.  Fate
cannot deprive us of THAT.  Others have seen Napoleon.  Some few still
exist who have beheld Frederick the Great, Doctor Johnson, Marie
Antoinette, \&c.---be it our reasonable boast to our children, that we
saw George the Good, the Magnificent, the Great.

Well, there came a happy day in Mrs.\ Rawdon Crawley's existence when
this angel was admitted into the paradise of a Court which she coveted,
her sister-in-law acting as her godmother.  On the appointed day, Sir
Pitt and his lady, in their great family carriage (just newly built,
and ready for the Baronet's assumption of the office of High Sheriff of
his county), drove up to the little house in Curzon Street, to the
edification of Raggles, who was watching from his greengrocer's shop,
and saw fine plumes within, and enormous bunches of flowers in the
breasts of the new livery-coats of the footmen.

Sir Pitt, in a glittering uniform, descended and went into Curzon
Street, his sword between his legs.  Little Rawdon stood with his face
against the parlour window-panes, smiling and nodding with all his
might to his aunt in the carriage within; and presently Sir Pitt issued
forth from the house again, leading forth a lady with grand feathers,
covered in a white shawl, and holding up daintily a train of
magnificent brocade.  She stepped into the vehicle as if she were a
princess and accustomed all her life to go to Court, smiling graciously
on the footman at the door and on Sir Pitt, who followed her into the
carriage.

Then Rawdon followed in his old Guards' uniform, which had grown
woefully shabby, and was much too tight.  He was to have followed the
procession and waited upon his sovereign in a cab, but that his
good-natured sister-in-law insisted that they should be a family party.
The coach was large, the ladies not very big, they would hold their
trains in their laps---finally, the four went fraternally together, and
their carriage presently joined the line of royal equipages which was
making its way down Piccadilly and St.\ James's Street, towards the old
brick palace where the Star of Brunswick was in waiting to receive his
nobles and gentlefolks.

Becky felt as if she could bless the people out of the carriage
windows, so elated was she in spirit, and so strong a sense had she of
the dignified position which she had at last attained in life. Even our
Becky had her weaknesses, and as one often sees how men pride
themselves upon excellences which others are slow to perceive: how, for
instance, Comus firmly believes that he is the greatest tragic actor in
England; how Brown, the famous novelist, longs to be considered, not a
man of genius, but a man of fashion; while Robinson, the great lawyer,
does not in the least care about his reputation in Westminster Hall,
but believes himself incomparable across country and at a five-barred
gate---so to be, and to be thought, a respectable woman was Becky's aim
in life, and she got up the genteel with amazing assiduity, readiness,
and success.  We have said, there were times when she believed herself
to be a fine lady and forgot that there was no money in the chest at
home---duns round the gate, tradesmen to coax and wheedle---no ground to
walk upon, in a word.  And as she went to Court in the carriage, the
family carriage, she adopted a demeanour so grand, self-satisfied,
deliberate, and imposing that it made even Lady Jane laugh.  She walked
into the royal apartments with a toss of the head which would have
befitted an empress, and I have no doubt had she been one, she would
have become the character perfectly.

We are authorized to state that Mrs.\ Rawdon Crawley's costume de cour
on the occasion of her presentation to the Sovereign was of the most
elegant and brilliant description.  Some ladies we may have seen---we
who wear stars and cordons and attend the St.\ James's assemblies, or
we, who, in muddy boots, dawdle up and down Pall Mall and peep into the
coaches as they drive up with the great folks in their feathers---some
ladies of fashion, I say, we may have seen, about two o'clock of the
forenoon of a levee day, as the laced-jacketed band of the Life Guards
are blowing triumphal marches seated on those prancing music-stools,
their cream-coloured chargers---who are by no means lovely and enticing
objects at that early period of noon.  A stout countess of sixty,
decolletee, painted, wrinkled with rouge up to her drooping eyelids,
and diamonds twinkling in her wig, is a wholesome and edifying, but not
a pleasant sight.  She has the faded look of a St.\ James's Street
illumination, as it may be seen of an early morning, when half the
lamps are out, and the others are blinking wanly, as if they were about
to vanish like ghosts before the dawn.  Such charms as those of which
we catch glimpses while her ladyship's carriage passes should appear
abroad at night alone.  If even Cynthia looks haggard of an afternoon,
as we may see her sometimes in the present winter season, with Phoebus
staring her out of countenance from the opposite side of the heavens,
how much more can old Lady Castlemouldy keep her head up when the sun
is shining full upon it through the chariot windows, and showing all
the chinks and crannies with which time has marked her face!  No.
Drawing-rooms should be announced for November, or the first foggy day,
or the elderly sultanas of our Vanity Fair should drive up in closed
litters, descend in a covered way, and make their curtsey to the
Sovereign under the protection of lamplight.

Our beloved Rebecca had no need, however, of any such a friendly halo
to set off her beauty.  Her complexion could bear any sunshine as yet,
and her dress, though if you were to see it now, any present lady of
Vanity Fair would pronounce it to be the most foolish and preposterous
attire ever worn, was as handsome in her eyes and those of the public,
some five-and-twenty years since, as the most brilliant costume of the
most famous beauty of the present season. A score of years hence that
too, that milliner's wonder, will have passed into the domain of the
absurd, along with all previous vanities.  But we are wandering too
much.  Mrs.\ Rawdon's dress was pronounced to be charmante on the
eventful day of her presentation. Even good little Lady Jane was forced
to acknowledge this effect, as she looked at her kinswoman, and owned
sorrowfully to herself that she was quite inferior in taste to Mrs.\ %
Becky.

She did not know how much care, thought, and genius Mrs.\ Rawdon had
bestowed upon that garment.  Rebecca had as good taste as any milliner
in Europe, and such a clever way of doing things as Lady Jane little
understood. The latter quickly spied out the magnificence of the
brocade of Becky's train, and the splendour of the lace on her dress.

The brocade was an old remnant, Becky said; and as for the lace, it was
a great bargain.  She had had it these hundred years.

``My dear Mrs.\ Crawley, it must have cost a little fortune,'' Lady Jane
said, looking down at her own lace, which was not nearly so good; and
then examining the quality of the ancient brocade which formed the
material of Mrs.\ Rawdon's Court dress, she felt inclined to say that
she could not afford such fine clothing, but checked that speech, with
an effort, as one uncharitable to her kinswoman.

And yet, if Lady Jane had known all, I think even her kindly temper
would have failed her.  The fact is, when she was putting Sir Pitt's
house in order, Mrs.\ Rawdon had found the lace and the brocade in old
wardrobes, the property of the former ladies of the house, and had
quietly carried the goods home, and had suited them to her own little
person.  Briggs saw her take them, asked no questions, told no stories;
but I believe quite sympathised with her on this matter, and so would
many another honest woman.

And the diamonds---``Where the doose did you get the diamonds, Becky?''
said her husband, admiring some jewels which he had never seen before
and which sparkled in her ears and on her neck with brilliance and
profusion.

Becky blushed a little and looked at him hard for a moment.  Pitt
Crawley blushed a little too, and looked out of window.  The fact is,
he had given her a very small portion of the brilliants; a pretty
diamond clasp, which confined a pearl necklace which she wore---and the
Baronet had omitted to mention the circumstance to his lady.

Becky looked at her husband, and then at Sir Pitt, with an air of saucy
triumph---as much as to say, ``Shall I betray you?''

``Guess!'' she said to her husband.  ``Why, you silly man,'' she continued,
``where do you suppose I got them?---all except the little clasp, which a
dear friend of mine gave me long ago.  I hired them, to be sure.  I
hired them at Mr.\ Polonius's, in Coventry Street. You don't suppose
that all the diamonds which go to Court belong to the wearers; like
those beautiful stones which Lady Jane has, and which are much
handsomer than any which I have, I am certain.''

``They are family jewels,'' said Sir Pitt, again looking uneasy.  And in
this family conversation the carriage rolled down the street, until its
cargo was finally discharged at the gates of the palace where the
Sovereign was sitting in state.

The diamonds, which had created Rawdon's admiration, never went back to
Mr.\ Polonius, of Coventry Street, and that gentleman never applied for
their restoration, but they retired into a little private repository,
in an old desk, which Amelia Sedley had given her years and years ago,
and in which Becky kept a number of useful and, perhaps, valuable
things, about which her husband knew nothing. To know nothing, or
little, is in the nature of some husbands.  To hide, in the nature of
how many women? Oh, ladies! how many of you have surreptitious
milliners' bills? How many of you have gowns and bracelets which you
daren't show, or which you wear trembling?---trembling, and coaxing
with smiles the husband by your side, who does not know the new velvet
gown from the old one, or the new bracelet from last year's, or has any
notion that the ragged-looking yellow lace scarf cost forty guineas and
that Madame Bobinot is writing dunning letters every week for the money!

Thus Rawdon knew nothing about the brilliant diamond ear-rings, or the
superb brilliant ornament which decorated the fair bosom of his lady;
but Lord Steyne, who was in his place at Court, as Lord of the Powder
Closet, and one of the great dignitaries and illustrious defences of
the throne of England, and came up with all his stars, garters,
collars, and cordons, and paid particular attention to the little
woman, knew whence the jewels came and who paid for them.

As he bowed over her he smiled, and quoted the hackneyed and beautiful
lines from The Rape of the Lock about Belinda's diamonds, ``which Jews
might kiss and infidels adore.''

``But I hope your lordship is orthodox,'' said the little lady with a
toss of her head.  And many ladies round about whispered and talked,
and many gentlemen nodded and whispered, as they saw what marked
attention the great nobleman was paying to the little adventuress.

What were the circumstances of the interview between Rebecca Crawley,
nee Sharp, and her Imperial Master, it does not become such a feeble
and inexperienced pen as mine to attempt to relate.  The dazzled eyes
close before that Magnificent Idea.  Loyal respect and decency tell
even the imagination not to look too keenly and audaciously about the
sacred audience-chamber, but to back away rapidly, silently, and
respectfully, making profound bows out of the August Presence.

This may be said, that in all London there was no more loyal heart than
Becky's after this interview.  The name of her king was always on her
lips, and he was proclaimed by her to be the most charming of men.  She
went to Colnaghi's and ordered the finest portrait of him that art had
produced, and credit could supply.  She chose that famous one in which
the best of monarchs is represented in a frock-coat with a fur collar,
and breeches and silk stockings, simpering on a sofa from under his
curly brown wig.  She had him painted in a brooch and wore it---indeed
she amused and somewhat pestered her acquaintance with her perpetual
talk about his urbanity and beauty. Who knows!  Perhaps the little
woman thought she might play the part of a Maintenon or a Pompadour.

But the finest sport of all after her presentation was to hear her talk
virtuously.  She had a few female acquaintances, not, it must be owned,
of the very highest reputation in Vanity Fair.  But being made an
honest woman of, so to speak, Becky would not consort any longer with
these dubious ones, and cut Lady Crackenbury when the latter nodded to
her from her opera-box, and gave Mrs.\ Washington White the go-by in the
Ring.  ``One must, my dear, show one is somebody,'' she said.  ``One
mustn't be seen with doubtful people.  I pity Lady Crackenbury from my
heart, and Mrs.\ Washington White may be a very good-natured person.
YOU may go and dine with them, as you like your rubber.  But I mustn't,
and won't; and you will have the goodness to tell Smith to say I am not
at home when either of them calls.''

The particulars of Becky's costume were in the newspapers---feathers,
lappets, superb diamonds, and all the rest.  Lady Crackenbury read the
paragraph in bitterness of spirit and discoursed to her followers about
the airs which that woman was giving herself.  Mrs.\ Bute Crawley and
her young ladies in the country had a copy of the Morning Post from
town, and gave a vent to their honest indignation. ``If you had been
sandy-haired, green-eyed, and a French rope-dancer's daughter,'' Mrs.\ %
Bute said to her eldest girl (who, on the contrary, was a very swarthy,
short, and snub-nosed young lady), ``You might have had superb diamonds
forsooth, and have been presented at Court by your cousin, the Lady
Jane.  But you're only a gentlewoman, my poor dear child.  You have
only some of the best blood in England in your veins, and good
principles and piety for your portion.  I, myself, the wife of a
Baronet's younger brother, too, never thought of such a thing as going
to Court---nor would other people, if good Queen Charlotte had been
alive.'' In this way the worthy Rectoress consoled herself, and her
daughters sighed and sat over the Peerage all night.

A few days after the famous presentation, another great and exceeding
honour was vouchsafed to the virtuous Becky.  Lady Steyne's carriage
drove up to Mr.\ Rawdon Crawley's door, and the footman, instead of
driving down the front of the house, as by his tremendous knocking he
appeared to be inclined to do, relented and only delivered in a couple
of cards, on which were engraven the names of the Marchioness of Steyne
and the Countess of Gaunt.  If these bits of pasteboard had been
beautiful pictures, or had had a hundred yards of Malines lace rolled
round them, worth twice the number of guineas, Becky could not have
regarded them with more pleasure. You may be sure they occupied a
conspicuous place in the china bowl on the drawing-room table, where
Becky kept the cards of her visitors.  Lord! lord! how poor Mrs.\ %
Washington White's card and Lady Crackenbury's card---which our little
friend had been glad enough to get a few months back, and of which the
silly little creature was rather proud once---Lord! lord! I say, how
soon at the appearance of these grand court cards, did those poor
little neglected deuces sink down to the bottom of the pack.  Steyne!
Bareacres, Johnes of Helvellyn!  and Caerylon of Camelot! we may be
sure that Becky and Briggs looked out those august names in the
Peerage, and followed the noble races up through all the ramifications
of the family tree.

My Lord Steyne coming to call a couple of hours afterwards, and looking
about him, and observing everything as was his wont, found his ladies'
cards already ranged as the trumps of Becky's hand, and grinned, as
this old cynic always did at any naive display of human weakness.
Becky came down to him presently; whenever the dear girl expected his
lordship, her toilette was prepared, her hair in perfect order, her
mouchoirs, aprons, scarfs, little morocco slippers, and other female
gimcracks arranged, and she seated in some artless and agreeable
posture ready to receive him---whenever she was surprised, of course,
she had to fly to her apartment to take a rapid survey of matters in
the glass, and to trip down again to wait upon the great peer.

She found him grinning over the bowl.  She was discovered, and she
blushed a little.  ``Thank you, Monseigneur,'' she said.  ``You see your
ladies have been here.  How good of you!  I couldn't come before---I was
in the kitchen making a pudding.''

``I know you were, I saw you through the area-railings as I drove up,''
replied the old gentleman.

``You see everything,'' she replied.

``A few things, but not that, my pretty lady,'' he said good-naturedly.
``You silly little fibster!  I heard you in the room overhead, where I
have no doubt you were putting a little rouge on---you must give some
of yours to my Lady Gaunt, whose complexion is quite preposterous---and
I heard the bedroom door open, and then you came downstairs.''

``Is it a crime to try and look my best when YOU come here?'' answered
Mrs.\ Rawdon plaintively, and she rubbed her cheek with her handkerchief
as if to show there was no rouge at all, only genuine blushes and
modesty in her case.  About this who can tell? I know there is some
rouge that won't come off on a pocket-handkerchief, and some so good
that even tears will not disturb it.

``Well,'' said the old gentleman, twiddling round his wife's card, ``you
are bent on becoming a fine lady. You pester my poor old life out to
get you into the world.  You won't be able to hold your own there, you
silly little fool.  You've got no money.''

``You will get us a place,'' interposed Becky, ``as quick as possible.''

``You've got no money, and you want to compete with those who have. You
poor little earthenware pipkin, you want to swim down the stream along
with the great copper kettles.  All women are alike. Everybody is
striving for what is not worth the having!  Gad!  I dined with the King
yesterday, and we had neck of mutton and turnips. A dinner of herbs is
better than a stalled ox very often. You will go to Gaunt House.  You
give an old fellow no rest until you get there.  It's not half so nice
as here. You'll be bored there.  I am.  My wife is as gay as Lady
Macbeth, and my daughters as cheerful as Regan and Goneril.  I daren't
sleep in what they call my bedroom. The bed is like the baldaquin of
St.\ Peter's, and the pictures frighten me.  I have a little brass bed
in a dressing-room, and a little hair mattress like an anchorite. I am
an anchorite. Ho!  ho!  You'll be asked to dinner next week.  And gare
aux femmes, look out and hold your own!  How the women will bully you!''
This was a very long speech for a man of few words like my Lord Steyne;
nor was it the first which he uttered for Becky's benefit on that day.

Briggs looked up from the work-table at which she was seated in the
farther room and gave a deep sigh as she heard the great Marquis speak
so lightly of her sex.

``If you don't turn off that abominable sheep-dog,'' said Lord Steyne,
with a savage look over his shoulder at her, ``I will have her poisoned.''

``I always give my dog dinner from my own plate,'' said Rebecca, laughing
mischievously; and having enjoyed for some time the discomfiture of my
lord, who hated poor Briggs for interrupting his tete-a-tete with the
fair Colonel's wife, Mrs.\ Rawdon at length had pity upon her admirer,
and calling to Briggs, praised the fineness of the weather to her and
bade her to take out the child for a walk.

``I can't send her away,'' Becky said presently, after a pause, and in a
very sad voice.  Her eyes filled with tears as she spoke, and she
turned away her head.

``You owe her her wages, I suppose?'' said the Peer.

``Worse than that,'' said Becky, still casting down her eyes; ``I have
ruined her.''

``Ruined her? Then why don't you turn her out?'' the gentleman asked.

``Men do that,'' Becky answered bitterly.  ``Women are not so bad as you.
Last year, when we were reduced to our last guinea, she gave us
everything.  She shall never leave me, until we are ruined utterly
ourselves, which does not seem far off, or until I can pay her the
utmost farthing.''

``--------- it, how much is it?'' said the Peer with an oath. And Becky,
reflecting on the largeness of his means, mentioned not only the sum
which she had borrowed from Miss Briggs, but one of nearly double the
amount.

This caused the Lord Steyne to break out in another brief and energetic
expression of anger, at which Rebecca held down her head the more and
cried bitterly.  ``I could not help it.  It was my only chance.  I dare
not tell my husband.  He would kill me if I told him what I have done.
I have kept it a secret from everybody but you---and you forced it from
me.  Ah, what shall I do, Lord Steyne? for I am very, very unhappy!''

Lord Steyne made no reply except by beating the devil's tattoo and
biting his nails.  At last he clapped his hat on his head and flung out
of the room.  Rebecca did not rise from her attitude of misery until
the door slammed upon him and his carriage whirled away.  Then she rose
up with the queerest expression of victorious mischief glittering in
her green eyes.  She burst out laughing once or twice to herself, as
she sat at work, and sitting down to the piano, she rattled away a
triumphant voluntary on the keys, which made the people pause under her
window to listen to her brilliant music.

That night, there came two notes from Gaunt House for the little woman,
the one containing a card of invitation from Lord and Lady Steyne to a
dinner at Gaunt House next Friday, while the other enclosed a slip of
gray paper bearing Lord Steyne's signature and the address of Messrs.
Jones, Brown, and Robinson, Lombard Street.

Rawdon heard Becky laughing in the night once or twice.  It was only
her delight at going to Gaunt House and facing the ladies there, she
said, which amused her so.  But the truth was that she was occupied
with a great number of other thoughts.  Should she pay off old Briggs
and give her her conge? Should she astonish Raggles by settling his
account? She turned over all these thoughts on her pillow, and on the
next day, when Rawdon went out to pay his morning visit to the Club,
Mrs.\ Crawley (in a modest dress with a veil on) whipped off in a
hackney-coach to the City:  and being landed at Messrs. Jones and
Robinson's bank, presented a document there to the authority at the
desk, who, in reply, asked her ``How she would take it?''

She gently said ``she would take a hundred and fifty pounds in small
notes and the remainder in one note'': and passing through St.\ Paul's
Churchyard stopped there and bought the handsomest black silk gown for
Briggs which money could buy; and which, with a kiss and the kindest
speeches, she presented to the simple old spinster.

Then she walked to Mr.\ Raggles, inquired about his children
affectionately, and gave him fifty pounds on account.  Then she went to
the livery-man from whom she jobbed her carriages and gratified him
with a similar sum.  ``And I hope this will be a lesson to you, Spavin,''
she said, ``and that on the next drawing-room day my brother, Sir Pitt,
will not be inconvenienced by being obliged to take four of us in his
carriage to wait upon His Majesty, because my own carriage is not
forthcoming.'' It appears there had been a difference on the last
drawing-room day.  Hence the degradation which the Colonel had almost
suffered, of being obliged to enter the presence of his Sovereign in a
hack cab.

These arrangements concluded, Becky paid a visit upstairs to the
before-mentioned desk, which Amelia Sedley had given her years and
years ago, and which contained a number of useful and valuable little
things---in which private museum she placed the one note which Messrs.
Jones and Robinson's cashier had given her.



\chapter{In Which We Enjoy Three Courses and a Dessert}

When the ladies of Gaunt House were at breakfast that morning, Lord
Steyne (who took his chocolate in private and seldom disturbed the
females of his household, or saw them except upon public days, or when
they crossed each other in the hall, or when from his pit-box at the
opera he surveyed them in their box on the grand tier) his lordship, we
say, appeared among the ladies and the children who were assembled over
the tea and toast, and a battle royal ensued apropos of Rebecca.

``My Lady Steyne,'' he said, ``I want to see the list for your dinner on
Friday; and I want you, if you please, to write a card for Colonel and
Mrs.\ Crawley.''

``Blanche writes them,'' Lady Steyne said in a flutter. ``Lady Gaunt
writes them.''

``I will not write to that person,'' Lady Gaunt said, a tall and stately
lady, who looked up for an instant and then down again after she had
spoken.  It was not good to meet Lord Steyne's eyes for those who had
offended him.

``Send the children out of the room.  Go!'' said he pulling at the
bell-rope.  The urchins, always frightened before him, retired: their
mother would have followed too.  ``Not you,'' he said.  ``You stop.''

``My Lady Steyne,'' he said, ``once more will you have the goodness to go
to the desk and write that card for your dinner on Friday?''

``My Lord, I will not be present at it,'' Lady Gaunt said; ``I will go
home.''

``I wish you would, and stay there.  You will find the bailiffs at
Bareacres very pleasant company, and I shall be freed from lending
money to your relations and from your own damned tragedy airs.  Who are
you to give orders here? You have no money.  You've got no brains.  You
were here to have children, and you have not had any. Gaunt's tired of
you, and George's wife is the only person in the family who doesn't
wish you were dead.  Gaunt would marry again if you were.''

``I wish I were,'' her Ladyship answered with tears and rage in her eyes.

``You, forsooth, must give yourself airs of virtue, while my wife, who
is an immaculate saint, as everybody knows, and never did wrong in her
life, has no objection to meet my young friend Mrs.\ Crawley. My Lady
Steyne knows that appearances are sometimes against the best of women;
that lies are often told about the most innocent of them. Pray, madam,
shall I tell you some little anecdotes about my Lady Bareacres, your
mamma?''

``You may strike me if you like, sir, or hit any cruel blow,'' Lady Gaunt
said.  To see his wife and daughter suffering always put his Lordship
into a good humour.

``My sweet Blanche,'' he said, ``I am a gentleman, and never lay my hand
upon a woman, save in the way of kindness.  I only wish to correct
little faults in your character.  You women are too proud, and sadly
lack humility, as Father Mole, I'm sure, would tell my Lady Steyne if
he were here.  You mustn't give yourselves airs; you must be meek and
humble, my blessings.  For all Lady Steyne knows, this calumniated,
simple, good-humoured Mrs.\ Crawley is quite innocent---even more
innocent than herself.  Her husband's character is not good, but it is
as good as Bareacres', who has played a little and not paid a great
deal, who cheated you out of the only legacy you ever had and left you
a pauper on my hands.  And Mrs.\ Crawley is not very well-born, but she
is not worse than Fanny's illustrious ancestor, the first de la Jones.''

``The money which I brought into the family, sir,'' Lady George cried
out---

``You purchased a contingent reversion with it,'' the Marquis said
darkly.  ``If Gaunt dies, your husband may come to his honours; your
little boys may inherit them, and who knows what besides? In the
meanwhile, ladies, be as proud and virtuous as you like abroad, but
don't give ME any airs.  As for Mrs.\ Crawley's character, I shan't
demean myself or that most spotless and perfectly irreproachable lady
by even hinting that it requires a defence.  You will be pleased to
receive her with the utmost cordiality, as you will receive all persons
whom I present in this house.  This house?'' He broke out with a laugh.
``Who is the master of it? and what is it? This Temple of Virtue belongs
to me.  And if I invite all Newgate or all Bedlam here, by --------- they
shall be welcome.''

After this vigorous allocution, to one of which sort Lord Steyne
treated his ``Hareem'' whenever symptoms of insubordination appeared in
his household, the crestfallen women had nothing for it but to obey.
Lady Gaunt wrote the invitation which his Lordship required, and she
and her mother-in-law drove in person, and with bitter and humiliated
hearts, to leave the cards on Mrs.\ Rawdon, the reception of which
caused that innocent woman so much pleasure.

There were families in London who would have sacrificed a year's income
to receive such an honour at the hands of those great ladies. Mrs.\ %
Frederick Bullock, for instance, would have gone on her knees from May
Fair to Lombard Street, if Lady Steyne and Lady Gaunt had been waiting
in the City to raise her up and say, ``Come to us next Friday''---not to
one of the great crushes and grand balls of Gaunt House, whither
everybody went, but to the sacred, unapproachable, mysterious,
delicious entertainments, to be admitted to one of which was a
privilege, and an honour, and a blessing indeed.

Severe, spotless, and beautiful, Lady Gaunt held the very highest rank
in Vanity Fair.  The distinguished courtesy with which Lord Steyne
treated her charmed everybody who witnessed his behaviour, caused the
severest critics to admit how perfect a gentleman he was, and to own
that his Lordship's heart at least was in the right place.

The ladies of Gaunt House called Lady Bareacres in to their aid, in
order to repulse the common enemy.  One of Lady Gaunt's carriages went
to Hill Street for her Ladyship's mother, all whose equipages were in
the hands of the bailiffs, whose very jewels and wardrobe, it was said,
had been seized by those inexorable Israelites. Bareacres Castle was
theirs, too, with all its costly pictures, furniture, and articles of
vertu---the magnificent Vandykes; the noble Reynolds pictures; the
Lawrence portraits, tawdry and beautiful, and, thirty years ago, deemed
as precious as works of real genius; the matchless Dancing Nymph of
Canova, for which Lady Bareacres had sat in her youth---Lady Bareacres
splendid then, and radiant in wealth, rank, and beauty---a toothless,
bald, old woman now---a mere rag of a former robe of state.  Her lord,
painted at the same time by Lawrence, as waving his sabre in front of
Bareacres Castle, and clothed in his uniform as Colonel of the
Thistlewood Yeomanry, was a withered, old, lean man in a greatcoat and
a Brutus wig, slinking about Gray's Inn of mornings chiefly and dining
alone at clubs.  He did not like to dine with Steyne now.  They had run
races of pleasure together in youth when Bareacres was the winner. But
Steyne had more bottom than he and had lasted him out.  The Marquis was
ten times a greater man now than the young Lord Gaunt of '85, and
Bareacres nowhere in the race---old, beaten, bankrupt, and broken down.
He had borrowed too much money of Steyne to find it pleasant to meet
his old comrade often.  The latter, whenever he wished to be merry,
used jeeringly to ask Lady Gaunt why her father had not come to see
her. ``He has not been here for four months,'' Lord Steyne would say.  ``I
can always tell by my cheque-book afterwards, when I get a visit from
Bareacres.  What a comfort it is, my ladies, I bank with one of my
sons' fathers-in-law, and the other banks with me!''

Of the other illustrious persons whom Becky had the honour to encounter
on this her first presentation to the grand world, it does not become
the present historian to say much.  There was his Excellency the Prince
of Peterwaradin, with his Princess---a nobleman tightly girthed, with a
large military chest, on which the plaque of his order shone
magnificently, and wearing the red collar of the Golden Fleece round
his neck.  He was the owner of countless flocks. ``Look at his face.  I
think he must be descended from a sheep,'' Becky whispered to Lord
Steyne.  Indeed, his Excellency's countenance, long, solemn, and white,
with the ornament round his neck, bore some resemblance to that of a
venerable bell-wether.

There was Mr.\ John Paul Jefferson Jones, titularly attached to the
American Embassy and correspondent of the New York Demagogue, who, by
way of making himself agreeable to the company, asked Lady Steyne,
during a pause in the conversation at dinner, how his dear friend,
George Gaunt, liked the Brazils? He and George had been most intimate
at Naples and had gone up Vesuvius together.  Mr.\ Jones wrote a full
and particular account of the dinner, which appeared duly in the
Demagogue.  He mentioned the names and titles of all the guests, giving
biographical sketches of the principal people.  He described the
persons of the ladies with great eloquence; the service of the table;
the size and costume of the servants; enumerated the dishes and wines
served; the ornaments of the sideboard; and the probable value of the
plate.  Such a dinner he calculated could not be dished up under
fifteen or eighteen dollars per head. And he was in the habit, until
very lately, of sending over proteges, with letters of recommendation
to the present Marquis of Steyne, encouraged to do so by the intimate
terms on which he had lived with his dear friend, the late lord.  He
was most indignant that a young and insignificant aristocrat, the Earl
of Southdown, should have taken the pas of him in their procession to
the dining-room.  ``Just as I was stepping up to offer my hand to a
very pleasing and witty fashionable, the brilliant and exclusive Mrs.\ %
Rawdon Crawley,''---he wrote---``the young patrician interposed between me
and the lady and whisked my Helen off without a word of apology. I was
fain to bring up the rear with the Colonel, the lady's husband, a stout
red-faced warrior who distinguished himself at Waterloo, where he had
better luck than befell some of his brother redcoats at New Orleans.''

The Colonel's countenance on coming into this polite society wore as
many blushes as the face of a boy of sixteen assumes when he is
confronted with his sister's schoolfellows.  It has been told before
that honest Rawdon had not been much used at any period of his life to
ladies' company.  With the men at the Club or the mess room, he was
well enough; and could ride, bet, smoke, or play at billiards with the
boldest of them.  He had had his time for female friendships too, but
that was twenty years ago, and the ladies were of the rank of those
with whom Young Marlow in the comedy is represented as having been
familiar before he became abashed in the presence of Miss Hardcastle.
The times are such that one scarcely dares to allude to that kind of
company which thousands of our young men in Vanity Fair are frequenting
every day, which nightly fills casinos and dancing-rooms, which is
known to exist as well as the Ring in Hyde Park or the Congregation at
St.\ James's---but which the most squeamish if not the most moral of
societies is determined to ignore.  In a word, although Colonel Crawley
was now five-and-forty years of age, it had not been his lot in life to
meet with a half dozen good women, besides his paragon of a wife.  All
except her and his kind sister Lady Jane, whose gentle nature had tamed
and won him, scared the worthy Colonel, and on occasion of his first
dinner at Gaunt House he was not heard to make a single remark except
to state that the weather was very hot.  Indeed Becky would have left
him at home, but that virtue ordained that her husband should be by her
side to protect the timid and fluttering little creature on her first
appearance in polite society.

On her first appearance Lord Steyne stepped forward, taking her hand,
and greeting her with great courtesy, and presenting her to Lady
Steyne, and their ladyships, her daughters.  Their ladyships made three
stately curtsies, and the elder lady to be sure gave her hand to the
newcomer, but it was as cold and lifeless as marble.

Becky took it, however, with grateful humility, and performing a
reverence which would have done credit to the best dancer-master, put
herself at Lady Steyne's feet, as it were, by saying that his Lordship
had been her father's earliest friend and patron, and that she, Becky,
had learned to honour and respect the Steyne family from the days of
her childhood.  The fact is that Lord Steyne had once purchased a
couple of pictures of the late Sharp, and the affectionate orphan could
never forget her gratitude for that favour.

The Lady Bareacres then came under Becky's cognizance---to whom the
Colonel's lady made also a most respectful obeisance:  it was returned
with severe dignity by the exalted person in question.

``I had the pleasure of making your Ladyship's acquaintance at Brussels,
ten years ago,'' Becky said in the most winning manner.  ``I had the good
fortune to meet Lady Bareacres at the Duchess of Richmond's ball, the
night before the Battle of Waterloo.  And I recollect your Ladyship,
and my Lady Blanche, your daughter, sitting in the carriage in the
porte-cochere at the Inn, waiting for horses. I hope your Ladyship's
diamonds are safe.''

Everybody's eyes looked into their neighbour's.  The famous diamonds
had undergone a famous seizure, it appears, about which Becky, of
course, knew nothing. Rawdon Crawley retreated with Lord Southdown into
a window, where the latter was heard to laugh immoderately, as Rawdon
told him the story of Lady Bareacres wanting horses and ``knuckling down
by Jove,'' to Mrs.\ Crawley.  ``I think I needn't be afraid of THAT
woman,'' Becky thought.  Indeed, Lady Bareacres exchanged terrified and
angry looks with her daughter and retreated to a table, where she began
to look at pictures with great energy.

When the Potentate from the Danube made his appearance, the
conversation was carried on in the French language, and the Lady
Bareacres and the younger ladies found, to their farther mortification,
that Mrs.\ Crawley was much better acquainted with that tongue, and
spoke it with a much better accent than they. Becky had met other
Hungarian magnates with the army in France in 1816-17.  She asked after
her friends with great interest. The foreign personages thought that she
was a lady of great distinction, and the Prince and the Princess asked
severally of Lord Steyne and the Marchioness, whom they conducted to
dinner, who was that petite dame who spoke so well?

Finally, the procession being formed in the order described by the
American diplomatist, they marched into the apartment where the banquet
was served, and which, as I have promised the reader he shall enjoy it,
he shall have the liberty of ordering himself so as to suit his fancy.

But it was when the ladies were alone that Becky knew the tug of war
would come.  And then indeed the little woman found herself in such a
situation as made her acknowledge the correctness of Lord Steyne's
caution to her to beware of the society of ladies above her own sphere.
As they say, the persons who hate Irishmen most are Irishmen; so,
assuredly, the greatest tyrants over women are women. When poor little
Becky, alone with the ladies, went up to the fire-place whither the
great ladies had repaired, the great ladies marched away and took
possession of a table of drawings.  When Becky followed them to the
table of drawings, they dropped off one by one to the fire again.  She
tried to speak to one of the children (of whom she was commonly fond in
public places), but Master George Gaunt was called away by his mamma;
and the stranger was treated with such cruelty finally, that even Lady
Steyne herself pitied her and went up to speak to the friendless little
woman.

``Lord Steyne,'' said her Ladyship, as her wan cheeks glowed with a
blush, ``says you sing and play very beautifully, Mrs.\ Crawley---I wish
you would do me the kindness to sing to me.''

``I will do anything that may give pleasure to my Lord Steyne or to
you,'' said Rebecca, sincerely grateful, and seating herself at the
piano, began to sing.

She sang religious songs of Mozart, which had been early favourites of
Lady Steyne, and with such sweetness and tenderness that the lady,
lingering round the piano, sat down by its side and listened until the
tears rolled down her eyes.  It is true that the opposition ladies at
the other end of the room kept up a loud and ceaseless buzzing and
talking, but the Lady Steyne did not hear those rumours.  She was a
child again---and had wandered back through a forty years' wilderness to
her convent garden.  The chapel organ had pealed the same tones, the
organist, the sister whom she loved best of the community, had taught
them to her in those early happy days.  She was a girl once more, and
the brief period of her happiness bloomed out again for an hour---she
started when the jarring doors were flung open, and with a loud laugh
from Lord Steyne, the men of the party entered full of gaiety.

He saw at a glance what had happened in his absence, and was grateful
to his wife for once.  He went and spoke to her, and called her by her
Christian name, so as again to bring blushes to her pale face---``My wife
says you have been singing like an angel,'' he said to Becky.  Now there
are angels of two kinds, and both sorts, it is said, are charming in
their way.

Whatever the previous portion of the evening had been, the rest of that
night was a great triumph for Becky.  She sang her very best, and it
was so good that every one of the men came and crowded round the piano.
The women, her enemies, were left quite alone. And Mr.\ Paul Jefferson
Jones thought he had made a conquest of Lady Gaunt by going up to her
Ladyship and praising her delightful friend's first-rate singing.



\chapter{Contains a Vulgar Incident}

The Muse, whoever she be, who presides over this Comic History must now
descend from the genteel heights in which she has been soaring and have
the goodness to drop down upon the lowly roof of John Sedley at
Brompton, and describe what events are taking place there. Here, too,
in this humble tenement, live care, and distrust, and dismay.  Mrs.\ %
Clapp in the kitchen is grumbling in secret to her husband about the
rent, and urging the good fellow to rebel against his old friend and
patron and his present lodger.  Mrs.\ Sedley has ceased to visit her
landlady in the lower regions now, and indeed is in a position to
patronize Mrs.\ Clapp no longer.  How can one be condescending to a lady
to whom one owes a matter of forty pounds, and who is perpetually
throwing out hints for the money? The Irish maidservant has not altered
in the least in her kind and respectful behaviour; but Mrs.\ Sedley
fancies that she is growing insolent and ungrateful, and, as the guilty
thief who fears each bush an officer, sees threatening innuendoes and
hints of capture in all the girl's speeches and answers.  Miss Clapp,
grown quite a young woman now, is declared by the soured old lady to be
an unbearable and impudent little minx.  Why Amelia can be so fond of
her, or have her in her room so much, or walk out with her so
constantly, Mrs.\ Sedley cannot conceive. The bitterness of poverty has
poisoned the life of the once cheerful and kindly woman.  She is
thankless for Amelia's constant and gentle bearing towards her; carps
at her for her efforts at kindness or service; rails at her for her
silly pride in her child and her neglect of her parents.  Georgy's
house is not a very lively one since Uncle Jos's annuity has been
withdrawn and the little family are almost upon famine diet.

Amelia thinks, and thinks, and racks her brain, to find some means of
increasing the small pittance upon which the household is starving.
Can she give lessons in anything? paint card-racks? do fine work? She
finds that women are working hard, and better than she can, for
twopence a day.  She buys a couple of begilt Bristol boards at the
Fancy Stationer's and paints her very best upon them---a shepherd with
a red waistcoat on one, and a pink face smiling in the midst of a
pencil landscape---a shepherdess on the other, crossing a little bridge,
with a little dog, nicely shaded.  The man of the Fancy Repository and
Brompton Emporium of Fine Arts (of whom she bought the screens, vainly
hoping that he would repurchase them when ornamented by her hand) can
hardly hide the sneer with which he examines these feeble works of art.
He looks askance at the lady who waits in the shop, and ties up the
cards again in their envelope of whitey-brown paper, and hands them to
the poor widow and Miss Clapp, who had never seen such beautiful things
in her life, and had been quite confident that the man must give at
least two guineas for the screens.  They try at other shops in the
interior of London, with faint sickening hopes.  ``Don't want 'em,'' says
one.  ``Be off,'' says another fiercely.  Three-and-sixpence has been
spent in vain---the screens retire to Miss Clapp's bedroom, who
persists in thinking them lovely.

She writes out a little card in her neatest hand, and after long
thought and labour of composition, in which the public is informed that
``A Lady who has some time at her disposal, wishes to undertake the
education of some little girls, whom she would instruct in English, in
French, in Geography, in History, and in Music---address A.  O., at Mr.\ %
Brown's''; and she confides the card to the gentleman of the Fine Art
Repository, who consents to allow it to lie upon the counter, where it
grows dingy and fly-blown.  Amelia passes the door wistfully many a
time, in hopes that Mr.\ Brown will have some news to give her, but he
never beckons her in.  When she goes to make little purchases, there is
no news for her.  Poor simple lady, tender and weak---how are you to
battle with the struggling violent world?

She grows daily more care-worn and sad, fixing upon her child alarmed
eyes, whereof the little boy cannot interpret the expression.  She
starts up of a night and peeps into his room stealthily, to see that he
is sleeping and not stolen away.  She sleeps but little now.  A
constant thought and terror is haunting her.  How she weeps and prays
in the long silent nights---how she tries to hide from herself the
thought which will return to her, that she ought to part with the boy,
that she is the only barrier between him and prosperity.  She can't,
she can't. Not now, at least.  Some other day.  Oh! it is too hard to
think of and to bear.

A thought comes over her which makes her blush and turn from
herself---her parents might keep the annuity---the curate would marry her
and give a home to her and the boy.  But George's picture and dearest
memory are there to rebuke her.  Shame and love say no to the
sacrifice.  She shrinks from it as from something unholy, and such
thoughts never found a resting-place in that pure and gentle bosom.

The combat, which we describe in a sentence or two, lasted for many
weeks in poor Amelia's heart, during which she had no confidante;
indeed, she could never have one, as she would not allow to herself the
possibility of yielding, though she was giving way daily before the
enemy with whom she had to battle.  One truth after another was
marshalling itself silently against her and keeping its ground. Poverty
and misery for all, want and degradation for her parents, injustice to
the boy---one by one the outworks of the little citadel were taken, in
which the poor soul passionately guarded her only love and treasure.

At the beginning of the struggle, she had written off a letter of
tender supplication to her brother at Calcutta, imploring him not to
withdraw the support which he had granted to their parents and painting
in terms of artless pathos their lonely and hapless condition.  She did
not know the truth of the matter.  The payment of Jos's annuity was
still regular, but it was a money-lender in the City who was receiving
it:  old Sedley had sold it for a sum of money wherewith to prosecute
his bootless schemes.  Emmy was calculating eagerly the time that would
elapse before the letter would arrive and be answered.  She had written
down the date in her pocket-book of the day when she dispatched it.  To
her son's guardian, the good Major at Madras, she had not communicated
any of her griefs and perplexities.  She had not written to him since
she wrote to congratulate him on his approaching marriage.  She thought
with sickening despondency, that that friend---the only one, the one who
had felt such a regard for her---was fallen away.

One day, when things had come to a very bad pass---when the creditors
were pressing, the mother in hysteric grief, the father in more than
usual gloom, the inmates of the family avoiding each other, each
secretly oppressed with his private unhappiness and notion of
wrong---the father and daughter happened to be left alone together, and
Amelia thought to comfort her father by telling him what she had done.
She had written to Joseph---an answer must come in three or four months.
He was always generous, though careless.  He could not refuse, when he
knew how straitened were the circumstances of his parents.

Then the poor old gentleman revealed the whole truth to her---that his
son was still paying the annuity, which his own imprudence had flung
away.  He had not dared to tell it sooner.  He thought Amelia's ghastly
and terrified look, when, with a trembling, miserable voice he made the
confession, conveyed reproaches to him for his concealment.  ``Ah!'' said
he with quivering lips and turning away, ``you despise your old father
now!''

``Oh, papa! it is not that,'' Amelia cried out, falling on his neck and
kissing him many times.  ``You are always good and kind.  You did it for
the best.  It is not for the money---it is---my God! my God! have mercy
upon me, and give me strength to bear this trial''; and she kissed him
again wildly and went away.

Still the father did not know what that explanation meant, and the
burst of anguish with which the poor girl left him.  It was that she
was conquered.  The sentence was passed.  The child must go from
her---to others---to forget her.  Her heart and her treasure---her joy,
hope, love, worship---her God, almost!  She must give him up, and
then---and then she would go to George, and they would watch over the
child and wait for him until he came to them in Heaven.

She put on her bonnet, scarcely knowing what she did, and went out to
walk in the lanes by which George used to come back from school, and
where she was in the habit of going on his return to meet the boy.  It
was May, a half-holiday.  The leaves were all coming out, the weather
was brilliant; the boy came running to her flushed with health,
singing, his bundle of school-books hanging by a thong. There he was.
Both her arms were round him.  No, it was impossible. They could not be
going to part.  ``What is the matter, Mother?'' said he; ``you look very
pale.''

``Nothing, my child,'' she said and stooped down and kissed him.

That night Amelia made the boy read the story of Samuel to her, and how
Hannah, his mother, having weaned him, brought him to Eli the High
Priest to minister before the Lord.  And he read the song of gratitude
which Hannah sang, and which says, who it is who maketh poor and maketh
rich, and bringeth low and exalteth---how the poor shall be raised up
out of the dust, and how, in his own might, no man shall be strong.
Then he read how Samuel's mother made him a little coat and brought it
to him from year to year when she came up to offer the yearly
sacrifice.  And then, in her sweet simple way, George's mother made
commentaries to the boy upon this affecting story.  How Hannah, though
she loved her son so much, yet gave him up because of her vow.  And how
she must always have thought of him as she sat at home, far away,
making the little coat; and Samuel, she was sure, never forgot his
mother; and how happy she must have been as the time came (and the
years pass away very quick) when she should see her boy and how good
and wise he had grown.  This little sermon she spoke with a gentle
solemn voice, and dry eyes, until she came to the account of their
meeting---then the discourse broke off suddenly, the tender heart
overflowed, and taking the boy to her breast, she rocked him in her
arms and wept silently over him in a sainted agony of tears.

Her mind being made up, the widow began to take such measures as seemed
right to her for advancing the end which she proposed.  One day, Miss
Osborne, in Russell Square (Amelia had not written the name or number
of the house for ten years---her youth, her early story came back to her
as she wrote the superscription) one day Miss Osborne got a letter from
Amelia which made her blush very much and look towards her father,
sitting glooming in his place at the other end of the table.

In simple terms, Amelia told her the reasons which had induced her to
change her mind respecting her boy. Her father had met with fresh
misfortunes which had entirely ruined him.  Her own pittance was so
small that it would barely enable her to support her parents and would
not suffice to give George the advantages which were his due. Great as
her sufferings would be at parting with him she would, by God's help,
endure them for the boy's sake.  She knew that those to whom he was
going would do all in their power to make him happy. She described his
disposition, such as she fancied it---quick and impatient of control or
harshness, easily to be moved by love and kindness.  In a postscript,
she stipulated that she should have a written agreement, that she
should see the child as often as she wished---she could not part with
him under any other terms.

``What? Mrs.\ Pride has come down, has she?'' old Osborne said, when with
a tremulous eager voice Miss Osborne read him the letter. ``Reg'lar
starved out, hey? Ha, ha!  I knew she would.'' He tried to keep his
dignity and to read his paper as usual---but he could not follow it.  He
chuckled and swore to himself behind the sheet.

At last he flung it down and, scowling at his daughter, as his wont
was, went out of the room into his study adjoining, from whence he
presently returned with a key.  He flung it to Miss Osborne.

``Get the room over mine---his room that was---ready,'' he said.  ``Yes,
sir,'' his daughter replied in a tremble. It was George's room.  It had
not been opened for more than ten years.  Some of his clothes, papers,
handkerchiefs, whips and caps, fishing-rods and sporting gear, were
still there.  An Army list of 1814, with his name written on the cover;
a little dictionary he was wont to use in writing; and the Bible his
mother had given him, were on the mantelpiece, with a pair of spurs and
a dried inkstand covered with the dust of ten years.  Ah! since that
ink was wet, what days and people had passed away!  The writing-book,
still on the table, was blotted with his hand.

Miss Osborne was much affected when she first entered this room with
the servants under her.  She sank quite pale on the little bed. ``This
is blessed news, m'am---indeed, m'am,'' the housekeeper said; ``and the
good old times is returning, m'am.  The dear little feller, to be sure,
m'am; how happy he will be!  But some folks in May Fair, m'am, will owe
him a grudge, m'am''; and she clicked back the bolt which held the
window-sash and let the air into the chamber.

``You had better send that woman some money,'' Mr.\ Osborne said, before
he went out.  ``She shan't want for nothing.  Send her a hundred pound.''

``And I'll go and see her to-morrow?'' Miss Osborne asked.

``That's your look out.  She don't come in here, mind. No, by ---------,
not for all the money in London.  But she mustn't want now.  So look
out, and get things right.'' With which brief speeches Mr.\ Osborne took
leave of his daughter and went on his accustomed way into the City.

``Here, Papa, is some money,'' Amelia said that night, kissing the old
man, her father, and putting a bill for a hundred pounds into his
hands.  ``And---and, Mamma, don't be harsh with Georgy.  He---he is not
going to stop with us long.'' She could say nothing more, and walked
away silently to her room.  Let us close it upon her prayers and her
sorrow.  I think we had best speak little about so much love and grief.

Miss Osborne came the next day, according to the promise contained in
her note, and saw Amelia.  The meeting between them was friendly. A
look and a few words from Miss Osborne showed the poor widow that, with
regard to this woman at least, there need be no fear lest she should
take the first place in her son's affection. She was cold, sensible,
not unkind.  The mother had not been so well pleased, perhaps, had the
rival been better looking, younger, more affectionate, warmer-hearted.
Miss Osborne, on the other hand, thought of old times and memories and
could not but be touched with the poor mother's pitiful situation.  She
was conquered, and laying down her arms, as it were, she humbly
submitted.  That day they arranged together the preliminaries of the
treaty of capitulation.

George was kept from school the next day, and saw his aunt.  Amelia
left them alone together and went to her room.  She was trying the
separation---as that poor gentle Lady Jane Grey felt the edge of the axe
that was to come down and sever her slender life.  Days were passed in
parleys, visits, preparations.  The widow broke the matter to Georgy
with great caution; she looked to see him very much affected by the
intelligence.  He was rather elated than otherwise, and the poor woman
turned sadly away.  He bragged about the news that day to the boys at
school; told them how he was going to live with his grandpapa, his
father's father, not the one who comes here sometimes; and that he
would be very rich, and have a carriage, and a pony, and go to a much
finer school, and when he was rich he would buy Leader's pencil-case
and pay the tart-woman.  The boy was the image of his father, as his
fond mother thought.

Indeed I have no heart, on account of our dear Amelia's sake, to go
through the story of George's last days at home.

At last the day came, the carriage drove up, the little humble packets
containing tokens of love and remembrance were ready and disposed in
the hall long since---George was in his new suit, for which the tailor
had come previously to measure him.  He had sprung up with the sun and
put on the new clothes, his mother hearing him from the room close by,
in which she had been lying, in speechless grief and watching.  Days
before she had been making preparations for the end, purchasing little
stores for the boy's use, marking his books and linen, talking with him
and preparing him for the change---fondly fancying that he needed
preparation.

So that he had change, what cared he? He was longing for it.  By a
thousand eager declarations as to what he would do, when he went to
live with his grandfather, he had shown the poor widow how little the
idea of parting had cast him down.  ``He would come and see his mamma
often on the pony,'' he said.  ``He would come and fetch her in the
carriage; they would drive in the park, and she should have everything
she wanted.'' The poor mother was fain to content herself with these
selfish demonstrations of attachment, and tried to convince herself how
sincerely her son loved her.  He must love her. All children were so:
a little anxious for novelty, and---no, not selfish, but self-willed.
Her child must have his enjoyments and ambition in the world.  She
herself, by her own selfishness and imprudent love for him had denied
him his just rights and pleasures hitherto.

I know few things more affecting than that timorous debasement and
self-humiliation of a woman.  How she owns that it is she and not the
man who is guilty; how she takes all the faults on her side; how she
courts in a manner punishment for the wrongs which she has not
committed and persists in shielding the real culprit!  It is those who
injure women who get the most kindness from them---they are born timid
and tyrants and maltreat those who are humblest before them.

So poor Amelia had been getting ready in silent misery for her son's
departure, and had passed many and many a long solitary hour in making
preparations for the end. George stood by his mother, watching her
arrangements without the least concern.  Tears had fallen into his
boxes; passages had been scored in his favourite books; old toys,
relics, treasures had been hoarded away for him, and packed with
strange neatness and care---and of all these things the boy took no
note.  The child goes away smiling as the mother breaks her heart.  By
heavens it is pitiful, the bootless love of women for children in
Vanity Fair.

A few days are past, and the great event of Amelia's life is
consummated.  No angel has intervened.  The child is sacrificed and
offered up to fate, and the widow is quite alone.

The boy comes to see her often, to be sure.  He rides on a pony with a
coachman behind him, to the delight of his old grandfather, Sedley, who
walks proudly down the lane by his side.  She sees him, but he is not
her boy any more.  Why, he rides to see the boys at the little school,
too, and to show off before them his new wealth and splendour.  In two
days he has adopted a slightly imperious air and patronizing manner.
He was born to command, his mother thinks, as his father was before him.

It is fine weather now.  Of evenings on the days when he does not come,
she takes a long walk into London---yes, as far as Russell Square, and
rests on the stone by the railing of the garden opposite Mr.\ Osborne's
house. It is so pleasant and cool.  She can look up and see the
drawing-room windows illuminated, and, at about nine o'clock, the
chamber in the upper story where Georgy sleeps.  She knows---he has told
her.  She prays there as the light goes out, prays with an humble
heart, and walks home shrinking and silent. She is very tired when she
comes home.  Perhaps she will sleep the better for that long weary
walk, and she may dream about Georgy.

One Sunday she happened to be walking in Russell Square, at some
distance from Mr.\ Osborne's house (she could see it from a distance
though) when all the bells of Sabbath were ringing, and George and his
aunt came out to go to church; a little sweep asked for charity, and
the footman, who carried the books, tried to drive him away; but Georgy
stopped and gave him money.  May God's blessing be on the boy!  Emmy
ran round the square and, coming up to the sweep, gave him her mite
too. All the bells of Sabbath were ringing, and she followed them until
she came to the Foundling Church, into which she went.  There she sat
in a place whence she could see the head of the boy under his father's
tombstone. Many hundred fresh children's voices rose up there and sang
hymns to the Father Beneficent, and little George's soul thrilled with
delight at the burst of glorious psalmody.  His mother could not see
him for awhile, through the mist that dimmed her eyes.



\chapter{In Which a Charade Is Acted Which May or May Not Puzzle the Reader}

After Becky's appearance at my Lord Steyne's private and select
parties, the claims of that estimable woman as regards fashion were
settled, and some of the very greatest and tallest doors in the
metropolis were speedily opened to her---doors so great and tall that
the beloved reader and writer hereof may hope in vain to enter at them.
Dear brethren, let us tremble before those august portals.  I fancy
them guarded by grooms of the chamber with flaming silver forks with
which they prong all those who have not the right of the entree. They
say the honest newspaper-fellow who sits in the hall and takes down the
names of the great ones who are admitted to the feasts dies after a
little time.  He can't survive the glare of fashion long.  It scorches
him up, as the presence of Jupiter in full dress wasted that poor
imprudent Semele---a giddy moth of a creature who ruined herself by
venturing out of her natural atmosphere. Her myth ought to be taken to
heart amongst the Tyburnians, the Belgravians---her story, and perhaps
Becky's too. Ah, ladies!---ask the Reverend Mr.\ Thurifer if Belgravia is
not a sounding brass and Tyburnia a tinkling cymbal.  These are
vanities. Even these will pass away.  And some day or other (but it
will be after our time, thank goodness) Hyde Park Gardens will be no
better known than the celebrated horticultural outskirts of Babylon,
and Belgrave Square will be as desolate as Baker Street, or Tadmor in
the wilderness.

Ladies, are you aware that the great Pitt lived in Baker Street? What
would not your grandmothers have given to be asked to Lady Hester's
parties in that now decayed mansion? I have dined in it---moi qui vous
parle, I peopled the chamber with ghosts of the mighty dead. As we sat
soberly drinking claret there with men of to-day, the spirits of the
departed came in and took their places round the darksome board.  The
pilot who weathered the storm tossed off great bumpers of spiritual
port; the shade of Dundas did not leave the ghost of a heeltap.
Addington sat bowing and smirking in a ghastly manner, and would not be
behindhand when the noiseless bottle went round; Scott, from under
bushy eyebrows, winked at the apparition of a beeswing; Wilberforce's
eyes went up to the ceiling, so that he did not seem to know how his
glass went up full to his mouth and came down empty; up to the ceiling
which was above us only yesterday, and which the great of the past days
have all looked at. They let the house as a furnished lodging now.
Yes, Lady Hester once lived in Baker Street, and lies asleep in the
wilderness. Eothen saw her there---not in Baker Street, but in the other
solitude.

It is all vanity to be sure, but who will not own to liking a little of
it? I should like to know what well-constituted mind, merely because it
is transitory, dislikes roast beef? That is a vanity, but may every man
who reads this have a wholesome portion of it through life, I beg:
aye, though my readers were five hundred thousand. Sit down, gentlemen,
and fall to, with a good hearty appetite; the fat, the lean, the gravy,
the horse-radish as you like it---don't spare it.  Another glass of
wine, Jones, my boy---a little bit of the Sunday side.  Yes, let us eat
our fill of the vain thing and be thankful therefor. And let us make
the best of Becky's aristocratic pleasures likewise---for these too,
like all other mortal delights, were but transitory.

The upshot of her visit to Lord Steyne was that His Highness the Prince
of Peterwaradin took occasion to renew his acquaintance with Colonel
Crawley, when they met on the next day at the Club, and to compliment
Mrs.\ Crawley in the Ring of Hyde Park with a profound salute of the
hat.  She and her husband were invited immediately to one of the
Prince's small parties at Levant House, then occupied by His Highness
during the temporary absence from England of its noble proprietor.  She
sang after dinner to a very little comite. The Marquis of Steyne was
present, paternally superintending the progress of his pupil.

At Levant House Becky met one of the finest gentlemen and greatest
ministers that Europe has produced---the Duc de la Jabotiere, then
Ambassador from the Most Christian King, and subsequently Minister to
that monarch.  I declare I swell with pride as these august names are
transcribed by my pen, and I think in what brilliant company my dear
Becky is moving.  She became a constant guest at the French Embassy,
where no party was considered to be complete without the presence of
the charming Madame Ravdonn Cravley.  Messieurs de Truffigny (of the
Perigord family) and Champignac, both attaches of the Embassy, were
straightway smitten by the charms of the fair Colonel's wife, and both
declared, according to the wont of their nation (for who ever yet met a
Frenchman, come out of England, that has not left half a dozen families
miserable, and brought away as many hearts in his pocket-book?), both,
I say, declared that they were au mieux with the charming Madame
Ravdonn.

But I doubt the correctness of the assertion.  Champignac was very fond
of ecarte, and made many parties with the Colonel of evenings, while
Becky was singing to Lord Steyne in the other room; and as for
Truffigny, it is a well-known fact that he dared not go to the
Travellers', where he owed money to the waiters, and if he had not had
the Embassy as a dining-place, the worthy young gentleman must have
starved.  I doubt, I say, that Becky would have selected either of
these young men as a person on whom she would bestow her special
regard.  They ran of her messages, purchased her gloves and flowers,
went in debt for opera-boxes for her, and made themselves amiable in a
thousand ways.  And they talked English with adorable simplicity, and
to the constant amusement of Becky and my Lord Steyne, she would mimic
one or other to his face, and compliment him on his advance in the
English language with a gravity which never failed to tickle the
Marquis, her sardonic old patron. Truffigny gave Briggs a shawl by way
of winning over Becky's confidante, and asked her to take charge of a
letter which the simple spinster handed over in public to the person to
whom it was addressed, and the composition of which amused everybody
who read it greatly. Lord Steyne read it, everybody but honest Rawdon,
to whom it was not necessary to tell everything that passed in the
little house in May Fair.

Here, before long, Becky received not only ``the best'' foreigners (as
the phrase is in our noble and admirable society slang), but some of
the best English people too. I don't mean the most virtuous, or indeed
the least virtuous, or the cleverest, or the stupidest, or the richest,
or the best born, but ``the best,''---in a word, people about whom there
is no question---such as the great Lady Fitz-Willis, that Patron Saint
of Almack's, the great Lady Slowbore, the great Lady Grizzel Macbeth
(she was Lady G.  Glowry, daughter of Lord Grey of Glowry), and the
like.  When the Countess of Fitz-Willis (her Ladyship is of the
Kingstreet family, see Debrett and Burke) takes up a person, he or she
is safe.  There is no question about them any more.  Not that my Lady
Fitz-Willis is any better than anybody else, being, on the contrary, a
faded person, fifty-seven years of age, and neither handsome, nor
wealthy, nor entertaining; but it is agreed on all sides that she is of
the ``best people.'' Those who go to her are of the best:  and from an
old grudge probably to Lady Steyne (for whose coronet her ladyship,
then the youthful Georgina Frederica, daughter of the Prince of Wales's
favourite, the Earl of Portansherry, had once tried), this great and
famous leader of the fashion chose to acknowledge Mrs.\ Rawdon Crawley;
made her a most marked curtsey at the assembly over which she presided;
and not only encouraged her son, St.\ Kitts (his lordship got his place
through Lord Steyne's interest), to frequent Mrs.\ Crawley's house, but
asked her to her own mansion and spoke to her twice in the most public
and condescending manner during dinner.  The important fact was known
all over London that night.  People who had been crying fie about Mrs.\ %
Crawley were silent.  Wenham, the wit and lawyer, Lord Steyne's
right-hand man, went about everywhere praising her:  some who had
hesitated, came forward at once and welcomed her; little Tom Toady, who
had warned Southdown about visiting such an abandoned woman, now
besought to be introduced to her.  In a word, she was admitted to be
among the ``best'' people.  Ah, my beloved readers and brethren, do not
envy poor Becky prematurely---glory like this is said to be fugitive.
It is currently reported that even in the very inmost circles, they are
no happier than the poor wanderers outside the zone; and Becky, who
penetrated into the very centre of fashion and saw the great George IV
face to face, has owned since that there too was Vanity.

We must be brief in descanting upon this part of her career.  As I
cannot describe the mysteries of freemasonry, although I have a shrewd
idea that it is a humbug, so an uninitiated man cannot take upon
himself to portray the great world accurately, and had best keep his
opinions to himself, whatever they are.

Becky has often spoken in subsequent years of this season of her life,
when she moved among the very greatest circles of the London fashion.
Her success excited, elated, and then bored her.  At first no
occupation was more pleasant than to invent and procure (the latter a
work of no small trouble and ingenuity, by the way, in a person of Mrs.\ %
Rawdon Crawley's very narrow means)---to procure, we say, the prettiest
new dresses and ornaments; to drive to fine dinner parties, where she
was welcomed by great people; and from the fine dinner parties to fine
assemblies, whither the same people came with whom she had been dining,
whom she had met the night before, and would see on the morrow---the
young men faultlessly appointed, handsomely cravatted, with the neatest
glossy boots and white gloves---the elders portly, brass-buttoned,
noble-looking, polite, and prosy---the young ladies blonde, timid, and
in pink---the mothers grand, beautiful, sumptuous, solemn, and in
diamonds.  They talked in English, not in bad French, as they do in the
novels.  They talked about each others' houses, and characters, and
families---just as the Joneses do about the Smiths.  Becky's former
acquaintances hated and envied her; the poor woman herself was yawning
in spirit. ``I wish I were out of it,'' she said to herself.  ``I would
rather be a parson's wife and teach a Sunday school than this; or a
sergeant's lady and ride in the regimental waggon; or, oh, how much
gayer it would be to wear spangles and trousers and dance before a
booth at a fair.''

``You would do it very well,'' said Lord Steyne, laughing. She used to
tell the great man her ennuis and perplexities in her artless way---they
amused him.

``Rawdon would make a very good Ecuyer---Master of the Ceremonies---what
do you call him---the man in the large boots and the uniform, who goes
round the ring cracking the whip? He is large, heavy, and of a military
figure.  I recollect,'' Becky continued pensively, ``my father took me to
see a show at Brookgreen Fair when I was a child, and when we came
home, I made myself a pair of stilts and danced in the studio to the
wonder of all the pupils.''

``I should have liked to see it,'' said Lord Steyne.

``I should like to do it now,'' Becky continued.  ``How Lady Blinkey would
open her eyes, and Lady Grizzel Macbeth would stare!  Hush! silence!
there is Pasta beginning to sing.'' Becky always made a point of being
conspicuously polite to the professional ladies and gentlemen who
attended at these aristocratic parties---of following them into the
corners where they sat in silence, and shaking hands with them, and
smiling in the view of all persons.  She was an artist herself, as she
said very truly; there was a frankness and humility in the manner in
which she acknowledged her origin, which provoked, or disarmed, or
amused lookers-on, as the case might be. ``How cool that woman is,'' said
one; ``what airs of independence she assumes, where she ought to sit
still and be thankful if anybody speaks to her!'' ``What an honest and
good-natured soul she is!'' said another. ``What an artful little minx''
said a third.  They were all right very likely, but Becky went her own
way, and so fascinated the professional personages that they would
leave off their sore throats in order to sing at her parties and give
her lessons for nothing.

Yes, she gave parties in the little house in Curzon Street.  Many
scores of carriages, with blazing lamps, blocked up the street, to the
disgust of No.\ 100, who could not rest for the thunder of the knocking,
and of 102, who could not sleep for envy.  The gigantic footmen who
accompanied the vehicles were too big to be contained in Becky's little
hall, and were billeted off in the neighbouring public-houses, whence,
when they were wanted, call-boys summoned them from their beer. Scores
of the great dandies of London squeezed and trod on each other on the
little stairs, laughing to find themselves there; and many spotless and
severe ladies of ton were seated in the little drawing-room, listening
to the professional singers, who were singing according to their wont,
and as if they wished to blow the windows down.  And the day after,
there appeared among the fashionable reunions in the Morning Post a
paragraph to the following effect:

``Yesterday, Colonel and Mrs.\ Crawley entertained a select party at
dinner at their house in May Fair.  Their Excellencies the Prince and
Princess of Peterwaradin, H. E. Papoosh Pasha, the Turkish Ambassador
(attended by Kibob Bey, dragoman of the mission), the Marquess of
Steyne, Earl of Southdown, Sir Pitt and Lady Jane Crawley, Mr.\ Wagg,
\&c.  After dinner Mrs.\ Crawley had an assembly which was attended by
the Duchess (Dowager) of Stilton, Duc de la Gruyere, Marchioness of
Cheshire, Marchese Alessandro Strachino, Comte de Brie, Baron
Schapzuger, Chevalier Tosti, Countess of Slingstone, and Lady F.
Macadam, Major-General and Lady G. Macbeth, and (2) Miss Macbeths;
Viscount Paddington, Sir Horace Fogey, Hon.  Sands Bedwin, Bobachy
Bahawder,'' and an \&c., which the reader may fill at his pleasure
through a dozen close lines of small type.

And in her commerce with the great our dear friend showed the same
frankness which distinguished her transactions with the lowly in
station.  On one occasion, when out at a very fine house, Rebecca was
(perhaps rather ostentatiously) holding a conversation in the French
language with a celebrated tenor singer of that nation, while the Lady
Grizzel Macbeth looked over her shoulder scowling at the pair.

``How very well you speak French,'' Lady Grizzel said, who herself spoke
the tongue in an Edinburgh accent most remarkable to hear.

``I ought to know it,'' Becky modestly said, casting down her eyes. ``I
taught it in a school, and my mother was a Frenchwoman.''

Lady Grizzel was won by her humility and was mollified towards the
little woman.  She deplored the fatal levelling tendencies of the age,
which admitted persons of all classes into the society of their
superiors, but her ladyship owned that this one at least was well
behaved and never forgot her place in life.  She was a very good woman:
good to the poor; stupid, blameless, unsuspicious. It is not her
ladyship's fault that she fancies herself better than you and me.  The
skirts of her ancestors' garments have been kissed for centuries; it is
a thousand years, they say, since the tartans of the head of the family
were embraced by the defunct Duncan's lords and councillors, when the
great ancestor of the House became King of Scotland.

Lady Steyne, after the music scene, succumbed before Becky, and perhaps
was not disinclined to her.  The younger ladies of the house of Gaunt
were also compelled into submission.  Once or twice they set people at
her, but they failed.  The brilliant Lady Stunnington tried a passage
of arms with her, but was routed with great slaughter by the intrepid
little Becky.  When attacked sometimes, Becky had a knack of adopting a
demure ingenue air, under which she was most dangerous.  She said the
wickedest things with the most simple unaffected air when in this mood,
and would take care artlessly to apologize for her blunders, so that
all the world should know that she had made them.

Mr.\ Wagg, the celebrated wit, and a led captain and trencher-man of my
Lord Steyne, was caused by the ladies to charge her; and the worthy
fellow, leering at his patronesses and giving them a wink, as much as
to say, ``Now look out for sport,'' one evening began an assault upon
Becky, who was unsuspiciously eating her dinner. The little woman,
attacked on a sudden, but never without arms, lighted up in an instant,
parried and riposted with a home-thrust, which made Wagg's face tingle
with shame; then she returned to her soup with the most perfect calm
and a quiet smile on her face.  Wagg's great patron, who gave him
dinners and lent him a little money sometimes, and whose election,
newspaper, and other jobs Wagg did, gave the luckless fellow such a
savage glance with the eyes as almost made him sink under the table and
burst into tears.  He looked piteously at my lord, who never spoke to
him during dinner, and at the ladies, who disowned him.  At last Becky
herself took compassion upon him and tried to engage him in talk. He
was not asked to dinner again for six weeks; and Fiche, my lord's
confidential man, to whom Wagg naturally paid a good deal of court, was
instructed to tell him that if he ever dared to say a rude thing to
Mrs.\ Crawley again, or make her the butt of his stupid jokes, Milor
would put every one of his notes of hand into his lawyer's hands and
sell him up without mercy.  Wagg wept before Fiche and implored his
dear friend to intercede for him.  He wrote a poem in favour of Mrs.\ R.
C., which appeared in the very next number of the Harum-scarum
Magazine, which he conducted.  He implored her good-will at parties
where he met her.  He cringed and coaxed Rawdon at the club.  He was
allowed to come back to Gaunt House after a while. Becky was always
good to him, always amused, never angry.

His lordship's vizier and chief confidential servant (with a seat in
parliament and at the dinner table), Mr.\ Wenham, was much more prudent
in his behaviour and opinions than Mr.\ Wagg.  However much he might be
disposed to hate all parvenus (Mr.\ Wenham himself was a staunch old
True Blue Tory, and his father a small coal-merchant in the north of
England), this aide-de-camp of the Marquis never showed any sort of
hostility to the new favourite, but pursued her with stealthy
kindnesses and a sly and deferential politeness which somehow made
Becky more uneasy than other people's overt hostilities.

How the Crawleys got the money which was spent upon the entertainments
with which they treated the polite world was a mystery which gave rise
to some conversation at the time, and probably added zest to these
little festivities.  Some persons averred that Sir Pitt Crawley gave
his brother a handsome allowance; if he did, Becky's power over the
Baronet must have been extraordinary indeed, and his character greatly
changed in his advanced age.  Other parties hinted that it was Becky's
habit to levy contributions on all her husband's friends: going to this
one in tears with an account that there was an execution in the house;
falling on her knees to that one and declaring that the whole family
must go to gaol or commit suicide unless such and such a bill could be
paid.  Lord Southdown, it was said, had been induced to give many
hundreds through these pathetic representations. Young Feltham, of the
---th Dragoons (and son of the firm of Tiler and Feltham, hatters and
army accoutrement makers), and whom the Crawleys introduced into
fashionable life, was also cited as one of Becky's victims in the
pecuniary way.  People declared that she got money from various simply
disposed persons, under pretence of getting them confidential
appointments under Government. Who knows what stories were or were not
told of our dear and innocent friend? Certain it is that if she had had
all the money which she was said to have begged or borrowed or stolen,
she might have capitalized and been honest for life, whereas,---but this
is advancing matters.

The truth is, that by economy and good management---by a sparing use of
ready money and by paying scarcely anybody---people can manage, for a
time at least, to make a great show with very little means: and it is
our belief that Becky's much-talked-of parties, which were not, after
all was said, very numerous, cost this lady very little more than the
wax candles which lighted the walls. Stillbrook and Queen's Crawley
supplied her with game and fruit in abundance.  Lord Steyne's cellars
were at her disposal, and that excellent nobleman's famous cooks
presided over her little kitchen, or sent by my lord's order the rarest
delicacies from their own.  I protest it is quite shameful in the world
to abuse a simple creature, as people of her time abuse Becky, and I
warn the public against believing one-tenth of the stories against her.
If every person is to be banished from society who runs into debt and
cannot pay---if we are to be peering into everybody's private life,
speculating upon their income, and cutting them if we don't approve of
their expenditure---why, what a howling wilderness and intolerable
dwelling Vanity Fair would be! Every man's hand would be against his
neighbour in this case, my dear sir, and the benefits of civilization
would be done away with. We should be quarrelling, abusing, avoiding
one another.  Our houses would become caverns, and we should go in rags
because we cared for nobody.  Rents would go down. Parties wouldn't be
given any more. All the tradesmen of the town would be bankrupt.  Wine,
wax-lights, comestibles, rouge, crinoline-petticoats, diamonds, wigs,
Louis-Quatorze gimcracks, and old china, park hacks, and splendid
high-stepping carriage horses---all the delights of life, I say,---would
go to the deuce, if people did but act upon their silly principles and
avoid those whom they dislike and abuse.  Whereas, by a little charity
and mutual forbearance, things are made to go on pleasantly enough:  we
may abuse a man as much as we like, and call him the greatest rascal
unhanged---but do we wish to hang him therefore? No. We shake hands when
we meet.  If his cook is good we forgive him and go and dine with him,
and we expect he will do the same by us.  Thus trade
flourishes---civilization advances; peace is kept; new dresses are
wanted for new assemblies every week; and the last year's vintage of
Lafitte will remunerate the honest proprietor who reared it.

At the time whereof we are writing, though the Great George was on the
throne and ladies wore gigots and large combs like tortoise-shell
shovels in their hair, instead of the simple sleeves and lovely wreaths
which are actually in fashion, the manners of the very polite world
were not, I take it, essentially different from those of the present
day:  and their amusements pretty similar.  To us, from the outside,
gazing over the policeman's shoulders at the bewildering beauties as
they pass into Court or ball, they may seem beings of unearthly
splendour and in the enjoyment of an exquisite happiness by us
unattainable. It is to console some of these dissatisfied beings that
we are narrating our dear Becky's struggles, and triumphs, and
disappointments, of all of which, indeed, as is the case with all
persons of merit, she had her share.

At this time the amiable amusement of acting charades had come among us
from France, and was considerably in vogue in this country, enabling
the many ladies amongst us who had beauty to display their charms, and
the fewer number who had cleverness to exhibit their wit. My Lord
Steyne was incited by Becky, who perhaps believed herself endowed with
both the above qualifications, to give an entertainment at Gaunt House,
which should include some of these little dramas---and we must take
leave to introduce the reader to this brilliant reunion, and, with a
melancholy welcome too, for it will be among the very last of the
fashionable entertainments to which it will be our fortune to conduct
him.

A portion of that splendid room, the picture gallery of Gaunt House,
was arranged as the charade theatre.  It had been so used when George
III was king; and a picture of the Marquis of Gaunt is still extant,
with his hair in powder and a pink ribbon, in a Roman shape, as it was
called, enacting the part of Cato in Mr.\ Addison's tragedy of that
name, performed before their Royal Highnesses the Prince of Wales, the
Bishop of Osnaburgh, and Prince William Henry, then children like the
actor. One or two of the old properties were drawn out of the garrets,
where they had lain ever since, and furbished up anew for the present
festivities.

Young Bedwin Sands, then an elegant dandy and Eastern traveller, was
manager of the revels.  An Eastern traveller was somebody in those
days, and the adventurous Bedwin, who had published his quarto and
passed some months under the tents in the desert, was a personage of no
small importance.  In his volume there were several pictures of Sands
in various oriental costumes; and he travelled about with a black
attendant of most unprepossessing appearance, just like another Brian
de Bois Guilbert.  Bedwin, his costumes, and black man, were hailed at
Gaunt House as very valuable acquisitions.

He led off the first charade.  A Turkish officer with an immense plume
of feathers (the Janizaries were supposed to be still in existence, and
the tarboosh had not as yet displaced the ancient and majestic
head-dress of the true believers) was seen couched on a divan, and
making believe to puff at a narghile, in which, however, for the sake
of the ladies, only a fragrant pastille was allowed to smoke.  The
Turkish dignitary yawns and expresses signs of weariness and idleness.
He claps his hands and Mesrour the Nubian appears, with bare arms,
bangles, yataghans, and every Eastern ornament---gaunt, tall, and
hideous.  He makes a salaam before my lord the Aga.

A thrill of terror and delight runs through the assembly. The ladies
whisper to one another.  The black slave was given to Bedwin Sands by
an Egyptian pasha in exchange for three dozen of Maraschino.  He has
sewn up ever so many odalisques in sacks and tilted them into the Nile.

``Bid the slave-merchant enter,'' says the Turkish voluptuary with a wave
of his hand.  Mesrour conducts the slave-merchant into my lord's
presence; he brings a veiled female with him.  He removes the veil.  A
thrill of applause bursts through the house.  It is Mrs.\ Winkworth (she
was a Miss Absolom) with the beautiful eyes and hair. She is in a
gorgeous oriental costume; the black braided locks are twined with
innumerable jewels; her dress is covered over with gold piastres.  The
odious Mahometan expresses himself charmed by her beauty.  She falls
down on her knees and entreats him to restore her to the mountains
where she was born, and where her Circassian lover is still deploring
the absence of his Zuleikah. No entreaties will move the obdurate
Hassan.  He laughs at the notion of the Circassian bridegroom. Zuleikah
covers her face with her hands and drops down in an attitude of the
most beautiful despair.  There seems to be no hope for her, when---when
the Kislar Aga appears.

The Kislar Aga brings a letter from the Sultan.  Hassan receives and
places on his head the dread firman.  A ghastly terror seizes him,
while on the Negro's face (it is Mesrour again in another costume)
appears a ghastly joy.  ``Mercy!  mercy!'' cries the Pasha:  while the
Kislar Aga, grinning horribly, pulls out---a bow-string.

The curtain draws just as he is going to use that awful weapon. Hassan
from within bawls out, ``First two syllables''---and Mrs.\ Rawdon Crawley,
who is going to act in the charade, comes forward and compliments Mrs.\ %
Winkworth on the admirable taste and beauty of her costume.

The second part of the charade takes place.  It is still an Eastern
scene.  Hassan, in another dress, is in an attitude by Zuleikah, who is
perfectly reconciled to him. The Kislar Aga has become a peaceful black
slave.  It is sunrise on the desert, and the Turks turn their heads
eastwards and bow to the sand.  As there are no dromedaries at hand,
the band facetiously plays ``The Camels are coming.'' An enormous
Egyptian head figures in the scene.  It is a musical one---and, to the
surprise of the oriental travellers, sings a comic song, composed by
Mr.\ Wagg.  The Eastern voyagers go off dancing, like Papageno and the
Moorish King in The Magic Flute.  ``Last two syllables,'' roars the head.

The last act opens.  It is a Grecian tent this time.  A tall and
stalwart man reposes on a couch there.  Above him hang his helmet and
shield.  There is no need for them now.  Ilium is down. Iphigenia is
slain.  Cassandra is a prisoner in his outer halls. The king of men (it
is Colonel Crawley, who, indeed, has no notion about the sack of Ilium
or the conquest of Cassandra), the anax andron is asleep in his chamber
at Argos.  A lamp casts the broad shadow of the sleeping warrior
flickering on the wall---the sword and shield of Troy glitter in its
light. The band plays the awful music of Don Juan, before the statue
enters.

Aegisthus steals in pale and on tiptoe.  What is that ghastly face
looking out balefully after him from behind the arras? He raises his
dagger to strike the sleeper, who turns in his bed, and opens his broad
chest as if for the blow.  He cannot strike the noble slumbering
chieftain. Clytemnestra glides swiftly into the room like an
apparition---her arms are bare and white---her tawny hair floats down her
shoulders---her face is deadly pale---and her eyes are lighted up with a
smile so ghastly that people quake as they look at her.

A tremor ran through the room.  ``Good God!'' somebody said, ``it's Mrs.\ %
Rawdon Crawley.''

Scornfully she snatches the dagger out of Aegisthus's hand and advances
to the bed.  You see it shining over her head in the glimmer of the
lamp, and---and the lamp goes out, with a groan, and all is dark.

The darkness and the scene frightened people.  Rebecca performed her
part so well, and with such ghastly truth, that the spectators were all
dumb, until, with a burst, all the lamps of the hall blazed out again,
when everybody began to shout applause.  ``Brava!  brava!'' old Steyne's
strident voice was heard roaring over all the rest.  ``By---, she'd do it
too,'' he said between his teeth. The performers were called by the
whole house, which sounded with cries of ``Manager! Clytemnestra!''
Agamemnon could not be got to show in his classical tunic, but stood in
the background with Aegisthus and others of the performers of the
little play.  Mr.\ Bedwin Sands led on Zuleikah and Clytemnestra.  A
great personage insisted on being presented to the charming
Clytemnestra.  ``Heigh ha? Run him through the body. Marry somebody
else, hay?'' was the apposite remark made by His Royal Highness.

``Mrs.\ Rawdon Crawley was quite killing in the part,'' said Lord Steyne.
Becky laughed, gay and saucy looking, and swept the prettiest little
curtsey ever seen.

Servants brought in salvers covered with numerous cool dainties, and
the performers disappeared to get ready for the second charade-tableau.

The three syllables of this charade were to be depicted in pantomime,
and the performance took place in the following wise:

First syllable.  Colonel Rawdon Crawley, C.B., with a slouched hat and
a staff, a great-coat, and a lantern borrowed from the stables, passed
across the stage bawling out, as if warning the inhabitants of the
hour.  In the lower window are seen two bagmen playing apparently at
the game of cribbage, over which they yawn much. To them enters one
looking like Boots (the Honourable G.  Ringwood), which character the
young gentleman performed to perfection, and divests them of their
lower coverings; and presently Chambermaid (the Right Honourable Lord
Southdown) with two candlesticks, and a warming-pan.  She ascends to
the upper apartment and warms the bed. She uses the warming-pan as a
weapon wherewith she wards off the attention of the bagmen. She exits.
They put on their night-caps and pull down the blinds.  Boots comes out
and closes the shutters of the ground-floor chamber.  You hear him
bolting and chaining the door within.  All the lights go out.  The
music plays Dormez, dormez, chers Amours.  A voice from behind the
curtain says, ``First syllable.''

Second syllable.  The lamps are lighted up all of a sudden.  The music
plays the old air from John of Paris, Ah quel plaisir d'etre en voyage.
It is the same scene. Between the first and second floors of the house
represented, you behold a sign on which the Steyne arms are painted.
All the bells are ringing all over the house. In the lower apartment
you see a man with a long slip of paper presenting it to another, who
shakes his fists, threatens and vows that it is monstrous.  ``Ostler,
bring round my gig,'' cries another at the door.  He chucks Chambermaid
(the Right Honourable Lord Southdown) under the chin; she seems to
deplore his absence, as Calypso did that of that other eminent
traveller Ulysses. Boots (the Honourable G.  Ringwood) passes with a
wooden box, containing silver flagons, and cries ``Pots'' with such
exquisite humour and naturalness that the whole house rings with
applause, and a bouquet is thrown to him.  Crack, crack, crack, go the
whips.   Landlord, chambermaid, waiter rush to the door, but just as
some distinguished guest is arriving, the curtains close, and the
invisible theatrical manager cries out ``Second syllable.''

``I think it must be 'Hotel,''' says Captain Grigg of the Life Guards;
there is a general laugh at the Captain's cleverness.  He is not very
far from the mark.

While the third syllable is in preparation, the band begins a nautical
medley---``All in the Downs,'' ``Cease Rude Boreas,'' ``Rule Britannia,'' ``In
the Bay of Biscay O!''---some maritime event is about to take place.  A
bell is heard ringing as the curtain draws aside. ``Now, gents, for the
shore!'' a voice exclaims.  People take leave of each other.  They point
anxiously as if towards the clouds, which are represented by a dark
curtain, and they nod their heads in fear. Lady Squeams (the Right
Honourable Lord Southdown), her lap-dog, her bags, reticules, and
husband sit down, and cling hold of some ropes. It is evidently a ship.

The Captain (Colonel Crawley, C.B.), with a cocked hat and a telescope,
comes in, holding his hat on his head, and looks out; his coat tails
fly about as if in the wind.  When he leaves go of his hat to use his
telescope, his hat flies off, with immense applause. It is blowing
fresh.  The music rises and whistles louder and louder; the mariners go
across the stage staggering, as if the ship was in severe motion.  The
Steward (the Honourable G. Ringwood) passes reeling by, holding six
basins.  He puts one rapidly by Lord Squeams---Lady Squeams, giving a
pinch to her dog, which begins to howl piteously, puts her
pocket-handkerchief to her face, and rushes away as for the cabin.  The
music rises up to the wildest pitch of stormy excitement, and the third
syllable is concluded.

There was a little ballet, ``Le Rossignol,'' in which Montessu and Noblet
used to be famous in those days, and which Mr.\ Wagg transferred to the
English stage as an opera, putting his verse, of which he was a skilful
writer, to the pretty airs of the ballet.  It was dressed in old French
costume, and little Lord Southdown now appeared admirably attired in
the disguise of an old woman hobbling about the stage with a faultless
crooked stick.

Trills of melody were heard behind the scenes, and gurgling from a
sweet pasteboard cottage covered with roses and trellis work.
``Philomele, Philomele,'' cries the old woman, and Philomele comes out.

More applause---it is Mrs.\ Rawdon Crawley in powder and patches, the
most ravissante little Marquise in the world.

She comes in laughing, humming, and frisks about the stage with all the
innocence of theatrical youth---she makes a curtsey.  Mamma says ``Why,
child, you are always laughing and singing,'' and away she goes, with---

  THE ROSE UPON MY BALCONY

  The rose upon my balcony the morning air perfuming
  Was leafless all the winter time and pining for the spring;
  You ask me why her breath is sweet and why her cheek is blooming,
  It is because the sun is out and birds begin to sing.

  The nightingale, whose melody is through the greenwood ringing,
  Was silent when the boughs were bare and winds were blowing keen:
  And if, Mamma, you ask of me the reason of his singing,
  It is because the sun is out and all the leaves are green.

  Thus each performs his part, Mamma, the birds have found their voices,
  The blowing rose a flush, Mamma, her bonny cheek to dye;
  And there's sunshine in my heart, Mamma, which wakens and rejoices,
  And so I sing and blush, Mamma, and that's the reason why.


During the intervals of the stanzas of this ditty, the good-natured
personage addressed as Mamma by the singer, and whose large whiskers
appeared under her cap, seemed very anxious to exhibit her maternal
affection by embracing the innocent creature who performed the
daughter's part.  Every caress was received with loud acclamations of
laughter by the sympathizing audience. At its conclusion (while the
music was performing a symphony as if ever so many birds were warbling)
the whole house was unanimous for an encore:  and applause and bouquets
without end were showered upon the Nightingale of the evening.  Lord
Steyne's voice of applause was loudest of all. Becky, the nightingale,
took the flowers which he threw to her and pressed them to her heart
with the air of a consummate comedian. Lord Steyne was frantic with
delight.  His guests' enthusiasm harmonized with his own.  Where was
the beautiful black-eyed Houri whose appearance in the first charade
had caused such delight? She was twice as handsome as Becky, but the
brilliancy of the latter had quite eclipsed her.  All voices were for
her.  Stephens, Caradori, Ronzi de Begnis, people compared her to one
or the other, and agreed with good reason, very likely, that had she
been an actress none on the stage could have surpassed her. She had
reached her culmination: her voice rose trilling and bright over the
storm of applause, and soared as high and joyful as her triumph.  There
was a ball after the dramatic entertainments, and everybody pressed
round Becky as the great point of attraction of the evening.  The Royal
Personage declared with an oath that she was perfection, and engaged
her again and again in conversation.  Little Becky's soul swelled with
pride and delight at these honours; she saw fortune, fame, fashion
before her.  Lord Steyne was her slave, followed her everywhere, and
scarcely spoke to any one in the room beside, and paid her the most
marked compliments and attention.  She still appeared in her Marquise
costume and danced a minuet with Monsieur de Truffigny, Monsieur Le Duc
de la Jabotiere's attache; and the Duke, who had all the traditions of
the ancient court, pronounced that Madame Crawley was worthy to have
been a pupil of Vestris, or to have figured at Versailles. Only a
feeling of dignity, the gout, and the strongest sense of duty and
personal sacrifice prevented his Excellency from dancing with her
himself, and he declared in public that a lady who could talk and dance
like Mrs.\ Rawdon was fit to be ambassadress at any court in Europe.  He
was only consoled when he heard that she was half a Frenchwoman by
birth.  ``None but a compatriot,'' his Excellency declared, ``could have
performed that majestic dance in such a way.''

Then she figured in a waltz with Monsieur de Klingenspohr, the Prince
of Peterwaradin's cousin and attache.  The delighted Prince, having
less retenue than his French diplomatic colleague, insisted upon taking
a turn with the charming creature, and twirled round the ball-room with
her, scattering the diamonds out of his boot-tassels and hussar jacket
until his Highness was fairly out of breath. Papoosh Pasha himself
would have liked to dance with her if that amusement had been the
custom of his country.  The company made a circle round her and
applauded as wildly as if she had been a Noblet or a Taglioni.
Everybody was in ecstacy; and Becky too, you may be sure.  She passed
by Lady Stunnington with a look of scorn.  She patronized Lady Gaunt
and her astonished and mortified sister-in-law---she ecrased all rival
charmers.  As for poor Mrs.\ Winkworth, and her long hair and great
eyes, which had made such an effect at the commencement of the
evening---where was she now? Nowhere in the race.  She might tear her
long hair and cry her great eyes out, but there was not a person to
heed or to deplore the discomfiture.

The greatest triumph of all was at supper time.  She was placed at the
grand exclusive table with his Royal Highness the exalted personage
before mentioned, and the rest of the great guests.  She was served on
gold plate.  She might have had pearls melted into her champagne if she
liked---another Cleopatra---and the potentate of Peterwaradin would have
given half the brilliants off his jacket for a kind glance from those
dazzling eyes. Jabotiere wrote home about her to his government.  The
ladies at the other tables, who supped off mere silver and marked Lord
Steyne's constant attention to her, vowed it was a monstrous
infatuation, a gross insult to ladies of rank.  If sarcasm could have
killed, Lady Stunnington would have slain her on the spot.

Rawdon Crawley was scared at these triumphs.  They seemed to separate
his wife farther than ever from him somehow.  He thought with a feeling
very like pain how immeasurably she was his superior.

When the hour of departure came, a crowd of young men followed her to
her carriage, for which the people without bawled, the cry being caught
up by the link-men who were stationed outside the tall gates of Gaunt
House, congratulating each person who issued from the gate and hoping
his Lordship had enjoyed this noble party.

Mrs.\ Rawdon Crawley's carriage, coming up to the gate after due
shouting, rattled into the illuminated court-yard and drove up to the
covered way.  Rawdon put his wife into the carriage, which drove off.
Mr.\ Wenham had proposed to him to walk home, and offered the Colonel
the refreshment of a cigar.

They lighted their cigars by the lamp of one of the many link-boys
outside, and Rawdon walked on with his friend Wenham.  Two persons
separated from the crowd and followed the two gentlemen; and when they
had walked down Gaunt Square a few score of paces, one of the men came
up and, touching Rawdon on the shoulder, said, ``Beg your pardon,
Colonel, I vish to speak to you most particular.'' This gentleman's
acquaintance gave a loud whistle as the latter spoke, at which signal a
cab came clattering up from those stationed at the gate of Gaunt
House---and the aide-de-camp ran round and placed himself in front of
Colonel Crawley.

That gallant officer at once knew what had befallen him.  He was in the
hands of the bailiffs.  He started back, falling against the man who
had first touched him.

``We're three on us---it's no use bolting,'' the man behind said.

``It's you, Moss, is it?'' said the Colonel, who appeared to know his
interlocutor.  ``How much is it?''

``Only a small thing,'' whispered Mr.\ Moss, of Cursitor Street, Chancery
Lane, and assistant officer to the Sheriff of Middlesex---``One hundred
and sixty-six, six and eight-pence, at the suit of Mr.\ Nathan.''

``Lend me a hundred, Wenham, for God's sake,'' poor Rawdon said---``I've
got seventy at home.''

``I've not got ten pounds in the world,'' said poor Mr.\ Wenham---``Good
night, my dear fellow.''

``Good night,'' said Rawdon ruefully.  And Wenham walked away---and Rawdon
Crawley finished his cigar as the cab drove under Temple Bar.



\chapter{In Which Lord Steyne Shows Himself in a Most Amiable Light}

When Lord Steyne was benevolently disposed, he did nothing by halves,
and his kindness towards the Crawley family did the greatest honour to
his benevolent discrimination.  His lordship extended his good-will to
little Rawdon:  he pointed out to the boy's parents the necessity of
sending him to a public school, that he was of an age now when
emulation, the first principles of the Latin language, pugilistic
exercises, and the society of his fellow-boys would be of the greatest
benefit to the boy.  His father objected that he was not rich enough to
send the child to a good public school; his mother that Briggs was a
capital mistress for him, and had brought him on (as indeed was the
fact) famously in English, the Latin rudiments, and in general
learning:  but all these objections disappeared before the generous
perseverance of the Marquis of Steyne.  His lordship was one of the
governors of that famous old collegiate institution called the
Whitefriars.  It had been a Cistercian Convent in old days, when the
Smithfield, which is contiguous to it, was a tournament ground.
Obstinate heretics used to be brought thither convenient for burning
hard by.  Henry VIII, the Defender of the Faith, seized upon the
monastery and its possessions and hanged and tortured some of the monks
who could not accommodate themselves to the pace of his reform.
Finally, a great merchant bought the house and land adjoining, in
which, and with the help of other wealthy endowments of land and money,
he established a famous foundation hospital for old men and children.
An extern school grew round the old almost monastic foundation, which
subsists still with its middle-age costume and usages---and all
Cistercians pray that it may long flourish.

Of this famous house, some of the greatest noblemen, prelates, and
dignitaries in England are governors:  and as the boys are very
comfortably lodged, fed, and educated, and subsequently inducted to
good scholarships at the University and livings in the Church, many
little gentlemen are devoted to the ecclesiastical profession from
their tenderest years, and there is considerable emulation to procure
nominations for the foundation.  It was originally intended for the
sons of poor and deserving clerics and laics, but many of the noble
governors of the Institution, with an enlarged and rather capricious
benevolence, selected all sorts of objects for their bounty. To get an
education for nothing, and a future livelihood and profession assured,
was so excellent a scheme that some of the richest people did not
disdain it; and not only great men's relations, but great men
themselves, sent their sons to profit by the chance---Right Rev.
prelates sent their own kinsmen or the sons of their clergy, while, on
the other hand, some great noblemen did not disdain to patronize the
children of their confidential servants---so that a lad entering this
establishment had every variety of youthful society wherewith to mingle.

Rawdon Crawley, though the only book which he studied was the Racing
Calendar, and though his chief recollections of polite learning were
connected with the floggings which he received at Eton in his early
youth, had that decent and honest reverence for classical learning
which all English gentlemen feel, and was glad to think that his son
was to have a provision for life, perhaps, and a certain opportunity of
becoming a scholar.  And although his boy was his chief solace and
companion, and endeared to him by a thousand small ties, about which he
did not care to speak to his wife, who had all along shown the utmost
indifference to their son, yet Rawdon agreed at once to part with him
and to give up his own greatest comfort and benefit for the sake of the
welfare of the little lad.  He did not know how fond he was of the
child until it became necessary to let him go away. When he was gone,
he felt more sad and downcast than he cared to own---far sadder than the
boy himself, who was happy enough to enter a new career and find
companions of his own age.  Becky burst out laughing once or twice when
the Colonel, in his clumsy, incoherent way, tried to express his
sentimental sorrows at the boy's departure.  The poor fellow felt that
his dearest pleasure and closest friend was taken from him.  He looked
often and wistfully at the little vacant bed in his dressing-room,
where the child used to sleep.  He missed him sadly of mornings and
tried in vain to walk in the park without him.  He did not know how
solitary he was until little Rawdon was gone.  He liked the people who
were fond of him, and would go and sit for long hours with his
good-natured sister Lady Jane, and talk to her about the virtues, and
good looks, and hundred good qualities of the child.

Young Rawdon's aunt, we have said, was very fond of him, as was her
little girl, who wept copiously when the time for her cousin's
departure came.  The elder Rawdon was thankful for the fondness of
mother and daughter.  The very best and honestest feelings of the man
came out in these artless outpourings of paternal feeling in which he
indulged in their presence, and encouraged by their sympathy.  He
secured not only Lady Jane's kindness, but her sincere regard, by the
feelings which he manifested, and which he could not show to his own
wife.  The two kinswomen met as seldom as possible. Becky laughed
bitterly at Jane's feelings and softness; the other's kindly and gentle
nature could not but revolt at her sister's callous behaviour.

It estranged Rawdon from his wife more than he knew or acknowledged to
himself.  She did not care for the estrangement.  Indeed, she did not
miss him or anybody. She looked upon him as her errand-man and humble
slave.  He might be ever so depressed or sulky, and she did not mark
his demeanour, or only treated it with a sneer.  She was busy thinking
about her position, or her pleasures, or her advancement in society;
she ought to have held a great place in it, that is certain.

It was honest Briggs who made up the little kit for the boy which he
was to take to school.  Molly, the housemaid, blubbered in the passage
when he went away---Molly kind and faithful in spite of a long arrear of
unpaid wages.  Mrs.\ Becky could not let her husband have the carriage
to take the boy to school.  Take the horses into the City!---such a
thing was never heard of.  Let a cab be brought. She did not offer to
kiss him when he went, nor did the child propose to embrace her; but
gave a kiss to old Briggs (whom, in general, he was very shy of
caressing), and consoled her by pointing out that he was to come home
on Saturdays, when she would have the benefit of seeing him.  As the
cab rolled towards the City, Becky's carriage rattled off to the park.
She was chattering and laughing with a score of young dandies by the
Serpentine as the father and son entered at the old gates of the
school---where Rawdon left the child and came away with a sadder purer
feeling in his heart than perhaps that poor battered fellow had ever
known since he himself came out of the nursery.

He walked all the way home very dismally, and dined alone with Briggs.
He was very kind to her and grateful for her love and watchfulness over
the boy.  His conscience smote him that he had borrowed Briggs's money
and aided in deceiving her.  They talked about little Rawdon a long
time, for Becky only came home to dress and go out to dinner---and then
he went off uneasily to drink tea with Lady Jane, and tell her of what
had happened, and how little Rawdon went off like a trump, and how he
was to wear a gown and little knee-breeches, and how young Blackball,
Jack Blackball's son, of the old regiment, had taken him in charge and
promised to be kind to him.

In the course of a week, young Blackball had constituted little Rawdon
his fag, shoe-black, and breakfast toaster; initiated him into the
mysteries of the Latin Grammar; and thrashed him three or four times,
but not severely.  The little chap's good-natured honest face won his
way for him.  He only got that degree of beating which was, no doubt,
good for him; and as for blacking shoes, toasting bread, and fagging in
general, were these offices not deemed to be necessary parts of every
young English gentleman's education?

Our business does not lie with the second generation and Master
Rawdon's life at school, otherwise the present tale might be carried to
any indefinite length.  The Colonel went to see his son a short time
afterwards and found the lad sufficiently well and happy, grinning and
laughing in his little black gown and little breeches.

His father sagaciously tipped Blackball, his master, a sovereign, and
secured that young gentleman's good-will towards his fag.  As a protege
of the great Lord Steyne, the nephew of a County member, and son of a
Colonel and C.B., whose name appeared in some of the most fashionable
parties in the Morning Post, perhaps the school authorities were
disposed not to look unkindly on the child.  He had plenty of
pocket-money, which he spent in treating his comrades royally to
raspberry tarts, and he was often allowed to come home on Saturdays to
his father, who always made a jubilee of that day. When free, Rawdon
would take him to the play, or send him thither with the footman; and
on Sundays he went to church with Briggs and Lady Jane and his cousins.
Rawdon marvelled over his stories about school, and fights, and
fagging.  Before long, he knew the names of all the masters and the
principal boys as well as little Rawdon himself.  He invited little
Rawdon's crony from school, and made both the children sick with
pastry, and oysters, and porter after the play.  He tried to look
knowing over the Latin grammar when little Rawdon showed him what part
of that work he was ``in.'' ``Stick to it, my boy,'' he said to him with
much gravity, ``there's nothing like a good classical education!
Nothing!''

Becky's contempt for her husband grew greater every day.  ``Do what you
like---dine where you please---go and have ginger-beer and sawdust at
Astley's, or psalm-singing with Lady Jane---only don't expect me to busy
myself with the boy.  I have your interests to attend to, as you can't
attend to them yourself.  I should like to know where you would have
been now, and in what sort of a position in society, if I had not
looked after you.'' Indeed, nobody wanted poor old Rawdon at the parties
whither Becky used to go.  She was often asked without him now.  She
talked about great people as if she had the fee-simple of May Fair, and
when the Court went into mourning, she always wore black.

Little Rawdon being disposed of, Lord Steyne, who took such a parental
interest in the affairs of this amiable poor family, thought that their
expenses might be very advantageously curtailed by the departure of
Miss Briggs, and that Becky was quite clever enough to take the
management of her own house.  It has been narrated in a former chapter
how the benevolent nobleman had given his protegee money to pay off her
little debt to Miss Briggs, who however still remained behind with her
friends; whence my lord came to the painful conclusion that Mrs.\ %
Crawley had made some other use of the money confided to her than that
for which her generous patron had given the loan.  However, Lord Steyne
was not so rude as to impart his suspicions upon this head to Mrs.\ %
Becky, whose feelings might be hurt by any controversy on the
money-question, and who might have a thousand painful reasons for
disposing otherwise of his lordship's generous loan.  But he determined
to satisfy himself of the real state of the case, and instituted the
necessary inquiries in a most cautious and delicate manner.

In the first place he took an early opportunity of pumping Miss Briggs.
That was not a difficult operation. A very little encouragement would
set that worthy woman to talk volubly and pour out all within her.  And
one day when Mrs.\ Rawdon had gone out to drive (as Mr.\ Fiche, his
lordship's confidential servant, easily learned at the livery stables
where the Crawleys kept their carriage and horses, or rather, where the
livery-man kept a carriage and horses for Mr.\ and Mrs.\ Crawley)---my
lord dropped in upon the Curzon Street house---asked Briggs for a cup of
coffee---told her that he had good accounts of the little boy at
school---and in five minutes found out from her that Mrs.\ Rawdon had
given her nothing except a black silk gown, for which Miss Briggs was
immensely grateful.

He laughed within himself at this artless story.  For the truth is, our
dear friend Rebecca had given him a most circumstantial narration of
Briggs's delight at receiving her money---eleven hundred and twenty-five
pounds---and in what securities she had invested it; and what a pang
Becky herself felt in being obliged to pay away such a delightful sum
of money.  ``Who knows,'' the dear woman may have thought within herself,
``perhaps he may give me a little more?'' My lord, however, made no such
proposal to the little schemer---very likely thinking that he had been
sufficiently generous already.

He had the curiosity, then, to ask Miss Briggs about the state of her
private affairs---and she told his lordship candidly what her position
was---how Miss Crawley had left her a legacy---how her relatives had had
part of it---how Colonel Crawley had put out another portion, for which
she had the best security and interest---and how Mr.\ and Mrs.\ Rawdon
had kindly busied themselves with Sir Pitt, who was to dispose of the
remainder most advantageously for her, when he had time.  My lord asked
how much the Colonel had already invested for her, and Miss Briggs at
once and truly told him that the sum was six hundred and odd pounds.

But as soon as she had told her story, the voluble Briggs repented of
her frankness and besought my lord not to tell Mr.\ Crawley of the
confessions which she had made.  ``The Colonel was so kind---Mr.\ Crawley
might be offended and pay back the money, for which she could get no
such good interest anywhere else.'' Lord Steyne, laughing, promised he
never would divulge their conversation, and when he and Miss Briggs
parted he laughed still more.

``What an accomplished little devil it is!'' thought he. ``What a splendid
actress and manager!  She had almost got a second supply out of me the
other day; with her coaxing ways.  She beats all the women I have ever
seen in the course of all my well-spent life. They are babies compared
to her.  I am a greenhorn myself, and a fool in her hands---an old fool.
She is unsurpassable in lies.'' His lordship's admiration for Becky rose
immeasurably at this proof of her cleverness.  Getting the money was
nothing---but getting double the sum she wanted, and paying nobody---it
was a magnificent stroke. And Crawley, my lord thought---Crawley is not
such a fool as he looks and seems.  He has managed the matter cleverly
enough on his side. Nobody would ever have supposed from his face and
demeanour that he knew anything about this money business; and yet he
put her up to it, and has spent the money, no doubt.  In this opinion
my lord, we know, was mistaken, but it influenced a good deal his
behaviour towards Colonel Crawley, whom he began to treat with even
less than that semblance of respect which he had formerly shown towards
that gentleman.  It never entered into the head of Mrs.\ Crawley's
patron that the little lady might be making a purse for herself; and,
perhaps, if the truth must be told, he judged of Colonel Crawley by his
experience of other husbands, whom he had known in the course of the
long and well-spent life which had made him acquainted with a great
deal of the weakness of mankind.  My lord had bought so many men during
his life that he was surely to be pardoned for supposing that he had
found the price of this one.

He taxed Becky upon the point on the very first occasion when he met
her alone, and he complimented her, good-humouredly, on her cleverness
in getting more than the money which she required.  Becky was only a
little taken aback.  It was not the habit of this dear creature to tell
falsehoods, except when necessity compelled, but in these great
emergencies it was her practice to lie very freely; and in an instant
she was ready with another neat plausible circumstantial story which
she administered to her patron.  The previous statement which she had
made to him was a falsehood---a wicked falsehood---she owned it.  But who
had made her tell it? ``Ah, my Lord,'' she said, ``you don't know all I
have to suffer and bear in silence; you see me gay and happy before
you---you little know what I have to endure when there is no protector
near me.  It was my husband, by threats and the most savage treatment,
forced me to ask for that sum about which I deceived you.  It was he
who, foreseeing that questions might be asked regarding the disposal of
the money, forced me to account for it as I did.  He took the money.
He told me he had paid Miss Briggs; I did not want, I did not dare to
doubt him. Pardon the wrong which a desperate man is forced to commit,
and pity a miserable, miserable woman.'' She burst into tears as she
spoke.  Persecuted virtue never looked more bewitchingly wretched.

They had a long conversation, driving round and round the Regent's Park
in Mrs.\ Crawley's carriage together, a conversation of which it is not
necessary to repeat the details, but the upshot of it was that, when
Becky came home, she flew to her dear Briggs with a smiling face and
announced that she had some very good news for her. Lord Steyne had
acted in the noblest and most generous manner.  He was always thinking
how and when he could do good.  Now that little Rawdon was gone to
school, a dear companion and friend was no longer necessary to her.
She was grieved beyond measure to part with Briggs, but her means
required that she should practise every retrenchment, and her sorrow
was mitigated by the idea that her dear Briggs would be far better
provided for by her generous patron than in her humble home.  Mrs.\ %
Pilkington, the housekeeper at Gauntly Hall, was growing exceedingly
old, feeble, and rheumatic: she was not equal to the work of
superintending that vast mansion, and must be on the look out for a
successor.  It was a splendid position. The family did not go to
Gauntly once in two years.  At other times the housekeeper was the
mistress of the magnificent mansion---had four covers daily for her
table; was visited by the clergy and the most respectable people of the
county---was the lady of Gauntly, in fact; and the two last housekeepers
before Mrs.\ Pilkington had married rectors of Gauntly---but Mrs.\ P.
could not, being the aunt of the present Rector.  The place was not to
be hers yet, but she might go down on a visit to Mrs.\ Pilkington and
see whether she would like to succeed her.

What words can paint the ecstatic gratitude of Briggs! All she
stipulated for was that little Rawdon should be allowed to come down
and see her at the Hall.  Becky promised this---anything.  She ran up to
her husband when he came home and told him the joyful news. Rawdon was
glad, deuced glad; the weight was off his conscience about poor
Briggs's money.  She was provided for, at any rate, but---but his mind
was disquiet.  He did not seem to be all right, somehow.  He told
little Southdown what Lord Steyne had done, and the young man eyed
Crawley with an air which surprised the latter.

He told Lady Jane of this second proof of Steyne's bounty, and she,
too, looked odd and alarmed; so did Sir Pitt.  ``She is too clever
and---and gay to be allowed to go from party to party without a
companion,'' both said.  ``You must go with her, Rawdon, wherever she
goes, and you must have somebody with her---one of the girls from
Queen's Crawley, perhaps, though they were rather giddy guardians for
her.''

Somebody Becky should have.  But in the meantime it was clear that
honest Briggs must not lose her chance of settlement for life, and so
she and her bags were packed, and she set off on her journey. And so
two of Rawdon's out-sentinels were in the hands of the enemy.

Sir Pitt went and expostulated with his sister-in-law upon the subject
of the dismissal of Briggs and other matters of delicate family
interest.  In vain she pointed out to him how necessary was the
protection of Lord Steyne for her poor husband; how cruel it would be
on their part to deprive Briggs of the position offered to her.
Cajolements, coaxings, smiles, tears could not satisfy Sir Pitt, and he
had something very like a quarrel with his once admired Becky.  He
spoke of the honour of the family, the unsullied reputation of the
Crawleys; expressed himself in indignant tones about her receiving
those young Frenchmen---those wild young men of fashion, my Lord Steyne
himself, whose carriage was always at her door, who passed hours daily
in her company, and whose constant presence made the world talk about
her.  As the head of the house he implored her to be more prudent.
Society was already speaking lightly of her.  Lord Steyne, though a
nobleman of the greatest station and talents, was a man whose
attentions would compromise any woman; he besought, he implored, he
commanded his sister-in-law to be watchful in her intercourse with that
nobleman.

Becky promised anything and everything Pitt wanted; but Lord Steyne
came to her house as often as ever, and Sir Pitt's anger increased. I
wonder was Lady Jane angry or pleased that her husband at last found
fault with his favourite Rebecca? Lord Steyne's visits continuing, his
own ceased, and his wife was for refusing all further intercourse with
that nobleman and declining the invitation to the charade-night which
the marchioness sent to her; but Sir Pitt thought it was necessary to
accept it, as his Royal Highness would be there.

Although he went to the party in question, Sir Pitt quitted it very
early, and his wife, too, was very glad to come away.  Becky hardly so
much as spoke to him or noticed her sister-in-law.  Pitt Crawley
declared her behaviour was monstrously indecorous, reprobated in strong
terms the habit of play-acting and fancy dressing as highly unbecoming
a British female, and after the charades were over, took his brother
Rawdon severely to task for appearing himself and allowing his wife to
join in such improper exhibitions.

Rawdon said she should not join in any more such amusements---but
indeed, and perhaps from hints from his elder brother and sister, he
had already become a very watchful and exemplary domestic character. He
left off his clubs and billiards.  He never left home.  He took Becky
out to drive; he went laboriously with her to all her parties. Whenever
my Lord Steyne called, he was sure to find the Colonel. And when Becky
proposed to go out without her husband, or received invitations for
herself, he peremptorily ordered her to refuse them: and there was that
in the gentleman's manner which enforced obedience.  Little Becky, to
do her justice, was charmed with Rawdon's gallantry.  If he was surly,
she never was. Whether friends were present or absent, she had always a
kind smile for him and was attentive to his pleasure and comfort.  It
was the early days of their marriage over again:  the same good humour,
prevenances, merriment, and artless confidence and regard.  ``How much
pleasanter it is,'' she would say, ``to have you by my side in the
carriage than that foolish old Briggs!  Let us always go on so, dear
Rawdon.  How nice it would be, and how happy we should always be, if we
had but the money!'' He fell asleep after dinner in his chair; he did
not see the face opposite to him, haggard, weary, and terrible; it
lighted up with fresh candid smiles when he woke.  It kissed him gaily.
He wondered that he had ever had suspicions.  No, he never had
suspicions; all those dumb doubts and surly misgivings which had been
gathering on his mind were mere idle jealousies.  She was fond of him;
she always had been.  As for her shining in society, it was no fault of
hers; she was formed to shine there. Was there any woman who could
talk, or sing, or do anything like her? If she would but like the boy!
Rawdon thought.  But the mother and son never could be brought together.

And it was while Rawdon's mind was agitated with these doubts and
perplexities that the incident occurred which was mentioned in the last
chapter, and the unfortunate Colonel found himself a prisoner away from
home.



\chapter{A Rescue and a Catastrophe}

Friend Rawdon drove on then to Mr.\ Moss's mansion in Cursitor Street,
and was duly inducted into that dismal place of hospitality. Morning
was breaking over the cheerful house-tops of Chancery Lane as the
rattling cab woke up the echoes there.  A little pink-eyed Jew-boy,
with a head as ruddy as the rising morn, let the party into the house,
and Rawdon was welcomed to the ground-floor apartments by Mr.\ Moss, his
travelling companion and host, who cheerfully asked him if he would
like a glass of something warm after his drive.

The Colonel was not so depressed as some mortals would be, who,
quitting a palace and a placens uxor, find themselves barred into a
spunging-house; for, if the truth must be told, he had been a lodger at
Mr.\ Moss's establishment once or twice before.  We have not thought it
necessary in the previous course of this narrative to mention these
trivial little domestic incidents:  but the reader may be assured that
they can't unfrequently occur in the life of a man who lives on nothing
a year.

Upon his first visit to Mr.\ Moss, the Colonel, then a bachelor, had
been liberated by the generosity of his aunt; on the second mishap,
little Becky, with the greatest spirit and kindness, had borrowed a sum
of money from Lord Southdown and had coaxed her husband's creditor (who
was her shawl, velvet-gown, lace pocket-handkerchief, trinket, and
gim-crack purveyor, indeed) to take a portion of the sum claimed and
Rawdon's promissory note for the remainder:  so on both these occasions
the capture and release had been conducted with the utmost gallantry on
all sides, and Moss and the Colonel were therefore on the very best of
terms.

``You'll find your old bed, Colonel, and everything comfortable,'' that
gentleman said, ``as I may honestly say. You may be pretty sure its kep
aired, and by the best of company, too.  It was slep in the night afore
last by the Honorable Capting Famish, of the Fiftieth Dragoons, whose
Mar took him out, after a fortnight, jest to punish him, she said.
But, Law bless you, I promise you, he punished my champagne, and had a
party ere every night---reglar tip-top swells, down from the clubs and
the West End---Capting Ragg, the Honorable Deuceace, who lives in the
Temple, and some fellers as knows a good glass of wine, I warrant you.
I've got a Doctor of Diwinity upstairs, five gents in the coffee-room,
and Mrs.\ Moss has a tably-dy-hoty at half-past five, and a little
cards or music afterwards, when we shall be most happy to see you.''

``I'll ring when I want anything,'' said Rawdon and went quietly to his
bedroom.  He was an old soldier, we have said, and not to be disturbed
by any little shocks of fate.  A weaker man would have sent off a
letter to his wife on the instant of his capture.  ``But what is the use
of disturbing her night's rest?'' thought Rawdon. ``She won't know
whether I am in my room or not.  It will be time enough to write to her
when she has had her sleep out, and I have had mine.  It's only a
hundred-and-seventy, and the deuce is in it if we can't raise that.''
And so, thinking about little Rawdon (whom he would not have know that
he was in such a queer place), the Colonel turned into the bed lately
occupied by Captain Famish and fell asleep.  It was ten o'clock when he
woke up, and the ruddy-headed youth brought him, with conscious pride,
a fine silver dressing-case, wherewith he might perform the operation
of shaving. Indeed Mr.\ Moss's house, though somewhat dirty, was
splendid throughout.  There were dirty trays, and wine-coolers en
permanence on the sideboard, huge dirty gilt cornices, with dingy
yellow satin hangings to the barred windows which looked into Cursitor
Street---vast and dirty gilt picture frames surrounding pieces sporting
and sacred, all of which works were by the greatest masters---and
fetched the greatest prices, too, in the bill transactions, in the
course of which they were sold and bought over and over again.  The
Colonel's breakfast was served to him in the same dingy and gorgeous
plated ware.  Miss Moss, a dark-eyed maid in curl-papers, appeared with
the teapot, and, smiling, asked the Colonel how he had slep? And she
brought him in the Morning Post, with the names of all the great people
who had figured at Lord Steyne's entertainment the night before.  It
contained a brilliant account of the festivities and of the beautiful
and accomplished Mrs.\ Rawdon Crawley's admirable personifications.

After a lively chat with this lady (who sat on the edge of the
breakfast table in an easy attitude displaying the drapery of her
stocking and an ex-white satin shoe, which was down at heel), Colonel
Crawley called for pens and ink, and paper, and being asked how many
sheets, chose one which was brought to him between Miss Moss's own
finger and thumb.  Many a sheet had that dark-eyed damsel brought in;
many a poor fellow had scrawled and blotted hurried lines of entreaty
and paced up and down that awful room until his messenger brought back
the reply.  Poor men always use messengers instead of the post.  Who
has not had their letters, with the wafers wet, and the announcement
that a person is waiting in the hall?

Now on the score of his application, Rawdon had not many misgivings.

DEAR BECKY, (Rawdon wrote)

I HOPE YOU SLEPT WELL.  Don't be FRIGHTENED if I don't bring you in
your COFFY.  Last night as I was coming home smoaking, I met with an
ACCADENT.  I was NABBED by Moss of Cursitor Street---from whose GILT AND
SPLENDID PARLER I write this---the same that had me this time two years.
Miss Moss brought in my tea---she is grown very FAT, and, as usual, had
her STOCKENS DOWN AT HEAL.

It's Nathan's business---a hundred-and-fifty---with costs,
hundred-and-seventy.  Please send me my desk and some CLOTHS---I'm in
pumps and a white tye (something like Miss M's stockings)---I've seventy
in it.  And as soon as you get this, Drive to Nathan's---offer him
seventy-five down, and ASK HIM TO RENEW---say I'll take wine---we may as
well have some dinner sherry; but not PICTURS, they're too dear.

If he won't stand it.  Take my ticker and such of your things as you
can SPARE, and send them to Balls---we must, of coarse, have the sum
to-night.  It won't do to let it stand over, as to-morrow's Sunday; the
beds here are not very CLEAN, and there may be other things out against
me---I'm glad it an't Rawdon's Saturday for coming home.  God bless you.

Yours in haste, R. C. P.S.  Make haste and come.

This letter, sealed with a wafer, was dispatched by one of the
messengers who are always hanging about Mr.\ Moss's establishment, and
Rawdon, having seen him depart, went out in the court-yard and smoked
his cigar with a tolerably easy mind---in spite of the bars
overhead---for Mr.\ Moss's court-yard is railed in like a cage, lest the
gentlemen who are boarding with him should take a fancy to escape from
his hospitality.

Three hours, he calculated, would be the utmost time required, before
Becky should arrive and open his prison doors, and he passed these
pretty cheerfully in smoking, in reading the paper, and in the
coffee-room with an acquaintance, Captain Walker, who happened to be
there, and with whom he cut for sixpences for some hours, with pretty
equal luck on either side.

But the day passed away and no messenger returned---no Becky.  Mr.\ %
Moss's tably-dy-hoty was served at the appointed hour of half-past
five, when such of the gentlemen lodging in the house as could afford
to pay for the banquet came and partook of it in the splendid front
parlour before described, and with which Mr.\ Crawley's temporary
lodging communicated, when Miss M.  (Miss Hem, as her papa called her)
appeared without the curl-papers of the morning, and Mrs.\ Hem did the
honours of a prime boiled leg of mutton and turnips, of which the
Colonel ate with a very faint appetite.  Asked whether he would ``stand''
a bottle of champagne for the company, he consented, and the ladies
drank to his 'ealth, and Mr.\ Moss, in the most polite manner, ``looked
towards him.''

In the midst of this repast, however, the doorbell was heard---young
Moss of the ruddy hair rose up with the keys and answered the summons,
and coming back, told the Colonel that the messenger had returned with
a bag, a desk and a letter, which he gave him.  ``No ceramony, Colonel,
I beg,'' said Mrs.\ Moss with a wave of her hand, and he opened the
letter rather tremulously.  It was a beautiful letter, highly scented,
on a pink paper, and with a light green seal.

MON PAUVRE CHER PETIT, (Mrs.\ Crawley wrote)

I could not sleep ONE WINK for thinking of what had become of my odious
old monstre, and only got to rest in the morning after sending for Mr.\ %
Blench (for I was in a fever), who gave me a composing draught and left
orders with Finette that I should be disturbed ON NO ACCOUNT.  So that
my poor old man's messenger, who had bien mauvaise mine Finette says,
and sentoit le Genievre, remained in the hall for some hours waiting my
bell. You may fancy my state when I read your poor dear old ill-spelt
letter.

Ill as I was, I instantly called for the carriage, and as soon as I was
dressed (though I couldn't drink a drop of chocolate---I assure you I
couldn't without my monstre to bring it to me), I drove ventre a terre
to Nathan's.  I saw him---I wept---I cried---I fell at his odious knees.
Nothing would mollify the horrid man. He would have all the money, he
said, or keep my poor monstre in prison.  I drove home with the
intention of paying that triste visite chez mon oncle (when every
trinket I have should be at your disposal though they would not fetch a
hundred pounds, for some, you know, are with ce cher oncle already),
and found Milor there with the Bulgarian old sheep-faced monster, who
had come to compliment me upon last night's performances. Paddington
came in, too, drawling and lisping and twiddling his hair; so did
Champignac, and his chef---everybody with foison of compliments and
pretty speeches---plaguing poor me, who longed to be rid of them, and
was thinking every moment of the time of mon pauvre prisonnier.

When they were gone, I went down on my knees to Milor; told him we were
going to pawn everything, and begged and prayed him to give me two
hundred pounds. He pish'd and psha'd in a fury---told me not to be such
a fool as to pawn---and said he would see whether he could lend me the
money.  At last he went away, promising that he would send it me in the
morning:  when I will bring it to my poor old monster with a kiss from
his affectionate

BECKY

I am writing in bed.  Oh I have such a headache and such a heartache!

When Rawdon read over this letter, he turned so red and looked so
savage that the company at the table d'hote easily perceived that bad
news had reached him.  All his suspicions, which he had been trying to
banish, returned upon him.  She could not even go out and sell her
trinkets to free him.  She could laugh and talk about compliments paid
to her, whilst he was in prison.  Who had put him there? Wenham had
walked with him.  Was there....  He could hardly bear to think of what
he suspected.  Leaving the room hurriedly, he ran into his own---opened
his desk, wrote two hurried lines, which he directed to Sir Pitt or
Lady Crawley, and bade the messenger carry them at once to Gaunt
Street, bidding him to take a cab, and promising him a guinea if he was
back in an hour.

In the note he besought his dear brother and sister, for the sake of
God, for the sake of his dear child and his honour, to come to him and
relieve him from his difficulty.  He was in prison, he wanted a hundred
pounds to set him free---he entreated them to come to him.

He went back to the dining-room after dispatching his messenger and
called for more wine.  He laughed and talked with a strange
boisterousness, as the people thought.  Sometimes he laughed madly at
his own fears and went on drinking for an hour, listening all the while
for the carriage which was to bring his fate back.

At the expiration of that time, wheels were heard whirling up to the
gate---the young janitor went out with his gate-keys.  It was a lady
whom he let in at the bailiff's door.

``Colonel Crawley,'' she said, trembling very much.  He, with a knowing
look, locked the outer door upon her---then unlocked and opened the
inner one, and calling out, ``Colonel, you're wanted,'' led her into the
back parlour, which he occupied.

Rawdon came in from the dining-parlour where all those people were
carousing, into his back room; a flare of coarse light following him
into the apartment where the lady stood, still very nervous.

``It is I, Rawdon,'' she said in a timid voice, which she strove to
render cheerful.  ``It is Jane.'' Rawdon was quite overcome by that kind
voice and presence.  He ran up to her---caught her in his arms---gasped
out some inarticulate words of thanks and fairly sobbed on her
shoulder.  She did not know the cause of his emotion.

The bills of Mr.\ Moss were quickly settled, perhaps to the
disappointment of that gentleman, who had counted on having the Colonel
as his guest over Sunday at least; and Jane, with beaming smiles and
happiness in her eyes, carried away Rawdon from the bailiff's house,
and they went homewards in the cab in which she had hastened to his
release.  ``Pitt was gone to a parliamentary dinner,'' she said, ``when
Rawdon's note came, and so, dear Rawdon, I---I came myself''; and she put
her kind hand in his.  Perhaps it was well for Rawdon Crawley that Pitt
was away at that dinner.  Rawdon thanked his sister a hundred times,
and with an ardour of gratitude which touched and almost alarmed that
soft-hearted woman. ``Oh,'' said he, in his rude, artless way, ``you---you
don't know how I'm changed since I've known you, and---and little Rawdy.
I---I'd like to change somehow.  You see I want---I want---to be---'' He did
not finish the sentence, but she could interpret it.  And that night
after he left her, and as she sat by her own little boy's bed, she
prayed humbly for that poor way-worn sinner.

Rawdon left her and walked home rapidly.  It was nine o'clock at night.
He ran across the streets and the great squares of Vanity Fair, and at
length came up breathless opposite his own house.  He started back and
fell against the railings, trembling as he looked up.  The drawing-room
windows were blazing with light.  She had said that she was in bed and
ill.  He stood there for some time, the light from the rooms on his
pale face.

He took out his door-key and let himself into the house.  He could hear
laughter in the upper rooms.  He was in the ball-dress in which he had
been captured the night before.  He went silently up the stairs,
leaning against the banisters at the stair-head.  Nobody was stirring
in the house besides---all the servants had been sent away. Rawdon heard
laughter within---laughter and singing. Becky was singing a snatch of
the song of the night before; a hoarse voice shouted ``Brava!
Brava!''---it was Lord Steyne's.

Rawdon opened the door and went in.  A little table with a dinner was
laid out---and wine and plate.  Steyne was hanging over the sofa on
which Becky sat.  The wretched woman was in a brilliant full toilette,
her arms and all her fingers sparkling with bracelets and rings, and
the brilliants on her breast which Steyne had given her. He had her
hand in his, and was bowing over it to kiss it, when Becky started up
with a faint scream as she caught sight of Rawdon's white face.  At the
next instant she tried a smile, a horrid smile, as if to welcome her
husband; and Steyne rose up, grinding his teeth, pale, and with fury in
his looks.

He, too, attempted a laugh---and came forward holding out his hand.
``What, come back!  How d'ye do, Crawley?'' he said, the nerves of his
mouth twitching as he tried to grin at the intruder.

There was that in Rawdon's face which caused Becky to fling herself
before him.  ``I am innocent, Rawdon,'' she said; ``before God, I am
innocent.'' She clung hold of his coat, of his hands; her own were all
covered with serpents, and rings, and baubles.  ``I am innocent. Say I
am innocent,'' she said to Lord Steyne.

He thought a trap had been laid for him, and was as furious with the
wife as with the husband.  ``You innocent!  Damn you,'' he screamed out.
``You innocent!  Why every trinket you have on your body is paid for by
me. I have given you thousands of pounds, which this fellow has spent
and for which he has sold you.  Innocent, by ---------! You're as innocent as
your mother, the ballet-girl, and your husband the bully.  Don't think
to frighten me as you have done others. Make way, sir, and let me
pass''; and Lord Steyne seized up his hat, and, with flame in his eyes,
and looking his enemy fiercely in the face, marched upon him, never for
a moment doubting that the other would give way.

But Rawdon Crawley springing out, seized him by the neckcloth, until
Steyne, almost strangled, writhed and bent under his arm.  ``You lie,
you dog!'' said Rawdon. ``You lie, you coward and villain!'' And he struck
the Peer twice over the face with his open hand and flung him bleeding
to the ground.  It was all done before Rebecca could interpose.  She
stood there trembling before him.  She admired her husband, strong,
brave, and victorious.

``Come here,'' he said.  She came up at once.

``Take off those things.'' She began, trembling, pulling the jewels from
her arms, and the rings from her shaking fingers, and held them all in
a heap, quivering and looking up at him.  ``Throw them down,'' he said,
and she dropped them.  He tore the diamond ornament out of her breast
and flung it at Lord Steyne.  It cut him on his bald forehead.  Steyne
wore the scar to his dying day.

``Come upstairs,'' Rawdon said to his wife.  ``Don't kill me, Rawdon,'' she
said.  He laughed savagely.  ``I want to see if that man lies about the
money as he has about me.  Has he given you any?''

``No,'' said Rebecca, ``that is---''

``Give me your keys,'' Rawdon answered, and they went out together.

Rebecca gave him all the keys but one, and she was in hopes that he
would not have remarked the absence of that.  It belonged to the little
desk which Amelia had given her in early days, and which she kept in a
secret place.  But Rawdon flung open boxes and wardrobes, throwing the
multifarious trumpery of their contents here and there, and at last he
found the desk.  The woman was forced to open it.  It contained papers,
love-letters many years old---all sorts of small trinkets and woman's
memoranda.  And it contained a pocket-book with bank-notes. Some of
these were dated ten years back, too, and one was quite a fresh one---a
note for a thousand pounds which Lord Steyne had given her.

``Did he give you this?'' Rawdon said.

``Yes,'' Rebecca answered.

``I'll send it to him to-day,'' Rawdon said (for day had dawned again,
and many hours had passed in this search), ``and I will pay Briggs, who
was kind to the boy, and some of the debts.  You will let me know where
I shall send the rest to you.  You might have spared me a hundred
pounds, Becky, out of all this---I have always shared with you.''

``I am innocent,'' said Becky.  And he left her without another word.

What were her thoughts when he left her? She remained for hours after
he was gone, the sunshine pouring into the room, and Rebecca sitting
alone on the bed's edge.  The drawers were all opened and their
contents scattered about---dresses and feathers, scarfs and trinkets, a
heap of tumbled vanities lying in a wreck.  Her hair was falling over
her shoulders; her gown was torn where Rawdon had wrenched the
brilliants out of it.  She heard him go downstairs a few minutes after
he left her, and the door slamming and closing on him.  She knew he
would never come back.  He was gone forever. Would he kill
himself?---she thought---not until after he had met Lord Steyne.  She
thought of her long past life, and all the dismal incidents of it.  Ah,
how dreary it seemed, how miserable, lonely and profitless!  Should she
take laudanum, and end it, to have done with all hopes, schemes, debts,
and triumphs? The French maid found her in this position---sitting in
the midst of her miserable ruins with clasped hands and dry eyes.  The
woman was her accomplice and in Steyne's pay.  ``Mon Dieu, madame, what
has happened?'' she asked.

What had happened? Was she guilty or not? She said not, but who could
tell what was truth which came from those lips, or if that corrupt
heart was in this case pure?

All her lies and her schemes, and her selfishness and her wiles, all her
wit and genius had come to this bankruptcy.  The woman closed the
curtains and, with some entreaty and show of kindness, persuaded her
mistress to lie down on the bed.  Then she went below and gathered up
the trinkets which had been lying on the floor since Rebecca dropped
them there at her husband's orders, and Lord Steyne went away.



\chapter{Sunday After the Battle}

The mansion of Sir Pitt Crawley, in Great Gaunt Street, was just
beginning to dress itself for the day, as Rawdon, in his evening
costume, which he had now worn two days, passed by the scared female
who was scouring the steps and entered into his brother's study. Lady
Jane, in her morning-gown, was up and above stairs in the nursery
superintending the toilettes of her children and listening to the
morning prayers which the little creatures performed at her knee.
Every morning she and they performed this duty privately, and before
the public ceremonial at which Sir Pitt presided and at which all the
people of the household were expected to assemble. Rawdon sat down in
the study before the Baronet's table, set out with the orderly blue
books and the letters, the neatly docketed bills and symmetrical
pamphlets, the locked account-books, desks, and dispatch boxes, the
Bible, the Quarterly Review, and the Court Guide, which all stood as if
on parade awaiting the inspection of their chief.

A book of family sermons, one of which Sir Pitt was in the habit of
administering to his family on Sunday mornings, lay ready on the study
table, and awaiting his judicious selection.  And by the sermon-book
was the Observer newspaper, damp and neatly folded, and for Sir Pitt's
own private use.  His gentleman alone took the opportunity of perusing
the newspaper before he laid it by his master's desk.  Before he had
brought it into the study that morning, he had read in the journal a
flaming account of ``Festivities at Gaunt House,'' with the names of all
the distinguished personages invited by the Marquis of Steyne to meet
his Royal Highness.  Having made comments upon this entertainment to
the housekeeper and her niece as they were taking early tea and hot
buttered toast in the former lady's apartment, and wondered how the
Rawding Crawleys could git on, the valet had damped and folded the
paper once more, so that it looked quite fresh and innocent against the
arrival of the master of the house.

Poor Rawdon took up the paper and began to try and read it until his
brother should arrive.  But the print fell blank upon his eyes, and he
did not know in the least what he was reading.  The Government news and
appointments (which Sir Pitt as a public man was bound to peruse,
otherwise he would by no means permit the introduction of Sunday papers
into his household), the theatrical criticisms, the fight for a hundred
pounds a side between the Barking Butcher and the Tutbury Pet, the
Gaunt House chronicle itself, which contained a most complimentary
though guarded account of the famous charades of which Mrs.\ Becky had
been the heroine---all these passed as in a haze before Rawdon, as he
sat waiting the arrival of the chief of the family.

Punctually, as the shrill-toned bell of the black marble study clock
began to chime nine, Sir Pitt made his appearance, fresh, neat, smugly
shaved, with a waxy clean face, and stiff shirt collar, his scanty hair
combed and oiled, trimming his nails as he descended the stairs
majestically, in a starched cravat and a grey flannel dressing-gown---a
real old English gentleman, in a word---a model of neatness and every
propriety.  He started when he saw poor Rawdon in his study in tumbled
clothes, with blood-shot eyes, and his hair over his face.  He thought
his brother was not sober, and had been out all night on some orgy.
``Good gracious, Rawdon,'' he said, with a blank face, ``what brings you
here at this time of the morning? Why ain't you at home?''

``Home,'' said Rawdon with a wild laugh.  ``Don't be frightened, Pitt. I'm
not drunk.  Shut the door; I want to speak to you.''

Pitt closed the door and came up to the table, where he sat down in the
other arm-chair---that one placed for the reception of the steward,
agent, or confidential visitor who came to transact business with the
Baronet---and trimmed his nails more vehemently than ever.

``Pitt, it's all over with me,'' the Colonel said after a pause.  ``I'm
done.''

``I always said it would come to this,'' the Baronet cried peevishly, and
beating a tune with his clean-trimmed nails.  ``I warned you a thousand
times.  I can't help you any more.  Every shilling of my money is tied
up.  Even the hundred pounds that Jane took you last night were
promised to my lawyer to-morrow morning, and the want of it will put me
to great inconvenience. I don't mean to say that I won't assist you
ultimately. But as for paying your creditors in full, I might as well
hope to pay the National Debt.  It is madness, sheer madness, to think
of such a thing.  You must come to a compromise.  It's a painful thing
for the family, but everybody does it.  There was George Kitely, Lord
Ragland's son, went through the Court last week, and was what they call
whitewashed, I believe. Lord Ragland would not pay a shilling for him,
and---''

``It's not money I want,'' Rawdon broke in.  ``I'm not come to you about
myself.  Never mind what happens to me.''

``What is the matter, then?'' said Pitt, somewhat relieved.

``It's the boy,'' said Rawdon in a husky voice.  ``I want you to promise
me that you will take charge of him when I'm gone.  That dear good wife
of yours has always been good to him; and he's fonder of her than he is
of his . . .---Damn it.  Look here, Pitt---you know that I was to have
had Miss Crawley's money.  I wasn't brought up like a younger brother,
but was always encouraged to be extravagant and kep idle.  But for this
I might have been quite a different man.  I didn't do my duty with the
regiment so bad.  You know how I was thrown over about the money, and
who got it.''

``After the sacrifices I have made, and the manner in which I have stood
by you, I think this sort of reproach is useless,'' Sir Pitt said.
``Your marriage was your own doing, not mine.''

``That's over now,'' said Rawdon.  ``That's over now.'' And the words were
wrenched from him with a groan, which made his brother start.

``Good God!  is she dead?'' Sir Pitt said with a voice of genuine alarm
and commiseration.

``I wish I was,'' Rawdon replied.  ``If it wasn't for little Rawdon I'd
have cut my throat this morning---and that damned villain's too.''

Sir Pitt instantly guessed the truth and surmised that Lord Steyne was
the person whose life Rawdon wished to take.  The Colonel told his
senior briefly, and in broken accents, the circumstances of the case.
``It was a regular plan between that scoundrel and her,'' he said.  ``The
bailiffs were put upon me; I was taken as I was going out of his house;
when I wrote to her for money, she said she was ill in bed and put me
off to another day. And when I got home I found her in diamonds and
sitting with that villain alone.'' He then went on to describe hurriedly
the personal conflict with Lord Steyne.  To an affair of that nature,
of course, he said, there was but one issue, and after his conference
with his brother, he was going away to make the necessary arrangements
for the meeting which must ensue.  ``And as it may end fatally with me,''
Rawdon said with a broken voice, ``and as the boy has no mother, I must
leave him to you and Jane, Pitt---only it will be a comfort to me if you
will promise me to be his friend.''

The elder brother was much affected, and shook Rawdon's hand with a
cordiality seldom exhibited by him. Rawdon passed his hand over his
shaggy eyebrows. ``Thank you, brother,'' said he.  ``I know I can trust
your word.''

``I will, upon my honour,'' the Baronet said.  And thus, and almost
mutely, this bargain was struck between them.

Then Rawdon took out of his pocket the little pocket-book which he had
discovered in Becky's desk, and from which he drew a bundle of the
notes which it contained. ``Here's six hundred,'' he said---``you didn't
know I was so rich.  I want you to give the money to Briggs, who lent
it to us---and who was kind to the boy---and I've always felt ashamed of
having taken the poor old woman's money.  And here's some more---I've
only kept back a few pounds---which Becky may as well have, to get on
with.'' As he spoke he took hold of the other notes to give to his
brother, but his hands shook, and he was so agitated that the
pocket-book fell from him, and out of it the thousand-pound note which
had been the last of the unlucky Becky's winnings.

Pitt stooped and picked them up, amazed at so much wealth.  ``Not that,''
Rawdon said.  ``I hope to put a bullet into the man whom that belongs
to.'' He had thought to himself, it would be a fine revenge to wrap a
ball in the note and kill Steyne with it.

After this colloquy the brothers once more shook hands and parted. Lady
Jane had heard of the Colonel's arrival, and was waiting for her
husband in the adjoining dining-room, with female instinct, auguring
evil.  The door of the dining-room happened to be left open, and the
lady of course was issuing from it as the two brothers passed out of
the study.  She held out her hand to Rawdon and said she was glad he
was come to breakfast, though she could perceive, by his haggard
unshorn face and the dark looks of her husband, that there was very
little question of breakfast between them.  Rawdon muttered some
excuses about an engagement, squeezing hard the timid little hand which
his sister-in-law reached out to him.  Her imploring eyes could read
nothing but calamity in his face, but he went away without another
word.  Nor did Sir Pitt vouchsafe her any explanation. The children
came up to salute him, and he kissed them in his usual frigid manner.
The mother took both of them close to herself, and held a hand of each
of them as they knelt down to prayers, which Sir Pitt read to them, and
to the servants in their Sunday suits or liveries, ranged upon chairs
on the other side of the hissing tea-urn. Breakfast was so late that
day, in consequence of the delays which had occurred, that the
church-bells began to ring whilst they were sitting over their meal;
and Lady Jane was too ill, she said, to go to church, though her
thoughts had been entirely astray during the period of family devotion.

Rawdon Crawley meanwhile hurried on from Great Gaunt Street, and
knocking at the great bronze Medusa's head which stands on the portal
of Gaunt House, brought out the purple Silenus in a red and silver
waistcoat who acts as porter of that palace.  The man was scared also
by the Colonel's dishevelled appearance, and barred the way as if
afraid that the other was going to force it.  But Colonel Crawley only
took out a card and enjoined him particularly to send it in to Lord
Steyne, and to mark the address written on it, and say that Colonel
Crawley would be all day after one o'clock at the Regent Club in St.\ %
James's Street---not at home.  The fat red-faced man looked after him
with astonishment as he strode away; so did the people in their Sunday
clothes who were out so early; the charity-boys with shining faces,
the greengrocer lolling at his door, and the publican shutting his
shutters in the sunshine, against service commenced.  The people joked
at the cab-stand about his appearance, as he took a carriage there, and
told the driver to drive him to Knightsbridge Barracks.

All the bells were jangling and tolling as he reached that place. He
might have seen his old acquaintance Amelia on her way from Brompton to
Russell Square, had he been looking out.  Troops of schools were on
their march to church, the shiny pavement and outsides of coaches in
the suburbs were thronged with people out upon their Sunday pleasure;
but the Colonel was much too busy to take any heed of these phenomena,
and, arriving at Knightsbridge, speedily made his way up to the room of
his old friend and comrade Captain Macmurdo, who Crawley found, to his
satisfaction, was in barracks.

Captain Macmurdo, a veteran officer and Waterloo man, greatly liked by
his regiment, in which want of money alone prevented him from attaining
the highest ranks, was enjoying the forenoon calmly in bed.  He had
been at a fast supper-party, given the night before by Captain the
Honourable George Cinqbars, at his house in Brompton Square, to several
young men of the regiment, and a number of ladies of the corps de
ballet, and old Mac, who was at home with people of all ages and ranks,
and consorted with generals, dog-fanciers, opera-dancers, bruisers, and
every kind of person, in a word, was resting himself after the night's
labours, and, not being on duty, was in bed.

His room was hung round with boxing, sporting, and dancing pictures,
presented to him by comrades as they retired from the regiment, and
married and settled into quiet life.  And as he was now nearly fifty
years of age, twenty-four of which he had passed in the corps, he had a
singular museum.  He was one of the best shots in England, and, for a
heavy man, one of the best riders; indeed, he and Crawley had been
rivals when the latter was in the Army.  To be brief, Mr.\ Macmurdo was
lying in bed, reading in Bell's Life an account of that very fight
between the Tutbury Pet and the Barking Butcher, which has been before
mentioned---a venerable bristly warrior, with a little close-shaved grey
head, with a silk nightcap, a red face and nose, and a great dyed
moustache.

When Rawdon told the Captain he wanted a friend, the latter knew
perfectly well on what duty of friendship he was called to act, and
indeed had conducted scores of affairs for his acquaintances with the
greatest prudence and skill.  His Royal Highness the late lamented
Commander-in-Chief had had the greatest regard for Macmurdo on this
account, and he was the common refuge of gentlemen in trouble.

``What's the row about, Crawley, my boy?'' said the old warrior.  ``No
more gambling business, hay, like that when we shot Captain Marker?''

``It's about---about my wife,'' Crawley answered, casting down his eyes
and turning very red.

The other gave a whistle.  ``I always said she'd throw you over,'' he
began---indeed there were bets in the regiment and at the clubs
regarding the probable fate of Colonel Crawley, so lightly was his
wife's character esteemed by his comrades and the world; but seeing the
savage look with which Rawdon answered the expression of this opinion,
Macmurdo did not think fit to enlarge upon it further.

``Is there no way out of it, old boy?'' the Captain continued in a grave
tone.  ``Is it only suspicion, you know, or---or what is it? Any letters?
Can't you keep it quiet? Best not make any noise about a thing of that
sort if you can help it.'' ``Think of his only finding her out now,'' the
Captain thought to himself, and remembered a hundred particular
conversations at the mess-table, in which Mrs.\ Crawley's reputation had
been torn to shreds.

``There's no way but one out of it,'' Rawdon replied---``and there's only a
way out of it for one of us, Mac---do you understand? I was put out of
the way---arrested---I found 'em alone together.  I told him he was a
liar and a coward, and knocked him down and thrashed him.''

``Serve him right,'' Macmurdo said.  ``Who is it?''

Rawdon answered it was Lord Steyne.

``The deuce!  a Marquis!  they said he---that is, they said you---''

``What the devil do you mean?'' roared out Rawdon; ``do you mean that you
ever heard a fellow doubt about my wife and didn't tell me, Mac?''

``The world's very censorious, old boy,'' the other replied.  ``What the
deuce was the good of my telling you what any tom-fools talked about?''

``It was damned unfriendly, Mac,'' said Rawdon, quite overcome; and,
covering his face with his hands, he gave way to an emotion, the sight
of which caused the tough old campaigner opposite him to wince with
sympathy. ``Hold up, old boy,'' he said; ``great man or not, we'll put a
bullet in him, damn him.  As for women, they're all so.''

``You don't know how fond I was of that one,'' Rawdon said,
half-inarticulately.  ``Damme, I followed her like a footman.  I gave up
everything I had to her.  I'm a beggar because I would marry her. By
Jove, sir, I've pawned my own watch in order to get her anything she
fancied; and she---she's been making a purse for herself all the time,
and grudged me a hundred pound to get me out of quod.'' He then fiercely
and incoherently, and with an agitation under which his counsellor had
never before seen him labour, told Macmurdo the circumstances of the
story.  His adviser caught at some stray hints in it. ``She may be
innocent, after all,'' he said.  ``She says so. Steyne has been a hundred
times alone with her in the house before.''

``It may be so,'' Rawdon answered sadly, ``but this don't look very
innocent'':  and he showed the Captain the thousand-pound note which he
had found in Becky's pocket-book.  ``This is what he gave her, Mac, and
she kep it unknown to me; and with this money in the house, she refused
to stand by me when I was locked up.'' The Captain could not but own
that the secreting of the money had a very ugly look.

Whilst they were engaged in their conference, Rawdon dispatched Captain
Macmurdo's servant to Curzon Street, with an order to the domestic
there to give up a bag of clothes of which the Colonel had great need.
And during the man's absence, and with great labour and a Johnson's
Dictionary, which stood them in much stead, Rawdon and his second
composed a letter, which the latter was to send to Lord Steyne.
Captain Macmurdo had the honour of waiting upon the Marquis of Steyne,
on the part of Colonel Rawdon Crawley, and begged to intimate that he
was empowered by the Colonel to make any arrangements for the meeting
which, he had no doubt, it was his Lordship's intention to demand, and
which the circumstances of the morning had rendered inevitable.
Captain Macmurdo begged Lord Steyne, in the most polite manner, to
appoint a friend, with whom he (Captain M.M.) might communicate, and
desired that the meeting might take place with as little delay as
possible.

In a postscript the Captain stated that he had in his possession a
bank-note for a large amount, which Colonel Crawley had reason to
suppose was the property of the Marquis of Steyne.  And he was anxious,
on the Colonel's behalf, to give up the note to its owner.

By the time this note was composed, the Captain's servant returned from
his mission to Colonel Crawley's house in Curzon Street, but without
the carpet-bag and portmanteau, for which he had been sent, and with a
very puzzled and odd face.

``They won't give 'em up,'' said the man; ``there's a regular shinty in
the house, and everything at sixes and sevens.  The landlord's come in
and took possession.  The servants was a drinkin' up in the
drawingroom.  They said---they said you had gone off with the plate,
Colonel''---the man added after a pause---``One of the servants is off
already.  And Simpson, the man as was very noisy and drunk indeed, says
nothing shall go out of the house until his wages is paid up.''

The account of this little revolution in May Fair astonished and gave a
little gaiety to an otherwise very triste conversation.  The two
officers laughed at Rawdon's discomfiture.

``I'm glad the little 'un isn't at home,'' Rawdon said, biting his nails.
``You remember him, Mac, don't you, in the Riding School? How he sat the
kicker to be sure! didn't he?''

``That he did, old boy,'' said the good-natured Captain.

Little Rawdon was then sitting, one of fifty gown boys, in the Chapel
of Whitefriars School, thinking, not about the sermon, but about going
home next Saturday, when his father would certainly tip him and perhaps
would take him to the play.

``He's a regular trump, that boy,'' the father went on, still musing
about his son.  ``I say, Mac, if anything goes wrong---if I drop---I
should like you to---to go and see him, you know, and say that I was
very fond of him, and that.  And---dash it---old chap, give him these
gold sleeve-buttons:  it's all I've got.'' He covered his face with his
black hands, over which the tears rolled and made furrows of white.
Mr.\ Macmurdo had also occasion to take off his silk night-cap and rub
it across his eyes.

``Go down and order some breakfast,'' he said to his man in a loud
cheerful voice.  ``What'll you have, Crawley? Some devilled kidneys and
a herring---let's say.  And, Clay, lay out some dressing things for the
Colonel:  we were always pretty much of a size, Rawdon, my boy, and
neither of us ride so light as we did when we first entered the corps.''
With which, and leaving the Colonel to dress himself, Macmurdo turned
round towards the wall, and resumed the perusal of Bell's Life, until
such time as his friend's toilette was complete and he was at liberty
to commence his own.

This, as he was about to meet a lord, Captain Macmurdo performed with
particular care.  He waxed his mustachios into a state of brilliant
polish and put on a tight cravat and a trim buff waistcoat, so that all
the young officers in the mess-room, whither Crawley had preceded his
friend, complimented Mac on his appearance at breakfast and asked if he
was going to be married that Sunday.



\chapter{In Which the Same Subject is Pursued}

Becky did not rally from the state of stupor and confusion in which the
events of the previous night had plunged her intrepid spirit until the
bells of the Curzon Street Chapels were ringing for afternoon service,
and rising from her bed she began to ply her own bell, in order to
summon the French maid who had left her some hours before.

Mrs.\ Rawdon Crawley rang many times in vain; and though, on the last
occasion, she rang with such vehemence as to pull down the bell-rope,
Mademoiselle Fifine did not make her appearance---no, not though her
mistress, in a great pet, and with the bell-rope in her hand, came out
to the landing-place with her hair over her shoulders and screamed out
repeatedly for her attendant.

The truth is, she had quitted the premises for many hours, and upon
that permission which is called French leave among us.  After picking up
the trinkets in the drawing-room, Mademoiselle had ascended to her own
apartments, packed and corded her own boxes there, tripped out and
called a cab for herself, brought down her trunks with her own hand,
and without ever so much as asking the aid of any of the other
servants, who would probably have refused it, as they hated her
cordially, and without wishing any one of them good-bye, had made her
exit from Curzon Street.

The game, in her opinion, was over in that little domestic
establishment.  Fifine went off in a cab, as we have known more exalted
persons of her nation to do under similar circumstances: but, more
provident or lucky than these, she secured not only her own property,
but some of her mistress's (if indeed that lady could be said to have
any property at all)---and not only carried off the trinkets before
alluded to, and some favourite dresses on which she had long kept her
eye, but four richly gilt Louis Quatorze candlesticks, six gilt albums,
keepsakes, and Books of Beauty, a gold enamelled snuff-box which had
once belonged to Madame du Barri, and the sweetest little inkstand and
mother-of-pearl blotting book, which Becky used when she composed her
charming little pink notes, had vanished from the premises in Curzon
Street together with Mademoiselle Fifine, and all the silver laid on
the table for the little festin which Rawdon interrupted.  The plated
ware Mademoiselle left behind her was too cumbrous, probably for which
reason, no doubt, she also left the fire irons, the chimney-glasses,
and the rosewood cottage piano.

A lady very like her subsequently kept a milliner's shop in the Rue du
Helder at Paris, where she lived with great credit and enjoyed the
patronage of my Lord Steyne.  This person always spoke of England as of
the most treacherous country in the world, and stated to her young
pupils that she had been affreusement vole by natives of that island.
It was no doubt compassion for her misfortunes which induced the
Marquis of Steyne to be so very kind to Madame de Saint-Amaranthe.  May
she flourish as she deserves---she appears no more in our quarter of
Vanity Fair.

Hearing a buzz and a stir below, and indignant at the impudence of
those servants who would not answer her summons, Mrs.\ Crawley flung her
morning robe round her and descended majestically to the drawing-room,
whence the noise proceeded.

The cook was there with blackened face, seated on the beautiful chintz
sofa by the side of Mrs.\ Raggles, to whom she was administering
Maraschino.  The page with the sugar-loaf buttons, who carried about
Becky's pink notes, and jumped about her little carriage with such
alacrity, was now engaged putting his fingers into a cream dish; the
footman was talking to Raggles, who had a face full of perplexity and
woe---and yet, though the door was open, and Becky had been screaming a
half-dozen of times a few feet off, not one of her attendants had
obeyed her call.  ``Have a little drop, do'ee now, Mrs.\ Raggles,'' the
cook was saying as Becky entered, the white cashmere dressing-gown
flouncing around her.

``Simpson!  Trotter!'' the mistress of the house cried in great wrath.
``How dare you stay here when you heard me call? How dare you sit down
in my presence? Where's my maid?'' The page withdrew his fingers from
his mouth with a momentary terror, but the cook took off a glass of
Maraschino, of which Mrs.\ Raggles had had enough, staring at Becky over
the little gilt glass as she drained its contents. The liquor appeared
to give the odious rebel courage.

``YOUR sofy, indeed!'' Mrs.\ Cook said.  ``I'm a settin' on Mrs.\ Raggles's
sofy.  Don't you stir, Mrs.\ Raggles, Mum. I'm a settin' on Mr.\ and Mrs.\ %
Raggles's sofy, which they bought with honest money, and very dear it
cost 'em, too.  And I'm thinkin' if I set here until I'm paid my wages,
I shall set a precious long time, Mrs.\ Raggles; and set I will,
too---ha!  ha!'' and with this she filled herself another glass of the
liquor and drank it with a more hideously satirical air.

``Trotter!  Simpson!  turn that drunken wretch out,'' screamed Mrs.\ %
Crawley.

``I shawn't,'' said Trotter the footman; ``turn out yourself.  Pay our
selleries, and turn me out too.  WE'LL go fast enough.''

``Are you all here to insult me?'' cried Becky in a fury; ``when Colonel
Crawley comes home I'll---''

At this the servants burst into a horse haw-haw, in which, however,
Raggles, who still kept a most melancholy countenance, did not join.
``He ain't a coming back,'' Mr.\ Trotter resumed.  ``He sent for his
things, and I wouldn't let 'em go, although Mr.\ Raggles would; and I
don't b'lieve he's no more a Colonel than I am.  He's hoff, and I
suppose you're a goin' after him.  You're no better than swindlers,
both on you.  Don't be a bullyin' ME.  I won't stand it.  Pay us our
selleries, I say.  Pay us our selleries.'' It was evident, from Mr.\ %
Trotter's flushed countenance and defective intonation, that he, too,
had had recourse to vinous stimulus.

``Mr.\ Raggles,'' said Becky in a passion of vexation, ``you will not
surely let me be insulted by that drunken man?'' ``Hold your noise,
Trotter; do now,'' said Simpson the page.  He was affected by his
mistress's deplorable situation, and succeeded in preventing an
outrageous denial of the epithet ``drunken'' on the footman's part.

``Oh, M'am,'' said Raggles, ``I never thought to live to see this year
day:  I've known the Crawley family ever since I was born.  I lived
butler with Miss Crawley for thirty years; and I little thought one of
that family was a goin' to ruing me---yes, ruing me''---said the poor
fellow with tears in his eyes.  ``Har you a goin' to pay me? You've
lived in this 'ouse four year.  You've 'ad my substance: my plate and
linning.  You ho me a milk and butter bill of two 'undred pound, you
must 'ave noo laid heggs for your homlets, and cream for your spanil
dog.''

``She didn't care what her own flesh and blood had,'' interposed the
cook.  ``Many's the time, he'd have starved but for me.''

``He's a charaty-boy now, Cooky,'' said Mr.\ Trotter, with a drunken ``ha!
ha!''---and honest Raggles continued, in a lamentable tone, an
enumeration of his griefs.  All he said was true.  Becky and her
husband had ruined him. He had bills coming due next week and no means
to meet them.  He would be sold up and turned out of his shop and his
house, because he had trusted to the Crawley family.  His tears and
lamentations made Becky more peevish than ever.

``You all seem to be against me,'' she said bitterly. ``What do you want?
I can't pay you on Sunday.  Come back to-morrow and I'll pay you
everything.  I thought Colonel Crawley had settled with you.  He will
to-morrow. I declare to you upon my honour that he left home this
morning with fifteen hundred pounds in his pocket-book. He has left me
nothing.  Apply to him.  Give me a bonnet and shawl and let me go out
and find him.  There was a difference between us this morning.  You all
seem to know it.  I promise you upon my word that you shall all be
paid.  He has got a good appointment.  Let me go out and find him.''

This audacious statement caused Raggles and the other personages
present to look at one another with a wild surprise, and with it
Rebecca left them.  She went upstairs and dressed herself this time
without the aid of her French maid.  She went into Rawdon's room, and
there saw that a trunk and bag were packed ready for removal, with a
pencil direction that they should be given when called for; then she
went into the Frenchwoman's garret; everything was clean, and all the
drawers emptied there. She bethought herself of the trinkets which had
been left on the ground and felt certain that the woman had fled. ``Good
Heavens!  was ever such ill luck as mine?'' she said; ``to be so near,
and to lose all.  Is it all too late?'' No; there was one chance more.

She dressed herself and went away unmolested this time, but alone. It
was four o'clock.  She went swiftly down the streets (she had no money
to pay for a carriage), and never stopped until she came to Sir Pitt
Crawley's door, in Great Gaunt Street.  Where was Lady Jane Crawley?
She was at church.  Becky was not sorry. Sir Pitt was in his study, and
had given orders not to be disturbed---she must see him---she slipped by
the sentinel in livery at once, and was in Sir Pitt's room before the
astonished Baronet had even laid down the paper.

He turned red and started back from her with a look of great alarm and
horror.

``Do not look so,'' she said.  ``I am not guilty, Pitt, dear Pitt; you
were my friend once.  Before God, I am not guilty.  I seem so.
Everything is against me.  And oh!  at such a moment!  just when all my
hopes were about to be realized:  just when happiness was in store for
us.''

``Is this true, what I see in the paper then?'' Sir Pitt said---a
paragraph in which had greatly surprised him.

``It is true.  Lord Steyne told me on Friday night, the night of that
fatal ball.  He has been promised an appointment any time these six
months.  Mr.\ Martyr, the Colonial Secretary, told him yesterday that it
was made out. That unlucky arrest ensued; that horrible meeting. I was
only guilty of too much devotedness to Rawdon's service.  I have
received Lord Steyne alone a hundred times before. I confess I had
money of which Rawdon knew nothing. Don't you know how careless he is
of it, and could I dare to confide it to him?'' And so she went on with
a perfectly connected story, which she poured into the ears of her
perplexed kinsman.

It was to the following effect.  Becky owned, and with perfect
frankness, but deep contrition, that having remarked Lord Steyne's
partiality for her (at the mention of which Pitt blushed), and being
secure of her own virtue, she had determined to turn the great peer's
attachment to the advantage of herself and her family.  ``I looked for a
peerage for you, Pitt,'' she said (the brother-in-law again turned red).
``We have talked about it.  Your genius and Lord Steyne's interest made
it more than probable, had not this dreadful calamity come to put an
end to all our hopes.  But, first, I own that it was my object to
rescue my dear husband---him whom I love in spite of all his ill usage
and suspicions of me---to remove him from the poverty and ruin which was
impending over us.  I saw Lord Steyne's partiality for me,'' she said,
casting down her eyes.  ``I own that I did everything in my power to
make myself pleasing to him, and as far as an honest woman may, to
secure his---his esteem. It was only on Friday morning that the news
arrived of the death of the Governor of Coventry Island, and my Lord
instantly secured the appointment for my dear husband. It was intended
as a surprise for him---he was to see it in the papers to-day.  Even
after that horrid arrest took place (the expenses of which Lord Steyne
generously said he would settle, so that I was in a manner prevented
from coming to my husband's assistance), my Lord was laughing with me,
and saying that my dearest Rawdon would be consoled when he read of his
appointment in the paper, in that shocking spun---bailiff's house. And
then---then he came home.  His suspicions were excited,---the dreadful
scene took place between my Lord and my cruel, cruel Rawdon---and, O my
God, what will happen next? Pitt, dear Pitt!  pity me, and reconcile
us!'' And as she spoke she flung herself down on her knees, and bursting
into tears, seized hold of Pitt's hand, which she kissed passionately.

It was in this very attitude that Lady Jane, who, returning from
church, ran to her husband's room directly she heard Mrs.\ Rawdon
Crawley was closeted there, found the Baronet and his sister-in-law.

``I am surprised that woman has the audacity to enter this house,'' Lady
Jane said, trembling in every limb and turning quite pale. (Her
Ladyship had sent out her maid directly after breakfast, who had
communicated with Raggles and Rawdon Crawley's household, who had told
her all, and a great deal more than they knew, of that story, and many
others besides).  ``How dare Mrs.\ Crawley to enter the house of---of an
honest family?''

Sir Pitt started back, amazed at his wife's display of vigour. Becky
still kept her kneeling posture and clung to Sir Pitt's hand.

``Tell her that she does not know all:  Tell her that I am innocent,
dear Pitt,'' she whimpered out.

``Upon my word, my love, I think you do Mrs.\ Crawley injustice,'' Sir
Pitt said; at which speech Rebecca was vastly relieved.  ``Indeed I
believe her to be---''

``To be what?'' cried out Lady Jane, her clear voice thrilling and, her
heart beating violently as she spoke. ``To be a wicked woman---a
heartless mother, a false wife? She never loved her dear little boy,
who used to fly here and tell me of her cruelty to him.  She never came
into a family but she strove to bring misery with her and to weaken the
most sacred affections with her wicked flattery and falsehoods.  She
has deceived her husband, as she has deceived everybody; her soul is
black with vanity, worldliness, and all sorts of crime.  I tremble when
I touch her.  I keep my children out of her sight.''

``Lady Jane!'' cried Sir Pitt, starting up, ``this is really language---''
``I have been a true and faithful wife to you, Sir Pitt,'' Lady Jane
continued, intrepidly; ``I have kept my marriage vow as I made it to God
and have been obedient and gentle as a wife should.  But righteous
obedience has its limits, and I declare that I will not bear that---that
woman again under my roof; if she enters it, I and my children will
leave it.  She is not worthy to sit down with Christian people.
You---you must choose, sir, between her and me''; and with this my Lady
swept out of the room, fluttering with her own audacity, and leaving
Rebecca and Sir Pitt not a little astonished at it.

As for Becky, she was not hurt; nay, she was pleased. ``It was the
diamond-clasp you gave me,'' she said to Sir Pitt, reaching him out her
hand; and before she left him (for which event you may be sure my Lady
Jane was looking out from her dressing-room window in the upper story)
the Baronet had promised to go and seek out his brother, and endeavour
to bring about a reconciliation.

Rawdon found some of the young fellows of the regiment seated in the
mess-room at breakfast, and was induced without much difficulty to
partake of that meal, and of the devilled legs of fowls and soda-water
with which these young gentlemen fortified themselves.  Then they had a
conversation befitting the day and their time of life: about the next
pigeon-match at Battersea, with relative bets upon Ross and
Osbaldiston; about Mademoiselle Ariane of the French Opera, and who had
left her, and how she was consoled by Panther Carr; and about the fight
between the Butcher and the Pet, and the probabilities that it was a
cross.  Young Tandyman, a hero of seventeen, laboriously endeavouring
to get up a pair of mustachios, had seen the fight, and spoke in the
most scientific manner about the battle and the condition of the men.
It was he who had driven the Butcher on to the ground in his drag and
passed the whole of the previous night with him.  Had there not been
foul play he must have won it.  All the old files of the Ring were in
it; and Tandyman wouldn't pay; no, dammy, he wouldn't pay.  It was but
a year since the young Cornet, now so knowing a hand in Cribb's
parlour, had a still lingering liking for toffy, and used to be birched
at Eton.

So they went on talking about dancers, fights, drinking, demireps,
until Macmurdo came down and joined the boys and the conversation. He
did not appear to think that any especial reverence was due to their
boyhood; the old fellow cut in with stories, to the full as choice as
any the youngest rake present had to tell---nor did his own grey hairs
nor their smooth faces detain him.  Old Mac was famous for his good
stories.  He was not exactly a lady's man; that is, men asked him to
dine rather at the houses of their mistresses than of their mothers.
There can scarcely be a life lower, perhaps, than his, but he was quite
contented with it, such as it was, and led it in perfect good nature,
simplicity, and modesty of demeanour.

By the time Mac had finished a copious breakfast, most of the others
had concluded their meal.  Young Lord Varinas was smoking an immense
Meerschaum pipe, while Captain Hugues was employed with a cigar: that
violent little devil Tandyman, with his little bull-terrier between his
legs, was tossing for shillings with all his might (that fellow was
always at some game or other) against Captain Deuceace; and Mac and
Rawdon walked off to the Club, neither, of course, having given any
hint of the business which was occupying their minds.  Both, on the
other hand, had joined pretty gaily in the conversation, for why should
they interrupt it? Feasting, drinking, ribaldry, laughter, go on
alongside of all sorts of other occupations in Vanity Fair---the crowds
were pouring out of church as Rawdon and his friend passed down St.\ %
James's Street and entered into their Club.

The old bucks and habitues, who ordinarily stand gaping and grinning
out of the great front window of the Club, had not arrived at their
posts as yet---the newspaper-room was almost empty.  One man was present
whom Rawdon did not know; another to whom he owed a little score for
whist, and whom, in consequence, he did not care to meet; a third was
reading the Royalist (a periodical famous for its scandal and its
attachment to Church and King) Sunday paper at the table, and looking
up at Crawley with some interest, said, ``Crawley, I congratulate you.''

``What do you mean?'' said the Colonel.

``It's in the Observer and the Royalist too,'' said Mr.\ Smith.

``What?'' Rawdon cried, turning very red.  He thought that the affair
with Lord Steyne was already in the public prints.  Smith looked up
wondering and smiling at the agitation which the Colonel exhibited as
he took up the paper and, trembling, began to read.

Mr.\ Smith and Mr.\ Brown (the gentleman with whom Rawdon had the
outstanding whist account) had been talking about the Colonel just
before he came in.

``It is come just in the nick of time,'' said Smith.  ``I suppose Crawley
had not a shilling in the world.''

``It's a wind that blows everybody good,'' Mr.\ Brown said.  ``He can't go
away without paying me a pony he owes me.''

``What's the salary?'' asked Smith.

``Two or three thousand,'' answered the other.  ``But the climate's so
infernal, they don't enjoy it long. Liverseege died after eighteen
months of it, and the man before went off in six weeks, I hear.''

``Some people say his brother is a very clever man.  I always found him
a d--------- bore,'' Smith ejaculated.  ``He must have good interest, though.
He must have got the Colonel the place.''

``He!'' said Brown, with a sneer.  ``Pooh.  It was Lord Steyne got it.''

``How do you mean?''

``A virtuous woman is a crown to her husband,'' answered the other
enigmatically, and went to read his papers.

Rawdon, for his part, read in the Royalist the following astonishing
paragraph:

GOVERNORSHIP OF COVENTRY ISLAND.---H.M.S. Yellowjack, Commander
Jaunders, has brought letters and papers from Coventry Island.  H. E.
Sir Thomas Liverseege had fallen a victim to the prevailing fever at
Swampton.  His loss is deeply felt in the flourishing colony.  We hear
that the Governorship has been offered to Colonel Rawdon Crawley, C.B.,
a distinguished Waterloo officer.  We need not only men of acknowledged
bravery, but men of administrative talents to superintend the affairs
of our colonies, and we have no doubt that the gentleman selected by
the Colonial Office to fill the lamented vacancy which has occurred at
Coventry Island is admirably calculated for the post which he is about
to occupy.


``Coventry Island!  Where was it? Who had appointed him to the
government? You must take me out as your secretary, old boy,'' Captain
Macmurdo said laughing; and as Crawley and his friend sat wondering and
perplexed over the announcement, the Club waiter brought in to the
Colonel a card on which the name of Mr.\ Wenham was engraved, who begged
to see Colonel Crawley.

The Colonel and his aide-de-camp went out to meet the gentleman,
rightly conjecturing that he was an emissary of Lord Steyne.  ``How d'ye
do, Crawley? I am glad to see you,'' said Mr.\ Wenham with a bland smile,
and grasping Crawley's hand with great cordiality.

``You come, I suppose, from---''

``Exactly,'' said Mr.\ Wenham.

``Then this is my friend Captain Macmurdo, of the Life Guards Green.''

``Delighted to know Captain Macmurdo, I'm sure,'' Mr.\ Wenham said and
tendered another smile and shake of the hand to the second, as he had
done to the principal. Mac put out one finger, armed with a buckskin
glove, and made a very frigid bow to Mr.\ Wenham over his tight cravat.
He was, perhaps, discontented at being put in communication with a
pekin, and thought that Lord Steyne should have sent him a Colonel at
the very least.

``As Macmurdo acts for me, and knows what I mean,'' Crawley said, ``I had
better retire and leave you together.''

``Of course,'' said Macmurdo.

``By no means, my dear Colonel,'' Mr.\ Wenham said; ``the interview which I
had the honour of requesting was with you personally, though the
company of Captain Macmurdo cannot fail to be also most pleasing.  In
fact, Captain, I hope that our conversation will lead to none but the
most agreeable results, very different from those which my friend
Colonel Crawley appears to anticipate.''

``Humph!'' said Captain Macmurdo.  Be hanged to these civilians, he
thought to himself, they are always for arranging and speechifying. Mr.\ %
Wenham took a chair which was not offered to him---took a paper from his
pocket, and resumed---

``You have seen this gratifying announcement in the papers this morning,
Colonel? Government has secured a most valuable servant, and you, if
you accept office, as I presume you will, an excellent appointment.
Three thousand a year, delightful climate, excellent government-house,
all your own way in the Colony, and a certain promotion.  I
congratulate you with all my heart.  I presume you know, gentlemen, to
whom my friend is indebted for this piece of patronage?''

``Hanged if I know,'' the Captain said; his principal turned very red.

``To one of the most generous and kindest men in the world, as he is one
of the greatest---to my excellent friend, the Marquis of Steyne.''

``I'll see him d--------- before I take his place,'' growled out Rawdon.

``You are irritated against my noble friend,'' Mr.\ Wenham calmly resumed;
``and now, in the name of common sense and justice, tell me why?''

``WHY?'' cried Rawdon in surprise.

``Why? Dammy!'' said the Captain, ringing his stick on the ground.

``Dammy, indeed,'' said Mr.\ Wenham with the most agreeable smile; ``still,
look at the matter as a man of the world---as an honest man---and see if
you have not been in the wrong.  You come home from a journey, and
find---what?---my Lord Steyne supping at your house in Curzon Street with
Mrs.\ Crawley.  Is the circumstance strange or novel? Has he not been a
hundred times before in the same position? Upon my honour and word as a
gentleman''---Mr.\ Wenham here put his hand on his waistcoat with a
parliamentary air---``I declare I think that your suspicions are
monstrous and utterly unfounded, and that they injure an honourable
gentleman who has proved his good-will towards you by a thousand
benefactions---and a most spotless and innocent lady.''

``You don't mean to say that---that Crawley's mistaken?'' said Mr.\ %
Macmurdo.

``I believe that Mrs.\ Crawley is as innocent as my wife, Mrs.\ Wenham,''
Mr.\ Wenham said with great energy.  ``I believe that, misled by an
infernal jealousy, my friend here strikes a blow against not only an
infirm and old man of high station, his constant friend and benefactor,
but against his wife, his own dearest honour, his son's future
reputation, and his own prospects in life.''

``I will tell you what happened,'' Mr.\ Wenham continued with great
solemnity; ``I was sent for this morning by my Lord Steyne, and found
him in a pitiable state, as, I need hardly inform Colonel Crawley, any
man of age and infirmity would be after a personal conflict with a man
of your strength.  I say to your face; it was a cruel advantage you
took of that strength, Colonel Crawley.  It was not only the body of my
noble and excellent friend which was wounded---his heart, sir, was
bleeding.  A man whom he had loaded with benefits and regarded with
affection had subjected him to the foulest indignity.  What was this
very appointment, which appears in the journals of to-day, but a proof
of his kindness to you? When I saw his Lordship this morning I found
him in a state pitiable indeed to see, and as anxious as you are to
revenge the outrage committed upon him, by blood.  You know he has
given his proofs, I presume, Colonel Crawley?''

``He has plenty of pluck,'' said the Colonel.  ``Nobody ever said he
hadn't.''

``His first order to me was to write a letter of challenge, and to carry
it to Colonel Crawley.  One or other of us,'' he said, ``must not survive
the outrage of last night.''

Crawley nodded.  ``You're coming to the point, Wenham,'' he said.

``I tried my utmost to calm Lord Steyne. 'Good God! sir,' I said, 'how I
regret that Mrs.\ Wenham and myself had not accepted Mrs.\ Crawley's
invitation to sup with her!'''

``She asked you to sup with her?'' Captain Macmurdo said.

``After the opera.  Here's the note of invitation---stop---no, this is
another paper---I thought I had it, but it's of no consequence, and I
pledge you my word to the fact.  If we had come---and it was only one of
Mrs.\ Wenham's headaches which prevented us---she suffers under them a
good deal, especially in the spring---if we had come, and you had
returned home, there would have been no quarrel, no insult, no
suspicion---and so it is positively because my poor wife has a headache
that you are to bring death down upon two men of honour and plunge two
of the most excellent and ancient families in the kingdom into disgrace
and sorrow.''

Mr.\ Macmurdo looked at his principal with the air of a man profoundly
puzzled, and Rawdon felt with a kind of rage that his prey was escaping
him.  He did not believe a word of the story, and yet, how discredit or
disprove it?

Mr.\ Wenham continued with the same fluent oratory, which in his place
in Parliament he had so often practised---``I sat for an hour or more by
Lord Steyne's bedside, beseeching, imploring Lord Steyne to forego his
intention of demanding a meeting.  I pointed out to him that the
circumstances were after all suspicious---they were suspicious.  I
acknowledge it---any man in your position might have been taken in---I
said that a man furious with jealousy is to all intents and purposes a
madman, and should be as such regarded---that a duel between you must
lead to the disgrace of all parties concerned---that a man of his
Lordship's exalted station had no right in these days, when the most
atrocious revolutionary principles, and the most dangerous levelling
doctrines are preached among the vulgar, to create a public scandal;
and that, however innocent, the common people would insist that he was
guilty.  In fine, I implored him not to send the challenge.''

``I don't believe one word of the whole story,'' said Rawdon, grinding
his teeth.  ``I believe it a d--------- lie, and that you're in it, Mr.\ %
Wenham.  If the challenge don't come from him, by Jove it shall come
from me.''

Mr.\ Wenham turned deadly pale at this savage interruption of the
Colonel and looked towards the door.

But he found a champion in Captain Macmurdo.  That gentleman rose up
with an oath and rebuked Rawdon for his language.  ``You put the affair
into my hands, and you shall act as I think fit, by Jove, and not as
you do. You have no right to insult Mr.\ Wenham with this sort of
language; and dammy, Mr.\ Wenham, you deserve an apology.  And as for a
challenge to Lord Steyne, you may get somebody else to carry it, I
won't.  If my lord, after being thrashed, chooses to sit still, dammy
let him. And as for the affair with---with Mrs.\ Crawley, my belief is,
there's nothing proved at all:  that your wife's innocent, as innocent
as Mr.\ Wenham says she is; and at any rate that you would be a d---fool
not to take the place and hold your tongue.''

``Captain Macmurdo, you speak like a man of sense,'' Mr.\ Wenham cried
out, immensely relieved---``I forget any words that Colonel Crawley has
used in the irritation of the moment.''

``I thought you would,'' Rawdon said with a sneer.

``Shut your mouth, you old stoopid,'' the Captain said good-naturedly.
``Mr.\ Wenham ain't a fighting man; and quite right, too.''

``This matter, in my belief,'' the Steyne emissary cried, ``ought to be
buried in the most profound oblivion.  A word concerning it should
never pass these doors.  I speak in the interest of my friend, as well
as of Colonel Crawley, who persists in considering me his enemy.''

``I suppose Lord Steyne won't talk about it very much,'' said Captain
Macmurdo; ``and I don't see why our side should.  The affair ain't a
very pretty one, any way you take it, and the less said about it the
better. It's you are thrashed, and not us; and if you are satisfied,
why, I think, we should be.''

Mr.\ Wenham took his hat, upon this, and Captain Macmurdo following him
to the door, shut it upon himself and Lord Steyne's agent, leaving
Rawdon chafing within.  When the two were on the other side, Macmurdo
looked hard at the other ambassador and with an expression of anything
but respect on his round jolly face.

``You don't stick at a trifle, Mr.\ Wenham,'' he said.

``You flatter me, Captain Macmurdo,'' answered the other with a smile.
``Upon my honour and conscience now, Mrs.\ Crawley did ask us to sup
after the opera.''

``Of course; and Mrs.\ Wenham had one of her head-aches.  I say, I've got
a thousand-pound note here, which I will give you if you will give me a
receipt, please; and I will put the note up in an envelope for Lord
Steyne. My man shan't fight him.  But we had rather not take his money.''

``It was all a mistake---all a mistake, my dear sir,'' the other said with
the utmost innocence of manner; and was bowed down the Club steps by
Captain Macmurdo, just as Sir Pitt Crawley ascended them. There was a
slight acquaintance between these two gentlemen, and the Captain, going
back with the Baronet to the room where the latter's brother was, told
Sir Pitt, in confidence, that he had made the affair all right between
Lord Steyne and the Colonel.

Sir Pitt was well pleased, of course, at this intelligence, and
congratulated his brother warmly upon the peaceful issue of the affair,
making appropriate moral remarks upon the evils of duelling and the
unsatisfactory nature of that sort of settlement of disputes.

And after this preface, he tried with all his eloquence to effect a
reconciliation between Rawdon and his wife. He recapitulated the
statements which Becky had made, pointed out the probabilities of their
truth, and asserted his own firm belief in her innocence.

But Rawdon would not hear of it.  ``She has kep money concealed from me
these ten years,'' he said ``She swore, last night only, she had none
from Steyne.  She knew it was all up, directly I found it.  If she's
not guilty, Pitt, she's as bad as guilty, and I'll never see her
again---never.'' His head sank down on his chest as he spoke the words,
and he looked quite broken and sad.

``Poor old boy,'' Macmurdo said, shaking his head.

Rawdon Crawley resisted for some time the idea of taking the place
which had been procured for him by so odious a patron, and was also for
removing the boy from the school where Lord Steyne's interest had
placed him.  He was induced, however, to acquiesce in these benefits by
the entreaties of his brother and Macmurdo, but mainly by the latter,
pointing out to him what a fury Steyne would be in to think that his
enemy's fortune was made through his means.

When the Marquis of Steyne came abroad after his accident, the Colonial
Secretary bowed up to him and congratulated himself and the Service
upon having made so excellent an appointment.  These congratulations
were received with a degree of gratitude which may be imagined on the
part of Lord Steyne.

The secret of the rencontre between him and Colonel Crawley was buried
in the profoundest oblivion, as Wenham said; that is, by the seconds
and the principals. But before that evening was over it was talked of
at fifty dinner-tables in Vanity Fair.  Little Cackleby himself went to
seven evening parties and told the story with comments and emendations
at each place.  How Mrs.\ Washington White revelled in it!  The
Bishopess of Ealing was shocked beyond expression; the Bishop went and
wrote his name down in the visiting-book at Gaunt House that very day.
Little Southdown was sorry; so you may be sure was his sister Lady
Jane, very sorry.  Lady Southdown wrote it off to her other daughter at
the Cape of Good Hope.  It was town-talk for at least three days, and
was only kept out of the newspapers by the exertions of Mr.\ Wagg,
acting upon a hint from Mr.\ Wenham.

The bailiffs and brokers seized upon poor Raggles in Curzon Street, and
the late fair tenant of that poor little mansion was in the
meanwhile---where? Who cared!  Who asked after a day or two? Was she
guilty or not? We all know how charitable the world is, and how the
verdict of Vanity Fair goes when there is a doubt.  Some people said
she had gone to Naples in pursuit of Lord Steyne, whilst others averred
that his Lordship quitted that city and fled to Palermo on hearing of
Becky's arrival; some said she was living in Bierstadt, and had become
a dame d'honneur to the Queen of Bulgaria; some that she was at
Boulogne; and others, at a boarding-house at Cheltenham.

Rawdon made her a tolerable annuity, and we may be sure that she was a
woman who could make a little money go a great way, as the saying is.
He would have paid his debts on leaving England, could he have got any
Insurance Office to take his life, but the climate of Coventry Island
was so bad that he could borrow no money on the strength of his salary.
He remitted, however, to his brother punctually, and wrote to his
little boy regularly every mail.  He kept Macmurdo in cigars and sent
over quantities of shells, cayenne pepper, hot pickles, guava jelly,
and colonial produce to Lady Jane. He sent his brother home the Swamp
Town Gazette, in which the new Governor was praised with immense
enthusiasm; whereas the Swamp Town Sentinel, whose wife was not asked
to Government House, declared that his Excellency was a tyrant,
compared to whom Nero was an enlightened philanthropist.  Little Rawdon
used to like to get the papers and read about his Excellency.

His mother never made any movement to see the child. He went home to
his aunt for Sundays and holidays; he soon knew every bird's nest about
Queen's Crawley, and rode out with Sir Huddlestone's hounds, which he
admired so on his first well-remembered visit to Hampshire.



\chapter{Georgy is Made a Gentleman}

Georgy Osborne was now fairly established in his grandfather's mansion
in Russell Square, occupant of his father's room in the house and heir
apparent of all the splendours there.  The good looks, gallant bearing,
and gentlemanlike appearance of the boy won the grandsire's heart for
him.  Mr.\ Osborne was as proud of him as ever he had been of the elder
George.

The child had many more luxuries and indulgences than had been awarded
his father.  Osborne's commerce had prospered greatly of late years.
His wealth and importance in the City had very much increased.  He had
been glad enough in former days to put the elder George to a good
private school; and a commission in the army for his son had been a
source of no small pride to him; for little George and his future
prospects the old man looked much higher.  He would make a gentleman of
the little chap, was Mr.\ Osborne's constant saying regarding little
Georgy.  He saw him in his mind's eye, a collegian, a Parliament man, a
Baronet, perhaps.  The old man thought he would die contented if he
could see his grandson in a fair way to such honours.  He would have
none but a tip-top college man to educate him---none of your quacks and
pretenders---no, no.  A few years before, he used to be savage, and
inveigh against all parsons, scholars, and the like declaring that they
were a pack of humbugs, and quacks that weren't fit to get their living
but by grinding Latin and Greek, and a set of supercilious dogs that
pretended to look down upon British merchants and gentlemen, who could
buy up half a hundred of 'em.  He would mourn now, in a very solemn
manner, that his own education had been neglected, and repeatedly point
out, in pompous orations to Georgy, the necessity and excellence of
classical acquirements.

When they met at dinner the grandsire used to ask the lad what he had
been reading during the day, and was greatly interested at the report
the boy gave of his own studies, pretending to understand little George
when he spoke regarding them.  He made a hundred blunders and showed
his ignorance many a time.  It did not increase the respect which the
child had for his senior. A quick brain and a better education
elsewhere showed the boy very soon that his grandsire was a dullard,
and he began accordingly to command him and to look down upon him; for
his previous education, humble and contracted as it had been, had made
a much better gentleman of Georgy than any plans of his grandfather
could make him.  He had been brought up by a kind, weak, and tender
woman, who had no pride about anything but about him, and whose heart
was so pure and whose bearing was so meek and humble that she could not
but needs be a true lady.  She busied herself in gentle offices and
quiet duties; if she never said brilliant things, she never spoke or
thought unkind ones; guileless and artless, loving and pure, indeed how
could our poor little Amelia be other than a real gentlewoman!

Young Georgy lorded over this soft and yielding nature; and the
contrast of its simplicity and delicacy with the coarse pomposity of
the dull old man with whom he next came in contact made him lord over
the latter too.  If he had been a Prince Royal he could not have been
better brought up to think well of himself.

Whilst his mother was yearning after him at home, and I do believe
every hour of the day, and during most hours of the sad lonely nights,
thinking of him, this young gentleman had a number of pleasures and
consolations administered to him, which made him for his part bear the
separation from Amelia very easily.  Little boys who cry when they are
going to school cry because they are going to a very uncomfortable
place.  It is only a few who weep from sheer affection.  When you think
that the eyes of your childhood dried at the sight of a piece of
gingerbread, and that a plum cake was a compensation for the agony of
parting with your mamma and sisters, oh my friend and brother, you need
not be too confident of your own fine feelings.

Well, then, Master George Osborne had every comfort and luxury that a
wealthy and lavish old grandfather thought fit to provide.  The
coachman was instructed to purchase for him the handsomest pony which
could be bought for money, and on this George was taught to ride, first
at a riding-school, whence, after having performed satisfactorily
without stirrups, and over the leaping-bar, he was conducted through
the New Road to Regent's Park, and then to Hyde Park, where he rode in
state with Martin the coachman behind him. Old Osborne, who took
matters more easily in the City now, where he left his affairs to his
junior partners, would often ride out with Miss O.  in the same
fashionable direction. As little Georgy came cantering up with his
dandified air and his heels down, his grandfather would nudge the lad's
aunt and say, ``Look, Miss O.'' And he would laugh, and his face would
grow red with pleasure, as he nodded out of the window to the boy, as
the groom saluted the carriage, and the footman saluted Master George.
Here too his aunt, Mrs.\ Frederick Bullock (whose chariot might daily be
seen in the Ring, with bullocks or emblazoned on the panels and
harness, and three pasty-faced little Bullocks, covered with cockades
and feathers, staring from the windows) Mrs.\ Frederick Bullock, I say,
flung glances of the bitterest hatred at the little upstart as he rode
by with his hand on his side and his hat on one ear, as proud as a lord.

Though he was scarcely eleven years of age, Master George wore straps
and the most beautiful little boots like a man.  He had gilt spurs, and
a gold-headed whip, and a fine pin in his handkerchief, and the neatest
little kid gloves which Lamb's Conduit Street could furnish. His mother
had given him a couple of neckcloths, and carefully hemmed and made
some little shirts for him; but when her Eli came to see the widow,
they were replaced by much finer linen. He had little jewelled buttons
in the lawn shirt fronts.  Her humble presents had been put aside---I
believe Miss Osborne had given them to the coachman's boy.  Amelia
tried to think she was pleased at the change.  Indeed, she was happy
and charmed to see the boy looking so beautiful.

She had had a little black profile of him done for a shilling, and this
was hung up by the side of another portrait over her bed.  One day the
boy came on his accustomed visit, galloping down the little street at
Brompton, and bringing, as usual, all the inhabitants to the windows to
admire his splendour, and with great eagerness and a look of triumph in
his face, he pulled a case out of his great-coat---it was a natty white
great-coat, with a cape and a velvet collar---pulled out a red morocco
case, which he gave her.

``I bought it with my own money, Mamma,'' he said. ``I thought you'd like
it.''

Amelia opened the case, and giving a little cry of delighted affection,
seized the boy and embraced him a hundred times.  It was a miniature of
himself, very prettily done (though not half handsome enough, we may be
sure, the widow thought).  His grandfather had wished to have a picture
of him by an artist whose works, exhibited in a shop-window, in
Southampton Row, had caught the old gentleman's eye; and George, who
had plenty of money, bethought him of asking the painter how much a
copy of the little portrait would cost, saying that he would pay for it
out of his own money and that he wanted to give it to his mother.  The
pleased painter executed it for a small price, and old Osborne himself,
when he heard of the incident, growled out his satisfaction and gave
the boy twice as many sovereigns as he paid for the miniature.

But what was the grandfather's pleasure compared to Amelia's ecstacy?
That proof of the boy's affection charmed her so that she thought no
child in the world was like hers for goodness.  For long weeks after,
the thought of his love made her happy.  She slept better with the
picture under her pillow, and how many many times did she kiss it and
weep and pray over it!  A small kindness from those she loved made that
timid heart grateful.  Since her parting with George she had had no
such joy and consolation.

At his new home Master George ruled like a lord; at dinner he invited
the ladies to drink wine with the utmost coolness, and took off his
champagne in a way which charmed his old grandfather.  ``Look at him,''
the old man would say, nudging his neighbour with a delighted purple
face, ``did you ever see such a chap? Lord, Lord! he'll be ordering a
dressing-case next, and razors to shave with; I'm blessed if he won't.''

The antics of the lad did not, however, delight Mr.\ Osborne's friends
so much as they pleased the old gentleman.  It gave Mr.\ Justice Coffin
no pleasure to hear Georgy cut into the conversation and spoil his
stories. Colonel Fogey was not interested in seeing the little boy half
tipsy.  Mr.\ Sergeant Toffy's lady felt no particular gratitude, when,
with a twist of his elbow, he tilted a glass of port-wine over her
yellow satin and laughed at the disaster; nor was she better pleased,
although old Osborne was highly delighted, when Georgy ``whopped'' her
third boy (a young gentleman a year older than Georgy, and by chance
home for the holidays from Dr. Tickleus's at Ealing School) in Russell
Square. George's grandfather gave the boy a couple of sovereigns for
that feat and promised to reward him further for every boy above his
own size and age whom he whopped in a similar manner.  It is difficult
to say what good the old man saw in these combats; he had a vague
notion that quarrelling made boys hardy, and that tyranny was a useful
accomplishment for them to learn.  English youth have been so educated
time out of mind, and we have hundreds of thousands of apologists and
admirers of injustice, misery, and brutality, as perpetrated among
children.  Flushed with praise and victory over Master Toffy, George
wished naturally to pursue his conquests further, and one day as he was
strutting about in prodigiously dandified new clothes, near St.\ %
Pancras, and a young baker's boy made sarcastic comments upon his
appearance, the youthful patrician pulled off his dandy jacket with
great spirit, and giving it in charge to the friend who accompanied him
(Master Todd, of Great Coram Street, Russell Square, son of the junior
partner of the house of Osborne and Co.), George tried to whop the
little baker.  But the chances of war were unfavourable this time, and
the little baker whopped Georgy, who came home with a rueful black eye
and all his fine shirt frill dabbled with the claret drawn from his own
little nose.  He told his grandfather that he had been in combat with a
giant, and frightened his poor mother at Brompton with long, and by no
means authentic, accounts of the battle.

This young Todd, of Coram Street, Russell Square, was Master George's
great friend and admirer.  They both had a taste for painting
theatrical characters; for hardbake and raspberry tarts; for sliding
and skating in the Regent's Park and the Serpentine, when the weather
permitted; for going to the play, whither they were often conducted, by
Mr.\ Osborne's orders, by Rowson, Master George's appointed
body-servant, with whom they sat in great comfort in the pit.

In the company of this gentleman they visited all the principal
theatres of the metropolis; knew the names of all the actors from Drury
Lane to Sadler's Wells; and performed, indeed, many of the plays to the
Todd family and their youthful friends, with West's famous characters,
on their pasteboard theatre.  Rowson, the footman, who was of a
generous disposition, would not unfrequently, when in cash, treat his
young master to oysters after the play, and to a glass of rum-shrub for
a night-cap. We may be pretty certain that Mr.\ Rowson profited in his
turn by his young master's liberality and gratitude for the pleasures
to which the footman inducted him.

A famous tailor from the West End of the town---Mr.\ Osborne would have
none of your City or Holborn bunglers, he said, for the boy (though a
City tailor was good enough for HIM)---was summoned to ornament little
George's person, and was told to spare no expense in so doing.  So, Mr.\ %
Woolsey, of Conduit Street, gave a loose to his imagination and sent
the child home fancy trousers, fancy waistcoats, and fancy jackets
enough to furnish a school of little dandies.  Georgy had little white
waistcoats for evening parties, and little cut velvet waistcoats for
dinners, and a dear little darling shawl dressing-gown, for all the
world like a little man. He dressed for dinner every day, ``like a
regular West End swell,'' as his grandfather remarked; one of the
domestics was affected to his special service, attended him at his
toilette, answered his bell, and brought him his letters always on a
silver tray.

Georgy, after breakfast, would sit in the arm-chair in the dining-room
and read the Morning Post, just like a grown-up man.  ``How he DU dam
and swear,'' the servants would cry, delighted at his precocity.  Those
who remembered the Captain his father, declared Master George was his
Pa, every inch of him.  He made the house lively by his activity, his
imperiousness, his scolding, and his good-nature.

George's education was confided to a neighbouring scholar and private
pedagogue who ``prepared young noblemen and gentlemen for the
Universities, the senate, and the learned professions:  whose system
did not embrace the degrading corporal severities still practised at
the ancient places of education, and in whose family the pupils would
find the elegances of refined society and the confidence and affection
of a home.'' It was in this way that the Reverend Lawrence Veal of Hart
Street, Bloomsbury, and domestic Chaplain to the Earl of Bareacres,
strove with Mrs.\ Veal his wife to entice pupils.

By thus advertising and pushing sedulously, the domestic Chaplain and
his Lady generally succeeded in having one or two scholars by them---who
paid a high figure and were thought to be in uncommonly comfortable
quarters.  There was a large West Indian, whom nobody came to see, with
a mahogany complexion, a woolly head, and an exceedingly dandyfied
appearance; there was another hulking boy of three-and-twenty whose
education had been neglected and whom Mr.\ and Mrs.\ Veal were to
introduce into the polite world; there were two sons of Colonel Bangles
of the East India Company's Service:  these four sat down to dinner at
Mrs.\ Veal's genteel board, when Georgy was introduced to her
establishment.

Georgy was, like some dozen other pupils, only a day boy; he arrived in
the morning under the guardianship of his friend Mr.\ Rowson, and if it
was fine, would ride away in the afternoon on his pony, followed by the
groom.  The wealth of his grandfather was reported in the school to be
prodigious.  The Rev. Mr.\ Veal used to compliment Georgy upon it
personally, warning him that he was destined for a high station; that
it became him to prepare, by sedulity and docility in youth, for the
lofty duties to which he would be called in mature age; that obedience
in the child was the best preparation for command in the man; and that
he therefore begged George would not bring toffee into the school and
ruin the health of the Masters Bangles, who had everything they wanted
at the elegant and abundant table of Mrs.\ Veal.

With respect to learning, ``the Curriculum,'' as Mr.\ Veal loved to call
it, was of prodigious extent, and the young gentlemen in Hart Street
might learn a something of every known science.  The Rev. Mr.\ Veal had
an orrery, an electrifying machine, a turning lathe, a theatre (in the
wash-house), a chemical apparatus, and what he called a select library
of all the works of the best authors of ancient and modern times and
languages. He took the boys to the British Museum and descanted upon
the antiquities and the specimens of natural history there, so that
audiences would gather round him as he spoke, and all Bloomsbury highly
admired him as a prodigiously well-informed man.  And whenever he spoke
(which he did almost always), he took care to produce the very finest
and longest words of which the vocabulary gave him the use, rightly
judging that it was as cheap to employ a handsome, large, and sonorous
epithet, as to use a little stingy one.

Thus he would say to George in school, ``I observed on my return home
from taking the indulgence of an evening's scientific conversation with
my excellent friend Doctor Bulders---a true archaeologian, gentlemen, a
true archaeologian---that the windows of your venerated grandfather's
almost princely mansion in Russell Square were illuminated as if for
the purposes of festivity.  Am I right in my conjecture that Mr.\ %
Osborne entertained a society of chosen spirits round his sumptuous
board last night?''

Little Georgy, who had considerable humour, and used to mimic Mr.\ Veal
to his face with great spirit and dexterity, would reply that Mr.\ V.
was quite correct in his surmise.

``Then those friends who had the honour of partaking of Mr.\ Osborne's
hospitality, gentlemen, had no reason, I will lay any wager, to
complain of their repast.  I myself have been more than once so
favoured.  (By the way, Master Osborne, you came a little late this
morning, and have been a defaulter in this respect more than once.) I
myself, I say, gentlemen, humble as I am, have been found not unworthy
to share Mr.\ Osborne's elegant hospitality.  And though I have feasted
with the great and noble of the world---for I presume that I may call my
excellent friend and patron, the Right Honourable George Earl of
Bareacres, one of the number---yet I assure you that the board of the
British merchant was to the full as richly served, and his reception as
gratifying and noble.  Mr.\ Bluck, sir, we will resume, if you please,
that passage of Eutropis, which was interrupted by the late arrival of
Master Osborne.''

To this great man George's education was for some time entrusted.
Amelia was bewildered by his phrases, but thought him a prodigy of
learning.  That poor widow made friends of Mrs.\ Veal, for reasons of
her own.  She liked to be in the house and see Georgy coming to school
there.  She liked to be asked to Mrs.\ Veal's conversazioni, which took
place once a month (as you were informed on pink cards, with A\textTheta HNH
engraved on them), and where the professor
welcomed his pupils and their friends to weak tea and scientific
conversation. Poor little Amelia never missed one of these
entertainments and thought them delicious so long as she might have
Georgy sitting by her. And she would walk from Brompton in any weather,
and embrace Mrs.\ Veal with tearful gratitude for the delightful evening
she had passed, when, the company having retired and Georgy gone off
with Mr.\ Rowson, his attendant, poor Mrs.\ Osborne put on her cloaks and
her shawls preparatory to walking home.

As for the learning which Georgy imbibed under this valuable master of
a hundred sciences, to judge from the weekly reports which the lad took
home to his grandfather, his progress was remarkable.  The names of a
score or more of desirable branches of knowledge were printed in a
table, and the pupil's progress in each was marked by the professor.
In Greek Georgy was pronounced aristos, in Latin optimus, in French
tres bien, and so forth; and everybody had prizes for everything at the
end of the year.  Even Mr.\ Swartz, the wooly-headed young gentleman,
and half-brother to the Honourable Mrs.\ Mac Mull, and Mr.\ Bluck, the
neglected young pupil of three-and-twenty from the agricultural
district, and that idle young scapegrace of a Master Todd before
mentioned, received little eighteen-penny books, with ``Athene'' engraved
on them, and a pompous Latin inscription from the professor to his
young friends.

The family of this Master Todd were hangers-on of the house of Osborne.
The old gentleman had advanced Todd from being a clerk to be a junior
partner in his establishment.

Mr.\ Osborne was the godfather of young Master Todd (who in subsequent
life wrote Mr.\ Osborne Todd on his cards and became a man of decided
fashion), while Miss Osborne had accompanied Miss Maria Todd to the
font, and gave her protegee a prayer-book, a collection of tracts, a
volume of very low church poetry, or some such memento of her goodness
every year.  Miss O.  drove the Todds out in her carriage now and then;
when they were ill, her footman, in large plush smalls and waistcoat,
brought jellies and delicacies from Russell Square to Coram Street.
Coram Street trembled and looked up to Russell Square indeed, and Mrs.\ %
Todd, who had a pretty hand at cutting out paper trimmings for haunches
of mutton, and could make flowers, ducks, \&c., out of turnips and
carrots in a very creditable manner, would go to ``the Square,'' as it
was called, and assist in the preparations incident to a great dinner,
without even so much as thinking of sitting down to the banquet.  If
any guest failed at the eleventh hour, Todd was asked to dine.  Mrs.\ %
Todd and Maria came across in the evening, slipped in with a muffled
knock, and were in the drawing-room by the time Miss Osborne and the
ladies under her convoy reached that apartment---and ready to fire off
duets and sing until the gentlemen came up.  Poor Maria Todd; poor
young lady!  How she had to work and thrum at these duets and sonatas
in the Street, before they appeared in public in the Square!

Thus it seemed to be decreed by fate that Georgy was to domineer over
everybody with whom he came in contact, and that friends, relatives,
and domestics were all to bow the knee before the little fellow.  It
must be owned that he accommodated himself very willingly to this
arrangement.  Most people do so.  And Georgy liked to play the part of
master and perhaps had a natural aptitude for it.

In Russell Square everybody was afraid of Mr.\ Osborne, and Mr.\ Osborne
was afraid of Georgy.  The boy's dashing manners, and offhand rattle
about books and learning, his likeness to his father (dead unreconciled
in Brussels yonder) awed the old gentleman and gave the young boy the
mastery.  The old man would start at some hereditary feature or tone
unconsciously used by the little lad, and fancy that George's father
was again before him.  He tried by indulgence to the grandson to make
up for harshness to the elder George.  People were surprised at his
gentleness to the boy.  He growled and swore at Miss Osborne as usual,
and would smile when George came down late for breakfast.

Miss Osborne, George's aunt, was a faded old spinster, broken down by
more than forty years of dulness and coarse usage.  It was easy for a
lad of spirit to master her. And whenever George wanted anything from
her, from the jam-pots in her cupboards to the cracked and dry old
colours in her paint-box (the old paint-box which she had had when she
was a pupil of Mr.\ Smee and was still almost young and blooming),
Georgy took possession of the object of his desire, which obtained, he
took no further notice of his aunt.

For his friends and cronies, he had a pompous old schoolmaster, who
flattered him, and a toady, his senior, whom he could thrash.  It was
dear Mrs.\ Todd's delight to leave him with her youngest daughter, Rosa
Jemima, a darling child of eight years old.  The little pair looked so
well together, she would say (but not to the folks in ``the Square,'' we
may be sure) ``who knows what might happen? Don't they make a pretty
little couple?'' the fond mother thought.

The broken-spirited, old, maternal grandfather was likewise subject to
the little tyrant.  He could not help respecting a lad who had such
fine clothes and rode with a groom behind him.  Georgy, on his side,
was in the constant habit of hearing coarse abuse and vulgar satire
levelled at John Sedley by his pitiless old enemy, Mr.\ Osborne.
Osborne used to call the other the old pauper, the old coal-man, the
old bankrupt, and by many other such names of brutal contumely.  How
was little George to respect a man so prostrate? A few months after he
was with his paternal grandfather, Mrs.\ Sedley died. There had been
little love between her and the child. He did not care to show much
grief.  He came down to visit his mother in a fine new suit of
mourning, and was very angry that he could not go to a play upon which
he had set his heart.

The illness of that old lady had been the occupation and perhaps the
safeguard of Amelia.  What do men know about women's martyrdoms? We
should go mad had we to endure the hundredth part of those daily pains
which are meekly borne by many women.  Ceaseless slavery meeting with
no reward; constant gentleness and kindness met by cruelty as constant;
love, labour, patience, watchfulness, without even so much as the
acknowledgement of a good word; all this, how many of them have to bear
in quiet, and appear abroad with cheerful faces as if they felt
nothing.  Tender slaves that they are, they must needs be hypocrites
and weak.

From her chair Amelia's mother had taken to her bed, which she had
never left, and from which Mrs.\ Osborne herself was never absent except
when she ran to see George.  The old lady grudged her even those rare
visits; she, who had been a kind, smiling, good-natured mother once, in
the days of her prosperity, but whom poverty and infirmities had broken
down.  Her illness or estrangement did not affect Amelia.  They rather
enabled her to support the other calamity under which she was
suffering, and from the thoughts of which she was kept by the ceaseless
calls of the invalid.  Amelia bore her harshness quite gently; smoothed
the uneasy pillow; was always ready with a soft answer to the watchful,
querulous voice; soothed the sufferer with words of hope, such as her
pious simple heart could best feel and utter, and closed the eyes that
had once looked so tenderly upon her.

Then all her time and tenderness were devoted to the consolation and
comfort of the bereaved old father, who was stunned by the blow which
had befallen him, and stood utterly alone in the world.  His wife, his
honour, his fortune, everything he loved best had fallen away from him.
There was only Amelia to stand by and support with her gentle arms the
tottering, heart-broken old man. We are not going to write the history:
it would be too dreary and stupid.  I can see Vanity Fair yawning over
it d'avance.

One day as the young gentlemen were assembled in the study at the Rev.
Mr.\ Veal's, and the domestic chaplain to the Right Honourable the Earl
of Bareacres was spouting away as usual, a smart carriage drove up to
the door decorated with the statue of Athene, and two gentlemen stepped
out.  The young Masters Bangles rushed to the window with a vague
notion that their father might have arrived from Bombay.  The great
hulking scholar of three-and-twenty, who was crying secretly over a
passage of Eutropius, flattened his neglected nose against the panes
and looked at the drag, as the laquais de place sprang from the box and
let out the persons in the carriage.

``It's a fat one and a thin one,'' Mr.\ Bluck said as a thundering knock
came to the door.

Everybody was interested, from the domestic chaplain himself, who hoped
he saw the fathers of some future pupils, down to Master Georgy, glad
of any pretext for laying his book down.

The boy in the shabby livery with the faded copper buttons, who always
thrust himself into the tight coat to open the door, came into the
study and said, ``Two gentlemen want to see Master Osborne.'' The
professor had had a trifling altercation in the morning with that young
gentleman, owing to a difference about the introduction of crackers in
school-time; but his face resumed its habitual expression of bland
courtesy as he said, ``Master Osborne, I give you full permission to go
and see your carriage friends---to whom I beg you to convey the
respectful compliments of myself and Mrs.\ Veal.''

Georgy went into the reception-room and saw two strangers, whom he
looked at with his head up, in his usual haughty manner.  One was fat,
with mustachios, and the other was lean and long, in a blue frock-coat,
with a brown face and a grizzled head.

``My God, how like he is!'' said the long gentleman with a start. ``Can
you guess who we are, George?''

The boy's face flushed up, as it did usually when he was moved, and his
eyes brightened.  ``I don't know the other,'' he said, ``but I should
think you must be Major Dobbin.''

Indeed it was our old friend.  His voice trembled with pleasure as he
greeted the boy, and taking both the other's hands in his own, drew the
lad to him.

``Your mother has talked to you about me---has she?'' he said.

``That she has,'' Georgy answered, ``hundreds and hundreds of times.''



\chapter{Eothen} % HOGG: Should this be typeset \foreign? What does this mean?

It was one of the many causes for personal pride with which old Osborne
chose to recreate himself that Sedley, his ancient rival, enemy, and
benefactor, was in his last days so utterly defeated and humiliated as
to be forced to accept pecuniary obligations at the hands of the man
who had most injured and insulted him.  The successful man of the world
cursed the old pauper and relieved him from time to time.  As he
furnished George with money for his mother, he gave the boy to
understand by hints, delivered in his brutal, coarse way, that George's
maternal grandfather was but a wretched old bankrupt and dependant, and
that John Sedley might thank the man to whom he already owed ever so
much money for the aid which his generosity now chose to administer.
George carried the pompous supplies to his mother and the shattered old
widower whom it was now the main business of her life to tend and
comfort.  The little fellow patronized the feeble and disappointed old
man.

It may have shown a want of ``proper pride'' in Amelia that she chose to
accept these money benefits at the hands of her father's enemy. But
proper pride and this poor lady had never had much acquaintance
together. A disposition naturally simple and demanding protection; a
long course of poverty and humility, of daily privations, and hard
words, of kind offices and no returns, had been her lot ever since
womanhood almost, or since her luckless marriage with George Osborne.
You who see your betters bearing up under this shame every day, meekly
suffering under the slights of fortune, gentle and unpitied, poor, and
rather despised for their poverty, do you ever step down from your
prosperity and wash the feet of these poor wearied beggars? The very
thought of them is odious and low.  ``There must be classes---there must
be rich and poor,'' Dives says, smacking his claret (it is well if he
even sends the broken meat out to Lazarus sitting under the window).
Very true; but think how mysterious and often unaccountable it is---that
lottery of life which gives to this man the purple and fine linen and
sends to the other rags for garments and dogs for comforters.

So I must own that, without much repining, on the contrary with
something akin to gratitude, Amelia took the crumbs that her father-in-law
let drop now and then, and with them fed her own parent.
Directly she understood it to be her duty, it was this young woman's
nature (ladies, she is but thirty still, and we choose to call her a
young woman even at that age) it was, I say, her nature to sacrifice
herself and to fling all that she had at the feet of the beloved
object.  During what long thankless nights had she worked out her
fingers for little Georgy whilst at home with her; what buffets,
scorns, privations, poverties had she endured for father and mother!
And in the midst of all these solitary resignations and unseen
sacrifices, she did not respect herself any more than the world
respected her, but I believe thought in her heart that she was a
poor-spirited, despicable little creature, whose luck in life was only
too good for her merits.  O you poor women!  O you poor secret martyrs
and victims, whose life is a torture, who are stretched on racks in
your bedrooms, and who lay your heads down on the block daily at the
drawing-room table; every man who watches your pains, or peers into
those dark places where the torture is administered to you, must pity
you---and---and thank God that he has a beard.  I recollect seeing, years
ago, at the prisons for idiots and madmen at Bicetre, near Paris, a
poor wretch bent down under the bondage of his imprisonment and his
personal infirmity, to whom one of our party gave a halfpenny worth of
snuff in a cornet or ``screw'' of paper.  The kindness was too much for
the poor epileptic creature. He cried in an anguish of delight and
gratitude:  if anybody gave you and me a thousand a year, or saved our
lives, we could not be so affected.  And so, if you properly tyrannize
over a woman, you will find a ha'p'orth of kindness act upon her and
bring tears into her eyes, as though you were an angel benefiting her.

Some such boons as these were the best which Fortune allotted to poor
little Amelia.  Her life, begun not unprosperously, had come down to
this---to a mean prison and a long, ignoble bondage.  Little George
visited her captivity sometimes and consoled it with feeble gleams of
encouragement.  Russell Square was the boundary of her prison:  she
might walk thither occasionally, but was always back to sleep in her
cell at night; to perform cheerless duties; to watch by thankless
sick-beds; to suffer the harassment and tyranny of querulous
disappointed old age.  How many thousands of people are there, women
for the most part, who are doomed to endure this long slavery?---who are
hospital nurses without wages---sisters of Charity, if you like, without
the romance and the sentiment of sacrifice---who strive, fast, watch,
and suffer, unpitied, and fade away ignobly and unknown.

The hidden and awful Wisdom which apportions the destinies of mankind
is pleased so to humiliate and cast down the tender, good, and wise,
and to set up the selfish, the foolish, or the wicked. Oh, be humble,
my brother, in your prosperity!  Be gentle with those who are less
lucky, if not more deserving.  Think, what right have you to be
scornful, whose virtue is a deficiency of temptation, whose success may
be a chance, whose rank may be an ancestor's accident, whose prosperity
is very likely a satire.

They buried Amelia's mother in the churchyard at Brompton, upon just
such a rainy, dark day as Amelia recollected when first she had been
there to marry George. Her little boy sat by her side in pompous new
sables. She remembered the old pew-woman and clerk.  Her thoughts were
away in other times as the parson read. But that she held George's hand
in her own, perhaps she would have liked to change places with....
Then, as usual, she felt ashamed of her selfish thoughts and prayed
inwardly to be strengthened to do her duty.

So she determined with all her might and strength to try and make her
old father happy.  She slaved, toiled, patched, and mended, sang and
played backgammon, read out the newspaper, cooked dishes for old
Sedley, walked him out sedulously into Kensington Gardens or the
Brompton Lanes, listened to his stories with untiring smiles and
affectionate hypocrisy, or sat musing by his side and communing with
her own thoughts and reminiscences, as the old man, feeble and
querulous, sunned himself on the garden benches and prattled about his
wrongs or his sorrows.  What sad, unsatisfactory thoughts those of the
widow were!  The children running up and down the slopes and broad
paths in the gardens reminded her of George, who was taken from her;
the first George was taken from her; her selfish, guilty love, in both
instances, had been rebuked and bitterly chastised. She strove to think
it was right that she should be so punished. She was such a miserable
wicked sinner.  She was quite alone in the world.

I know that the account of this kind of solitary imprisonment is
insufferably tedious, unless there is some cheerful or humorous
incident to enliven it---a tender gaoler, for instance, or a waggish
commandant of the fortress, or a mouse to come out and play about
Latude's beard and whiskers, or a subterranean passage under the
castle, dug by Trenck with his nails and a toothpick:  the historian
has no such enlivening incident to relate in the narrative of Amelia's
captivity.  Fancy her, if you please, during this period, very sad, but
always ready to smile when spoken to; in a very mean, poor, not to say
vulgar position of life; singing songs, making puddings, playing cards,
mending stockings, for her old father's benefit.  So, never mind,
whether she be a heroine or no; or you and I, however old, scolding,
and bankrupt---may we have in our last days a kind soft shoulder on
which to lean and a gentle hand to soothe our gouty old pillows.

Old Sedley grew very fond of his daughter after his wife's death, and
Amelia had her consolation in doing her duty by the old man.

But we are not going to leave these two people long in such a low and
ungenteel station of life.  Better days, as far as worldly prosperity
went, were in store for both. Perhaps the ingenious reader has guessed
who was the stout gentleman who called upon Georgy at his school in
company with our old friend Major Dobbin. It was another old
acquaintance returned to England, and at a time when his presence was
likely to be of great comfort to his relatives there.

Major Dobbin having easily succeeded in getting leave from his
good-natured commandant to proceed to Madras, and thence probably to
Europe, on urgent private affairs, never ceased travelling night and day
until he reached his journey's end, and had directed his march with such
celerity that he arrived at Madras in a high fever.  His servants who
accompanied him brought him to the house of the friend with whom he had
resolved to stay until his departure for Europe in a state of delirium;
and it was thought for many, many days that he would never travel
farther than the burying-ground of the church of St.\ George's, where
the troops should fire a salvo over his grave, and where many a gallant
officer lies far away from his home.

Here, as the poor fellow lay tossing in his fever, the people who
watched him might have heard him raving about Amelia.  The idea that he
should never see her again depressed him in his lucid hours.  He
thought his last day was come, and he made his solemn preparations for
departure, setting his affairs in this world in order and leaving the
little property of which he was possessed to those whom he most desired
to benefit.  The friend in whose house he was located witnessed his
testament.  He desired to be buried with a little brown hair-chain
which he wore round his neck and which, if the truth must be known, he
had got from Amelia's maid at Brussels, when the young widow's hair was
cut off, during the fever which prostrated her after the death of
George Osborne on the plateau at Mount St.\ John.

He recovered, rallied, relapsed again, having undergone such a process
of blood-letting and calomel as showed the strength of his original
constitution.  He was almost a skeleton when they put him on board the
Ramchunder East Indiaman, Captain Bragg, from Calcutta, touching at
Madras, and so weak and prostrate that his friend who had tended him
through his illness prophesied that the honest Major would never
survive the voyage, and that he would pass some morning, shrouded in
flag and hammock, over the ship's side, and carrying down to the sea
with him the relic that he wore at his heart.  But whether it was the
sea air, or the hope which sprung up in him afresh, from the day that
the ship spread her canvas and stood out of the roads towards home, our
friend began to amend, and he was quite well (though as gaunt as a
greyhound) before they reached the Cape.  ``Kirk will be disappointed of
his majority this time,'' he said with a smile; ``he will expect to find
himself gazetted by the time the regiment reaches home.'' For it must be
premised that while the Major was lying ill at Madras, having made such
prodigious haste to go thither, the gallant ---th, which had passed many
years abroad, which after its return from the West Indies had been
baulked of its stay at home by the Waterloo campaign, and had been
ordered from Flanders to India, had received orders home; and the Major
might have accompanied his comrades, had he chosen to wait for their
arrival at Madras.

Perhaps he was not inclined to put himself in his exhausted state again
under the guardianship of Glorvina. ``I think Miss O'Dowd would have
done for me,'' he said laughingly to a fellow-passenger, ``if we had had
her on board, and when she had sunk me, she would have fallen upon you,
depend upon it, and carried you in as a prize to Southampton, Jos, my
boy.''

For indeed it was no other than our stout friend who was also a
passenger on board the Ramchunder.  He had passed ten years in Bengal.
Constant dinners, tiffins, pale ale and claret, the prodigious labour
of cutcherry, and the refreshment of brandy-pawnee which he was forced
to take there, had their effect upon Waterloo Sedley. A voyage to
Europe was pronounced necessary for him---and having served his full
time in India and had fine appointments which had enabled him to lay by
a considerable sum of money, he was free to come home and stay with a
good pension, or to return and resume that rank in the service to which
his seniority and his vast talents entitled him.

He was rather thinner than when we last saw him, but had gained in
majesty and solemnity of demeanour. He had resumed the mustachios to
which his services at Waterloo entitled him, and swaggered about on
deck in a magnificent velvet cap with a gold band and a profuse
ornamentation of pins and jewellery about his person. He took breakfast
in his cabin and dressed as solemnly to appear on the quarter-deck as
if he were going to turn out for Bond Street, or the Course at
Calcutta.  He brought a native servant with him, who was his valet and
pipe-bearer and who wore the Sedley crest in silver on his turban.
That oriental menial had a wretched life under the tyranny of Jos
Sedley.  Jos was as vain of his person as a woman, and took as long a
time at his toilette as any fading beauty.  The youngsters among the
passengers, Young Chaffers of the 150th, and poor little Ricketts,
coming home after his third fever, used to draw out Sedley at the
cuddy-table and make him tell prodigious stories about himself and his
exploits against tigers and Napoleon. He was great when he visited the
Emperor's tomb at Longwood, when to these gentlemen and the young
officers of the ship, Major Dobbin not being by, he described the whole
battle of Waterloo and all but announced that Napoleon never would have
gone to Saint Helena at all but for him, Jos Sedley.

After leaving St.\ Helena he became very generous, disposing of a great
quantity of ship stores, claret, preserved meats, and great casks
packed with soda-water, brought out for his private delectation.  There
were no ladies on board; the Major gave the pas of precedency to the
civilian, so that he was the first dignitary at table, and treated by
Captain Bragg and the officers of the Ramchunder with the respect which
his rank warranted.  He disappeared rather in a panic during a
two-days' gale, in which he had the portholes of his cabin battened
down, and remained in his cot reading the Washerwoman of Finchley
Common, left on board the Ramchunder by the Right Honourable the Lady
Emily Hornblower, wife of the Rev. Silas Hornblower, when on their
passage out to the Cape, where the Reverend gentleman was a missionary;
but, for common reading, he had brought a stock of novels and plays
which he lent to the rest of the ship, and rendered himself agreeable
to all by his kindness and condescension.

Many and many a night as the ship was cutting through the roaring dark
sea, the moon and stars shining overhead and the bell singing out the
watch, Mr.\ Sedley and the Major would sit on the quarter-deck of the
vessel talking about home, as the Major smoked his cheroot and the
civilian puffed at the hookah which his servant prepared for him.

In these conversations it was wonderful with what perseverance and
ingenuity Major Dobbin would manage to bring the talk round to the
subject of Amelia and her little boy.  Jos, a little testy about his
father's misfortunes and unceremonious applications to him, was soothed
down by the Major, who pointed out the elder's ill fortunes and old
age.  He would not perhaps like to live with the old couple, whose ways
and hours might not agree with those of a younger man, accustomed to
different society (Jos bowed at this compliment); but, the Major
pointed out, how advantageous it would be for Jos Sedley to have a
house of his own in London, and not a mere bachelor's establishment as
before; how his sister Amelia would be the very person to preside over
it; how elegant, how gentle she was, and of what refined good manners.
He recounted stories of the success which Mrs.\ George Osborne had had
in former days at Brussels, and in London, where she was much admired
by people of very great fashion; and he then hinted how becoming it
would be for Jos to send Georgy to a good school and make a man of him,
for his mother and her parents would be sure to spoil him.  In a word,
this artful Major made the civilian promise to take charge of Amelia
and her unprotected child.  He did not know as yet what events had
happened in the little Sedley family, and how death had removed the
mother, and riches had carried off George from Amelia.  But the fact is
that every day and always, this love-smitten and middle-aged gentleman
was thinking about Mrs.\ Osborne, and his whole heart was bent upon
doing her good.  He coaxed, wheedled, cajoled, and complimented Jos
Sedley with a perseverance and cordiality of which he was not aware
himself, very likely; but some men who have unmarried sisters or
daughters even, may remember how uncommonly agreeable gentlemen are to
the male relations when they are courting the females; and perhaps this
rogue of a Dobbin was urged by a similar hypocrisy.

The truth is, when Major Dobbin came on board the Ramchumder, very
sick, and for the three days she lay in the Madras Roads, he did not
begin to rally, nor did even the appearance and recognition of his old
acquaintance, Mr.\ Sedley, on board much cheer him, until after a
conversation which they had one day, as the Major was laid languidly on
the deck.  He said then he thought he was doomed; he had left a little
something to his godson in his will, and he trusted Mrs.\ Osborne would
remember him kindly and be happy in the marriage she was about to make.
``Married? not the least,'' Jos answered; ``he had heard from her:  she
made no mention of the marriage, and by the way, it was curious, she
wrote to say that Major Dobbin was going to be married, and hoped that
HE would be happy.'' What were the dates of Sedley's letters from
Europe? The civilian fetched them. They were two months later than the
Major's; and the ship's surgeon congratulated himself upon the
treatment adopted by him towards his new patient, who had been
consigned to shipboard by the Madras practitioner with very small hopes
indeed; for, from that day, the very day that he changed the draught,
Major Dobbin began to mend. And thus it was that deserving officer,
Captain Kirk, was disappointed of his majority.

After they passed St.\ Helena, Major Dobbin's gaiety and strength was
such as to astonish all his fellow passengers.  He larked with the
midshipmen, played single-stick with the mates, ran up the shrouds like
a boy, sang a comic song one night to the amusement of the whole party
assembled over their grog after supper, and rendered himself so gay,
lively, and amiable that even Captain Bragg, who thought there was
nothing in his passenger, and considered he was a poor-spirited feller
at first, was constrained to own that the Major was a reserved but
well-informed and meritorious officer.  ``He ain't got distangy manners,
dammy,'' Bragg observed to his first mate; ``he wouldn't do at Government
House, Roper, where his Lordship and Lady William was as kind to me,
and shook hands with me before the whole company, and asking me at
dinner to take beer with him, before the Commander-in-Chief himself; he
ain't got manners, but there's something about him---'' And thus Captain
Bragg showed that he possessed discrimination as a man, as well as
ability as a commander.

But a calm taking place when the Ramchunder was within ten days' sail
of England, Dobbin became so impatient and ill-humoured as to surprise
those comrades who had before admired his vivacity and good temper. He
did not recover until the breeze sprang up again, and was in a highly
excited state when the pilot came on board.  Good God, how his heart
beat as the two friendly spires of Southampton came in sight.



\chapter{Our Friend the Major}

Our Major had rendered himself so popular on board the Ramchunder that
when he and Mr.\ Sedley descended into the welcome shore-boat which was
to take them from the ship, the whole crew, men and officers, the great
Captain Bragg himself leading off, gave three cheers for Major Dobbin,
who blushed very much and ducked his head in token of thanks.  Jos, who
very likely thought the cheers were for himself, took off his
gold-laced cap and waved it majestically to his friends, and they were
pulled to shore and landed with great dignity at the pier, whence they
proceeded to the Royal George Hotel.

Although the sight of that magnificent round of beef, and the silver
tankard suggestive of real British home-brewed ale and porter, which
perennially greet the eyes of the traveller returning from foreign
parts who enters the coffee-room of the George, are so invigorating and
delightful that a man entering such a comfortable snug homely English
inn might well like to stop some days there, yet Dobbin began to talk
about a post-chaise instantly, and was no sooner at Southampton than he
wished to be on the road to London.  Jos, however, would not hear of
moving that evening.  Why was he to pass a night in a post-chaise
instead of a great large undulating downy feather-bed which was there
ready to replace the horrid little narrow crib in which the portly
Bengal gentleman had been confined during the voyage? He could not
think of moving till his baggage was cleared, or of travelling until he
could do so with his chillum.  So the Major was forced to wait over
that night, and dispatched a letter to his family announcing his
arrival, entreating from Jos a promise to write to his own friends.
Jos promised, but didn't keep his promise.  The Captain, the surgeon,
and one or two passengers came and dined with our two gentlemen at the
inn, Jos exerting himself in a sumptuous way in ordering the dinner and
promising to go to town the next day with the Major. The landlord said
it did his eyes good to see Mr.\ Sedley take off his first pint of
porter.  If I had time and dared to enter into digressions, I would
write a chapter about that first pint of porter drunk upon English
ground. Ah, how good it is!  It is worth-while to leave home for a
year, just to enjoy that one draught.

Major Dobbin made his appearance the next morning very neatly shaved
and dressed, according to his wont. Indeed, it was so early in the
morning that nobody was up in the house except that wonderful Boots of
an inn who never seems to want sleep; and the Major could hear the
snores of the various inmates of the house roaring through the
corridors as he creaked about in those dim passages.  Then the
sleepless Boots went shirking round from door to door, gathering up at
each the Bluchers, Wellingtons, Oxonians, which stood outside. Then
Jos's native servant arose and began to get ready his master's
ponderous dressing apparatus and prepare his hookah; then the
maidservants got up, and meeting the dark man in the passages,
shrieked, and mistook him for the devil.  He and Dobbin stumbled over
their pails in the passages as they were scouring the decks of the
Royal George.  When the first unshorn waiter appeared and unbarred the
door of the inn, the Major thought that the time for departure was
arrived, and ordered a post-chaise to be fetched instantly, that they
might set off.

He then directed his steps to Mr.\ Sedley's room and opened the curtains
of the great large family bed wherein Mr.\ Jos was snoring. ``Come, up!
Sedley,'' the Major said, ``it's time to be off; the chaise will be at
the door in half an hour.''

Jos growled from under the counterpane to know what the time was; but
when he at last extorted from the blushing Major (who never told fibs,
however they might be to his advantage) what was the real hour of the
morning, he broke out into a volley of bad language, which we will not
repeat here, but by which he gave Dobbin to understand that he would
jeopardy his soul if he got up at that moment, that the Major might go
and be hanged, that he would not travel with Dobbin, and that it was
most unkind and ungentlemanlike to disturb a man out of his sleep in
that way; on which the discomfited Major was obliged to retreat,
leaving Jos to resume his interrupted slumbers.

The chaise came up presently, and the Major would wait no longer.

If he had been an English nobleman travelling on a pleasure tour, or a
newspaper courier bearing dispatches (government messages are generally
carried much more quietly), he could not have travelled more quickly.
The post-boys wondered at the fees he flung amongst them. How happy and
green the country looked as the chaise whirled rapidly from mile-stone
to mile-stone, through neat country towns where landlords came out to
welcome him with smiles and bows; by pretty roadside inns, where the
signs hung on the elms, and horses and waggoners were drinking under
the chequered shadow of the trees; by old halls and parks; rustic
hamlets clustered round ancient grey churches---and through the charming
friendly English landscape.  Is there any in the world like it? To a
traveller returning home it looks so kind---it seems to shake hands with
you as you pass through it. Well, Major Dobbin passed through all this
from Southampton to London, and without noting much beyond the
milestones along the road.  You see he was so eager to see his parents
at Camberwell.

He grudged the time lost between Piccadilly and his old haunt at the
Slaughters', whither he drove faithfully. Long years had passed since
he saw it last, since he and George, as young men, had enjoyed many a
feast, and held many a revel there.  He had now passed into the stage
of old-fellow-hood.  His hair was grizzled, and many a passion and
feeling of his youth had grown grey in that interval. There, however,
stood the old waiter at the door, in the same greasy black suit, with
the same double chin and flaccid face, with the same huge bunch of
seals at his fob, rattling his money in his pockets as before, and
receiving the Major as if he had gone away only a week ago.  ``Put the
Major's things in twenty-three, that's his room,'' John said, exhibiting
not the least surprise.  ``Roast fowl for your dinner, I suppose.  You
ain't got married? They said you was married---the Scotch surgeon of
yours was here.  No, it was Captain Humby of the thirty-third, as was
quartered with the ---th in Injee. Like any warm water? What do you come
in a chay for---ain't the coach good enough?'' And with this, the
faithful waiter, who knew and remembered every officer who used the
house, and with whom ten years were but as yesterday, led the way up to
Dobbin's old room, where stood the great moreen bed, and the shabby
carpet, a thought more dingy, and all the old black furniture covered
with faded chintz, just as the Major recollected them in his youth.

He remembered George pacing up and down the room, and biting his nails,
and swearing that the Governor must come round, and that if he didn't,
he didn't care a straw, on the day before he was married. He could
fancy him walking in, banging the door of Dobbin's room, and his own
hard by---

``You ain't got young,'' John said, calmly surveying his friend of former
days.

Dobbin laughed.  ``Ten years and a fever don't make a man young, John,''
he said.  ``It is you that are always young---no, you are always old.''

``What became of Captain Osborne's widow?'' John said.  ``Fine young
fellow that.  Lord, how he used to spend his money.  He never came back
after that day he was marched from here.  He owes me three pound at
this minute.  Look here, I have it in my book.  'April 10, 1815,
Captain Osborne:  3 pounds.' I wonder whether his father would pay
me,'' and so saying, John of the Slaughters' pulled out the very morocco
pocket-book in which he had noted his loan to the Captain, upon a
greasy faded page still extant, with many other scrawled memoranda
regarding the bygone frequenters of the house.

Having inducted his customer into the room, John retired with perfect
calmness; and Major Dobbin, not without a blush and a grin at his own
absurdity, chose out of his kit the very smartest and most becoming
civil costume he possessed, and laughed at his own tanned face and grey
hair, as he surveyed them in the dreary little toilet-glass on the
dressing-table.

``I'm glad old John didn't forget me,'' he thought. ``She'll know me, too,
I hope.'' And he sallied out of the inn, bending his steps once more in
the direction of Brompton.

Every minute incident of his last meeting with Amelia was present to
the constant man's mind as he walked towards her house.  The arch and
the Achilles statue were up since he had last been in Piccadilly; a
hundred changes had occurred which his eye and mind vaguely noted.  He
began to tremble as he walked up the lane from Brompton, that
well-remembered lane leading to the street where she lived.  Was she
going to be married or not? If he were to meet her with the little
boy---Good God, what should he do? He saw a woman coming to him with a
child of five years old---was that she? He began to shake at the mere
possibility.  When he came up to the row of houses, at last, where she
lived, and to the gate, he caught hold of it and paused.  He might have
heard the thumping of his own heart. ``May God Almighty bless her,
whatever has happened,'' he thought to himself.  ``Psha!  she may be gone
from here,'' he said and went in through the gate.

The window of the parlour which she used to occupy was open, and there
were no inmates in the room.  The Major thought he recognized the
piano, though, with the picture over it, as it used to be in former
days, and his perturbations were renewed.  Mr.\ Clapp's brass plate was
still on the door, at the knocker of which Dobbin performed a summons.

A buxom-looking lass of sixteen, with bright eyes and purple cheeks,
came to answer the knock and looked hard at the Major as he leant back
against the little porch.

He was as pale as a ghost and could hardly falter out the words---``Does
Mrs.\ Osborne live here?''

She looked him hard in the face for a moment---and then turning white
too---said, ``Lord bless me---it's Major Dobbin.'' She held out both her
hands shaking---``Don't you remember me?'' she said.  ``I used to call you
Major Sugarplums.'' On which, and I believe it was for the first time
that he ever so conducted himself in his life, the Major took the girl
in his arms and kissed her. She began to laugh and cry hysterically,
and calling out ``Ma, Pa!'' with all her voice, brought up those worthy
people, who had already been surveying the Major from the casement of
the ornamental kitchen, and were astonished to find their daughter in
the little passage in the embrace of a great tall man in a blue
frock-coat and white duck trousers.

``I'm an old friend,'' he said---not without blushing though.  ``Don't you
remember me, Mrs.\ Clapp, and those good cakes you used to make for tea?
Don't you recollect me, Clapp? I'm George's godfather, and just come
back from India.'' A great shaking of hands ensued---Mrs.\ Clapp was
greatly affected and delighted; she called upon heaven to interpose a
vast many times in that passage.

The landlord and landlady of the house led the worthy Major into the
Sedleys' room (whereof he remembered every single article of furniture,
from the old brass ornamented piano, once a natty little instrument,
Stothard maker, to the screens and the alabaster miniature tombstone,
in the midst of which ticked Mr.\ Sedley's gold watch), and there, as he
sat down in the lodger's vacant arm-chair, the father, the mother, and
the daughter, with a thousand ejaculatory breaks in the narrative,
informed Major Dobbin of what we know already, but of particulars in
Amelia's history of which he was not aware---namely of Mrs.\ Sedley's
death, of George's reconcilement with his grandfather Osborne, of the
way in which the widow took on at leaving him, and of other particulars
of her life. Twice or thrice he was going to ask about the marriage
question, but his heart failed him. He did not care to lay it bare to
these people.  Finally, he was informed that Mrs.\ O.  was gone to walk
with her pa in Kensington Gardens, whither she always went with the old
gentleman (who was very weak and peevish now, and led her a sad life,
though she behaved to him like an angel, to be sure), of a fine
afternoon, after dinner.

``I'm very much pressed for time,'' the Major said, ``and have business
to-night of importance.  I should like to see Mrs.\ Osborne tho'.
Suppose Miss Polly would come with me and show me the way?''

Miss Polly was charmed and astonished at this proposal.  She knew the
way.  She would show Major Dobbin.  She had often been with Mr.\ Sedley
when Mrs.\ O. was gone---was gone Russell Square way---and knew the bench
where he liked to sit.  She bounced away to her apartment and appeared
presently in her best bonnet and her mamma's yellow shawl and large
pebble brooch, of which she assumed the loan in order to make herself a
worthy companion for the Major.

That officer, then, in his blue frock-coat and buckskin gloves, gave
the young lady his arm, and they walked away very gaily.  He was glad
to have a friend at hand for the scene which he dreaded somehow.  He
asked a thousand more questions from his companion about Amelia:  his
kind heart grieved to think that she should have had to part with her
son.  How did she bear it? Did she see him often? Was Mr.\ Sedley pretty
comfortable now in a worldly point of view? Polly answered all these
questions of Major Sugarplums to the very best of her power.

And in the midst of their walk an incident occurred which, though very
simple in its nature, was productive of the greatest delight to Major
Dobbin.  A pale young man with feeble whiskers and a stiff white
neckcloth came walking down the lane, en sandwich---having a lady, that
is, on each arm.  One was a tall and commanding middle-aged female,
with features and a complexion similar to those of the clergyman of the
Church of England by whose side she marched, and the other a stunted
little woman with a dark face, ornamented by a fine new bonnet and
white ribbons, and in a smart pelisse, with a rich gold watch in the
midst of her person.  The gentleman, pinioned as he was by these two
ladies, carried further a parasol, shawl, and basket, so that his arms
were entirely engaged, and of course he was unable to touch his hat in
acknowledgement of the curtsey with which Miss Mary Clapp greeted him.

He merely bowed his head in reply to her salutation, which the two
ladies returned with a patronizing air, and at the same time looking
severely at the individual in the blue coat and bamboo cane who
accompanied Miss Polly.

``Who's that?'' asked the Major, amused by the group, and after he had
made way for the three to pass up the lane.  Mary looked at him rather
roguishly.

``That is our curate, the Reverend Mr.\ Binny (a twitch from Major
Dobbin), and his sister Miss B.  Lord bless us, how she did use to
worret us at Sunday-school; and the other lady, the little one with a
cast in her eye and the handsome watch, is Mrs.\ Binny---Miss Grits that
was; her pa was a grocer, and kept the Little Original Gold Tea Pot in
Kensington Gravel Pits.  They were married last month, and are just
come back from Margate.  She's five thousand pound to her fortune; but
her and Miss B., who made the match, have quarrelled already.''

If the Major had twitched before, he started now, and slapped the
bamboo on the ground with an emphasis which made Miss Clapp cry, ``Law,''
and laugh too.  He stood for a moment, silent, with open mouth, looking
after the retreating young couple, while Miss Mary told their history;
but he did not hear beyond the announcement of the reverend gentleman's
marriage; his head was swimming with felicity.  After this rencontre he
began to walk double quick towards the place of his destination---and
yet they were too soon (for he was in a great tremor at the idea of a
meeting for which he had been longing any time these ten
years)---through the Brompton lanes, and entering at the little old
portal in Kensington Garden wall.

``There they are,'' said Miss Polly, and she felt him again start back on
her arm.  She was a confidante at once of the whole business. She knew
the story as well as if she had read it in one of her favourite
novel-books---Fatherless Fanny, or the Scottish Chiefs.

``Suppose you were to run on and tell her,'' the Major said.  Polly ran
forward, her yellow shawl streaming in the breeze.

Old Sedley was seated on a bench, his handkerchief placed over his
knees, prattling away, according to his wont, with some old story about
old times to which Amelia had listened and awarded a patient smile many
a time before.  She could of late think of her own affairs, and smile
or make other marks of recognition of her father's stories, scarcely
hearing a word of the old man's tales. As Mary came bouncing along, and
Amelia caught sight of her, she started up from her bench.  Her first
thought was that something had happened to Georgy, but the sight of the
messenger's eager and happy face dissipated that fear in the timorous
mother's bosom.

``News!  News!'' cried the emissary of Major Dobbin. ``He's come!  He's
come!''

``Who is come?'' said Emmy, still thinking of her son.

``Look there,'' answered Miss Clapp, turning round and pointing; in which
direction Amelia looking, saw Dobbin's lean figure and long shadow
stalking across the grass.  Amelia started in her turn, blushed up,
and, of course, began to cry.  At all this simple little creature's
fetes, the grandes eaux were accustomed to play. He looked at her---oh,
how fondly---as she came running towards him, her hands before her,
ready to give them to him.  She wasn't changed. She was a little pale,
a little stouter in figure.  Her eyes were the same, the kind trustful
eyes.  There were scarce three lines of silver in her soft brown hair.
She gave him both her hands as she looked up flushing and smiling
through her tears into his honest homely face.  He took the two little
hands between his two and held them there.  He was speechless for a
moment.  Why did he not take her in his arms and swear that he would
never leave her? She must have yielded:  she could not but have obeyed
him.

``I---I've another arrival to announce,'' he said after a pause.

``Mrs.\ Dobbin?'' Amelia said, making a movement back---why didn't he speak?

``No,'' he said, letting her hands go:  ``Who has told you those lies? I
mean, your brother Jos came in the same ship with me, and is come home
to make you all happy.''

``Papa, Papa!'' Emmy cried out, ``here are news!  My brother is in
England.  He is come to take care of you. Here is Major Dobbin.''

Mr.\ Sedley started up, shaking a great deal and gathering up his
thoughts.  Then he stepped forward and made an old-fashioned bow to the
Major, whom he called Mr.\ Dobbin, and hoped his worthy father, Sir
William, was quite well.  He proposed to call upon Sir William, who had
done him the honour of a visit a short time ago.  Sir William had not
called upon the old gentleman for eight years---it was that visit he was
thinking of returning.

``He is very much shaken,'' Emmy whispered as Dobbin went up and
cordially shook hands with the old man.

Although he had such particular business in London that evening, the
Major consented to forego it upon Mr.\ Sedley's invitation to him to
come home and partake of tea.  Amelia put her arm under that of her
young friend with the yellow shawl and headed the party on their return
homewards, so that Mr.\ Sedley fell to Dobbin's share. The old man
walked very slowly and told a number of ancient histories about himself
and his poor Bessy, his former prosperity, and his bankruptcy.  His
thoughts, as is usual with failing old men, were quite in former times.
The present, with the exception of the one catastrophe which he felt,
he knew little about.  The Major was glad to let him talk on.  His eyes
were fixed upon the figure in front of him---the dear little figure
always present to his imagination and in his prayers, and visiting his
dreams wakeful or slumbering.

Amelia was very happy, smiling, and active all that evening, performing
her duties as hostess of the little entertainment with the utmost grace
and propriety, as Dobbin thought.  His eyes followed her about as they
sat in the twilight.  How many a time had he longed for that moment and
thought of her far away under hot winds and in weary marches, gentle
and happy, kindly ministering to the wants of old age, and decorating
poverty with sweet submission---as he saw her now.  I do not say that
his taste was the highest, or that it is the duty of great intellects
to be content with a bread-and-butter paradise, such as sufficed our
simple old friend; but his desires were of this sort, whether for good
or bad, and, with Amelia to help him, he was as ready to drink as many
cups of tea as Doctor Johnson.

Amelia seeing this propensity, laughingly encouraged it and looked
exceedingly roguish as she administered to him cup after cup.  It is
true she did not know that the Major had had no dinner and that the
cloth was laid for him at the Slaughters', and a plate laid thereon to
mark that the table was retained, in that very box in which the Major
and George had sat many a time carousing, when she was a child just
come home from Miss Pinkerton's school.

The first thing Mrs.\ Osborne showed the Major was Georgy's miniature,
for which she ran upstairs on her arrival at home.  It was not half
handsome enough of course for the boy, but wasn't it noble of him to
think of bringing it to his mother? Whilst her papa was awake she did
not talk much about Georgy.  To hear about Mr.\ Osborne and Russell
Square was not agreeable to the old man, who very likely was
unconscious that he had been living for some months past mainly on the
bounty of his richer rival, and lost his temper if allusion was made to
the other.

Dobbin told him all, and a little more perhaps than all, that had
happened on board the Ramchunder, and exaggerated Jos's benevolent
dispositions towards his father and resolution to make him comfortable
in his old days.  The truth is that during the voyage the Major had
impressed this duty most strongly upon his fellow-passenger and
extorted promises from him that he would take charge of his sister and
her child.  He soothed Jos's irritation with regard to the bills which
the old gentleman had drawn upon him, gave a laughing account of his
own sufferings on the same score and of the famous consignment of wine
with which the old man had favoured him, and brought Mr.\ Jos, who was
by no means an ill-natured person when well-pleased and moderately
flattered, to a very good state of feeling regarding his relatives in
Europe.

And in fine I am ashamed to say that the Major stretched the truth so
far as to tell old Mr.\ Sedley that it was mainly a desire to see his
parent which brought Jos once more to Europe.

At his accustomed hour Mr.\ Sedley began to doze in his chair, and then
it was Amelia's opportunity to commence her conversation, which she did
with great eagerness---it related exclusively to Georgy.  She did not
talk at all about her own sufferings at breaking from him, for indeed,
this worthy woman, though she was half-killed by the separation from
the child, yet thought it was very wicked in her to repine at losing
him; but everything concerning him, his virtues, talents, and
prospects, she poured out.  She described his angelic beauty; narrated
a hundred instances of his generosity and greatness of mind whilst
living with her; how a Royal Duchess had stopped and admired him in
Kensington Gardens; how splendidly he was cared for now, and how he had
a groom and a pony; what quickness and cleverness he had, and what a
prodigiously well-read and delightful person the Reverend Lawrence Veal
was, George's master.  ``He knows EVERYTHING,'' Amelia said.  ``He has the
most delightful parties.  You who are so learned yourself, and have
read so much, and are so clever and accomplished---don't shake your head
and say no---HE always used to say you were---you will be charmed with
Mr.\ Veal's parties. The last Tuesday in every month.  He says there is
no place in the bar or the senate that Georgy may not aspire to.  Look
here,'' and she went to the piano-drawer and drew out a theme of
Georgy's composition.  This great effort of genius, which is still in
the possession of George's mother, is as follows:

On Selfishness---Of all the vices which degrade the human character,
Selfishness is the most odious and contemptible.  An undue love of Self
leads to the most monstrous crimes and occasions the greatest
misfortunes both in States and Families.  As a selfish man will
impoverish his family and often bring them to ruin, so a selfish king
brings ruin on his people and often plunges them into war.

Example:  The selfishness of Achilles, as remarked by the poet Homer,
occasioned a thousand woes to the Greeks---muri Achaiois alge
etheke---(Hom. Il. A. 2). The selfishness of the late Napoleon Bonaparte
occasioned innumerable wars in Europe and caused him to perish,
himself, in a miserable island---that of Saint Helena in the Atlantic
Ocean.

We see by these examples that we are not to consult our own interest
and ambition, but that we are to consider the interests of others as
well as our own.

George S.  Osborne Athene House, 24 April, 1827

``Think of him writing such a hand, and quoting Greek too, at his age,''
the delighted mother said.  ``Oh, William,'' she added, holding out her
hand to the Major, ``what a treasure Heaven has given me in that boy!
He is the comfort of my life---and he is the image of---of him that's
gone!''

``Ought I to be angry with her for being faithful to him?'' William
thought.  ``Ought I to be jealous of my friend in the grave, or hurt
that such a heart as Amelia's can love only once and for ever? Oh,
George, George, how little you knew the prize you had, though.'' This
sentiment passed rapidly through William's mind as he was holding
Amelia's hand, whilst the handkerchief was veiling her eyes.

``Dear friend,'' she said, pressing the hand which held hers, ``how good,
how kind you always have been to me! See!  Papa is stirring. You will
go and see Georgy tomorrow, won't you?''

``Not to-morrow,'' said poor old Dobbin.  ``I have business.'' He did not
like to own that he had not as yet been to his parents' and his dear
sister Anne---a remissness for which I am sure every well-regulated
person will blame the Major.  And presently he took his leave, leaving
his address behind him for Jos, against the latter's arrival.  And so
the first day was over, and he had seen her.

When he got back to the Slaughters', the roast fowl was of course cold,
in which condition he ate it for supper.  And knowing what early hours
his family kept, and that it would be needless to disturb their
slumbers at so late an hour, it is on record, that Major Dobbin treated
himself to half-price at the Haymarket Theatre that evening, where let
us hope he enjoyed himself.



\chapter{The Old Piano}

The Major's visit left old John Sedley in a great state of agitation
and excitement.  His daughter could not induce him to settle down to
his customary occupations or amusements that night.  He passed the
evening fumbling amongst his boxes and desks, untying his papers with
trembling hands, and sorting and arranging them against Jos's arrival.
He had them in the greatest order---his tapes and his files, his
receipts, and his letters with lawyers and correspondents; the
documents relative to the wine project (which failed from a most
unaccountable accident, after commencing with the most splendid
prospects), the coal project (which only a want of capital prevented
from becoming the most successful scheme ever put before the public),
the patent saw-mills and sawdust consolidation project, \&c., \&c.  All
night, until a very late hour, he passed in the preparation of these
documents, trembling about from one room to another, with a quivering
candle and shaky hands.  Here's the wine papers, here's the sawdust,
here's the coals; here's my letters to Calcutta and Madras, and replies
from Major Dobbin, C.B., and Mr.\ Joseph Sedley to the same.  ``He shall
find no irregularity about ME, Emmy,'' the old gentleman said.

Emmy smiled.  ``I don't think Jos will care about seeing those papers,
Papa,'' she said.

``You don't know anything about business, my dear,'' answered the sire,
shaking his head with an important air.  And it must be confessed that
on this point Emmy was very ignorant, and that is a pity some people
are so knowing.  All these twopenny documents arranged on a side table,
old Sedley covered them carefully over with a clean bandanna
handkerchief (one out of Major Dobbin's lot) and enjoined the maid and
landlady of the house, in the most solemn way, not to disturb those
papers, which were arranged for the arrival of Mr.\ Joseph Sedley the
next morning, ``Mr.\ Joseph Sedley of the Honourable East India Company's
Bengal Civil Service.''

Amelia found him up very early the next morning, more eager, more
hectic, and more shaky than ever.  ``I didn't sleep much, Emmy, my
dear,'' he said.  ``I was thinking of my poor Bessy.  I wish she was
alive, to ride in Jos's carriage once again.  She kept her own and
became it very well.'' And his eyes filled with tears, which trickled
down his furrowed old face.  Amelia wiped them away, and smilingly
kissed him, and tied the old man's neckcloth in a smart bow, and put
his brooch into his best shirt frill, in which, in his Sunday suit of
mourning, he sat from six o'clock in the morning awaiting the arrival
of his son.

However, when the postman made his appearance, the little party were
put out of suspense by the receipt of a letter from Jos to his sister,
who announced that he felt a little fatigued after his voyage, and
should not be able to move on that day, but that he would leave
Southampton early the next morning and be with his father and mother at
evening.  Amelia, as she read out the letter to her father, paused over
the latter word; her brother, it was clear, did not know what had
happened in the family. Nor could he, for the fact is that, though the
Major rightly suspected that his travelling companion never would be
got into motion in so short a space as twenty-four hours, and would
find some excuse for delaying, yet Dobbin had not written to Jos to
inform him of the calamity which had befallen the Sedley family, being
occupied in talking with Amelia until long after post-hour.

There are some splendid tailors' shops in the High Street of
Southampton, in the fine plate-glass windows of which hang gorgeous
waistcoats of all sorts, of silk and velvet, and gold and crimson, and
pictures of the last new fashions, in which those wonderful gentlemen
with quizzing glasses, and holding on to little boys with the exceeding
large eyes and curly hair, ogle ladies in riding habits prancing by the
Statue of Achilles at Apsley House.  Jos, although provided with some
of the most splendid vests that Calcutta could furnish, thought he
could not go to town until he was supplied with one or two of these
garments, and selected a crimson satin, embroidered with gold
butterflies, and a black and red velvet tartan with white stripes and a
rolling collar, with which, and a rich blue satin stock and a gold pin,
consisting of a five-barred gate with a horseman in pink enamel jumping
over it, he thought he might make his entry into London with some
dignity.  For Jos's former shyness and blundering blushing timidity had
given way to a more candid and courageous self-assertion of his worth.
``I don't care about owning it,'' Waterloo Sedley would say to his
friends, ``I am a dressy man''; and though rather uneasy if the ladies
looked at him at the Government House balls, and though he blushed and
turned away alarmed under their glances, it was chiefly from a dread
lest they should make love to him that he avoided them, being averse to
marriage altogether.  But there was no such swell in Calcutta as
Waterloo Sedley, I have heard say, and he had the handsomest turn-out,
gave the best bachelor dinners, and had the finest plate in the whole
place.

To make these waistcoats for a man of his size and dignity took at
least a day, part of which he employed in hiring a servant to wait upon
him and his native and in instructing the agent who cleared his
baggage, his boxes, his books, which he never read, his chests of
mangoes, chutney, and curry-powders, his shawls for presents to people
whom he didn't know as yet, and the rest of his Persicos apparatus.

At length, he drove leisurely to London on the third day and in the new
waistcoat, the native, with chattering teeth, shuddering in a shawl on
the box by the side of the new European servant; Jos puffing his pipe
at intervals within and looking so majestic that the little boys cried
Hooray, and many people thought he must be a Governor-General.  HE, I
promise, did not decline the obsequious invitation of the landlords to
alight and refresh himself in the neat country towns.  Having partaken
of a copious breakfast, with fish, and rice, and hard eggs, at
Southampton, he had so far rallied at Winchester as to think a glass of
sherry necessary.  At Alton he stepped out of the carriage at his
servant's request and imbibed some of the ale for which the place is
famous.  At Farnham he stopped to view the Bishop's Castle and to
partake of a light dinner of stewed eels, veal cutlets, and French
beans, with a bottle of claret.  He was cold over Bagshot Heath, where
the native chattered more and more, and Jos Sahib took some
brandy-and-water; in fact, when he drove into town he was as full of
wine, beer, meat, pickles, cherry-brandy, and tobacco as the steward's
cabin of a steam-packet. It was evening when his carriage thundered up
to the little door in Brompton, whither the affectionate fellow drove
first, and before hieing to the apartments secured for him by Mr.\ %
Dobbin at the Slaughters'.

All the faces in the street were in the windows; the little maidservant
flew to the wicket-gate; the Mesdames Clapp looked out from the
casement of the ornamented kitchen; Emmy, in a great flutter, was in
the passage among the hats and coats; and old Sedley in the parlour
inside, shaking all over.  Jos descended from the post-chaise and down
the creaking swaying steps in awful state, supported by the new valet
from Southampton and the shuddering native, whose brown face was now
livid with cold and of the colour of a turkey's gizzard.  He created an
immense sensation in the passage presently, where Mrs.\ and Miss Clapp,
coming perhaps to listen at the parlour door, found Loll Jewab shaking
upon the hall-bench under the coats, moaning in a strange piteous way,
and showing his yellow eyeballs and white teeth.

For, you see, we have adroitly shut the door upon the meeting between
Jos and the old father and the poor little gentle sister inside.  The
old man was very much affected; so, of course, was his daughter; nor
was Jos without feeling.  In that long absence of ten years, the most
selfish will think about home and early ties. Distance sanctifies both.
Long brooding over those lost pleasures exaggerates their charm and
sweetness.  Jos was unaffectedly glad to see and shake the hand of his
father, between whom and himself there had been a coolness---glad to see
his little sister, whom he remembered so pretty and smiling, and pained
at the alteration which time, grief, and misfortune had made in the
shattered old man.  Emmy had come out to the door in her black clothes
and whispered to him of her mother's death, and not to speak of it to
their father. There was no need of this caution, for the elder Sedley
himself began immediately to speak of the event, and prattled about it,
and wept over it plenteously. It shocked the Indian not a little and
made him think of himself less than the poor fellow was accustomed to
do.

The result of the interview must have been very satisfactory, for when
Jos had reascended his post-chaise and had driven away to his hotel,
Emmy embraced her father tenderly, appealing to him with an air of
triumph, and asking the old man whether she did not always say that her
brother had a good heart?

Indeed, Joseph Sedley, affected by the humble position in which he
found his relations, and in the expansiveness and overflowing of heart
occasioned by the first meeting, declared that they should never suffer
want or discomfort any more, that he was at home for some time at any
rate, during which his house and everything he had should be theirs:
and that Amelia would look very pretty at the head of his table---until
she would accept one of her own.

She shook her head sadly and had, as usual, recourse to the waterworks.
She knew what he meant.  She and her young confidante, Miss Mary, had
talked over the matter most fully, the very night of the Major's visit,
beyond which time the impetuous Polly could not refrain from talking of
the discovery which she had made, and describing the start and tremor
of joy by which Major Dobbin betrayed himself when Mr.\ Binny passed
with his bride and the Major learned that he had no longer a rival to
fear.  ``Didn't you see how he shook all over when you asked if he was
married and he said, 'Who told you those lies?' Oh, M'am,'' Polly said,
``he never kept his eyes off you, and I'm sure he's grown grey athinking
of you.''

But Amelia, looking up at her bed, over which hung the portraits of her
husband and son, told her young protegee never, never, to speak on that
subject again; that Major Dobbin had been her husband's dearest friend
and her own and George's most kind and affectionate guardian; that she
loved him as a brother---but that a woman who had been married to such
an angel as that, and she pointed to the wall, could never think of any
other union.  Poor Polly sighed:  she thought what she should do if
young Mr.\ Tomkins, at the surgery, who always looked at her so at
church, and who, by those mere aggressive glances had put her timorous
little heart into such a flutter that she was ready to surrender at
once,---what she should do if he were to die? She knew he was
consumptive, his cheeks were so red and he was so uncommon thin in the
waist.

Not that Emmy, being made aware of the honest Major's passion, rebuffed
him in any way, or felt displeased with him.  Such an attachment from
so true and loyal a gentleman could make no woman angry. Desdemona was
not angry with Cassio, though there is very little doubt she saw the
Lieutenant's partiality for her (and I for my part believe that many
more things took place in that sad affair than the worthy Moorish
officer ever knew of); why, Miranda was even very kind to Caliban, and
we may be pretty sure for the same reason. Not that she would encourage
him in the least---the poor uncouth monster---of course not.  No more
would Emmy by any means encourage her admirer, the Major.  She would
give him that friendly regard, which so much excellence and fidelity
merited; she would treat him with perfect cordiality and frankness
until he made his proposals, and THEN it would be time enough for her
to speak and to put an end to hopes which never could be realized.

She slept, therefore, very soundly that evening, after the conversation
with Miss Polly, and was more than ordinarily happy, in spite of Jos's
delaying.  ``I am glad he is not going to marry that Miss O'Dowd,'' she
thought. ``Colonel O'Dowd never could have a sister fit for such an
accomplished man as Major William.'' Who was there amongst her little
circle who would make him a good wife? Not Miss Binny, she was too old
and ill-tempered; Miss Osborne? too old too. Little Polly was too
young. Mrs.\ Osborne could not find anybody to suit the Major before she
went to sleep.

The same morning brought Major Dobbin a letter to the Slaughters'
Coffee-house from his friend at Southampton, begging dear Dob to excuse
Jos for being in a rage when awakened the day before (he had a
confounded headache, and was just in his first sleep), and entreating
Dob to engage comfortable rooms at the Slaughters' for Mr.\ Sedley and
his servants.  The Major had become necessary to Jos during the voyage.
He was attached to him, and hung upon him.  The other passengers were
away to London. Young Ricketts and little Chaffers went away on the
coach that day---Ricketts on the box, and taking the reins from Botley;
the Doctor was off to his family at Portsea; Bragg gone to town to his
co-partners; and the first mate busy in the unloading of the
Ramchunder.  Mr.\ Joe was very lonely at Southampton, and got the
landlord of the George to take a glass of wine with him that day, at
the very hour at which Major Dobbin was seated at the table of his
father, Sir William, where his sister found out (for it was impossible
for the Major to tell fibs) that he had been to see Mrs.\ George Osborne.

Jos was so comfortably situated in St.\ Martin's Lane, he could enjoy
his hookah there with such perfect ease, and could swagger down to the
theatres, when minded, so agreeably, that, perhaps, he would have
remained altogether at the Slaughters' had not his friend, the Major,
been at his elbow.  That gentleman would not let the Bengalee rest
until he had executed his promise of having a home for Amelia and his
father.  Jos was a soft fellow in anybody's hands, Dobbin most active
in anybody's concerns but his own; the civilian was, therefore, an easy
victim to the guileless arts of this good-natured diplomatist and was
ready to do, to purchase, hire, or relinquish whatever his friend
thought fit.  Loll Jewab, of whom the boys about St.\ Martin's Lane
used to make cruel fun whenever he showed his dusky countenance in the
street, was sent back to Calcutta in the Lady Kicklebury East Indiaman,
in which Sir William Dobbin had a share, having previously taught Jos's
European the art of preparing curries, pilaus, and pipes.  It was a
matter of great delight and occupation to Jos to superintend the
building of a smart chariot which he and the Major ordered in the
neighbouring Long Acre:  and a pair of handsome horses were jobbed,
with which Jos drove about in state in the park, or to call upon his
Indian friends.  Amelia was not seldom by his side on these excursions,
when also Major Dobbin would be seen in the back seat of the carriage.
At other times old Sedley and his daughter took advantage of it, and
Miss Clapp, who frequently accompanied her friend, had great pleasure
in being recognized as she sat in the carriage, dressed in the famous
yellow shawl, by the young gentleman at the surgery, whose face might
commonly be seen over the window-blinds as she passed.

Shortly after Jos's first appearance at Brompton, a dismal scene,
indeed, took place at that humble cottage at which the Sedleys had
passed the last ten years of their life.  Jos's carriage (the temporary
one, not the chariot under construction) arrived one day and carried
off old Sedley and his daughter---to return no more.  The tears that
were shed by the landlady and the landlady's daughter at that event
were as genuine tears of sorrow as any that have been outpoured in the
course of this history. In their long acquaintanceship and intimacy
they could not recall a harsh word that had been uttered by Amelia. She
had been all sweetness and kindness, always thankful, always gentle,
even when Mrs.\ Clapp lost her own temper and pressed for the rent.
When the kind creature was going away for good and all, the landlady
reproached herself bitterly for ever having used a rough expression to
her---how she wept, as they stuck up with wafers on the window, a paper
notifying that the little rooms so long occupied were to let!  They
never would have such lodgers again, that was quite clear.  After-life
proved the truth of this melancholy prophecy, and Mrs.\ Clapp revenged
herself for the deterioration of mankind by levying the most savage
contributions upon the tea-caddies and legs of mutton of her
locataires.  Most of them scolded and grumbled; some of them did not
pay; none of them stayed. The landlady might well regret those old, old
friends, who had left her.

As for Miss Mary, her sorrow at Amelia's departure was such as I shall
not attempt to depict.  From childhood upwards she had been with her
daily and had attached herself so passionately to that dear good lady
that when the grand barouche came to carry her off into splendour, she
fainted in the arms of her friend, who was indeed scarcely less
affected than the good-natured girl.  Amelia loved her like a daughter.
During eleven years the girl had been her constant friend and
associate.  The separation was a very painful one indeed to her.  But
it was of course arranged that Mary was to come and stay often at the
grand new house whither Mrs.\ Osborne was going, and where Mary was sure
she would never be so happy as she had been in their humble cot, as
Miss Clapp called it, in the language of the novels which she loved.

Let us hope she was wrong in her judgement.  Poor Emmy's days of
happiness had been very few in that humble cot.  A gloomy Fate had
oppressed her there.  She never liked to come back to the house after
she had left it, or to face the landlady who had tyrannized over her
when ill-humoured and unpaid, or when pleased had treated her with a
coarse familiarity scarcely less odious. Her servility and fulsome
compliments when Emmy was in prosperity were not more to that lady's
liking.  She cast about notes of admiration all over the new house,
extolling every article of furniture or ornament; she fingered Mrs.\ %
Osborne's dresses and calculated their price. Nothing could be too good
for that sweet lady, she vowed and protested.  But in the vulgar
sycophant who now paid court to her, Emmy always remembered the coarse
tyrant who had made her miserable many a time, to whom she had been
forced to put up petitions for time, when the rent was overdue; who
cried out at her extravagance if she bought delicacies for her ailing
mother or father; who had seen her humble and trampled upon her.

Nobody ever heard of these griefs, which had been part of our poor
little woman's lot in life.  She kept them secret from her father,
whose improvidence was the cause of much of her misery.  She had to
bear all the blame of his misdoings, and indeed was so utterly gentle
and humble as to be made by nature for a victim.

I hope she is not to suffer much more of that hard usage.  And, as in
all griefs there is said to be some consolation, I may mention that
poor Mary, when left at her friend's departure in a hysterical
condition, was placed under the medical treatment of the young fellow
from the surgery, under whose care she rallied after a short period.
Emmy, when she went away from Brompton, endowed Mary with every article
of furniture that the house contained, only taking away her pictures
(the two pictures over the bed) and her piano---that little old piano
which had now passed into a plaintive jingling old age, but which she
loved for reasons of her own.  She was a child when first she played on
it, and her parents gave it her.  It had been given to her again since,
as the reader may remember, when her father's house was gone to ruin
and the instrument was recovered out of the wreck.

Major Dobbin was exceedingly pleased when, as he was superintending the
arrangements of Jos's new house---which the Major insisted should be
very handsome and comfortable---the cart arrived from Brompton, bringing
the trunks and bandboxes of the emigrants from that village, and with
them the old piano.  Amelia would have it up in her sitting-room, a
neat little apartment on the second floor, adjoining her father's
chamber, and where the old gentleman sat commonly of evenings.

When the men appeared then bearing this old music-box, and Amelia gave
orders that it should be placed in the chamber aforesaid, Dobbin was
quite elated.  ``I'm glad you've kept it,'' he said in a very sentimental
manner.  ``I was afraid you didn't care about it.''

``I value it more than anything I have in the world,'' said Amelia.

``Do you, Amelia?'' cried the Major.  The fact was, as he had bought it
himself, though he never said anything about it, it never entered into
his head to suppose that Emmy should think anybody else was the
purchaser, and as a matter of course he fancied that she knew the gift
came from him.  ``Do you, Amelia?'' he said; and the question, the great
question of all, was trembling on his lips, when Emmy replied---

``Can I do otherwise?---did not he give it me?''

``I did not know,'' said poor old Dob, and his countenance fell.

Emmy did not note the circumstance at the time, nor take immediate heed
of the very dismal expression which honest Dobbin's countenance
assumed, but she thought of it afterwards.  And then it struck her,
with inexpressible pain and mortification too, that it was William who
was the giver of the piano, and not George, as she had fancied. It was
not George's gift; the only one which she had received from her lover,
as she thought---the thing she had cherished beyond all others---her
dearest relic and prize.  She had spoken to it about George; played his
favourite airs upon it; sat for long evening hours, touching, to the
best of her simple art, melancholy harmonies on the keys, and weeping
over them in silence. It was not George's relic.  It was valueless now.
The next time that old Sedley asked her to play, she said it was
shockingly out of tune, that she had a headache, that she couldn't play.

Then, according to her custom, she rebuked herself for her pettishness
and ingratitude and determined to make a reparation to honest William
for the slight she had not expressed to him, but had felt for his
piano. A few days afterwards, as they were seated in the drawing-room,
where Jos had fallen asleep with great comfort after dinner, Amelia
said with rather a faltering voice to Major Dobbin---

``I have to beg your pardon for something.''

``About what?'' said he.

``About---about that little square piano.  I never thanked you for it
when you gave it me, many, many years ago, before I was married.  I
thought somebody else had given it.  Thank you, William.'' She held out
her hand, but the poor little woman's heart was bleeding; and as for
her eyes, of course they were at their work.

But William could hold no more.  ``Amelia, Amelia,'' he said, ``I did buy
it for you.  I loved you then as I do now.  I must tell you.  I think I
loved you from the first minute that I saw you, when George brought me
to your house, to show me the Amelia whom he was engaged to.  You were
but a girl, in white, with large ringlets; you came down singing---do
you remember?---and we went to Vauxhall.  Since then I have thought of
but one woman in the world, and that was you.  I think there is no hour
in the day has passed for twelve years that I haven't thought of you.
I came to tell you this before I went to India, but you did not care,
and I hadn't the heart to speak.  You did not care whether I stayed or
went.''

``I was very ungrateful,'' Amelia said.

``No, only indifferent,'' Dobbin continued desperately. ``I have nothing
to make a woman to be otherwise.  I know what you are feeling now.  You
are hurt in your heart at the discovery about the piano, and that it
came from me and not from George.  I forgot, or I should never have
spoken of it so.  It is for me to ask your pardon for being a fool for
a moment, and thinking that years of constancy and devotion might have
pleaded with you.''

``It is you who are cruel now,'' Amelia said with some spirit. ``George is
my husband, here and in heaven.  How could I love any other but him? I
am his now as when you first saw me, dear William. It was he who told
me how good and generous you were, and who taught me to love you as a
brother.  Have you not been everything to me and my boy? Our dearest,
truest, kindest friend and protector? Had you come a few months sooner
perhaps you might have spared me that---that dreadful parting.  Oh, it
nearly killed me, William---but you didn't come, though I wished and
prayed for you to come, and they took him too away from me.  Isn't he a
noble boy, William? Be his friend still and mine''---and here her voice
broke, and she hid her face on his shoulder.

The Major folded his arms round her, holding her to him as if she was a
child, and kissed her head.  ``I will not change, dear Amelia,'' he said.
``I ask for no more than your love.  I think I would not have it
otherwise. Only let me stay near you and see you often.''

``Yes, often,'' Amelia said.  And so William was at liberty to look and
long---as the poor boy at school who has no money may sigh after the
contents of the tart-woman's tray.



\chapter{Returns to the Genteel World}

Good fortune now begins to smile upon Amelia.  We are glad to get her
out of that low sphere in which she has been creeping hitherto and
introduce her into a polite circle---not so grand and refined as that in
which our other female friend, Mrs.\ Becky, has appeared, but still
having no small pretensions to gentility and fashion.  Jos's friends
were all from the three presidencies, and his new house was in the
comfortable Anglo-Indian district of which Moira Place is the centre.
Minto Square, Great Clive Street, Warren Street, Hastings Street,
Ochterlony Place, Plassy Square, Assaye Terrace (``gardens'' was a
felicitous word not applied to stucco houses with asphalt terraces in
front, so early as 1827)---who does not know these respectable abodes of
the retired Indian aristocracy, and the quarter which Mr.\ Wenham calls
the Black Hole, in a word? Jos's position in life was not grand enough
to entitle him to a house in Moira Place, where none can live but
retired Members of Council, and partners of Indian firms (who break,
after having settled a hundred thousand pounds on their wives, and
retire into comparative penury to a country place and four thousand a
year); he engaged a comfortable house of a second- or third-rate order
in Gillespie Street, purchasing the carpets, costly mirrors, and
handsome and appropriate planned furniture by Seddons from the
assignees of Mr.\ Scape, lately admitted partner into the great Calcutta
House of Fogle, Fake, and Cracksman, in which poor Scape had embarked
seventy thousand pounds, the earnings of a long and honourable life,
taking Fake's place, who retired to a princely park in Sussex (the
Fogles have been long out of the firm, and Sir Horace Fogle is about to
be raised to the peerage as Baron Bandanna)---admitted, I say, partner
into the great agency house of Fogle and Fake two years before it
failed for a million and plunged half the Indian public into misery and
ruin.

Scape, ruined, honest, and broken-hearted at sixty-five years of age,
went out to Calcutta to wind up the affairs of the house. Walter Scape
was withdrawn from Eton and put into a merchant's house.  Florence
Scape, Fanny Scape, and their mother faded away to Boulogne, and will
be heard of no more.  To be brief, Jos stepped in and bought their
carpets and sideboards and admired himself in the mirrors which had
reflected their kind handsome faces.  The Scape tradesmen, all
honourably paid, left their cards, and were eager to supply the new
household.  The large men in white waistcoats who waited at Scape's
dinners, greengrocers, bank-porters, and milkmen in their private
capacity, left their addresses and ingratiated themselves with the
butler.  Mr.\ Chummy, the chimney-purifier, who had swept the last three
families, tried to coax the butler and the boy under him, whose duty it
was to go out covered with buttons and with stripes down his trousers,
for the protection of Mrs.\ Amelia whenever she chose to walk abroad.

It was a modest establishment.  The butler was Jos's valet also, and
never was more drunk than a butler in a small family should be who has
a proper regard for his master's wine.  Emmy was supplied with a maid,
grown on Sir William Dobbin's suburban estate; a good girl, whose
kindness and humility disarmed Mrs.\ Osborne, who was at first terrified
at the idea of having a servant to wait upon herself, who did not in
the least know how to use one, and who always spoke to domestics with
the most reverential politeness.  But this maid was very useful in the
family, in dexterously tending old Mr.\ Sedley, who kept almost entirely
to his own quarter of the house and never mixed in any of the gay
doings which took place there.

Numbers of people came to see Mrs.\ Osborne.  Lady Dobbin and daughters
were delighted at her change of fortune, and waited upon her.  Miss
Osborne from Russell Square came in her grand chariot with the flaming
hammer-cloth emblazoned with the Leeds arms.  Jos was reported to be
immensely rich.  Old Osborne had no objection that Georgy should
inherit his uncle's property as well as his own. ``Damn it, we will make
a man of the feller,'' he said; ``and I'll see him in Parliament before I
die.  You may go and see his mother, Miss O., though I'll never set
eyes on her'':  and Miss Osborne came. Emmy, you may be sure, was very
glad to see her, and so be brought nearer to George.  That young fellow
was allowed to come much more frequently than before to visit his
mother.  He dined once or twice a week in Gillespie Street and bullied
the servants and his relations there, just as he did in Russell Square.

He was always respectful to Major Dobbin, however, and more modest in
his demeanour when that gentleman was present.  He was a clever lad and
afraid of the Major.  George could not help admiring his friend's
simplicity, his good humour, his various learning quietly imparted, his
general love of truth and justice.  He had met no such man as yet in
the course of his experience, and he had an instinctive liking for a
gentleman.  He hung fondly by his godfather's side, and it was his
delight to walk in the parks and hear Dobbin talk.  William told George
about his father, about India and Waterloo, about everything but
himself.  When George was more than usually pert and conceited, the
Major made jokes at him, which Mrs.\ Osborne thought very cruel.  One
day, taking him to the play, and the boy declining to go into the pit
because it was vulgar, the Major took him to the boxes, left him there,
and went down himself to the pit.  He had not been seated there very
long before he felt an arm thrust under his and a dandy little hand in
a kid glove squeezing his arm.  George had seen the absurdity of his
ways and come down from the upper region.  A tender laugh of
benevolence lighted up old Dobbin's face and eyes as he looked at the
repentant little prodigal.  He loved the boy, as he did everything that
belonged to Amelia.  How charmed she was when she heard of this
instance of George's goodness!  Her eyes looked more kindly on Dobbin
than they ever had done.  She blushed, he thought, after looking at him
so.

Georgy never tired of his praises of the Major to his mother.  ``I like
him, Mamma, because he knows such lots of things; and he ain't like old
Veal, who is always bragging and using such long words, don't you know?
The chaps call him 'Longtail' at school.  I gave him the name; ain't it
capital? But Dob reads Latin like English, and French and that; and
when we go out together he tells me stories about my Papa, and never
about himself; though I heard Colonel Buckler, at Grandpapa's, say that
he was one of the bravest officers in the army, and had distinguished
himself ever so much.  Grandpapa was quite surprised, and said, 'THAT
feller!  Why, I didn't think he could say Bo to a goose'---but I know he
could, couldn't he, Mamma?''

Emmy laughed:  she thought it was very likely the Major could do thus
much.

If there was a sincere liking between George and the Major, it must be
confessed that between the boy and his uncle no great love existed.
George had got a way of blowing out his cheeks, and putting his hands
in his waistcoat pockets, and saying, ``God bless my soul, you don't say
so,'' so exactly after the fashion of old Jos that it was impossible to
refrain from laughter.  The servants would explode at dinner if the
lad, asking for something which wasn't at table, put on that
countenance and used that favourite phrase.  Even Dobbin would shoot
out a sudden peal at the boy's mimicry.  If George did not mimic his
uncle to his face, it was only by Dobbin's rebukes and Amelia's
terrified entreaties that the little scapegrace was induced to desist.
And the worthy civilian being haunted by a dim consciousness that the
lad thought him an ass, and was inclined to turn him into ridicule,
used to be extremely timorous and, of course, doubly pompous and
dignified in the presence of Master Georgy.  When it was announced that
the young gentleman was expected in Gillespie Street to dine with his
mother, Mr.\ Jos commonly found that he had an engagement at the Club.
Perhaps nobody was much grieved at his absence.  On those days Mr.\ %
Sedley would commonly be induced to come out from his place of refuge
in the upper stories, and there would be a small family party, whereof
Major Dobbin pretty generally formed one.  He was the ami de la
maison---old Sedley's friend, Emmy's friend, Georgy's friend, Jos's
counsel and adviser. ``He might almost as well be at Madras for anything
WE see of him,'' Miss Ann Dobbin remarked at Camberwell.  Ah!  Miss Ann,
did it not strike you that it was not YOU whom the Major wanted to
marry?

Joseph Sedley then led a life of dignified otiosity such as became a
person of his eminence.  His very first point, of course, was to become
a member of the Oriental Club, where he spent his mornings in the
company of his brother Indians, where he dined, or whence he brought
home men to dine.

Amelia had to receive and entertain these gentlemen and their ladies.
From these she heard how soon Smith would be in Council; how many lacs
Jones had brought home with him, how Thomson's House in London had
refused the bills drawn by Thomson, Kibobjee, and Co., the Bombay
House, and how it was thought the Calcutta House must go too; how very
imprudent, to say the least of it, Mrs.\ Brown's conduct (wife of Brown
of the Ahmednuggur Irregulars) had been with young Swankey of the Body
Guard, sitting up with him on deck until all hours, and losing
themselves as they were riding out at the Cape; how Mrs.\ Hardyman had
had out her thirteen sisters, daughters of a country curate, the Rev:
Felix Rabbits, and married eleven of them, seven high up in the
service; how Hornby was wild because his wife would stay in Europe, and
Trotter was appointed Collector at Ummerapoora.  This and similar talk
took place at the grand dinners all round.  They had the same
conversation; the same silver dishes; the same saddles of mutton,
boiled turkeys, and entrees.  Politics set in a short time after
dessert, when the ladies retired upstairs and talked about their
complaints and their children.

Mutato nomine, it is all the same.  Don't the barristers' wives talk
about Circuit? Don't the soldiers' ladies gossip about the Regiment?
Don't the clergymen's ladies discourse about Sunday-schools and who
takes whose duty? Don't the very greatest ladies of all talk about that
small clique of persons to whom they belong? And why should our Indian
friends not have their own conversation?---only I admit it is slow for
the laymen whose fate it sometimes is to sit by and listen.

Before long Emmy had a visiting-book, and was driving about regularly
in a carriage, calling upon Lady Bludyer (wife of Major-General Sir
Roger Bludyer, K.C.B., Bengal Army); Lady Huff, wife of Sir G.  Huff,
Bombay ditto; Mrs.\ Pice, the Lady of Pice the Director, \&c.  We are not
long in using ourselves to changes in life.  That carriage came round
to Gillespie Street every day; that buttony boy sprang up and down from
the box with Emmy's and Jos's visiting-cards; at stated hours Emmy and
the carriage went for Jos to the Club and took him an airing; or,
putting old Sedley into the vehicle, she drove the old man round the
Regent's Park.  The lady's maid and the chariot, the visiting-book and
the buttony page, became soon as familiar to Amelia as the humble
routine of Brompton.  She accommodated herself to one as to the other.
If Fate had ordained that she should be a Duchess, she would even have
done that duty too.  She was voted, in Jos's female society, rather a
pleasing young person---not much in her, but pleasing, and that sort of
thing.

The men, as usual, liked her artless kindness and simple refined
demeanour.  The gallant young Indian dandies at home on furlough---immense
dandies these---chained and moustached---driving in tearing cabs,
the pillars of the theatres, living at West End hotels---nevertheless
admired Mrs.\ Osborne, liked to bow to her carriage in the park, and to
be admitted to have the honour of paying her a morning visit.  Swankey
of the Body Guard himself, that dangerous youth, and the greatest buck
of all the Indian army now on leave, was one day discovered by Major
Dobbin tete-a-tete with Amelia, and describing the sport of
pig-sticking to her with great humour and eloquence; and he spoke
afterwards of a d---d king's officer that's always hanging about the
house---a long, thin, queer-looking, oldish fellow---a dry fellow though,
that took the shine out of a man in the talking line.

Had the Major possessed a little more personal vanity he would have
been jealous of so dangerous a young buck as that fascinating Bengal
Captain.  But Dobbin was of too simple and generous a nature to have
any doubts about Amelia.  He was glad that the young men should pay her
respect, and that others should admire her.  Ever since her womanhood
almost, had she not been persecuted and undervalued? It pleased him to
see how kindness bought out her good qualities and how her spirits
gently rose with her prosperity.  Any person who appreciated her paid a
compliment to the Major's good judgement---that is, if a man may be
said to have good judgement who is under the influence of Love's
delusion.

After Jos went to Court, which we may be sure he did as a loyal subject
of his Sovereign (showing himself in his full court suit at the Club,
whither Dobbin came to fetch him in a very shabby old uniform) he who
had always been a staunch Loyalist and admirer of George IV, became
such a tremendous Tory and pillar of the State that he was for having
Amelia to go to a Drawing-room, too.  He somehow had worked himself up
to believe that he was implicated in the maintenance of the public
welfare and that the Sovereign would not be happy unless Jos Sedley and
his family appeared to rally round him at St.\ James's.

Emmy laughed.  ``Shall I wear the family diamonds, Jos?'' she said.

``I wish you would let me buy you some,'' thought the Major.  ``I should
like to see any that were too good for you.''



\chapter{In Which Two Lights are Put Out}

There came a day when the round of decorous pleasures and solemn
gaieties in which Mr.\ Jos Sedley's family indulged was interrupted by
an event which happens in most houses.  As you ascend the staircase of
your house from the drawing towards the bedroom floors, you may have
remarked a little arch in the wall right before you, which at once
gives light to the stair which leads from the second story to the third
(where the nursery and servants' chambers commonly are) and serves for
another purpose of utility, of which the undertaker's men can give you
a notion.  They rest the coffins upon that arch, or pass them through
it so as not to disturb in any unseemly manner the cold tenant
slumbering within the black ark.

That second-floor arch in a London house, looking up and down the well
of the staircase and commanding the main thoroughfare by which the
inhabitants are passing; by which cook lurks down before daylight to
scour her pots and pans in the kitchen; by which young master
stealthily ascends, having left his boots in the hall, and let himself
in after dawn from a jolly night at the Club; down which miss comes
rustling in fresh ribbons and spreading muslins, brilliant and
beautiful, and prepared for conquest and the ball; or Master Tommy
slides, preferring the banisters for a mode of conveyance, and
disdaining danger and the stair; down which the mother is fondly
carried smiling in her strong husband's arms, as he steps steadily step
by step, and followed by the monthly nurse, on the day when the medical
man has pronounced that the charming patient may go downstairs; up
which John lurks to bed, yawning, with a sputtering tallow candle, and
to gather up before sunrise the boots which are awaiting him in the
passages---that stair, up or down which babies are carried, old people
are helped, guests are marshalled to the ball, the parson walks to the
christening, the doctor to the sick-room, and the undertaker's men to
the upper floor---what a memento of Life, Death, and Vanity it is---that
arch and stair---if you choose to consider it, and sit on the landing,
looking up and down the well!  The doctor will come up to us too for
the last time there, my friend in motley.  The nurse will look in at
the curtains, and you take no notice---and then she will fling open the
windows for a little and let in the air.  Then they will pull down all
the front blinds of the house and live in the back rooms---then they
will send for the lawyer and other men in black, \&c.  Your comedy and
mine will have been played then, and we shall be removed, oh, how far,
from the trumpets, and the shouting, and the posture-making.  If we
are gentlefolks they will put hatchments over our late domicile, with
gilt cherubim, and mottoes stating that there is ``Quiet in Heaven.''
Your son will new furnish the house, or perhaps let it, and go into a
more modern quarter; your name will be among the ``Members Deceased'' in
the lists of your clubs next year. However much you may be mourned,
your widow will like to have her weeds neatly made---the cook will send
or come up to ask about dinner---the survivor will soon bear to look at
your picture over the mantelpiece, which will presently be deposed from
the place of honour, to make way for the portrait of the son who reigns.

Which of the dead are most tenderly and passionately deplored? Those
who love the survivors the least, I believe.  The death of a child
occasions a passion of grief and frantic tears, such as your end,
brother reader, will never inspire.  The death of an infant which
scarce knew you, which a week's absence from you would have caused to
forget you, will strike you down more than the loss of your closest
friend, or your first-born son---a man grown like yourself, with
children of his own.  We may be harsh and stern with Judah and
Simeon---our love and pity gush out for Benjamin, the little one. And if
you are old, as some reader of this may be or shall be old and rich, or
old and poor---you may one day be thinking for yourself---``These people
are very good round about me, but they won't grieve too much when I am
gone.  I am very rich, and they want my inheritance---or very poor, and
they are tired of supporting me.''

The period of mourning for Mrs.\ Sedley's death was only just concluded,
and Jos scarcely had had time to cast off his black and appear in the
splendid waistcoats which he loved, when it became evident to those
about Mr.\ Sedley that another event was at hand, and that the old man
was about to go seek for his wife in the dark land whither she had
preceded him.  ``The state of my father's health,'' Jos Sedley solemnly
remarked at the Club, ``prevents me from giving any LARGE parties this
season:  but if you will come in quietly at half-past six, Chutney, my
boy, and fake a homely dinner with one or two of the old set---I shall
be always glad to see you.'' So Jos and his acquaintances dined and
drank their claret among themselves in silence, whilst the sands of
life were running out in the old man's glass upstairs.  The
velvet-footed butler brought them their wine, and they composed
themselves to a rubber after dinner, at which Major Dobbin would
sometimes come and take a hand; and Mrs.\ Osborne would occasionally
descend, when her patient above was settled for the night, and had
commenced one of those lightly troubled slumbers which visit the pillow
of old age.

The old man clung to his daughter during this sickness.  He would take
his broths and medicines from scarcely any other hand.  To tend him
became almost the sole business of her life.  Her bed was placed close
by the door which opened into his chamber, and she was alive at the
slightest noise or disturbance from the couch of the querulous invalid.
Though, to do him justice, he lay awake many an hour, silent and
without stirring, unwilling to awaken his kind and vigilant nurse.

He loved his daughter with more fondness now, perhaps, than ever he had
done since the days of her childhood.  In the discharge of gentle
offices and kind filial duties, this simple creature shone most
especially.  ``She walks into the room as silently as a sunbeam,'' Mr.\ %
Dobbin thought as he saw her passing in and out from her father's room,
a cheerful sweetness lighting up her face as she moved to and fro,
graceful and noiseless.  When women are brooding over their children,
or busied in a sick-room, who has not seen in their faces those sweet
angelic beams of love and pity?

A secret feud of some years' standing was thus healed, and with a tacit
reconciliation.  In these last hours, and touched by her love and
goodness, the old man forgot all his grief against her, and wrongs
which he and his wife had many a long night debated:  how she had given
up everything for her boy; how she was careless of her parents in their
old age and misfortune, and only thought of the child; how absurdly and
foolishly, impiously indeed, she took on when George was removed from
her.  Old Sedley forgot these charges as he was making up his last
account, and did justice to the gentle and uncomplaining little martyr.
One night when she stole into his room, she found him awake, when the
broken old man made his confession.  ``Oh, Emmy, I've been thinking we
were very unkind and unjust to you,'' he said and put out his cold and
feeble hand to her. She knelt down and prayed by his bedside, as he did
too, having still hold of her hand.  When our turn comes, friend, may
we have such company in our prayers!

Perhaps as he was lying awake then, his life may have passed before
him---his early hopeful struggles, his manly successes and prosperity,
his downfall in his declining years, and his present helpless
condition---no chance of revenge against Fortune, which had had the
better of him---neither name nor money to bequeath---a spent-out,
bootless life of defeat and disappointment, and the end here! Which, I
wonder, brother reader, is the better lot, to die prosperous and
famous, or poor and disappointed? To have, and to be forced to yield;
or to sink out of life, having played and lost the game? That must be a
strange feeling, when a day of our life comes and we say, ``To-morrow,
success or failure won't matter much, and the sun will rise, and all
the myriads of mankind go to their work or their pleasure as usual, but
I shall be out of the turmoil.''

So there came one morning and sunrise when all the world got up and set
about its various works and pleasures, with the exception of old John
Sedley, who was not to fight with fortune, or to hope or scheme any
more, but to go and take up a quiet and utterly unknown residence in a
churchyard at Brompton by the side of his old wife.

Major Dobbin, Jos, and Georgy followed his remains to the grave, in a
black cloth coach.  Jos came on purpose from the Star and Garter at
Richmond, whither he retreated after the deplorable event.  He did not
care to remain in the house, with the---under the circumstances, you
understand.  But Emmy stayed and did her duty as usual.  She was bowed
down by no especial grief, and rather solemn than sorrowful.  She
prayed that her own end might be as calm and painless, and thought with
trust and reverence of the words which she had heard from her father
during his illness, indicative of his faith, his resignation, and his
future hope.

Yes, I think that will be the better ending of the two, after all.
Suppose you are particularly rich and well-to-do and say on that last
day, ``I am very rich; I am tolerably well known; I have lived all my
life in the best society, and thank Heaven, come of a most respectable
family.  I have served my King and country with honour. I was in
Parliament for several years, where, I may say, my speeches were
listened to and pretty well received. I don't owe any man a shilling:
on the contrary, I lent my old college friend, Jack Lazarus, fifty
pounds, for which my executors will not press him.  I leave my
daughters with ten thousand pounds apiece---very good portions for
girls; I bequeath my plate and furniture, my house in Baker Street,
with a handsome jointure, to my widow for her life; and my landed
property, besides money in the funds, and my cellar of well-selected
wine in Baker Street, to my son.  I leave twenty pound a year to my
valet; and I defy any man after I have gone to find anything against my
character.'' Or suppose, on the other hand, your swan sings quite a
different sort of dirge and you say, ``I am a poor blighted,
disappointed old fellow, and have made an utter failure through life.
I was not endowed either with brains or with good fortune, and confess
that I have committed a hundred mistakes and blunders. I own to having
forgotten my duty many a time.  I can't pay what I owe.  On my last bed
I lie utterly helpless and humble, and I pray forgiveness for my
weakness and throw myself, with a contrite heart, at the feet of the
Divine Mercy.'' Which of these two speeches, think you, would be the
best oration for your own funeral? Old Sedley made the last; and in
that humble frame of mind, and holding by the hand of his daughter,
life and disappointment and vanity sank away from under him.

``You see,'' said old Osborne to George, ``what comes of merit, and
industry, and judicious speculations, and that.  Look at me and my
banker's account.  Look at your poor Grandfather Sedley and his
failure.  And yet he was a better man than I was, this day twenty
years---a better man, I should say, by ten thousand pound.''

Beyond these people and Mr.\ Clapp's family, who came over from Brompton
to pay a visit of condolence, not a single soul alive ever cared a
penny piece about old John Sedley, or remembered the existence of such
a person.

When old Osborne first heard from his friend Colonel Buckler (as little
Georgy had already informed us) how distinguished an officer Major
Dobbin was, he exhibited a great deal of scornful incredulity and
expressed his surprise how ever such a feller as that should possess
either brains or reputation.  But he heard of the Major's fame from
various members of his society.  Sir William Dobbin had a great opinion
of his son and narrated many stories illustrative of the Major's
learning, valour, and estimation in the world's opinion. Finally, his
name appeared in the lists of one or two great parties of the nobility,
and this circumstance had a prodigious effect upon the old aristocrat
of Russell Square.

The Major's position, as guardian to Georgy, whose possession had been
ceded to his grandfather, rendered some meetings between the two
gentlemen inevitable; and it was in one of these that old Osborne, a
keen man of business, looking into the Major's accounts with his ward
and the boy's mother, got a hint, which staggered him very much, and at
once pained and pleased him, that it was out of William Dobbin's own
pocket that a part of the fund had been supplied upon which the poor
widow and the child had subsisted.

When pressed upon the point, Dobbin, who could not tell lies, blushed
and stammered a good deal and finally confessed.  ``The marriage,'' he
said (at which his interlocutor's face grew dark) ``was very much my
doing.  I thought my poor friend had gone so far that retreat from his
engagement would have been dishonour to him and death to Mrs.\ Osborne,
and I could do no less, when she was left without resources, than give
what money I could spare to maintain her.''

``Major D.,'' Mr.\ Osborne said, looking hard at him and turning very red
too---``you did me a great injury; but give me leave to tell you, sir,
you are an honest feller. There's my hand, sir, though I little thought
that my flesh and blood was living on you---'' and the pair shook hands,
with great confusion on Major Dobbin's part, thus found out in his act
of charitable hypocrisy.

He strove to soften the old man and reconcile him towards his son's
memory.  ``He was such a noble fellow,'' he said, ``that all of us loved
him, and would have done anything for him.  I, as a young man in those
days, was flattered beyond measure by his preference for me, and was
more pleased to be seen in his company than in that of the
Commander-in-Chief.  I never saw his equal for pluck and daring and all
the qualities of a soldier''; and Dobbin told the old father as many
stories as he could remember regarding the gallantry and achievements
of his son.  ``And Georgy is so like him,'' the Major added.

``He's so like him that he makes me tremble sometimes,'' the grandfather
said.

On one or two evenings the Major came to dine with Mr.\ Osborne (it was
during the time of the sickness of Mr.\ Sedley), and as the two sat
together in the evening after dinner, all their talk was about the
departed hero. The father boasted about him according to his wont,
glorifying himself in recounting his son's feats and gallantry, but his
mood was at any rate better and more charitable than that in which he
had been disposed until now to regard the poor fellow; and the
Christian heart of the kind Major was pleased at these symptoms of
returning peace and good-will.  On the second evening old Osborne
called Dobbin William, just as he used to do at the time when Dobbin
and George were boys together, and the honest gentleman was pleased by
that mark of reconciliation.

On the next day at breakfast, when Miss Osborne, with the asperity of
her age and character, ventured to make some remark reflecting
slightingly upon the Major's appearance or behaviour---the master of the
house interrupted her.  ``You'd have been glad enough to git him for
yourself, Miss O.  But them grapes are sour.  Ha!  ha! Major William is
a fine feller.''

``That he is, Grandpapa,'' said Georgy approvingly; and going up close to
the old gentleman, he took a hold of his large grey whiskers, and
laughed in his face good-humouredly, and kissed him.  And he told the
story at night to his mother, who fully agreed with the boy. ``Indeed he
is,'' she said.  ``Your dear father always said so. He is one of the best
and most upright of men.'' Dobbin happened to drop in very soon after
this conversation, which made Amelia blush perhaps, and the young
scapegrace increased the confusion by telling Dobbin the other part of
the story.  ``I say, Dob,'' he said, ``there's such an uncommon nice girl
wants to marry you.  She's plenty of tin; she wears a front; and she
scolds the servants from morning till night.'' ``Who is it?'' asked
Dobbin.   ``It's Aunt O.,'' the boy answered. ``Grandpapa said so.  And I
say, Dob, how prime it would be to have you for my uncle.'' Old Sedley's
quavering voice from the next room at this moment weakly called for
Amelia, and the laughing ended.

That old Osborne's mind was changing was pretty clear. He asked George
about his uncle sometimes, and laughed at the boy's imitation of the
way in which Jos said ``God-bless-my-soul'' and gobbled his soup.  Then
he said, ``It's not respectful, sir, of you younkers to be imitating of
your relations.  Miss O., when you go out adriving to-day, leave my
card upon Mr.\ Sedley, do you hear? There's no quarrel betwigst me and
him anyhow.''

The card was returned, and Jos and the Major were asked to dinner---to
a dinner the most splendid and stupid that perhaps ever Mr.\ Osborne
gave; every inch of the family plate was exhibited, and the best
company was asked.  Mr.\ Sedley took down Miss O.  to dinner, and she
was very gracious to him; whereas she hardly spoke to the Major, who
sat apart from her, and by the side of Mr.\ Osborne, very timid.  Jos
said, with great solemnity, it was the best turtle soup he had ever
tasted in his life, and asked Mr.\ Osborne where he got his Madeira.

``It is some of Sedley's wine,'' whispered the butler to his master.
``I've had it a long time, and paid a good figure for it, too,'' Mr.\ %
Osborne said aloud to his guest, and then whispered to his right-hand
neighbour how he had got it ``at the old chap's sale.''

More than once he asked the Major about---about Mrs.\ George Osborne---a
theme on which the Major could be very eloquent when he chose.  He told
Mr.\ Osborne of her sufferings---of her passionate attachment to her
husband, whose memory she worshipped still---of the tender and dutiful
manner in which she had supported her parents, and given up her boy,
when it seemed to her her duty to do so.  ``You don't know what she
endured, sir,'' said honest Dobbin with a tremor in his voice, ``and I
hope and trust you will be reconciled to her.  If she took your son
away from you, she gave hers to you; and however much you loved your
George, depend on it, she loved hers ten times more.''

``By God, you are a good feller, sir,'' was all Mr.\ Osborne said.  It had
never struck him that the widow would feel any pain at parting from the
boy, or that his having a fine fortune could grieve her.  A
reconciliation was announced as speedy and inevitable, and Amelia's
heart already began to beat at the notion of the awful meeting with
George's father.

It was never, however, destined to take place.  Old Sedley's lingering
illness and death supervened, after which a meeting was for some time
impossible.  That catastrophe and other events may have worked upon Mr.\ %
Osborne.  He was much shaken of late, and aged, and his mind was
working inwardly.  He had sent for his lawyers, and probably changed
something in his will.  The medical man who looked in pronounced him
shaky, agitated, and talked of a little blood and the seaside; but he
took neither of these remedies.

One day when he should have come down to breakfast, his servant missing
him, went into his dressing-room and found him lying at the foot of the
dressing-table in a fit.  Miss Osborne was apprised; the doctors were
sent for; Georgy stopped away from school; the bleeders and cuppers
came.  Osborne partially regained cognizance, but never could speak
again, though he tried dreadfully once or twice, and in four days he
died.  The doctors went down, and the undertaker's men went up the
stairs, and all the shutters were shut towards the garden in Russell
Square.  Bullock rushed from the City in a hurry. ``How much money had
he left to that boy? Not half, surely? Surely share and share alike
between the three?'' It was an agitating moment.

What was it that poor old man tried once or twice in vain to say? I
hope it was that he wanted to see Amelia and be reconciled before he
left the world to one dear and faithful wife of his son:  it was most
likely that, for his will showed that the hatred which he had so long
cherished had gone out of his heart.

They found in the pocket of his dressing-gown the letter with the great
red seal which George had written him from Waterloo.  He had looked at
the other papers too, relative to his son, for the key of the box in
which he kept them was also in his pocket, and it was found the seals
and envelopes had been broken---very likely on the night before the
seizure---when the butler had taken him tea into his study, and found
him reading in the great red family Bible.

When the will was opened, it was found that half the property was left
to George, and the remainder between the two sisters.  Mr.\ Bullock to
continue, for their joint benefit, the affairs of the commercial house,
or to go out, as he thought fit.  An annuity of five hundred pounds,
chargeable on George's property, was left to his mother, ``the widow of
my beloved son, George Osborne,'' who was to resume the guardianship of
the boy.

``Major William Dobbin, my beloved son's friend,'' was appointed
executor; ``and as out of his kindness and bounty, and with his own
private funds, he maintained my grandson and my son's widow, when they
were otherwise without means of support'' (the testator went on to say)
``I hereby thank him heartily for his love and regard for them, and
beseech him to accept such a sum as may be sufficient to purchase his
commission as a Lieutenant-Colonel, or to be disposed of in any way he
may think fit.''

When Amelia heard that her father-in-law was reconciled to her, her
heart melted, and she was grateful for the fortune left to her.  But
when she heard how Georgy was restored to her, and knew how and by
whom, and how it was William's bounty that supported her in poverty,
how it was William who gave her her husband and her son---oh, then she
sank on her knees, and prayed for blessings on that constant and kind
heart; she bowed down and humbled herself, and kissed the feet, as it
were, of that beautiful and generous affection.

And gratitude was all that she had to pay back for such admirable
devotion and benefits---only gratitude!  If she thought of any other
return, the image of George stood up out of the grave and said, ``You
are mine, and mine only, now and forever.''

William knew her feelings:  had he not passed his whole life in
divining them?

When the nature of Mr.\ Osborne's will became known to the world, it was
edifying to remark how Mrs.\ George Osborne rose in the estimation of
the people forming her circle of acquaintance.  The servants of Jos's
establishment, who used to question her humble orders and say they
would ``ask Master'' whether or not they could obey, never thought now of
that sort of appeal.  The cook forgot to sneer at her shabby old gowns
(which, indeed, were quite eclipsed by that lady's finery when she was
dressed to go to church of a Sunday evening), the others no longer
grumbled at the sound of her bell, or delayed to answer that summons.
The coachman, who grumbled that his 'osses should be brought out and
his carriage made into an hospital for that old feller and Mrs.\ O.,
drove her with the utmost alacrity now, and trembling lest he should be
superseded by Mr.\ Osborne's coachman, asked ``what them there Russell
Square coachmen knew about town, and whether they was fit to sit on a
box before a lady?'' Jos's friends, male and female, suddenly became
interested about Emmy, and cards of condolence multiplied on her hall
table.  Jos himself, who had looked on her as a good-natured harmless
pauper, to whom it was his duty to give victuals and shelter, paid her
and the rich little boy, his nephew, the greatest respect---was anxious
that she should have change and amusement after her troubles and
trials, ``poor dear girl''---and began to appear at the breakfast-table,
and most particularly to ask how she would like to dispose of the day.

In her capacity of guardian to Georgy, she, with the consent of the
Major, her fellow-trustee, begged Miss Osborne to live in the Russell
Square house as long as ever she chose to dwell there; but that lady,
with thanks, declared that she never could think of remaining alone in
that melancholy mansion, and departed in deep mourning to Cheltenham,
with a couple of her old domestics. The rest were liberally paid and
dismissed, the faithful old butler, whom Mrs.\ Osborne proposed to
retain, resigning and preferring to invest his savings in a
public-house, where, let us hope, he was not unprosperous. Miss Osborne
not choosing to live in Russell Square, Mrs.\ Osborne also, after
consultation, declined to occupy the gloomy old mansion there.  The
house was dismantled; the rich furniture and effects, the awful
chandeliers and dreary blank mirrors packed away and hidden, the rich
rosewood drawing-room suite was muffled in straw, the carpets were
rolled up and corded, the small select library of well-bound books was
stowed into two wine-chests, and the whole paraphernalia rolled away in
several enormous vans to the Pantechnicon, where they were to lie until
Georgy's majority.  And the great heavy dark plate-chests went off to
Messrs.  Stumpy and Rowdy, to lie in the cellars of those eminent
bankers until the same period should arrive.

One day Emmy, with George in her hand and clad in deep sables, went to
visit the deserted mansion which she had not entered since she was a
girl.  The place in front was littered with straw where the vans had
been laden and rolled off.  They went into the great blank rooms, the
walls of which bore the marks where the pictures and mirrors had hung.
Then they went up the great blank stone staircases into the upper
rooms, into that where grandpapa died, as George said in a whisper, and
then higher still into George's own room.  The boy was still clinging
by her side, but she thought of another besides him.  She knew that it
had been his father's room as well as his own.

She went up to one of the open windows (one of those at which she used
to gaze with a sick heart when the child was first taken from her), and
thence as she looked out she could see, over the trees of Russell
Square, the old house in which she herself was born, and where she had
passed so many happy days of sacred youth. They all came back to her,
the pleasant holidays, the kind faces, the careless, joyful past times,
and the long pains and trials that had since cast her down. She thought
of these and of the man who had been her constant protector, her good
genius, her sole benefactor, her tender and generous friend.

``Look here, Mother,'' said Georgy, ``here's a G.O. scratched on the glass
with a diamond, I never saw it before, I never did it.''

``It was your father's room long before you were born, George,'' she
said, and she blushed as she kissed the boy.

She was very silent as they drove back to Richmond, where they had
taken a temporary house:  where the smiling lawyers used to come
bustling over to see her (and we may be sure noted the visit in the
bill):  and where of course there was a room for Major Dobbin too, who
rode over frequently, having much business to transact on behalf of his
little ward.

Georgy at this time was removed from Mr.\ Veal's on an unlimited
holiday, and that gentleman was engaged to prepare an inscription for a
fine marble slab, to be placed up in the Foundling under the monument
of Captain George Osborne.

The female Bullock, aunt of Georgy, although despoiled by that little
monster of one-half of the sum which she expected from her father,
nevertheless showed her charitableness of spirit by being reconciled to
the mother and the boy.  Roehampton is not far from Richmond, and one
day the chariot, with the golden bullocks emblazoned on the panels, and
the flaccid children within, drove to Amelia's house at Richmond; and
the Bullock family made an irruption into the garden, where Amelia was
reading a book, Jos was in an arbour placidly dipping strawberries into
wine, and the Major in one of his Indian jackets was giving a back to
Georgy, who chose to jump over him.  He went over his head and bounded
into the little advance of Bullocks, with immense black bows in their
hats, and huge black sashes, accompanying their mourning mamma.

``He is just of the age for Rosa,'' the fond parent thought, and glanced
towards that dear child, an unwholesome little miss of seven years of
age.

``Rosa, go and kiss your dear cousin,'' Mrs.\ Frederick said.  ``Don't you
know me, George? I am your aunt.''

``I know you well enough,'' George said; ``but I don't like kissing,
please''; and he retreated from the obedient caresses of his cousin.

``Take me to your dear mamma, you droll child,'' Mrs.\ Frederick said, and
those ladies accordingly met, after an absence of more than fifteen
years.  During Emmy's cares and poverty the other had never once
thought about coming to see her, but now that she was decently
prosperous in the world, her sister-in-law came to her as a matter of
course.

So did numbers more.  Our old friend, Miss Swartz, and her husband came
thundering over from Hampton Court, with flaming yellow liveries, and
was as impetuously fond of Amelia as ever.  Miss Swartz would have
liked her always if she could have seen her.  One must do her that
justice.  But, que voulez vous?---in this vast town one has not the time
to go and seek one's friends; if they drop out of the rank they
disappear, and we march on without them.  Who is ever missed in Vanity
Fair?

But so, in a word, and before the period of grief for Mr.\ Osborne's
death had subsided, Emmy found herself in the centre of a very genteel
circle indeed, the members of which could not conceive that anybody
belonging to it was not very lucky.  There was scarce one of the ladies
that hadn't a relation a Peer, though the husband might be a drysalter
in the City.  Some of the ladies were very blue and well informed,
reading Mrs.\ Somerville and frequenting the Royal Institution; others
were severe and Evangelical, and held by Exeter Hall. Emmy, it must be
owned, found herself entirely at a loss in the midst of their clavers,
and suffered woefully on the one or two occasions on which she was
compelled to accept Mrs.\ Frederick Bullock's hospitalities.  That lady
persisted in patronizing her and determined most graciously to form
her.  She found Amelia's milliners for her and regulated her household
and her manners.  She drove over constantly from Roehampton and
entertained her friend with faint fashionable fiddle-faddle and feeble
Court slip-slop. Jos liked to hear it, but the Major used to go off
growling at the appearance of this woman, with her twopenny gentility.
He went to sleep under Frederick Bullock's bald head, after dinner, at
one of the banker's best parties (Fred was still anxious that the
balance of the Osborne property should be transferred from Stumpy and
Rowdy's to them), and whilst Amelia, who did not know Latin, or who
wrote the last crack article in the Edinburgh, and did not in the least
deplore, or otherwise, Mr.\ Peel's late extraordinary tergiversation on
the fatal Catholic Relief Bill, sat dumb amongst the ladies in the
grand drawing-room, looking out upon velvet lawns, trim gravel walks,
and glistening hot-houses.

``She seems good-natured but insipid,'' said Mrs.\ Rowdy; ``that Major
seems to be particularly epris.''

``She wants ton sadly,'' said Mrs.\ Hollyock.  ``My dear creature, you
never will be able to form her.''

``She is dreadfully ignorant or indifferent,'' said Mrs.\ Glowry with a
voice as if from the grave, and a sad shake of the head and turban. ``I
asked her if she thought that it was in 1836, according to Mr.\ Jowls,
or in 1839, according to Mr.\ Wapshot, that the Pope was to fall: and
she said---'Poor Pope!  I hope not---What has he done?'''

``She is my brother's widow, my dear friends,'' Mrs.\ Frederick replied,
``and as such I think we're all bound to give her every attention and
instruction on entering into the world.  You may fancy there can be no
MERCENARY motives in those whose DISAPPOINTMENTS are well known.''

``That poor dear Mrs.\ Bullock,'' said Rowdy to Hollyock, as they drove
away together---``she is always scheming and managing.  She wants Mrs.\ %
Osborne's account to be taken from our house to hers---and the way in
which she coaxes that boy and makes him sit by that blear-eyed little
Rosa is perfectly ridiculous.''

``I wish Glowry was choked with her Man of Sin and her Battle of
Armageddon,'' cried the other, and the carriage rolled away over Putney
Bridge.

But this sort of society was too cruelly genteel for Emmy, and all
jumped for joy when a foreign tour was proposed.



\chapter{\foreign{Am Rhein}}

The above everyday events had occurred, and a few weeks had passed,
when on one fine morning, Parliament being over, the summer advanced,
and all the good company in London about to quit that city for their
annual tour in search of pleasure or health, the Batavier steamboat
left the Tower-stairs laden with a goodly company of English fugitives.
The quarter-deck awnings were up, and the benches and gangways crowded
with scores of rosy children, bustling nursemaids; ladies in the
prettiest pink bonnets and summer dresses; gentlemen in travelling caps
and linen-jackets, whose mustachios had just begun to sprout for the
ensuing tour; and stout trim old veterans with starched neckcloths and
neat-brushed hats, such as have invaded Europe any time since the
conclusion of the war, and carry the national Goddem into every city of
the Continent.  The congregation of hat-boxes, and Bramah desks, and
dressing-cases was prodigious.  There were jaunty young Cambridge-men
travelling with their tutor, and going for a reading excursion to
Nonnenwerth or Konigswinter; there were Irish gentlemen, with the most
dashing whiskers and jewellery, talking about horses incessantly, and
prodigiously polite to the young ladies on board, whom, on the
contrary, the Cambridge lads and their pale-faced tutor avoided with
maiden coyness; there were old Pall Mall loungers bound for Ems and
Wiesbaden and a course of waters to clear off the dinners of the
season, and a little roulette and trente-et-quarante to keep the
excitement going; there was old Methuselah, who had married his young
wife, with Captain Papillon of the Guards holding her parasol and
guide-books; there was young May who was carrying off his bride on a
pleasure tour (Mrs.\ Winter that was, and who had been at school with
May's grandmother); there was Sir John and my Lady with a dozen
children, and corresponding nursemaids; and the great grandee Bareacres
family that sat by themselves near the wheel, stared at everybody, and
spoke to no one.  Their carriages, emblazoned with coronets and heaped
with shining imperials, were on the foredeck, locked in with a dozen
more such vehicles:  it was difficult to pass in and out amongst them;
and the poor inmates of the fore-cabin had scarcely any space for
locomotion.  These consisted of a few magnificently attired gentlemen
from Houndsditch, who brought their own provisions, and could have
bought half the gay people in the grand saloon; a few honest fellows
with mustachios and portfolios, who set to sketching before they had
been half an hour on board; one or two French femmes de chambre who
began to be dreadfully ill by the time the boat had passed Greenwich; a
groom or two who lounged in the neighbourhood of the horse-boxes under
their charge, or leaned over the side by the paddle-wheels, and talked
about who was good for the Leger, and what they stood to win or lose
for the Goodwood cup.

All the couriers, when they had done plunging about the ship and had
settled their various masters in the cabins or on the deck, congregated
together and began to chatter and smoke; the Hebrew gentlemen joining
them and looking at the carriages.  There was Sir John's great carriage
that would hold thirteen people; my Lord Methuselah's carriage, my Lord
Bareacres' chariot, britzska, and fourgon, that anybody might pay for
who liked. It was a wonder how my Lord got the ready money to pay for
the expenses of the journey. The Hebrew gentlemen knew how he got it.
They knew what money his Lordship had in his pocket at that instant,
and what interest he paid for it, and who gave it him.  Finally there
was a very neat, handsome travelling carriage, about which the
gentlemen speculated.

``A qui cette voiture la?'' said one gentleman-courier with a large
morocco money-bag and ear-rings to another with ear-rings and a large
morocco money-bag.

``C'est a Kirsch je bense---je l'ai vu toute a l'heure---qui brenoit des
sangviches dans la voiture,'' said the courier in a fine German French.

Kirsch emerging presently from the neighbourhood of the hold, where he
had been bellowing instructions intermingled with polyglot oaths to the
ship's men engaged in secreting the passengers' luggage, came to give
an account of himself to his brother interpreters.  He informed them
that the carriage belonged to a Nabob from Calcutta and Jamaica
enormously rich, and with whom he was engaged to travel; and at this
moment a young gentleman who had been warned off the bridge between the
paddle-boxes, and who had dropped thence on to the roof of Lord
Methuselah's carriage, from which he made his way over other carriages
and imperials until he had clambered on to his own, descended thence
and through the window into the body of the carriage, to the applause
of the couriers looking on.

``Nous allons avoir une belle traversee, Monsieur George,'' said the
courier with a grin, as he lifted his gold-laced cap.

``D--------- your French,'' said the young gentleman, ``where's the biscuits,
ay?'' Whereupon Kirsch answered him in the English language or in such
an imitation of it as he could command---for though he was familiar with
all languages, Mr.\ Kirsch was not acquainted with a single one, and
spoke all with indifferent volubility and incorrectness.

The imperious young gentleman who gobbled the biscuits (and indeed it
was time to refresh himself, for he had breakfasted at Richmond full
three hours before) was our young friend George Osborne.  Uncle Jos and
his mamma were on the quarter-deck with a gentleman of whom they used
to see a good deal, and the four were about to make a summer tour.

Jos was seated at that moment on deck under the awning, and pretty
nearly opposite to the Earl of Bareacres and his family, whose
proceedings absorbed the Bengalee almost entirely.  Both the noble
couple looked rather younger than in the eventful year '15, when Jos
remembered to have seen them at Brussels (indeed, he always gave out in
India that he was intimately acquainted with them).  Lady Bareacres'
hair, which was then dark, was now a beautiful golden auburn, whereas
Lord Bareacres' whiskers, formerly red, were at present of a rich black
with purple and green reflections in the light.  But changed as they
were, the movements of the noble pair occupied Jos's mind entirely.
The presence of a Lord fascinated him, and he could look at nothing
else.

``Those people seem to interest you a good deal,'' said Dobbin, laughing
and watching him.  Amelia too laughed. She was in a straw bonnet with
black ribbons, and otherwise dressed in mourning, but the little bustle
and holiday of the journey pleased and excited her, and she looked
particularly happy.

``What a heavenly day!'' Emmy said and added, with great originality, ``I
hope we shall have a calm passage.''

Jos waved his hand, scornfully glancing at the same time under his
eyelids at the great folks opposite.  ``If you had made the voyages we
have,'' he said, ``you wouldn't much care about the weather.'' But
nevertheless, traveller as he was, he passed the night direfully sick
in his carriage, where his courier tended him with brandy-and-water
and every luxury.

In due time this happy party landed at the quays of Rotterdam, whence
they were transported by another steamer to the city of Cologne.  Here
the carriage and the family took to the shore, and Jos was not a little
gratified to see his arrival announced in the Cologne newspapers as
``Herr Graf Lord von Sedley nebst Begleitung aus London.'' He had his
court dress with him; he had insisted that Dobbin should bring his
regimental paraphernalia; he announced that it was his intention to be
presented at some foreign courts, and pay his respects to the
Sovereigns of the countries which he honoured with a visit.

Wherever the party stopped, and an opportunity was offered, Mr.\ Jos
left his own card and the Major's upon ``Our Minister.'' It was with
great difficulty that he could be restrained from putting on his cocked
hat and tights to wait upon the English consul at the Free City of
Judenstadt, when that hospitable functionary asked our travellers to
dinner.  He kept a journal of his voyage and noted elaborately the
defects or excellences of the various inns at which he put up, and of
the wines and dishes of which he partook.

As for Emmy, she was very happy and pleased.  Dobbin used to carry
about for her her stool and sketch-book, and admired the drawings of
the good-natured little artist as they never had been admired before.
She sat upon steamers' decks and drew crags and castles, or she mounted
upon donkeys and ascended to ancient robber-towers, attended by her two
aides-de-camp, Georgy and Dobbin.  She laughed, and the Major did too,
at his droll figure on donkey-back, with his long legs touching the
ground.  He was the interpreter for the party; having a good military
knowledge of the German language, and he and the delighted George
fought the campaigns of the Rhine and the Palatinate.  In the course of
a few weeks, and by assiduously conversing with Herr Kirsch on the box
of the carriage, Georgy made prodigious advance in the knowledge of
High Dutch, and could talk to hotel waiters and postilions in a way
that charmed his mother and amused his guardian.

Mr.\ Jos did not much engage in the afternoon excursions of his
fellow-travellers.  He slept a good deal after dinner, or basked in the
arbours of the pleasant inn-gardens.  Pleasant Rhine gardens! Fair
scenes of peace and sunshine---noble purple mountains, whose crests are
reflected in the magnificent stream---who has ever seen you that has not
a grateful memory of those scenes of friendly repose and beauty? To lay
down the pen and even to think of that beautiful Rhineland makes one
happy.  At this time of summer evening, the cows are trooping down from
the hills, lowing and with their bells tinkling, to the old town, with
its old moats, and gates, and spires, and chestnut-trees, with long
blue shadows stretching over the grass; the sky and the river below
flame in crimson and gold; and the moon is already out, looking pale
towards the sunset.  The sun sinks behind the great castle-crested
mountains, the night falls suddenly, the river grows darker and darker,
lights quiver in it from the windows in the old ramparts, and twinkle
peacefully in the villages under the hills on the opposite shore.

So Jos used to go to sleep a good deal with his bandanna over his face
and be very comfortable, and read all the English news, and every word
of Galignani's admirable newspaper (may the blessings of all Englishmen
who have ever been abroad rest on the founders and proprietors of that
piratical print!  ) and whether he woke or slept, his friends did not
very much miss him.  Yes, they were very happy.  They went to the opera
often of evenings---to those snug, unassuming, dear old operas in the
German towns, where the noblesse sits and cries, and knits stockings on
the one side, over against the bourgeoisie on the other; and His
Transparency the Duke and his Transparent family, all very fat and
good-natured, come and occupy the great box in the middle; and the pit
is full of the most elegant slim-waisted officers with straw-coloured
mustachios, and twopence a day on full pay. Here it was that Emmy found
her delight, and was introduced for the first time to the wonders of
Mozart and Cimarosa. The Major's musical taste has been before alluded
to, and his performances on the flute commended. But perhaps the chief
pleasure he had in these operas was in watching Emmy's rapture while
listening to them. A new world of love and beauty broke upon her when
she was introduced to those divine compositions; this lady had the
keenest and finest sensibility, and how could she be indifferent when
she heard Mozart? The tender parts of ``Don Juan'' awakened in her
raptures so exquisite that she would ask herself when she went to say
her prayers of a night whether it was not wicked to feel so much
delight as that with which ``Vedrai Carino'' and ``Batti Batti'' filled her
gentle little bosom? But the Major, whom she consulted upon this head,
as her theological adviser (and who himself had a pious and reverent
soul), said that for his part, every beauty of art or nature made him
thankful as well as happy, and that the pleasure to be had in listening
to fine music, as in looking at the stars in the sky, or at a beautiful
landscape or picture, was a benefit for which we might thank Heaven as
sincerely as for any other worldly blessing.  And in reply to some
faint objections of Mrs.\ Amelia's (taken from certain theological works
like the Washerwoman of Finchley Common and others of that school, with
which Mrs.\ Osborne had been furnished during her life at Brompton) he
told her an Eastern fable of the Owl who thought that the sunshine was
unbearable for the eyes and that the Nightingale was a most overrated
bird.  ``It is one's nature to sing and the other's to hoot,'' he said,
laughing, ``and with such a sweet voice as you have yourself, you must
belong to the Bulbul faction.''

I like to dwell upon this period of her life and to think that she was
cheerful and happy.  You see, she has not had too much of that sort of
existence as yet, and has not fallen in the way of means to educate her
tastes or her intelligence.  She has been domineered over hitherto by
vulgar intellects.  It is the lot of many a woman. And as every one of
the dear sex is the rival of the rest of her kind, timidity passes for
folly in their charitable judgments; and gentleness for dulness; and
silence---which is but timid denial of the unwelcome assertion of ruling
folks, and tacit protestantism---above all, finds no mercy at the hands
of the female Inquisition. Thus, my dear and civilized reader, if you
and I were to find ourselves this evening in a society of greengrocers,
let us say, it is probable that our conversation would not be
brilliant; if, on the other hand, a greengrocer should find himself at
your refined and polite tea-table, where everybody was saying witty
things, and everybody of fashion and repute tearing her friends to
pieces in the most delightful manner, it is possible that the stranger
would not be very talkative and by no means interesting or interested.

And it must be remembered that this poor lady had never met a gentleman
in her life until this present moment.  Perhaps these are rarer
personages than some of us think for.  Which of us can point out many
such in his circle---men whose aims are generous, whose truth is
constant, and not only constant in its kind but elevated in its degree;
whose want of meanness makes them simple; who can look the world
honestly in the face with an equal manly sympathy for the great and the
small? We all know a hundred whose coats are very well made, and a
score who have excellent manners, and one or two happy beings who are
what they call in the inner circles, and have shot into the very centre
and bull's-eye of the fashion; but of gentlemen how many? Let us take a
little scrap of paper and each make out his list.

My friend the Major I write, without any doubt, in mine.  He had very
long legs, a yellow face, and a slight lisp, which at first was rather
ridiculous.  But his thoughts were just, his brains were fairly good,
his life was honest and pure, and his heart warm and humble.  He
certainly had very large hands and feet, which the two George Osbornes
used to caricature and laugh at; and their jeers and laughter perhaps
led poor little Emmy astray as to his worth.  But have we not all been
misled about our heroes and changed our opinions a hundred times? Emmy,
in this happy time, found that hers underwent a very great change in
respect of the merits of the Major.

Perhaps it was the happiest time of both their lives, indeed, if they
did but know it---and who does? Which of us can point out and say that
was the culmination---that was the summit of human joy? But at all
events, this couple were very decently contented, and enjoyed as
pleasant a summer tour as any pair that left England that year. Georgy
was always present at the play, but it was the Major who put Emmy's
shawl on after the entertainment; and in the walks and excursions the
young lad would be on ahead, and up a tower-stair or a tree, whilst the
soberer couple were below, the Major smoking his cigar with great
placidity and constancy, whilst Emmy sketched the site or the ruin.  It
was on this very tour that I, the present writer of a history of which
every word is true, had the pleasure to see them first and to make
their acquaintance.

It was at the little comfortable Ducal town of Pumpernickel (that very
place where Sir Pitt Crawley had been so distinguished as an attache;
but that was in early early days, and before the news of the Battle of
Austerlitz sent all the English diplomatists in Germany to the right
about) that I first saw Colonel Dobbin and his party.  They had arrived
with the carriage and courier at the Erbprinz Hotel, the best of the
town, and the whole party dined at the table d'hote.  Everybody
remarked the majesty of Jos and the knowing way in which he sipped, or
rather sucked, the Johannisberger, which he ordered for dinner.  The
little boy, too, we observed, had a famous appetite, and consumed
schinken, and braten, and kartoffeln, and cranberry jam, and salad, and
pudding, and roast fowls, and sweetmeats, with a gallantry that did
honour to his nation.  After about fifteen dishes, he concluded the
repast with dessert, some of which he even carried out of doors, for
some young gentlemen at table, amused with his coolness and gallant
free-and-easy manner, induced him to pocket a handful of macaroons,
which he discussed on his way to the theatre, whither everybody went in
the cheery social little German place. The lady in black, the boy's
mamma, laughed and blushed, and looked exceedingly pleased and shy as
the dinner went on, and at the various feats and instances of
espieglerie on the part of her son.  The Colonel---for so he became very
soon afterwards---I remember joked the boy with a great deal of grave
fun, pointing out dishes which he hadn't tried, and entreating him not
to baulk his appetite, but to have a second supply of this or that.

It was what they call a gast-rolle night at the Royal Grand Ducal
Pumpernickelisch Hof---or Court theatre---and Madame Schroeder Devrient,
then in the bloom of her beauty and genius, performed the part of the
heroine in the wonderful opera of Fidelio.  From our places in the
stalls we could see our four friends of the table d'hote in the loge
which Schwendler of the Erbprinz kept for his best guests, and I could
not help remarking the effect which the magnificent actress and music
produced upon Mrs.\ Osborne, for so we heard the stout gentleman in the
mustachios call her.  During the astonishing Chorus of the Prisoners,
over which the delightful voice of the actress rose and soared in the
most ravishing harmony, the English lady's face wore such an expression
of wonder and delight that it struck even little Fipps, the blase
attache, who drawled out, as he fixed his glass upon her, ``Gayd, it
really does one good to see a woman caypable of that stayt of
excaytement.'' And in the Prison Scene, where Fidelio, rushing to her
husband, cries, ``Nichts, nichts, mein Florestan,'' she fairly lost
herself and covered her face with her handkerchief.  Every woman in the
house was snivelling at the time, but I suppose it was because it was
predestined that I was to write this particular lady's memoirs that I
remarked her.

The next day they gave another piece of Beethoven, Die Schlacht bei
Vittoria.  Malbrook is introduced at the beginning of the performance,
as indicative of the brisk advance of the French army. Then come drums,
trumpets, thunders of artillery, and groans of the dying, and at last,
in a grand triumphal swell, ``God Save the King'' is performed.

There may have been a score of Englishmen in the house, but at the
burst of that beloved and well-known music, every one of them, we young
fellows in the stalls, Sir John and Lady Bullminster (who had taken a
house at Pumpernickel for the education of their nine children), the
fat gentleman with the mustachios, the long Major in white duck
trousers, and the lady with the little boy upon whom he was so sweet,
even Kirsch, the courier in the gallery, stood bolt upright in their
places and proclaimed themselves to be members of the dear old British
nation.  As for Tapeworm, the Charge d'Affaires, he rose up in his box
and bowed and simpered, as if he would represent the whole empire.
Tapeworm was nephew and heir of old Marshal Tiptoff, who has been
introduced in this story as General Tiptoff, just before Waterloo, who
was Colonel of the ---th regiment in which Major Dobbin served, and who
died in this year full of honours, and of an aspic of plovers' eggs;
when the regiment was graciously given by his Majesty to Colonel Sir
Michael O'Dowd, K.C.B.  who had commanded it in many glorious fields.

Tapeworm must have met with Colonel Dobbin at the house of the
Colonel's Colonel, the Marshal, for he recognized him on this night at
the theatre, and with the utmost condescension, his Majesty's minister
came over from his own box and publicly shook hands with his new-found
friend.

``Look at that infernal sly-boots of a Tapeworm,'' Fipps whispered,
examining his chief from the stalls. ``Wherever there's a pretty woman
he always twists himself in.'' And I wonder what were diplomatists made
for but for that?

``Have I the honour of addressing myself to Mrs.\ Dobbin?'' asked the
Secretary with a most insinuating grin.

Georgy burst out laughing and said, ``By Jove, that was a good 'un.''
Emmy and the Major blushed:  we saw them from the stalls.

``This lady is Mrs.\ George Osborne,'' said the Major, ``and this is her
brother, Mr.\ Sedley, a distinguished officer of the Bengal Civil
Service:  permit me to introduce him to your lordship.''

My lord nearly sent Jos off his legs with the most fascinating smile.
``Are you going to stop in Pumpernickel?'' he said.  ``It is a dull place,
but we want some nice people, and we would try and make it SO agreeable
to you.  Mr.---Ahum---Mrs.---Oho.  I shall do myself the honour of calling
upon you to-morrow at your inn.'' And he went away with a Parthian grin
and glance which he thought must finish Mrs.\ Osborne completely.

The performance over, the young fellows lounged about the lobbies, and
we saw the society take its departure. The Duchess Dowager went off in
her jingling old coach, attended by two faithful and withered old maids
of honour, and a little snuffy spindle-shanked gentleman in waiting, in
a brown jasey and a green coat covered with orders---of which the star
and the grand yellow cordon of the order of St.\ Michael of Pumpernickel
were most conspicuous.  The drums rolled, the guards saluted, and the
old carriage drove away.

Then came his Transparency the Duke and Transparent family, with his
great officers of state and household.  He bowed serenely to everybody.
And amid the saluting of the guards and the flaring of the torches of
the running footmen, clad in scarlet, the Transparent carriages drove
away to the old Ducal schloss, with its towers and pinacles standing on
the schlossberg.  Everybody in Pumpernickel knew everybody.  No sooner
was a foreigner seen there than the Minister of Foreign Affairs, or
some other great or small officer of state, went round to the Erbprinz
and found out the name of the new arrival.

We watched them, too, out of the theatre.  Tapeworm had just walked
off, enveloped in his cloak, with which his gigantic chasseur was
always in attendance, and looking as much as possible like Don Juan.
The Prime Minister's lady had just squeezed herself into her sedan, and
her daughter, the charming Ida, had put on her calash and clogs; when
the English party came out, the boy yawning drearily, the Major taking
great pains in keeping the shawl over Mrs.\ Osborne's head, and Mr.\ %
Sedley looking grand, with a crush opera-hat on one side of his head
and his hand in the stomach of a voluminous white waistcoat.  We took
off our hats to our acquaintances of the table d'hote, and the lady, in
return, presented us with a little smile and a curtsey, for which
everybody might be thankful.

The carriage from the inn, under the superintendence of the bustling
Mr.\ Kirsch, was in waiting to convey the party; but the fat man said he
would walk and smoke his cigar on his way homewards, so the other
three, with nods and smiles to us, went without Mr.\ Sedley, Kirsch,
with the cigar case, following in his master's wake.

We all walked together and talked to the stout gentleman about the
agremens of the place.  It was very agreeable for the English. There
were shooting-parties and battues; there was a plenty of balls and
entertainments at the hospitable Court; the society was generally good;
the theatre excellent; and the living cheap.

``And our Minister seems a most delightful and affable person,'' our new
friend said.  ``With such a representative, and---and a good medical man,
I can fancy the place to be most eligible.  Good-night, gentlemen.'' And
Jos creaked up the stairs to bedward, followed by Kirsch with a
flambeau.  We rather hoped that nice-looking woman would be induced to
stay some time in the town.



\chapter{In Which We Meet an Old Acquaintance}

Such polite behaviour as that of Lord Tapeworm did not fail to have the
most favourable effect upon Mr.\ Sedley's mind, and the very next
morning, at breakfast, he pronounced his opinion that Pumpernickel was
the pleasantest little place of any which he had visited on their tour.
Jos's motives and artifices were not very difficult of comprehension,
and Dobbin laughed in his sleeve, like a hypocrite as he was, when he
found, by the knowing air of the civilian and the offhand manner in
which the latter talked about Tapeworm Castle and the other members of
the family, that Jos had been up already in the morning, consulting his
travelling Peerage.  Yes, he had seen the Right Honourable the Earl of
Bagwig, his lordship's father; he was sure he had, he had met him
at---at the Levee---didn't Dob remember? and when the Diplomatist called
on the party, faithful to his promise, Jos received him with such a
salute and honours as were seldom accorded to the little Envoy.  He
winked at Kirsch on his Excellency's arrival, and that emissary,
instructed before-hand, went out and superintended an entertainment of
cold meats, jellies, and other delicacies, brought in upon trays, and
of which Mr.\ Jos absolutely insisted that his noble guest should
partake.

Tapeworm, so long as he could have an opportunity of admiring the
bright eyes of Mrs.\ Osborne (whose freshness of complexion bore
daylight remarkably well) was not ill pleased to accept any invitation
to stay in Mr.\ Sedley's lodgings; he put one or two dexterous questions
to him about India and the dancing-girls there; asked Amelia about that
beautiful boy who had been with her; and complimented the astonished
little woman upon the prodigious sensation which she had made in the
house; and tried to fascinate Dobbin by talking of the late war and the
exploits of the Pumpernickel contingent under the command of the
Hereditary Prince, now Duke of Pumpernickel.

Lord Tapeworm inherited no little portion of the family gallantry, and
it was his happy belief that almost every woman upon whom he himself
cast friendly eyes was in love with him.  He left Emmy under the
persuasion that she was slain by his wit and attractions and went home
to his lodgings to write a pretty little note to her.  She was not
fascinated, only puzzled, by his grinning, his simpering, his scented
cambric handkerchief, and his high-heeled lacquered boots.  She did not
understand one-half the compliments which he paid; she had never, in
her small experience of mankind, met a professional ladies' man as yet,
and looked upon my lord as something curious rather than pleasant; and
if she did not admire, certainly wondered at him.  Jos, on the
contrary, was delighted. ``How very affable his Lordship is,'' he said;
``How very kind of his Lordship to say he would send his medical man!
Kirsch, you will carry our cards to the Count de Schlusselback
directly; the Major and I will have the greatest pleasure in paying our
respects at Court as soon as possible.  Put out my uniform,
Kirsch---both our uniforms.  It is a mark of politeness which every
English gentleman ought to show to the countries which he visits to pay
his respects to the sovereigns of those countries as to the
representatives of his own.''

When Tapeworm's doctor came, Doctor von Glauber, Body Physician to
H.S.H.  the Duke, he speedily convinced Jos that the Pumpernickel
mineral springs and the Doctor's particular treatment would infallibly
restore the Bengalee to youth and slimness.  ``Dere came here last
year,'' he said, ``Sheneral Bulkeley, an English Sheneral, tvice so pic
as you, sir.  I sent him back qvite tin after tree months, and he
danced vid Baroness Glauber at the end of two.''

Jos's mind was made up; the springs, the Doctor, the Court, and the
Charge d'Affaires convinced him, and he proposed to spend the autumn in
these delightful quarters.  And punctual to his word, on the next day
the Charge d'Affaires presented Jos and the Major to Victor Aurelius
XVII, being conducted to their audience with that sovereign by the
Count de Schlusselback, Marshal of the Court.

They were straightway invited to dinner at Court, and their intention
of staying in the town being announced, the politest ladies of the
whole town instantly called upon Mrs.\ Osborne; and as not one of these,
however poor they might be, was under the rank of a Baroness, Jos's
delight was beyond expression.  He wrote off to Chutney at the Club to
say that the Service was highly appreciated in Germany, that he was
going to show his friend, the Count de Schlusselback, how to stick a
pig in the Indian fashion, and that his august friends, the Duke and
Duchess, were everything that was kind and civil.

Emmy, too, was presented to the august family, and as mourning is not
admitted in Court on certain days, she appeared in a pink crape dress
with a diamond ornament in the corsage, presented to her by her
brother, and she looked so pretty in this costume that the Duke and
Court (putting out of the question the Major, who had scarcely ever
seen her before in an evening dress, and vowed that she did not look
five-and-twenty) all admired her excessively.

In this dress she walked a Polonaise with Major Dobbin at a Court ball,
in which easy dance Mr.\ Jos had the honour of leading out the Countess
of Schlusselback, an old lady with a hump back, but with sixteen good
quarters of nobility and related to half the royal houses of Germany.

Pumpernickel stands in the midst of a happy valley through which
sparkles---to mingle with the Rhine somewhere, but I have not the map at
hand to say exactly at what point---the fertilizing stream of the Pump.
In some places the river is big enough to support a ferry-boat, in
others to turn a mill; in Pumpernickel itself, the last Transparency
but three, the great and renowned Victor Aurelius XIV built a
magnificent bridge, on which his own statue rises, surrounded by
water-nymphs and emblems of victory, peace, and plenty; he has his foot
on the neck of a prostrate Turk---history says he engaged and ran a
Janissary through the body at the relief of Vienna by Sobieski---but,
quite undisturbed by the agonies of that prostrate Mahometan, who
writhes at his feet in the most ghastly manner, the Prince smiles
blandly and points with his truncheon in the direction of the Aurelius
Platz, where he began to erect a new palace that would have been the
wonder of his age had the great-souled Prince but had funds to
complete it.  But the completion of Monplaisir (Monblaisir the honest
German folks call it) was stopped for lack of ready money, and it and
its park and garden are now in rather a faded condition, and not more
than ten times big enough to accommodate the Court of the reigning
Sovereign.

The gardens were arranged to emulate those of Versailles, and amidst
the terraces and groves there are some huge allegorical waterworks
still, which spout and froth stupendously upon fete-days, and frighten
one with their enormous aquatic insurrections.  There is the
Trophonius' cave in which, by some artifice, the leaden Tritons are
made not only to spout water, but to play the most dreadful groans out
of their lead conchs---there is the nymphbath and the Niagara cataract,
which the people of the neighbourhood admire beyond expression, when
they come to the yearly fair at the opening of the Chamber, or to the
fetes with which the happy little nation still celebrates the birthdays
and marriage-days of its princely governors.

Then from all the towns of the Duchy, which stretches for nearly ten
mile---from Bolkum, which lies on its western frontier bidding defiance
to Prussia, from Grogwitz, where the Prince has a hunting-lodge, and
where his dominions are separated by the Pump River from those of the
neighbouring Prince of Potzenthal; from all the little villages, which
besides these three great cities, dot over the happy principality---from
the farms and the mills along the Pump come troops of people in red
petticoats and velvet head-dresses, or with three-cornered hats and
pipes in their mouths, who flock to the Residenz and share in the
pleasures of the fair and the festivities there.  Then the theatre is
open for nothing, then the waters of Monblaisir begin to play (it is
lucky that there is company to behold them, for one would be afraid to
see them alone)---then there come mountebanks and riding troops (the way
in which his Transparency was fascinated by one of the horse-riders is
well known, and it is believed that La Petite Vivandiere, as she was
called, was a spy in the French interest), and the delighted people are
permitted to march through room after room of the Grand Ducal palace
and admire the slippery floor, the rich hangings, and the spittoons at
the doors of all the innumerable chambers.  There is one Pavilion at
Monblaisir which Aurelius Victor XV had arranged---a great Prince but
too fond of pleasure---and which I am told is a perfect wonder of
licentious elegance. It is painted with the story of Bacchus and
Ariadne, and the table works in and out of the room by means of a
windlass, so that the company was served without any intervention of
domestics.  But the place was shut up by Barbara, Aurelius XV's widow,
a severe and devout Princess of the House of Bolkum and Regent of the
Duchy during her son's glorious minority, and after the death of her
husband, cut off in the pride of his pleasures.

The theatre of Pumpernickel is known and famous in that quarter of
Germany.  It languished a little when the present Duke in his youth
insisted upon having his own operas played there, and it is said one
day, in a fury, from his place in the orchestra, when he attended a
rehearsal, broke a bassoon on the head of the Chapel Master, who was
conducting, and led too slow; and during which time the Duchess Sophia
wrote domestic comedies, which must have been very dreary to witness.
But the Prince executes his music in private now, and the Duchess only
gives away her plays to the foreigners of distinction who visit her
kind little Court.

It is conducted with no small comfort and splendour. When there are
balls, though there may be four hundred people at supper, there is a
servant in scarlet and lace to attend upon every four, and every one is
served on silver.  There are festivals and entertainments going
continually on, and the Duke has his chamberlains and equerries, and
the Duchess her mistress of the wardrobe and ladies of honour, just
like any other and more potent potentates.

The Constitution is or was a moderate despotism, tempered by a Chamber
that might or might not be elected.  I never certainly could hear of
its sitting in my time at Pumpernickel.  The Prime Minister had
lodgings in a second floor, and the Foreign Secretary occupied the
comfortable lodgings over Zwieback's Conditorey.  The army consisted of
a magnificent band that also did duty on the stage, where it was quite
pleasant to see the worthy fellows marching in Turkish dresses with
rouge on and wooden scimitars, or as Roman warriors with ophicleides
and trombones---to see them again, I say, at night, after one had
listened to them all the morning in the Aurelius Platz, where they
performed opposite the cafe where we breakfasted.  Besides the band,
there was a rich and numerous staff of officers, and, I believe, a few
men.  Besides the regular sentries, three or four men, habited as
hussars, used to do duty at the Palace, but I never saw them on
horseback, and au fait, what was the use of cavalry in a time of
profound peace?---and whither the deuce should the hussars ride?

Everybody---everybody that was noble of course, for as for the bourgeois
we could not quite be expected to take notice of THEM---visited his
neighbour.  H. E. Madame de Burst received once a week, H. E. Madame de
Schnurrbart had her night---the theatre was open twice a week, the Court
graciously received once, so that a man's life might in fact be a
perfect round of pleasure in the unpretending Pumpernickel way.

That there were feuds in the place, no one can deny. Politics ran very
high at Pumpernickel, and parties were very bitter.  There was the
Strumpff faction and the Lederlung party, the one supported by our
envoy and the other by the French Charge d'Affaires, M. de Macabau.
Indeed it sufficed for our Minister to stand up for Madame Strumpff,
who was clearly the greater singer of the two, and had three more notes
in her voice than Madame Lederlung her rival---it sufficed, I say, for
our Minister to advance any opinion to have it instantly contradicted
by the French diplomatist.

Everybody in the town was ranged in one or other of these factions. The
Lederlung was a prettyish little creature certainly, and her voice
(what there was of it) was very sweet, and there is no doubt that the
Strumpff was not in her first youth and beauty, and certainly too
stout; when she came on in the last scene of the Sonnambula, for
instance, in her night-chemise with a lamp in her hand, and had to go
out of the window, and pass over the plank of the mill, it was all she
could do to squeeze out of the window, and the plank used to bend and
creak again under her weight---but how she poured out the finale of the
opera!  and with what a burst of feeling she rushed into Elvino's
arms---almost fit to smother him! Whereas the little Lederlung---but a
truce to this gossip---the fact is that these two women were the two
flags of the French and the English party at Pumpernickel, and the
society was divided in its allegiance to those two great nations.

We had on our side the Home Minister, the Master of the Horse, the
Duke's Private Secretary, and the Prince's Tutor; whereas of the French
party were the Foreign Minister, the Commander-in-Chief's Lady, who had
served under Napoleon, and the Hof-Marschall and his wife, who was glad
enough to get the fashions from Paris, and always had them and her caps
by M. de Macabau's courier.  The Secretary of his Chancery was little
Grignac, a young fellow, as malicious as Satan, and who made
caricatures of Tapeworm in all the albums of the place.

Their headquarters and table d'hote were established at the Pariser
Hof, the other inn of the town; and though, of course, these gentlemen
were obliged to be civil in public, yet they cut at each other with
epigrams that were as sharp as razors, as I have seen a couple of
wrestlers in Devonshire, lashing at each other's shins and never
showing their agony upon a muscle of their faces.  Neither Tapeworm nor
Macabau ever sent home a dispatch to his government without a most
savage series of attacks upon his rival.  For instance, on our side we
would write, ``The interests of Great Britain in this place, and
throughout the whole of Germany, are perilled by the continuance in
office of the present French envoy; this man is of a character so
infamous that he will stick at no falsehood, or hesitate at no crime,
to attain his ends.  He poisons the mind of the Court against the
English minister, represents the conduct of Great Britain in the most
odious and atrocious light, and is unhappily backed by a minister whose
ignorance and necessities are as notorious as his influence is fatal.''
On their side they would say, ``M. de Tapeworm continues his system of
stupid insular arrogance and vulgar falsehood against the greatest
nation in the world.  Yesterday he was heard to speak lightly of Her
Royal Highness Madame the Duchess of Berri; on a former occasion he
insulted the heroic Duke of Angouleme and dared to insinuate that
H.R.H.  the Duke of Orleans was conspiring against the august throne of
the lilies.  His gold is prodigated in every direction which his stupid
menaces fail to frighten. By one and the other, he has won over
creatures of the Court here---and, in fine, Pumpernickel will not be
quiet, Germany tranquil, France respected, or Europe content until this
poisonous viper be crushed under heel'':  and so on.  When one side or
the other had written any particularly spicy dispatch, news of it was
sure to slip out.

Before the winter was far advanced, it is actually on record that Emmy
took a night and received company with great propriety and modesty.
She had a French master, who complimented her upon the purity of her
accent and her facility of learning; the fact is she had learned long
ago and grounded herself subsequently in the grammar so as to be able
to teach it to George; and Madam Strumpff came to give her lessons in
singing, which she performed so well and with such a true voice that
the Major's windows, who had lodgings opposite under the Prime
Minister, were always open to hear the lesson. Some of the German
ladies, who are very sentimental and simple in their tastes, fell in
love with her and began to call her du at once.  These are trivial
details, but they relate to happy times.  The Major made himself
George's tutor and read Caesar and mathematics with him, and they had a
German master and rode out of evenings by the side of Emmy's
carriage---she was always too timid, and made a dreadful outcry at the
slightest disturbance on horse-back.  So she drove about with one of
her dear German friends, and Jos asleep on the back-seat of the
barouche.

He was becoming very sweet upon the Grafinn Fanny de Butterbrod, a very
gentle tender-hearted and unassuming young creature, a Canoness and
Countess in her own right, but with scarcely ten pounds per year to her
fortune, and Fanny for her part declared that to be Amelia's sister was
the greatest delight that Heaven could bestow on her, and Jos might
have put a Countess's shield and coronet by the side of his own arms on
his carriage and forks; when---when events occurred, and those grand
fetes given upon the marriage of the Hereditary Prince of Pumpernickel
with the lovely Princess Amelia of Humbourg-Schlippenschloppen took
place.

At this festival the magnificence displayed was such as had not been
known in the little German place since the days of the prodigal Victor
XIV.  All the neighbouring Princes, Princesses, and Grandees were
invited to the feast.  Beds rose to half a crown per night in
Pumpernickel, and the Army was exhausted in providing guards of honour
for the Highnesses, Serenities, and Excellencies who arrived from all
quarters.  The Princess was married by proxy, at her father's
residence, by the Count de Schlusselback.  Snuff-boxes were given away
in profusion (as we learned from the Court jeweller, who sold and
afterwards bought them again), and bushels of the Order of Saint
Michael of Pumpernickel were sent to the nobles of the Court, while
hampers of the cordons and decorations of the Wheel of St.\ Catherine of
Schlippenschloppen were brought to ours.  The French envoy got both.
``He is covered with ribbons like a prize cart-horse,'' Tapeworm said,
who was not allowed by the rules of his service to take any
decorations:  ``Let him have the cordons; but with whom is the victory?''
The fact is, it was a triumph of British diplomacy, the French party
having proposed and tried their utmost to carry a marriage with a
Princess of the House of Potztausend-Donnerwetter, whom, as a matter
of course, we opposed.

Everybody was asked to the fetes of the marriage. Garlands and
triumphal arches were hung across the road to welcome the young bride.
The great Saint Michael's Fountain ran with uncommonly sour wine, while
that in the Artillery Place frothed with beer.  The great waters
played; and poles were put up in the park and gardens for the happy
peasantry, which they might climb at their leisure, carrying off
watches, silver forks, prize sausages hung with pink ribbon, \&c., at
the top.  Georgy got one, wrenching it off, having swarmed up the pole
to the delight of the spectators, and sliding down with the rapidity of
a fall of water.  But it was for the glory's sake merely.  The boy gave
the sausage to a peasant, who had very nearly seized it, and stood at
the foot of the mast, blubbering, because he was unsuccessful.

At the French Chancellerie they had six more lampions in their
illumination than ours had; but our transparency, which represented the
young Couple advancing and Discord flying away, with the most ludicrous
likeness to the French Ambassador, beat the French picture hollow; and
I have no doubt got Tapeworm the advancement and the Cross of the Bath
which he subsequently attained.

Crowds of foreigners arrived for the fetes, and of English, of course.
Besides the Court balls, public balls were given at the Town Hall and
the Redoute, and in the former place there was a room for
trente-et-quarante and roulette established, for the week of the
festivities only, and by one of the great German companies from Ems or
Aix-la-Chapelle.  The officers or inhabitants of the town were not
allowed to play at these games, but strangers, peasants, ladies were
admitted, and any one who chose to lose or win money.

That little scapegrace Georgy Osborne amongst others, whose pockets
were always full of dollars and whose relations were away at the grand
festival of the Court, came to the Stadthaus Ball in company of his
uncle's courier, Mr.\ Kirsch, and having only peeped into a play-room at
Baden-Baden when he hung on Dobbin's arm, and where, of course, he was
not permitted to gamble, came eagerly to this part of the entertainment
and hankered round the tables where the croupiers and the punters were
at work.  Women were playing; they were masked, some of them; this
license was allowed in these wild times of carnival.

A woman with light hair, in a low dress by no means so fresh as it had
been, and with a black mask on, through the eyelets of which her eyes
twinkled strangely, was seated at one of the roulette-tables with a
card and a pin and a couple of florins before her.  As the croupier
called out the colour and number, she pricked on the card with great
care and regularity, and only ventured her money on the colours after
the red or black had come up a certain number of times.  It was strange
to look at her.

But in spite of her care and assiduity she guessed wrong and the last
two florins followed each other under the croupier's rake, as he cried
out with his inexorable voice the winning colour and number.  She gave
a sigh, a shrug with her shoulders, which were already too much out of
her gown, and dashing the pin through the card on to the table, sat
thrumming it for a while.  Then she looked round her and saw Georgy's
honest face staring at the scene.  The little scamp!  What business had
he to be there?

When she saw the boy, at whose face she looked hard through her shining
eyes and mask, she said, ``Monsieur n'est pas joueur?''

``Non, Madame,'' said the boy; but she must have known, from his accent,
of what country he was, for she answered him with a slight foreign
tone.  ``You have nevare played---will you do me a littl' favor?''

``What is it?'' said Georgy, blushing again.  Mr.\ Kirsch was at work for
his part at the rouge et noir and did not see his young master.

``Play this for me, if you please; put it on any number, any number.''
And she took from her bosom a purse, and out of it a gold piece, the
only coin there, and she put it into George's hand.  The boy laughed
and did as he was bid.

The number came up sure enough.  There is a power that arranges that,
they say, for beginners.

``Thank you,'' said she, pulling the money towards her, ``thank you. What
is your name?''

``My name's Osborne,'' said Georgy, and was fingering in his own pockets
for dollars, and just about to make a trial, when the Major, in his
uniform, and Jos, en Marquis, from the Court ball, made their
appearance.  Other people, finding the entertainment stupid and
preferring the fun at the Stadthaus, had quitted the Palace ball
earlier; but it is probable the Major and Jos had gone home and found
the boy's absence, for the former instantly went up to him and, taking
him by the shoulder, pulled him briskly back from the place of
temptation.  Then, looking round the room, he saw Kirsch employed as we
have said, and going up to him, asked how he dared to bring Mr.\ George
to such a place.

``Laissez-moi tranquille,'' said Mr.\ Kirsch, very much excited by play
and wine.  ``Il faut s'amuser, parbleu. Je ne suis pas au service de
Monsieur.''

Seeing his condition the Major did not choose to argue with the man,
but contented himself with drawing away George and asking Jos if he
would come away.  He was standing close by the lady in the mask, who
was playing with pretty good luck now, and looking on much interested
at the game.

``Hadn't you better come, Jos,'' the Major said, ``with George and me?''

``I'll stop and go home with that rascal, Kirsch,'' Jos said; and for the
same reason of modesty, which he thought ought to be preserved before
the boy, Dobbin did not care to remonstrate with Jos, but left him and
walked home with Georgy.

``Did you play?'' asked the Major when they were out and on their way
home.

The boy said ``No.''

``Give me your word of honour as a gentleman that you never will.''

``Why?'' said the boy; ``it seems very good fun.'' And, in a very eloquent
and impressive manner, the Major showed him why he shouldn't, and would
have enforced his precepts by the example of Georgy's own father, had
he liked to say anything that should reflect on the other's memory.
When he had housed him, he went to bed and saw his light, in the little
room outside of Amelia's, presently disappear.  Amelia's followed half
an hour afterwards.  I don't know what made the Major note it so
accurately.

Jos, however, remained behind over the play-table; he was no gambler,
but not averse to the little excitement of the sport now and then, and
he had some Napoleons chinking in the embroidered pockets of his court
waistcoat.  He put down one over the fair shoulder of the little
gambler before him, and they won.  She made a little movement to make
room for him by her side, and just took the skirt of her gown from a
vacant chair there.

``Come and give me good luck,'' she said, still in a foreign accent,
quite different from that frank and perfectly English ``Thank you,'' with
which she had saluted Georgy's coup in her favour.  The portly
gentleman, looking round to see that nobody of rank observed him, sat
down; he muttered---``Ah, really, well now, God bless my soul. I'm very
fortunate; I'm sure to give you good fortune,'' and other words of
compliment and confusion. ``Do you play much?'' the foreign mask said.

``I put a Nap or two down,'' said Jos with a superb air, flinging down a
gold piece.

``Yes; ay nap after dinner,'' said the mask archly.  But Jos looking
frightened, she continued, in her pretty French accent, ``You do not
play to win.  No more do I. I play to forget, but I cannot.  I cannot
forget old times, monsieur.  Your little nephew is the image of his
father; and you---you are not changed---but yes, you are. Everybody
changes, everybody forgets; nobody has any heart.''

``Good God, who is it?'' asked Jos in a flutter.

``Can't you guess, Joseph Sedley?'' said the little woman in a sad voice,
and undoing her mask, she looked at him.  ``You have forgotten me.''

``Good heavens!  Mrs.\ Crawley!'' gasped out Jos.

``Rebecca,'' said the other, putting her hand on his; but she followed
the game still, all the time she was looking at him.

``I am stopping at the Elephant,'' she continued.  ``Ask for Madame de
Raudon.  I saw my dear Amelia to-day; how pretty she looked, and how
happy!  So do you! Everybody but me, who am wretched, Joseph Sedley.''
And she put her money over from the red to the black, as if by a chance
movement of her hand, and while she was wiping her eyes with a
pocket-handkerchief fringed with torn lace.

The red came up again, and she lost the whole of that stake. ``Come
away,'' she said.  ``Come with me a little---we are old friends, are we
not, dear Mr.\ Sedley?''

And Mr.\ Kirsch having lost all his money by this time, followed his
master out into the moonlight, where the illuminations were winking out
and the transparency over our mission was scarcely visible.



\chapter{A Vagabond Chapter}

We must pass over a part of Mrs.\ Rebecca Crawley's biography with that
lightness and delicacy which the world demands---the moral world, that
has, perhaps, no particular objection to vice, but an insuperable
repugnance to hearing vice called by its proper name. There are things
we do and know perfectly well in Vanity Fair, though we never speak of
them:  as the Ahrimanians worship the devil, but don't mention him:
and a polite public will no more bear to read an authentic description
of vice than a truly refined English or American female will permit the
word breeches to be pronounced in her chaste hearing.  And yet, madam,
both are walking the world before our faces every day, without much
shocking us.  If you were to blush every time they went by, what
complexions you would have!  It is only when their naughty names are
called out that your modesty has any occasion to show alarm or sense of
outrage, and it has been the wish of the present writer, all through
this story, deferentially to submit to the fashion at present
prevailing, and only to hint at the existence of wickedness in a light,
easy, and agreeable manner, so that nobody's fine feelings may be
offended.  I defy any one to say that our Becky, who has certainly some
vices, has not been presented to the public in a perfectly genteel and
inoffensive manner.  In describing this Siren, singing and smiling,
coaxing and cajoling, the author, with modest pride, asks his readers
all round, has he once forgotten the laws of politeness, and showed the
monster's hideous tail above water? No!  Those who like may peep down
under waves that are pretty transparent and see it writhing and
twirling, diabolically hideous and slimy, flapping amongst bones, or
curling round corpses; but above the waterline, I ask, has not
everything been proper, agreeable, and decorous, and has any the most
squeamish immoralist in Vanity Fair a right to cry fie? When, however,
the Siren disappears and dives below, down among the dead men, the
water of course grows turbid over her, and it is labour lost to look
into it ever so curiously.  They look pretty enough when they sit upon
a rock, twanging their harps and combing their hair, and sing, and
beckon to you to come and hold the looking-glass; but when they sink
into their native element, depend on it, those mermaids are about no
good, and we had best not examine the fiendish marine cannibals,
revelling and feasting on their wretched pickled victims.  And so, when
Becky is out of the way, be sure that she is not particularly well
employed, and that the less that is said about her doings is in fact
the better.

If we were to give a full account of her proceedings during a couple of
years that followed after the Curzon Street catastrophe, there might be
some reason for people to say this book was improper.  The actions of
very vain, heartless, pleasure-seeking people are very often improper
(as are many of yours, my friend with the grave face and spotless
reputation---but that is merely by the way); and what are those of a
woman without faith---or love---or character? And I am inclined to think
that there was a period in Mrs Becky's life when she was seized, not by
remorse, but by a kind of despair, and absolutely neglected her person
and did not even care for her reputation.

This abattement and degradation did not take place all at once; it was
brought about by degrees, after her calamity, and after many struggles
to keep up---as a man who goes overboard hangs on to a spar whilst any
hope is left, and then flings it away and goes down, when he finds that
struggling is in vain.

She lingered about London whilst her husband was making preparations
for his departure to his seat of government, and it is believed made
more than one attempt to see her brother-in-law, Sir Pitt Crawley, and
to work upon his feelings, which she had almost enlisted in her favour.
As Sir Pitt and Mr.\ Wenham were walking down to the House of Commons,
the latter spied Mrs.\ Rawdon in a black veil, and lurking near the
palace of the legislature.  She sneaked away when her eyes met those of
Wenham, and indeed never succeeded in her designs upon the Baronet.

Probably Lady Jane interposed.  I have heard that she quite astonished
her husband by the spirit which she exhibited in this quarrel, and her
determination to disown Mrs.\ Becky.  Of her own movement, she invited
Rawdon to come and stop in Gaunt Street until his departure for
Coventry Island, knowing that with him for a guard Mrs.\ Becky would not
try to force her door; and she looked curiously at the superscriptions
of all the letters which arrived for Sir Pitt, lest he and his
sister-in-law should be corresponding.  Not but that Rebecca could have
written had she a mind, but she did not try to see or to write to Pitt
at his own house, and after one or two attempts consented to his demand
that the correspondence regarding her conjugal differences should be
carried on by lawyers only.

The fact was that Pitt's mind had been poisoned against her.  A short
time after Lord Steyne's accident Wenham had been with the Baronet and
given him such a biography of Mrs.\ Becky as had astonished the member
for Queen's Crawley.  He knew everything regarding her: who her father
was; in what year her mother danced at the opera; what had been her
previous history; and what her conduct during her married life---as I
have no doubt that the greater part of the story was false and dictated
by interested malevolence, it shall not be repeated here.  But Becky
was left with a sad sad reputation in the esteem of a country gentleman
and relative who had been once rather partial to her.

The revenues of the Governor of Coventry Island are not large.  A part
of them were set aside by his Excellency for the payment of certain
outstanding debts and liabilities, the charges incident on his high
situation required considerable expense; finally, it was found that he
could not spare to his wife more than three hundred pounds a year,
which he proposed to pay to her on an undertaking that she would never
trouble him. Otherwise, scandal, separation, Doctors' Commons would
ensue.  But it was Mr.\ Wenham's business, Lord Steyne's business,
Rawdon's, everybody's---to get her out of the country, and hush up a
most disagreeable affair.

She was probably so much occupied in arranging these affairs of
business with her husband's lawyers that she forgot to take any step
whatever about her son, the little Rawdon, and did not even once
propose to go and see him.  That young gentleman was consigned to the
entire guardianship of his aunt and uncle, the former of whom had
always possessed a great share of the child's affection.  His mamma
wrote him a neat letter from Boulogne, when she quitted England, in
which she requested him to mind his book, and said she was going to
take a Continental tour, during which she would have the pleasure of
writing to him again.  But she never did for a year afterwards, and
not, indeed, until Sir Pitt's only boy, always sickly, died of
hooping-cough and measles---then Rawdon's mamma wrote the most
affectionate composition to her darling son, who was made heir of
Queen's Crawley by this accident, and drawn more closely than ever to
the kind lady, whose tender heart had already adopted him.  Rawdon
Crawley, then grown a tall, fine lad, blushed when he got the letter.
``Oh, Aunt Jane, you are my mother!'' he said; ``and not---and not that
one.'' But he wrote back a kind and respectful letter to Mrs.\ Rebecca,
then living at a boarding-house at Florence. But we are advancing
matters.

Our darling Becky's first flight was not very far.  She perched upon
the French coast at Boulogne, that refuge of so much exiled English
innocence, and there lived in rather a genteel, widowed manner, with a
femme de chambre and a couple of rooms, at an hotel.  She dined at the
table d'hote, where people thought her very pleasant, and where she
entertained her neighbours by stories of her brother, Sir Pitt, and her
great London acquaintance, talking that easy, fashionable slip-slop
which has so much effect upon certain folks of small breeding.  She
passed with many of them for a person of importance; she gave little
tea-parties in her private room and shared in the innocent amusements
of the place in sea-bathing, and in jaunts in open carriages, in
strolls on the sands, and in visits to the play. Mrs.\ Burjoice, the
printer's lady, who was boarding with her family at the hotel for the
summer, and to whom her Burjoice came of a Saturday and Sunday, voted
her charming, until that little rogue of a Burjoice began to pay her
too much attention.  But there was nothing in the story, only that
Becky was always affable, easy, and good-natured---and with men
especially.

Numbers of people were going abroad as usual at the end of the season,
and Becky had plenty of opportunities of finding out by the behaviour
of her acquaintances of the great London world the opinion of ``society''
as regarded her conduct.  One day it was Lady Partlet and her daughters
whom Becky confronted as she was walking modestly on Boulogne pier, the
cliffs of Albion shining in the distance across the deep blue sea.
Lady Partlet marshalled all her daughters round her with a sweep of her
parasol and retreated from the pier, darting savage glances at poor
little Becky who stood alone there.

On another day the packet came in.  It had been blowing fresh, and it
always suited Becky's humour to see the droll woe-begone faces of the
people as they emerged from the boat.  Lady Slingstone happened to be
on board this day.  Her ladyship had been exceedingly ill in her
carriage, and was greatly exhausted and scarcely fit to walk up the
plank from the ship to the pier.  But all her energies rallied the
instant she saw Becky smiling roguishly under a pink bonnet, and giving
her a glance of scorn such as would have shrivelled up most women, she
walked into the Custom House quite unsupported.  Becky only laughed:
but I don't think she liked it.  She felt she was alone, quite alone,
and the far-off shining cliffs of England were impassable to her.

The behaviour of the men had undergone too I don't know what change.
Grinstone showed his teeth and laughed in her face with a familiarity
that was not pleasant. Little Bob Suckling, who was cap in hand to her
three months before, and would walk a mile in the rain to see for her
carriage in the line at Gaunt House, was talking to Fitzoof of the
Guards (Lord Heehaw's son) one day upon the jetty, as Becky took her
walk there. Little Bobby nodded to her over his shoulder, without
moving his hat, and continued his conversation with the heir of Heehaw.
Tom Raikes tried to walk into her sitting-room at the inn with a cigar
in his mouth, but she closed the door upon him, and would have locked
it, only that his fingers were inside.  She began to feel that she was
very lonely indeed.  ``If HE'D been here,'' she said, ``those cowards
would never have dared to insult me.'' She thought about ``him'' with
great sadness and perhaps longing---about his honest, stupid, constant
kindness and fidelity; his never-ceasing obedience; his good humour;
his bravery and courage.  Very likely she cried, for she was
particularly lively, and had put on a little extra rouge, when she came
down to dinner.

She rouged regularly now; and---and her maid got Cognac for her besides
that which was charged in the hotel bill.

Perhaps the insults of the men were not, however, so intolerable to her
as the sympathy of certain women. Mrs.\ Crackenbury and Mrs.\ Washington
White passed through Boulogne on their way to Switzerland.  The party
were protected by Colonel Horner, young Beaumoris, and of course old
Crackenbury, and Mrs.\ White's little girl. THEY did not avoid her.
They giggled, cackled, tattled, condoled, consoled, and patronized her
until they drove her almost wild with rage.  To be patronized by THEM!
she thought, as they went away simpering after kissing her.  And she
heard Beaumoris's laugh ringing on the stair and knew quite well how to
interpret his hilarity.

It was after this visit that Becky, who had paid her weekly bills,
Becky who had made herself agreeable to everybody in the house, who
smiled at the landlady, called the waiters ``monsieur,'' and paid the
chambermaids in politeness and apologies, what far more than
compensated for a little niggardliness in point of money (of which
Becky never was free), that Becky, we say, received a notice to quit
from the landlord, who had been told by some one that she was quite an
unfit person to have at his hotel, where English ladies would not sit
down with her.  And she was forced to fly into lodgings of which the
dulness and solitude were most wearisome to her.

Still she held up, in spite of these rebuffs, and tried to make a
character for herself and conquer scandal.  She went to church very
regularly and sang louder than anybody there.  She took up the cause of
the widows of the shipwrecked fishermen, and gave work and drawings for
the Quashyboo Mission; she subscribed to the Assembly and WOULDN'T
waltz.  In a word, she did everything that was respectable, and that is
why we dwell upon this part of her career with more fondness than upon
subsequent parts of her history, which are not so pleasant. She saw
people avoiding her, and still laboriously smiled upon them; you never
could suppose from her countenance what pangs of humiliation she might
be enduring inwardly.

Her history was after all a mystery.  Parties were divided about her.
Some people who took the trouble to busy themselves in the matter said
that she was the criminal, whilst others vowed that she was as innocent
as a lamb and that her odious husband was in fault. She won over a good
many by bursting into tears about her boy and exhibiting the most
frantic grief when his name was mentioned, or she saw anybody like him.
She gained good Mrs.\ Alderney's heart in that way, who was rather the
Queen of British Boulogne and gave the most dinners and balls of all
the residents there, by weeping when Master Alderney came from Dr.
Swishtail's academy to pass his holidays with his mother.  ``He and her
Rawdon were of the same age, and so like,'' Becky said in a voice
choking with agony; whereas there was five years' difference between
the boys' ages, and no more likeness between them than between my
respected reader and his humble servant.  Wenham, when he was going
abroad, on his way to Kissingen to join Lord Steyne, enlightened Mrs.\ %
Alderney on this point and told her how he was much more able to
describe little Rawdon than his mamma, who notoriously hated him and
never saw him; how he was thirteen years old, while little Alderney was
but nine, fair, while the other darling was dark---in a word, caused the
lady in question to repent of her good humour.

Whenever Becky made a little circle for herself with incredible toils
and labour, somebody came and swept it down rudely, and she had all her
work to begin over again.  It was very hard; very hard; lonely and
disheartening.

There was Mrs.\ Newbright, who took her up for some time, attracted by
the sweetness of her singing at church and by her proper views upon
serious subjects, concerning which in former days, at Queen's Crawley,
Mrs.\ Becky had had a good deal of instruction.  Well, she not only took
tracts, but she read them.  She worked flannel petticoats for the
Quashyboos---cotton night-caps for the Cocoanut Indians---painted
handscreens for the conversion of the Pope and the Jews---sat under Mr.\ %
Rowls on Wednesdays, Mr.\ Huggleton on Thursdays, attended two Sunday
services at church, besides Mr.\ Bawler, the Darbyite, in the evening,
and all in vain.  Mrs.\ Newbright had occasion to correspond with the
Countess of Southdown about the Warmingpan Fund for the Fiji Islanders
(for the management of which admirable charity both these ladies formed
part of a female committee), and having mentioned her ``sweet friend,''
Mrs.\ Rawdon Crawley, the Dowager Countess wrote back such a letter
regarding Becky, with such particulars, hints, facts, falsehoods, and
general comminations, that intimacy between Mrs.\ Newbright and Mrs.\ %
Crawley ceased forthwith, and all the serious world of Tours, where
this misfortune took place, immediately parted company with the
reprobate.  Those who know the English Colonies abroad know that we
carry with us us our pride, pills, prejudices, Harvey-sauces,
cayenne-peppers, and other Lares, making a little Britain wherever we
settle down.

From one colony to another Becky fled uneasily.  From Boulogne to
Dieppe, from Dieppe to Caen, from Caen to Tours---trying with all her
might to be respectable, and alas!  always found out some day or other
and pecked out of the cage by the real daws.

Mrs.\ Hook Eagles took her up at one of these places---a woman without a
blemish in her character and a house in Portman Square.  She was
staying at the hotel at Dieppe, whither Becky fled, and they made each
other's acquaintance first at sea, where they were swimming together,
and subsequently at the table d'hote of the hotel.  Mrs Eagles had
heard---who indeed had not?---some of the scandal of the Steyne affair;
but after a conversation with Becky, she pronounced that Mrs.\ Crawley
was an angel, her husband a ruffian, Lord Steyne an unprincipled
wretch, as everybody knew, and the whole case against Mrs.\ Crawley an
infamous and wicked conspiracy of that rascal Wenham.  ``If you were a
man of any spirit, Mr.\ Eagles, you would box the wretch's ears the next
time you see him at the Club,'' she said to her husband.  But Eagles was
only a quiet old gentleman, husband to Mrs.\ Eagles, with a taste for
geology, and not tall enough to reach anybody's ears.

The Eagles then patronized Mrs.\ Rawdon, took her to live with her at
her own house at Paris, quarrelled with the ambassador's wife because
she would not receive her protegee, and did all that lay in woman's
power to keep Becky straight in the paths of virtue and good repute.

Becky was very respectable and orderly at first, but the life of
humdrum virtue grew utterly tedious to her before long.  It was the
same routine every day, the same dulness and comfort, the same drive
over the same stupid Bois de Boulogne, the same company of an evening,
the same Blair's Sermon of a Sunday night---the same opera always being
acted over and over again; Becky was dying of weariness, when, luckily
for her, young Mr.\ Eagles came from Cambridge, and his mother, seeing
the impression which her little friend made upon him, straightway gave
Becky warning.

Then she tried keeping house with a female friend; then the double
menage began to quarrel and get into debt.  Then she determined upon a
boarding-house existence and lived for some time at that famous mansion
kept by Madame de Saint Amour, in the Rue Royale, at Paris, where she
began exercising her graces and fascinations upon the shabby dandies
and fly-blown beauties who frequented her landlady's salons.  Becky
loved society and, indeed, could no more exist without it than an
opium-eater without his dram, and she was happy enough at the period of
her boarding-house life.  ``The women here are as amusing as those in
May Fair,'' she told an old London friend who met her, ``only, their
dresses are not quite so fresh.  The men wear cleaned gloves, and are
sad rogues, certainly, but they are not worse than Jack This and Tom
That.  The mistress of the house is a little vulgar, but I don't think
she is so vulgar as Lady ---------'' and here she named the name of a great
leader of fashion that I would die rather than reveal.  In fact, when
you saw Madame de Saint Amour's rooms lighted up of a night, men with
plaques and cordons at the ecarte tables, and the women at a little
distance, you might fancy yourself for a while in good society, and
that Madame was a real Countess.  Many people did so fancy, and Becky
was for a while one of the most dashing ladies of the Countess's salons.

But it is probable that her old creditors of 1815 found her out and
caused her to leave Paris, for the poor little woman was forced to fly
from the city rather suddenly, and went thence to Brussels.

How well she remembered the place!  She grinned as she looked up at the
little entresol which she had occupied, and thought of the Bareacres
family, bawling for horses and flight, as their carriage stood in the
porte-cochere of the hotel.  She went to Waterloo and to Laeken, where
George Osborne's monument much struck her.  She made a little sketch of
it.  ``That poor Cupid!'' she said; ``how dreadfully he was in love with
me, and what a fool he was!  I wonder whether little Emmy is alive.  It
was a good little creature; and that fat brother of hers.  I have his
funny fat picture still among my papers.  They were kind simple people.''

At Brussels Becky arrived, recommended by Madame de Saint Amour to her
friend, Madame la Comtesse de Borodino, widow of Napoleon's General,
the famous Count de Borodino, who was left with no resource by the
deceased hero but that of a table d'hote and an ecarte table.
Second-rate dandies and roues, widow-ladies who always have a lawsuit,
and very simple English folks, who fancy they see ``Continental society''
at these houses, put down their money, or ate their meals, at Madame de
Borodino's tables.  The gallant young fellows treated the company round
to champagne at the table d'hote, rode out with the women, or hired
horses on country excursions, clubbed money to take boxes at the play
or the opera, betted over the fair shoulders of the ladies at the
ecarte tables, and wrote home to their parents in Devonshire about
their felicitous introduction to foreign society.

Here, as at Paris, Becky was a boarding-house queen, and ruled in
select pensions.  She never refused the champagne, or the bouquets, or
the drives into the country, or the private boxes; but what she
preferred was the ecarte at night,---and she played audaciously. First
she played only for a little, then for five-franc pieces, then for
Napoleons, then for notes:  then she would not be able to pay her
month's pension:  then she borrowed from the young gentlemen: then she
got into cash again and bullied Madame de Borodino, whom she had coaxed
and wheedled before:  then she was playing for ten sous at a time, and
in a dire state of poverty:  then her quarter's allowance would come
in, and she would pay off Madame de Borodino's score and would once
more take the cards against Monsieur de Rossignol, or the Chevalier de
Raff.

When Becky left Brussels, the sad truth is that she owed three months'
pension to Madame de Borodino, of which fact, and of the gambling, and
of the drinking, and of the going down on her knees to the Reverend Mr.\ %
Muff, Ministre Anglican, and borrowing money of him, and of her coaxing
and flirting with Milor Noodle, son of Sir Noodle, pupil of the Rev.
Mr.\ Muff, whom she used to take into her private room, and of whom she
won large sums at ecarte---of which fact, I say, and of a hundred of her
other knaveries, the Countess de Borodino informs every English person
who stops at her establishment, and announces that Madame Rawdon was no
better than a vipere.

So our little wanderer went about setting up her tent in various cities
of Europe, as restless as Ulysses or Bampfylde Moore Carew. Her taste
for disrespectability grew more and more remarkable.  She became a
perfect Bohemian ere long, herding with people whom it would make your
hair stand on end to meet.

There is no town of any mark in Europe but it has its little colony of
English raffs---men whose names Mr.\ Hemp the officer reads out
periodically at the Sheriffs' Court---young gentlemen of very good
family often, only that the latter disowns them; frequenters of
billiard-rooms and estaminets, patrons of foreign races and
gaming-tables.  They people the debtors' prisons---they drink and
swagger---they fight and brawl---they run away without paying---they have
duels with French and German officers---they cheat Mr.\ Spooney at
ecarte---they get the money and drive off to Baden in magnificent
britzkas---they try their infallible martingale and lurk about the tables
with empty pockets, shabby bullies, penniless bucks, until they can
swindle a Jew banker with a sham bill of exchange, or find another Mr.\ %
Spooney to rob. The alternations of splendour and misery which these
people undergo are very queer to view.  Their life must be one of great
excitement.  Becky---must it be owned?---took to this life, and took to
it not unkindly.  She went about from town to town among these
Bohemians.  The lucky Mrs.\ Rawdon was known at every play-table in
Germany.  She and Madame de Cruchecassee kept house at Florence
together.  It is said she was ordered out of Munich, and my friend Mr.\ %
Frederick Pigeon avers that it was at her house at Lausanne that he was
hocussed at supper and lost eight hundred pounds to Major Loder and the
Honourable Mr.\ Deuceace.  We are bound, you see, to give some account
of Becky's biography, but of this part, the less, perhaps, that is said
the better.

They say that, when Mrs.\ Crawley was particularly down on her luck, she
gave concerts and lessons in music here and there.  There was a Madame
de Raudon, who certainly had a matinee musicale at Wildbad, accompanied
by Herr Spoff, premier pianist to the Hospodar of Wallachia, and my
little friend Mr.\ Eaves, who knew everybody and had travelled
everywhere, always used to declare that he was at Strasburg in the year
1830, when a certain Madame Rebecque made her appearance in the opera
of the Dame Blanche, giving occasion to a furious row in the theatre
there.  She was hissed off the stage by the audience, partly from her
own incompetency, but chiefly from the ill-advised sympathy of some
persons in the parquet, (where the officers of the garrison had their
admissions); and Eaves was certain that the unfortunate debutante in
question was no other than Mrs.\ Rawdon Crawley.

She was, in fact, no better than a vagabond upon this earth.  When she
got her money she gambled; when she had gambled it she was put to
shifts to live; who knows how or by what means she succeeded? It is
said that she was once seen at St.\ Petersburg, but was summarily
dismissed from that capital by the police, so that there cannot be any
possibility of truth in the report that she was a Russian spy at
Toplitz and Vienna afterwards.  I have even been informed that at Paris
she discovered a relation of her own, no less a person than her
maternal grandmother, who was not by any means a Montmorenci, but a
hideous old box-opener at a theatre on the Boulevards.  The meeting
between them, of which other persons, as it is hinted elsewhere, seem
to have been acquainted, must have been a very affecting interview.
The present historian can give no certain details regarding the event.

It happened at Rome once that Mrs.\ de Rawdon's half-year's salary had
just been paid into the principal banker's there, and, as everybody who
had a balance of above five hundred scudi was invited to the balls
which this prince of merchants gave during the winter, Becky had the
honour of a card, and appeared at one of the Prince and Princess
Polonia's splendid evening entertainments. The Princess was of the
family of Pompili, lineally descended from the second king of Rome, and
Egeria of the house of Olympus, while the Prince's grandfather,
Alessandro Polonia, sold wash-balls, essences, tobacco, and
pocket-handkerchiefs, ran errands for gentlemen, and lent money in a
small way.  All the great company in Rome thronged to his
saloons---Princes, Dukes, Ambassadors, artists, fiddlers, monsignori,
young bears with their leaders---every rank and condition of man. His
halls blazed with light and magnificence; were resplendent with gilt
frames (containing pictures), and dubious antiques; and the enormous
gilt crown and arms of the princely owner, a gold mushroom on a crimson
field (the colour of the pocket-handkerchiefs which he sold), and the
silver fountain of the Pompili family shone all over the roof, doors,
and panels of the house, and over the grand velvet baldaquins prepared
to receive Popes and Emperors.

So Becky, who had arrived in the diligence from Florence, and was
lodged at an inn in a very modest way, got a card for Prince Polonia's
entertainment, and her maid dressed her with unusual care, and she went
to this fine ball leaning on the arm of Major Loder, with whom she
happened to be travelling at the time---(the same man who shot Prince
Ravoli at Naples the next year, and was caned by Sir John Buckskin for
carrying four kings in his hat besides those which he used in playing
at ecarte )---and this pair went into the rooms together, and Becky saw
a number of old faces which she remembered in happier days, when she
was not innocent, but not found out. Major Loder knew a great number of
foreigners, keen-looking whiskered men with dirty striped ribbons in
their buttonholes, and a very small display of linen; but his own
countrymen, it might be remarked, eschewed the Major.  Becky, too, knew
some ladies here and there---French widows, dubious Italian countesses,
whose husbands had treated them ill---faugh---what shall we say, we who
have moved among some of the finest company of Vanity Fair, of this
refuse and sediment of rascals? If we play, let it be with clean cards,
and not with this dirty pack.  But every man who has formed one of the
innumerable army of travellers has seen these marauding irregulars
hanging on, like Nym and Pistol, to the main force, wearing the king's
colours and boasting of his commission, but pillaging for themselves,
and occasionally gibbeted by the roadside.

Well, she was hanging on the arm of Major Loder, and they went through
the rooms together, and drank a great quantity of champagne at the
buffet, where the people, and especially the Major's irregular corps,
struggled furiously for refreshments, of which when the pair had had
enough, they pushed on until they reached the Duchess's own pink velvet
saloon, at the end of the suite of apartments (where the statue of the
Venus is, and the great Venice looking-glasses, framed in silver), and
where the princely family were entertaining their most distinguished
guests at a round table at supper.  It was just such a little select
banquet as that of which Becky recollected that she had partaken at
Lord Steyne's---and there he sat at Polonia's table, and she saw him.
The scar cut by the diamond on his white, bald, shining forehead made a
burning red mark; his red whiskers were dyed of a purple hue, which
made his pale face look still paler.  He wore his collar and orders,
his blue ribbon and garter.  He was a greater Prince than any there,
though there was a reigning Duke and a Royal Highness, with their
princesses, and near his Lordship was seated the beautiful Countess of
Belladonna, nee de Glandier, whose husband (the Count Paolo della
Belladonna), so well known for his brilliant entomological collections,
had been long absent on a mission to the Emperor of Morocco.

When Becky beheld that familiar and illustrious face, how vulgar all of
a sudden did Major Loder appear to her, and how that odious Captain
Rook did smell of tobacco!  In one instant she reassumed her
fine-ladyship and tried to look and feel as if she were in May Fair
once more.  ``That woman looks stupid and ill-humoured,'' she thought; ``I
am sure she can't amuse him.  No, he must be bored by her---he never was
by me.'' A hundred such touching hopes, fears, and memories palpitated
in her little heart, as she looked with her brightest eyes (the rouge
which she wore up to her eyelids made them twinkle) towards the great
nobleman.  Of a Star and Garter night Lord Steyne used also to put on
his grandest manner and to look and speak like a great prince, as he
was. Becky admired him smiling sumptuously, easy, lofty, and stately.
Ah, bon Dieu, what a pleasant companion he was, what a brilliant wit,
what a rich fund of talk, what a grand manner!---and she had exchanged
this for Major Loder, reeking of cigars and brandy-and-water, and
Captain Rook with his horsejockey jokes and prize-ring slang, and their
like.  ``I wonder whether he will know me,'' she thought.  Lord Steyne
was talking and laughing with a great and illustrious lady at his side,
when he looked up and saw Becky.

She was all over in a flutter as their eyes met, and she put on the
very best smile she could muster, and dropped him a little, timid,
imploring curtsey.  He stared aghast at her for a minute, as Macbeth
might on beholding Banquo's sudden appearance at his ball-supper, and
remained looking at her with open mouth, when that horrid Major Loder
pulled her away.

``Come away into the supper-room, Mrs.\ R.,'' was that gentleman's remark:
``seeing these nobs grubbing away has made me peckish too. Let's go and
try the old governor's champagne.'' Becky thought the Major had had a
great deal too much already.

The day after she went to walk on the Pincian Hill---the Hyde Park of
the Roman idlers---possibly in hopes to have another sight of Lord
Steyne.  But she met another acquaintance there:  it was Mr.\ Fiche, his
lordship's confidential man, who came up nodding to her rather
familiarly and putting a finger to his hat.  ``I knew that Madame was
here,'' he said; ``I followed her from her hotel.  I have some advice to
give Madame.''

``From the Marquis of Steyne?'' Becky asked, resuming as much of her
dignity as she could muster, and not a little agitated by hope and
expectation.

``No,'' said the valet; ``it is from me.  Rome is very unwholesome.''

``Not at this season, Monsieur Fiche---not till after Easter.''

``I tell Madame it is unwholesome now.  There is always malaria for some
people.  That cursed marsh wind kills many at all seasons. Look, Madame
Crawley, you were always bon enfant, and I have an interest in you,
parole d'honneur.  Be warned.  Go away from Rome, I tell you---or you
will be ill and die.''

Becky laughed, though in rage and fury.  ``What! assassinate poor little
me?'' she said.  ``How romantic!  Does my lord carry bravos for couriers,
and stilettos in the fourgons? Bah!  I will stay, if but to plague him.
I have those who will defend me whilst I am here.''

It was Monsieur Fiche's turn to laugh now.  ``Defend you,'' he said, ``and
who? The Major, the Captain, any one of those gambling men whom Madame
sees would take her life for a hundred louis.  We know things about
Major Loder (he is no more a Major than I am my Lord the Marquis) which
would send him to the galleys or worse.  We know everything and have
friends everywhere. We know whom you saw at Paris, and what relations
you found there.  Yes, Madame may stare, but we do.  How was it that no
minister on the Continent would receive Madame? She has offended
somebody:  who never forgives---whose rage redoubled when he saw you.
He was like a madman last night when he came home.  Madame de
Belladonna made him a scene about you and fired off in one of her
furies.''

``Oh, it was Madame de Belladonna, was it?'' Becky said, relieved a
little, for the information she had just got had scared her.

``No---she does not matter---she is always jealous.  I tell you it was
Monseigneur.  You did wrong to show yourself to him.  And if you stay
here you will repent it.  Mark my words.  Go.  Here is my lord's
carriage''---and seizing Becky's arm, he rushed down an alley of the
garden as Lord Steyne's barouche, blazing with heraldic devices, came
whirling along the avenue, borne by the almost priceless horses, and
bearing Madame de Belladonna lolling on the cushions, dark, sulky, and
blooming, a King Charles in her lap, a white parasol swaying over her
head, and old Steyne stretched at her side with a livid face and
ghastly eyes.  Hate, or anger, or desire caused them to brighten now
and then still, but ordinarily, they gave no light, and seemed tired of
looking out on a world of which almost all the pleasure and all the
best beauty had palled upon the worn-out wicked old man.

``Monseigneur has never recovered the shock of that night, never,''
Monsieur Fiche whispered to Mrs.\ Crawley as the carriage flashed by,
and she peeped out at it from behind the shrubs that hid her.  ``That
was a consolation at any rate,'' Becky thought.

Whether my lord really had murderous intentions towards Mrs.\ Becky as
Monsieur Fiche said (since Monseigneur's death he has returned to his
native country, where he lives much respected, and has purchased from
his Prince the title of Baron Ficci), and the factotum objected to have
to do with assassination; or whether he simply had a commission to
frighten Mrs.\ Crawley out of a city where his Lordship proposed to pass
the winter, and the sight of her would be eminently disagreeable to the
great nobleman, is a point which has never been ascertained:  but the
threat had its effect upon the little woman, and she sought no more to
intrude herself upon the presence of her old patron.

Everybody knows the melancholy end of that nobleman, which befell at
Naples two months after the French Revolution of 1830; when the Most
Honourable George Gustavus, Marquis of Steyne, Earl of Gaunt and of
Gaunt Castle, in the Peerage of Ireland, Viscount Hellborough, Baron
Pitchley and Grillsby, a Knight of the Most Noble Order of the Garter,
of the Golden Fleece of Spain, of the Russian Order of Saint Nicholas
of the First Class, of the Turkish Order of the Crescent, First Lord of
the Powder Closet and Groom of the Back Stairs, Colonel of the Gaunt or
Regent's Own Regiment of Militia, a Trustee of the British Museum, an
Elder Brother of the Trinity House, a Governor of the White Friars, and
D.C.L.---died after a series of fits brought on, as the papers said, by
the shock occasioned to his lordship's sensibilities by the downfall of
the ancient French monarchy.

An eloquent catalogue appeared in a weekly print, describing his
virtues, his magnificence, his talents, and his good actions.  His
sensibility, his attachment to the illustrious House of Bourbon, with
which he claimed an alliance, were such that he could not survive the
misfortunes of his august kinsmen.  His body was buried at Naples, and
his heart---that heart which always beat with every generous and noble
emotion was brought back to Castle Gaunt in a silver urn.  ``In him,''
Mr.\ Wagg said, ``the poor and the Fine Arts have lost a beneficent
patron, society one of its most brilliant ornaments, and England one of
her loftiest patriots and statesmen,'' \&c., \&c.

His will was a good deal disputed, and an attempt was made to force
from Madame de Belladonna the celebrated jewel called the ``Jew's-eye''
diamond, which his lordship always wore on his forefinger, and which it
was said that she removed from it after his lamented demise. But his
confidential friend and attendant, Monsieur Fiche proved that the ring
had been presented to the said Madame de Belladonna two days before the
Marquis's death, as were the bank-notes, jewels, Neapolitan and French
bonds, \&c., found in his lordship's secretaire and claimed by his heirs
from that injured woman.



\chapter{Full of Business and Pleasure}

The day after the meeting at the play-table, Jos had himself arrayed
with unusual care and splendour, and without thinking it necessary to
say a word to any member of his family regarding the occurrences of the
previous night, or asking for their company in his walk, he sallied
forth at an early hour, and was presently seen making inquiries at the
door of the Elephant Hotel.  In consequence of the fetes the house was
full of company, the tables in the street were already surrounded by
persons smoking and drinking the national small-beer, the public rooms
were in a cloud of smoke, and Mr.\ Jos having, in his pompous way, and
with his clumsy German, made inquiries for the person of whom he was in
search, was directed to the very top of the house, above the
first-floor rooms where some travelling pedlars had lived, and were
exhibiting their jewellery and brocades; above the second-floor
apartments occupied by the etat major of the gambling firm; above the
third-floor rooms, tenanted by the band of renowned Bohemian vaulters
and tumblers; and so on to the little cabins of the roof, where, among
students, bagmen, small tradesmen, and country-folks come in for the
festival, Becky had found a little nest---as dirty a little refuge as
ever beauty lay hid in.

Becky liked the life.  She was at home with everybody in the place,
pedlars, punters, tumblers, students and all. She was of a wild, roving
nature, inherited from father and mother, who were both Bohemians, by
taste and circumstance; if a lord was not by, she would talk to his
courier with the greatest pleasure; the din, the stir, the drink, the
smoke, the tattle of the Hebrew pedlars, the solemn, braggart ways of
the poor tumblers, the sournois talk of the gambling-table officials,
the songs and swagger of the students, and the general buzz and hum of
the place had pleased and tickled the little woman, even when her luck
was down and she had not wherewithal to pay her bill.  How pleasant was
all the bustle to her now that her purse was full of the money which
little Georgy had won for her the night before!

As Jos came creaking and puffing up the final stairs, and was
speechless when he got to the landing, and began to wipe his face and
then to look for No.\ 92, the room where he was directed to seek for the
person he wanted, the door of the opposite chamber, No.\ 90, was open,
and a student, in jack-boots and a dirty schlafrock, was lying on the
bed smoking a long pipe; whilst another student in long yellow hair and
a braided coat, exceeding smart and dirty too, was actually on his
knees at No.\ 92, bawling through the keyhole supplications to the
person within.

``Go away,'' said a well-known voice, which made Jos thrill, ``I expect
somebody; I expect my grandpapa.  He mustn't see you there.''

``Angel Englanderinn!'' bellowed the kneeling student with the whity-brown
ringlets and the large finger-ring, ``do take compassion upon us.
Make an appointment. Dine with me and Fritz at the inn in the park.  We
will have roast pheasants and porter, plum-pudding and French wine.  We
shall die if you don't.''

``That we will,'' said the young nobleman on the bed; and this colloquy
Jos overheard, though he did not comprehend it, for the reason that he
had never studied the language in which it was carried on.

``Newmero kattervang dooze, si vous plait,'' Jos said in his grandest
manner, when he was able to speak.

``Quater fang tooce!'' said the student, starting up, and he bounced into
his own room, where he locked the door, and where Jos heard him
laughing with his comrade on the bed.

The gentleman from Bengal was standing, disconcerted by this incident,
when the door of the 92 opened of itself and Becky's little head peeped
out full of archness and mischief.  She lighted on Jos.  ``It's you,''
she said, coming out.  ``How I have been waiting for you!  Stop! not
yet---in one minute you shall come in.'' In that instant she put a
rouge-pot, a brandy bottle, and a plate of broken meat into the bed,
gave one smooth to her hair, and finally let in her visitor.

She had, by way of morning robe, a pink domino, a trifle faded and
soiled, and marked here and there with pomaturn; but her arms shone out
from the loose sleeves of the dress very white and fair, and it was
tied round her little waist so as not ill to set off the trim little
figure of the wearer.  She led Jos by the hand into her garret. ``Come
in,'' she said.  ``Come and talk to me.  Sit yonder on the chair''; and
she gave the civilian's hand a little squeeze and laughingly placed him
upon it.  As for herself, she placed herself on the bed---not on the
bottle and plate, you may be sure---on which Jos might have reposed, had
he chosen that seat; and so there she sat and talked with her old
admirer.   ``How little years have changed you,'' she said with a look of
tender interest.  ``I should have known you anywhere.  What a comfort it
is amongst strangers to see once more the frank honest face of an old
friend!''

The frank honest face, to tell the truth, at this moment bore any
expression but one of openness and honesty:  it was, on the contrary,
much perturbed and puzzled in look.  Jos was surveying the queer little
apartment in which he found his old flame.  One of her gowns hung over
the bed, another depending from a hook of the door; her bonnet obscured
half the looking-glass, on which, too, lay the prettiest little pair of
bronze boots; a French novel was on the table by the bedside, with a
candle, not of wax.  Becky thought of popping that into the bed too,
but she only put in the little paper night-cap with which she had put
the candle out on going to sleep.

``I should have known you anywhere,'' she continued; ``a woman never
forgets some things.  And you were the first man I ever---I ever saw.''

``Was I really?'' said Jos.  ``God bless my soul, you---you don't say so.''

``When I came with your sister from Chiswick, I was scarcely more than a
child,'' Becky said.  ``How is that, dear love? Oh, her husband was a sad
wicked man, and of course it was of me that the poor dear was jealous.
As if I cared about him, heigho!  when there was somebody---but
no---don't let us talk of old times''; and she passed her handkerchief
with the tattered lace across her eyelids.

``Is not this a strange place,'' she continued, ``for a woman, who has
lived in a very different world too, to be found in? I have had so many
griefs and wrongs, Joseph Sedley; I have been made to suffer so cruelly
that I am almost made mad sometimes.  I can't stay still in any place,
but wander about always restless and unhappy. All my friends have been
false to me---all.  There is no such thing as an honest man in the
world.  I was the truest wife that ever lived, though I married my
husband out of pique, because somebody else---but never mind that.  I
was true, and he trampled upon me and deserted me.  I was the fondest
mother.  I had but one child, one darling, one hope, one joy, which I
held to my heart with a mother's affection, which was my life, my
prayer, my---my blessing; and they---they tore it from me---tore it from
me''; and she put her hand to her heart with a passionate gesture of
despair, burying her face for a moment on the bed.

The brandy-bottle inside clinked up against the plate which held the
cold sausage.  Both were moved, no doubt, by the exhibition of so much
grief.  Max and Fritz were at the door, listening with wonder to Mrs.\ %
Becky's sobs and cries.  Jos, too, was a good deal frightened and
affected at seeing his old flame in this condition. And she began,
forthwith, to tell her story---a tale so neat, simple, and artless that
it was quite evident from hearing her that if ever there was a
white-robed angel escaped from heaven to be subject to the infernal
machinations and villainy of fiends here below, that spotless
being---that miserable unsullied martyr, was present on the bed before
Jos---on the bed, sitting on the brandy-bottle.

They had a very long, amicable, and confidential talk there, in the
course of which Jos Sedley was somehow made aware (but in a manner that
did not in the least scare or offend him) that Becky's heart had first
learned to beat at his enchanting presence; that George Osborne had
certainly paid an unjustifiable court to HER, which might account for
Amelia's jealousy and their little rupture; but that Becky never gave
the least encouragement to the unfortunate officer, and that she had
never ceased to think about Jos from the very first day she had seen
him, though, of course, her duties as a married woman were
paramount---duties which she had always preserved, and would, to her
dying day, or until the proverbially bad climate in which Colonel
Crawley was living should release her from a yoke which his cruelty had
rendered odious to her.

Jos went away, convinced that she was the most virtuous, as she was one
of the most fascinating of women, and revolving in his mind all sorts
of benevolent schemes for her welfare.  Her persecutions ought to be
ended: she ought to return to the society of which she was an ornament.
He would see what ought to be done.  She must quit that place and take
a quiet lodging.  Amelia must come and see her and befriend her.  He
would go and settle about it, and consult with the Major.  She wept
tears of heart-felt gratitude as she parted from him, and pressed his
hand as the gallant stout gentleman stooped down to kiss hers.

So Becky bowed Jos out of her little garret with as much grace as if it
was a palace of which she did the honours; and that heavy gentleman
having disappeared down the stairs, Max and Fritz came out of their
hole, pipe in mouth, and she amused herself by mimicking Jos to them as
she munched her cold bread and sausage and took draughts of her
favourite brandy-and-water.

Jos walked over to Dobbin's lodgings with great solemnity and there
imparted to him the affecting history with which he had just been made
acquainted, without, however, mentioning the play business of the night
before. And the two gentlemen were laying their heads together and
consulting as to the best means of being useful to Mrs.\ Becky, while
she was finishing her interrupted dejeuner a la fourchette.

How was it that she had come to that little town? How was it that she
had no friends and was wandering about alone? Little boys at school are
taught in their earliest Latin book that the path of Avernus is very
easy of descent.  Let us skip over the interval in the history of her
downward progress.  She was not worse now than she had been in the days
of her prosperity---only a little down on her luck.

As for Mrs.\ Amelia, she was a woman of such a soft and foolish
disposition that when she heard of anybody unhappy, her heart
straightway melted towards the sufferer; and as she had never thought
or done anything mortally guilty herself, she had not that abhorrence
for wickedness which distinguishes moralists much more knowing.  If she
spoiled everybody who came near her with kindness and compliments---if
she begged pardon of all her servants for troubling them to answer the
bell---if she apologized to a shopboy who showed her a piece of silk, or
made a curtsey to a street-sweeper with a complimentary remark upon
the elegant state of his crossing---and she was almost capable of every
one of these follies---the notion that an old acquaintance was
miserable was sure to soften her heart; nor would she hear of anybody's
being deservedly unhappy. A world under such legislation as hers would
not be a very orderly place of abode; but there are not many women, at
least not of the rulers, who are of her sort.  This lady, I believe,
would have abolished all gaols, punishments, handcuffs, whippings,
poverty, sickness, hunger, in the world, and was such a mean-spirited
creature that---we are obliged to confess it---she could even forget a
mortal injury.

When the Major heard from Jos of the sentimental adventure which had
just befallen the latter, he was not, it must be owned, nearly as much
interested as the gentleman from Bengal.  On the contrary, his
excitement was quite the reverse from a pleasurable one; he made use of
a brief but improper expression regarding a poor woman in distress,
saying, in fact, ``The little minx, has she come to light again?'' He
never had had the slightest liking for her, but had heartily mistrusted
her from the very first moment when her green eyes had looked at, and
turned away from, his own.

``That little devil brings mischief wherever she goes,'' the Major said
disrespectfully.  ``Who knows what sort of life she has been leading?
And what business has she here abroad and alone? Don't tell me about
persecutors and enemies; an honest woman always has friends and never
is separated from her family.  Why has she left her husband? He may
have been disreputable and wicked, as you say.  He always was.  I
remember the confounded blackleg and the way in which he used to cheat
and hoodwink poor George.  Wasn't there a scandal about their
separation? I think I heard something,'' cried out Major Dobbin, who did
not care much about gossip, and whom Jos tried in vain to convince that
Mrs.\ Becky was in all respects a most injured and virtuous female.

``Well, well; let's ask Mrs.\ George,'' said that arch-diplomatist of a
Major.  ``Only let us go and consult her. I suppose you will allow that
she is a good judge at any rate, and knows what is right in such
matters.''

``Hm!  Emmy is very well,'' said Jos, who did not happen to be in love
with his sister.

``Very well? By Gad, sir, she's the finest lady I ever met in my life,''
bounced out the Major.  ``I say at once, let us go and ask her if this
woman ought to be visited or not---I will be content with her verdict.''
Now this odious, artful rogue of a Major was thinking in his own mind
that he was sure of his case.  Emmy, he remembered, was at one time
cruelly and deservedly jealous of Rebecca, never mentioned her name but
with a shrinking and terror---a jealous woman never forgives, thought
Dobbin:  and so the pair went across the street to Mrs.\ George's house,
where she was contentedly warbling at a music lesson with Madame
Strumpff.

When that lady took her leave, Jos opened the business with his usual
pomp of words.  ``Amelia, my dear,'' said he, ``I have just had the most
extraordinary---yes---God bless my soul!  the most extraordinary
adventure---an old friend---yes, a most interesting old friend of yours,
and I may say in old times, has just arrived here, and I should like
you to see her.''

``Her!'' said Amelia, ``who is it? Major Dobbin, if you please not to
break my scissors.'' The Major was twirling them round by the little
chain from which they sometimes hung to their lady's waist, and was
thereby endangering his own eye.

``It is a woman whom I dislike very much,'' said the Major, doggedly, ``and
whom you have no cause to love.''

``It is Rebecca, I'm sure it is Rebecca,'' Amelia said, blushing and
being very much agitated.

``You are right; you always are,'' Dobbin answered. Brussels, Waterloo,
old, old times, griefs, pangs, remembrances, rushed back into Amelia's
gentle heart and caused a cruel agitation there.

``Don't let me see her,'' Emmy continued.  ``I couldn't see her.''

``I told you so,'' Dobbin said to Jos.

``She is very unhappy, and---and that sort of thing,'' Jos urged.  ``She is
very poor and unprotected, and has been ill---exceedingly ill---and that
scoundrel of a husband has deserted her.''

``Ah!'' said Amelia.

``She hasn't a friend in the world,'' Jos went on, not undexterously,
``and she said she thought she might trust in you.  She's so miserable,
Emmy.  She has been almost mad with grief.  Her story quite affected
me---'pon my word and honour, it did---never was such a cruel persecution
borne so angelically, I may say.  Her family has been most cruel to
her.''

``Poor creature!'' Amelia said.

``And if she can get no friend, she says she thinks she'll die,'' Jos
proceeded in a low tremulous voice.  ``God bless my soul!  do you know
that she tried to kill herself? She carries laudanum with her---I saw
the bottle in her room---such a miserable little room---at a third-rate
house, the Elephant, up in the roof at the top of all.  I went there.''

This did not seem to affect Emmy.  She even smiled a little. Perhaps
she figured Jos to herself panting up the stair.

``She's beside herself with grief,'' he resumed.  ``The agonies that woman
has endured are quite frightful to hear of.  She had a little boy, of
the same age as Georgy.''

``Yes, yes, I think I remember,'' Emmy remarked. ``Well?''

``The most beautiful child ever seen,'' Jos said, who was very fat, and
easily moved, and had been touched by the story Becky told; ``a perfect
angel, who adored his mother.  The ruffians tore him shrieking out of
her arms, and have never allowed him to see her.''

``Dear Joseph,'' Emmy cried out, starting up at once, ``let us go and see
her this minute.'' And she ran into her adjoining bedchamber, tied on
her bonnet in a flutter, came out with her shawl on her arm, and
ordered Dobbin to follow.

He went and put her shawl---it was a white cashmere, consigned to her by
the Major himself from India---over her shoulders.  He saw there was
nothing for it but to obey, and she put her hand into his arm, and they
went away.

``It is number 92, up four pair of stairs,'' Jos said, perhaps not very
willing to ascend the steps again; but he placed himself in the window
of his drawing-room, which commands the place on which the Elephant
stands, and saw the pair marching through the market.

It was as well that Becky saw them too from her garret, for she and the
two students were chattering and laughing there; they had been joking
about the appearance of Becky's grandpapa---whose arrival and departure
they had witnessed---but she had time to dismiss them, and have her
little room clear before the landlord of the Elephant, who knew that
Mrs.\ Osborne was a great favourite at the Serene Court, and respected
her accordingly, led the way up the stairs to the roof story,
encouraging Miladi and the Herr Major as they achieved the ascent.

``Gracious lady, gracious lady!'' said the landlord, knocking at Becky's
door; he had called her Madame the day before, and was by no means
courteous to her.

``Who is it?'' Becky said, putting out her head, and she gave a little
scream.  There stood Emmy in a tremble, and Dobbin, the tall Major,
with his cane.

He stood still watching, and very much interested at the scene; but
Emmy sprang forward with open arms towards Rebecca, and forgave her at
that moment, and embraced her and kissed her with all her heart. Ah,
poor wretch, when was your lip pressed before by such pure kisses?



\chapter{\foreign{Amantium Irae}}

Frankness and kindness like Amelia's were likely to touch even such a
hardened little reprobate as Becky.  She returned Emmy's caresses and
kind speeches with something very like gratitude, and an emotion which,
if it was not lasting, for a moment was almost genuine.  That was a
lucky stroke of hers about the child ``torn from her arms shrieking.'' It
was by that harrowing misfortune that Becky had won her friend back,
and it was one of the very first points, we may be certain, upon which
our poor simple little Emmy began to talk to her new-found acquaintance.

``And so they took your darling child from you?'' our simpleton cried
out.  ``Oh, Rebecca, my poor dear suffering friend, I know what it is to
lose a boy, and to feel for those who have lost one.  But please Heaven
yours will be restored to you, as a merciful merciful Providence has
brought me back mine.''

``The child, my child? Oh, yes, my agonies were frightful,'' Becky owned,
not perhaps without a twinge of conscience. It jarred upon her to be
obliged to commence instantly to tell lies in reply to so much
confidence and simplicity.  But that is the misfortune of beginning
with this kind of forgery.  When one fib becomes due as it were, you
must forge another to take up the old acceptance; and so the stock of
your lies in circulation inevitably multiplies, and the danger of
detection increases every day.

``My agonies,'' Becky continued, ``were terrible (I hope she won't sit
down on the bottle) when they took him away from me; I thought I should
die; but I fortunately had a brain fever, during which my doctor gave
me up, and---and I recovered, and---and here I am, poor and friendless.''

``How old is he?'' Emmy asked.

``Eleven,'' said Becky.

``Eleven!'' cried the other.  ``Why, he was born the same year with
Georgy, who is---''

``I know, I know,'' Becky cried out, who had in fact quite forgotten all
about little Rawdon's age.  ``Grief has made me forget so many things,
dearest Amelia.  I am very much changed:  half-wild sometimes.  He was
eleven when they took him away from me.  Bless his sweet face; I have
never seen it again.''

``Was he fair or dark?'' went on that absurd little Emmy.  ``Show me his
hair.''

Becky almost laughed at her simplicity.  ``Not to-day, love---some other
time, when my trunks arrive from Leipzig, whence I came to this
place---and a little drawing of him, which I made in happy days.''

``Poor Becky, poor Becky!'' said Emmy.  ``How thankful, how thankful I
ought to be''; (though I doubt whether that practice of piety inculcated
upon us by our womankind in early youth, namely, to be thankful because
we are better off than somebody else, be a very rational religious
exercise) and then she began to think, as usual, how her son was the
handsomest, the best, and the cleverest boy in the whole world.

``You will see my Georgy,'' was the best thing Emmy could think of to
console Becky.  If anything could make her comfortable that would.

And so the two women continued talking for an hour or more, during
which Becky had the opportunity of giving her new friend a full and
complete version of her private history.  She showed how her marriage
with Rawdon Crawley had always been viewed by the family with feelings
of the utmost hostility; how her sister-in-law (an artful woman) had
poisoned her husband's mind against her; how he had formed odious
connections, which had estranged his affections from her:  how she had
borne everything---poverty, neglect, coldness from the being whom she
most loved---and all for the sake of her child; how, finally, and by the
most flagrant outrage, she had been driven into demanding a separation
from her husband, when the wretch did not scruple to ask that she
should sacrifice her own fair fame so that he might procure advancement
through the means of a very great and powerful but unprincipled
man---the Marquis of Steyne, indeed.  The atrocious monster!

This part of her eventful history Becky gave with the utmost feminine
delicacy and the most indignant virtue. Forced to fly her husband's
roof by this insult, the coward had pursued his revenge by taking her
child from her. And thus Becky said she was a wanderer, poor,
unprotected, friendless, and wretched.

Emmy received this story, which was told at some length, as those
persons who are acquainted with her character may imagine that she
would.  She quivered with indignation at the account of the conduct of
the miserable Rawdon and the unprincipled Steyne.  Her eyes made notes
of admiration for every one of the sentences in which Becky described
the persecutions of her aristocratic relatives and the falling away of
her husband. (Becky did not abuse him.  She spoke rather in sorrow than
in anger.  She had loved him only too fondly: and was he not the father
of her boy?) And as for the separation scene from the child, while
Becky was reciting it, Emmy retired altogether behind her
pocket-handkerchief, so that the consummate little tragedian must have
been charmed to see the effect which her performance produced on her
audience.

Whilst the ladies were carrying on their conversation, Amelia's
constant escort, the Major (who, of course, did not wish to interrupt
their conference, and found himself rather tired of creaking about the
narrow stair passage of which the roof brushed the nap from his hat)
descended to the ground-floor of the house and into the great room
common to all the frequenters of the Elephant, out of which the stair
led.  This apartment is always in a fume of smoke and liberally
sprinkled with beer.  On a dirty table stand scores of corresponding
brass candlesticks with tallow candles for the lodgers, whose keys hang
up in rows over the candles.  Emmy had passed blushing through the room
anon, where all sorts of people were collected; Tyrolese glove-sellers
and Danubian linen-merchants, with their packs; students recruiting
themselves with butterbrods and meat; idlers, playing cards or dominoes
on the sloppy, beery tables; tumblers refreshing during the cessation
of their performances---in a word, all the fumum and strepitus of a
German inn in fair time.  The waiter brought the Major a mug of beer,
as a matter of course, and he took out a cigar and amused himself with
that pernicious vegetable and a newspaper until his charge should come
down to claim him.

Max and Fritz came presently downstairs, their caps on one side, their
spurs jingling, their pipes splendid with coats of arms and full-blown
tassels, and they hung up the key of No.\ 90 on the board and called for
the ration of butterbrod and beer.  The pair sat down by the Major and
fell into a conversation of which he could not help hearing somewhat.
It was mainly about ``Fuchs'' and ``Philister,'' and duels and
drinking-bouts at the neighbouring University of Schoppenhausen, from
which renowned seat of learning they had just come in the Eilwagen,
with Becky, as it appeared, by their side, and in order to be present
at the bridal fetes at Pumpernickel.

``The title Englanderinn seems to be en bays de gonnoisance,'' said Max,
who knew the French language, to Fritz, his comrade.  ``After the fat
grandfather went away, there came a pretty little compatriot.  I heard
them chattering and whimpering together in the little woman's chamber.''

``We must take the tickets for her concert,'' Fritz said. ``Hast thou any
money, Max?''

``Bah,'' said the other, ``the concert is a concert in nubibus.  Hans said
that she advertised one at Leipzig, and the Burschen took many tickets.
But she went off without singing.  She said in the coach yesterday that
her pianist had fallen ill at Dresden.  She cannot sing, it is my
belief: her voice is as cracked as thine, O thou beer-soaking Renowner!''

``It is cracked; I hear her trying out of her window a schrecklich
English ballad, called 'De Rose upon de Balgony.'''

``Saufen and singen go not together,'' observed Fritz with the red nose,
who evidently preferred the former amusement.  ``No, thou shalt take
none of her tickets. She won money at the trente and quarante last
night.  I saw her:  she made a little English boy play for her. We will
spend thy money there or at the theatre, or we will treat her to French
wine or Cognac in the Aurelius Garden, but the tickets we will not buy.
What sayest thou? Yet, another mug of beer?'' and one and another
successively having buried their blond whiskers in the mawkish draught,
curled them and swaggered off into the fair.

The Major, who had seen the key of No.\ 90 put up on its hook and had
heard the conversation of the two young University bloods, was not at a
loss to understand that their talk related to Becky.  ``The little devil
is at her old tricks,'' he thought, and he smiled as he recalled old
days, when he had witnessed the desperate flirtation with Jos and the
ludicrous end of that adventure. He and George had often laughed over
it subsequently, and until a few weeks after George's marriage, when he
also was caught in the little Circe's toils, and had an understanding
with her which his comrade certainly suspected, but preferred to
ignore.  William was too much hurt or ashamed to ask to fathom that
disgraceful mystery, although once, and evidently with remorse on his
mind, George had alluded to it. It was on the morning of Waterloo, as
the young men stood together in front of their line, surveying the
black masses of Frenchmen who crowned the opposite heights, and as the
rain was coming down, ``I have been mixing in a foolish intrigue with a
woman,'' George said. ``I am glad we were marched away.  If I drop, I
hope Emmy will never know of that business.  I wish to God it had never
been begun!'' And William was pleased to think, and had more than once
soothed poor George's widow with the narrative, that Osborne, after
quitting his wife, and after the action of Quatre Bras, on the first
day, spoke gravely and affectionately to his comrade of his father and
his wife.  On these facts, too, William had insisted very strongly in
his conversations with the elder Osborne, and had thus been the means
of reconciling the old gentleman to his son's memory, just at the close
of the elder man's life.

``And so this devil is still going on with her intrigues,'' thought
William.  ``I wish she were a hundred miles from here.  She brings
mischief wherever she goes.'' And he was pursuing these forebodings and
this uncomfortable train of thought, with his head between his hands,
and the Pumpernickel Gazette of last week unread under his nose, when
somebody tapped his shoulder with a parasol, and he looked up and saw
Mrs.\ Amelia.

This woman had a way of tyrannizing over Major Dobbin (for the weakest
of all people will domineer over somebody), and she ordered him about,
and patted him, and made him fetch and carry just as if he was a great
Newfoundland dog.  He liked, so to speak, to jump into the water if she
said ``High, Dobbin!'' and to trot behind her with her reticule in his
mouth.  This history has been written to very little purpose if the
reader has not perceived that the Major was a spooney.

``Why did you not wait for me, sir, to escort me downstairs?'' she said,
giving a little toss of her head and a most sarcastic curtsey.

``I couldn't stand up in the passage,'' he answered with a comical
deprecatory look; and, delighted to give her his arm and to take her
out of the horrid smoky place, he would have walked off without even so
much as remembering the waiter, had not the young fellow run after him
and stopped him on the threshold of the Elephant to make him pay for
the beer which he had not consumed.  Emmy laughed:  she called him a
naughty man, who wanted to run away in debt, and, in fact, made some
jokes suitable to the occasion and the small-beer. She was in high
spirits and good humour, and tripped across the market-place very
briskly.  She wanted to see Jos that instant.  The Major laughed at the
impetuous affection Mrs.\ Amelia exhibited; for, in truth, it was not
very often that she wanted her brother ``that instant.''  They found the
civilian in his saloon on the first-floor; he had been pacing the room,
and biting his nails, and looking over the market-place towards the
Elephant a hundred times at least during the past hour whilst Emmy was
closeted with her friend in the garret and the Major was beating the
tattoo on the sloppy tables of the public room below, and he was, on
his side too, very anxious to see Mrs.\ Osborne.

``Well?'' said he.

``The poor dear creature, how she has suffered!'' Emmy said.

``God bless my soul, yes,'' Jos said, wagging his head, so that his
cheeks quivered like jellies.

``She may have Payne's room, who can go upstairs,'' Emmy continued. Payne
was a staid English maid and personal attendant upon Mrs.\ Osborne, to
whom the courier, as in duty bound, paid court, and whom Georgy used to
``lark'' dreadfully with accounts of German robbers and ghosts.  She
passed her time chiefly in grumbling, in ordering about her mistress,
and in stating her intention to return the next morning to her native
village of Clapham.  ``She may have Payne's room,'' Emmy said.

``Why, you don't mean to say you are going to have that woman into the
house?'' bounced out the Major, jumping up.

``Of course we are,'' said Amelia in the most innocent way in the world.
``Don't be angry and break the furniture, Major Dobbin.  Of course we
are going to have her here.''

``Of course, my dear,'' Jos said.

``The poor creature, after all her sufferings,'' Emmy continued; ``her
horrid banker broken and run away; her husband---wicked wretch---having
deserted her and taken her child away from her'' (here she doubled her
two little fists and held them in a most menacing attitude before her,
so that the Major was charmed to see such a dauntless virago) ``the poor
dear thing!  quite alone and absolutely forced to give lessons in
singing to get her bread---and not have her here!''

``Take lessons, my dear Mrs.\ George,'' cried the Major, ``but don't have
her in the house.  I implore you don't.''

``Pooh,'' said Jos.

``You who are always good and kind---always used to be at any rate---I'm
astonished at you, Major William,'' Amelia cried.  ``Why, what is the
moment to help her but when she is so miserable? Now is the time to be
of service to her.  The oldest friend I ever had, and not---''

``She was not always your friend, Amelia,'' the Major said, for he was
quite angry.  This allusion was too much for Emmy, who, looking the
Major almost fiercely in the face, said, ``For shame, Major Dobbin!'' and
after having fired this shot, she walked out of the room with a most
majestic air and shut her own door briskly on herself and her outraged
dignity.

``To allude to THAT!'' she said, when the door was closed.  ``Oh, it was
cruel of him to remind me of it,'' and she looked up at George's
picture, which hung there as usual, with the portrait of the boy
underneath.  ``It was cruel of him.  If I had forgiven it, ought he to
have spoken?  No.  And it is from his own lips that I know how wicked
and groundless my jealousy was; and that you were pure---oh, yes, you
were pure, my saint in heaven!''

She paced the room, trembling and indignant.  She went and leaned on
the chest of drawers over which the picture hung, and gazed and gazed
at it.  Its eyes seemed to look down on her with a reproach that
deepened as she looked. The early dear, dear memories of that brief
prime of love rushed back upon her.  The wound which years had scarcely
cicatrized bled afresh, and oh, how bitterly!  She could not bear the
reproaches of the husband there before her.  It couldn't be.  Never,
never.

Poor Dobbin; poor old William!  That unlucky word had undone the work
of many a year---the long laborious edifice of a life of love and
constancy---raised too upon what secret and hidden foundations, wherein
lay buried passions, uncounted struggles, unknown sacrifices---a little
word was spoken, and down fell the fair palace of hope---one word, and
away flew the bird which he had been trying all his life to lure!

William, though he saw by Amelia's looks that a great crisis had come,
nevertheless continued to implore Sedley, in the most energetic terms,
to beware of Rebecca; and he eagerly, almost frantically, adjured Jos
not to receive her.  He besought Mr.\ Sedley to inquire at least
regarding her; told him how he had heard that she was in the company of
gamblers and people of ill repute; pointed out what evil she had done
in former days, how she and Crawley had misled poor George into ruin,
how she was now parted from her husband, by her own confession, and,
perhaps, for good reason.  What a dangerous companion she would be for
his sister, who knew nothing of the affairs of the world!  William
implored Jos, with all the eloquence which he could bring to bear, and
a great deal more energy than this quiet gentleman was ordinarily in
the habit of showing, to keep Rebecca out of his household.

Had he been less violent, or more dexterous, he might have succeeded in
his supplications to Jos; but the civilian was not a little jealous of
the airs of superiority which the Major constantly exhibited towards
him, as he fancied (indeed, he had imparted his opinions to Mr.\ Kirsch,
the courier, whose bills Major Dobbin checked on this journey, and who
sided with his master), and he began a blustering speech about his
competency to defend his own honour, his desire not to have his affairs
meddled with, his intention, in fine, to rebel against the Major, when
the colloquy---rather a long and stormy one---was put an end to in the
simplest way possible, namely, by the arrival of Mrs.\ Becky, with a
porter from the Elephant Hotel in charge of her very meagre baggage.

She greeted her host with affectionate respect and made a shrinking,
but amicable salutation to Major Dobbin, who, as her instinct assured
her at once, was her enemy, and had been speaking against her; and the
bustle and clatter consequent upon her arrival brought Amelia out of
her room.  Emmy went up and embraced her guest with the greatest
warmth, and took no notice of the Major, except to fling him an angry
look---the most unjust and scornful glance that had perhaps ever
appeared in that poor little woman's face since she was born.  But she
had private reasons of her own, and was bent upon being angry with him.
And Dobbin, indignant at the injustice, not at the defeat, went off,
making her a bow quite as haughty as the killing curtsey with which the
little woman chose to bid him farewell.

He being gone, Emmy was particularly lively and affectionate to
Rebecca, and bustled about the apartments and installed her guest in
her room with an eagerness and activity seldom exhibited by our placid
little friend.  But when an act of injustice is to be done, especially
by weak people, it is best that it should be done quickly, and Emmy
thought she was displaying a great deal of firmness and proper feeling
and veneration for the late Captain Osborne in her present behaviour.

Georgy came in from the fetes for dinner-time and found four covers
laid as usual; but one of the places was occupied by a lady, instead of
by Major Dobbin. ``Hullo!  where's Dob?'' the young gentleman asked with
his usual simplicity of language.  ``Major Dobbin is dining out, I
suppose,'' his mother said, and, drawing the boy to her, kissed him a
great deal, and put his hair off his forehead, and introduced him to
Mrs.\ Crawley.  ``This is my boy, Rebecca,'' Mrs.\ Osborne said---as much as
to say---can the world produce anything like that? Becky looked at him
with rapture and pressed his hand fondly. ``Dear boy!'' she said---``he is
just like my---'' Emotion choked her further utterance, but Amelia
understood, as well as if she had spoken, that Becky was thinking of
her own blessed child.  However, the company of her friend consoled
Mrs.\ Crawley, and she ate a very good dinner.

During the repast, she had occasion to speak several times, when Georgy
eyed her and listened to her.  At the desert Emmy was gone out to
superintend further domestic arrangements; Jos was in his great chair
dozing over Galignani; Georgy and the new arrival sat close to each
other---he had continued to look at her knowingly more than once, and at
last he laid down the nutcrackers.

``I say,'' said Georgy.

``What do you say?'' Becky said, laughing.

``You're the lady I saw in the mask at the Rouge et Noir.''

``Hush!  you little sly creature,'' Becky said, taking up his hand and
kissing it.  ``Your uncle was there too, and Mamma mustn't know.''

``Oh, no---not by no means,'' answered the little fellow.

``You see we are quite good friends already,'' Becky said to Emmy, who
now re-entered; and it must be owned that Mrs.\ Osborne had introduced a
most judicious and amiable companion into her house.

William, in a state of great indignation, though still unaware of all
the treason that was in store for him, walked about the town wildly
until he fell upon the Secretary of Legation, Tapeworm, who invited him
to dinner.  As they were discussing that meal, he took occasion to ask
the Secretary whether he knew anything about a certain Mrs.\ Rawdon
Crawley, who had, he believed, made some noise in London; and then
Tapeworm, who of course knew all the London gossip, and was besides a
relative of Lady Gaunt, poured out into the astonished Major's ears
such a history about Becky and her husband as astonished the querist,
and supplied all the points of this narrative, for it was at that very
table years ago that the present writer had the pleasure of hearing the
tale.  Tufto, Steyne, the Crawleys, and their history---everything
connected with Becky and her previous life passed under the record of
the bitter diplomatist. He knew everything and a great deal besides,
about all the world---in a word, he made the most astounding revelations
to the simple-hearted Major.  When Dobbin said that Mrs.\ Osborne and
Mr.\ Sedley had taken her into their house, Tapeworm burst into a peal
of laughter which shocked the Major, and asked if they had not better
send into the prison and take in one or two of the gentlemen in shaved
heads and yellow jackets who swept the streets of Pumpernickel, chained
in pairs, to board and lodge, and act as tutor to that little
scapegrace Georgy.

This information astonished and horrified the Major not a little. It
had been agreed in the morning (before meeting with Rebecca) that
Amelia should go to the Court ball that night.  There would be the
place where he should tell her.  The Major went home, and dressed
himself in his uniform, and repaired to Court, in hopes to see Mrs.\ %
Osborne.  She never came.  When he returned to his lodgings all the
lights in the Sedley tenement were put out.  He could not see her till
the morning.  I don't know what sort of a night's rest he had with this
frightful secret in bed with him.

At the earliest convenient hour in the morning he sent his servant
across the way with a note, saying that he wished very particularly to
speak with her.  A message came back to say that Mrs.\ Osborne was
exceedingly unwell and was keeping her room.

She, too, had been awake all that night.  She had been thinking of a
thing which had agitated her mind a hundred times before.  A hundred
times on the point of yielding, she had shrunk back from a sacrifice
which she felt was too much for her.  She couldn't, in spite of his
love and constancy and her own acknowledged regard, respect, and
gratitude.  What are benefits, what is constancy, or merit? One curl of
a girl's ringlet, one hair of a whisker, will turn the scale against
them all in a minute. They did not weigh with Emmy more than with other
women.  She had tried them; wanted to make them pass; could not; and
the pitiless little woman had found a pretext, and determined to be
free.

When at length, in the afternoon, the Major gained admission to Amelia,
instead of the cordial and affectionate greeting, to which he had been
accustomed now for many a long day, he received the salutation of a
curtsey, and of a little gloved hand, retracted the moment after it was
accorded to him.

Rebecca, too, was in the room, and advanced to meet him with a smile
and an extended hand.  Dobbin drew back rather confusedly, ``I---I beg
your pardon, m'am,'' he said; ``but I am bound to tell you that it is not
as your friend that I am come here now.''

``Pooh!  damn; don't let us have this sort of thing!'' Jos cried out,
alarmed, and anxious to get rid of a scene.

``I wonder what Major Dobbin has to say against Rebecca?'' Amelia said in
a low, clear voice with a slight quiver in it, and a very determined
look about the eyes.

``I will not have this sort of thing in my house,'' Jos again interposed.
``I say I will not have it; and Dobbin, I beg, sir, you'll stop it.'' And
he looked round, trembling and turning very red, and gave a great puff,
and made for his door.

``Dear friend!'' Rebecca said with angelic sweetness, ``do hear what Major
Dobbin has to say against me.''

``I will not hear it, I say,'' squeaked out Jos at the top of his voice,
and, gathering up his dressing-gown, he was gone.

``We are only two women,'' Amelia said.  ``You can speak now, sir.''

``This manner towards me is one which scarcely becomes you, Amelia,'' the
Major answered haughtily; ``nor I believe am I guilty of habitual
harshness to women.  It is not a pleasure to me to do the duty which I
am come to do.''

``Pray proceed with it quickly, if you please, Major Dobbin,'' said
Amelia, who was more and more in a pet.  The expression of Dobbin's
face, as she spoke in this imperious manner, was not pleasant.

``I came to say---and as you stay, Mrs.\ Crawley, I must say it in your
presence---that I think you---you ought not to form a member of the
family of my friends.  A lady who is separated from her husband, who
travels not under her own name, who frequents public gaming-tables---''

``It was to the ball I went,'' cried out Becky.

``---is not a fit companion for Mrs.\ Osborne and her son,'' Dobbin went
on:  ``and I may add that there are people here who know you, and who
profess to know that regarding your conduct about which I don't even
wish to speak before---before Mrs.\ Osborne.''

``Yours is a very modest and convenient sort of calumny, Major Dobbin,''
Rebecca said.  ``You leave me under the weight of an accusation which,
after all, is unsaid. What is it? Is it unfaithfulness to my husband? I
scorn it and defy anybody to prove it---I defy you, I say.  My honour is
as untouched as that of the bitterest enemy who ever maligned me.  Is
it of being poor, forsaken, wretched, that you accuse me? Yes, I am
guilty of those faults, and punished for them every day.  Let me go,
Emmy.  It is only to suppose that I have not met you, and I am no worse
to-day than I was yesterday.  It is only to suppose that the night is
over and the poor wanderer is on her way.  Don't you remember the song
we used to sing in old, dear old days? I have been wandering ever since
then---a poor castaway, scorned for being miserable, and insulted
because I am alone.  Let me go:  my stay here interferes with the plans
of this gentleman.''

``Indeed it does, madam,'' said the Major.  ``If I have any authority in
this house---''

``Authority, none!'' broke out Amelia ``Rebecca, you stay with me.  I
won't desert you because you have been persecuted, or insult you
because---because Major Dobbin chooses to do so.  Come away, dear.'' And
the two women made towards the door.

William opened it.  As they were going out, however, he took Amelia's
hand and said---``Will you stay a moment and speak to me?''

``He wishes to speak to you away from me,'' said Becky, looking like a
martyr.  Amelia gripped her hand in reply.

``Upon my honour it is not about you that I am going to speak,'' Dobbin
said.  ``Come back, Amelia,'' and she came.  Dobbin bowed to Mrs.\ %
Crawley, as he shut the door upon her.  Amelia looked at him, leaning
against the glass:  her face and her lips were quite white.

``I was confused when I spoke just now,'' the Major said after a pause,
``and I misused the word authority.''

``You did,'' said Amelia with her teeth chattering.

``At least I have claims to be heard,'' Dobbin continued.

``It is generous to remind me of our obligations to you,'' the woman
answered.

``The claims I mean are those left me by George's father,'' William said.

``Yes, and you insulted his memory.  You did yesterday. You know you
did.  And I will never forgive you.  Never!'' said Amelia.  She shot out
each little sentence in a tremor of anger and emotion.

``You don't mean that, Amelia?'' William said sadly. ``You don't mean that
these words, uttered in a hurried moment, are to weigh against a whole
life's devotion? I think that George's memory has not been injured by
the way in which I have dealt with it, and if we are come to bandying
reproaches, I at least merit none from his widow and the mother of his
son.  Reflect, afterwards when---when you are at leisure, and your
conscience will withdraw this accusation.  It does even now.'' Amelia
held down her head.

``It is not that speech of yesterday,'' he continued, ``which moves you.
That is but the pretext, Amelia, or I have loved you and watched you
for fifteen years in vain. Have I not learned in that time to read all
your feelings and look into your thoughts? I know what your heart is
capable of:  it can cling faithfully to a recollection and cherish a
fancy, but it can't feel such an attachment as mine deserves to mate
with, and such as I would have won from a woman more generous than you.
No, you are not worthy of the love which I have devoted to you. I knew
all along that the prize I had set my life on was not worth the
winning; that I was a fool, with fond fancies, too, bartering away my
all of truth and ardour against your little feeble remnant of love.  I
will bargain no more:  I withdraw.  I find no fault with you.  You are
very good-natured, and have done your best, but you couldn't---you
couldn't reach up to the height of the attachment which I bore you, and
which a loftier soul than yours might have been proud to share.
Good-bye, Amelia! I have watched your struggle.  Let it end.  We are
both weary of it.''

Amelia stood scared and silent as William thus suddenly broke the chain
by which she held him and declared his independence and superiority.
He had placed himself at her feet so long that the poor little woman
had been accustomed to trample upon him.  She didn't wish to marry him,
but she wished to keep him.  She wished to give him nothing, but that
he should give her all.  It is a bargain not unfrequently levied in
love.

William's sally had quite broken and cast her down. HER assault was
long since over and beaten back.

``Am I to understand then, that you are going---away, William?'' she said.

He gave a sad laugh.  ``I went once before,'' he said, ``and came back
after twelve years.  We were young then, Amelia.  Good-bye.  I have
spent enough of my life at this play.''

Whilst they had been talking, the door into Mrs.\ Osborne's room had
opened ever so little; indeed, Becky had kept a hold of the handle and
had turned it on the instant when Dobbin quitted it, and she heard
every word of the conversation that had passed between these two. ``What
a noble heart that man has,'' she thought, ``and how shamefully that
woman plays with it!'' She admired Dobbin; she bore him no rancour for
the part he had taken against her.  It was an open move in the game,
and played fairly.  ``Ah!'' she thought, ``if I could have had such a
husband as that---a man with a heart and brains too!  I would not have
minded his large feet''; and running into her room, she absolutely
bethought herself of something, and wrote him a note, beseeching him to
stop for a few days---not to think of going---and that she could serve
him with A.

The parting was over.  Once more poor William walked to the door and
was gone; and the little widow, the author of all this work, had her
will, and had won her victory, and was left to enjoy it as she best
might.  Let the ladies envy her triumph.

At the romantic hour of dinner, Mr.\ Georgy made his appearance and
again remarked the absence of ``Old Dob.'' The meal was eaten in silence
by the party.  Jos's appetite not being diminished, but Emmy taking
nothing at all.

After the meal, Georgy was lolling in the cushions of the old window, a
large window, with three sides of glass abutting from the gable, and
commanding on one side the market-place, where the Elephant is, his
mother being busy hard by, when he remarked symptoms of movement at the
Major's house on the other side of the street.

``Hullo!'' said he, ``there's Dob's trap---they are bringing it out of the
court-yard.'' The ``trap'' in question was a carriage which the Major had
bought for six pounds sterling, and about which they used to rally him
a good deal.

Emmy gave a little start, but said nothing.

``Hullo!'' Georgy continued, ``there's Francis coming out with the
portmanteaus, and Kunz, the one-eyed postilion, coming down the market
with three schimmels. Look at his boots and yellow jacket---ain't he a
rum one? Why---they're putting the horses to Dob's carriage. Is he going
anywhere?''

``Yes,'' said Emmy, ``he is going on a journey.''

``Going on a journey; and when is he coming back?''

``He is---not coming back,'' answered Emmy.

``Not coming back!'' cried out Georgy, jumping up. ``Stay here, sir,''
roared out Jos.  ``Stay, Georgy,'' said his mother with a very sad face.
The boy stopped, kicked about the room, jumped up and down from the
window-seat with his knees, and showed every symptom of uneasiness and
curiosity.

The horses were put to.  The baggage was strapped on.  Francis came out
with his master's sword, cane, and umbrella tied up together, and laid
them in the well, and his desk and old tin cocked-hat case, which he
placed under the seat.  Francis brought out the stained old blue cloak
lined with red camlet, which had wrapped the owner up any time these
fifteen years, and had manchen Sturm erlebt, as a favourite song of
those days said.  It had been new for the campaign of Waterloo and had
covered George and William after the night of Quatre Bras.

Old Burcke, the landlord of the lodgings, came out, then Francis, with
more packages---final packages---then Major William---Burcke wanted to
kiss him.  The Major was adored by all people with whom he had to do.
It was with difficulty he could escape from this demonstration of
attachment.

``By Jove, I will go!'' screamed out George.  ``Give him this,'' said
Becky, quite interested, and put a paper into the boy's hand.  He had
rushed down the stairs and flung across the street in a minute---the
yellow postilion was cracking his whip gently.

William had got into the carriage, released from the embraces of his
landlord.  George bounded in afterwards, and flung his arms round the
Major's neck (as they saw from the window), and began asking him
multiplied questions.  Then he felt in his waistcoat pocket and gave
him a note.  William seized at it rather eagerly, he opened it
trembling, but instantly his countenance changed, and he tore the paper
in two and dropped it out of the carriage.  He kissed Georgy on the
head, and the boy got out, doubling his fists into his eyes, and with
the aid of Francis.  He lingered with his hand on the panel.  Fort,
Schwager!  The yellow postilion cracked his whip prodigiously, up
sprang Francis to the box, away went the schimmels, and Dobbin with his
head on his breast.  He never looked up as they passed under Amelia's
window, and Georgy, left alone in the street, burst out crying in the
face of all the crowd.

Emmy's maid heard him howling again during the night and brought him
some preserved apricots to console him.  She mingled her lamentations
with his.  All the poor, all the humble, all honest folks, all good men
who knew him, loved that kind-hearted and simple gentleman.

As for Emmy, had she not done her duty? She had her picture of George
for a consolation.



\chapter{Which Contains Births, Marriages, and Deaths}

Whatever Becky's private plan might be by which Dobbin's true love was
to be crowned with success, the little woman thought that the secret
might keep, and indeed, being by no means so much interested about
anybody's welfare as about her own, she had a great number of things
pertaining to herself to consider, and which concerned her a great deal
more than Major Dobbin's happiness in this life.

She found herself suddenly and unexpectedly in snug comfortable
quarters, surrounded by friends, kindness, and good-natured simple
people such as she had not met with for many a long day; and, wanderer
as she was by force and inclination, there were moments when rest was
pleasant to her.  As the most hardened Arab that ever careered across
the desert over the hump of a dromedary likes to repose sometimes under
the date-trees by the water, or to come into the cities, walk into the
bazaars, refresh himself in the baths, and say his prayers in the
mosques, before he goes out again marauding, so Jos's tents and pilau
were pleasant to this little Ishmaelite. She picketed her steed, hung
up her weapons, and warmed herself comfortably by his fire.  The halt
in that roving, restless life was inexpressibly soothing and pleasant
to her.

So, pleased herself, she tried with all her might to please everybody;
and we know that she was eminent and successful as a practitioner in
the art of giving pleasure.  As for Jos, even in that little interview
in the garret at the Elephant Inn, she had found means to win back a
great deal of his good-will.  In the course of a week, the civilian was
her sworn slave and frantic admirer.  He didn't go to sleep after
dinner, as his custom was in the much less lively society of Amelia.
He drove out with Becky in his open carriage.  He asked little parties
and invented festivities to do her honour.

Tapeworm, the Charge d'Affaires, who had abused her so cruelly, came to
dine with Jos, and then came every day to pay his respects to Becky.
Poor Emmy, who was never very talkative, and more glum and silent than
ever after Dobbin's departure, was quite forgotten when this superior
genius made her appearance.  The French Minister was as much charmed
with her as his English rival. The German ladies, never particularly
squeamish as regards morals, especially in English people, were
delighted with the cleverness and wit of Mrs.\ Osborne's charming
friend, and though she did not ask to go to Court, yet the most august
and Transparent Personages there heard of her fascinations and were
quite curious to know her.  When it became known that she was noble, of
an ancient English family, that her husband was a Colonel of the Guard,
Excellenz and Governor of an island, only separated from his lady by
one of those trifling differences which are of little account in a
country where Werther is still read and the Wahlverwandtschaften of
Goethe is considered an edifying moral book, nobody thought of refusing
to receive her in the very highest society of the little Duchy; and the
ladies were even more ready to call her du and to swear eternal
friendship for her than they had been to bestow the same inestimable
benefits upon Amelia.  Love and Liberty are interpreted by those simple
Germans in a way which honest folks in Yorkshire and Somersetshire
little understand, and a lady might, in some philosophic and civilized
towns, be divorced ever so many times from her respective husbands and
keep her character in society.  Jos's house never was so pleasant since
he had a house of his own as Rebecca caused it to be. She sang, she
played, she laughed, she talked in two or three languages, she brought
everybody to the house, and she made Jos believe that it was his own
great social talents and wit which gathered the society of the place
round about him.

As for Emmy, who found herself not in the least mistress of her own
house, except when the bills were to be paid, Becky soon discovered the
way to soothe and please her.  She talked to her perpetually about
Major Dobbin sent about his business, and made no scruple of declaring
her admiration for that excellent, high-minded gentleman, and of
telling Emmy that she had behaved most cruelly regarding him. Emmy
defended her conduct and showed that it was dictated only by the purest
religious principles; that a woman once, \&c., and to such an angel as
him whom she had had the good fortune to marry, was married forever;
but she had no objection to hear the Major praised as much as ever
Becky chose to praise him, and indeed, brought the conversation round
to the Dobbin subject a score of times every day.

Means were easily found to win the favour of Georgy and the servants.
Amelia's maid, it has been said, was heart and soul in favour of the
generous Major.  Having at first disliked Becky for being the means of
dismissing him from the presence of her mistress, she was reconciled to
Mrs.\ Crawley subsequently, because the latter became William's most
ardent admirer and champion.  And in those nightly conclaves in which
the two ladies indulged after their parties, and while Miss Payne was
``brushing their 'airs,'' as she called the yellow locks of the one and
the soft brown tresses of the other, this girl always put in her word
for that dear good gentleman Major Dobbin.  Her advocacy did not make
Amelia angry any more than Rebecca's admiration of him.  She made
George write to him constantly and persisted in sending Mamma's kind
love in a postscript.  And as she looked at her husband's portrait of
nights, it no longer reproached her---perhaps she reproached it, now
William was gone.

Emmy was not very happy after her heroic sacrifice. She was very
distraite, nervous, silent, and ill to please. The family had never
known her so peevish.  She grew pale and ill.  She used to try to sing
certain songs (``Einsam bin ich nicht alleine,'' was one of them, that
tender love-song of Weber's which in old-fashioned days, young ladies,
and when you were scarcely born, showed that those who lived before you
knew too how to love and to sing) certain songs, I say, to which the
Major was partial; and as she warbled them in the twilight in the
drawing-room, she would break off in the midst of the song, and walk
into her neighbouring apartment, and there, no doubt, take refuge in
the miniature of her husband.

Some books still subsisted, after Dobbin's departure, with his name
written in them; a German dictionary, for instance, with ``William
Dobbin, ---th Reg.,'' in the fly-leaf; a guide-book with his initials;
and one or two other volumes which belonged to the Major.  Emmy cleared
these away and put them on the drawers, where she placed her work-box,
her desk, her Bible, and prayer-book, under the pictures of the two
Georges.  And the Major, on going away, having left his gloves behind
him, it is a fact that Georgy, rummaging his mother's desk some time
afterwards, found the gloves neatly folded up and put away in what they
call the secret-drawers of the desk.

Not caring for society, and moping there a great deal, Emmy's chief
pleasure in the summer evenings was to take long walks with Georgy
(during which Rebecca was left to the society of Mr.\ Joseph), and then
the mother and son used to talk about the Major in a way which even
made the boy smile.  She told him that she thought Major William was
the best man in all the world---the gentlest and the kindest, the
bravest and the humblest.  Over and over again she told him how they
owed everything which they possessed in the world to that kind friend's
benevolent care of them; how he had befriended them all through their
poverty and misfortunes; watched over them when nobody cared for them;
how all his comrades admired him though he never spoke of his own
gallant actions; how Georgy's father trusted him beyond all other men,
and had been constantly befriended by the good William.  ``Why, when
your papa was a little boy,'' she said, ``he often told me that it was
William who defended him against a tyrant at the school where they
were; and their friendship never ceased from that day until the last,
when your dear father fell.''

``Did Dobbin kill the man who killed Papa?'' Georgy said.  ``I'm sure he
did, or he would if he could have caught him, wouldn't he, Mother? When
I'm in the Army, won't I hate the French?---that's all.''

In such colloquies the mother and the child passed a great deal of
their time together.  The artless woman had made a confidant of the
boy.  He was as much William's friend as everybody else who knew him
well.

By the way, Mrs.\ Becky, not to be behind hand in sentiment, had got a
miniature too hanging up in her room, to the surprise and amusement of
most people, and the delight of the original, who was no other than our
friend Jos.  On her first coming to favour the Sedleys with a visit,
the little woman, who had arrived with a remarkably small shabby kit,
was perhaps ashamed of the meanness of her trunks and bandboxes, and
often spoke with great respect about her baggage left behind at
Leipzig, which she must have from that city.  When a traveller talks to
you perpetually about the splendour of his luggage, which he does not
happen to have with him, my son, beware of that traveller!  He is, ten
to one, an impostor.

Neither Jos nor Emmy knew this important maxim.  It seemed to them of
no consequence whether Becky had a quantity of very fine clothes in
invisible trunks; but as her present supply was exceedingly shabby,
Emmy supplied her out of her own stores, or took her to the best
milliner in the town and there fitted her out.  It was no more torn
collars now, I promise you, and faded silks trailing off at the
shoulder.  Becky changed her habits with her situation in life---the
rouge-pot was suspended---another excitement to which she had accustomed
herself was also put aside, or at least only indulged in in privacy, as
when she was prevailed on by Jos of a summer evening, Emmy and the boy
being absent on their walks, to take a little spirit-and-water.  But if
she did not indulge---the courier did: that rascal Kirsch could not be
kept from the bottle, nor could he tell how much he took when he
applied to it.  He was sometimes surprised himself at the way in which
Mr.\ Sedley's Cognac diminished.  Well, well, this is a painful subject.
Becky did not very likely indulge so much as she used before she
entered a decorous family.

At last the much-bragged-about boxes arrived from Leipzig; three of
them not by any means large or splendid; nor did Becky appear to take
out any sort of dresses or ornaments from the boxes when they did
arrive.  But out of one, which contained a mass of her papers (it was
that very box which Rawdon Crawley had ransacked in his furious hunt
for Becky's concealed money), she took a picture with great glee, which
she pinned up in her room, and to which she introduced Jos.  It was the
portrait of a gentleman in pencil, his face having the advantage of
being painted up in pink.  He was riding on an elephant away from some
cocoa-nut trees and a pagoda: it was an Eastern scene.

``God bless my soul, it is my portrait,'' Jos cried out. It was he
indeed, blooming in youth and beauty, in a nankeen jacket of the cut of
1804.  It was the old picture that used to hang up in Russell Square.

``I bought it,'' said Becky in a voice trembling with emotion; ``I went to
see if I could be of any use to my kind friends.  I have never parted
with that picture---I never will.''

``Won't you?'' Jos cried with a look of unutterable rapture and
satisfaction.  ``Did you really now value it for my sake?''

``You know I did, well enough,'' said Becky; ``but why speak---why
think---why look back!  It is too late now!''

That evening's conversation was delicious for Jos. Emmy only came in to
go to bed very tired and unwell. Jos and his fair guest had a charming
tete-a-tete, and his sister could hear, as she lay awake in her
adjoining chamber, Rebecca singing over to Jos the old songs of 1815.
He did not sleep, for a wonder, that night, any more than Amelia.

It was June, and, by consequence, high season in London; Jos, who read
the incomparable Galignani (the exile's best friend) through every day,
used to favour the ladies with extracts from his paper during their
breakfast.  Every week in this paper there is a full account of
military movements, in which Jos, as a man who had seen service, was
especially interested.  On one occasion he read out---``Arrival of the
---th regiment.  Gravesend, June 20.---The Ramchunder, East Indiaman,
came into the river this morning, having on board 14 officers, and 132
rank and file of this gallant corps.  They have been absent from
England fourteen years, having been embarked the year after Waterloo,
in which glorious conflict they took an active part, and having
subsequently distinguished themselves in the Burmese war.  The veteran
colonel, Sir Michael O'Dowd, K.C.B., with his lady and sister, landed
here yesterday, with Captains Posky, Stubble, Macraw, Malony;
Lieutenants Smith, Jones, Thompson, F. Thomson; Ensigns Hicks and
Grady; the band on the pier playing the national anthem, and the crowd
loudly cheering the gallant veterans as they went into Wayte's hotel,
where a sumptuous banquet was provided for the defenders of Old
England.  During the repast, which we need not say was served up in
Wayte's best style, the cheering continued so enthusiastically that
Lady O'Dowd and the Colonel came forward to the balcony and drank the
healths of their fellow-countrymen in a bumper of Wayte's best claret.''

On a second occasion Jos read a brief announcement---Major Dobbin had
joined the ---th regiment at Chatham; and subsequently he promulgated
accounts of the presentations at the Drawing-room of Colonel Sir
Michael O'Dowd, K.C.B., Lady O'Dowd (by Mrs.\ Malloy Malony of
Ballymalony), and Miss Glorvina O'Dowd (by Lady O'Dowd).  Almost
directly after this, Dobbin's name appeared among the Lieutenant-Colonels:
for old Marshal Tiptoff had died during the passage of the
---th from Madras, and the Sovereign was pleased to advance Colonel Sir
Michael O'Dowd to the rank of Major-General on his return to England,
with an intimation that he should be Colonel of the distinguished
regiment which he had so long commanded.

Amelia had been made aware of some of these movements.  The
correspondence between George and his guardian had not ceased by any
means:  William had even written once or twice to her since his
departure, but in a manner so unconstrainedly cold that the poor woman
felt now in her turn that she had lost her power over him and that, as
he had said, he was free.  He had left her, and she was wretched.  The
memory of his almost countless services, and lofty and affectionate
regard, now presented itself to her and rebuked her day and night.  She
brooded over those recollections according to her wont, saw the purity
and beauty of the affection with which she had trifled, and reproached
herself for having flung away such a treasure.

It was gone indeed.  William had spent it all out.  He loved her no
more, he thought, as he had loved her. He never could again.  That sort
of regard, which he had proffered to her for so many faithful years,
can't be flung down and shattered and mended so as to show no scars.
The little heedless tyrant had so destroyed it.  No, William thought
again and again, ``It was myself I deluded and persisted in cajoling;
had she been worthy of the love I gave her, she would have returned it
long ago.  It was a fond mistake.  Isn't the whole course of life made
up of such? And suppose I had won her, should I not have been
disenchanted the day after my victory? Why pine, or be ashamed of my
defeat?'' The more he thought of this long passage of his life, the more
clearly he saw his deception.  ``I'll go into harness again,'' he said,
``and do my duty in that state of life in which it has pleased Heaven to
place me.  I will see that the buttons of the recruits are properly
bright and that the sergeants make no mistakes in their accounts.  I
will dine at mess and listen to the Scotch surgeon telling his stories.
When I am old and broken, I will go on half-pay, and my old sisters
shall scold me.  I have geliebt und gelebet, as the girl in
'Wallenstein' says.  I am done.  Pay the bills and get me a cigar:
find out what there is at the play to-night, Francis; to-morrow we
cross by the Batavier.'' He made the above speech, whereof Francis only
heard the last two lines, pacing up and down the Boompjes at Rotterdam.
The Batavier was lying in the basin.  He could see the place on the
quarter-deck where he and Emmy had sat on the happy voyage out.  What
had that little Mrs.\ Crawley to say to him? Psha; to-morrow we will put
to sea, and return to England, home, and duty!

After June all the little Court Society of Pumpernickel used to
separate, according to the German plan, and make for a hundred
watering-places, where they drank at the wells, rode upon donkeys,
gambled at the redoutes if they had money and a mind, rushed with
hundreds of their kind to gourmandise at the tables d'hote, and idled
away the summer.  The English diplomatists went off to Teoplitz and
Kissingen, their French rivals shut up their chancellerie and whisked
away to their darling Boulevard de Gand. The Transparent reigning
family took too to the waters, or retired to their hunting lodges.
Everybody went away having any pretensions to politeness, and of
course, with them, Doctor von Glauber, the Court Doctor, and his
Baroness.  The seasons for the baths were the most productive periods
of the Doctor's practice---he united business with pleasure, and his
chief place of resort was Ostend, which is much frequented by Germans,
and where the Doctor treated himself and his spouse to what he called a
``dib'' in the sea.

His interesting patient, Jos, was a regular milch-cow to the Doctor,
and he easily persuaded the civilian, both for his own health's sake
and that of his charming sister, which was really very much shattered,
to pass the summer at that hideous seaport town.  Emmy did not care
where she went much.  Georgy jumped at the idea of a move.  As for
Becky, she came as a matter of course in the fourth place inside of the
fine barouche Mr.\ Jos had bought, the two domestics being on the box in
front. She might have some misgivings about the friends whom she should
meet at Ostend, and who might be likely to tell ugly stories---but bah!
she was strong enough to hold her own.  She had cast such an anchor in
Jos now as would require a strong storm to shake.  That incident of the
picture had finished him.  Becky took down her elephant and put it into
the little box which she had had from Amelia ever so many years ago.
Emmy also came off with her Lares---her two pictures---and the party,
finally, were, lodged in an exceedingly dear and uncomfortable house at
Ostend.

There Amelia began to take baths and get what good she could from them,
and though scores of people of Becky's acquaintance passed her and cut
her, yet Mrs.\ Osborne, who walked about with her, and who knew nobody,
was not aware of the treatment experienced by the friend whom she had
chosen so judiciously as a companion; indeed, Becky never thought fit
to tell her what was passing under her innocent eyes.

Some of Mrs.\ Rawdon Crawley's acquaintances, however, acknowledged her
readily enough,---perhaps more readily than she would have desired.
Among those were Major Loder (unattached), and Captain Rook (late of
the Rifles), who might be seen any day on the Dike, smoking and staring
at the women, and who speedily got an introduction to the hospitable
board and select circle of Mr.\ Joseph Sedley.  In fact they would take
no denial; they burst into the house whether Becky was at home or not,
walked into Mrs.\ Osborne's drawing-room, which they perfumed with their
coats and mustachios, called Jos ``Old buck,'' and invaded his
dinner-table, and laughed and drank for long hours there.

``What can they mean?'' asked Georgy, who did not like these gentlemen.
``I heard the Major say to Mrs.\ Crawley yesterday, 'No, no, Becky, you
shan't keep the old buck to yourself.  We must have the bones in, or,
dammy, I'll split.' What could the Major mean, Mamma?''

``Major!  don't call him Major!'' Emmy said.  ``I'm sure I can't tell what
he meant.'' His presence and that of his friend inspired the little lady
with intolerable terror and aversion.  They paid her tipsy compliments;
they leered at her over the dinner-table.  And the Captain made her
advances that filled her with sickening dismay, nor would she ever see
him unless she had George by her side.

Rebecca, to do her justice, never would let either of these men remain
alone with Amelia; the Major was disengaged too, and swore he would be
the winner of her. A couple of ruffians were fighting for this innocent
creature, gambling for her at her own table, and though she was not
aware of the rascals' designs upon her, yet she felt a horror and
uneasiness in their presence and longed to fly.

She besought, she entreated Jos to go.  Not he.  He was slow of
movement, tied to his Doctor, and perhaps to some other leading-strings.
At least Becky was not anxious to go to England.

At last she took a great resolution---made the great plunge.  She wrote
off a letter to a friend whom she had on the other side of the water, a
letter about which she did not speak a word to anybody, which she
carried herself to the post under her shawl; nor was any remark made
about it, only that she looked very much flushed and agitated when
Georgy met her, and she kissed him, and hung over him a great deal that
night.  She did not come out of her room after her return from her
walk. Becky thought it was Major Loder and the Captain who frightened
her.

``She mustn't stop here,'' Becky reasoned with herself. ``She must go
away, the silly little fool.  She is still whimpering after that gaby
of a husband---dead (and served right!) these fifteen years. She shan't
marry either of these men.  It's too bad of Loder.  No; she shall marry
the bamboo cane, I'll settle it this very night.''

So Becky took a cup of tea to Amelia in her private apartment and found
that lady in the company of her miniatures, and in a most melancholy
and nervous condition.  She laid down the cup of tea.

``Thank you,'' said Amelia.

``Listen to me, Amelia,'' said Becky, marching up and down the room
before the other and surveying her with a sort of contemptuous
kindness.  ``I want to talk to you. You must go away from here and from
the impertinences of these men.  I won't have you harassed by them:
and they will insult you if you stay.  I tell you they are rascals: men
fit to send to the hulks.  Never mind how I know them. I know
everybody.  Jos can't protect you; he is too weak and wants a protector
himself.  You are no more fit to live in the world than a baby in arms.
You must marry, or you and your precious boy will go to ruin.  You must
have a husband, you fool; and one of the best gentlemen I ever saw has
offered you a hundred times, and you have rejected him, you silly,
heartless, ungrateful little creature!''

``I tried---I tried my best, indeed I did, Rebecca,'' said Amelia
deprecatingly, ``but I couldn't forget---''; and she finished the sentence
by looking up at the portrait.

``Couldn't forget HIM!'' cried out Becky, ``that selfish humbug, that
low-bred cockney dandy, that padded booby, who had neither wit, nor
manners, nor heart, and was no more to be compared to your friend with
the bamboo cane than you are to Queen Elizabeth.  Why, the man was
weary of you, and would have jilted you, but that Dobbin forced him to
keep his word.  He owned it to me.  He never cared for you. He used to
sneer about you to me, time after time, and made love to me the week
after he married you.''

``It's false!  It's false!  Rebecca,'' cried out Amelia, starting up.

``Look there, you fool,'' Becky said, still with provoking good humour,
and taking a little paper out of her belt, she opened it and flung it
into Emmy's lap.  ``You know his handwriting.  He wrote that to
me---wanted me to run away with him---gave it me under your nose, the day
before he was shot---and served him right!'' Becky repeated.

Emmy did not hear her; she was looking at the letter. It was that which
George had put into the bouquet and given to Becky on the night of the
Duchess of Richmond's ball.  It was as she said:  the foolish young man
had asked her to fly.

Emmy's head sank down, and for almost the last time in which she shall
be called upon to weep in this history, she commenced that work.  Her
head fell to her bosom, and her hands went up to her eyes; and there
for a while, she gave way to her emotions, as Becky stood on and
regarded her.  Who shall analyse those tears and say whether they were
sweet or bitter? Was she most grieved because the idol of her life was
tumbled down and shivered at her feet, or indignant that her love had
been so despised, or glad because the barrier was removed which modesty
had placed between her and a new, a real affection? ``There is nothing
to forbid me now,'' she thought. ``I may love him with all my heart now.
Oh, I will, I will, if he will but let me and forgive me.'' I believe it
was this feeling rushed over all the others which agitated that gentle
little bosom.

Indeed, she did not cry so much as Becky expected---the other soothed
and kissed her---a rare mark of sympathy with Mrs.\ Becky.  She treated
Emmy like a child and patted her head.  ``And now let us get pen and ink
and write to him to come this minute,'' she said.

``I---I wrote to him this morning,'' Emmy said, blushing exceedingly.
Becky screamed with laughter---``Un biglietto,'' she sang out with Rosina,
``eccolo qua!''---the whole house echoed with her shrill singing.

Two mornings after this little scene, although the day was rainy and
gusty, and Amelia had had an exceedingly wakeful night, listening to
the wind roaring, and pitying all travellers by land and by water, yet
she got up early and insisted upon taking a walk on the Dike with
Georgy; and there she paced as the rain beat into her face, and she
looked out westward across the dark sea line and over the swollen
billows which came tumbling and frothing to the shore. Neither spoke
much, except now and then, when the boy said a few words to his timid
companion, indicative of sympathy and protection.

``I hope he won't cross in such weather,'' Emmy said.

``I bet ten to one he does,'' the boy answered.  ``Look, Mother, there's
the smoke of the steamer.'' It was that signal, sure enough.

But though the steamer was under way, he might not be on board; he
might not have got the letter; he might not choose to come.  A hundred
fears poured one over the other into the little heart, as fast as the
waves on to the Dike.

The boat followed the smoke into sight.  Georgy had a dandy telescope
and got the vessel under view in the most skilful manner. And he made
appropriate nautical comments upon the manner of the approach of the
steamer as she came nearer and nearer, dipping and rising in the water.
The signal of an English steamer in sight went fluttering up to the
mast on the pier.  I daresay Mrs.\ Amelia's heart was in a similar
flutter.

Emmy tried to look through the telescope over George's shoulder, but
she could make nothing of it. She only saw a black eclipse bobbing up
and down before her eyes.

George took the glass again and raked the vessel. ``How she does pitch!''
he said.  ``There goes a wave slap over her bows.  There's only two
people on deck besides the steersman.  There's a man lying down, and
a---chap in a---cloak with a---Hooray!---it's Dob, by Jingo!'' He clapped to
the telescope and flung his arms round his mother.  As for that lady,
let us say what she did in the words of a favourite poet---``Dakruoen
gelasasa.'' She was sure it was William.  It could be no other.  What
she had said about hoping that he would not come was all hypocrisy.  Of
course he would come; what could he do else but come? She knew he would
come.

The ship came swiftly nearer and nearer.  As they went in to meet her
at the landing-place at the quay, Emmy's knees trembled so that she
scarcely could run.  She would have liked to kneel down and say her
prayers of thanks there.  Oh, she thought, she would be all her life
saying them!

It was such a bad day that as the vessel came alongside of the quay
there were no idlers abroad, scarcely even a commissioner on the look
out for the few passengers in the steamer.  That young scapegrace
George had fled too, and as the gentleman in the old cloak lined with
red stuff stepped on to the shore, there was scarcely any one present
to see what took place, which was briefly this:

A lady in a dripping white bonnet and shawl, with her two little hands
out before her, went up to him, and in the next minute she had
altogether disappeared under the folds of the old cloak, and was
kissing one of his hands with all her might; whilst the other, I
suppose, was engaged in holding her to his heart (which her head just
about reached) and in preventing her from tumbling down.  She was
murmuring something about---forgive---dear William---dear, dear, dearest
friend---kiss, kiss, kiss, and so forth---and in fact went on under the
cloak in an absurd manner.

When Emmy emerged from it, she still kept tight hold of one of
William's hands, and looked up in his face.  It was full of sadness and
tender love and pity.  She understood its reproach and hung down her
head.

``It was time you sent for me, dear Amelia,'' he said.

``You will never go again, William?''

``No, never,'' he answered, and pressed the dear little soul once more to
his heart.

As they issued out of the custom-house precincts, Georgy broke out on
them, with his telescope up to his eye, and a loud laugh of welcome; he
danced round the couple and performed many facetious antics as he led
them up to the house.  Jos wasn't up yet; Becky not visible (though she
looked at them through the blinds). Georgy ran off to see about
breakfast.  Emmy, whose shawl and bonnet were off in the passage in the
hands of Mrs.\ Payne, now went to undo the clasp of William's cloak,
and---we will, if you please, go with George, and look after breakfast
for the Colonel.  The vessel is in port. He has got the prize he has
been trying for all his life. The bird has come in at last.  There it
is with its head on his shoulder, billing and cooing close up to his
heart, with soft outstretched fluttering wings.  This is what he has
asked for every day and hour for eighteen years.  This is what he pined
after.  Here it is---the summit, the end---the last page of the third
volume. Good-bye, Colonel---God bless you, honest William!---Farewell,
dear Amelia---Grow green again, tender little parasite, round the rugged
old oak to which you cling!

Perhaps it was compunction towards the kind and simple creature, who
had been the first in life to defend her, perhaps it was a dislike to
all such sentimental scenes---but Rebecca, satisfied with her part in
the transaction, never presented herself before Colonel Dobbin and the
lady whom he married.  ``Particular business,'' she said, took her to
Bruges, whither she went, and only Georgy and his uncle were present at
the marriage ceremony. When it was over, and Georgy had rejoined his
parents, Mrs.\ Becky returned (just for a few days) to comfort the
solitary bachelor, Joseph Sedley.  He preferred a continental life, he
said, and declined to join in housekeeping with his sister and her
husband.

Emmy was very glad in her heart to think that she had written to her
husband before she read or knew of that letter of George's.  ``I knew it
all along,'' William said; ``but could I use that weapon against the poor
fellow's memory? It was that which made me suffer so when you---''

``Never speak of that day again,'' Emmy cried out, so contrite and humble
that William turned off the conversation by his account of Glorvina and
dear old Peggy O'Dowd, with whom he was sitting when the letter of
recall reached him.  ``If you hadn't sent for me,'' he added with a
laugh, ``who knows what Glorvina's name might be now?''

At present it is Glorvina Posky (now Mrs.\ Major Posky); she took him on
the death of his first wife, having resolved never to marry out of the
regiment.  Lady O'Dowd is also so attached to it that, she says, if
anything were to happen to Mick, bedad she'd come back and marry some
of 'em.  But the Major-General is quite well and lives in great
splendour at O'Dowdstown, with a pack of beagles, and (with the
exception of perhaps their neighbour, Hoggarty of Castle Hoggarty) he
is the first man of his county.  Her Ladyship still dances jigs, and
insisted on standing up with the Master of the Horse at the Lord
Lieutenant's last ball.  Both she and Glorvina declared that Dobbin had
used the latter SHEAMFULLY, but Posky falling in, Glorvina was
consoled, and a beautiful turban from Paris appeased the wrath of Lady
O'Dowd.

When Colonel Dobbin quitted the service, which he did immediately after
his marriage, he rented a pretty little country place in Hampshire, not
far from Queen's Crawley, where, after the passing of the Reform Bill,
Sir Pitt and his family constantly resided now. All idea of a Peerage
was out of the question, the Baronet's two seats in Parliament being
lost.  He was both out of pocket and out of spirits by that
catastrophe, failed in his health, and prophesied the speedy ruin of
the Empire.

Lady Jane and Mrs.\ Dobbin became great friends---there was a perpetual
crossing of pony-chaises between the Hall and the Evergreens, the
Colonel's place (rented of his friend Major Ponto, who was abroad with
his family).  Her Ladyship was godmother to Mrs.\ Dobbin's child, which
bore her name, and was christened by the Rev. James Crawley, who
succeeded his father in the living: and a pretty close friendship
subsisted between the two lads, George and Rawdon, who hunted and shot
together in the vacations, were both entered of the same college at
Cambridge, and quarrelled with each other about Lady Jane's daughter,
with whom they were both, of course, in love. A match between George
and that young lady was long a favourite scheme of both the matrons,
though I have heard that Miss Crawley herself inclined towards her
cousin.

Mrs.\ Rawdon Crawley's name was never mentioned by either family. There
were reasons why all should be silent regarding her.  For wherever Mr.\ %
Joseph Sedley went, she travelled likewise, and that infatuated man
seemed to be entirely her slave.  The Colonel's lawyers informed him
that his brother-in-law had effected a heavy insurance upon his life,
whence it was probable that he had been raising money to discharge
debts.  He procured prolonged leave of absence from the East India
House, and indeed, his infirmities were daily increasing.

On hearing the news about the insurance, Amelia, in a good deal of
alarm, entreated her husband to go to Brussels, where Jos then was, and
inquire into the state of his affairs.  The Colonel quitted home with
reluctance (for he was deeply immersed in his History of the Punjaub
which still occupies him, and much alarmed about his little daughter,
whom he idolizes, and who was just recovering from the chicken-pox) and
went to Brussels and found Jos living at one of the enormous hotels in
that city.  Mrs.\ Crawley, who had her carriage, gave entertainments,
and lived in a very genteel manner, occupied another suite of
apartments in the same hotel.

The Colonel, of course, did not desire to see that lady, or even think
proper to notify his arrival at Brussels, except privately to Jos by a
message through his valet.  Jos begged the Colonel to come and see him
that night, when Mrs.\ Crawley would be at a soiree, and when they could
meet alone.  He found his brother-in-law in a condition of pitiable
infirmity---and dreadfully afraid of Rebecca, though eager in his
praises of her.  She tended him through a series of unheard-of
illnesses with a fidelity most admirable.  She had been a daughter to
him.  ``But---but---oh, for God's sake, do come and live near me,
and---and---see me sometimes,'' whimpered out the unfortunate man.

The Colonel's brow darkened at this.  ``We can't, Jos,'' he said.
``Considering the circumstances, Amelia can't visit you.''

``I swear to you---I swear to you on the Bible,'' gasped out Joseph,
wanting to kiss the book, ``that she is as innocent as a child, as
spotless as your own wife.''

``It may be so,'' said the Colonel gloomily, ``but Emmy can't come to you.
Be a man, Jos:  break off this disreputable connection.  Come home to
your family.  We hear your affairs are involved.''

``Involved!'' cried Jos.  ``Who has told such calumnies? All my money is
placed out most advantageously.  Mrs.\ Crawley---that is---I mean---it is
laid out to the best interest.''

``You are not in debt, then? Why did you insure your life?''

``I thought---a little present to her---in case anything happened; and you
know my health is so delicate---common gratitude you know---and I intend
to leave all my money to you---and I can spare it out of my income,
indeed I can,'' cried out William's weak brother-in-law.

The Colonel besought Jos to fly at once---to go back to India, whither
Mrs.\ Crawley could not follow him; to do anything to break off a
connection which might have the most fatal consequences to him.

Jos clasped his hands and cried, ``He would go back to India.  He would
do anything, only he must have time: they mustn't say anything to Mrs.\ %
Crawley---she'd---she'd kill me if she knew it.  You don't know what a
terrible woman she is,'' the poor wretch said.

``Then, why not come away with me?'' said Dobbin in reply; but Jos had
not the courage.  ``He would see Dobbin again in the morning; he must on
no account say that he had been there.  He must go now.  Becky might
come in.'' And Dobbin quitted him, full of forebodings.

He never saw Jos more.  Three months afterwards Joseph Sedley died at
\foreign{Aix-la-Chapelle}.  It was found that all his property had been muddled
away in speculations, and was represented by valueless shares in
different bubble companies.  All his available assets were the two
thousand pounds for which his life was insured, and which were left
equally between his beloved ``sister Amelia, wife of, \&c., and his
friend and invaluable attendant during sickness, Rebecca, wife of
Lieutenant-Colonel Rawdon Crawley, C.B.,'' who was appointed
administratrix.

The solicitor of the insurance company swore it was the blackest case
that ever had come before him, talked of sending a commission to Aix to
examine into the death, and the Company refused payment of the policy.
But Mrs.,\ or Lady Crawley, as she styled herself, came to town at once
(attended with her solicitors, Messrs.\ Burke, Thurtell, and Hayes, of
Thavies Inn) and dared the Company to refuse the payment.  They invited
examination, they declared that she was the object of an infamous
conspiracy, which had been pursuing her all through life, and triumphed
finally.  The money was paid, and her character established, but
Colonel Dobbin sent back his share of the legacy to the insurance
office and rigidly declined to hold any communication with Rebecca.

She never was Lady Crawley, though she continued so to call herself.
His Excellency Colonel Rawdon Crawley died of yellow fever at Coventry
Island, most deeply beloved and deplored, and six weeks before the
demise of his brother, Sir Pitt.  The estate consequently devolved upon
the present Sir Rawdon Crawley, Bart.

He, too, has declined to see his mother, to whom he makes a liberal
allowance, and who, besides, appears to be very wealthy.  The Baronet
lives entirely at Queen's Crawley, with Lady Jane and her daughter,
whilst Rebecca, Lady Crawley, chiefly hangs about Bath and Cheltenham,
where a very strong party of excellent people consider her to be a most
injured woman.  She has her enemies.  Who has not? Her life is her
answer to them. She busies herself in works of piety.  She goes to
church, and never without a footman.  Her name is in all the Charity
Lists.  The destitute orange-girl, the neglected washerwoman, the
distressed muffin-man find in her a fast and generous friend.  She is
always having stalls at Fancy Fairs for the benefit of these hapless
beings.  Emmy, her children, and the Colonel, coming to London some
time back, found themselves suddenly before her at one of these fairs.
She cast down her eyes demurely and smiled as they started away from
her; Emmy scurrying off on the arm of George (now grown a dashing young
gentleman) and the Colonel seizing up his little Janey, of whom he is
fonder than of anything in the world---fonder even than of his History
of the Punjaub.

``Fonder than he is of me,'' Emmy thinks with a sigh. But he never said a
word to Amelia that was not kind and gentle, or thought of a want of
hers that he did not try to gratify.

Ah!  \foreign{Vanitas Vanitatum!}  which of us is happy in this world? Which of
us has his desire? or, having it, is satisfied?---come, children, let us
shut up the box and the puppets, for our play is played out.


\cleardoublepage
~\markboth{}{}

\cleardoublepage
\appendix
\renewcommand{\thechapter}{\Alph{chapter}.}
\addtocontents{toc}{\vspace{\normalbaselineskip}}
\chapter{A note about this book}\label{about}

HOGG:  How did I make this?

HOGG: What decisions did I make (about quotations, dashes, emphasis,
money, foreign words, italics, etc.)?

HOGG:  Thank \LaTeX.

I would like to thank Project Gutengerg (HOGG: say more).

I appreciated comparing with my Modern Library Classics edition (HOGG: give full citation).

HOGG: I got ideas from \textit{The Elements of Typographic Style} by Bringhurst (?).

\chapter{Information from Project Gutenberg}
The following is preserved \foreign{verbatim} from the Project Gutenberg
text file from which the \LaTeX\ code source for this book was generated.

{\footnotesize
\begin{verbatim}
The Project Gutenberg EBook of Vanity Fair, by William Makepeace Thackeray

This eBook is for the use of anyone anywhere at no cost and with
almost no restrictions whatsoever.  You may copy it, give it away or
re-use it under the terms of the Project Gutenberg License included
with this eBook or online at www.gutenberg.net

Title: Vanity Fair

Author: William Makepeace Thackeray

Posting Date: August 30, 2008 [EBook #599]
Release Date: July, 1996
[Last updated: September 10, 2015]

Language: English

This and all associated files of various formats will be found in:
        http://www.gutenberg.org/5/9/599/

Produced by Juli Rew.

Updated editions will replace the previous one--the old editions
will be renamed.

Creating the works from public domain print editions means that no
one owns a United States copyright in these works, so the Foundation
(and you!) can copy and distribute it in the United States without
permission and without paying copyright royalties.  Special rules,
set forth in the General Terms of Use part of this license, apply to
copying and distributing Project Gutenberg-tm electronic works to
protect the PROJECT GUTENBERG-tm concept and trademark.  Project
Gutenberg is a registered trademark, and may not be used if you
charge for the eBooks, unless you receive specific permission.  If you
do not charge anything for copies of this eBook, complying with the
rules is very easy.  You may use this eBook for nearly any purpose
such as creation of derivative works, reports, performances and
research.  They may be modified and printed and given away--you may do
practically ANYTHING with public domain eBooks.  Redistribution is
subject to the trademark license, especially commercial
redistribution.

*** START: FULL LICENSE ***

THE FULL PROJECT GUTENBERG LICENSE
PLEASE READ THIS BEFORE YOU DISTRIBUTE OR USE THIS WORK

To protect the Project Gutenberg-tm mission of promoting the free
distribution of electronic works, by using or distributing this work
(or any other work associated in any way with the phrase "Project
Gutenberg"), you agree to comply with all the terms of the Full Project
Gutenberg-tm License (available with this file or online at
http://gutenberg.net/license).

Section 1.  General Terms of Use and Redistributing Project Gutenberg-tm
electronic works

1.A.  By reading or using any part of this Project Gutenberg-tm
electronic work, you indicate that you have read, understand, agree to
and accept all the terms of this license and intellectual property
(trademark/copyright) agreement.  If you do not agree to abide by all
the terms of this agreement, you must cease using and return or destroy
all copies of Project Gutenberg-tm electronic works in your possession.
If you paid a fee for obtaining a copy of or access to a Project
Gutenberg-tm electronic work and you do not agree to be bound by the
terms of this agreement, you may obtain a refund from the person or
entity to whom you paid the fee as set forth in paragraph 1.E.8.

1.B.  "Project Gutenberg" is a registered trademark.  It may only be
used on or associated in any way with an electronic work by people who
agree to be bound by the terms of this agreement.  There are a few
things that you can do with most Project Gutenberg-tm electronic works
even without complying with the full terms of this agreement.  See
paragraph 1.C below.  There are a lot of things you can do with Project
Gutenberg-tm electronic works if you follow the terms of this agreement
and help preserve free future access to Project Gutenberg-tm electronic
works.  See paragraph 1.E below.

1.C.  The Project Gutenberg Literary Archive Foundation ("the Foundation"
or PGLAF), owns a compilation copyright in the collection of Project
Gutenberg-tm electronic works.  Nearly all the individual works in the
collection are in the public domain in the United States.  If an
individual work is in the public domain in the United States and you are
located in the United States, we do not claim a right to prevent you from
copying, distributing, performing, displaying or creating derivative
works based on the work as long as all references to Project Gutenberg
are removed.  Of course, we hope that you will support the Project
Gutenberg-tm mission of promoting free access to electronic works by
freely sharing Project Gutenberg-tm works in compliance with the terms of
this agreement for keeping the Project Gutenberg-tm name associated with
the work.  You can easily comply with the terms of this agreement by
keeping this work in the same format with its attached full Project
Gutenberg-tm License when you share it without charge with others.

1.D.  The copyright laws of the place where you are located also govern
what you can do with this work.  Copyright laws in most countries are in
a constant state of change.  If you are outside the United States, check
the laws of your country in addition to the terms of this agreement
before downloading, copying, displaying, performing, distributing or
creating derivative works based on this work or any other Project
Gutenberg-tm work.  The Foundation makes no representations concerning
the copyright status of any work in any country outside the United
States.

1.E.  Unless you have removed all references to Project Gutenberg:

1.E.1.  The following sentence, with active links to, or other immediate
access to, the full Project Gutenberg-tm License must appear prominently
whenever any copy of a Project Gutenberg-tm work (any work on which the
phrase "Project Gutenberg" appears, or with which the phrase "Project
Gutenberg" is associated) is accessed, displayed, performed, viewed,
copied or distributed:

This eBook is for the use of anyone anywhere at no cost and with
almost no restrictions whatsoever.  You may copy it, give it away or
re-use it under the terms of the Project Gutenberg License included
with this eBook or online at www.gutenberg.net

1.E.2.  If an individual Project Gutenberg-tm electronic work is derived
from the public domain (does not contain a notice indicating that it is
posted with permission of the copyright holder), the work can be copied
and distributed to anyone in the United States without paying any fees
or charges.  If you are redistributing or providing access to a work
with the phrase "Project Gutenberg" associated with or appearing on the
work, you must comply either with the requirements of paragraphs 1.E.1
through 1.E.7 or obtain permission for the use of the work and the
Project Gutenberg-tm trademark as set forth in paragraphs 1.E.8 or
1.E.9.

1.E.3.  If an individual Project Gutenberg-tm electronic work is posted
with the permission of the copyright holder, your use and distribution
must comply with both paragraphs 1.E.1 through 1.E.7 and any additional
terms imposed by the copyright holder.  Additional terms will be linked
to the Project Gutenberg-tm License for all works posted with the
permission of the copyright holder found at the beginning of this work.

1.E.4.  Do not unlink or detach or remove the full Project Gutenberg-tm
License terms from this work, or any files containing a part of this
work or any other work associated with Project Gutenberg-tm.

1.E.5.  Do not copy, display, perform, distribute or redistribute this
electronic work, or any part of this electronic work, without
prominently displaying the sentence set forth in paragraph 1.E.1 with
active links or immediate access to the full terms of the Project
Gutenberg-tm License.

1.E.6.  You may convert to and distribute this work in any binary,
compressed, marked up, nonproprietary or proprietary form, including any
word processing or hypertext form.  However, if you provide access to or
distribute copies of a Project Gutenberg-tm work in a format other than
"Plain Vanilla ASCII" or other format used in the official version
posted on the official Project Gutenberg-tm web site (www.gutenberg.net),
you must, at no additional cost, fee or expense to the user, provide a
copy, a means of exporting a copy, or a means of obtaining a copy upon
request, of the work in its original "Plain Vanilla ASCII" or other
form.  Any alternate format must include the full Project Gutenberg-tm
License as specified in paragraph 1.E.1.

1.E.7.  Do not charge a fee for access to, viewing, displaying,
performing, copying or distributing any Project Gutenberg-tm works
unless you comply with paragraph 1.E.8 or 1.E.9.

1.E.8.  You may charge a reasonable fee for copies of or providing
access to or distributing Project Gutenberg-tm electronic works provided
that

- You pay a royalty fee of 20% of the gross profits you derive from
     the use of Project Gutenberg-tm works calculated using the method
     you already use to calculate your applicable taxes.  The fee is
     owed to the owner of the Project Gutenberg-tm trademark, but he
     has agreed to donate royalties under this paragraph to the
     Project Gutenberg Literary Archive Foundation.  Royalty payments
     must be paid within 60 days following each date on which you
     prepare (or are legally required to prepare) your periodic tax
     returns.  Royalty payments should be clearly marked as such and
     sent to the Project Gutenberg Literary Archive Foundation at the
     address specified in Section 4, "Information about donations to
     the Project Gutenberg Literary Archive Foundation."

- You provide a full refund of any money paid by a user who notifies
     you in writing (or by e-mail) within 30 days of receipt that s/he
     does not agree to the terms of the full Project Gutenberg-tm
     License.  You must require such a user to return or
     destroy all copies of the works possessed in a physical medium
     and discontinue all use of and all access to other copies of
     Project Gutenberg-tm works.

- You provide, in accordance with paragraph 1.F.3, a full refund of any
     money paid for a work or a replacement copy, if a defect in the
     electronic work is discovered and reported to you within 90 days
     of receipt of the work.

- You comply with all other terms of this agreement for free
     distribution of Project Gutenberg-tm works.

1.E.9.  If you wish to charge a fee or distribute a Project Gutenberg-tm
electronic work or group of works on different terms than are set
forth in this agreement, you must obtain permission in writing from
both the Project Gutenberg Literary Archive Foundation and Michael
Hart, the owner of the Project Gutenberg-tm trademark.  Contact the
Foundation as set forth in Section 3 below.

1.F.

1.F.1.  Project Gutenberg volunteers and employees expend considerable
effort to identify, do copyright research on, transcribe and proofread
public domain works in creating the Project Gutenberg-tm
collection.  Despite these efforts, Project Gutenberg-tm electronic
works, and the medium on which they may be stored, may contain
"Defects," such as, but not limited to, incomplete, inaccurate or
corrupt data, transcription errors, a copyright or other intellectual
property infringement, a defective or damaged disk or other medium, a
computer virus, or computer codes that damage or cannot be read by
your equipment.

1.F.2.  LIMITED WARRANTY, DISCLAIMER OF DAMAGES - Except for the "Right
of Replacement or Refund" described in paragraph 1.F.3, the Project
Gutenberg Literary Archive Foundation, the owner of the Project
Gutenberg-tm trademark, and any other party distributing a Project
Gutenberg-tm electronic work under this agreement, disclaim all
liability to you for damages, costs and expenses, including legal
fees.  YOU AGREE THAT YOU HAVE NO REMEDIES FOR NEGLIGENCE, STRICT
LIABILITY, BREACH OF WARRANTY OR BREACH OF CONTRACT EXCEPT THOSE
PROVIDED IN PARAGRAPH F3.  YOU AGREE THAT THE FOUNDATION, THE
TRADEMARK OWNER, AND ANY DISTRIBUTOR UNDER THIS AGREEMENT WILL NOT BE
LIABLE TO YOU FOR ACTUAL, DIRECT, INDIRECT, CONSEQUENTIAL, PUNITIVE OR
INCIDENTAL DAMAGES EVEN IF YOU GIVE NOTICE OF THE POSSIBILITY OF SUCH
DAMAGE.

1.F.3.  LIMITED RIGHT OF REPLACEMENT OR REFUND - If you discover a
defect in this electronic work within 90 days of receiving it, you can
receive a refund of the money (if any) you paid for it by sending a
written explanation to the person you received the work from.  If you
received the work on a physical medium, you must return the medium with
your written explanation.  The person or entity that provided you with
the defective work may elect to provide a replacement copy in lieu of a
refund.  If you received the work electronically, the person or entity
providing it to you may choose to give you a second opportunity to
receive the work electronically in lieu of a refund.  If the second copy
is also defective, you may demand a refund in writing without further
opportunities to fix the problem.

1.F.4.  Except for the limited right of replacement or refund set forth
in paragraph 1.F.3, this work is provided to you 'AS-IS' WITH NO OTHER
WARRANTIES OF ANY KIND, EXPRESS OR IMPLIED, INCLUDING BUT NOT LIMITED TO
WARRANTIES OF MERCHANTIBILITY OR FITNESS FOR ANY PURPOSE.

1.F.5.  Some states do not allow disclaimers of certain implied
warranties or the exclusion or limitation of certain types of damages.
If any disclaimer or limitation set forth in this agreement violates the
law of the state applicable to this agreement, the agreement shall be
interpreted to make the maximum disclaimer or limitation permitted by
the applicable state law.  The invalidity or unenforceability of any
provision of this agreement shall not void the remaining provisions.

1.F.6.  INDEMNITY - You agree to indemnify and hold the Foundation, the
trademark owner, any agent or employee of the Foundation, anyone
providing copies of Project Gutenberg-tm electronic works in accordance
with this agreement, and any volunteers associated with the production,
promotion and distribution of Project Gutenberg-tm electronic works,
harmless from all liability, costs and expenses, including legal fees,
that arise directly or indirectly from any of the following which you do
or cause to occur: (a) distribution of this or any Project Gutenberg-tm
work, (b) alteration, modification, or additions or deletions to any
Project Gutenberg-tm work, and (c) any Defect you cause.

Section  2.  Information about the Mission of Project Gutenberg-tm

Project Gutenberg-tm is synonymous with the free distribution of
electronic works in formats readable by the widest variety of computers
including obsolete, old, middle-aged and new computers.  It exists
because of the efforts of hundreds of volunteers and donations from
people in all walks of life.

Volunteers and financial support to provide volunteers with the
assistance they need, is critical to reaching Project Gutenberg-tm's
goals and ensuring that the Project Gutenberg-tm collection will
remain freely available for generations to come.  In 2001, the Project
Gutenberg Literary Archive Foundation was created to provide a secure
and permanent future for Project Gutenberg-tm and future generations.
To learn more about the Project Gutenberg Literary Archive Foundation
and how your efforts and donations can help, see Sections 3 and 4
and the Foundation web page at http://www.pglaf.org.

Section 3.  Information about the Project Gutenberg Literary Archive
Foundation

The Project Gutenberg Literary Archive Foundation is a non profit
501(c)(3) educational corporation organized under the laws of the
state of Mississippi and granted tax exempt status by the Internal
Revenue Service.  The Foundation's EIN or federal tax identification
number is 64-6221541.  Its 501(c)(3) letter is posted at
http://pglaf.org/fundraising.  Contributions to the Project Gutenberg
Literary Archive Foundation are tax deductible to the full extent
permitted by U.S. federal laws and your state's laws.

The Foundation's principal office is located at 4557 Melan Dr. S.
Fairbanks, AK, 99712., but its volunteers and employees are scattered
throughout numerous locations.  Its business office is located at
809 North 1500 West, Salt Lake City, UT 84116, (801) 596-1887, email
business@pglaf.org.  Email contact links and up to date contact
information can be found at the Foundation's web site and official
page at http://pglaf.org

For additional contact information:
     Dr. Gregory B. Newby
     Chief Executive and Director
     gbnewby@pglaf.org

Section 4.  Information about Donations to the Project Gutenberg
Literary Archive Foundation

Project Gutenberg-tm depends upon and cannot survive without wide
spread public support and donations to carry out its mission of
increasing the number of public domain and licensed works that can be
freely distributed in machine readable form accessible by the widest
array of equipment including outdated equipment.  Many small donations
($1 to $5,000) are particularly important to maintaining tax exempt
status with the IRS.

The Foundation is committed to complying with the laws regulating
charities and charitable donations in all 50 states of the United
States.  Compliance requirements are not uniform and it takes a
considerable effort, much paperwork and many fees to meet and keep up
with these requirements.  We do not solicit donations in locations
where we have not received written confirmation of compliance.  To
SEND DONATIONS or determine the status of compliance for any
particular state visit http://pglaf.org

While we cannot and do not solicit contributions from states where we
have not met the solicitation requirements, we know of no prohibition
against accepting unsolicited donations from donors in such states who
approach us with offers to donate.

International donations are gratefully accepted, but we cannot make
any statements concerning tax treatment of donations received from
outside the United States.  U.S. laws alone swamp our small staff.

Please check the Project Gutenberg Web pages for current donation
methods and addresses.  Donations are accepted in a number of other
ways including including checks, online payments and credit card
donations.  To donate, please visit: http://pglaf.org/donate

Section 5.  General Information About Project Gutenberg-tm electronic
works.

Professor Michael S. Hart is the originator of the Project Gutenberg-tm
concept of a library of electronic works that could be freely shared
with anyone.  For thirty years, he produced and distributed Project
Gutenberg-tm eBooks with only a loose network of volunteer support.

Project Gutenberg-tm eBooks are often created from several printed
editions, all of which are confirmed as Public Domain in the U.S.
unless a copyright notice is included.  Thus, we do not necessarily
keep eBooks in compliance with any particular paper edition.

Most people start at our Web site which has the main PG search facility:

     http://www.gutenberg.net

This Web site includes information about Project Gutenberg-tm,
including how to make donations to the Project Gutenberg Literary
Archive Foundation, how to help produce our new eBooks, and how to
subscribe to our email newsletter to hear about new eBooks.
\end{verbatim}
}


\end{document}
